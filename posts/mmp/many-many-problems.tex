% Options for packages loaded elsewhere
\PassOptionsToPackage{unicode}{hyperref}
\PassOptionsToPackage{hyphens}{url}
\PassOptionsToPackage{dvipsnames,svgnames,x11names}{xcolor}
%
\documentclass[
  11pt,
  letterpaper,
  DIV=11,
  numbers=noendperiod,
  oneside]{scrartcl}

\usepackage{amsmath,amssymb}
\usepackage{iftex}
\ifPDFTeX
  \usepackage[T1]{fontenc}
  \usepackage[utf8]{inputenc}
  \usepackage{textcomp} % provide euro and other symbols
\else % if luatex or xetex
  \ifXeTeX
    \usepackage{mathspec} % this also loads fontspec
  \else
    \usepackage{unicode-math} % this also loads fontspec
  \fi
  \defaultfontfeatures{Scale=MatchLowercase}
  \defaultfontfeatures[\rmfamily]{Ligatures=TeX,Scale=1}
\fi
\usepackage{lmodern}
\ifPDFTeX\else  
    % xetex/luatex font selection
  \setmainfont[Scale = MatchLowercase]{Scala Pro}
  \setsansfont[]{Scala Sans Pro}
  \ifXeTeX
    \setmathfont(Digits,Latin,Greek)[]{Scala Pro}
  \else
    \setmathfont[]{Scala Pro}
  \fi
\fi
% Use upquote if available, for straight quotes in verbatim environments
\IfFileExists{upquote.sty}{\usepackage{upquote}}{}
\IfFileExists{microtype.sty}{% use microtype if available
  \usepackage[]{microtype}
  \UseMicrotypeSet[protrusion]{basicmath} % disable protrusion for tt fonts
}{}
\makeatletter
\@ifundefined{KOMAClassName}{% if non-KOMA class
  \IfFileExists{parskip.sty}{%
    \usepackage{parskip}
  }{% else
    \setlength{\parindent}{0pt}
    \setlength{\parskip}{6pt plus 2pt minus 1pt}}
}{% if KOMA class
  \KOMAoptions{parskip=half}}
\makeatother
\usepackage{xcolor}
\usepackage[left=1in,marginparwidth=2.0666666666667in,textwidth=4.1333333333333in,marginparsep=0.3in]{geometry}
\setlength{\emergencystretch}{3em} % prevent overfull lines
\setcounter{secnumdepth}{3}
% Make \paragraph and \subparagraph free-standing
\ifx\paragraph\undefined\else
  \let\oldparagraph\paragraph
  \renewcommand{\paragraph}[1]{\oldparagraph{#1}\mbox{}}
\fi
\ifx\subparagraph\undefined\else
  \let\oldsubparagraph\subparagraph
  \renewcommand{\subparagraph}[1]{\oldsubparagraph{#1}\mbox{}}
\fi


\providecommand{\tightlist}{%
  \setlength{\itemsep}{0pt}\setlength{\parskip}{0pt}}\usepackage{longtable,booktabs,array}
\usepackage{calc} % for calculating minipage widths
% Correct order of tables after \paragraph or \subparagraph
\usepackage{etoolbox}
\makeatletter
\patchcmd\longtable{\par}{\if@noskipsec\mbox{}\fi\par}{}{}
\makeatother
% Allow footnotes in longtable head/foot
\IfFileExists{footnotehyper.sty}{\usepackage{footnotehyper}}{\usepackage{footnote}}
\makesavenoteenv{longtable}
\usepackage{graphicx}
\makeatletter
\def\maxwidth{\ifdim\Gin@nat@width>\linewidth\linewidth\else\Gin@nat@width\fi}
\def\maxheight{\ifdim\Gin@nat@height>\textheight\textheight\else\Gin@nat@height\fi}
\makeatother
% Scale images if necessary, so that they will not overflow the page
% margins by default, and it is still possible to overwrite the defaults
% using explicit options in \includegraphics[width, height, ...]{}
\setkeys{Gin}{width=\maxwidth,height=\maxheight,keepaspectratio}
% Set default figure placement to htbp
\makeatletter
\def\fps@figure{htbp}
\makeatother
% definitions for citeproc citations
\NewDocumentCommand\citeproctext{}{}
\NewDocumentCommand\citeproc{mm}{%
  \begingroup\def\citeproctext{#2}\cite{#1}\endgroup}
\makeatletter
 % allow citations to break across lines
 \let\@cite@ofmt\@firstofone
 % avoid brackets around text for \cite:
 \def\@biblabel#1{}
 \def\@cite#1#2{{#1\if@tempswa , #2\fi}}
\makeatother
\newlength{\cslhangindent}
\setlength{\cslhangindent}{1.5em}
\newlength{\csllabelwidth}
\setlength{\csllabelwidth}{3em}
\newenvironment{CSLReferences}[2] % #1 hanging-indent, #2 entry-spacing
 {\begin{list}{}{%
  \setlength{\itemindent}{0pt}
  \setlength{\leftmargin}{0pt}
  \setlength{\parsep}{0pt}
  % turn on hanging indent if param 1 is 1
  \ifodd #1
   \setlength{\leftmargin}{\cslhangindent}
   \setlength{\itemindent}{-1\cslhangindent}
  \fi
  % set entry spacing
  \setlength{\itemsep}{#2\baselineskip}}}
 {\end{list}}
\usepackage{calc}
\newcommand{\CSLBlock}[1]{\hfill\break#1\hfill\break}
\newcommand{\CSLLeftMargin}[1]{\parbox[t]{\csllabelwidth}{\strut#1\strut}}
\newcommand{\CSLRightInline}[1]{\parbox[t]{\linewidth - \csllabelwidth}{\strut#1\strut}}
\newcommand{\CSLIndent}[1]{\hspace{\cslhangindent}#1}

\setlength\heavyrulewidth{0ex}
\setlength\lightrulewidth{0ex}
\makeatletter
\def\@maketitle{%
\newpage
\null
\vskip 2em%
\begin{center}%
\let \footnote \thanks
  {\LARGE \@title \par}%
  \vskip 1.5em%
  {\large
    \lineskip .5em%
    \begin{tabular}[t]{c}%
      \@author
    \end{tabular}\par}%
  %\vskip 1em%
  %{\large \@date}%
\end{center}%
\par
\vskip 1.5em}
\makeatother 
\KOMAoption{captions}{tableheading}
\makeatletter
\@ifpackageloaded{caption}{}{\usepackage{caption}}
\AtBeginDocument{%
\ifdefined\contentsname
  \renewcommand*\contentsname{Table of contents}
\else
  \newcommand\contentsname{Table of contents}
\fi
\ifdefined\listfigurename
  \renewcommand*\listfigurename{List of Figures}
\else
  \newcommand\listfigurename{List of Figures}
\fi
\ifdefined\listtablename
  \renewcommand*\listtablename{List of Tables}
\else
  \newcommand\listtablename{List of Tables}
\fi
\ifdefined\figurename
  \renewcommand*\figurename{Figure}
\else
  \newcommand\figurename{Figure}
\fi
\ifdefined\tablename
  \renewcommand*\tablename{Table}
\else
  \newcommand\tablename{Table}
\fi
}
\@ifpackageloaded{float}{}{\usepackage{float}}
\floatstyle{ruled}
\@ifundefined{c@chapter}{\newfloat{codelisting}{h}{lop}}{\newfloat{codelisting}{h}{lop}[chapter]}
\floatname{codelisting}{Listing}
\newcommand*\listoflistings{\listof{codelisting}{List of Listings}}
\makeatother
\makeatletter
\makeatother
\makeatletter
\@ifpackageloaded{caption}{}{\usepackage{caption}}
\@ifpackageloaded{subcaption}{}{\usepackage{subcaption}}
\makeatother
\makeatletter
\@ifpackageloaded{sidenotes}{}{\usepackage{sidenotes}}
\@ifpackageloaded{marginnote}{}{\usepackage{marginnote}}
\makeatother
\ifLuaTeX
  \usepackage{selnolig}  % disable illegal ligatures
\fi
\IfFileExists{bookmark.sty}{\usepackage{bookmark}}{\usepackage{hyperref}}
\IfFileExists{xurl.sty}{\usepackage{xurl}}{} % add URL line breaks if available
\urlstyle{same} % disable monospaced font for URLs
\hypersetup{
  pdftitle={Many Many Problems},
  pdfauthor={Brian Weatherson},
  colorlinks=true,
  linkcolor={black},
  filecolor={Maroon},
  citecolor={Blue},
  urlcolor={Blue},
  pdfcreator={LaTeX via pandoc}}

\title{Many Many Problems}
\author{Brian Weatherson}
\date{2003-08-29}

\begin{document}
\maketitle
\subsection{Schiffer's Problem}\label{schiffers-problem}

Stephen Schiffer suggests the following argument refutes
supervaluationism. The central point is that, allegedly, the
supervaluational theory of vague singular terms says false things about
singular terms in speech reports.

\marginnote{\begin{footnotesize}

Published in \emph{Philosophical Quarterly} 55: 481-501.

Photo by \href{https://www.flickr.com/photos/25570425@N07}{Yoni Lerner}
via
\href{https://search.creativecommons.org/photos/39a2c491-c2e6-4454-92e8-cb45e7499d30}{Creative
Commons}.

\end{footnotesize}}

\begin{quote}
Pointing in a certain direction, Alice says to Bob, `There is where
Harold and I first danced the rumba.' Later that day, while pointing in
the same direction, Bob says to Carla, `There is where Alice said she
and Harold first danced the rumba.' Now consider the following argument:

Bob's utterance was true.

If the supervaluational semantics were correct, Bob's utterance wouldn't
be true.

\(\therefore\) The supervaluational semantics isn't correct.
(\citeproc{ref-Schiffer2000b}{Schiffer 2000, 321})
\end{quote}

Assuming Bob did point in pretty much the same direction as Alice, it
seems implausible to deny (1). The argument is valid. So the issue is
whether (2) is correct. Schiffer has a quick argument for (2), which I
will paraphrase here. On supervaluational semantics, a sentence is true
iff each of its acceptable precisifications is true. In this case, this
means that if Bob's utterance is true then it must be true however we
precisify `there'. Each precisification of `there' will be a (precise)
place, and since `there' is rather vague, many of these precisifications
will be acceptable. For Bob's utterance to be true, then, Alice must
have said of every one of those places that it was the place where
Harold and her first danced the rumba. But Alice couldn't have said all
those things, so (2) is true.

Schiffer suggests that one way out of this problem would be to accept
the existence of a vague object: the place where Harold and Alice first
danced the rumba. I will note in section four several reasons for
thinking the cost of this move is excessive. Fortunately, there is a
cheaper way home.

Schiffer underestimates the scope of supervaluationism. On Schiffer's
vision of the theory, a precisification assigns a precise content to a
word, and hence to a sentence, then the world determines whether that
content is satisfied, and hence whether the sentence is true on that
precisification. This is hardly an unorthodox view of how
supervaluationism works, it seems for instance to be exactly the view
defended in Keefe (\citeproc{ref-Keefe2000}{2000}), but it is neither
the only way, nor the best way, forward. We could say, rather, that a
precisification assigns content to \emph{every linguistic token in the
world}, and the truth conditions of every one of these tokens is then
determined relative to that global assignment of content. So if a
precisification \emph{P} assigns a place \emph{x} to Bob's word `there',
Bob's utterance is true according to that precisification iff \emph{P}
also assigns \emph{x} to Alice's utterance of `there'. That is, Bob's
utterance is true according to \emph{P} iff the precisification of his
words by \emph{P} just is what Alice said \emph{according to
P}.\sidenote{\footnotesize Following Schiffer, we ignore the vagueness in `is where
  Harold and I first danced the rumba.' This phrase is vague, but its
  vagueness raises no extra issues of philosophical importance.}

It is a dramatic widening of the scope of precisifications to claim that
they assign content to every linguistic token in the world, rather than
just words in the sentence under consideration, but it can be
justified.\sidenote{\footnotesize Thanks to John Hawthorne for the following argument.}
Consider how we would react if later in the day, pointing in the crucial
direction, Alice said, `Harold and I never danced the rumba there.' We
would think that Alice had contradicted herself -- that between her two
statements she must have said something false. A standard
supervaluationist account, where sentences are precisified one at a
time, cannot deliver this result. On such a view, it might be that each
of Alice's utterances are true on some precisifications, so they are
both neither true nor false. On my theory, each precisification applies
to both of Alice's utterances (as well as every other utterance ever
made) and since on each precisification one or other of the utterances
is false, it turns out supertrue that Alice said something false, as
desired. The current view allows for penumbral connections between
sentences, as well as penumbral connections within sentences. Just as
someone who says, ``That is red and orange'' says something false, my
theory decrees that someone who says, ``That is red. That is orange,''
while pointing at the same thing says something false, even if the
object is in the vague area `between' red and orange.

It is crucial for this response to work that on every precisification,
Alice and Bob's demonstratives are co--referential. It does not seem
like a particular expansion of supervaluational theory to posit this as
a penumbral connection between the two words. At least, it seems
plausible enough to do this if Alice and Bob really are pointing in a
similar direction. If their demonstrations are only \emph{roughly}
co-directional, then on some precisifications they may well pick out
different objects. This will definitely happen if some admissible
precisification of Alice's `there' is not an admissible precisification
of Bob's `there'. In such a case, the theory here predicts that Bob's
utterance will be indeterminate in truth value. But if Alice and Bob
only vaguely pointed in the same direction this is the correct
prediction.

\subsection{Natural Properties}\label{natural-properties}

Schiffer's problem seems to have been solved with a minimum of fuss, but
there is still a little work to do. Above I posited a penumbral
connection between Alice's and Bob's words without \emph{explaining} how
such a connection could arise. This connection can be explained by some
general considerations about content, considerations closely tied to the
view of vagueness as semantic indecision that provides the best
motivation for supervaluationism. As a few writers have pointed out
(\citeproc{ref-Quine1960}{Quine 1960}; \citeproc{ref-Putnam1980}{Putnam
1980}; \citeproc{ref-Kripke1982}{Kripke 1982}), there is not enough in
our dispositions to use words to fix a precise content all terms in our
lexicon. This does not immediately imply a thorough-going content
scepticism because, as a few writers have also pointed out
(\citeproc{ref-Putnam1973}{Putnam 1973};
\citeproc{ref-Kripke1980}{Kripke 1980}; \citeproc{ref-Lewis1983e}{Lewis
1983}, \citeproc{ref-Lewis1984a}{1984}), meanings ain't (entirely) in
the head. Sometimes our words refer to a particular property or object
rather than another not because our dispositions make this so, but
because of some particular feature of that property or object. David
Lewis calls this extra feature `naturalness': some properties and
objects are more natural than others, and when our verbal dispositions
do not discriminate between different possible contents, naturalness
steps in to finish the job and the more natural property or object gets
to be the content.

Well, that's what happens when things go well. Vagueness happens when
things don't go well. Sometimes our verbal dispositions are
indiscriminate between several different contents, and no one of these
is more natural than all the rest. In these cases there will be many
unnatural contents not eliminated by our dispositions that naturalness
does manage to eliminate, but there will be still be many contents left
uneliminated. Consider, for example, all the possible properties we
\emph{might} denote by `tall woman'. As far as our usage dispositions
go, it might denote any one of the following properties: woman taller
than 1680mm, woman taller than 1681mm, woman taller than 1680.719mm,
\emph{etc}. And it does not seem that any of these properties are more
natural than any other. Hence there is no precise fact about what the
phrase denotes. Hence it is vague. In sum, our dispositions are never
enough to settle the content of a term. In some cases, such as `water',
`rabbit', `plus', `brain' and `vat', nature is kind enough to, more or
less, finish the job. In others it is not, and vagueness is the result.

(The above reasoning has a surprising consequence. Perhaps our verbal
dispositions are consistent with the predicate \emph{Tall X} denoting
the property of being in the top quartile of \emph{X}s by height. Unlike
each of the properties mentioned in the text, this is a more natural
property than many of its competitors. So if this kind of approach to
vagueness is right, there \emph{might} not be quite as much vagueness as
we expected.)

If this is how vagueness is created, then there is a natural way to
understand how precisifications remove vagueness. Vagueness arises
because \emph{more natural than} is a partial order on putative
contents, and hence there might be no most natural content consistent
with our verbal dispositions. If this relation only defined a strict
ordering, so whatever the candidate meanings were, one of them would be
most natural, vagueness might be defeated. Well, that isn't true in
reality, but it is true on each precisification. Every precisification
is a completion of the `naturalness' partial order. That is, each
precisification \emph{P} defines a strict order, \emph{more natural-P
than}, on possible contents of terms such that \emph{o}\textsubscript{1}
is more natural-\emph{P} than \emph{o}\textsubscript{2} if (but not only
if) \emph{o}\textsubscript{1} is more natural than
\emph{o}\textsubscript{2}. The particular contents of terms according to
\emph{P} is then defined by using the more natural-\emph{P} than
relation where the more natural than relation is used in the real theory
of content.

This conjecture meshes nicely with my theory of the role of
precisifications. First, it explains why precisifications apply to the
whole of language. Since a precisification does not just remedy a defect
in a particular word, but a defect in the content generation mechanism,
precisifications are most naturally applied not just to a single word,
but to every contentful entity. Secondly, it explains why we have the
particular penumbral connections we actually have. Recall that it was
left a little unexplained above why Alice's and Bob's use of `there'
denoted the same precise place. On the current conjecture, Alice's term
refers to a particular place \emph{x} according to \emph{P} because
\emph{x} is more natural--\emph{P} than all the other places to which
Alice might have referred. If this is so, then \emph{x} will be more
natural--\emph{P} than all the other places to which Bob might have
referred, so it will also be the referent according to \emph{P} of Bob's
there. Hence according to every precisification, Bob's utterance will be
true, as Schiffer required.

We can also explain some other unexplained penumbral connections by
appeal to naturalness. Consider the sentence \emph{David Chalmers is
conscious}. Unless this is supertrue, supervaluationism is in trouble.
It is vague just which object is denoted by \emph{David Chalmers}. On
every precisification, there are other objects that massively overlap
David Chalmers. Indeed, these very objects are denoted by `David
Chalmers' on other precisifications. These objects are not conscious,
since if one did there would be two conscious objects where,
intuitively, there is just one. But each of these rogue objects must be
in the extension of `conscious' on the precisifications where it is the
denotation of `David Chalmers'. So `conscious' must be vague in slightly
unexpected ways, and there must be a penumbral connection between it and
`David Chalmers': on every precisification, whatever object is denoted
by that name is in the extension of `conscious', while no other
potential denotata of `David Chalmers' is in the extension. How is this
penumbral connection to be explained? Not by appeal to the meanings of
the terms! Even if `David Chalmers' has descriptive content, it is
highly implausible that this includes \emph{being conscious}. (After
all, unless medicine improves a bit in a thousand years Chalmers will
not be conscious.) Rather, this penumbral connection is explained by the
fact that the very same thing, naturalness, is used in resolving the
vagueness in the terms `conscious' and `David Chalmers'. If the
precisification makes one particular possible precisification of `David
Chalmers', say \emph{d}\textsubscript{1}, more natural than another,
\emph{d}\textsubscript{2}, then it will make properties satisfied by
\emph{d}\textsubscript{1­} more natural than those satisfied by
\emph{d}\textsubscript{2}, so every precisification will make the
denotation of `David Chalmers' fall into the extension of `conscious'.

We can say the same thing about Alice's original statement: \emph{That
is where Harold and I first danced the rumba.} Since one can't
\emph{first} dance the rumba with Harold in two different places, it
seems Alice's statement can't be true relative to more than one
precisification of `That'. But really the phrase after `is' is also
vague, and there is a penumbral connection (via naturalness) between it
and the demonstrative. Hence we can say Alice's statement is supertrue
without appealing to any \emph{mysterious} penumbral connections.

\subsection{McGee and McLaughlin's
Challenge}\label{mcgee-and-mclaughlins-challenge}

Vann McGee and Brian McLaughlin (\citeproc{ref-McGee2000}{2000}) raise a
challenge for supervlauational approaches to the Problem of the Many
that uses belief reports in much the way that Schiffer's problem uses
speech reports. They fear that without further development, the
supervaluational theory cannot distinguish between the \emph{de re} and
\emph{de dicto} readings of (4).

\begin{enumerate}
\def\labelenumi{\arabic{enumi}.}
\tightlist
\item
  Ralph believes that there is a snow-capped mountain within sight of
  the equator.
\end{enumerate}

They claim, correctly, that (4) should have both a \emph{de dicto}
reading and a \emph{de re} reading, where in the latter case it is a
belief about Kilimanjaro. The problem with the latter case is unclear
how Ralph's belief can be about Kilimanjaro itself. To press the point,
they consider an atom at or around the base of Kilimanjaro, called
Sparky, and define ``Kilimanjaro(+) to be the body of land constituted
\ldots{} by the atoms that make up Kilimanjaro together with Sparky
{[}and{]} Kilimanjaro(-) {[}to{]} be the body of land constituted
\ldots{} by the atoms that make up Kilimanjaro other than Sparky.''
(129) The problem with taking (4) to be true on a \emph{de re} reading
is that ``there isn't anything, either in his mental state or in his
neural state or in his causal relations with his environment that would
make one of Kilimanjaro(+) and Kilimanjaro(-), rather than the other,
the thing that Ralph's belief is about.'' (146) So if the truth of (4)
on a \emph{de re} reading requires that Ralph believes a singular, or
object-dependent, proposition, about one of Kilimanjaro(+) and
Kilimanjaro(-), then (4) cannot be true. Even worse, if the truth of (4)
requires that Ralph both that Ralph believes a singular proposition
about Kilimanjaro(+), that it is a snow-capped mountain within sight of
the equator, and the same proposition about Kilimanjaro(-), then given
some knowledge about mountains on Ralph's part, (4) cannot be true,
because that would require Ralph to mistakenly believe there are two
mountains located roughly where Kilimanjaro is located.

We should not be so easily dissuaded. It is hard to identify exactly
which features of Ralph's ``mental state or neural state or causal
relations with his environment'' that make it the case that he believes
that two plus two equals four, but does not believe that two quus two
equals four. (I assume Ralph is no philosopher, so lacks the concept
QUUS.) I doubt, for example, that the concept PLUS has some causal
influence over Ralph that the concept QUUS lacks. But Ralph does have
the belief involving PLUS, and not the belief involving QUUS. He has
this belief not merely in virtue of his mental or neural states, or his
causal interactions with his environment, but in virtue of the fact that
PLUS is a more natural concept than QUUS, and hence is more eligible to
be a constituent of his belief.

So if Kilimanjaro(+) is more natural than Kilimanjaro(-), it will be a
constituent of Ralph's belief, despite the fact that there is no other
reason to say his belief is about one rather than the other. Now, in
reality Kilimanjaro(+) is no more natural than Kilimanjaro(-). But
according to any precisification, one of them will be more natural than
the other, for precisifications determine content by determining
relative naturalness. Hence if Ralph has a belief with the right
structure, in particular a belief with a place for an object (roughly,
Kilimanjaro) and the property \emph{being within sight of the equator},
then on every precisification he has a singular belief that a
Kilimanjaro-like mountain is within sight of the equator. And notice
that since naturalness determines both mental content and verbal
content, on every precisification the constituent of that belief will be
the referent of `Kilimanjaro'. So even on a \emph{de re} reading, (4)
will be true.

Schiffer's problem showed that we should not take precisifications to be
defined merely over single sentences. McGee and McLaughlin's problem
shows that we should take precisifications to set the content not just
of sentences, but of mental states as well. Precisifications do not just
assign precise content to every contentful linguistic token, but to
every contentful entity in the world, including beliefs. This makes the
issue of penumbral connections that we discussed in section two rather
pressing. We already noted the need to establish penumbral connections
between separate uses of demonstratives. Now we must establish penumbral
connections between words and beliefs. The idea that precisifications
determine content by determining relative naturalness establishes these
connections.

To sum up, McGee and McLaughlin raise three related problems concerning
\emph{de re} belief. Two of these concern belief reports. First, how can
we distinguish between \emph{de re} and \emph{de dicto} reports? If I am
right, we can distinguish between these just the way Russell suggested,
by specifying the scope of the quantifiers. McGee and McLaughlin suspect
this will not work because in general we cannot argue from (5) to (6),
given the vagueness of `Kilimanjaro'.

\begin{enumerate}
\def\labelenumi{\arabic{enumi}.}
\item
  Kilimanjaro is such that Ralph believes it to be within sight of the
  equator.
\item
  There is a mountain such that Ralph believes it to be within sight of
  the equator.
\end{enumerate}

Whether or not we want to accept a semantics in which we must restrict
existential generalisation in this way as a general rule, we can give an
independent argument that (6) is true whenever (4) is true on a \emph{de
re} reading (i.e.~whenever (5) is true). The argument is just that on
every precisification, the subject of Ralph's salient singular belief is
a mountain, so (6) is true on every precisification. This argument
assumes that there is a penumbral connection between the subject of this
belief, as we might say the referent of `Kilimanjaro' in his language of
thought\sidenote{\footnotesize I do not mean here to commit myself to anything like
  the language of thought hypothesis. This is just being used as a
  convenient shorthand.}, and the word `mountain'. But since we have
already established that there is such a connection between
`Kilimanjaro' in his language of thought and `Kilimanjaro' in public
language, and there is obviously a connection between `Kilimanjaro' in
public language and the word `mountain', as `Kilimanjaro is a mountain'
is supertrue, this assumption is safe. So the second puzzle McGee and
McLaughlin raise, how it can be that the relevant \emph{de re} reports
can be \emph{true}, has also been addressed.

There is a third puzzle McGee and McLaughlin raise that the reader might
think I have not addressed. How can it be that Ralph can actually have a
\emph{de re} belief concerning Kilimanjaro? I have so far concentrated
on belief reports, not merely on beliefs, and my theory has relied
crucially on correlations between the vagueness in these reports and the
vagueness in the underlying belief. It might be thought that I have
excluded the most interesting case, the one where Ralph has a particular
belief with Kilimanjaro itself as a constituent. While I will end up
denying Ralph can have such a belief, I doubt this a problematic feature
of my view. The theory outlined here denies that Ralph has
object--dependent beliefs, but not that he has \emph{de re} beliefs. I
deny that Ralph has a belief that has Kilimanjaro(+) as a constituent,
but it is hard to see how Ralph \emph{could} have such a belief, since
it very hard to see how he could have had a belief that has
Kilimanjaro(+) rather than Kilimanjaro(-) as its subject. (This was
McGee and McLaughlin's fundamental point.) If we think that having a
\emph{de re} belief implies having a belief whose content is an
object--dependent proposition, then we must deny that there are \emph{de
re} beliefs about Kilimanjaro. Since there is no object that is
determinately a constituent of the proposition Ralph believes, it is a
little hard to maintain that he believes an object--dependent
proposition.\sidenote{\footnotesize This is hard, but not perhaps impossible. One
  might say that on every precisification, Ralph believes a proposition
  that has a mountain as a constituent, and hence as an essential part.}
But this is not the only way to make sense of \emph{de re} beliefs.

Robin Jeshion has argued that whether a belief is \emph{de re} depends
essentially on its role in cognition. ``What distinguishes \emph{de re}
thought is its structural or organisational role in thought'' {[}Jeshion
(\citeproc{ref-Jeshion2002}{2002}) 67{]}\sidenote{\footnotesize I don't know if
  Jeshion would accept the corollary that if belief is too unstructured
  to allow for the possibility of such organisational roles, then there
  is no \emph{de re} belief, but I do.} I won't rehearse Jeshion's
arguments here, just their more interesting conclusions. We can have
\emph{de re} beliefs about an object iff we have a certain kind of
mental file folder for the object. This folder need not be generated by
acquaintance with the object, so acquiantanceless \emph{de re} belief is
possible. Indeed, the folder could have been created defectively, so
there is no object that the information in the folder is
about.\sidenote{\footnotesize Which is not just to say that there is no object that
  has all the properties in the folder. This is neither necessary nor
  sufficient for the folder to be about the object, as Kripke's
  discussion of `famous deeds' descriptivism should make clear.} In this
case, the contents of the folder are subjectless \emph{de re} beliefs.
Jeshion doesn't discuss this, but presumably the folder must not have
been created purely to be the repository for information about the
bearer of a certain property, whoever or whatever that is. We have to
rule out this option if we follow Szabó (\citeproc{ref-Szabo2000}{2000})
in thinking the folder metaphor plays a crucial role in explaining our
talk and thought involving descriptions. Provided the folder was created
with the intent that it record information about some object, rather
than merely information about whatever object has a particular property,
its contents are \emph{de re} beliefs. (To allow for distinct folders
`about' non-existent objects, we must allow that it is possible that
such folders \emph{do} have their reference fixed by their contents, but
as long as this was not the intent in creation these folders can suffice
for \emph{de re} belief. This point lets us distinguish between my
folder for \emph{Vulcan} and my folder for \emph{The planet causing the
perturbations of Mercury}. Both are individuated by the fact that they
contain the proposition \emph{This} \emph{causes the perturbations of
Mercury}. It is this feature of the folder that fixes their reference,
or in this case their non-reference. Only in the latter case, however,
was this the intent in creating the folder, so its contents are \emph{de
dicto} beliefs, while the contents of the former are \emph{de re}
beliefs.)

Now we have the resources to show how Ralph can have \emph{de re}
beliefs concerning Kilimanjaro. When Ralph hears about it, or sees it,
he opens a file folder for Kilimanjaro. This is not intended to merely
be a folder for the mountain he just heard about, or saw. It is intended
to be a folder for \emph{that}. (Imagine here that I am demonstrating
the mountain in question.) The Kripkenstein point about referential
indeterminacy applies to folders as much as to words. This point is
closely related to Kripke's insistence that his indeterminacy argument
does not rely on behaviourism. So if Ralph's folder is to have a
reference, it must be fixed in part by the naturalness of various
putative referents. But that is consistent with Ralph's folder
containing \emph{de re} beliefs, since unless Ralph is a certain odd
kind of philosopher, he will not have in his folder that Kilimanjaro is
peculiarly eligible to be a referent. So the referent of the folder is
not fixed by its contents (as the referent for a folder about \emph{The
mountain over there, whatever it is}, would be, or how the referent for
a folder about \emph{The natural object over there, whatever it is},
would be), and the contents of this folder are still \emph{de re}
beliefs Ralph has about Kilimanjaro. This was a bit roundabout, but we
have seen that the Problem of the Many threatens neither the possibility
that Ralph is the subject of true \emph{de re} belief ascriptions, nor
that he actually has \emph{de re} beliefs.

\subsection{Vague Objects}\label{vague-objects}

\begin{quote}
``I think the principle that to be is to be determinate is \emph{a
priori}, and hence that it is \emph{a priori} that there is no \emph{de
re} vagueness''. (\citeproc{ref-Jackson2001}{Jackson 2001, 657--58})
\end{quote}

So do I. I also think there are a few arguments for this claim, though
some of them may seem question-begging to the determined defender of
indeterminate objects. Most of these arguments I will just mention,
since I assume the reader has little desire to see them detailed
\emph{again}. One argument is just that it is obvious that there is no
\emph{de re} vagueness. Such `arguments' are not worthless. The best
argument that there are no true contradictions is of just this form, as
Priest (\citeproc{ref-Priest1998}{1998}) shows. And it's a good
argument! Secondly, Russell's point that \emph{most} arguments for
\emph{de re} vagueness involve confusing what is represented with its
representation still seems fair (\citeproc{ref-Russell1923}{Russell
1923}). Thirdly, even though the literature on this is a rather large,
it still looks like the Evans-Salmon argument against vague identities
works, at least under the interpretation David Lewis gives it, and this
makes it hard to see how there could be vague objects
(\citeproc{ref-Evans1978}{Evans 1978}; \citeproc{ref-Salmon1981}{Salmon
1981}; \citeproc{ref-Lewis1988g}{Lewis 1988}). Fourthly, Mark Heller
(\citeproc{ref-Heller1996}{1996}) argues that we have to allow that
referential terms are semantically vague. He says we have to do so to
explain context dependence but there are a few other explanatory
projects that would do just as well. Since semantic conceptions of
vagueness can explain all the data that are commonly taken to support
ontological vagueness, it seems theoretically unparsimonious to
postulate ontological vagueness too. That's probably enough, but let me
add one more argument to the mix. Accepting that Kilimanjaro is be a
vague material object distinct from both Kilimanjaro(+) and
Kilimanjaro(-) has either metaphysical or logical costs. To prove this,
I derive some rather unpleasant metaphysical conclusions from the
assumption that Kilimanjaro is vague. The proofs will use some
contentious principles of classical logic, but rejecting those, and
hence rejecting classical logic, would be a substantial logical cost.
The most contentious such principle used will be an instance of excluded
middle: Sparky is or is not a part of Kilimanjaro. I also assume that if
for all \emph{x} other than Sparky that \emph{x} is a part of \emph{y}
iff it is a part of \emph{z}, then if Sparky is part of both \emph{y}
and \emph{z}, or part of neither \emph{y} nor \emph{z}, then \emph{y}
and \emph{z} coincide. If someone can contrive a mereological theory
that rejects this principle, it will be immune to these arguments.

It is very plausible that material objects are individuated by the
materials from which they are composed, so any coincident material
objects are identical. Properly understood, that is a good account of
what it is to be material. The problem is getting a proper
understanding. Sider (\citeproc{ref-Sider1996-SIDATW}{1996}) interprets
it as saying that no two non-identical material objects coincide
\emph{right now}. His project ends up running aground over concerns
about sentences involving counting, but his project, of finding a strong
interpretation of the principle is intuitively compelling. David
(\citeproc{ref-Lewis1986a}{Lewis 1986} Ch. 4) defends a slight weaker
version: no two non-identical material objects coincide \emph{at all
times}. Call this the \emph{strong composition principle} (scp). The scp
is (classically) inconsistent with the hypothesis that Kilimanjaro is
vague. If Sparky is part of Kilimanjaro, then Kilimanjaro and
Kilimanjaro(+) always coincide. If Sparky is not part of Kilimanjaro
then Kilimanjaro and Kilimanjaro(-) always coincide. Either way, two
non-identical objects always coincide, which the scp does not allow.

Some think the scp is refuted by Gibbard's example of Lumpl and Goliath
(\citeproc{ref-Gibbard1975}{Gibbard 1975}). The most natural response to
Gibbard's example is to weaken our individuation principle again, this
time to: no two non-identical material objects coincide \emph{in all
worlds} \emph{at all times}. Call this the \emph{weak compositional
principle} (wcp). Since there are worlds in which Goliath is composed of
bronze, but Lumpl is still a lump of clay in those worlds, Lumpl and
Goliath do not refute the wcp. Some may think that even the wcp is too
strong\sidenote{\footnotesize Kit Fine (\citeproc{ref-Fine1994}{1994}) does exactly
  this.}, but most would agree that if vague objects violated the wcp,
that would be a reason to believe they don't exist.

Given a plausible metaphysical principle, which I call Crossover, vague
objects will violate the wcp. As shown above, Kilimanjaro actually
(always) coincides with Kilimanjaro(+) or Kilimanjaro(-), but is not
identical with either. Crossover is the following principle:

\begin{description}
\item[Crossover]
For any actual material objects \emph{x} and \emph{y} there is an object
\emph{z} that coincides with \emph{x} in the actual world and \emph{y}
in all other worlds.
\end{description}

Given that arbitrary fusions exist, Crossover is entailed by, but does
not entail, the doctrine of arbitrary modal parts: that for any object
\emph{o} and world \emph{w}, if \emph{o} exists in \emph{w} then
\emph{o} has a part that only exists in \emph{w}. But Crossover does not
have the most surprising consequence of the doctrine of arbitrary modal
parts: that for any object \emph{o} there is an object that has
essentially all the properties \emph{o} actually has.

Let K1 be the object that coincides with Kilimanjaro in this world and
Kilimanjaro(+) in all other worlds. Let K2 be the object that coincides
with Kilimanjaro in this world and Kilimanjaro(-) in all other worlds.
If Sparky is part of Kilimanjaro then K1 and Kilimanjaro(+) coincide in
all worlds, but they are not identical, since it is determinate that
Sparky is actually part of Kilimanjaro(+) and not determinate that it is
part of K1. If Sparky is not part of Kilimanjaro then K2 and
Kilimanjaro(-) coincide in all worlds, but they are not identical, since
it is determinate that Sparky is not actually part of Kilimanjaro(-) and
not determinate that it is not part of K2. Either way, we have a
violation of the wcp. So the following three claims are (classically)
inconsistent.

\begin{enumerate}
\def\labelenumi{\arabic{enumi}.}
\item
  Crossover.
\item
  The wcp.
\item
  Kilimanjaro is a vague object that indeterminately has Sparky as a
  part.
\end{enumerate}

I think the first two are highly plausible, so accepting (c) is costly.
I already noted the plausibility of the wcp, so the focus should be on
Crossover. On Lewis's account of modality, it is clearly true, as is the
stronger doctrine of arbitrary modal parts. On a fictionalist theory of
modality based on Lewis's account, it is still true, or at least true in
the fiction that we must adopt to make sense of modal talk. So the
principle is not without merits. And dialectically, opposing Crossover
will be problematic for the believer in vague objects. Either an
object's modal profile is determined by its categorical properties or it
isn't. If it is, then the wcp will entail the scp, so by the above
reasoning vague objects will be inconsistent with the wcp. If it is not,
then it is hard to see why an object could not have a completely
arbitrary modal profile, say the profile of some other ordinary material
object. But that means Crossover is true, and again we cannot have both
the wcp and vague objects. Probably the best way out for the believer in
vague objects will be to short-circuit this reasoning by abandoning
classical logic, presumably by declining to endorse the version of
excluded middle with which I started. But that is undoubtedly a costly
move, particularly for a supervaluationist.

\subsection{McKinnon on Coins and
Precisifications}\label{mckinnon-on-coins-and-precisifications}

Most of our discussions of the Problem of the Many relate to the
vagueness in a single singular term, and a single ordinary object. As
McKinnon reminds us, however, there is not just one mountain in the
world, there are many of them, and supervaluationists are obliged to say
plausible things about statements that are about many mountains. Or, to
focus on McKinnon's example, we must not only have a plausible theory of
coins, but of coin exhibitions. These do raise distinctive problems.
Imagine we have an exhibition with, as we would ordinarily say, 2547
coins, each numbered in the catalogue. So to each number \emph{n} there
correspond millions of coin-like entities, coin*s in Sider's helpful
phrase (\citeproc{ref-Sider2001}{Sider 2001}), and each precisification
assigns a coin* to a number. In general, Sider holds that something is
an \emph{F}* iff it has all the properties necessary and sufficient for
being an \emph{F} except the property of not massively overlapping
another \emph{F}. There are some interesting questions about how
independent these assignments can be. If one precisification assigns
coin* \emph{c}\textsubscript{1} to \emph{n}\textsubscript{1}, and
another assigns coin* \emph{c}\textsubscript{2} to
\emph{n}\textsubscript{2} (distinct from \emph{n}\textsubscript{1}) then
is there a guaranteed to be a precisification that assigns both
\emph{c}\textsubscript{1} to \emph{n}\textsubscript{1} and
\emph{c}\textsubscript{2} to \emph{n}\textsubscript{2}? In other words,
may the precisifications of each numeral (construed as a coin
denotation) be \emph{independent} of each other? The following example
suggests not. Say \emph{C\textsubscript{j}} is the set of coin*s that
are possible precisifications of \emph{j}. This set may be vague because
of higher--order vagueness, but set those difficulties aside. If every
member of \emph{C}\textsubscript{1728} has a duplicate in
\emph{C}\textsubscript{1729}, then presumably only precisifications that
assigned duplicates to `1728' and `1729' would be admissible. If the
exhibition has two Edward I pennies on display to show the obverse and
reverse, and miraculously these coins are duplicates, such a situation
will arise.

This case is fanciful, so we don't know whether in reality the
precisifications of the numerals are independent. We probably can't
answer this question, but this is no major concern. McKinnon has found a
question which the supervaluationist should feel a need to answer, but
to which neither answer seems appropriate. Say that a precisification is
\emph{principled} iff there is some not-too-disjunctive property
\emph{F} such that for each numeral \emph{n}, the precisification
assigns to \emph{n} the \emph{F}-est coin* in \emph{C\textsubscript{n}}.
If \emph{F} does not come in degrees, then the precisification assigns
to \emph{n} the \emph{F} in \emph{C\textsubscript{n}}. McKinnon's
question to the supervaluationist is: Are all precisifications
principled? He aims to show either answering `yes' or `no' gets the
supervaluationist in trouble. `Yes' leads to there being too few
precisifications; `No' leads to there being too many. Let us look at
these in order.

I have little to say for now on the first horn of this dilemma.
McKinnon's survey of principled precisifications only considers cases
where \emph{F} is intrinsic, and I postpone for now investigation of
extrinsic principles. Nevertheless, he does show that if \emph{F} must
be intrinsic, then there are not enough principled precisifications to
generate all the indeterminacy our coin exhibit intuitively displays.
The other horn is trickier.

A precisification must not only assign a plausible coin* to each
numeral, it must do so in such a way that respects penumbral
connections. McKinnon thinks that unprincipled, or \emph{arbitrary}
precisifications, will violate (NAD) and (NAS).

\begin{description}
\item[Non-Arbitrary Differences (NAD)]
For any coin and non-coin, there is a principled difference between them
which forms the basis for one being a coin and the other being a
non-coin.
\item[Non-Arbitrary Similarities (NAS)]
For any pair of coins, there is a principled similarity between them
which forms the basis for their both being coins.
\end{description}

McKinnon holds these are true, so they should be true on all
precisifications, but they are not true on unprincipled
precisifications, so unprincipled precisifications are unacceptable. The
motivation for (NAD) and (NAS) is clear. When we list the fundamental
properties of the universe, we will not include \emph{being a coin}.
Coinness doesn't go that deep. So if some things are coins, they must be
so in virtue of their other properties. From this (NAD) and (NAS)
follow.

The last step looks dubious. Consider any coin, for definiteness say the
referent of `1728', and a coin* that massively overlaps it. The coin* is
not a coin, so (a) one of these is a coin and the other is not, and (b)
the minute differences between them cannot form the basis for a
distinction between coins and non-coins. Hence (NAD) and (NAS) fail. At
best, it seems, we can justify the following claims. If something is a
coin* and something else is not, \emph{then} there is a principled
difference between them that makes one of them a coin* and the other
not. Something is a coin iff it is a coin* that does not excessively
overlap a coin. If this is the best we can do at defining `coin', then
the prospects for a \emph{reductive} physicalism about coins might look
a little dim, though this is no threat to a physicalism about coins that
stays neutral on the question of reduction. (I trust no reader is an
anti-physicalist about coins, but it is worth noting how vexing
questions of reduction can get even when questions of physicalism are
settled.)

So I think this example refutes (NAD) and (NAS). Do I beg some questions
here? Well, my counterexample turns crucially on the existence of kinds
of objects, massively overlapping coin*s, that some people reject, and
indeed that some find the most objectionable aspect of the
supervaluationist solution. But this gets the burden of proof the wrong
way around. I was not trying to refute (NAD) and (NAS). I just aimed to
parry an argument based on those principles. I am allowed to appeal to
aspects of my theory in doing so without begging questions. I do not
want to rest too much weight on this point, however, for issues to do
with who bears the burden of proof are rarely easily resolved, so let us
move on.

My main response to McKinnon's dilemma is another dilemma. If the
principled similarities and differences in (NAD) and (NAS) must be
intrinsic properties, then those principles are false, because there is
no principled intrinsic difference between a coin and a token, or a coin
and a medal. If the principled similarities and differences in (NAD) and
(NAS) may be extrinsic properties, then those principles may be true,
but then the argument that there are not enough principled
precisifications fail, since now we must consider precisifications based
on extrinsic principles. Let's look at the two halves of that dilemma in
detail, in order.

A subway token is not a coin. Nor is a medal.\sidenote{\footnotesize Some people I
  have asked think tokens are coins, but no one thinks medals are coins,
  so if you (mistakenly) think tokens are coins, imagine all my
  subsequent arguments are phrased using medals rather than tokens.} But
in their intrinsic respects, subway tokens often resemble certain coins
more than some coins resemble other coins. Imagine we had a Boston
subway token (which looks a bit like an American penny, but larger), an
American penny, a British 20p piece (which is roughly heptagonal) and an
early Australian holey dollar (which has a hole in it). There is no
non-disjunctive classification of these by intrinsic properties that
includes the penny, the 20p piece and the holey dollar in one group, and
the subway token in the other. Any group that includes the penny and the
other coins will include the token as well. So if we restrict attention
to intrinsic similarities and differences, (NAD) and (NAS) are false.

There is a difference between these coins and the subway token. The
coins were produced with the intent of being legal tender, the token was
not. Perhaps we can find a difference between coins and non-coins based
on the intent of their creator.\sidenote{\footnotesize Note that I say little here
  about what the intent of the creator must be. I don't think that the
  intent must always be to create legal tender. A ceremonial coin that
  is created, for example, to be tossed before the start of a sporting
  match is still a coin, although it is not intended to be tender. But
  intent still matters. If someone had made a duplicate of that
  ceremonial coin with the intent of awarding it as a medal to the
  victorious captain, it would be a medal and not a coin.} This might
make (NAD) and (NAS) true. But note that given the theory of
precisifications developed in section 3, on every precisification, one
and only one of the precisifications of `1728' will be the subject of an
intention on the part of its manufacturer. Just which of the objects is
the subject of this intent will vary from precisification to
precisification, but there is only one on every precisification. So we
can say that on every precisification, \emph{the} coin is the one where
the intent of its creator was that it be used in a certain way. Indeed,
on any precisification we may have antecedently thought to have existed,
we can show that precisification to be principled by taking \emph{F} to
be the property \emph{being created with intent of being used in a
coin-like way}.\sidenote{\footnotesize Because of the problems raised in the previous
  footnote, I will not try and say just what this intention amounts to.
  There are complications when (a) the creator is a corporate entity
  rather than an individual and (b) the coins are mass--produced rather
  than produced individually. But since the story is essentially the
  same, I leave the gruesome details out here.} So now we can say that
restricting attention to the principled precisifications does not unduly
delimit the class of precisifications.

Let's sum up. To argue against the possibility of unprincipled
precisifications, McKinnon needed to justify (NAD) and (NAS). But these
are only true when we allow `principled differences' to include
differences in creatorial intent. And if we do that we can see that
every \emph{prima facie} admissible precisification is principled, so we
can give an affirmative answer to McKinnon's question.

It might be objected that this move relies heavily on the fact that for
many artefacts creative intent is constitutive of being the kind of
thing that it is. But a Problem of the Many does not arise only for
artefacts, so my solution does not generalise. This is little reason for
concern since McKinnon's problem does not generalise either. (NAD) and
(NAS) are clearly false when we substitute `mountain' for `coin'.
Consider a fairly typical case where it is indeterminate whether we have
one mountain or two.\sidenote{\footnotesize This case is rather important in the
  history of the problem, because its discussion in Quine
  (\citeproc{ref-Quine1960}{1960}) is one of the earliest presentations
  in print of anything like the problem of the many.} In this case it
might be not clear whether, for example, we have one mountain with a
southern and a northern peak, or two mountains, one of them a little
north of the other. Whether there is one mountain here or two, clearly
the two peaks exist, and their fusion exists too. The real question is
which of these three things is a mountain. However this question is
resolved, a substitution instance of (NAD) with the two objects being
the southern peak and the fusion of the two peaks will be false. So in
this case a relatively unprincipled precisification will be acceptable.
The point here is that mountain*s that are not mountains exist (either
the peaks or their fusion will do as examples), and that suffices to
refute McKinnon's alleged penumbral connections and allow, in this case,
a negative answer to his question.

\subsection{Sorensen on Direct
Reference}\label{sorensen-on-direct-reference}

According to orthodoxy, we can use descriptions to determine the
reference of names without those descriptions becoming part of the
meaning of the name. This, apparently, is what happened when Leverrier
introduced `Neptune' to \emph{name}, not merely describe, the planet
causing certain perturbations, and when someone introduced `Jack the
Ripper' to \emph{name}, not merely describe, the person performing
certain murders. So let us introduce `Acme' as the name for the first
tributary of the river Enigma. As Sorensen suggests, this can create
certain problems.

\begin{quote}
When {[}explorers{]} first travel up the river Enigma they finally reach
the first pair of river branches. They name one branch `Sumo' and the
other `Wilt'. Sumo is shorter but more voluminous than Wilt. This makes
Sumo and Wilt borderline cases of `tributary' \ldots{} `Acme' definitely
refers to something, even though it is vague whether it refers to Sumo
and vague whether it refers to Wilt.
(\citeproc{ref-Sorensen2000}{Sorensen 2000, 180})
\end{quote}

If `Acme', `Sumo' and `Wilt' are all vague names related in this way,
Sorensen thinks the supervaluationist has a problem. The sentences `Acme
is Sumo' and `Acme is Wilt' both express propositions of the form
\(\langle x = y \rangle\). For exactly one of them, \emph{x} is
\emph{y}. Since the proposition contains just the objects \emph{x} and
\emph{y} (and the identity relation) but not their route into the
proposition, there is no vagueness in the proposition. Hence there is no
way to precisify either proposition. So a supervaluationist cannot
explain how these propositions are vague.

This is no problem for supervaluationism, since supervaluationism says
that \emph{sentences}, not \emph{propositions}, are vague. Indeed, most
supervaluationists would say that no proposition is ever vague. Thinking
they are vague is just another instance of the fallacy Russell
identified: attributing properties of the representation to the entity,
in this case a proposition, represented.

But maybe there is a problem in the area. One natural way of spelling
out the idea that names directly refer to objects is to say that the
\emph{meaning} of a name is its referent. And one quite plausible
principle about precisifications is that precisifications must not
\emph{change} the meaning of a term, they may merely provide a meaning
where none exists. Now the supervaluationist has a problem. For it is
true that one of Sorensen's identity sentences is true in virtue of its
meaning, since its meaning determines that it expresses a proposition of
the form \(\langle x = x \rangle\). But each sentence is false on some
precisifications, so some precisifications change the meaning of the
terms involved.

The best way to respond to this objection is simply to bite the bullet.
We can accept that some precisifications alter meanings provided we can
provide some other criteria for acceptability of precisifications. I
offered one such proposal in section 2. An acceptable precisification
takes the partial order \emph{more natural than}, turns it into a
complete order without changing any of the relations that already exist,
and uses this new relation to generate meanings. If we proceed in this
way it is possible, for all we have hitherto said, that on every
precisification the proposition expressed by `Acme is Sumo' will be of
the form \(\langle x = y \rangle\), so just the named object, rather
than the method of naming, gets into the proposition. The central point
is that since precisifications apply to the processes that turn semantic
intentions into meanings, rather than to sentences with meanings, there
is no guarantee they will preserve meanings. But if we like directly
referential theories of names we should think this perfectly natural. If
names are directly referential then Sorensen's argument that there are
vague sentences that are true in virtue of their meaning works. But this
is consistent with supervaluationism.

One challenge remains. If precisifications change meanings, why should
we care about them, or about what is true on all of them? This is not a
new challenge; it is a central plank in Jerry Fodor and Ernest Lepore's
(\citeproc{ref-Fodor1996}{1996}) attack on supervaluationism. A simple
response is just to say that we should care about precisifications
because this method delivers the right results in all core cases, and an
intuitively plausible set of results in contentious cases. This kind of
instrumentalism about the foundations of a theory is not always
satisfying.\sidenote{\footnotesize The largest debate in the history of philosophy of
  economics concerned whether we could, or should, be instrumentalists
  about the ideally rational agents at the core of mainstream
  microeconomic theory. See Friedman (\citeproc{ref-Friedman1953}{1953})
  for the classic statement of the instrumentalist position, and Hausman
  (\citeproc{ref-Hausman1992}{1992}) for the most amusing and
  enlightening of the countably many responses.} But if that's the
biggest problem supervaluationists have, they should be able to sleep a
lot easier than the rest of us.

\subsection{Conclusions and
Confessions}\label{conclusions-and-confessions}

I have spent a fair bit of time arguing that supervaluationism is not
vulnerable to a few challenges based on the Problem of the Many. Despite
doing all this, I don't believe supervaluationism to be quite true. So
why spend this time? Because the true theory of vagueness will be a
\emph{classical semantic} theory, and everything I say about
supervaluationism above applies \emph{mutatis mutandis} to all classical
semantic theories. I focussed on supervaluationism because it is more
familiar and more popular, but I need not have.

What is a classical semantic theory? That's easy - it's a theory that is
both classical and semantic. What is a classical theory? It is one that
incorporates vagueness while preserving classical logic. How much of
classical logic must we preserve? That's a hard question, though it is
relevant to determining whether supervaluationism is (as it is often
advertised) a classical theory. Williamson
(\citeproc{ref-Williamson1994-WILV}{1994}) notes that supervaluationism
does not preserve classical inference rules, and Hyde
(\citeproc{ref-Hyde1997}{1997}) notes that it does not preserve some
classically valid multiple--conclusion sequents. Keefe
(\citeproc{ref-Keefe2000}{2000}) argues that neither of these
constitutes an important deviation from classical logic. I'm inclined to
disagree with Keefe on both points. Following Read (2000), I take it
that the best response to the anti-classical arguments in Dummett
(\citeproc{ref-Dummett1991}{1991}) takes the essential features of
classical logic to be its inferential rules as formulated in a
multiple--conclusion logic. But we need not adjudicate this dispute
here. Why should we want a classical theory? The usual arguments for it
are based on epistemic conservatism, and I think these arguments are
fairly compelling. I also think that no non--classical theory will be
able to provide a plausible account of quantification.\sidenote{\footnotesize See the
  last section of Weatherson
  (\citeproc{ref-Weatherson2005-WEATTT}{2005}) for a detailed defence of
  this claim.}

What is a semantic theory? It is one that makes vagueness a semantic
phenomenon. It is not necessarily one that makes vagueness a
\emph{linguistic} phenomenon. That would be absurd in any case, since
clearly some non--linguistic entities, maps, beliefs and pictures for
example, are vague. But the more general idea that vagueness is a
property only of representations is quite attractive. It links up well
with the theory of content Lewis outlines in ``Languages and Language''
- all Languages (in his technical sense) are precise, vagueness in
natural language is a result of indecision about which Language we are
speaking.

Trenton Merricks (\citeproc{ref-Merricks2001}{2001}) argues against this
picture, claiming that all semantic vagueness (he says `linguistic', but
ignore that) must arise because of metaphysical or epistemic vagueness.
He claims that if (17) is vague, then so is (18), and (18)'s vagueness
must be either metaphysical or semantic.

\begin{enumerate}
\def\labelenumi{\arabic{enumi}.}
\item
  Harry is bald.
\item
  `Bald' describes Harry.
\end{enumerate}

One might question the inference from (17)`s vagueness to (18) - on some
supervaluational theories if (17) is vague then (18) is false. But I
will let that pass, for there is a simpler problem in the argument.
Merricks claims that if (18) is vague, then it is vague whether 'Bald'
has the property \emph{describing Harry}, and this is a kind of
metaphysical vagueness. It is hard to see how this follows. If there is
metaphysical vagueness, there is presumably some object \emph{o} and
some property \emph{F} such that it is vague whether the object has the
property. Presumably the object here is the word `bald' and the property
is \emph{describing Harry}. But words alone do not have properties like
\emph{describing Harry}. At best, words in languages do so. So maybe the
object can be the ordered pair \(\langle \text{`Bald'}, l \rangle\),
where \emph{l} is a language. But which one? Not one of Lewis's
Languages, for then it is determinate whether \textless{}`Bald',
\emph{l}\textgreater{} has the property \emph{describing Harry}. So
maybe a natural language, perhaps English! But it is doubly unclear that
English is an object. First, it is unclear whether we should reify
natural languages to such a degree that we accept that `English' refers
to anything at all. Secondly, if we say `English' does refer, why not
say that it refers to one of Lewis's Languages, thought it is vague
which one? That way we can say that the sentence \emph{`Bald' in English
describes Harry} is vague without there being any object that vaguely
instantiates a property. Now on a supervaluational theory this approach
may have the unwanted consequence that ``English is a precise language''
is true, since it is true on all precisifications. It does not seem that
\emph{this} problem for the supervaluationist generalises to be a
problem for all semantic theories of vagueness, so Merricks has raised
no general problem for semantic theories of vagueness. (The problem for
the supervaluationist here is not new. For some discussion see Lewis's
response, in ``Many, but Almost One'' to the objection, there attributed
to Kripke, that the supervaluationist account makes it true that all
words are precise.)

If we have a classical semantic theory that provides a concept of
determinateness, then we can define acceptable precisifications as
maximal consistent extensions of the set of determinate truths. Given
that, it follows pretty quickly that determinate truth implies truth on
all precisifications. And this is sufficient for the major objections
canvassed above to get a foothold, and hence be worthy of response,
though as we have seen none of them will ultimately succeed. Still, our
theory may differ from supervaluationism in many ways. For one thing, it
might \emph{explain} determinateness in ways quite different from those
in supervaluationism. For example, the theory in Field
(\citeproc{ref-Field2000}{2000}) is a classical semantic
theory\sidenote{\footnotesize At least, it strikes me as a classical semantic theory.
  Ryan Wasserman has tried to convince me that properly understood, it
  is really an epistemic theory. Space prevents a thorough account of
  why I think Field's theory is flawed. Briefly, I think the point in
  Leeds (\citeproc{ref-Leeds2000}{2000}) that Field's concept of a
  numerical degree of belief needs \emph{substantially} more explanation
  than Field gives it can be developed into a conclusive refutation.},
but it clearly goes beyond supervaluational theory because it has an
interesting, if ultimately flawed, explanation of determinateness in
terms of Shafer functions. Other classical semantic theories may differ
from supervaluationism by providing distinctive theories of higher order
vagueness.

The most promising research programs in vagueness are within the
classical semantic framework. Like all research programs, these programs
need a defensive component, to fend off potential refutations and
crisis. This avoids unwanted crises in the program, and as we have seen
here we can learn a bit from seeing how to defend against certain
attacks. There will undoubtedly be more challenges in the time ahead,
but for now the moves in this paper brings the defensive side of the
program up to date.

\phantomsection\label{refs}
\begin{CSLReferences}{1}{0}
\bibitem[\citeproctext]{ref-Dummett1991}
Dummett, Michael. 1991. \emph{The Logical Basis of
Metaphysics}.Cambridge, MA: Harvard.

\bibitem[\citeproctext]{ref-Evans1978}
Evans, Gareth. 1978. {``Can There Be Vague Objects?''} \emph{Analysis}
38 (4): 208. \url{https://doi.org/10.1093/analys/38.4.208}.

\bibitem[\citeproctext]{ref-Field2000}
Field, Hartry. 2000. {``Indeterminacy, Degree of Belief, and Excluded
Middle.''} \emph{No{û}s} 34 (1): 1--30.
\url{https://doi.org/10.1111/0029-4624.00200}.

\bibitem[\citeproctext]{ref-Fine1994}
Fine, Kit. 1994. {``Compounds and Aggregates.''} \emph{No{û}s} 28 (2):
137--58. \url{https://doi.org/10.2307/2216046}.

\bibitem[\citeproctext]{ref-Fodor1996}
Fodor, Jerry A., and Ernest Lepore. 1996. {``What Cannot Be Valuated
Cannot Be Valuated, and It Cannot Be Supervaluated Either.''}
\emph{Journal of Philosophy} 93 (10): 516--35.
\url{https://doi.org/10.5840/jphil1996931013}.

\bibitem[\citeproctext]{ref-Friedman1953}
Friedman, Milton. 1953. {``The Methodology of Positive Economics.''} In
\emph{Essays in Positive Economics}, 3--43. Chicago: University of
Chicago Press.

\bibitem[\citeproctext]{ref-Gibbard1975}
Gibbard, Allan. 1975. {``Contingent Identity.''} \emph{Journal of
Philosophical Logic} 4 (2): 187--221.
\url{https://doi.org/10.1007/bf00693273}.

\bibitem[\citeproctext]{ref-Hausman1992}
Hausman, Daniel. 1992. {``Why Look Under the Hood?''} In \emph{Essays in
Philosophy and Economic Methodology}, 70--73. Cambridge: Cambridge
University Press.

\bibitem[\citeproctext]{ref-Heller1996}
Heller, Mark. 1996. {``Against Metaphysical Vagueness.''}
\emph{Philosophical Perspectives} 10: 177--85.
\url{https://doi.org/10.2307/2216242}.

\bibitem[\citeproctext]{ref-Hyde1997}
Hyde, Dominic. 1997. {``From Heaps and Gaps to Heaps of Gluts.''}
\emph{Mind} 106 (424): 641--60.
\url{https://doi.org/10.1093/mind/106.424.641}.

\bibitem[\citeproctext]{ref-Jackson2001}
Jackson, Frank. 2001. {``Responses.''} \emph{Philosophy and
Phenomenological Research} 62 (3): 653--64.
\url{https://doi.org/10.2307/2653545}.

\bibitem[\citeproctext]{ref-Jeshion2002}
Jeshion, Robin. 2002. {``Acquiantanceless \emph{de Re} Belief'.''} In
\emph{Meaning and Truth: Investigations in Philosophical Semantics},
edited by Joseph Keim Campbell, Michael O'Rourke, and David Shier,
53--74. New York: Seven Bridges Press.

\bibitem[\citeproctext]{ref-Keefe2000}
Keefe, Rosanna. 2000. \emph{Theories of Vagueness}. Cambridge: Cambridge
University Press.

\bibitem[\citeproctext]{ref-Kripke1980}
Kripke, Saul. 1980. \emph{Naming and Necessity}. Cambridge: Harvard
University Press.

\bibitem[\citeproctext]{ref-Kripke1982}
---------. 1982. \emph{Wittgenstein on Rules and Private Language}.
Oxford: Basil Blackwell.

\bibitem[\citeproctext]{ref-Leeds2000}
Leeds, Stephen. 2000. {``A Disquotationalist Looks at Vagueness.''}
\emph{Philosophical Topics} 28 (1): 107--28.
\url{https://doi.org/10.5840/philtopics200028119}.

\bibitem[\citeproctext]{ref-Lewis1983e}
Lewis, David. 1983. {``New Work for a Theory of Universals.''}
\emph{Australasian Journal of Philosophy} 61 (4): 343--77.
\url{https://doi.org/10.1080/00048408312341131}.

\bibitem[\citeproctext]{ref-Lewis1984a}
---------. 1984. {``Devil's Bargains and the Real World.''} In \emph{The
Security Gamble: Deterrence in the Nuclear Age}, edited by Douglas
Maclean, 141--54. Totowa, NJ: Rowman; Allenheld.

\bibitem[\citeproctext]{ref-Lewis1986a}
---------. 1986. \emph{On the Plurality of Worlds}. Oxford: Blackwell
Publishers.

\bibitem[\citeproctext]{ref-Lewis1988g}
---------. 1988. {``Vague Identity: Evans Misunderstood.''}
\emph{Analysis} 48 (3): 128--30.
\url{https://doi.org/10.1093/analys/48.3.128}.

\bibitem[\citeproctext]{ref-McGee2000}
McGee, Vann, and Brian McLaughlin. 2000. {``The Lessons of the Many.''}
\emph{Philosophical Topics} 28 (1): 129--51.
\url{https://doi.org/10.5840/philtopics200028120}.

\bibitem[\citeproctext]{ref-Merricks2001}
Merricks, Trenton. 2001. {``Varieties of Vagueness.''} \emph{Philosophy
and Phenomenological Research} 62 (1): 145--57.
\url{https://doi.org/10.2307/2653593}.

\bibitem[\citeproctext]{ref-Priest1998}
Priest, Graham. 1998. {``What Is so Bad about Contradictions?''}
\emph{Journal of Philosophy} 95 (8): 410--26.
\url{https://doi.org/10.2307/2564636}.

\bibitem[\citeproctext]{ref-Putnam1973}
Putnam, Hilary. 1973. {``Meaning and Reference.''} \emph{Journal of
Philosophy} 70 (19): 699--711. \url{https://doi.org/10.2307/2025079}.

\bibitem[\citeproctext]{ref-Putnam1980}
---------. 1980. {``Models and Reality.''} \emph{Journal of Symbolic
Logic} 45 (3): 464--82. \url{https://doi.org/10.2307/2273415}.

\bibitem[\citeproctext]{ref-Quine1960}
Quine, W. V. O. 1960. \emph{Word and Object}. Cambridge, MA.: MIT Press.

\bibitem[\citeproctext]{ref-Russell1923}
Russell, Bertrand. 1923. {``Vagueness.''} \emph{Australasian Journal of
Philosophy and Psychology} 1 (2): 84--92.
\url{https://doi.org/10.1080/00048402308540623}.

\bibitem[\citeproctext]{ref-Salmon1981}
Salmon, Nathan. 1981. \emph{Reference and Essence}. Princeton: Princeton
University Press.

\bibitem[\citeproctext]{ref-Schiffer2000b}
Schiffer, Stephen. 2000. {``Replies.''} \emph{Philosophical Issues} 10
(1): 320--43. \url{https://doi.org/10.1111/j.1758-2237.2000.tb00029.x}.

\bibitem[\citeproctext]{ref-Sider1996-SIDATW}
Sider, Theodore. 1996. {``All the World's a Stage.''} \emph{Australasian
Journal of Philosophy} 74 (3): 433--53.
\url{https://doi.org/10.1080/00048409612347421}.

\bibitem[\citeproctext]{ref-Sider2001}
---------. 2001. {``Maximality and Intrinsic Properties.''}
\emph{Philosophy and Phenomenological Research} 63 (2): 357--64.
\url{https://doi.org/10.1111/j.1933-1592.2001.tb00109.x}.

\bibitem[\citeproctext]{ref-Sorensen2000}
Sorensen, Roy. 2000. {``Direct Reference and Vague Identity.''}
\emph{Philosophical Topics} 28 (1): 177--94.
\url{https://doi.org/10.5840/philtopics200028123}.

\bibitem[\citeproctext]{ref-Szabo2000}
Szabó, Zoltan Gendler. 2000. {``Descriptions and Uniqueness.''}
\emph{Philosophical Studies} 101 (1): 29--57.
\url{https://doi.org/10.1023/A:1026437211756}.

\bibitem[\citeproctext]{ref-Weatherson2005-WEATTT}
Weatherson, Brian. 2005. {``{True, Truer, Truest}.''}
\emph{Philosophical Studies} 123 (1-2): 47--70.
\url{https://doi.org/10.1007/s11098-004-5218-x}.

\bibitem[\citeproctext]{ref-Williamson1994-WILV}
Williamson, Timothy. 1994. \emph{{Vagueness}}. Routledge.

\end{CSLReferences}



\end{document}
