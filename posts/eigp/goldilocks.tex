% Options for packages loaded elsewhere
\PassOptionsToPackage{unicode}{hyperref}
\PassOptionsToPackage{hyphens}{url}
%
\documentclass[
  10pt,
  letterpaper,
  DIV=11,
  numbers=noendperiod,
  twoside]{scrartcl}

\usepackage{amsmath,amssymb}
\usepackage{setspace}
\usepackage{iftex}
\ifPDFTeX
  \usepackage[T1]{fontenc}
  \usepackage[utf8]{inputenc}
  \usepackage{textcomp} % provide euro and other symbols
\else % if luatex or xetex
  \usepackage{unicode-math}
  \defaultfontfeatures{Scale=MatchLowercase}
  \defaultfontfeatures[\rmfamily]{Ligatures=TeX,Scale=1}
\fi
\usepackage{lmodern}
\ifPDFTeX\else  
    % xetex/luatex font selection
  \setmainfont[ItalicFont=EB Garamond Italic,BoldFont=EB Garamond
Bold]{EB Garamond Math}
  \setsansfont[]{Europa-Bold}
  \setmathfont[]{Garamond-Math}
\fi
% Use upquote if available, for straight quotes in verbatim environments
\IfFileExists{upquote.sty}{\usepackage{upquote}}{}
\IfFileExists{microtype.sty}{% use microtype if available
  \usepackage[]{microtype}
  \UseMicrotypeSet[protrusion]{basicmath} % disable protrusion for tt fonts
}{}
\usepackage{xcolor}
\usepackage[left=1in, right=1in, top=0.8in, bottom=0.8in,
paperheight=9.5in, paperwidth=6.5in, includemp=TRUE, marginparwidth=0in,
marginparsep=0in]{geometry}
\setlength{\emergencystretch}{3em} % prevent overfull lines
\setcounter{secnumdepth}{3}
% Make \paragraph and \subparagraph free-standing
\ifx\paragraph\undefined\else
  \let\oldparagraph\paragraph
  \renewcommand{\paragraph}[1]{\oldparagraph{#1}\mbox{}}
\fi
\ifx\subparagraph\undefined\else
  \let\oldsubparagraph\subparagraph
  \renewcommand{\subparagraph}[1]{\oldsubparagraph{#1}\mbox{}}
\fi


\providecommand{\tightlist}{%
  \setlength{\itemsep}{0pt}\setlength{\parskip}{0pt}}\usepackage{longtable,booktabs,array}
\usepackage{calc} % for calculating minipage widths
% Correct order of tables after \paragraph or \subparagraph
\usepackage{etoolbox}
\makeatletter
\patchcmd\longtable{\par}{\if@noskipsec\mbox{}\fi\par}{}{}
\makeatother
% Allow footnotes in longtable head/foot
\IfFileExists{footnotehyper.sty}{\usepackage{footnotehyper}}{\usepackage{footnote}}
\makesavenoteenv{longtable}
\usepackage{graphicx}
\makeatletter
\def\maxwidth{\ifdim\Gin@nat@width>\linewidth\linewidth\else\Gin@nat@width\fi}
\def\maxheight{\ifdim\Gin@nat@height>\textheight\textheight\else\Gin@nat@height\fi}
\makeatother
% Scale images if necessary, so that they will not overflow the page
% margins by default, and it is still possible to overwrite the defaults
% using explicit options in \includegraphics[width, height, ...]{}
\setkeys{Gin}{width=\maxwidth,height=\maxheight,keepaspectratio}
% Set default figure placement to htbp
\makeatletter
\def\fps@figure{htbp}
\makeatother
% definitions for citeproc citations
\NewDocumentCommand\citeproctext{}{}
\NewDocumentCommand\citeproc{mm}{%
  \begingroup\def\citeproctext{#2}\cite{#1}\endgroup}
\makeatletter
 % allow citations to break across lines
 \let\@cite@ofmt\@firstofone
 % avoid brackets around text for \cite:
 \def\@biblabel#1{}
 \def\@cite#1#2{{#1\if@tempswa , #2\fi}}
\makeatother
\newlength{\cslhangindent}
\setlength{\cslhangindent}{1.5em}
\newlength{\csllabelwidth}
\setlength{\csllabelwidth}{3em}
\newenvironment{CSLReferences}[2] % #1 hanging-indent, #2 entry-spacing
 {\begin{list}{}{%
  \setlength{\itemindent}{0pt}
  \setlength{\leftmargin}{0pt}
  \setlength{\parsep}{0pt}
  % turn on hanging indent if param 1 is 1
  \ifodd #1
   \setlength{\leftmargin}{\cslhangindent}
   \setlength{\itemindent}{-1\cslhangindent}
  \fi
  % set entry spacing
  \setlength{\itemsep}{#2\baselineskip}}}
 {\end{list}}
\usepackage{calc}
\newcommand{\CSLBlock}[1]{\hfill\break\parbox[t]{\linewidth}{\strut\ignorespaces#1\strut}}
\newcommand{\CSLLeftMargin}[1]{\parbox[t]{\csllabelwidth}{\strut#1\strut}}
\newcommand{\CSLRightInline}[1]{\parbox[t]{\linewidth - \csllabelwidth}{\strut#1\strut}}
\newcommand{\CSLIndent}[1]{\hspace{\cslhangindent}#1}

\setlength\heavyrulewidth{0ex}
\setlength\lightrulewidth{0ex}
\usepackage[automark]{scrlayer-scrpage}
\clearpairofpagestyles
\cehead{
  Brian Weatherson
  }
\cohead{
  Explanation, Idealisation and the Goldilocks Problem
  }
\ohead{\bfseries \pagemark}
\cfoot{}
\makeatletter
\newcommand*\NoIndentAfterEnv[1]{%
  \AfterEndEnvironment{#1}{\par\@afterindentfalse\@afterheading}}
\makeatother
\NoIndentAfterEnv{itemize}
\NoIndentAfterEnv{enumerate}
\NoIndentAfterEnv{description}
\NoIndentAfterEnv{quote}
\NoIndentAfterEnv{equation}
\NoIndentAfterEnv{longtable}
\NoIndentAfterEnv{abstract}
\renewenvironment{abstract}
 {\vspace{-1.25cm}
 \quotation\small\noindent\rule{\linewidth}{.5pt}\par\smallskip
 \noindent }
 {\par\noindent\rule{\linewidth}{.5pt}\endquotation}
\KOMAoption{captions}{tableheading}
\makeatletter
\@ifpackageloaded{caption}{}{\usepackage{caption}}
\AtBeginDocument{%
\ifdefined\contentsname
  \renewcommand*\contentsname{Table of contents}
\else
  \newcommand\contentsname{Table of contents}
\fi
\ifdefined\listfigurename
  \renewcommand*\listfigurename{List of Figures}
\else
  \newcommand\listfigurename{List of Figures}
\fi
\ifdefined\listtablename
  \renewcommand*\listtablename{List of Tables}
\else
  \newcommand\listtablename{List of Tables}
\fi
\ifdefined\figurename
  \renewcommand*\figurename{Figure}
\else
  \newcommand\figurename{Figure}
\fi
\ifdefined\tablename
  \renewcommand*\tablename{Table}
\else
  \newcommand\tablename{Table}
\fi
}
\@ifpackageloaded{float}{}{\usepackage{float}}
\floatstyle{ruled}
\@ifundefined{c@chapter}{\newfloat{codelisting}{h}{lop}}{\newfloat{codelisting}{h}{lop}[chapter]}
\floatname{codelisting}{Listing}
\newcommand*\listoflistings{\listof{codelisting}{List of Listings}}
\makeatother
\makeatletter
\makeatother
\makeatletter
\@ifpackageloaded{caption}{}{\usepackage{caption}}
\@ifpackageloaded{subcaption}{}{\usepackage{subcaption}}
\makeatother
\ifLuaTeX
  \usepackage{selnolig}  % disable illegal ligatures
\fi
\usepackage{bookmark}

\IfFileExists{xurl.sty}{\usepackage{xurl}}{} % add URL line breaks if available
\urlstyle{same} % disable monospaced font for URLs
\hypersetup{
  pdftitle={Explanation, Idealisation and the Goldilocks Problem},
  pdfauthor={Brian Weatherson},
  hidelinks,
  pdfcreator={LaTeX via pandoc}}

\title{Explanation, Idealisation and the Goldilocks Problem}
\author{Brian Weatherson}
\date{2012}

\begin{document}
\maketitle
\begin{abstract}
A contribution to a symposium on Michael Strevens's book \emph{Depth}.
\end{abstract}

\setstretch{1.1}
Michael Strevens's book \emph{Depth} is a great achievement.\footnote{All
  page references, unless otherwise noted, are to Strevens
  (\citeproc{ref-Strevens2008}{2008}).} To say anything interesting,
useful and true about explanation requires taking on fundamental issues
in the metaphysics and epistemology of science. So this book not only
tells us a lot about scientific explanation, it has a lot to say about
causation, lawhood, probability and the relation between the physical
and the special sciences. It should be read by anyone interested in any
of those questions, which includes presumably the vast majority of
readers of this journal.

One of its many virtues is that it lets us see more clearly what
questions about explanation, causation, lawhood and so on need
answering, and frames those questions in perspicuous ways. I'm going to
focus on one of these questions, what I'll call the Goldilocks problem.
As it turns out, I'm not going to agree with all the details of
Strevens's answer to this problem, though I suspect that something
\emph{like} his answer is right. At least, I hope something like his
answer is right; if it isn't, I'm not sure where else we can look.

\section{The Goldilocks Problem}\label{the-goldilocks-problem}

Sam has engaged in some unhealthy activity, and is now profusely
vomiting in the bathroom. Here are three things that are true of the
buildup to this unfortunate turn of events.

\begin{enumerate}
\def\labelenumi{\arabic{enumi}.}
\tightlist
\item
  Sam either ate a carton of raw eggs, or drank a bottle of vodka.
\item
  Sam ate a carton of raw eggs.
\item
  Sam ate a carton of raw eggs that were bought at midday.
\end{enumerate}

All three of these claims are interesting things to know about the
buildup to the vomiting. But intuitively, or at least according to my
intuitions, (2) is the best \emph{explanation} of the lot. That's
because intuitively, (1) is too weak, and (3) is too strong, while (2)
is just right.

Let's assume for now these intuitions are correct. We then have the
puzzle of explaining why explanations of moderate strength, like (2),
are strictly better than either weaker explanations, like (1), or
stronger explanations, like (3). Put another way, we have to explain
what makes (2) `just right'. Call this the Goldilocks
problem.\footnote{Strevens calls the problem of how to explain why (2)
  is a better explanation than (1) `the disjunction problem'. Given that
  the problem arises in the context of a theory that aims to explain why
  (2) is better than (3), I think the disjunction problem and the
  Goldilocks problem~are not particularly distinct.}

If the Goldilocks problem~was merely a matter of first-pass intuitions,
then perhaps the right way to solve it would be to explain why we have
quirky intuitons about explanations. But I think we can see that it
turns on deeper features than that.

On the one hand, we want explanations, particularly of single events, to
locate those events in the causal structure of the world. That's why
we're pushed towards saying that (3) is the best explanation of Sam's
current activity. Indeed, in his defence of a causal theory of
explanation, David Lewis (\citeproc{ref-Lewis1986f}{1986}) says that (3)
is really the best explanation, though we might prefer to use, or to
offer, (2) for pragmatic reasons.

On the other hand, we want explanations that unify disparate phenomena.
If we see that an event is just one instance of the right kind of
pattern, it feels more explicable. That pushes us towards explanations
that encompass more and more actual and possible outcomes. This pushes
us away from (3) as an explanation, and towards (2), but also away from
(2) and towards (1). After all, if we accepted (1) as the best
explanation for what's going on, we would have an explanation that
encompasses even more events.\footnote{For more on explanation as
  unification, see Friedman (\citeproc{ref-Friedman1974}{1974}) and,
  especially, Kitcher (\citeproc{ref-Kitcher1989}{1989}).}

We can also get pushed towards (1) as being the ideal explanation by
considering ways in which (2) is a better explanation than (3). There is
a sense in which some of the information in (3) is \emph{redundant}. No
matter when Sam bought the eggs, the vomiting would have resulted given
that they were eaten. Here is one principle we might draw from that. If
\(E^\prime\) is logically weaker than \(E\), and the outcome \(O\) would
have happened even if \(E^\prime\) had happened but \(E\) had not, then
\(E^\prime\) is a better explanation than \(E\). This will get the right
result that (2) is a better explanation than (3). But it will get the
wrong result that (1) is a better explanation than (2).

To some extent, the observations of the last three paragraphs point to a
solution to the Goldilocks problem. There are virtues that (2) has over
(3), in not being too specific, and over (1), in being specific enough
for the task at hand. But as a moment's reflection will show, attempting
to turn these ideas into a theory is not exactly trivial. It's much too
easy to come up with principles that end up implying that (2) has all
the \emph{vices} of (1) and (3), and is really \emph{worse} than each,
rather than better. (The attempt to use counterfactuals to give a
sufficient condition for superiority of explanation in the last
paragraph is illustrative of how we might end up theorising this way.)
Having a theory of explanation that avoids these traps is both
desirable, and difficult.

\section{Idealisations in
Explanation}\label{idealisations-in-explanation}

Much more familiar than the Goldilocks problem~is the problem of
accounting for the role of idealisations in explanation. Explanations
seem, after all, factive. The sentence \emph{p because q} just entails
both \emph{p} and \(q\). And yet explanations involving idealisations
seem to be false. Here's an illustrative example.

On a busy suburban corner, there are four gas stations.\footnote{`Petrol
  stations' if that fits your dialect better.} Although the price for
which they offer gas fluctuates a lot from day to day, the four usually
have the same price, even to the nearest tenth of a cent. Why might that
be? One might suspect collusion, but we'll stipulate that this is a real
free market, and the stations are actually competing, not colluding.
Another might be that the stations are using `cost-plus' pricing. But in
fact, given the many and varied ways in which the stations (or their
corporate parents) have used derivatives to hedge their costs, the four
actually face very different input costs. And in any case, a `cost-plus'
theory can't explain the fluctuation of prices.

The real explanation is relatively simple. If any station charges a
higher price than its rivals, then no one will come to that station. And
that's something the station desparately wants to avoid. So no station
charges a higher price than the others. And that means they all charge
the same price.

Now why, might we ask, is it that if any station charges a higher price
than its rivals, then no one will come to that station? There's a simple
explanation here too. First, customers know the prices at each of the
four stations, or at least if they don't the cost of getting those
prices is zero. Second, the customers are each utility-maximisers who
prefer having more money to less. And third, the goods that the stations
are offering are perfect substitutes. Those three premises entail that a
station with a higher price than the others will have zero customers.

But just wait! Precisely none of those three premises are perfectly
true. There is some cost in figuring out the prices at each. If there
weren't, we couldn't explain why stations put up such big signs
advertising their prices. The point of those signs is to \emph{reduce}
the cost of acquiring price information. And, as philosophers of
economics never tire of pointing out, customers aren't perfect utility
maximisers. And, finally, the goods aren't perfect substitutes. The
stations might have different queue lengths, or reputations for quality,
or associations with firms that pollute the Gulf of Mexico, and so
on.\footnote{Given the last point, we'd expect that after the BP
  disaster in the Gulf of Mexico, stations weren't too worried about
  being undercut on price by a nearby BP station.}

Strevens has a nice story to tell here about what we should say about
the explanations like the one I just offered. When the explainer says
that, for instance, the cost of acquiring price information is zero, we
should interpret them charitably, and loosely. We should apply the same
principles as we apply when interpreting someone's claim that Brazil is
triangular. The truth-conditional content of the claim is not that the
cost of acquiring price information is \emph{precisely} zero. Rather, it
is that the cost is in a not-too-large range that includes zero. How
large is `not-too-large'? That depends on what the person is trying to
explain? If they are trying to explain the size of gas station signage,
it will be a small range; if they are trying to explain the dynamics of
gas station pricing, it will be somewhat larger.

Strevens's theory here is \emph{hermeneutic}, not \emph{revolutionary}.
He doesn't say that we should replace the explanations that economists
give, which are full of freely available information, perfectly
substitutable goods, utility maximising agents and so on, with
explanations that involve low cost information, highly substitutable
goods, and agents who usually choose high utility outcomes. Rather, he
is saying that the explanations those economists give already involve
low cost (but not necessarily free) information, highly (but not
necessarily perfectly) substitutable goods, and so on. This seems
entirely right to me. Well known results showing the limitations of
human rationality simply don't undermine the stories like the one I told
explaining the correlation between prices at nearby gas stations, even
though a cursory glance at those explanations might appear to involve
appeal to perfectly rational buyers.

Now what happens when we interpret an explanation as saying not that
some value is zero, but that it is near zero? Well, we get an instance
of the Goldilocks problem~back. We could imagine an explanation of the
gas station prices that includes the exact value of the cost of
acquiring information about each station's price. That explanation would
be more precise than the explanation that merely says the cost of
acquiring price information is low. But despite that increase in
precision, it would be a \emph{worse} explanation, and it would be worse
for just the same reason that (3) is a worse explanation than (2). (Of
course, we haven't yet said just what that reason is!)

So puzzles about idealisations in explanation reduce, given Strevens's
nice hermeneutic suggestion, to the Goldilocks problem. That raises the
interest in solving the Goldilocks problem, so let's turn to Strevens's
own solution to it.

\section{The Kairetic Theory of
Explanation}\label{the-kairetic-theory-of-explanation}

I'm going to have to simplify a lot in sketching Strevens's theory of
explanation, but I hope the following offers a not-too-inaccurate
picture. For Strevens, explanations of individual events are causal
models. (Explanations of regularities are basically explanations of the
events that make up the regularity.) A causal model is a valid argument,
whose premises are all true, and whose conclusion is the event to be
explained, such that the conclusion can be derived from the premises
using (more-or-less) nothing but modus ponens, with every such step,
from \emph{C} and \emph{C} \(\rightarrow\) \emph{E} to \emph{E}, being
such that in reality \emph{C} caused \emph{E}. When an argument has this
property, Strevens says that the premises \emph{causally entail} the
conclusion. In practice, these models typically have three (kinds of)
premises: a specification of initial conditions, a law (or set of laws)
linking those conditions to the eventual result, and a `no defeaters'
condition, since the laws in question will usually not guarantee any
outcome.\footnote{For instance, the gravitational law says that there is
  a downward force on my coffee cup, but it doesn't guarantee that it
  moves downwards. And, indeed, there are currently sufficiently many
  forces acting on it that it remains suspended 80 feet above the
  ground. The `no defeaters' clause is intended to rule out such
  mischief.}

There will usually be many such explanations. For instance, we could
start with either (1), (2) or (3), add an appropriate law and a no
defeaters condition, and causally derive that Sam is nauseous. Strevens
then puts two extra conditions on causal models, one of which provides a
ranking of explanations, the other of which is a necessary condition for
an explanation being satisfactory.

The ranking condition is that the weaker the set of initial conditions
are, the better the explanation is. If we weaken the initial conditions,
but can still causally derive the explanandum, then the stronger set of
initial conditions contained redundant information and better
explanations excise redundant information. The condition that some
information is necessary for the causal entailment to go through is what
Strevens calls `the kairetic condition' on explanatory relevance, and
that in turn is why the theory is called a kairetic theory of
explanation.

Once we loosen the specification of the initial conditions, a range of
different possible causal pathways are compatible with the argument
being a causal entailment. The necessary condition Strevens adds is that
these possible pathways must be \emph{coherent}. And he defines cohesion
as ``dynamic contiguity'' (105). That is, if we situate all the possible
causal chains in a possible space, an argument satisfies the cohesion
condition if the set of chains consistent with the argument's premises
causally entailing the conclusion form a contiguous set.

Note that contiguity is not that closely related to a similarity
condition. The set of all possible causal pathways is perfectly
contiguous, although its members are severely dissimilar. On the other
hand, some small sets of causal pathways are not contiguous. So consider
(4) and (5) below. Arguably the set of worlds in which (4) is true is
not contiguous -- there is a disconnect between the worlds where Suzy
throws and the worlds where Billy throws -- while the set of worlds in
which (5) is true is contiguous.

\begin{enumerate}
\def\labelenumi{\arabic{enumi}.}
\setcounter{enumi}{3}
\tightlist
\item
  Either Billy or Suzy threw a brick at the window at exactly \(2\pi\)
  mph.
\item
  Suzy threw a brick at the window at between 5 and 30 mph.
\end{enumerate}

Although the worlds where Suzy throws hard are very dissimilar from the
worlds where Suzy throws softly, there is a chain of worlds connecting
the two. And each member of the chain is very similar to the next
member. That suffices for contiguity.

It's important to what follows that Strevens takes contiguity here to be
\emph{physical} contiguity. That is, two worlds (or causal pathways) are
contiguous iff they are contiguous from the perspective of fundamental
physics. Contiguity is not meant to be something defined in terms of
explanations, and nor is it meant to be contiguity in terms of
properties of interest to the special sciences. This will be important
for what follows.

We're now in a position to see Strevens's solution to the Goldilocks
problem. The detail about when Sam bought the eggs is irrelevant to the
conclusion that Sam is nauseous. As long as the eggs were bought, and
eaten, Sam's nausea will exist. Indeed, its existence will be guaranteed
by a causal law, given the appropriate `no defeaters' condition. So the
kairetic condition says we improve the explanation of Sam's nausea by
dropping the time at which the eggs were bought.\footnote{Of course, if
  we wanted to explain the \emph{time} of Sam's nausea, and not just its
  existence, the extra details in (3) might matter.} Now if we start
with (1), there will still be a causal law that lets us derive Sam's
nausea. But the space of causal pathways consistent with the argument we
generate will not be contiguous. It will contain the worlds where the
eggs cause nausea, and the worlds where the vodka causes nausea, and
nothing in between. So it isn't an eligible explanation. So the kairetic
account predicts, correctly, that the best explanation of Sam's nausea
starts with (2). QED.

\section{Equilibrium Explanations in
Economics}\label{equilibrium-explanations-in-economics}

But there's a difficulty looming for this nice theory. It isn't at all
clear how we're going to generalise it to cover explanations in the
social sciences. It's perhaps easiest to see this if we look at an
example. This example is originally from Hendricks and Porter
(\citeproc{ref-HendricksPorter1988}{1988}), though much of my discussion
of it leans heavily on the exposition in Sutton
(\citeproc{ref-Sutton2000}{2000, 47--56}).

The fact to be explained concerns the amount that oil exploration firms
pay for, and eventually earn from, licences to drill in various tracts
of the Gulf of Mexico. At various times, the government opens up the
rights to drill on new tracts of sea bed. Firms are allowed to make a
single bid for the rights to these tracts, and the highest bid wins.
Some firms that bid have, prior to the opening of the new tract,
drilling rights to some adjacent tract, and some do not. Having drilling
rights to an adjacted tract is useful, because oil deposits tend not to
follow the sharp lines on government surveyors' maps. If you have
already been working on an area adjacent to the one being auctioned, you
have a pretty good idea how much oil that tract contains. If you don't,
then you have to make a guess based on more general features of that
region of the Gulf. The stylised fact to be explained is that firms that
bid on tracts adjacent to their existing tracts made a large profit, on
average, while firms that bid on non-adjacent tracts made no net profit.
(In fact they averaged a small loss, but the amount is close enough to
zero that it's worth treating their net returns as zero.) Why might this
be?

The explanation that Hendricks and Porter offer starts with the
following game, from Wilson (\citeproc{ref-Wilson1967}{1967}). Assume
that two players, \(A\) and \(B\), are bidding on a good of some value
in \([0, 1]\). \(A\) knows exactly how valuable the good is - call this
value \emph{x}. \(B\) has no idea how valuable the good is; her
credences about its possible value are distributed evenly over
\([0, 1]\). Both \(A\) and \(B\) know these facts about each other. What
should each of them do?

Standard game theory has an answer. The game has a single Nash
equilibrium. \(A\) bids \(\frac{x}{2}\), and \(B\) plays a mixed
strategy, randomly choosing a bid from \([0, \frac{1}{2}]\). If each of
them play these strategies, then \(A\) has an expected return of
\(\frac{x^2}{2}\), and \(B\) has an expected return of 0. Moreover,
given each of them is playing those strategies, the other party cannot
do better by changing their strategy. (That's just what it means for the
strategies to form a Nash equilibrium.)

Now Hendricks and Porter go on to suggest that the drilling rights
auctions are more or less like these games, with \(A\)'s role being
filled by the firm with an adjacent tract, and \(B\)'s role by the firm
with no adjacent tract.\footnote{There are a lot of technical details
  I'm suppressing here, many of which Hendricks and Porter take into
  account, and many of which they rightly suppress. Sutton characterises
  Hendricks and Porter's model as having considerably fewer added
  complications to the simple game Wilson develops than Hendricks and
  Porter themselves do. For instance, Sutton suppreses, while Hendricks
  and Porter explicitly consider, the possible efficiency gains
  derivable from owning adjacent tracts, but this only makes a small
  difference to the final result. On the kairetic theory of explanation,
  these simplifications actually improve the explanation considerably.}
If we apply that model, we get plausible results for how much profit the
two kinds of firms should earn, including a nice story about why the
non-adjacent firms earn no profit. Indeed, we even get a satisfactory
(at least to economists) story about why firms without adjacent tracts
continue to bid even though they earn no net profit by doing so. If they
didn't bid, then firms with adjacent tracts could win the bidding by
bidding a penny, and then it would be valuable to bid. In other words,
the only equilibrium solution requires them to bid, even though they get
no gain from it.

There is obviously a bit of work to do to show that this game provides a
good model of Gulf of Mexico auctions. For one thing, we have to show
that we can treat the auction as having effectively two players.
Hendricks and Porter suggest that the behaviour of firms with adjacent
tracts is sufficiently cooperative that this is a legitimate
idealisation. There are other idealisations too, all of which I think
can be fit nicely into Strevens's kairetic story. We have to treat the
firms with adjacent tracts as knowing the value of the tract, when
really they'll only know the approximate value. But it is plausible that
treating their ignorance as being zero-valued, i.e., treating their
knowledge as being perfect, makes no difference to what we need to
explain. Similarly, it is not really true that the other firms have no
idea how valuable the tracts are. But their knowledge levels are close
enough to being represented by a flat probability distribution over the
possible values of the tract that it doesn't make a difference to this
story to model their knowledge more precisely.

(There is an interesting technical point here. Strevens focuses on cases
where the idealisations involve giving some variable a ``zero, infinite
or some other extreme or default value'' (318). In social sciences, one
useful `default' value is that the variable is represented by a flat
probability function over some interval. This will rarely be exactly
right; whether we interpret the probability function metaphysically or
epistemically the `right' function will presumably have some bumps or
kinks in it. But it is an acceptable idealisation.)

So far so good. We started with an interesting set of facts, we found a
nice mathematical model that has the facts as a consequence, and we
argued (or at least hinted at how one could argue) that the deviation
between the model and the facts was irrelevant to the outcome to be
explained. So we've arguably fit a widely accepted economic explanation
into the kairetic framework.

But once we start looking at the details, some problems start to emerge.
Remember that on the kairetic account, explanations must be causal
derivations. It doesn't look at first like we've got any causation in
the economic explanation. But I think that's wrong. After all, there's a
reason why the two types of firms bid they way they do. The structure of
the auction, along with other facts, causes them to make these bids. It
isn't something you'll see highlighted in Hendricks and Porter, but it's
arguable their story is a causal story.

The problem is the `other facts' you need to cite to complete this
causal explanation. Those don't seem to be sufficiently `cohesive' for
Strevens's story to hold up. What we know is that if the actors follow
equilibrium strategies, then we'll get the results that are actually
observed. But why should we think that actors will do just that? There
are several possible reasons; too many reasons it might seem for the
kairetic theory to work.

Possibly the actors are perfectly rational, and perfectly rational
beings play Nash equilibrium strategies.\footnote{The normative claim
  here, that perfectly rational beings play Nash equilibrium strategies,
  seems implausible to me for reasons similar to those set out by
  Stalnaker (\citeproc{ref-Stalnaker1996}{1996},
  \citeproc{ref-Stalnaker1998}{1998},
  \citeproc{ref-Stalnaker1999}{1999}).} Possibly the actors are worried
about their strategies leaking out, and are maximising expected utility
relative to that assumption.\footnote{Note that maximising expected
  utility does not entail playing equilibrium strategies without some
  extra assumption about strategies leaking, since a mixed strategy can
  be part of a unique equilibrium, but can never be uniquely utility
  maximising.} Possibly there are a number of actors playing other
strategies, but they don't tend to survive economically, and so the
statistics are dominated by firms that do survive, and the survivors
generally play equilibrium strategies.\footnote{Philosophers tend to
  overstate how much economists rely on rationality assumptions. One of
  the attractions of game-theoretic explanations is that they don't
  require all the agents to be perfectly rational. After all,
  game-theoretic explanations work well in evolutionary biology, and the
  players there are certainly not perfectly rational. For more on this
  point, and especially on how much work economists do to weaken
  rationality postulates, see Hahn (\citeproc{ref-Hahn1996}{1996}).}
Possibly the firms are run by a lot of game theorists, and ``game theory
is an excellent way of predicting the behaviour of professional game
theorists.''\footnote{The quote is from a blog post by Daniel Davies on
  October 8, 2010. See
  \url{http://d-squareddigest.blogspot.com/2010/10/on-not-being-obliged-to-vote-for.html}.}
More likely, some combination of these four reasons, and even some
others, is causally relevant to the establishment and maintenance of
this equilibrium.

And that is something that's hard to fit into the kairetic framework. We
can show how the background facts about the case (i.e., the risks and
rewards facing the competing oil firms), and a general causal law (i.e.,
that firms tend to end up playing equilibrium strategies) entail the
conclusions that various firms bid on newly released tracts despite
having zero expected profit. The problem is that many distinct causal
pathways are compatible with this loosely described causal structure,
and these pathways are not `cohesive'. So the kairetic theory of
explanation predicts that the explanation offered in Hendricks and
Porter (\citeproc{ref-HendricksPorter1988}{1988}) is not a good
explanation of the observed behaviour in the auctions. That should worry
anyone who either finds it intuitively plausible that it is a good
explanation, or thinks that we should defer somewhat to the salient
experts on what is a good explanation.

\section{Possible Responses}\label{possible-responses}

I think these equilibrium explanations are a challenge to Strevens's
solution to the Goldilocks problem, and I think that's a problem given
the importance of solving the Goldilocks problem~to the broader aims of
the kairetic theory of explanation. But there are a number of ways
Strevens could respond to this challenge. Indeed, we can see three
responses already made in \emph{Depth}. So I'll end by noting why I
don't think those three responses work.

First, it is true that some equilibrium explanations are cohesive in
Strevens's sense. Strevens discusses an example proposed by Elliot Sober
(\citeproc{ref-Sober1983}{1983}). Here is how Strevens describes the
case.

\begin{quote}
Consider a ball released at the inside lip of a basin. The ball rolls
down into, then back and forth inside, the basin, eventually coming to
rest at its lowest point. This will happen no matter what the ball's
release point. \ldots{} Sober claims, quite rightly, that \emph{an}
equilibrium explanation \ldots{} is the best explanation of the ball's
final resting place. (267-8)
\end{quote}

Now there are many ways in which the ball might have reached its
equilibrium state. But note that these ways are all fairly similar to
one another. The ways are, collectively, \emph{cohesive} in just the
sense needed for the kairetic theory.\footnote{Actually, this sentence
  isn't as obviously true as it seems. Strevens's discussion of the case
  brings out some unexpected difficulties in accommodating Sober's claim
  in the kairetic theory. But this doesn't affect my point, which is
  that the case is relatively \emph{easy} for the kairetic theory to
  accommodate.} But this is surely an accident of the example. The ways
in which agents reach a game-theoretic equilibrium are very different
from one another, which makes that case rather unlike the case of a ball
descending to the bottom of a basin. In short, while some equilibrium
explanations will be suitably cohesive many, perhaps even most, will
not.

Second, sometimes we don't want to fully explain why \emph{p} is true,
but merely why \emph{p} is true rather than \(q\), or why \emph{p} is
true given that \(r\) is true. In these cases, Strevens says that we
exploit `explanatory frameworks' (149) which fix certain facts as given
for the purposes of explanation. So we might take \(p \vee q\), or
\(r\), to simply be fixed background facts; part of the framework
relative to which explanations are made. When a proposition is part of
the framework, its presence in the derivation of the intended outcome
does not contribute to incohesiveness (163). So if we say that, for
instance, the fact that games like the tract auction end up at
equilibrium is part of the framework, then the orthodox explanation of,
say, why firms bid despite a zero expected profit, can work. In short,
the story about why firms bid is incohesive, but the story about why
firms bid given that firms play equilibrium strategies is cohesive, and
it is the latter that economists are trying to explain.\footnote{This
  might be reading too much into Strevens's discussion. What he says
  about a related example is that the existence of communicative
  channels within firms is part of the `framework' when making economic
  explanations. I don't know whether he would extend this story to cover
  all means by which firms get to equilibrium.}

Now perhaps that's true of what some economists are doing some of the
time. But it seems too defeatist to me. Part of the appeal of game
theoretic explanations is that they are supposed to explain why we get
to, and stay at, equilibrium. I don't think a practicing economist would
say that they are merely presupposing that players in a game reach
equilibrium, as opposed to offering a theory where that fact falls out
as a nice explanandum. It's true that economists do leave some things in
the framework. They generally assume that economic actors are agents,
while leaving the story about how agency might be physically realised to
other disciplines. But it seems wrong to me to say that all the facts
about how equilibrium is established and preserved are simply framework
questions.

Finally, Strevens notes that we can often refer to causal processes in
explanations without being able to fully describe them. If someone asks
why the temperature in this room stays so even while the temperature in
other rooms fluctuates, I can explain the stability by saying that a
thermostat regulates the temperature. Now at first this might look like
a very incohesive explanation. There are many ways that a thermostat
might work, and they don't form anything like a coherent set. But
perhaps that's the wrong way to take my explanation. We could take the
explanation as \emph{referring} to the particular thermostat that is
present, and the particular way in which it regulates the temperature.
That explanation will be \emph{very} cohesive; indeed, the real worry is
that it is too precise. Of course, I might not be able to
\emph{describe} the process by which the thermostat regulates
temperature. But this is no barrier to my being able to refer to it, any
more than ignorance of chemistry is a barrier to my being able to refer
to H\(_2\)O.

Could this help with the tract auction we are discussing? At first
glance it seems like it might. Perhaps the explanation can simply refer
to the means by which a particular firm ends up playing an equilibrium
strategy, even if it cannot describe that means. But the second glance
is more troubling. Remember that what we're trying to explain here is an
average, not a particular firm's behaviour. And it is meant to be
consistent with the explanations that different firms get to equilibrium
in very different ways. So we can't really just refer to those different
methods; we can only describe what they have in common. And that leaves
us back with an incohesive explanation. Indeed, Strevens notes this
point in a similar context when he says that ``in aggregative and
regularity explanation \ldots~there is a real risk'' that we won't pick
out a cohesive causal mechanism. (154)

So I'm left thinking that we need somehow to supplement the story
Strevens offers to make it plausible as an account of explanation in the
special sciences. The kind of equilibrium explanations game theorists
offer of economic outcomes are at least sometimes good explanations. But
what makes them good is not the cohesiveness of their underlying
physical mechanisms. It is, at least intuitively, the cohesiveness of
the explanations from the perspective of the special science in
question. If that intuition is right, we theorists still have work to do
in characterising this notion of cohesiveness.

\subsection*{References}\label{references}
\addcontentsline{toc}{subsection}{References}

\phantomsection\label{refs}
\begin{CSLReferences}{1}{0}
\bibitem[\citeproctext]{ref-Friedman1974}
Friedman, Michael. 1974. {``Explanation and Scientific Understanding.''}
\emph{Journal of Philosophy} 71 (1): 5--19. doi:
\href{https://doi.org/10.2307/2024924}{10.2307/2024924}.

\bibitem[\citeproctext]{ref-Hahn1996}
Hahn, Frank. 1996. {``Rerum Cognoscere Causas.''} \emph{Economics and
Philosophy} 12 (2): 183--95. doi:
\href{https://doi.org/10.1017/S0266267100004156}{10.1017/S0266267100004156}.

\bibitem[\citeproctext]{ref-HendricksPorter1988}
Hendricks, Kenneth, and Robert H. Porter. 1988. {``An Empirical Study of
an Auction with Asymmetric Information.''} \emph{The American Economic
Review} 78 (5): 865--83.

\bibitem[\citeproctext]{ref-Kitcher1989}
Kitcher, Philip. 1989. {``Explanatory Unification and the Causal
Structure of the World.''} In \emph{Scientific Explanation}, edited by
Philip Kitcher and Wesley Salmon, 13:410--505. Minnesota Studies in
Philosophy of Science. Minneapolis: University of Minnesota Press.

\bibitem[\citeproctext]{ref-Lewis1986f}
Lewis, David. 1986. {``Causal Explanation.''} In \emph{Philosophical
Papers}, II:214--40. Oxford: Oxford University Press.

\bibitem[\citeproctext]{ref-Sober1983}
Sober, Elliot. 1983. {``Equilibrium Explanation.''} \emph{Philosophical
Studies} 43 (2): 201--10. doi:
\href{https://doi.org/10.1007/BF00372383}{10.1007/BF00372383}.

\bibitem[\citeproctext]{ref-Stalnaker1996}
Stalnaker, Robert. 1996. {``Knowledge, Belief and Counterfactual
Reasoning in Games.''} \emph{Economics and Philosophy} 12: 133--63. doi:
\href{https://doi.org/10.1017/S0266267100004132}{10.1017/S0266267100004132}.

\bibitem[\citeproctext]{ref-Stalnaker1998}
---------. 1998. {``Belief Revision in Games: Forward and Backward
Induction.''} \emph{Mathematical Social Sciences} 36 (1): 31--56. doi:
\href{https://doi.org/10.1016/S0165-4896(98)00007-9}{10.1016/S0165-4896(98)00007-9}.

\bibitem[\citeproctext]{ref-Stalnaker1999}
---------. 1999. {``Extensive and Strategic Forms: Games and Models for
Games.''} \emph{Research in Economics} 53 (3): 293--319. doi:
\href{https://doi.org/10.1006/reec.1999.0200}{10.1006/reec.1999.0200}.

\bibitem[\citeproctext]{ref-Strevens2008}
Strevens, Michael. 2008. \emph{Depth: An Account of Scientific
Explanations}. Cambridge, MA: Harvard University Press.

\bibitem[\citeproctext]{ref-Sutton2000}
Sutton, John. 2000. \emph{Marshall's Tendencies: What Can Economists
Know?} Cambridge, MA: {MIT} Press.

\bibitem[\citeproctext]{ref-Wilson1967}
Wilson, Robert B. 1967. {``Competitive Bidding with Asymmetric
Information.''} \emph{Management Science} 13 (11): 816--20. doi:
\href{https://doi.org/10.1287/mnsc.13.11.816}{10.1287/mnsc.13.11.816}.

\end{CSLReferences}



\noindent Published in\emph{
Philosophy and Phenomenological Research}, 2012, pp. 461-473.

\end{document}
