% Options for packages loaded elsewhere
% Options for packages loaded elsewhere
\PassOptionsToPackage{unicode}{hyperref}
\PassOptionsToPackage{hyphens}{url}
%
\documentclass[
  11pt,
  letterpaper,
  DIV=11,
  numbers=noendperiod,
  twoside]{scrartcl}
\usepackage{xcolor}
\usepackage[left=1.1in, right=1in, top=0.8in, bottom=0.8in,
paperheight=9.5in, paperwidth=7in, includemp=TRUE, marginparwidth=0in,
marginparsep=0in]{geometry}
\usepackage{amsmath,amssymb}
\setcounter{secnumdepth}{3}
\usepackage{iftex}
\ifPDFTeX
  \usepackage[T1]{fontenc}
  \usepackage[utf8]{inputenc}
  \usepackage{textcomp} % provide euro and other symbols
\else % if luatex or xetex
  \usepackage{unicode-math} % this also loads fontspec
  \defaultfontfeatures{Scale=MatchLowercase}
  \defaultfontfeatures[\rmfamily]{Ligatures=TeX,Scale=1}
\fi
\usepackage{lmodern}
\ifPDFTeX\else
  % xetex/luatex font selection
  \setmainfont[ItalicFont=EB Garamond Italic,BoldFont=EB Garamond
SemiBold]{EB Garamond Math}
  \setsansfont[]{EB Garamond}
  \setmathfont[]{Garamond-Math}
\fi
% Use upquote if available, for straight quotes in verbatim environments
\IfFileExists{upquote.sty}{\usepackage{upquote}}{}
\IfFileExists{microtype.sty}{% use microtype if available
  \usepackage[]{microtype}
  \UseMicrotypeSet[protrusion]{basicmath} % disable protrusion for tt fonts
}{}
\usepackage{setspace}
% Make \paragraph and \subparagraph free-standing
\makeatletter
\ifx\paragraph\undefined\else
  \let\oldparagraph\paragraph
  \renewcommand{\paragraph}{
    \@ifstar
      \xxxParagraphStar
      \xxxParagraphNoStar
  }
  \newcommand{\xxxParagraphStar}[1]{\oldparagraph*{#1}\mbox{}}
  \newcommand{\xxxParagraphNoStar}[1]{\oldparagraph{#1}\mbox{}}
\fi
\ifx\subparagraph\undefined\else
  \let\oldsubparagraph\subparagraph
  \renewcommand{\subparagraph}{
    \@ifstar
      \xxxSubParagraphStar
      \xxxSubParagraphNoStar
  }
  \newcommand{\xxxSubParagraphStar}[1]{\oldsubparagraph*{#1}\mbox{}}
  \newcommand{\xxxSubParagraphNoStar}[1]{\oldsubparagraph{#1}\mbox{}}
\fi
\makeatother


\usepackage{longtable,booktabs,array}
\usepackage{calc} % for calculating minipage widths
% Correct order of tables after \paragraph or \subparagraph
\usepackage{etoolbox}
\makeatletter
\patchcmd\longtable{\par}{\if@noskipsec\mbox{}\fi\par}{}{}
\makeatother
% Allow footnotes in longtable head/foot
\IfFileExists{footnotehyper.sty}{\usepackage{footnotehyper}}{\usepackage{footnote}}
\makesavenoteenv{longtable}
\usepackage{graphicx}
\makeatletter
\newsavebox\pandoc@box
\newcommand*\pandocbounded[1]{% scales image to fit in text height/width
  \sbox\pandoc@box{#1}%
  \Gscale@div\@tempa{\textheight}{\dimexpr\ht\pandoc@box+\dp\pandoc@box\relax}%
  \Gscale@div\@tempb{\linewidth}{\wd\pandoc@box}%
  \ifdim\@tempb\p@<\@tempa\p@\let\@tempa\@tempb\fi% select the smaller of both
  \ifdim\@tempa\p@<\p@\scalebox{\@tempa}{\usebox\pandoc@box}%
  \else\usebox{\pandoc@box}%
  \fi%
}
% Set default figure placement to htbp
\def\fps@figure{htbp}
\makeatother


% definitions for citeproc citations
\NewDocumentCommand\citeproctext{}{}
\NewDocumentCommand\citeproc{mm}{%
  \begingroup\def\citeproctext{#2}\cite{#1}\endgroup}
\makeatletter
 % allow citations to break across lines
 \let\@cite@ofmt\@firstofone
 % avoid brackets around text for \cite:
 \def\@biblabel#1{}
 \def\@cite#1#2{{#1\if@tempswa , #2\fi}}
\makeatother
\newlength{\cslhangindent}
\setlength{\cslhangindent}{1.5em}
\newlength{\csllabelwidth}
\setlength{\csllabelwidth}{3em}
\newenvironment{CSLReferences}[2] % #1 hanging-indent, #2 entry-spacing
 {\begin{list}{}{%
  \setlength{\itemindent}{0pt}
  \setlength{\leftmargin}{0pt}
  \setlength{\parsep}{0pt}
  % turn on hanging indent if param 1 is 1
  \ifodd #1
   \setlength{\leftmargin}{\cslhangindent}
   \setlength{\itemindent}{-1\cslhangindent}
  \fi
  % set entry spacing
  \setlength{\itemsep}{#2\baselineskip}}}
 {\end{list}}
\usepackage{calc}
\newcommand{\CSLBlock}[1]{\hfill\break\parbox[t]{\linewidth}{\strut\ignorespaces#1\strut}}
\newcommand{\CSLLeftMargin}[1]{\parbox[t]{\csllabelwidth}{\strut#1\strut}}
\newcommand{\CSLRightInline}[1]{\parbox[t]{\linewidth - \csllabelwidth}{\strut#1\strut}}
\newcommand{\CSLIndent}[1]{\hspace{\cslhangindent}#1}



\setlength{\emergencystretch}{3em} % prevent overfull lines

\providecommand{\tightlist}{%
  \setlength{\itemsep}{0pt}\setlength{\parskip}{0pt}}



 


\setlength\heavyrulewidth{0ex}
\setlength\lightrulewidth{0ex}
\usepackage[automark]{scrlayer-scrpage}
\clearpairofpagestyles
\cehead{
  Brian Weatherson
  }
\cohead{
  Intrinsic Properties and Combinatorial Principles
  }
\ohead{\bfseries \pagemark}
\cfoot{}
\makeatletter
\newcommand*\NoIndentAfterEnv[1]{%
  \AfterEndEnvironment{#1}{\par\@afterindentfalse\@afterheading}}
\makeatother
\NoIndentAfterEnv{itemize}
\NoIndentAfterEnv{enumerate}
\NoIndentAfterEnv{description}
\NoIndentAfterEnv{quote}
\NoIndentAfterEnv{equation}
\NoIndentAfterEnv{longtable}
\NoIndentAfterEnv{abstract}
\renewenvironment{abstract}
 {\vspace{-1.25cm}
 \quotation\small\noindent\emph{Abstract}:}
 {\endquotation}
\newfontfamily\tfont{EB Garamond}
\addtokomafont{disposition}{\rmfamily}
\addtokomafont{title}{\normalfont\itshape}
\let\footnoterule\relax

\makeatletter
\renewcommand{\@maketitle}{%
  \newpage
  \null
  \vskip 2em%
  \begin{center}%
  \let \footnote \thanks
    {\itshape\huge\@title \par}%
    \vskip 0.5em%  % Reduced from default
    {\large
      \lineskip 0.3em%  % Reduced from default 0.5em
      \begin{tabular}[t]{c}%
        \@author
      \end{tabular}\par}%
    \vskip 0.5em%  % Reduced from default
    {\large \@date}%
  \end{center}%
  \par
  }
\makeatother
\RequirePackage{lettrine}

\renewenvironment{abstract}
 {\quotation\small\noindent\emph{Abstract}:}
 {\endquotation\vspace{-0.02cm}}
\KOMAoption{captions}{tableheading}
\makeatletter
\@ifpackageloaded{caption}{}{\usepackage{caption}}
\AtBeginDocument{%
\ifdefined\contentsname
  \renewcommand*\contentsname{Table of contents}
\else
  \newcommand\contentsname{Table of contents}
\fi
\ifdefined\listfigurename
  \renewcommand*\listfigurename{List of Figures}
\else
  \newcommand\listfigurename{List of Figures}
\fi
\ifdefined\listtablename
  \renewcommand*\listtablename{List of Tables}
\else
  \newcommand\listtablename{List of Tables}
\fi
\ifdefined\figurename
  \renewcommand*\figurename{Figure}
\else
  \newcommand\figurename{Figure}
\fi
\ifdefined\tablename
  \renewcommand*\tablename{Table}
\else
  \newcommand\tablename{Table}
\fi
}
\@ifpackageloaded{float}{}{\usepackage{float}}
\floatstyle{ruled}
\@ifundefined{c@chapter}{\newfloat{codelisting}{h}{lop}}{\newfloat{codelisting}{h}{lop}[chapter]}
\floatname{codelisting}{Listing}
\newcommand*\listoflistings{\listof{codelisting}{List of Listings}}
\makeatother
\makeatletter
\makeatother
\makeatletter
\@ifpackageloaded{caption}{}{\usepackage{caption}}
\@ifpackageloaded{subcaption}{}{\usepackage{subcaption}}
\makeatother
\usepackage{bookmark}
\IfFileExists{xurl.sty}{\usepackage{xurl}}{} % add URL line breaks if available
\urlstyle{same}
\hypersetup{
  pdftitle={Intrinsic Properties and Combinatorial Principles},
  pdfauthor={Brian Weatherson},
  hidelinks,
  pdfcreator={LaTeX via pandoc}}


\title{Intrinsic Properties and Combinatorial Principles\thanks{Thanks
to David Lewis, Europa Malynicz, Dan Marshall, Daniel Nolan, Josh
Parsons and Ted Sider for helpful discussions.}}
\author{Brian Weatherson}
\date{2001}
\begin{document}
\maketitle
\begin{abstract}
Three objections have recently been levelled at the analysis of
intrinsicness offered by Rae Langton and David Lewis. While these
objections do seem telling against the particular theory Langton and
Lewis offer, they do not threaten the broader strategy Langton and Lewis
adopt: defining intrinsicness in terms of combinatorial features of
properties. I show how to amend their theory to overcome the objections
without abandoning the strategy.
\end{abstract}


\setstretch{1.1}
Three objections have recently been levelled at the analysis of
intrinsicness in Rae Langton and David Lewis's ``Defining
`Intrinsic'\,''. Yablo (\citeproc{ref-Yablo1993}{1999}) has objected
that the theory rests on ``controversial and (apparently) irrelevant''
judgements about the relative naturalness of various properties. Dan
Marshall and Josh Parsons Marshall and Parsons
(\citeproc{ref-Marshall2001}{2001}) have argued that quantification
properties, such as \emph{being accompanied by an cube}, are
counterexamples to Langton and Lewis's theory. And Theodore Sider Sider
(\citeproc{ref-Sider2001}{2001}) has argued that maximal properties,
like \emph{being a rock}, provide counterexamples to the theory. In this
paper I suggest a number of amendments to Langton and Lewis's theory to
overcome these counterexamples. The suggestions are meant to be friendly
in that the basic theory with which we are left shares a structure with
the theory proposed by Langton and Lewis. However, the suggestions are
not meant to be \emph{ad hoc} stipulations designed solely to avoid
theoretical punctures, but developments of principles that follow
naturally from the considerations adduced by Langton and Lewis.

\section{Langton and Lewis's Theory}\label{langton-and-lewiss-theory}

Langton and Lewis base their theory on a combinatorial principle about
intrinsicness. If a property \emph{F} is intrinsic, then whether a
particular object is \emph{F} is independent whether there are other
things in the world. This is just a specific instance of the general
principle that if \emph{F} is intrinsic then whether some particular is
\emph{F} is independent of the way the rest of the world is. So if
\emph{F} is intrinsic, then the following four conditions are met:

\begin{enumerate}
\def\labelenumi{\arabic{enumi}.}
\tightlist
\item
  Some lonely object is \emph{F};
\item
  Some lonely object is not-\emph{F};
\item
  Some accompanied object is \emph{F}; and
\item
  Some accompanied object is not-\emph{F}.
\end{enumerate}

The quantifiers in the conditions range across objects in all possible
worlds, and indeed this will be the quantifier domain in everything that
follows (except where indicated). An object is `lonely' if there are no
wholly distinct contingent things in its world. The effect of including
`distinct' in this definition is that an object can be lonely even if it
has proper parts; an object is not identical with its parts, but nor is
it distinct from them. Following Langton and Lewis, I will say that any
property that meets the four conditions is `independent of
accompaniment'.

All intrinsic properties are independent of accompaniment, but so are
some extrinsic properties. For example, the property \emph{being the
only round thing} is extrinsic, but independent of accompaniment. So
Langton and Lewis do not say that independence of accompaniment is
sufficient for intrinsicness. However, within a certain class of
properties, what we might call the \emph{basic} properties, they do say
that any property independent of accompaniment is intrinsic. A property
is \emph{basic} if it is neither \emph{disjunctive} nor the negation of
a disjunctive property. Langton and Lewis define the disjunctive
properties as follows:

\begin{quote}
{[}L{]}et us define the \emph{disjunctive} properties as those
properties that can be expressed by a disjunction of (conjunctions of)
natural properties; but that are not themselves natural properties. (Or,
if naturalness admits of degrees, they are much less natural than the
disjuncts in terms of which they can be expressed.) Langton and Lewis
(\citeproc{ref-Lewis2001Langton}{2001})
\end{quote}

Langton and Lewis assume here that there is some theory of naturalness
that can be plugged in here, but they are explicitly ecumenical about
what the theory may be. They mention three possibilities: naturalness
might be primitive; it might be defined in terms of which universals and
tropes exist, if you admit such into your ontology; or it might be
defined in terms of which properties play a special role in our theory.
Call the first the primitivist conception, the second the ontological
conception, and the third the pragmatic conception. (One can generate
different versions of the pragmatic theory by altering what one takes to
be `our theory'. In Taylor (\citeproc{ref-Taylor1993}{1993}), which
Langton and Lewis credit as the canonical statement of the pragmatic
conception, naturalness is relativised to a theory, and the theories he
focuses on are `regimented common sense' and `unified science'.) Langton
and Lewis's intention is to be neutral as to the correct interpretation
of naturalness whenever they appeal to it, and I will follow their
policy.

With these concepts, we can now define intriniscness. A property is
\emph{basic intrinsic} iff it is basic and independent of accompaniment.
Two objects are \emph{duplicates} iff they have the same basic intrinsic
properties. And a property is \emph{intrinsic} iff there are no two
duplicates that differ with respect to it.

Langton and Lewis make one qualification to this definition: it is only
meant to apply to \emph{pure}, or qualitative, properties, as opposed to
\emph{impure}, or haeccceitistic, properties. One reason for this
restriction is that if there are any impure intrinsic properties, such
as \emph{being John Malkovich}, they will not have the combinatorial
features distinctive of pure intrinsic properties. If \emph{F} is a pure
intrinsic property then there can be two wholly distinct things in a
world that are \emph{F}. This fact will be crucial to the revised
definition of intrinsicness offered below. However, it is impossible to
have wholly distinct things in the same world such that each is John
Malkovich. So for now I will follow Langton and Lewis and just say what
it takes for a pure property to be intrinsic. As Langton and Lewis note,
it would be nice to complete the definition by giving conditions under
which impure properties are intrinsic, but the little task of working
out the conditions under which pure properties are intrinsic will be
hard enough for now.

\section{Three Objections}\label{three-objections}

Stephen Yablo (\citeproc{ref-Yablo1993}{Yablo 1999}) criticises the
judgements of naturalness on which this theory rests. Consider again the
property \emph{being the only round thing}, which is extrinsic despite
being independent of accompaniment. If Langton and Lewis are right, this
must not be a basic property. Indeed, Langton and Lewis explicitly say
that it is the negation of a disjunctive property, since its negation
can be expressed as: \emph{being round and accompanied by a round thing
or being not-round}. Yablo's criticism is that it is far from obvious
that the existence of this expansion shows that \emph{being the only
round thing} is disjunctive. For simplicity, let us name all the salient
properties:

\begin{quote}
\emph{R} =\textsubscript{df} being the only round thing\\
\emph{S} =\textsubscript{df} being not the only round thing\\
\emph{T} =\textsubscript{df} being round and accompanied by a round
thing\\
\emph{U} =\textsubscript{df} being not-round
\end{quote}

(Something is accompanied by an \emph{F} iff one of its distinct
worldmates is \emph{F}.) Langton and Lewis claim that since
\emph{S}~=~\emph{T}~\({\vee}\)~\emph{U}, and \emph{S} is much less
natural than \emph{T} and than \emph{U}, \emph{S} is disjunctive, so
\emph{R} is not basic. Yablo notes that we can also express \emph{S} as
\emph{being round if accompanied by a round thing}, so it differs from
\emph{T} only in that it has an \emph{if} where \emph{T} has an
\emph{and}. Given this expansion, we should be dubious of the claim that
\emph{S} is much less natural than \emph{T}. But without that claim,
\emph{R} already provides a counterexample to Langton and Lewis's
theory, unless there is some other expression of \emph{R} or \emph{S}
that shows they are disjunctive.\footnote{It would be no good to say
  that Langton and Lewis should be more liberal with their definition of
  disjunctiveness, and say instead that a property is disjunctive iff it
  can be expressed as a disjunction. Any property \emph{F} can be
  expressed as the disjunction \emph{F} \emph{and G} \emph{or F and not
  G}, or for that matter, \emph{F or F}, so this would make every
  property disjunctive.

  I do not want to dismiss out of hand the possibility that there is
  another expression of \emph{S} that shows it is disjunctive. Josh
  Parsons suggested that if we define \emph{T}′ to be \emph{being
  accompanied by a round thing}, then \emph{S} is
  \emph{T}′~\({\vee}\)~\emph{U}, and there is some chance that \emph{T}′
  is more natural than \emph{S} on \emph{some} conceptions of
  naturalness. So we cannot derive a decisive counterexample from
  Yablo's discussion. Still, Langton and Lewis need it to be the case
  that on any account of naturalness, there is an expression that shows
  \emph{S} or \emph{R} to be disjunctive, and unless \emph{T}′ is much
  more natural than \emph{S} on \emph{all} conceptions of naturalness,
  this task is still far from complete.}

Dan Marshall and Josh Parsons Marshall and Parsons
(\citeproc{ref-Marshall2001}{2001}) argue that the same kind of
difficulties arise when we consider certain kinds of quantificational
properties. For example, let \emph{E} be the property \emph{being such
that a cube} \emph{exists}. This is independent of accompaniment, since
a lonely cube is \emph{E}, a lonely sphere is not \emph{E}, each of us
is accompanied and \emph{E}, and each of Max Black's two spheres is
accompanied and not \emph{E}. So it is a counterexample to Langton and
Lewis if it is basic. Marshall and Parsons note that, like all
properties, it does have disjunctive expressions. For example \emph{x}
is \emph{E} iff \emph{x} is a cube or \emph{x} is accompanied by a cube.
And \emph{E} is a less natural property than \emph{being a cube}. But it
is not at all intuitive that \emph{E} is much less natural than the
property \emph{being accompanied by a cube}. This does not just show
that Langton and Lewis have to cease being ecumenical about naturalness,
because on some conceptions of naturalness it is not clear that \emph{E}
is much less natural than \emph{being accompanied by a cube}. Rather,
this example shows that there is no conception of naturalness that could
play the role that Langton and Lewis want. The properties \emph{E} and
\emph{being accompanied by a cube} seem just as natural as each other on
the ontological conception of naturalness, on the pragmatic conception
of naturalness, and, as far as anyone can tell, on the primitivist
conception. This is not because \emph{E} is particularly natural on any
of these conceptions. It certainly does not, for example, correspond to
a universal, and it does not play a special role in our thinking or in
ideal science. But since there is no universal for \emph{being
accompanied by a cube}, and that property does not play a special role
in our thinking or in ideal science, it seems likely that each property
is as natural as the other.

Theodore Sider (\citeproc{ref-Sider2001}{2001}) notes that similar
problems arise for \emph{maximal} properties, like \emph{being a rock}.
A property \emph{F} is maximal iff large parts of \emph{F}s are
typically not \emph{F}s. For example, \emph{being a house} is maximal; a
very large part of a house, say a house minus one window ledge, is not a
house, it is just a large part of a house. Purported proof: call the
house minus one window ledge house-. If Katie buys the house she
undoubtedly buys house-, but she does not thereby buy two houses, so
house- is not a house. As Sider notes, this is not an entirely
conclusive proof, but it surely has some persuasive force. Maximal
properties could easily raise a problem for Langton and Lewis's
definition. All maximal properties are extrinsic; whether \emph{a} is a
house depends not just on how \emph{a} is, but on what surrounds
\emph{a}. Compare: House- would be a house if the extra window ledge did
not exist; in that case it would be \emph{the} house that Katie buys.
But some maximal properties are independent of accompaniment.
\emph{Being a rock} is presumably maximal: large parts of rocks are not
rocks. If they were then presumably tossing one rock up into the air and
catching it would constitute juggling seventeen rocks, making an
apparently tricky feat somewhat trivial. But there can be lonely rocks.
A rock from our planet would still be a rock if it were lonely. Indeed,
some large rock parts that are not rocks would be rocks if they were
lonely. And it is clear there are be lonely non-rocks (like our
universe), accompanied rocks (like Uluru) and accompanied non-rocks
(like me).

Since \emph{being a rock} is independent of accompaniment and extrinsic,
it is a counterexample if it is basic. Still, one might think it is not
basic. Perhaps \emph{being a rock} is not natural on the primitivist
conception. (Who is to say it is?) And perhaps it does not correspond to
a genuine universal, or to a collection of tropes, so it is a
disjunctive property on the ontological conception of naturalness. Sider
notes, however, that on at least one pragmatic conception, where natural
properties are those that play a special role in regimented common
sense, it does seem particularly natural. Certainly it is hard to find
properties such that \emph{being a rock} can be expressed as a
disjunction of properties that are more central to our thinking than
\emph{being a rock}. So this really does seem to be a counterexample to
Langton and Lewis's theory.

\section{The Set of Intrinsic
Properties}\label{the-set-of-intrinsic-properties}

It is a platitude that a property \emph{F} is intrinsic iff whether an
object is \emph{F} does not depend on the way the rest of the world is.
Ideally this platitude could be morphed into a definition. One obstacle
is that it is hard to define \emph{the way the rest of the world is}
without appeal to intrinsic properties. For example, even if \emph{F} is
intrinsic, whether \emph{a} is \emph{F} is not independent of whether
other objects have the property \emph{not being accompanied by an F},
which I will call \emph{G}. To the extent that having \emph{G} is a
feature of the way the rest of the world is, properties like \emph{G}
constitute counterexamples to the platitude. Since platitudes are meant
to be interpreted to be immune from counterexamples, it is wrong to
interpret the platitude so that \emph{G} is a feature of the way the
rest of the world is. The correct interpretation is that \emph{F} is
intrinsic iff whether an object is \emph{F} does not depend on which
\emph{intrinsic} properties are instantiated elsewhere in the world.

If what I call the independence platitude is to be platitudinous, we
must not treat independence in exactly the same way as Langton and Lewis
do. On one definition, whether \emph{a} is \emph{F} is independent of
whether the rest of the world is \emph{H} iff it is possible that
\emph{a} is \emph{F} and the rest of the world \emph{H}, possible that
\emph{a} is not-\emph{F} and the rest of the world \emph{H}, possible
that \emph{a} is \emph{F} and the rest of the world not-\emph{H}, and
possible that \emph{a} is not-\emph{F} and the rest of the world
not-\emph{H}. On another, whether \emph{a} if \emph{F} is independent of
whether the rest of the world is \emph{H} iff whether \emph{a} is
\emph{F} is entirely determined by the way \emph{a} itself, and nothing
else, is, and whether the rest of the world is \emph{H} is determined by
how it, and not \emph{a}, is. This latter definition is very informal;
hence the need for the formal theory that follows. But it does clearly
differ from the earlier definition in a couple of cases. The two
definitions may come apart if \emph{F} and \emph{H} are excessively
disjunctive. More importantly, for present purposes, they come apart if
\emph{F} is the necessary property (that everything has), or the
impossible property (that nothing has). In these cases, whether \emph{a}
is \emph{F} is entirely settled by the way \emph{a}, and nothing else
is, so in the latter sense it is independent of whether the rest of the
world is \emph{H}. But it is not the case that all four possibilities in
the former definition are possible, so it is not independent of whether
the rest of the world is \emph{H} in that sense. Since there is some
possibility of confusion here, it is worthwhile being clear about
terminology. When I talk about independence here, I will always mean the
latter, informal, definition, and I will refer to principles about which
combinations of intrinsic properties are possible, principles such as
Langton and Lewis's principle that basic intrinsic properties are
independent of accompaniment, as combinatorial principles. So, in the
terminology I am using, the combinatorial principles are attempts to
formally capture the true, but elusive, independence platitude with
which I opened this section.

Since the platitude is a biconditional with \emph{intrinsic} on either
side, it will be a little tricky to morph it into a definition. But we
can make progress by noting that the platitude tells us about relations
that hold between some intrinsic properties, and hence about what the
set of intrinsic properties, which I will call \emph{SI}, must look
like.

For example, from the platitude it follows that \emph{SI} is closed
under Boolean operations. Say that \emph{F} and \emph{G} are intrinsic.
This means that whether some individual \emph{a} is \emph{F} is
independent of how the world outside \emph{a} happens to be. And it
means that whether \emph{a} is \emph{G} is independent of the way the
world outside \emph{a} happens to be. This implies that whether \emph{a}
is \emph{F and G} is independent of the way the world outside \emph{a}
happens to be, because whether \emph{a} is \emph{F and G} is a function
of whether \emph{a} is \emph{F} and whether \emph{a} is \emph{G}. And
that means that \emph{F and G} is intrinsic. Similar reasoning shows
that \emph{F or G}, and \emph{not F} are also intrinsic. Call this
condition \emph{Boolean closure}.

Another implication of the independence platitude is that \emph{SI} must
be closed under various mereological operations. If \emph{F} is
intrinsic then whether \emph{a} is \emph{F} is independent of the
outside world. If some part of \emph{a} is \emph{F}, that means, however
the world outside that part happens to be, that part will be \emph{F}.
So that means that however the world outside \emph{a} is, \emph{a} will
have a part that is \emph{F}. Conversely, if \emph{a} does not have a
part that is \emph{F}, that means all of \emph{a}'s parts are \emph{not
F}. As we saw above, if \emph{F} is intrinsic, so is \emph{not F}. Hence
it is independent of the world outside \emph{a} that all of its parts
are \emph{not F}. That is, it is independent of the world outside
\emph{a} that \emph{a} does not have a part that is \emph{F}. In sum,
whether \emph{a} has a part that is \emph{F} is independent of how the
world outside \emph{a} turns out to be. And that means \emph{having a
part that is F} is intrinsic. By similar reasoning, the property
\emph{Having n parts that are F} will be intrinsic if \emph{F} is for
any value of \emph{n}. Finally, the same reasoning shows that the
property, \emph{being entirely composed of n things that are each F} is
intrinsic if \emph{F} is intrinsic. The only assumption used here is
that it is independent of everything outside \emph{b} that \emph{b} is
entirely composed of the particular things that it is composed of, but
again this seems to be a reasonable assumption. So, formally, if
\emph{F}~\({\in}\)~\emph{SI}, then \emph{Having n parts that are
F}~\({\in}\)~\emph{SI}, and \emph{Being entirely composed of n things
that are F}~\({\in}\)~\emph{SI}. Call this condition \emph{mereological
closure}.

Finally, and most importantly, various combinatorial principles follow
from the independence platitude. One of these, that all intrinsic
properties are independent of accompaniment, forms the centrepiece of
Langton and Lewis's theory. The counterexamples provided by Marshall and
Parsons, and by Sider, suggest that we need to draw two more
combinatorial principles from the platitude. The first is that if
\emph{F} and \emph{G} are intrinsic properties, then whether some
particular object \emph{a} is \emph{F} should be independent of how many
other things in the world are \emph{G}. More carefully, if \emph{F} and
\emph{G} are intrinsic properties that are somewhere instantiated then,
for any \emph{n} such that there is a world with \emph{n}+1 things,
there is a world constituted by exactly \emph{n}+1 pairwise distinct
things, one of which is \emph{F}, and the other \emph{n} of which are
all \emph{G}. When I say the world is constituted by exactly \emph{n}+1
things, I do not mean that there are only \emph{n}+1 things in the
world; some of the \emph{n}+1 things that constitute the world might
have proper parts. What I mean more precisely is that every contingent
thing in the world is a fusion of parts of some of these \emph{n}+1
things. Informally, every intrinsic property is not only independent of
accompaniment, it is independent of accompaniment by every intrinsic
property. As we will see, this combinatorial principle, combined with
the Boolean closure principle, suffices to show that Marshall and
Parsons's example, \emph{being such that a cube exists}, is extrinsic.

Sometimes the fact that a property \emph{F} is extrinsic is revealed by
the fact that nothing that is \emph{F} can be worldmates with things of
a certain type. So the property \emph{being lonely} is extrinsic because
nothing that is lonely can be worldmates with anything at all. But some
extrinsic properties are perfectly liberal about which other properties
can be instantiated in their world; they are extrinsic because their
satisfaction excludes (or entails) the satisfaction of other properties
in their immediate neighbourhood. Sider's maximal properties are like
this. That \emph{a} is a rock tells us nothing at all about what other
properties are instantiated in \emph{a}'s world. However, that \emph{a}
is a rock does tell us something about what happens around \emph{a}. In
particular, it tells us that there is no rock enveloping \emph{a}. If
there were a rock enveloping \emph{a}, then \emph{a} would not be a
rock, but rather a part of a rock. If \emph{being a rock} were
intrinsic, then we would expect there could be two rocks such that the
first envelops the second.\footnote{I assume here that there are rocks
  with rock-shaped holes in their interior. This seems like a reasonable
  assumption, though without much knowledge of geology I do not want to
  be too bold here.} The reason that \emph{being a rock} is extrinsic is
that it violates this combinatorial principle. (As a corollary to this,
a theory which ruled out \emph{being a rock} from the class of the
intrinsic just because it is somehow unnatural would be getting the
right result for the wrong reason. \emph{Being a rock} is not a property
like \emph{being a lonely electron or an accompanied non-electron} that
satisfies the independence platitude in the wrong way; rather, it fails
to satisfy the independence platitude, and our theory should reflect
this.)

So we need a second combinatorial principle that rules out properties
like \emph{being a rock}. The following principle does the job, although
at some cost in complexity. Assume there is some world
\emph{w}\textsubscript{1}, which has some kind of spacetimelike
structure.\footnote{Perhaps all worlds have some kind of spacetimelike
  structure, in which case this qualification is unnecessary, but at
  this stage it is best not to take a stand on such a contentious issue.}
Let \emph{d}\textsubscript{1} and \emph{d}\textsubscript{2} be shapes of
two disjoint spacetimelike regions in \emph{w}\textsubscript{1} that
stand in relation \emph{A}. Further, suppose \emph{F} and \emph{G} are
intrinsic properties such that in some world there is an \emph{F} that
wholly occupies a region with shape \emph{d}\textsubscript{1}, and in
some world, perhaps not the same one, there is a \emph{G} that wholly
occupies a region with shape \emph{d}\textsubscript{2}. By `wholly
occupies' I mean that the \emph{F} takes up all the `space' in
\emph{d}\textsubscript{1}, and does not take up any other `space'.
(There is an assumption here that we can identify shapes of
spacetimelike regions across possible worlds, and while this assumption
seems a little contentious, I hope it is acceptable in this context.) If
\emph{F}, \emph{G}, \emph{d}\textsubscript{1}, \emph{d}\textsubscript{2}
and \emph{A} are set up in this way, then there is a world where
\emph{d}\textsubscript{1} and \emph{d}\textsubscript{2} stand in
\emph{A}, and an \emph{F} wholly occupies a region of shape
\emph{d}\textsubscript{1} in that world, and a \emph{G} wholly occupies
a region of shape \emph{d}\textsubscript{2} in that world. In short, if
you could have an \emph{F} in \emph{d}\textsubscript{1}, and you could
have a \emph{G} in \emph{d}\textsubscript{2}, and
\emph{d}\textsubscript{1} and \emph{d}\textsubscript{2} could stand in
\emph{A}, then all three of those things could happen in one world. This
kind of combinatorial principle has been endorsed by many writers on
modality (for example Lewis 1986 and Armstrong 1989), and it seems
something we should endorse in a theory on intrinsic properties.

In sum, the set of intrinsic properties, \emph{SI}, has the following
four properties:

\begin{description}
\tightlist
\item[(B)]
If \emph{F}~\({\in}\)~\emph{SI}~and \emph{G}~\({\in}\) \emph{SI} then
\emph{F and G}~\({\in}\)~\emph{SI} and \emph{F or G}~\({\in}\)~\emph{SI}
and \emph{not F}~\({\in}\)~\emph{SI}.
\item[(M)]
If \emph{F}~\({\in}\)~\emph{SI} then \emph{Having n parts that are
F}~\({\in}\)~\emph{SI} and \emph{Being entirely composed of exactly n
things that are F}~\({\in}\)~\emph{SI}.
\item[(T)]
If \emph{F}~\({\in}\)~\emph{SI} and \emph{G}~\({\in}\)~\emph{SI} and
there is a possible world with \emph{n}+1 pairwise distinct things, and
something in some world is \emph{F} and something in some world is
\emph{G}, then there is a world with exactly \emph{n}+1 pairwise
distinct things such that one is \emph{F} and the other \emph{n} are
\emph{G}.
\item[(S)]
If \emph{F}~\({\in}\)~\emph{SI} and \emph{G}~\({\in}\)~\emph{SI} and it
is possible that regions with shapes \emph{d}\textsubscript{1} and
\emph{d}\textsubscript{2} stand in relation \emph{A}, and it is possible
that an \emph{F} wholly occupy a region with shape
\emph{d}\textsubscript{1} and a \emph{G} wholly occupy a region with
shape \emph{d}\textsubscript{2}, then there is a world where regions
with shapes \emph{d}\textsubscript{1} and \emph{d}\textsubscript{2}
stand in \emph{A}, and an \emph{F} wholly occupies the region with shape
\emph{d}\textsubscript{1} and a \emph{G} wholly occupies the region with
shape \emph{d}\textsubscript{2}.
\end{description}

Many other sets than \emph{SI} satisfy (B), (M), (T) and (S). That is,
there are many sets \emph{I\textsubscript{k}} such that each condition
would still be true if we were to substitute \emph{I\textsubscript{k}}
for \emph{SI} wherever it appears. Say that any such set is an
\emph{I}-set. Then \emph{F} is intrinsic only if \emph{F} is an element
of some \emph{I}-set. Is every element of every \emph{I}-set intrinsic?
As we will see, sadly the answer is \emph{no}. However, most of the
counterexamples proposed to Langton and Lewis's theory are not elements
of any \emph{I}-set, so we already have the resources to show they are
extrinsic.

\section{Responding to
Counterexamples}\label{responding-to-counterexamples}

Marshall and Parsons noted that \emph{E}, the property \emph{being such
that a cube exists}, is independent of accompaniment. However, it is not
part of any \emph{I}-set. To see this, assume it is in
\emph{I\textsubscript{k}}, which is an \emph{I}-set. By (B), \emph{not
E} is also in \emph{I\textsubscript{k}}. So by (T), there is a world
where something is \emph{E}, and there are two things, one of which is
\emph{E} and the other of which is \emph{not} \emph{E}. But clearly this
cannot be the case: if something in a world is \emph{E}, so is
everything else in the world. Hence \emph{I\textsubscript{k}} cannot be
an \emph{I}-set, contrary to our assumption. Intuitively, \emph{E} is
extrinsic because whether it is satisfied by an individual is not
independent of whether other individuals satisfy it.

Some other quantificational properties, such as \emph{being one of at
most seventeen cubes}, require a different argument to show that they
are not in any \emph{I}-set. Call that property \emph{E17}. (Note, by
the way, that \emph{E17} is independent of accompaniment, and not
obviously disjunctive.) If \emph{E17} is in an \emph{I}-set, then by (T)
there is a world containing exactly 18 things, each of which is
\emph{E17}. But this is clearly impossible, since everything that is
\emph{E17} is a cube, and everything that is \emph{E17} is in a world
containing at most seventeen cubes. So \emph{E17} is not in any
\emph{I}-set, and hence is extrinsic. Similarly, \emph{being the only
round thing} cannot be in an \emph{I}-set, because if it were by (T)
there would be a world in which two things are the only round thing,
which is impossible. So a definition of intrinsicness in terms of
\emph{I}-sets need not make the odd postulations about naturalness that
Yablo found objectionable.

Assume, for \emph{reductio}, that \emph{being a rock} is in an
\emph{I}-set. There is a rock that is roughly spherical, and there is a
rock that has a roughly spherical hollow in its interior. (Actually,
there are many rocks of each type, but we only need one of each.) Let
\emph{d}\textsubscript{1} be the region the first rock takes up, and
assume that the shape of the hollow in the second is also
\emph{d}\textsubscript{1}. If it is not, we could always find another
rock with a hollow this shape, so the assumption is harmless. Let
\emph{d}\textsubscript{2} be the region the second rock, the one with
this nicely shaped hollow, takes up. If \emph{being a rock} is an
\emph{I}-set, then by (S) there is a world where
\emph{d}\textsubscript{2} exactly surrounds \emph{d}\textsubscript{1},
there is a rock wholly occupying \emph{d}\textsubscript{1} and a rock
wholly occupying \emph{d}\textsubscript{2}. But this is impossible; if
there were rock-like things in both \emph{d}\textsubscript{1} and
\emph{d}\textsubscript{2}, they would both be parts of a single large
rock, that extends outside both \emph{d}\textsubscript{1} and
\emph{d}\textsubscript{2} and if there were not a rock-like thing in one
or the other region, then there would not be a rock in that region. So
no set satisfying (S) contains \emph{being a rock}, so that property is
not in any \emph{I}-set, and hence is extrinsic.

The first extrinsic property independent of accompaniment that Langton
and Lewis consider is \emph{CS}: \emph{being spherical and lonely or
cubical and accompanied}. This too is not in any \emph{I}-set. Again,
assume for reductio that it is. In the actual world, there are
(accompanied) cubes that are entirely composed of eight smaller cubes.
Both the large cube and the eight smaller cubes are accompanied, so they
are both \emph{CS}. Hence there is a \emph{CS} that is entirely composed
of eight things that are \emph{CS}. By (M), \emph{being entirely
composed of exactly eight things that are CS} is in the \emph{I}-set. By
(B), \emph{being CS and entirely composed of exactly eight things that
are CS} is in the \emph{I}-set. So by (T), there is a world in which
something has that property, and there is nothing else. (To see that (T)
entails this, let \emph{G} be any element of the \emph{I}-set, and let
\emph{n} be zero.) That is, there is a lonely \emph{CS} that is composed
of eight things that are \emph{CS}. But this is impossible. A lonely
\emph{CS} is a sphere, but its eight parts are not lonely, and are
\emph{CS}, so they must be cubes. And no sphere is entirely composed of
exactly eight cubes. So \emph{CS} cannot be in an \emph{I}-set, and
hence is extrinsic.

\section{Problem Cases and Disjunctive Properties}\label{sec-problem}

Those five successes might make us think that only intrinsic properties
are ever in \emph{I}-sets. However there are still some extrinsic
properties that can slip into \emph{I}-sets. For an example, consider
the property \emph{LCS}, defined as follows:

\begin{quote}
\emph{x} is \emph{LCS} \({\leftrightarrow}\) (\emph{x} is cubical and
not both lonely and simple) or (\emph{x} is lonely, simple and
spherical)
\end{quote}

The smallest set containing \emph{LCS} and satisfying (B) and (M) is an
\emph{I}-set. There is an important reason for this. Define a
\emph{simple world} as a world containing just one mereological simple,
and a compound world as a world that is not a simple world. Whether a
property satisfies (T) and (S) (or, more precisely, whether a set
containing that property can satisfy (T) and (S)) depends on just how
the property interacts with other properties in compound worlds and
whether it is ever instantiated in simple worlds. Since the same things
are \emph{LCS} as are cubical in compound worlds, these two properties,
\emph{LCS} and \emph{being cubical}, interact with other properties in
compound worlds in the same way. And each property is instantiated in
simple worlds, although they are instantiated in different simple
worlds. In sum, the properties are similar enough to be
indistinguishable by (T) and (S), and that means we will not be able to
show that \emph{LCS} is extrinsic using just those considerations.

Any property that agrees with an intrinsic property, like \emph{being
cubical}, in the compound worlds, and is somehow extended so it is
instantiated in simple worlds, will be in an \emph{I}-set. This is not
just because we have not put enough restrictions on what makes an
\emph{I}-set. There are just no combinatorial principles we could deduce
from the independence platitude that \emph{LCS} violates. This is
because any such principle would, like (T) and (S), be satisfied or not
depending just on how the property interacts with other properties in
worlds where there are things to interact with, i.e.~the compound
worlds, and whether it is instantiated in the simple worlds. It is to
the good that our deductions from the independence platitude did not
show that \emph{LCS} is extrinsic, because in an important sense
\emph{LCS}, like all properties that agree with some intrinsic property
in all compound worlds, satisfies the platitude.

So at this point appeal to disjunctive and non-disjunctive properties is
needed. Intuitively, intrinsic properties are not only capable of being
instantiated in all possible combinations with other intrinsic
properties, they are capable of being so instantiated in \emph{the same
way} in all these possible combinations. We need to distinguish between
the disjunctive and the non-disjunctive properties in order to say which
properties are instantiated the same way in all these different
combinations.

It might be thought at this stage that we could just adopt Langton and
Lewis's definition of the disjunctive properties. If that definition
worked, we could say the basic intrinsic properties are the
non-disjunctive properties that are in \emph{I}-sets, then define
duplication and intrinsicness as they do in terms of basic intrinsics.
The definition does not, it seems, work as it stands because it does not
show that \emph{LCS} is disjunctive. This will be easier to follow if we
name all the components of \emph{LCS}, as follows:

\begin{quote}
\emph{C} =\textsubscript{df} being cubical\\
\emph{L} =\textsubscript{df} being lonely\\
\emph{M} =\textsubscript{df} being simple\\
\emph{H} =\textsubscript{df} being spherical\\
\emph{LCS} =\textsubscript{df} (\emph{C}
\(\wedge \neg\)(\emph{L}~∧~\emph{M})) ∨ (\emph{L}~∧~\emph{C}~∧~\emph{H})
\end{quote}

Let us agree that \emph{LCS} is not a natural property, if naturalness
is an on/off state, or is very unnatural, if naturalness comes in
degrees. On Langton and Lewis's first definition, it is disjunctive if
it is a disjunction of conjunctions of natural properties. This seems
unlikely: \({\neg}\)(\emph{L}~∧~\emph{M}) is not a natural property.
This is the property of \emph{being in a compound world}, hardly a
natural property. Similarly,
\emph{C}~\(\wedge \neg\)(\emph{L}~∧~\emph{M}), \emph{being a cube in a
compound world}, is hardly natural either. We could insist that these
properties are natural, but at this point Yablo's complaint, that clear
facts like the extrinsicness of \emph{LCS} are being made to rest on
rather obscure facts, like the putative naturalness of \emph{being in a
compound world}, returns to haunt us. (I assume, for the sake of the
argument, that \emph{L} ∧~\emph{M} ∧~\emph{H} is a natural property,
though this assumption could be easily questioned.) On the second
definition, \emph{LCS} is disjunctive if it is much less natural than
\({\neg}\)(\emph{L} \& \emph{M}), or than
\emph{C}~\(\wedge \neg\)(\emph{L}~∧~\emph{M}). Again, it seems unlikely
that this is the case. These properties seem rather unnatural. I have
defined enough terms that we can state in the lexicon of this paper just
what \({\neg}\)(\emph{L}~∧~\emph{M}) amounts to, i.e.~\emph{being in a
compound world}, but the apparent simplicity of this definition should
not make us think that the properties are natural. It is true in natural
languages that predicates that are easy to express are often natural,
but this fact does not extend across to the technical language that is
employed here.

The way out is to change the definition of disjunctive properties. A
property is disjunctive, intuitively, if it can be instantiated in two
quite different ways. Most properties of the form:
(\emph{N}\textsubscript{1}~∧~\emph{U}\textsubscript{1})~\({\vee}\)~(\emph{N}\textsubscript{2}~∧~\emph{U}\textsubscript{2}),
where \emph{N}\textsubscript{1} and \emph{N}\textsubscript{2} pick out
distinct (relatively) natural properties, and \emph{U}\textsubscript{1}
and \emph{U}\textsubscript{2} pick out distinct (relatively) unnatural
properties that are independent of \emph{N}\textsubscript{1} and
\emph{N}\textsubscript{2}, will be like this. If we name this predicate
\emph{F}, there will be two quite different types of \emph{F}s: those
that are \emph{N}\textsubscript{1} and those that are
\emph{N}\textsubscript{2}. Note that this will be true no matter how
unnatural \emph{U}\textsubscript{1} and \emph{U}\textsubscript{2} are;
provided some \emph{F}s are \emph{N}\textsubscript{1}, and some are
\emph{N}\textsubscript{2}, there will be these two ways to be \emph{F}.
So I suggest we amend Langton and Lewis's definition of disjunctiveness
as follows:

\begin{quote}
A property \emph{F} is disjunctive iff it can be expressed as a
disjunction of conjunctions, i.e.:
(\emph{A}\textsubscript{11}~∧~\ldots~∧~\emph{A}\textsubscript{1\emph{n}})~\({\vee}\)
\ldots{} \({\vee}\)
(\emph{A\textsubscript{k}}\textsubscript{1}~∧~\ldots~∧~\emph{A\textsubscript{km}})
and in each disjunct, at least one of the conjuncts is much more natural
than \emph{F}.
\end{quote}

On this definition it is clear that \emph{LCS} is disjunctive, since it
is much less natural than \emph{being cubical} and than \emph{being
spherical}, and in its expression above, \emph{being cubical} is one of
the conjuncts in the first disjunct, and \emph{being spherical} is one
of the conjuncts in the second disjunct. These kinds of comparisons of
naturalness do not seem contentious, or any less obvious than the
conclusions about extrinsicness we use them to generate. Further, the
new definition of disjunctiveness is not meant to be an \emph{ad hoc}
fix. Rather this requirement that only one conjunct in each disjunct
need be much more natural than \emph{F} seems to follow directly from
the reason we introduced the concept of disjunctiveness to begin with.
For each \emph{F} that satisfies the combinatorial principle (either
independence of accompaniment in Langton and Lewis's theory, or being in
an \emph{I}-set in my theory), we wanted to know whether it only does
this because there are two or more ways to be an \emph{F}. If \emph{F}
satisfies the definition of disjunctiveness I offer here, it seems there
are two or more ways to be an \emph{F}, so the fact that it can be in an
\emph{I}-set should not lead us to believe it is intrinsic.

Using this definition of disjunctiveness, we can say that the basic
intrinsic properties are those that are neither disjunctive nor the
negation of a disjunctive property, and are in at least one
\emph{I}-set, then say duplicates are things that share all basic
intrinsic properties, and finally that intrinsic properties are
properties shared by all duplicates. There are two reasons for thinking
that this definition might well work. First, as we have seen it handles
a wide range of hard cases. More importantly, the way that the hard
cases were falling gave us reason to suspect that the only extrinsic
properties that will be in \emph{I}-sets are properties like \emph{LCS}:
properties that agree with some intrinsic property in all compound
worlds. It is reasonably clear that these properties will be disjunctive
according to the above definition. To see this, let \emph{F} be the
extrinsic property in an \emph{I}-set, and let \emph{G} be the intrinsic
property it agrees with in all compound worlds. Then for some \emph{J},
\emph{F} can be expressed as
(\emph{G}~∧~\({\lnot}\)(\emph{L}~∧~\emph{M}))
\({\vee}\)~(\emph{L}~∧~\emph{M}~∧~\emph{J}), and it will presumably be
much less natural than \emph{G}, probably much less natural than
\emph{J}, and almost certainly much less natural than \emph{being
simple}, our \emph{L}. So if these are the only kind of extrinsic
properties in \emph{I}-sets, our definition is correct.

Indeed, if these are the only kinds of extrinsic properties in
\emph{I}-sets, we may not even need to worry about which properties
should count as disjunctive. Say that a property \emph{F} \emph{blocks}
another property \emph{G} iff both \emph{F} and \emph{G} are in
\emph{I}-sets, but there is no \emph{I}-set containing both \emph{F} and
\emph{G}. If \emph{F} and \emph{G} were both intrinsic, then there would
be an \emph{I}-set they are both in, such as say \emph{SI}, so the fact
that there is no such \emph{I}-set shows that one of them is extrinsic.
Note that \emph{LCS} blocks \emph{being cubical}. To prove this, assume
\emph{LCS} and \emph{being cubical} are in an \emph{I}-set, say
\emph{I\textsubscript{k}}. By two applications of (B), \emph{LCS and not
cubical} is in \emph{I\textsubscript{k}}. This property is instantiated
in some possible worlds: it is instantiated by all lonely spheres. So by
(T) there should be a world containing two things that satisfy \emph{LCS
and not cubical}. But only lonely, simple spheres satisfy this property,
so there is no world where two things satisfy it, contradicting our
assumption that \emph{LCS} and \emph{being cubical} can be in the same
\emph{I}-set. The proof here seems perfectly general: if \emph{G} is
intrinsic and \emph{F} differs from \emph{G} only in which things in
simple worlds satisfy it, and \emph{G} is in an \emph{I}-set, then
\emph{F} will block \emph{G}. Blocking, as defined, is symmetric, so the
fact that \emph{F} blocks \emph{G} is no evidence that \emph{F} is
extrinsic, as opposed to \emph{G}. Still, if \emph{G} is much more
natural than \emph{F}, then in all probability the reason \emph{F}
blocks \emph{G} is that they agree about all cases in compound worlds,
and disagree just about the simple worlds. In that case, it seems that
\emph{F} is extrinsic, and \emph{G} is intrinsic. So I think the
following conjecture has merit: \emph{F} is intrinsic iff it is in an
\emph{I}-set and does not block any property much more natural than
itself. If the conjecture works, the only kind of naturalness
comparisons we need to make will be between properties like \emph{LCS}
and properties like \emph{being cubical}. Again, I think these kinds of
comparisons should be fairly uncontentious.

\section{Back to Basics?}\label{back-to-basics}

Most of the work in my theory is done by the concept of \emph{I}-sets.
It might be wondered whether we can do without them. In particular, it
might be thought that the new definition of disjunctivenes I offer in
Section~\ref{sec-problem} will be enough to rescue Langton and Lewis's
theory from the objections I have been fretting about. Indeed, the new
definition of disjunctiveness \emph{does} suffice for responding to
Yablo's objection. However, it will not do on its own, and I think it
will end up being essential to define intrinsicness in terms of
\emph{I}-sets.

Yablo notes that a property like \emph{being the only red thing} is
independent of accompaniment, and that the way Langton and Lewis suggest
showing it is disjunctive is by expressing its negation as \emph{being
red and accompanied by a red thing, or not being red}. Yablo criticises
the claim that the first of these disjuncts really is a natural
property. Above I agreed that this was a good objection. However, on the
new definition of disjunctiveness, it is beside the point.

To show that \emph{not being the only red thing} is disjunctive, we need
only express it as a disjunction of conjunctions such that at least one
conjunct in each disjunct is much more natural than it is. We have the
disjunctive expansion of \emph{not being the only red thing}, and the
first disjunct is \emph{being red and accompanied by a red thing}. Now
this disjunct as a whole may not be particularly natural, but the first
conjunct, \emph{being red}, is much more natural than \emph{not being
the only red thing}. So all we need to show is that one of the conjuncts
in the second disjunct is much more natural than the whole disjunction.
Since the second disjunct has only one conjunct, this means we have to
show \emph{not being red} is much more natural than \emph{not being the
only red thing}. However, there seems to be no simple way to show this.
It is just entirely unclear how natural properties like \emph{not being
red} should seem to be. My guess (for what it is worth) is that like
most properties that can be expressed by negations of English
predicates, it is very unnatural. Certainly it is very unnatural if we
suppose, as seems fair in \emph{this} context, that \emph{F} is only a
natural property if all the things that are \emph{F} resemble each other
in some important way. The class of things that are not red is as
heterogeneous a class as you can hope to find; blue berries, green
leaves, silver Beetles, colourless gases and immaterial souls all find
their way in. It is true that in \emph{New Work for a Theory of
Universals}, David Lewis provides two importantly distinct criteria for
naturalness. One is the resemblance criterion just mentioned. The other
is that \emph{F} is only perfectly natural if it is fundamental. It
might be thought that when we look at this criterion, it does turn out
that \emph{not being red} is much more fundamental than \emph{being the
only red thing}. Even if this is the case, it is not clear that it does
help, or more importantly, that it \emph{should} help. The problem
Langton and Lewis were trying to handle is that \emph{not being the only
red thing} satisfies a particular combinatorial principle (independence
of accompaniment), but only, they say, because there are two different
ways of instantiating that property: \emph{not being red} and
\emph{being accompanied by a red thing}. The problem is that \emph{not
being red} is not a way to instantiate a property, because it is not a
way that something could be. It seems very intuitive that `ways things
could be', in this sense, are resemblance properties: they are
properties that make for resemblance amongst their instantiators. And
even if we can defend the claim that \emph{not being red} is a
fundamental property, the fact that it is not a resemblance property
seems to undercut Langton and Lewis's case here.

The new definition of disjunctiveness does not provide a defender of
Langton and Lewis's theory with a response to Yablo's criticism. On the
new definition of disjunctiveness, we do not have to show that
\emph{being red and accompanied by a red thing} is more natural than
\emph{not being the only red thing} in order to show that the latter is
disjunctive. However, in order to show that \emph{not being the only red
thing} is disjunctive, we still need to show that \emph{not being red}
is a moderately natural property, and this does not seem to be true.

\section{Conclusion}\label{conclusion}

There are four major differences between the analysis of intrinsic
properties provided here and the one provided by Langton and Lewis.
Three of these are reflected in the difference between the combinatorial
principle they use, independence of accompaniment, and the combinatorial
principle I use, membership in an \emph{I}-set. All properties that are
in \emph{I}-sets are independent of accompaniment, but they also have a
few other nice features. First, membership in an \emph{I}-set guarantees
not just independence of whether there are other things, but
independence of what other types of things there are. This is the
independence principle encoded in condition (T) on \emph{I}-sets.
Secondly, membership in an \emph{I}-set guarantees independence of where
the other things are. This is the principle encoded in condition (S).
Third, the mereological principle (M) has no parallel in Langton and
Lewis's theory.

The effect of these extra three restrictions is that I have to make many
fewer appeals to naturalness than do Langton and Lewis. The fourth
difference between their theory and mine is in the role naturalness
considerations play in determining which properties are intrinsic. In
Section~\ref{sec-problem} I offer two ways of finishing the analysis
using naturalness. The first is in the new definition of
disjunctiveness; with this definition in hand we can finish the story
just as Langton and Lewis suggest. The second is in terms of blocking:
\emph{F} is intrinsic iff it is in an \emph{I}-set and does not block
any property that it is much less natural than. Both ways are designed
to deal with a quite specific problem: properties that differ only in
which things instantiate them in simple worlds have the same
combinatorial features, so a definition of intrinsicness in terms of
combinatorial features (as is Langton and Lewis's, and as is mine) will
not be able to distinguish them. Still, both solutions seem likely to
provide the same answer in all the hard cases: the right answer.

\subsection*{References}\label{references}
\addcontentsline{toc}{subsection}{References}

\phantomsection\label{refs}
\begin{CSLReferences}{1}{0}
\bibitem[\citeproctext]{ref-Lewis2001Langton}
Langton, Rae, and David Lewis. 2001. {``Marshall and {P}arsons on
{`Intrinsic'}.''} \emph{Philosophy and Phenomenological Research} 63
(2): 353--55. doi:
\href{https://doi.org/10.2307/3071068}{10.2307/3071068}.

\bibitem[\citeproctext]{ref-Marshall2001}
Marshall, Dan, and Josh Parsons. 2001. {``Langton and Lewis on
{`Intrinsic'}.''} \emph{Philosophy and Phenomenological Research} 63
(2): 347--51. doi:
\href{https://doi.org/10.2307/3071067}{10.2307/3071067}.

\bibitem[\citeproctext]{ref-Sider2001}
Sider, Theodore. 2001. {``Maximality and Intrinsic Properties.''}
\emph{Philosophy and Phenomenological Research} 63 (2): 357--64. doi:
\href{https://doi.org/10.1111/j.1933-1592.2001.tb00109.x}{10.1111/j.1933-1592.2001.tb00109.x}.

\bibitem[\citeproctext]{ref-Taylor1993}
Taylor, Barry. 1993. {``On Natural Properties in Metaphysics.''}
\emph{Mind} 102 (405): 81--100. doi:
\href{https://doi.org/10.1093/mind/102.405.81}{10.1093/mind/102.405.81}.

\bibitem[\citeproctext]{ref-Yablo1993}
Yablo, Stephen. 1999. {``Intrinsicness.''} \emph{Philosophical Topics}
26 (1): 479--505. doi:
\href{https://doi.org/10.5840/philtopics1999261/234}{10.5840/philtopics1999261/234}.

\end{CSLReferences}



\noindent Published in\emph{
Philosophy and Phenomenological Research}, 2001, pp. 365-380.


\end{document}
