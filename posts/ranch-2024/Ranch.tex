% Options for packages loaded elsewhere
\PassOptionsToPackage{unicode}{hyperref}
\PassOptionsToPackage{hyphens}{url}
\PassOptionsToPackage{dvipsnames,svgnames,x11names}{xcolor}
%
\documentclass[
  11pt,
  letterpaper,
  DIV=11,
  numbers=noendperiod,
  oneside]{scrartcl}

\usepackage{amsmath,amssymb}
\usepackage{iftex}
\ifPDFTeX
  \usepackage[T1]{fontenc}
  \usepackage[utf8]{inputenc}
  \usepackage{textcomp} % provide euro and other symbols
\else % if luatex or xetex
  \ifXeTeX
    \usepackage{mathspec} % this also loads fontspec
  \else
    \usepackage{unicode-math} % this also loads fontspec
  \fi
  \defaultfontfeatures{Scale=MatchLowercase}
  \defaultfontfeatures[\rmfamily]{Ligatures=TeX,Scale=1}
\fi
\usepackage{lmodern}
\ifPDFTeX\else  
    % xetex/luatex font selection
  \setmainfont[Scale = MatchLowercase]{Scala Pro}
  \setsansfont[]{Scala Sans Pro}
  \ifXeTeX
    \setmathfont(Digits,Latin,Greek)[]{Scala Pro}
  \else
    \setmathfont[]{Scala Pro}
  \fi
\fi
% Use upquote if available, for straight quotes in verbatim environments
\IfFileExists{upquote.sty}{\usepackage{upquote}}{}
\IfFileExists{microtype.sty}{% use microtype if available
  \usepackage[]{microtype}
  \UseMicrotypeSet[protrusion]{basicmath} % disable protrusion for tt fonts
}{}
\makeatletter
\@ifundefined{KOMAClassName}{% if non-KOMA class
  \IfFileExists{parskip.sty}{%
    \usepackage{parskip}
  }{% else
    \setlength{\parindent}{0pt}
    \setlength{\parskip}{6pt plus 2pt minus 1pt}}
}{% if KOMA class
  \KOMAoptions{parskip=half}}
\makeatother
\usepackage{xcolor}
\usepackage[left=1in,marginparwidth=2.0666666666667in,textwidth=4.1333333333333in,marginparsep=0.3in]{geometry}
\setlength{\emergencystretch}{3em} % prevent overfull lines
\setcounter{secnumdepth}{3}
% Make \paragraph and \subparagraph free-standing
\ifx\paragraph\undefined\else
  \let\oldparagraph\paragraph
  \renewcommand{\paragraph}[1]{\oldparagraph{#1}\mbox{}}
\fi
\ifx\subparagraph\undefined\else
  \let\oldsubparagraph\subparagraph
  \renewcommand{\subparagraph}[1]{\oldsubparagraph{#1}\mbox{}}
\fi


\providecommand{\tightlist}{%
  \setlength{\itemsep}{0pt}\setlength{\parskip}{0pt}}\usepackage{longtable,booktabs,array}
\usepackage{calc} % for calculating minipage widths
% Correct order of tables after \paragraph or \subparagraph
\usepackage{etoolbox}
\makeatletter
\patchcmd\longtable{\par}{\if@noskipsec\mbox{}\fi\par}{}{}
\makeatother
% Allow footnotes in longtable head/foot
\IfFileExists{footnotehyper.sty}{\usepackage{footnotehyper}}{\usepackage{footnote}}
\makesavenoteenv{longtable}
\usepackage{graphicx}
\makeatletter
\def\maxwidth{\ifdim\Gin@nat@width>\linewidth\linewidth\else\Gin@nat@width\fi}
\def\maxheight{\ifdim\Gin@nat@height>\textheight\textheight\else\Gin@nat@height\fi}
\makeatother
% Scale images if necessary, so that they will not overflow the page
% margins by default, and it is still possible to overwrite the defaults
% using explicit options in \includegraphics[width, height, ...]{}
\setkeys{Gin}{width=\maxwidth,height=\maxheight,keepaspectratio}
% Set default figure placement to htbp
\makeatletter
\def\fps@figure{htbp}
\makeatother

\setlength\heavyrulewidth{0ex}
\setlength\lightrulewidth{0ex}
\makeatletter
\def\@maketitle{%
\newpage
\null
\vskip 2em%
\begin{center}%
\let \footnote \thanks
  {\LARGE \@title \par}%
  \vskip 1.5em%
  {\large
    \lineskip .5em%
    \begin{tabular}[t]{c}%
      \@author
    \end{tabular}\par}%
  %\vskip 1em%
  %{\large \@date}%
\end{center}%
\par
\vskip 1.5em}
\makeatother
\KOMAoption{captions}{tableheading}
\makeatletter
\@ifpackageloaded{caption}{}{\usepackage{caption}}
\AtBeginDocument{%
\ifdefined\contentsname
  \renewcommand*\contentsname{Table of contents}
\else
  \newcommand\contentsname{Table of contents}
\fi
\ifdefined\listfigurename
  \renewcommand*\listfigurename{List of Figures}
\else
  \newcommand\listfigurename{List of Figures}
\fi
\ifdefined\listtablename
  \renewcommand*\listtablename{List of Tables}
\else
  \newcommand\listtablename{List of Tables}
\fi
\ifdefined\figurename
  \renewcommand*\figurename{Figure}
\else
  \newcommand\figurename{Figure}
\fi
\ifdefined\tablename
  \renewcommand*\tablename{Table}
\else
  \newcommand\tablename{Table}
\fi
}
\@ifpackageloaded{float}{}{\usepackage{float}}
\floatstyle{ruled}
\@ifundefined{c@chapter}{\newfloat{codelisting}{h}{lop}}{\newfloat{codelisting}{h}{lop}[chapter]}
\floatname{codelisting}{Listing}
\newcommand*\listoflistings{\listof{codelisting}{List of Listings}}
\makeatother
\makeatletter
\makeatother
\makeatletter
\@ifpackageloaded{caption}{}{\usepackage{caption}}
\@ifpackageloaded{subcaption}{}{\usepackage{subcaption}}
\makeatother
\makeatletter
\@ifpackageloaded{sidenotes}{}{\usepackage{sidenotes}}
\@ifpackageloaded{marginnote}{}{\usepackage{marginnote}}
\makeatother
\ifLuaTeX
  \usepackage{selnolig}  % disable illegal ligatures
\fi
\IfFileExists{bookmark.sty}{\usepackage{bookmark}}{\usepackage{hyperref}}
\IfFileExists{xurl.sty}{\usepackage{xurl}}{} % add URL line breaks if available
\urlstyle{same} % disable monospaced font for URLs
\hypersetup{
  pdftitle={Comments for Ranch Metaphysics 2024},
  pdfauthor={Brian Weatherson},
  colorlinks=true,
  linkcolor={blue},
  filecolor={Maroon},
  citecolor={black},
  urlcolor={black},
  pdfcreator={LaTeX via pandoc}}

\title{Comments for Ranch Metaphysics 2024}
\usepackage{etoolbox}
\makeatletter
\providecommand{\subtitle}[1]{% add subtitle to \maketitle
  \apptocmd{\@title}{\par {\large #1 \par}}{}{}
}
\makeatother
\subtitle{Commentary on Casey O'Callaghan, What's To Fear in Losing a
Sense}
\author{Brian Weatherson}
\date{2024}

\begin{document}
\maketitle

Thanks to Casey for a really thoughtful, and thought-provoking, paper.
I'm going to mostly talk about his first question, then say a little
about the second question at the end.

A very small point first. You need a \emph{strong} version of the
mere-difference view to get the first question puzzle off the ground.
Here's a simple reason to fear becoming disabled: disabled people are
treated awfully in this society. Barnes's mere-difference view says that
being disabled wouldn't be an all-things-considered bad if society
weren't so awful. To really get the puzzle going, you need that it isn't
an all-things-considered bad right now. Maybe some of the testimony from
disabled people can support that claim, but it's a \emph{really} strong
claim. Still, let's see what happens if we take it on board.

As a somewhat old-fashioned philosopher, my first thought when I saw the
title was to ask ``What is fear?''. There are two somewhat different
types of things we call fears. The first are what I'll call (apologies
if this is new terminology for old concepts), acute fear. It's what
happens in animals like us when they are confronted with certain kinds
of danger, and it typically involves bunch of characteristic
physiological reactions.\sidenote{\footnotesize Recent work on this includes Bronwyn
  Finnegan's ``Fear is Anticipatory: A Buddhist Analysis'', Davide
  Bordini \& Giuliano Torrengo's ``Frightening Times'', and Anthony
  Hatzimoysis's ``Passive Fear''.} The other I'll call chronic fear, and
it's how we describe someone who has an elevated disvalue on a kind of
harm. Fear of death is the really paradigmatic case here.\sidenote{\footnotesize Tom
  Cochrane's ``Fear of Death and the Will to Live'' is a useful paper on
  recent work on this.} There are cases that mix the two - fear of
flying and fear of needles for example.\sidenote{\footnotesize Fear that a harm has
  befallen a loved one, as Jesse Prinz discusses in \emph{Gut
  Reactions}, is an interesting in-between case too.}

But it's striking how different the extremes are. Acute fear has
distinctive reactions, and has been widely studies in psychology. Many
of these reactions are not actions, they are involuntary movements. Fear
of death is not like that at all. Most people who fear death don't stop
acting, if anything they over-act, and they don't lose means-end
coherence in their actions. They just have a very large disvalue on a
particular end. The philosophical literature on the fear of death has
almost no overlap with the philosophical, or more extensive
psychological, literature on acute fears. Apart from the word `fear',
they have almost nothing in common.

Fearing a loss of a sense is chronic, not acute. It's a lot like fear of
death, but perhaps with less intensity. One initial challenge here for
the normative version of Casey's first question is that some people seem
to take chronic fears as to involve irrationality. Fear of death is
meant to be more than a desire for longer life. What's the `more'?
Sometimes it seems like it's irrationality. This might be
terminological, and I'll assume that chronic fears can be more-or-less
rational.

Chronic fears are aspectual. To fear something doesn't require regarding
it as all-things-considered bad. It merely requires one of two things.

First, it is sufficient that there's a non-trivial probability of a bad
outcome. That's what happens in fear of flying.

Second, it is sufficient that there's a harm that's part of the outcome,
even if it's outweighed by other goods. That's what happens when a
person with a fear of needles gets an important vaccination.

Either of these points suffices to explain how fear of losing a sense is
compatible with the mere difference view, and even how that fear can be
recalcitrant. Losing a sense is psychologically devastating for some
people. Maybe not that many, but a higher proportion than the proportion
of passengers who die in plane crashes.

And there are definitely bads that go with losing a sense. Maybe for
many people there are offsetting goods, and maybe the bads aren't as bad
as feared, but this isn't like losing \$100 somewhere and gaining \$100
somewhere else. The goods and bads involved don't offset that easily.

And I think Casey's paper does a great job setting out what some of
those bads are. There are things that one simply can't do without a
sense. If I lost my sight, I wouldn't be able to \emph{watch} football.
I would be able to listen to descriptions, and that's fun too, but it's
not the \emph{same}.

Chronic fears frequently involve fearing the unknown. As Casey points
out, that's particularly relevant here. As he says, losing a sense is
\emph{transformative}. And one thing about transformative experiences is
that it makes sense to fear what will happen, even if you expect that
all-things-considered things will not be worse. Indeed, it makes sense
to fear what will happen, even if you expect that all-things-considered
things will be better. Fear with a positive expected gain can be
mundane: think about a scary but fun amusement park ride. But the
transformative cases are more striking.

Consider two paradigm cases of transformative experience: migrating, and
having a child. Both of them are frequently tinged with fear, even for
the willing migrant and excited parent. You don't know what it's like on
the other side, and that's a bit scary.

I think all these reflections help explain not just the fear, but what
Casey rightly points out is the big thing to explain here: the
recalcitrance of the fear. Tell me all the statistics you like about the
distribution of hedonic outcomes; fill me in on the goods that go along
with the bads, or losses, I can more easily comprehend. It won't make
migrating, or having a first child, or losing a sense, not be scary.
There's still a greater than plane crash probability of things going
badly wrong, and there still will be losses - even if they are offset.
And that's enough for fear.

I'll end on an autobiographical note relevant to the second question.
One sense I wouldn't fear losing is smell. There are some smells I'd
miss - kitchens full of curry or coffee. But I'm basically a city person
at heart, and cities \ldots{} don't smell great. The car fumes, the dog
poop, the cigarettes, that distinctive tang of underpasses; I could do
without it all. There is information in there that I'd somewhat miss,
though even there I wouldn't mind not being reminded of some parts. I
suspect there's a generalisation here; if the aesthetic value of a sense
isn't that great, it's easy to imagine doing without it. It's much
harder being able to imagine doing without the things I enjoy
\emph{watching} or \emph{listening to}.

Thanks again to Casey for a great paper, and I'm looking forward to the
discussion!





\end{document}
