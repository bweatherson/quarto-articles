% Options for packages loaded elsewhere
\PassOptionsToPackage{unicode}{hyperref}
\PassOptionsToPackage{hyphens}{url}
%
\documentclass[
  11pt,
  letterpaper,
  DIV=11,
  numbers=noendperiod,
  twoside]{scrartcl}

\usepackage{amsmath,amssymb}
\usepackage{setspace}
\usepackage{iftex}
\ifPDFTeX
  \usepackage[T1]{fontenc}
  \usepackage[utf8]{inputenc}
  \usepackage{textcomp} % provide euro and other symbols
\else % if luatex or xetex
  \usepackage{unicode-math}
  \defaultfontfeatures{Scale=MatchLowercase}
  \defaultfontfeatures[\rmfamily]{Ligatures=TeX,Scale=1}
\fi
\usepackage{lmodern}
\ifPDFTeX\else  
    % xetex/luatex font selection
    \setmainfont[ItalicFont=EB Garamond Italic,BoldFont=EB Garamond
Bold]{EB Garamond Math}
    \setsansfont[]{EB Garamond}
  \setmathfont[]{Garamond-Math}
\fi
% Use upquote if available, for straight quotes in verbatim environments
\IfFileExists{upquote.sty}{\usepackage{upquote}}{}
\IfFileExists{microtype.sty}{% use microtype if available
  \usepackage[]{microtype}
  \UseMicrotypeSet[protrusion]{basicmath} % disable protrusion for tt fonts
}{}
\usepackage{xcolor}
\usepackage[left=1.1in, right=1in, top=0.8in, bottom=0.8in,
paperheight=9.5in, paperwidth=7in, includemp=TRUE, marginparwidth=0in,
marginparsep=0in]{geometry}
\setlength{\emergencystretch}{3em} % prevent overfull lines
\setcounter{secnumdepth}{3}
% Make \paragraph and \subparagraph free-standing
\makeatletter
\ifx\paragraph\undefined\else
  \let\oldparagraph\paragraph
  \renewcommand{\paragraph}{
    \@ifstar
      \xxxParagraphStar
      \xxxParagraphNoStar
  }
  \newcommand{\xxxParagraphStar}[1]{\oldparagraph*{#1}\mbox{}}
  \newcommand{\xxxParagraphNoStar}[1]{\oldparagraph{#1}\mbox{}}
\fi
\ifx\subparagraph\undefined\else
  \let\oldsubparagraph\subparagraph
  \renewcommand{\subparagraph}{
    \@ifstar
      \xxxSubParagraphStar
      \xxxSubParagraphNoStar
  }
  \newcommand{\xxxSubParagraphStar}[1]{\oldsubparagraph*{#1}\mbox{}}
  \newcommand{\xxxSubParagraphNoStar}[1]{\oldsubparagraph{#1}\mbox{}}
\fi
\makeatother


\providecommand{\tightlist}{%
  \setlength{\itemsep}{0pt}\setlength{\parskip}{0pt}}\usepackage{longtable,booktabs,array}
\usepackage{calc} % for calculating minipage widths
% Correct order of tables after \paragraph or \subparagraph
\usepackage{etoolbox}
\makeatletter
\patchcmd\longtable{\par}{\if@noskipsec\mbox{}\fi\par}{}{}
\makeatother
% Allow footnotes in longtable head/foot
\IfFileExists{footnotehyper.sty}{\usepackage{footnotehyper}}{\usepackage{footnote}}
\makesavenoteenv{longtable}
\usepackage{graphicx}
\makeatletter
\newsavebox\pandoc@box
\newcommand*\pandocbounded[1]{% scales image to fit in text height/width
  \sbox\pandoc@box{#1}%
  \Gscale@div\@tempa{\textheight}{\dimexpr\ht\pandoc@box+\dp\pandoc@box\relax}%
  \Gscale@div\@tempb{\linewidth}{\wd\pandoc@box}%
  \ifdim\@tempb\p@<\@tempa\p@\let\@tempa\@tempb\fi% select the smaller of both
  \ifdim\@tempa\p@<\p@\scalebox{\@tempa}{\usebox\pandoc@box}%
  \else\usebox{\pandoc@box}%
  \fi%
}
% Set default figure placement to htbp
\def\fps@figure{htbp}
\makeatother
% definitions for citeproc citations
\NewDocumentCommand\citeproctext{}{}
\NewDocumentCommand\citeproc{mm}{%
  \begingroup\def\citeproctext{#2}\cite{#1}\endgroup}
\makeatletter
 % allow citations to break across lines
 \let\@cite@ofmt\@firstofone
 % avoid brackets around text for \cite:
 \def\@biblabel#1{}
 \def\@cite#1#2{{#1\if@tempswa , #2\fi}}
\makeatother
\newlength{\cslhangindent}
\setlength{\cslhangindent}{1.5em}
\newlength{\csllabelwidth}
\setlength{\csllabelwidth}{3em}
\newenvironment{CSLReferences}[2] % #1 hanging-indent, #2 entry-spacing
 {\begin{list}{}{%
  \setlength{\itemindent}{0pt}
  \setlength{\leftmargin}{0pt}
  \setlength{\parsep}{0pt}
  % turn on hanging indent if param 1 is 1
  \ifodd #1
   \setlength{\leftmargin}{\cslhangindent}
   \setlength{\itemindent}{-1\cslhangindent}
  \fi
  % set entry spacing
  \setlength{\itemsep}{#2\baselineskip}}}
 {\end{list}}
\usepackage{calc}
\newcommand{\CSLBlock}[1]{\hfill\break\parbox[t]{\linewidth}{\strut\ignorespaces#1\strut}}
\newcommand{\CSLLeftMargin}[1]{\parbox[t]{\csllabelwidth}{\strut#1\strut}}
\newcommand{\CSLRightInline}[1]{\parbox[t]{\linewidth - \csllabelwidth}{\strut#1\strut}}
\newcommand{\CSLIndent}[1]{\hspace{\cslhangindent}#1}

\setlength\heavyrulewidth{0ex}
\setlength\lightrulewidth{0ex}
\usepackage[automark]{scrlayer-scrpage}
\clearpairofpagestyles
\cehead{
  Brian Weatherson
  }
\cohead{
  Analytic-Synthetic and A Priori-A Posteriori
  }
\ohead{\bfseries \pagemark}
\cfoot{}
\makeatletter
\newcommand*\NoIndentAfterEnv[1]{%
  \AfterEndEnvironment{#1}{\par\@afterindentfalse\@afterheading}}
\makeatother
\NoIndentAfterEnv{itemize}
\NoIndentAfterEnv{enumerate}
\NoIndentAfterEnv{description}
\NoIndentAfterEnv{quote}
\NoIndentAfterEnv{equation}
\NoIndentAfterEnv{longtable}
\NoIndentAfterEnv{abstract}
\renewenvironment{abstract}
 {\vspace{-1.25cm}
 \quotation\small\noindent\emph{Abstract}:}
 {\endquotation}
\newfontfamily\tfont{EB Garamond}
\addtokomafont{disposition}{\rmfamily}
\addtokomafont{title}{\normalfont\itshape}
\let\footnoterule\relax
\KOMAoption{captions}{tableheading}
\makeatletter
\@ifpackageloaded{caption}{}{\usepackage{caption}}
\AtBeginDocument{%
\ifdefined\contentsname
  \renewcommand*\contentsname{Table of contents}
\else
  \newcommand\contentsname{Table of contents}
\fi
\ifdefined\listfigurename
  \renewcommand*\listfigurename{List of Figures}
\else
  \newcommand\listfigurename{List of Figures}
\fi
\ifdefined\listtablename
  \renewcommand*\listtablename{List of Tables}
\else
  \newcommand\listtablename{List of Tables}
\fi
\ifdefined\figurename
  \renewcommand*\figurename{Figure}
\else
  \newcommand\figurename{Figure}
\fi
\ifdefined\tablename
  \renewcommand*\tablename{Table}
\else
  \newcommand\tablename{Table}
\fi
}
\@ifpackageloaded{float}{}{\usepackage{float}}
\floatstyle{ruled}
\@ifundefined{c@chapter}{\newfloat{codelisting}{h}{lop}}{\newfloat{codelisting}{h}{lop}[chapter]}
\floatname{codelisting}{Listing}
\newcommand*\listoflistings{\listof{codelisting}{List of Listings}}
\makeatother
\makeatletter
\makeatother
\makeatletter
\@ifpackageloaded{caption}{}{\usepackage{caption}}
\@ifpackageloaded{subcaption}{}{\usepackage{subcaption}}
\makeatother

\usepackage{bookmark}

\IfFileExists{xurl.sty}{\usepackage{xurl}}{} % add URL line breaks if available
\urlstyle{same} % disable monospaced font for URLs
\hypersetup{
  pdftitle={Analytic-Synthetic and A Priori-A Posteriori},
  pdfauthor={Brian Weatherson},
  hidelinks,
  pdfcreator={LaTeX via pandoc}}


\title{Analytic-Synthetic and A Priori-A Posteriori}
\author{Brian Weatherson}
\date{2016}

\begin{document}
\maketitle
\begin{abstract}
This article focuses on the distinction between analytic truths and
synthetic truths (i.e.~every truth that isn't analytic), and between a
priori truths and a posteriori truths (i.e.~every truth that isn't a
priori) in philosophy, beginning with a brief historical survey of work
on the two distinctions, their relationship to each other, and to the
necessary/contingent distinction. Four important stops in the history
are considered: two involving Kant and W. V. O. Quine, and two relating
to logical positivism and semantic externalism. The article then
examines questions that have been raised about the analytic--synthetic
and a priori--a posteriori distinctions, such as whether all
distinctively philosophical truths fall on one side of the line and
whether the distinction is relevant to philosophy. It also discusses the
argument that there is a lot more a priori knowledge than we ever
thought, and concludes by describing epistemological accounts of
analyticity.
\end{abstract}


\setstretch{1.1}
\section{History}\label{history}

It's easy to give a rough gloss of the notions of analyticity and a
priority.

\begin{itemize}
\tightlist
\item
  Something is an analytic truth iff it is true in virtue of its
  meaning.
\item
  Something is an a priori truth iff it is knowably true without
  justification by experience.
\end{itemize}

And this yields us two distinctions, between analytic truths and
synthetic truths (i.e., every truth that isn't analytic), and between a
priori truths and a posteriori truths (i.e., every truth that isn't a
priori). But fleshing out these distinctions takes some work, as we'll
see. Let's start by a quick historical survey of work on the two
distinctions, their relationship to each other, and their relationship
to the necessary/contingent distinction. There are four important stops
in the history.

The distinction between ``analytic'' and ``synthetic'' traces back to
Kant. He thought that both distinctions in our title (analytic/synthetic
and a priori/a posteriori) were real, and that they were not the same
distinction. In particular, he held that most interesting philosophical
and mathematical claims were synthetic a priori. This is because he (at
least most of the time) worked with a fairly narrow notion of
analyticity. A subject-predicate sentence \emph{A is B} is analytic if
``the predicate \emph{B} belongs to the subject \emph{A} as something
that is (covertly) contained in this concept \emph{A}''
~(\citeproc{ref-KantFirstCritique}{Kant 1781/1787/1999, 6}). But there
can be plenty of a priori truths that do not fall under this narrow
category.

Let's look at one example of current importance. Consider the claim
\emph{Whatever is known is true}. It is at least plausible that that is
a priori; that we don't need to look into the world to know that
knowledge implies truth. But is it analytic for Kant? It is iff there is
an analysis of knowledge, and the analysis is of the form \emph{S knows
that p iff p is true, and X}, for some value of \emph{X}. Most
epistemologists nowadays would reject the idea that there is an analysis
of knowledge. But even among those who hold out hope for such an
analysis, an argument by Linda Zagzebski
(\citeproc{ref-Zagzebski1994}{1994}) has convinced most people that the
analysis cannot be of this form. In particular, Zagzebski argued that if
there is such an analysis, \emph{X} must entail that \emph{p} is true,
eliminating the need for this clause. So \emph{Whatever is known is
true} will turn out, by Kantian standards, to be synthetic, even if it
is a priori.

Although Kant did not think the two distinctions we are focussing on are
equivalent, most scholars take him to have thought all and only
necessary truths are a priori knowable. (Though see Strang
(\citeproc{ref-Strang2011}{2011}) for a dissent.) This will be a common
theme throughout much of the history.

Our second stop on the history is logical positivism, most clearly
represented in English by Ayer (\citeproc{ref-Ayer1936}{1936}). The
positivists thought that all three distinctions were in a fairly deep
sense equivalent. In particular, they thought all three were very close
to the distinction between theorems and non-theorems of logic. The
positivists were, self-consciously, building on the tradition of British
empiricism. But unlike some empiricists, they didn't want to insist on
an empirical basis for logical and mathematical knowledge.\footnote{Actually,
  the history of what pre-positivist empiricists believed about
  mathematics is a little more complicated than the standard story. See
  Whitmore (\citeproc{ref-Whitmore1945}{1945}) for some details.} The
solution was to build on the logicism about mathematics developed by
Frege and Russell.

Borrowing a term from Boghossian
(\citeproc{ref-Boghossian1996-BOGAR}{1996}), let's say that a sentence
is \emph{Frege-analytic} iff it can be converted to a logical truth by
the substitutions of synonyms. The positivists thought that all a
priori, necessary and analytic truths were the Frege-analytic truths.
Without logicism, this would be wildly implausible, since mathematical
truths would be an exception. But Frege and Russell had done enough to
make that possibility less worrying.

The positivists' view has some epistemological attractiveness. How can
we know things without having empirical input? And how can we know that
some things are true not just in this world, but in all worlds? Well,
say the positivists, by knowing the language (which we learn
empirically) we learn what sentences are related by the substitution of
synonyms. And then the puzzles about knowledge of analytic or necessary
truths just reduce to puzzles about the epistemology of logic.

Our third major stop then is with Quine, who questioned both steps of
this attempted explanation. First, Quine
(\citeproc{ref-QuineTruthConvention}{1936}) noted that the story still
needs an epistemology of logic. The obvious expansion of the story told
so far won't work. It can't just be by learning the meanings of the
logical connectives that we come to learn which the logical truths are.
That's because we need to be able to derive the consequences of those
meanings, and for that we need logic.

But second, Quine (\citeproc{ref-QuineTwoDogmas}{1951}) argued that we
have no independent way to make sense of the notion of synonymy that is
at the heart of Frege-analyticity. This is the most famous part of
Quine's attack on the empiricists epistemology of logic and mathematics,
but it isn't the strongest part of it. Indeed, the argument in ``Two
Dogmas'' is both strange and self-undermining.\footnote{The next few
  sentences follow the arguments of Sober
  (\citeproc{ref-Sober2000}{2000}) fairly closely.} Quine's primary
complaint in that paper about the notions of analyticity, synonymy and
meaning is that the only way we have of understanding these notions is
in terms of the others. But that would only be a problem if we thought
we needed to understand them in terms of something else. Arguably we
need not; the notions could be theoretical primitives. And especially if
one is a confirmation holist (and part of the point of ``Two Dogmas'' is
to defend holism) we shouldn't worry about circularity cropping up near
the core of our epistemology. Relatedly, a naturalist like Quine
shouldn't care about whether we can give a definition of terms like
`meaning', but rather about whether it is a useful concept in a science
like linguistics or cognitive science.

To avoid attributing an incoherent position to Quine, we should
interpret Quine the argument of ``Two Dogmas'' as part of his larger
argument against the appeal to meanings and analyticities.\footnote{This
  paragraph follows closely the discussion of Quine in Russell
  (\citeproc{ref-Russell2008}{2008}).} Quine's larger point, as defended
in ~(\citeproc{ref-Quine1960}{Quine 1960}), was that meanings were
unnecessary scientific postulates. He thinks that we simply don't need
them to explain all the facts about cognition and communication that
need explaining. Now it isn't clear how many people will share Quine's
view that meanings are unnecessary for these sciences, since without his
behaviourism the attempt to do without meanings looks unsuccessful. But
the larger point is that Quine isn't simply relying on an argument from
the irreducibility of analyticity to a dismissal of the
analytic/synthetic distinction.

The last stop on our history tour is semantic externalism. The
externalists complicated the above story in two overlapping ways. First,
they developed convincing arguments that necessity and a priority were
dissociable. The most compelling of these arguments were the examples of
necessary a posteriori truths, such as \emph{Water contains oxygen}. No
matter what surface characteristics or functional roles a substance
might play, if it does not contain oxygen, it could not be water.

Second, they showed that the pre-theoretical notion of meaning, which
had seemed good enough for much prior philosophical theorising,
contained a number of distinct ideas. Here is how Gillian Russell (whose
writings I've leaned heavily on in this introduction) puts it,

\begin{quote}
In three astonishingly influential pieces of philosophical writing,
Hilary Putnam (\citeproc{ref-Putnam1973}{1973}) argued that meaning
couldn't be both what a speaker grasped and what determined extension,
Kaplan (\citeproc{ref-Kaplan1989}{1989}) argued that what determines
extension (character) and what got contributed to what a sentence said
(content) came apart in the cases of indexicals and demonstratives, and
Kripke (\citeproc{ref-Kripke1980}{1980}) argued that what determined the
extension of a name or natural kind term need not be known in order for
a speaker to understand the expression, nor was it what was contributed
to the proposition expressed by a sentence containing one. Each was
suggesting that the roles attributed to a single thing-the expression's
meaning-in the {[}pre-theoretical{]} picture, can be played by distinct
things. (\citeproc{ref-Russell2008}{Russell 2008} x)
\end{quote}

From this point on, when we talk about truth in virtue of meaning, we
have to clarify which aspect of `meaning' we mean. With that in mind,
let's turn to the questions that have been raised about the
distinctions.

\section{Five Questions}\label{fivequestions}

To focus our discussion, let's start with five questions we could ask
about either the purported distinction between analytic and synthetic,
or between a priori and a posteriori.

\begin{enumerate}
\def\labelenumi{\arabic{enumi}.}
\tightlist
\item
  Is there a sensible distinction here?
\item
  Are there truths on either side of the line?
\item
  Does the distinction track something of independent significance?
\item
  Do all distinctively philosophical truths fall on one side of the
  line?
\item
  Is the distinction relevant to philosophy?
\end{enumerate}

The questions are obviously not independent; a negative answer to the
first suggests that we better not offer a positive answer to any of the
rest, for example. But there are more degrees of freedom here than might
immediately be apparent.

A negative answer to the second question, for instance, need not imply a
negative answer to the first. If one held, with Phillip Kitcher
(\citeproc{ref-Kitcher1980}{1980}) that a priori warrant is by its
nature indefeasible, and as a matter of fact no warrants are
indefeasible, then one would think the a priori/a posteriori distinction
is sensible, but in fact everything falls on one side of it.

With respect to the a priori/a posteriori distinction, I'll argue below
that while the answer to question 4 is clearly negative, the answer to
question 5 is positive. The a priori/a posteriori distinction may be
relevant to philosophy even if it isn't relevant to, for example,
demarcating philosophy from non-philosophy. Alternatively, a positive
answer to question 5 may follow from a \emph{negative} answer to one of
the earlier questions. (Williamson (\citeproc{ref-Williamson2013}{2013})
suggests, but ultimately I think does not endorse, the view suggested in
the following sentences.) If we learned that all knowledge was a
posteriori, i.e., that all knowledge depended in an epistemologically
significant way on experience, that would be epistemologically
interesting. So the distinction might have a valuable role in
articulating and perhaps defending a key philosophical insight, even if
all the actual cases fall on one side of the distinction.

The point of raising these questions at the start is to ward off a
possible confusion that can easily arise when discussing distinctions.
It is common to hear about `attacks' on a distinction, or `scepticism'
about a distinction, but a moment's reflection shows that it isn't clear
what this comes to. I think that most of the `attacks' on either of our
two distinctions are arguments for a negative answer to one of these
five questions. (We'll see some instances of this as we go through the
entry.) But different attackers may argue for different negative
answers, and different defenders defend different positive answers. So
it is, I think, helpful to have these distinct questions in mind before
we begin.

\section{The Traditional Notion of the A
Priori}\label{thetraditionalnotionoftheapriori}

The traditional notion of the a priori makes the best sense, I think, if
you start with the following three assumptions.

\begin{enumerate}
\def\labelenumi{\arabic{enumi}.}
\tightlist
\item
  There is a notion of justification that is distinct from, but a
  constituent of, knowledge.
\item
  Whether a belief is justified, in this sense, depends just on the
  evidence the believer has.
\item
  Evidence about the external world consists solely of perceptual
  experiences.
\end{enumerate}

From 2 and 3 we get the idea that there could be some beliefs whose
justification does not depend on any perceptual experience, i.e.,
beliefs that are justified by a null set of perceptual experiences.
These are the beliefs that are justified first, i.e., a priori. Then by
1 we can say that these beliefs satisfy a part, possibly a large part,
of the conditions for being knowledge. And this is the a priori
knowledge.

The problem, as will probably be clear to most readers, is that all
three of the assumptions I started with are contentious. As noted in the
introduction, Linda Zagzebski (\citeproc{ref-Zagzebski1994}{1994}) has
shown that there cannot be any non-factive notion of justification that
is a constituent of knowledge. Timothy Williamson
(\citeproc{ref-Williamson2000-WILKAI}{2000}, Ch. 8) has argued
convincingly against the phenomenal account of evidence; our evidence
consists of facts about the world, not just facts about our experience.
Point 2 is less clearly mistaken, but is still far from obvious. (See
Conee and Feldman (\citeproc{ref-ConeeFeldman2004}{2004}) for a long
defence of point 2, as well as discussion of several problems with it.)

Once we drop the three ideas though, or even just the first and third,
what could be left to say about the a priori? A natural first move is to
think about what \emph{explains} a person's knowledge, rather than what
\emph{constitutes} it. On the classical picture I just sketched, Bob's
knowledge that there are tigers nearby might be constituted by his
experience of hearing tiger-like growls. On that picture, having that
experience is (partially) constitutive of being justified in believing
that there are tigers nearby, and that justification is (partially)
constitutive of his knowing there are tigers nearby, and it is these
constitutive connections that make his knowledge a posteriori. We don't
need to make assumptions that are nearly so strong to conclude that the
experience partially \emph{explains} his knowledge. The experience could
(partially) explain why he is justified, without being any part of the
justification, and the justification could (partially) explain why he
knows, without being any part of the knowledge.

But there's a problem with this move too. I know that all tigers are
tigers. On a standard view about the a priori, this will be a piece of a
priori knowledge. But to explain why I have that knowledge, you have to
appeal to some experiences I have had. After all, with no experiences, I
would not be able to think about tigers. So maybe nothing will end up a
priori.

There's another usual response here. The experiences I have
\emph{enable} me to think about tigers, without doing anything to
\emph{justify} my belief that all tigers are tigers. So maybe a priori
knowledge is that knowledge where experiences do not play any
justificatory role, although they may play an enabling role.

That distinction between justifying and enabling will do a lot of work
in what follows, so it is worth pausing over it. Perhaps we can say a
bit more precisely what it means. An experience is a mere enabler if it
explains why a person knows that \emph{p}, but not in virtue of
explaining how it is they can believe that \emph{p}. I think something
like that is plausibly true, but it still makes a rather large
epistemological assumption, namely that justification is explanatorily
prior to knowledge. That's something that will be rejected by those who
accept the `knowledge first' epistemology of Williamson
(\citeproc{ref-Williamson2000-WILKAI}{2000}).

If you don't accept that justification is explanatorily prior to
knowledge, this route at least to articulating the difference between
experiences that enable, and experiences that justify, is closed off.
And perhaps the enabling/justifying distinction is too obscure to do
much work. That's what Williamson has recently argued, and in the next
section we'll look at his argument.

\section{A Priori Knowledge and Practical
Skills}\label{aprioriknowledgeandpracticalskills}

The a priori/a posteriori distinction, on the best way of freeing it
from outdated epistemological assumptions, relies on the idea we can
make sense of the idea that some experiences are necessary for knowledge
because they \emph{enable} that knowledge, rather than that they
\emph{justify} that knowledge. Timothy Williamson
(\citeproc{ref-Williamson2013}{2013}) has argued that this distinction
is too unclear to be useful, and as a result the a priori/a posteriori
distinction cannot do the work epistemologists need.

I find the example Williamson uses, involving Norman and \emph{Who's
Who} ~(\citeproc{ref-Williamson2013}{Williamson 2013, 295}), rather
unintuitive, so I'll substitute a different example that I think makes
the same point. Diane is a great basketball player. One of her great
skills is being able to anticipate the moves a defence will make, and
responding with a move that will maximise her team's chance of scoring.
This is a skill she's honed through years of practice and competition.
And her most common manifestation of it comes in game situations, when
she sees an opposing defender and realizes what move will maximise her
team's expected points. But she can also manifest this skill `off-line',
when she considers conditional questions of the form \emph{If the
opposing team were to do this, what should I do?}.

Williamson notes that in some such cases, these questions will be solved
through the use of imagination, which is surely right at least for some
sense of `imagination'. And in these cases, there won't be any
particular experience that justifies the answer. Yet Diane can acquire
knowledge by these acts of the imagination. Is the knowledge she gets a
priori or a posteriori? Williamson thinks there is no good answer to
this question, since Diane's experiences play a role in honing her
skills that goes beyond the enabling role, but this role very different
to the role experiences play in classical examples where we can point to
a particular experience that justifies the answer.

Now it might seem that there's a simple move to make here. Diane's
knowledge is obviously a posteriori because it is explained by her years
of experience playing basketball. It is a case of (massive)
overdetermination, but that doesn't mean the experiences collectively
are not an essential part of the explanation. Williamson's response is
that if we go down this route, some paradigmatic instances of a priori
reasoning will turn out to be a posteriori. For instance, our ability to
engage in logical reasoning might turn out to be dependent on our
ability to track identity of objects (or even just terms) across time.
In general, this kind of response threatens to drive the a priori out of
philosophy altogether.\footnote{Note this argument is distinct from the
  argument Williamson (\citeproc{ref-Williamson2007-WILTPO-17}{2007})
  makes about the role of knowledge of counterfactuals in philosophical
  reasoning, and its implications for the a priori status of
  philosophical knowledge. We'll return to that argument, and the
  response by Ichikawa and Jarvis
  (\citeproc{ref-IchikawaJarvis2009}{2009}) below. The key point is that
  this argument only turns on the idea that philosophical reasoning
  might rest on empirically acquired and honed skills.}

This is not, I hasten to note, Williamson's conclusion. Williamson
thinks that some of our logical knowledge is a priori, and Diane's
knowledge is a posteriori, but the salient explanations of how those
pieces of knowledge are obtained and sustained are similar in
epistemologically salient respects. So he concludes the a priori/a
posteriori distinction does not track anything of epistemological
significance.

\section{Innate Knowledge}\label{innateknowledge}

In the previous section we considered an argument that there is much
less a priori knowledge than we usually assume. In this section we'll
look at an argument, tracing back to work by John Hawthorne
(\citeproc{ref-Hawthorne2007}{2007}) that there is a lot more a priori
knowledge than we ever thought, in principle a lot lot more.

Recall that we've argued, on pain of losing all a priori knowledge, that
we must understand a priori knowledge as knowledge that is in some sense
\emph{prior} to experience, not knowledge that is \emph{independent} of
experience. So now consider beliefs that really are prior to experience,
namely innate beliefs. There is a lot of evidence that neonates have
differential reaction to (right way up) human faces than they do to
other objects. (See Chien (\citeproc{ref-Chien2011}{2011}) and
Heron-Delaney, Wirth, and Pascalis
(\citeproc{ref-HeronDelaney2011}{2011}) for some recent studies on this
and citations to many more.) It is natural to explain this by positing
an internal representation in the neonate of the structure of human
faces; i.e., a belief about how human faces are structured. Since these
beliefs are true, and are in a good sense held because they are true,
they seem to amount to knowledge. Yet they are clearly not grounded in,
or explained by, the experiences of the neonate. So they look like a
priori knowledge.

This is obviously very different to the standard conception of what is a
priori knowledge. As we noted at the top, and will expand on in the next
section, there is a lot of interest in the possibility of a priori
knowledge of contingent truths. But even the most enthusiastic
supporters of the a priori don't think we have a priori knowledge of
facial structure of conspecifics.

The problem is actually worse than this. We don't normally focus on what
is actually known a priori, but what is a priori \emph{knowable}. The
reason for this is fairly simple. Most of us cannot know complicated
enough multiplications without the aid of empirical evidence.\footnote{I
  think that when one carries out multiplication by hand, using the
  techniques taught at school, the marks on the paper play a justifying
  and not an enabling role. But arguing for that would be beyond the
  scope of this entry. It should be less controversial that
  multiplications carried out by machine give us a posteriori knowledge
  of the answer.} But this doesn't compromise the idea that mathematical
truths are in a deep sense a priori. That's because one could, in
principle, know them a priori, even if creatures with small brains or
limited skills need assistance from their perceptions.

But if that's right, then we can imagine creatures with all sorts of
different innate beliefs. Indeed, for any law about the world, we can
imagine a creature who innately believes that law to hold, and whose
belief has the right kind of evolutionary explanation for it to count as
knowledge. (Why the restriction to laws? Well, beyond that there might
be issues about whether the innate beliefs are accidentally true. In any
case, I'm not claiming that \emph{only} laws could be known a priori
this way.)

There isn't any obvious way out here. I think the best thing to do is to
say that when we say that something is a priori knowable, we have to
mean that it is a priori knowable for creatures like us. That rules out
the possibility of having all the laws be a priori, but at the cost of
making some arithmetic truths a posteriori.

The arguments by Williamson and Hawthorne I've discussed in the last two
sections challenge the utility of the traditional notion of the a
priori. But for the rest of this entry I'll set them aside, and discuss
what ways we might modify, or use, the traditional notion should we find
responses to these challenges.

\section{Substantive A Priori}\label{substantiveapriori}

As we noted in the introduction, a common thread through much of the
history of this topic was a belief in a close relationship between a
priority and necessity. Most writers take Kant to have treated them as
co-extensive notions, the positivists thought they were identical, and
Quine took them to suffer from similar defects. It is only with the
externalists that we see a gap appearing between the two.

Even once the externalists appear, the gap is not as wide as it may be
for two interlocking reasons. The first is that the argument from
externalism to the existence of the necessary a posteriori is clearer
than the argument from externalism to the existence of the contingent a
priori. The second is that externalism may only give a `shallow'
distinction between necessity and a priority. Let's take these in turn.

Assuming externalism, we can identify examples of the necessary a
posteriori by using familiar natural kind terms. To take a famous
example, it is necessary and a posteriori that water is
H\textsubscript{2}O. It is harder to even identify the contingent a
priori. The rough idea is clear enough. We take the characteristics by
which ordinary language users identify water, and say it is a priori
that water has those characteristics. But what are those
characteristics? Water is the stuff which falls from the sky, fills the
rivers, lakes and oceans, and so on. Is any one of these a priori? Not
really. It could turn out that nothing had all these properties. (Is it
really water in the oceans anyway, or is salt water a different
substance?) So we could introduce a new term, Chalmers
(\citeproc{ref-Chalmers1996}{1996}) suggests `watery', for the long
disjunction of conjunctions of properties that a substance must have if
we are to identify it as water. Perhaps we come up with a list such that
\emph{Water is watery} will be a priori, though since
H\textsubscript{2}O need not have been watery, it will be contingent.
Note we will, at the least, have to introduce new vocabulary to identify
this kind of contingent a priority.

The other worry is that the gap opened up here is `shallow' in the sense
of Gareth Evans (\citeproc{ref-Evans1979}{1979}). Given the way things
turned out, water must be H\textsubscript{2}O. But in some intuitive
sense, things could have turned out differently. (I'm taking the helpful
locution `could have turned out' from Yablo
(\citeproc{ref-Yablo2002}{2002}).) It could have turned out that the
stuff in the rivers, oceans etc was XYZ. So while it is necessary that
water is H\textsubscript{2}O, it could have turned out that this not
only wasn't necessary, it wasn't even true. If that all sounds plausible
to you, you may well think that the a priori truths are all and only
those truths which couldn't have turned out to be false. This way of
thinking is behind the important \emph{two-dimensionalist} approach.
Important works in this tradition, as well as Evans
(\citeproc{ref-Evans1979}{1979}), include Davies and Humberstone
(\citeproc{ref-Davies1980}{1980}), Chalmers
(\citeproc{ref-Chalmers1996}{1996}) and Jackson
(\citeproc{ref-Jackson1998}{1998}).

Now there are significant challenges facing two-dimensionalists, several
of which are set out in Block and Stalnaker
(\citeproc{ref-Block1999-BLOCAD}{1999}). But my sense is that several of
these challenges are very similar to the challenges facing anyone trying
to get an argument from semantic externalism to the contingent a priori.
If we could say more clearly what it is for something to be watery, it
would be easier to say whether a particular world is one where water
turned out to be XYZ. So I suspect if externalism gives us a reason to
believe in the contingent a priori, it will be a fairly shallow
distinction. (This doesn't extend to the argument from externalism to
the necessary a posteriori; we don't need to shore up two-dimensionalism
to say that \emph{Water contains oxygen} is necessary a posteriori.)

But that's not the only way that a priority might outrun necessity. In
recent years there has been a surge of interest in the idea that we can
know a priori various anti-sceptical propositions. This idea was
advanced in detail by John Hawthorne
(\citeproc{ref-Hawthorne2002}{2002}), and then suggested as a way out of
sceptical problems by Roger White (\citeproc{ref-White2006}{2006}) and
Brian Weatherson (\citeproc{ref-WeathersonSRE}{2005}).

Recently, Stewart Cohen (\citeproc{ref-Cohen2010}{2010}) and Sinan
Dogramaci (\citeproc{ref-Dogramaci2010}{2010}) have suggested that
(assuming inductive scepticism is false), we can use ampliative
inferential steps in suppositional reasoning. That is, if it possible to
inductively infer \emph{B} when we know \emph{A}, it is possible to
infer \emph{B} on the supposition that \emph{A}, and go on to infer the
material conditional \emph{A} ⊃ \emph{B}. That conditional will be
contingent if the inference was ampliative, but since we've discharged
the only supposition we used, it could be a priori in a good sense. I
have doubts about this route to the contingent a priori
~(\citeproc{ref-Weatherson2012-WEAIAS}{Weatherson 2012}), but I think
the general idea is plausible.

To make things more concrete, consider `bubble worlds'. A bubble world
consists of a person, and the space immediately around them. If you
think that evidence supervenes on sensory irritation, then you have a
duplicate in a bubble world who has the same evidence as you.\footnote{If
  you prefer a wider conception of evidence, so that for instance two
  people who are looking at distinct duplicates have distinct evidence,
  just make the bubble a little bigger, and this argument will still go
  through.} But you're not in a bubble world, and you know it. There's a
well known probabilistic argument that your evidence can't be grounds
for ruling out possibilities that entail you have just that
evidence.\footnote{See White (\citeproc{ref-White2006}{2006}). David
  Jehle and I have argued that this argument uses distinctively
  classical logical principles in a way that might be problematic
  ~(\citeproc{ref-JehleWeatherson}{Jehle and Weatherson 2012}). And I've
  argued that even slight weakenings of the assumptions about how to
  update credences make the argument fail.
  ~(\citeproc{ref-Weatherson2007}{Weatherson 2007}).} So your evidence
doesn't rule out that you're in a bubble world. But you know you're not.
Hence that knowledge is a priori. So \emph{I'm not in a bubble world}
might be contingent a priori.\footnote{I'm assuming here that the
  semantic response to scepticism, as defended by
  ~(\citeproc{ref-Putnam1981}{Hillary Putnam 1981}), doesn't work for
  ruling out bubble worlds. Defending this assumption would take us too
  far from the current topic.} Moreover, it's not a `shallow'
contingency. It could have turned out that you were in a bubble world.
Indeed, with some more evidence you might even know this.

I'm not going to defend here the claim that it's contingent a priori
that you're not in a bubble world. Indeed, I don't even believe the
probabilistic argument for that conclusion that I just referenced. But
it is worth noting this trend towards taking seriously the possibility
of substantive a priori knowledge.

\section{The A Priori In Philosophy}\label{theaprioriinphilosophy}

I've argued so far that the best sense we can make of the a priori
allows for a lot of a priori knowledge. Once we realise that a priori
knowledge, like any other kind of knowledge, is defeasible, and
fallible, it seems possible that an agent could have a lot of
foundational knowledge of contingent matters. And that foundational
knowledge does seem to be a priori. Of course, such an agent would not
be very much like \emph{us}; so there is still a question of what agents
like us could know a priori. And it might seem that the class of such
pieces of knowledge might be relatively small and interesting.

In particular, one might think that philosophical knowledge, or at least
some interesting part of philosophical knowledge, might be a priori.
Herman Cappelen (\citeproc{ref-Cappelen2012}{2012}) notes that a wide
range of philosophers, with very different commitments, end up with the
view that the a priori has a distinctive role to play in philosophy.
(See, especially, chapters 1 and 6 of that book.) But, as Cappelen also
shows, these philosophers are mistaken; outside perhaps of philosophical
logic, the a priori doesn't have a particularly special role to play in
philosophical inquiry. We can see this, I think, by working through one
recent debate.

Timothy Williamson (\citeproc{ref-Williamson2007-WILTPO-17}{2007}) noted
that philosophical thought experiments are almost always incomplete. The
text of an example doesn't guarantee that conclusions that are usually
drawn from it. To use his example, to guarantee that the subject in one
of Gettier's examples has justified beliefs (that don't amount to
knowledge), we have to suppose that there are no defeaters in the
vicinity, but that isn't stated in the example. Williamson's solution to
this is that we should read the example as a certain kind of
counterfactual. What we know, Williamson argues, is that in the nearest
world where the example was instantiated, the subject would have
justified true belief without knowledge. (I'm ignoring here some
complications involving names, and donkey anaphora, that are not
relevant to this debate.)

Jonathan Ichikawa and Benjamin Jarvis
(\citeproc{ref-IchikawaJarvis2009}{2009}) object that this makes
philosophical knowledge a posteriori. We have to know what the world is
like to know what the nearest world in which the Gettier case is
instantiated is like. Ichikawa and Jarvis reject this because they want
to defend a ``traditional'' conception of thought experiments on which
they provide a priori knowledge. I'm not convinced that this really is
part of philosophical tradition; it seems to me the thought experiments
in Hobbes, Hume, Mill and many others in the canon rely on empirical
knowledge. But I won't press that point here. If one does want to avoid
empirical knowledge coming in via the route Williamson suggests,
Ichikawa and Jarvis develop a nice way of doing so.

They say that thought experiments are little fictions. We need some
empirical knowledge to interpret the fiction. But, they insist, once we
are given the fiction, it is a priori that the fiction is possible, and
that in it the subject has a justified true belief without knowledge.
And, note this point, it is a priori that these two facts entail that it
is not necessary that all justified true beliefs amount to knowledge. On
the last point, they agree with Williamson. It is also a priori that if
the Gettier case could happen, and if it were to happen then there would
be justified true belief without knowledge, then it is not necessary
that all justified true beliefs are knowledge.

But let's try and generalise this to other thought experiments. Start,
for example, with the famous violinist described by Thomson
(\citeproc{ref-Thomson1971}{1971}). That violinist plays a key role in
an argument whose conclusion is that abortion is often morally
permissible. And one premise of the argument is something about an
imagined violinist. Williamson will say that premise is an a posteriori
counterfactual proposition. Ichikawa and Jarvis will say it is an a
priori proposition about what's true in a fiction. Perhaps there's
another premise about the possibility of the example, and maybe that's a
priori too. But those premises don't come close to supporting Thomson's
conclusion. We need another premise about the analogy between the
violinist and a woman contemplating having an abortion to get Thomson's
conclusion. Any such premise will not be a priori, unless some detailed
facts about human biology are a priori. It's a little tricky, but it
isn't obvious the soundness of the abductive inference from those
premises to Thomson's conclusion is a priori either. (See Pargetter and
Bigelow (\citeproc{ref-PargetterBigelow1997}{1997}) for some discussion
of this point.)

I think Thomson's example is more typical of philosophical reasoning
than Gettier's. We don't just use thought experiments to dismiss
theories, like the JTB theory of knowledge. We also use them to defend
philosophical conclusions, such as the permissibility of abortion. And
in general inferences from a thought experiment to the truth of a theory
will involve some a posteriori steps. So even if Ichikawa and Jarvis are
right that we can know a lot about thought experiments a priori, it
won't follow that in general philosophical knowledge derived from
thought experiments is a priori. Get away from special cases where the
facts about the thought experiment entail the philosophically
interesting result, and this should be reasonably clear.

That doesn't mean that there's no use for the a priori in philosophy. It
might be a very helpful concept to use in argument, even if it isn't
true that our conclusions are generally a priori. I'll illustrate with
one example from my own work. One way to support the sceptical intuition
that we don't know we aren't brains in vats is to ask, how could one
possibly know that? Rhetorical questions are not arguments, the received
wisdom of undergraduates notwithstanding, so the sceptic needs to find
some way to extract argumentative force from that question. An
attractive option is argue that \emph{these} are all the ways to know
something, and you can't know you're not a brain it a vat any of
\emph{these} ways. Such an argument typically runs into problems at the
first step; arguing that one has exhausted all possible ways of getting
knowledge is not easy.

Hume (\citeproc{ref-HumeTreatise}{1739/1978}) had the best idea for how
to overcome this step. Don't list the ways someone can know something;
use some property of knowledge gathering methods to partition the
methods. Then argue that in no cell of the partition can one find a
method that allows knowledge of the undesired kind. If the partition
just consists of the presence or absence of some property, you're
guaranteed at least to have covered the field. I've argued Weatherson
(\citeproc{ref-Weatherson2007}{2007}) that you get an interesting
argument by letting the property in question be \emph{is an a priori
method}. By `interesting' I certainly don't mean sound. (And nor do I
insist that Hume equated interesting sceptical arguments with sound
ones.) But I think you get epistemological insight by thinking about
whether knowledge that we're not brains in vats could be a priori, or
could be a posteriori. You don't have to think that philosophical
conclusions themselves are a priori to think this could be a useful
philosophical approach. That last point is probably obvious. We could
have developed the sceptical argument by asking whether knowledge of
nonenvattedness is innate or acquired. But suggesting that's an
interesting argument wouldn't imply the very traditional view that
philosophical knowledge is typically innate.

\section{Metaphysical Accounts of
Analyticity}\label{metaphysicalaccountsofanalyticity}

Let's turn now from the a priori to the analytic. As we noted at the
start, the traditional notion is that a sentence is analytic iff it is
true in virtue of meaning. And, as we saw at the end of the
introduction, this notion is complicated by the fact that traditional
theories of meaning conflated several things that should be kept
separate.

Paul Boghossian (\citeproc{ref-Boghossian1996-BOGAR}{1996}) makes a
distinction that has been highly influential between
\textbf{metaphysical} and \textbf{epistemological} conceptions of
analyticity, and I will follow many contemporary writers in splitting
the topic up in this way. The metaphysical conception is the one most
continuous with the traditional notion of analyticity, and also the one
least popular with contemporary theorists, so we will start with that.
It is the notion that some sentences are true merely in virtue of their
meaning.

Boghossian, following Quine, argues that this notion is either
nonsensical or trivial. Consider a simple example of a putatively
analytic truth, say \emph{Everything is self-identical}. Why is this
true? In part, because it means that everything is self-identical. But
that can't be what we mean to say that it is analytic. \emph{Paris is
beautiful} is true in part because it means that Paris is beautiful, but
that doesn't make the sentence an analytic truth. What we need is that
this is the only thing needed for the sentence to be true. And that
isn't the case for either sentence. \emph{Everything is self-identical}
is true because of what it means and the fact that everything is indeed
self-identical, and \emph{Paris is beautiful} is true because of what it
means and the fact that Paris is indeed beautiful. We haven't yet found
a difference between the two.

It might be easy to see a response here. Start with a less
discriminating treatment of truth makers than I supposed in the previous
paragraph. Say that a sentence is true in virtue of what it means, and
the way the world is. So both of our examples are true in virtue of
their meaning and the way this world is. But for \emph{Everything is
self-identical}, it doesn't matter how the world actually is, any way it
could have been would have made the sentence true. The contribution of
the world is like the contribution of the 5 in \emph{What is 0 times 5?}
You need a second number there, or the question doesn't make sense, but
it doesn't matter which. In some good sense, the 0 does all the work.
(This example, and most of the discussion in the rest of this section,
lean heavily on chapter 2 of ~(\citeproc{ref-Russell2008}{Russell
2008}).)

But this won't do as a conception of analyticity either, because of the
examples of the necessary a posteriori. Consider the example \emph{Gold
has atomic number 79}. It is true in virtue of what it means, thay gold
has atomic number 79, and how the world is. But it doesn't matter which
world we choose; in any world gold has atomic number 79. Yet it is not,
intuitively, analytic.

Russell suggests a solution to this problem that draws on the
developments in externalist theories of meaning that we discussed in the
introduction. Start with the following three way distinction. (These
definitions are a quote from page x of
~(\citeproc{ref-Russell2008}{Russell 2008}).)

\begin{itemize}
\tightlist
\item
  \textbf{Character}: The thing speakers must know (perhaps tacitly) to
  count as understanding an expression.
\item
  \textbf{Content}: What the word contributes to what a sentence
  containing it says (the proposition it expresses).
\item
  \textbf{Reference Determiner}: A condition which an object must meet
  in order to be the reference of, or fall in the extension of, an
  expression.
\end{itemize}

These can all come apart. In the case of pure indexicals like \emph{I},
the content comes apart from the character and reference determiner in
familiar ways. But it is tempting in those cases to equate character and
reference determiner. What makes it the case that a token of \emph{I}
picks out me is that I use it, and that relation between usage and
content is what someone must know to understand the term. But that's an
all too special case. I can be a competent user of the name `Alex' as a
name for my friend Alex without knowing whether she got that name at
birth in the normal way, or knowing whether she acquired it later.
Competence may require that I know the reference of Alex was somehow
determined to be her, but I need not know what that reference determiner
was.

Moreover, the character and reference determiner relate to contexts in
different ways. It is a familiar point that a sentence like \emph{If you
were speaking, I would have been speaking} may be false. That's because
when we evaluate the \emph{I} in the consequent, we don't look to who
the speaker is in the context of evaluation, i.e., the world where you
are speaking, but to the world of utterance, i.e., the context of my
utterance. Just like this familiar distinction between contexts of
utterance and contexts of evaluation, Russell requires us to think about
contexts of introduction. An example helps bring this out.

Say I, on Monday, introduce the name `Inigo' for the shortest sword
fighter. When the name is used on Tuesday, it need not pick out the
shortest sword fighter even in the context of utterance. Inigo might
have grown, or a shorter person may have taken up sword fighting. Of
course, when I use the name in counterfactuals, it might pick out
someone who was never a sword fighter. So we to distinguish the context
the term was introduced in, in this case Monday, from the context it is
uttered in, in this case Tuesday.

With these distinctions in mind, we can give Russell's first pass at a
definition of analyticity.

\begin{quote}
A sentence \emph{S} is true in virtue of meaning just in case for all
pairs of context of introduction and context of utterance, the
proposition expressed by \emph{S} with respect to those contexts is true
in the context of evaluation. (\citeproc{ref-Russell2008}{Russell 2008,
56})
\end{quote}

This will, says Russell, solve the problem about gold. It is true that
when someone now utters \emph{Gold has atomic number 79}, they express a
necessary truth. But we could have introduced the terms in the very same
way, and had the world not cooperated, this sentence would have been
false. Indeed, Russell splits analyticity from necessity twice over. She
thinks that \emph{I am here now} will be analytic in this sense though
it expresses a contingent proposition.

This does rely on understanding what it is for terms in different worlds
to have the same reference determiner. Perhaps one could object that had
we been pointing at something else when we introduced the term
\emph{gold}, we would have been using a crucially different reference
determiner. But the issues here about the metaphysics of words and
demonstrations, are subtle, and Russell's view that the same reference
determiner could determine different contents in different worlds seems
plausible.

Russell says that this is a perfectly good notion of truth in virtue of
meaning. Of course, as she says, it is really a kind of truth in virtue
of what determines meaning, not meaning itself. Reference determiners
are part of meta-semantics, not semantics. But that seems continuous
enough with the tradition. And there is no reason to think that analytic
sentences, so understood, will be epistemologically distinctive. In this
respect we may end up agreeing with the primary conclusion of the
discussion of metaphysical analyticity in chapter 3 of Williamson
(\citeproc{ref-Williamson2007-WILTPO-17}{2007}), namely that it isn't
directly relevant to philosophical methodology. But it could be an
interesting notion in its own right, and as discussed in the previous
section, it could be philosophically useful without playing its
traditional role.

I have considerably simplified the presentation of Russell's view,
however. The definition so far implies that some theorems of geometry,
and perhaps fundamental laws of ethics, will be analytic. Like Kant,
Russell wants these to be synthetic. Her solution is to say that
analytic truths must not just satisfy the constraint given above, but
that they must do so because the reference determiners of their parts
stand in the right kind of containment relations. But spelling this part
of her view out will take too much space, so instead I'll close with a
discussion of epistemological analyticity.

\section{Epistemological Accounts of
Analyticity}\label{epistemologicalaccountsofanalyticity}

In a series of influential articles, Paul
(\citeproc{ref-Boghossian1996-BOGAR}{Boghossian 1996},
\citeproc{ref-Boghossian1997}{1997},
\citeproc{ref-Boghossian2003-BOGEAA}{2003}) argued that we should accept
that Quine's argument against metaphysical versions of the
analytic/synthetic distinction, but that Quine's arguments left
untouched an \textbf{epistemic} understanding of the distinction. On
this way of understanding the distinction, a sentence is analytic iff it
is knowably true merely in virtue of understanding it. Consider, for
instance, this sentence.

\begin{description}
\tightlist
\item[(E)]
If frogs bark and ducks howl, then frogs bark.
\end{description}

Now make the following four assumptions.

\begin{enumerate}
\def\labelenumi{\arabic{enumi}.}
\tightlist
\item
  Understanding the non-logical terms in (E) suffices to see it is of
  the form (\emph{A}~∧~\emph{B})~→~\emph{A}.
\item
  For logical terms like ∧ and →, understanding involves accepting,
  perhaps implicitly in one's inferential practices, the basic
  introduction and elimination rules they license.
\item
  For ∧, the basic rules are the familiar introduction and elimination
  rules.
\item
  For →, the basic rules are modus ponens and conditional proof.
\end{enumerate}

Then anyone who understands (E) is in a position to prove it to be true
by a trivial three line proof. Generalizing this example, we can get an
argument that all logically true sentences are analytic. Generalizing a
bit further, we may be able to argue that the propositions they express
are a priori knowable, but this requires resolving many of the issues we
have already discussed in the discussion of the a priori, and I will set
it aside for the remainder of this entry.\footnote{There is an
  interesting worry around here that the three-line argument for (E) is
  circular, and so cannot justify (E), and this fact undermines
  Boghossian's argument that it is a priori. See Ebert
  (\citeproc{ref-Ebert2005}{2005}) and Jenkins
  (\citeproc{ref-Jenkins2008-JENBAE}{2008}) for two ways of developing
  this worry.}

The problem we will focus on is that assumptions 2 and 4, and hence
presumably 3, are not clearly true. Actually 4 as stated is almost
surely false. If the basic rules for → are modus ponens and conditional
proof, then → is material implication. But → was meant to be our symbol
for natural language `if', which is not material implication. So the
rule must be something else. It isn't clear what this rule could be. It
is plausible that we can use a restricted version of conditional proof
when reasoning about `if', such as a version which requires that there
be no undischarged assumptions when we apply conditional proof. That
will make the proof of (E) go through, but it is unlikely to be a basic
rule in the relevant sense, since it does not combine with an
elimination rule (i.e., modus ponens) to pick out a unique meaning for
`if'.

Disagreement about the introduction rule for `if' is endemic to the
literature on conditionals. But there is almost a consensus that the
elimination rule is modus ponens. Almost, but not quite - Vann McGee
(\citeproc{ref-McGee1985}{1985}) is a notable dissenter. Timothy
Williamson (\citeproc{ref-Williamson2007-WILTPO-17}{2007}) uses the
existence of notable dissenters like McGee to mount a sustained assault
on Boghossian's position. It is a consequence of the assumptions we have
made, and which Boghossian needs, that anyone who doesn't accept modus
ponens does not understand `if'. But that seems implausible. By any
familiar standard, McGee understands conditionals quite well. Indeed, he
is an expert on them.

This point generalizes, as Williamson stresses. On the inferentialist
view about the meaning of logical terms, in any debate about the
correctness of fundamental logical principles, either one party doesn't
understand the key terms, or the parties are speaking at cross purposes.
The intuitionist mathematician endorses the sentence ``All functions are
continuous'', and the classical mathematician rejects it. But it isn't
plausible that one party fails to understand `all', `functions' or
`continuous', or that they are speaking at cross purposes in that they
are assigning different meanings to one of these terms. (I'm assuming
the context makes it clear that both parties are speaking of functions
whose domain is the reals, and whose range is subset of the reals.) I've
used an example from real analysis, but we could make the same point
less pithily using Pierce's Law if we wanted to stick to propositional
logic.

I'll close with two replies on behalf of the defender of epistemic
analyticity, and some reasons for being dissatisfied with each. The
discussion will follow somewhat the recent exchange between Boghossian
(\citeproc{ref-Boghossian2011-BOGWOT}{2011}) and Williamson
(\citeproc{ref-Williamson2011-WILRTB-3}{2011}).

The first response says that we shouldn't have said if a sentence is
epistemically analytic, then \emph{understanding} it is sufficient for
knowing that a sentence is true. Rather, we should have said that
\emph{knowing the meaning} is sufficient for knowing the sentence is
true. Notably, Boghossian does not make this defence in his exchange
with Williamson, so it seems he accepts that Williamson was right to
take epistemic analyticity to involve a connection between understanding
and knowability. And this seems to be right. Consider again \emph{Water
contains oxygen}. In one sense of meaning, the meaning of `water' is
H\textsubscript{2}O. So anyone who knows what `water' means in that
sense knows that \emph{Water contains oxygen} is true. But it doesn't
feel like this claim is analytic, especially not in the epistemic sense
that interests Boghossian. It is possible that there is some other sense
of meaning that will be more useful for Boghossian's project, but it
isn't clear that knowing the meaning in this other sense will differ
particularly from understanding.

The second response, and one that Boghossian has used on several
occasions, is that the only plausible theory of meaning for the logical
connectives is inferentialist, and on an inferentialist theory of
meaning it will be true that anyone who understands a connective is
disposed to reason correctly with it. That last sentence is deliberately
sloppy, much more so than any statement of the response in Boghossian's
own work. But the sloppiness is there because it makes a potential
equivocation more easily visible.

Consider a theory that says the meaning of a logical connective is
either constituted by, or at least constitutively connected to, its
appropriate inferential rules. But to understand the term is not to
grasp the meaning in this sense, any more than to understand the term
`water' one has to know it is H\textsubscript{2}O. Rather, understanding
involves participating in the right kind of way in a social practice,
and it is that social practice (plus perhaps some facts about the nature
of logic, if such facts there be) that determines the appropriate
inferential rules for the connective.

Is the theory in the previous paragraph inferentialist? If not, then it
is false that no theory other than inferentialism is plausible as an
account of the meaning of the logical connectives. For this kind of
socialised theory of meaning is, it seems to me, highly plausible.
(Williamson (\citeproc{ref-Williamson2011-WILRTB-3}{2011}) notes that a
socialised theory of meaning for the connectives is plausible, though I
don't think he would sign up for the view that the result of such
socialization is a theory in terms of inferential rules.) If, on the
other hand, the theory is inferentialist, then it doesn't follow that
understanding requires a disposition to use the rules. Perhaps
understanding requires being part of a community many members of which
have the appropriate dispositions, but it does not require that any one
member have these dispositions. So it won't be true that mere
understanding puts one in a position to know. At best, understanding a
logical truth means one is in a community in which some people are in a
position to the sentence is true. But that doesn't do much to rescue the
notion of epistemic analyticity.

\subsection*{References}\label{references}
\addcontentsline{toc}{subsection}{References}

\phantomsection\label{refs}
\begin{CSLReferences}{1}{0}
\bibitem[\citeproctext]{ref-Ayer1936}
Ayer, Alfred. 1936. \emph{Language, Truth and Logic.} London: Gollantz.

\bibitem[\citeproctext]{ref-Block1999-BLOCAD}
Block, Ned, and Robert Stalnaker. 1999. {``{Conceptual Analysis,
Dualism, and the Explanatory Gap}.''} \emph{Philosophical Review} 108
(1): 1--46. doi:
\href{https://doi.org/10.2307/2998259}{10.2307/2998259}.

\bibitem[\citeproctext]{ref-Boghossian1996-BOGAR}
Boghossian, Paul. 1996. {``Analyticity Reconsidered.''} \emph{No{û}s} 30
(3): 360--91. doi:
\href{https://doi.org/10.2307/2216275}{10.2307/2216275}.

\bibitem[\citeproctext]{ref-Boghossian1997}
---------. 1997. {``Analyticity.''} In \emph{A Companion to the
Philosophy of Language}, edited by Bob Hale and Crispin Wright, 331--68.
Oxford: Blackwell.

\bibitem[\citeproctext]{ref-Boghossian2003-BOGEAA}
---------. 2003. {``Epistemic Analyticity: A Defense.''} \emph{Grazer
Philosophische Studien} 66 (1): 15--35. doi:
\href{https://doi.org/10.1163/18756735-90000810}{10.1163/18756735-90000810}.

\bibitem[\citeproctext]{ref-Boghossian2011-BOGWOT}
---------. 2011. {``Williamson on the a Priori and the Analytic.''}
\emph{Philosophy and Phenomenological Research} 82 (2): 488--97. doi:
\href{https://doi.org/10.1111/j.1933-1592.2010.00395.x}{10.1111/j.1933-1592.2010.00395.x}.

\bibitem[\citeproctext]{ref-Cappelen2012}
Cappelen, Herman. 2012. \emph{Philosophy Without Intuitions}. Oxford:
Oxford University Press.

\bibitem[\citeproctext]{ref-Chalmers1996}
Chalmers, David J. 1996. \emph{The Conscious Mind}. Oxford: Oxford
University Press.

\bibitem[\citeproctext]{ref-Chien2011}
Chien, Sarina Hui-Lin. 2011. {``No More Top-Heavy Bias: Infants and
Adults Prefer Upright Faces but Not Top-Heavy Geometric or Face-Like
Patterns.''} \emph{Journal of Vision} 11 (13): 1--14. doi:
\href{https://doi.org/10.1167/11.6.13}{10.1167/11.6.13}.

\bibitem[\citeproctext]{ref-Cohen2010}
Cohen, Stewart. 2010. {``Bootstrapping, Defeasible Reasoning and \emph{a
Priori} Justification.''} \emph{Philosophical Perspectives} 24 (1):
141--59. doi:
\href{https://doi.org/10.1111/j.1520-8583.2010.00188.x}{10.1111/j.1520-8583.2010.00188.x}.

\bibitem[\citeproctext]{ref-ConeeFeldman2004}
Conee, Earl, and Richard Feldman. 2004. \emph{Evidentialism: Essays in
Epistemology}. Oxford: Oxford University Press.

\bibitem[\citeproctext]{ref-Davies1980}
Davies, Martin, and I. L. Humberstone. 1980. {``Two Notions of
Necessity.''} \emph{Philosophical Studies} 38 (1): 1--30. doi:
\href{https://doi.org/10.1007/bf00354523}{10.1007/bf00354523}.

\bibitem[\citeproctext]{ref-Dogramaci2010}
Dogramaci, Sinan. 2010. {``Knowledge of Validity.''} \emph{No{û}s} 44
(3): 403--32. doi:
\href{https://doi.org/0.1111/j.1468-0068.2010.00746.x}{0.1111/j.1468-0068.2010.00746.x}.

\bibitem[\citeproctext]{ref-Ebert2005}
Ebert, Philip. 2005. {``Transmission of Warrant Failure and the Notion
of Epistemic Analyticity.''} \emph{Australasian Journal of Philosophy}
83 (4): 505--22. doi:
\href{https://doi.org/10.1080/00048400500338724}{10.1080/00048400500338724}.

\bibitem[\citeproctext]{ref-Evans1979}
Evans, Gareth. 1979. {``Reference and Contingency.''} \emph{Monist} 62:
161--89.

\bibitem[\citeproctext]{ref-Hawthorne2002}
Hawthorne, John. 2002. {``Deeply Contingent a Priori Knowledge.''}
\emph{Philosophy and Phenomenological Research} 65 (2): 247--69. doi:
\href{https://doi.org/10.1111/j.1933-1592.2002.tb00201.x}{10.1111/j.1933-1592.2002.tb00201.x}.

\bibitem[\citeproctext]{ref-Hawthorne2007}
---------. 2007. {``Craziness and Metasemantics.''} \emph{Philosophical
Review} 116 (3): 427--40. doi:
\href{https://doi.org/10.1215/00318108-2007-004}{10.1215/00318108-2007-004}.

\bibitem[\citeproctext]{ref-HeronDelaney2011}
Heron-Delaney, Michelle, Sylvia Wirth, and Olivier Pascalis. 2011.
{``Infants' Knowledge of Their Own Species.''} \emph{Philosophical
Transactions of the Royal Society B} 366 (1571): 1753--63. doi:
\href{https://doi.org/10.1098/rstb.2010.0371}{10.1098/rstb.2010.0371}.

\bibitem[\citeproctext]{ref-HumeTreatise}
Hume, David. 1739/1978. \emph{A Treatise on Human Nature}. Edited by L.
A. Selby-Bigge and P. H. Nidditch. Second. Oxford: Clarendon Press.

\bibitem[\citeproctext]{ref-IchikawaJarvis2009}
Ichikawa, Jonathan, and Benjamin Jarvis. 2009. {``Thought-Experiment
Intuitions and Truth in Fiction.''} \emph{Philosophical Studies} 142
(2): 221--46. doi:
\href{https://doi.org/10.1007/s11098-007-9184-y}{10.1007/s11098-007-9184-y}.

\bibitem[\citeproctext]{ref-Jackson1998}
Jackson, Frank. 1998. \emph{From Metaphysics to Ethics: A Defence of
Conceptual Analysis}. Clarendon Press: Oxford.

\bibitem[\citeproctext]{ref-JehleWeatherson}
Jehle, David, and Brian Weatherson. 2012. {``Dogmatism, Probability and
Logical Uncertainty.''} In \emph{New Waves in Philosophical Logic},
edited by Greg Restall and Gillian Russell, 95--111. Bassingstoke:
Palgrave Macmillan.

\bibitem[\citeproctext]{ref-Jenkins2008-JENBAE}
Jenkins, C. S. 2008. {``Boghossian and Epistemic Analyticity.''}
\emph{Croatian Journal of Philosophy} 8 (22): 113--27.

\bibitem[\citeproctext]{ref-KantFirstCritique}
Kant, Immanuel. 1781/1787/1999. \emph{Critique of Pure Reason}. Edited
by Paul Guyer and Allen Wood. Cambridge: Cambridge University Press.

\bibitem[\citeproctext]{ref-Kaplan1989}
Kaplan, David. 1989. {``Demonstratives.''} In \emph{Themes from Kaplan},
edited by Joseph Almog, John Perry, and Howard Wettstein, 481--563.
Oxford: Oxford University Press.

\bibitem[\citeproctext]{ref-Kitcher1980}
Kitcher, Philip. 1980. {``A Priori Knowledge.''} \emph{Philosophical
Review} 89 (1): 3--23. doi:
\href{https://doi.org/10.2307/2184861}{10.2307/2184861}.

\bibitem[\citeproctext]{ref-Kripke1980}
Kripke, Saul. 1980. \emph{Naming and Necessity}. Cambridge: Harvard
University Press.

\bibitem[\citeproctext]{ref-McGee1985}
McGee, Vann. 1985. {``A Counterexample to Modus Ponens.''} \emph{Journal
of Philosophy} 82 (9): 462--71. doi:
\href{https://doi.org/10.2307/2026276}{10.2307/2026276}.

\bibitem[\citeproctext]{ref-PargetterBigelow1997}
Pargetter, Robert, and John Bigelow. 1997. {``The Validation of
Induction.''} \emph{Australasian Journal of Philosophy} 75 (1): 62--76.
doi:
\href{https://doi.org/10.1080/00048409712347671}{10.1080/00048409712347671}.

\bibitem[\citeproctext]{ref-Putnam1973}
Putnam, Hilary. 1973. {``Meaning and Reference.''} \emph{Journal of
Philosophy} 70 (19): 699--711. doi:
\href{https://doi.org/10.2307/2025079}{10.2307/2025079}.

\bibitem[\citeproctext]{ref-Putnam1981}
Putnam, Hillary. 1981. \emph{Reason, Truth and History}. Cambridge:
Cambridge University Press.

\bibitem[\citeproctext]{ref-QuineTruthConvention}
Quine, W. V. O. 1936. {``Truth by Convention.''} In \emph{Philosophical
Essays for a. N. Whitehead}, edited by O. H. Lee, 90--124. New York:
Longmans.

\bibitem[\citeproctext]{ref-QuineTwoDogmas}
---------. 1951. {``Two Dogmas of Empiricism.''} \emph{Philosophical
Review} 60 (1): 20--43. doi:
\href{https://doi.org/10.2307/2181906}{10.2307/2181906}.

\bibitem[\citeproctext]{ref-Quine1960}
---------. 1960. \emph{Word and Object}. Cambridge, MA.: MIT Press.

\bibitem[\citeproctext]{ref-Russell2008}
Russell, Gillian. 2008. \emph{Truth in Virtue of Meaning: A Defence of
the Analytic/Synthetic Distinction}. Oxford: Oxford University Press.

\bibitem[\citeproctext]{ref-Sober2000}
Sober, Elliot. 2000. {``Quine's Two Dogmas.''} \emph{Aristotelian
Society Supplementary Volume} 74 (1): 237--80. doi:
\href{https://doi.org/10.1111/1467-8349.00071}{10.1111/1467-8349.00071}.

\bibitem[\citeproctext]{ref-Strang2011}
Strang, Nicholas F. 2011. {``Did Kant Conflate the Necessary and the
\emph{a Priori}.''} \emph{No{û}s} 45 (3): 443--71. doi:
\href{https://doi.org/10.1111/j.1468-0068.2010.00809.x}{10.1111/j.1468-0068.2010.00809.x}.

\bibitem[\citeproctext]{ref-Thomson1971}
Thomson, Judith Jarvis. 1971. {``A Defense of Abortion.''}
\emph{Philosophy and Public Affairs} 1 (1): 47--66.

\bibitem[\citeproctext]{ref-WeathersonSRE}
Weatherson, Brian. 2005. {``Scepticism, Rationalism and Externalism.''}
\emph{Oxford Studies in Epistemology} 1: 311--31.

\bibitem[\citeproctext]{ref-Weatherson2007}
---------. 2007. {``The Bayesian and the Dogmatist.''} \emph{Proceedings
of the Aristotelian Society} 107: 169--85. doi:
\href{https://doi.org/10.1111/j.1467-9264.2007.00217.x}{10.1111/j.1467-9264.2007.00217.x}.

\bibitem[\citeproctext]{ref-Weatherson2012-WEAIAS}
---------. 2012. {``Induction and Supposition.''} \emph{The Reasoner} 6
(6): 78--80.

\bibitem[\citeproctext]{ref-White2006}
White, Roger. 2006. {``Problems for Dogmatism.''} \emph{Philosophical
Studies} 131 (3): 525--57. doi:
\href{https://doi.org/10.1007/s11098-004-7487-9}{10.1007/s11098-004-7487-9}.

\bibitem[\citeproctext]{ref-Whitmore1945}
Whitmore, Charles E. 1945. {``Mill and Mathematics: An Historical
Note.''} \emph{Journal of the History of Ideas} 6 (1): 109--12. doi:
\href{https://doi.org/10.2307/2707061}{10.2307/2707061}.

\bibitem[\citeproctext]{ref-Williamson2000-WILKAI}
Williamson, Timothy. 2000. \emph{{Knowledge and its Limits}}. Oxford
University Press.

\bibitem[\citeproctext]{ref-Williamson2007-WILTPO-17}
---------. 2007. \emph{{The Philosophy of Philosophy}}. Blackwell.

\bibitem[\citeproctext]{ref-Williamson2011-WILRTB-3}
---------. 2011. {``Reply to Boghossian.''} \emph{Philosophy and
Phenomenological Research} 82 (2): 498--506. doi:
\href{https://doi.org/10.1111/j.1933-1592.2010.00400.x}{10.1111/j.1933-1592.2010.00400.x}.

\bibitem[\citeproctext]{ref-Williamson2013}
---------. 2013. {``How Deep Is the Distinction Between a Priori and a
Posteriori Knowledge.''} In \emph{The a Priori in Philosophy}, edited by
Albert Casullo and Joshua C. Thurow, 291--312. Oxford: Oxford University
Press.

\bibitem[\citeproctext]{ref-Yablo2002}
Yablo, Stephen. 2002. {``Coulda, Woulda, Shoulda.''} In
\emph{Conceivability and Possibility}, edited by Tamar Szabó Gendler and
John Hawthorne, 441--92. Oxford: Oxford University Press.

\bibitem[\citeproctext]{ref-Zagzebski1994}
Zagzebski, Linda. 1994. {``The Inescapability of Gettier Problems.''}
\emph{The Philosophical Quarterly} 44 (174): 65--73. doi:
\href{https://doi.org/10.2307/2220147}{10.2307/2220147}.

\end{CSLReferences}



\noindent Published in\emph{
Oxford Handbook of Philosophical Methodlology}, 2016, pp. 231-248.


\end{document}
