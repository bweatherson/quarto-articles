% Options for packages loaded elsewhere
\PassOptionsToPackage{unicode}{hyperref}
\PassOptionsToPackage{hyphens}{url}
%
\documentclass[
  10pt,
  letterpaper,
  DIV=11,
  numbers=noendperiod,
  twoside]{scrartcl}

\usepackage{amsmath,amssymb}
\usepackage{setspace}
\usepackage{iftex}
\ifPDFTeX
  \usepackage[T1]{fontenc}
  \usepackage[utf8]{inputenc}
  \usepackage{textcomp} % provide euro and other symbols
\else % if luatex or xetex
  \usepackage{unicode-math}
  \defaultfontfeatures{Scale=MatchLowercase}
  \defaultfontfeatures[\rmfamily]{Ligatures=TeX,Scale=1}
\fi
\usepackage{lmodern}
\ifPDFTeX\else  
    % xetex/luatex font selection
  \setmainfont[ItalicFont=EB Garamond Italic,BoldFont=EB Garamond
Bold]{EB Garamond Math}
  \setsansfont[]{Europa-Bold}
  \setmathfont[]{Garamond-Math}
\fi
% Use upquote if available, for straight quotes in verbatim environments
\IfFileExists{upquote.sty}{\usepackage{upquote}}{}
\IfFileExists{microtype.sty}{% use microtype if available
  \usepackage[]{microtype}
  \UseMicrotypeSet[protrusion]{basicmath} % disable protrusion for tt fonts
}{}
\usepackage{xcolor}
\usepackage[left=1in, right=1in, top=0.8in, bottom=0.8in,
paperheight=9.5in, paperwidth=6.5in, includemp=TRUE, marginparwidth=0in,
marginparsep=0in]{geometry}
\setlength{\emergencystretch}{3em} % prevent overfull lines
\setcounter{secnumdepth}{3}
% Make \paragraph and \subparagraph free-standing
\ifx\paragraph\undefined\else
  \let\oldparagraph\paragraph
  \renewcommand{\paragraph}[1]{\oldparagraph{#1}\mbox{}}
\fi
\ifx\subparagraph\undefined\else
  \let\oldsubparagraph\subparagraph
  \renewcommand{\subparagraph}[1]{\oldsubparagraph{#1}\mbox{}}
\fi


\providecommand{\tightlist}{%
  \setlength{\itemsep}{0pt}\setlength{\parskip}{0pt}}\usepackage{longtable,booktabs,array}
\usepackage{calc} % for calculating minipage widths
% Correct order of tables after \paragraph or \subparagraph
\usepackage{etoolbox}
\makeatletter
\patchcmd\longtable{\par}{\if@noskipsec\mbox{}\fi\par}{}{}
\makeatother
% Allow footnotes in longtable head/foot
\IfFileExists{footnotehyper.sty}{\usepackage{footnotehyper}}{\usepackage{footnote}}
\makesavenoteenv{longtable}
\usepackage{graphicx}
\makeatletter
\def\maxwidth{\ifdim\Gin@nat@width>\linewidth\linewidth\else\Gin@nat@width\fi}
\def\maxheight{\ifdim\Gin@nat@height>\textheight\textheight\else\Gin@nat@height\fi}
\makeatother
% Scale images if necessary, so that they will not overflow the page
% margins by default, and it is still possible to overwrite the defaults
% using explicit options in \includegraphics[width, height, ...]{}
\setkeys{Gin}{width=\maxwidth,height=\maxheight,keepaspectratio}
% Set default figure placement to htbp
\makeatletter
\def\fps@figure{htbp}
\makeatother
% definitions for citeproc citations
\NewDocumentCommand\citeproctext{}{}
\NewDocumentCommand\citeproc{mm}{%
  \begingroup\def\citeproctext{#2}\cite{#1}\endgroup}
\makeatletter
 % allow citations to break across lines
 \let\@cite@ofmt\@firstofone
 % avoid brackets around text for \cite:
 \def\@biblabel#1{}
 \def\@cite#1#2{{#1\if@tempswa , #2\fi}}
\makeatother
\newlength{\cslhangindent}
\setlength{\cslhangindent}{1.5em}
\newlength{\csllabelwidth}
\setlength{\csllabelwidth}{3em}
\newenvironment{CSLReferences}[2] % #1 hanging-indent, #2 entry-spacing
 {\begin{list}{}{%
  \setlength{\itemindent}{0pt}
  \setlength{\leftmargin}{0pt}
  \setlength{\parsep}{0pt}
  % turn on hanging indent if param 1 is 1
  \ifodd #1
   \setlength{\leftmargin}{\cslhangindent}
   \setlength{\itemindent}{-1\cslhangindent}
  \fi
  % set entry spacing
  \setlength{\itemsep}{#2\baselineskip}}}
 {\end{list}}
\usepackage{calc}
\newcommand{\CSLBlock}[1]{\hfill\break\parbox[t]{\linewidth}{\strut\ignorespaces#1\strut}}
\newcommand{\CSLLeftMargin}[1]{\parbox[t]{\csllabelwidth}{\strut#1\strut}}
\newcommand{\CSLRightInline}[1]{\parbox[t]{\linewidth - \csllabelwidth}{\strut#1\strut}}
\newcommand{\CSLIndent}[1]{\hspace{\cslhangindent}#1}

\setlength\heavyrulewidth{0ex}
\setlength\lightrulewidth{0ex}
\usepackage[automark]{scrlayer-scrpage}
\clearpairofpagestyles
\cehead{
  Brian Weatherson
  }
\cohead{
  True, Truer, Truest
  }
\ohead{\bfseries \pagemark}
\cfoot{}
\makeatletter
\newcommand*\NoIndentAfterEnv[1]{%
  \AfterEndEnvironment{#1}{\par\@afterindentfalse\@afterheading}}
\makeatother
\NoIndentAfterEnv{itemize}
\NoIndentAfterEnv{enumerate}
\NoIndentAfterEnv{description}
\NoIndentAfterEnv{quote}
\NoIndentAfterEnv{equation}
\NoIndentAfterEnv{longtable}
\NoIndentAfterEnv{abstract}
\renewenvironment{abstract}
 {\vspace{-1.25cm}
 \quotation\small\noindent\rule{\linewidth}{.5pt}\par\smallskip
 \noindent }
 {\par\noindent\rule{\linewidth}{.5pt}\endquotation}
\addtokomafont{descriptionlabel}{\normalfont\bfseries}
\KOMAoption{captions}{tableheading}
\makeatletter
\@ifpackageloaded{caption}{}{\usepackage{caption}}
\AtBeginDocument{%
\ifdefined\contentsname
  \renewcommand*\contentsname{Table of contents}
\else
  \newcommand\contentsname{Table of contents}
\fi
\ifdefined\listfigurename
  \renewcommand*\listfigurename{List of Figures}
\else
  \newcommand\listfigurename{List of Figures}
\fi
\ifdefined\listtablename
  \renewcommand*\listtablename{List of Tables}
\else
  \newcommand\listtablename{List of Tables}
\fi
\ifdefined\figurename
  \renewcommand*\figurename{Figure}
\else
  \newcommand\figurename{Figure}
\fi
\ifdefined\tablename
  \renewcommand*\tablename{Table}
\else
  \newcommand\tablename{Table}
\fi
}
\@ifpackageloaded{float}{}{\usepackage{float}}
\floatstyle{ruled}
\@ifundefined{c@chapter}{\newfloat{codelisting}{h}{lop}}{\newfloat{codelisting}{h}{lop}[chapter]}
\floatname{codelisting}{Listing}
\newcommand*\listoflistings{\listof{codelisting}{List of Listings}}
\makeatother
\makeatletter
\makeatother
\makeatletter
\@ifpackageloaded{caption}{}{\usepackage{caption}}
\@ifpackageloaded{subcaption}{}{\usepackage{subcaption}}
\makeatother
\ifLuaTeX
  \usepackage{selnolig}  % disable illegal ligatures
\fi
\usepackage{bookmark}

\IfFileExists{xurl.sty}{\usepackage{xurl}}{} % add URL line breaks if available
\urlstyle{same} % disable monospaced font for URLs
\hypersetup{
  pdftitle={True, Truer, Truest},
  pdfauthor={Brian Weatherson},
  hidelinks,
  pdfcreator={LaTeX via pandoc}}

\title{True, Truer, Truest\thanks{Thanks to Juan Comesaña, John
Hawthorne, Allan Hazlett, Alyssa Ney, and audiences at the 2002 APA
Central, the University of Edinburgh and especially the 2003 BSPC for
helpful comments. From the latter I'm especially grateful to Jonathan
Bennett, Alex Byrne, Cian Dorr, Andy Egan, Elizabeth Harman, Robin
Jeshion, Mike Nelson, Jonathan Schaffer, Ted Sider and Gabriel Uzquiano.
I'm also grateful to two very helpful referee reports and to Restall
(\citeproc{ref-Restall2000}{1999}), without which I would never have
known enough about lattices to write this paper.}}
\author{Brian Weatherson}
\date{2005}

\begin{document}
\maketitle
\begin{abstract}
My theory of vagueness.
\end{abstract}

\setstretch{1.1}
What the world needs now is another theory of vagueness. Not because the
old theories are useless. Quite the contrary, the old theories provide
many of the materials we need to construct the truest theory of
vagueness ever seen. The theory shall be similar in motivation to
supervaluationism, but more akin to many-valued theories in
conceptualisation. What I take from the many-valued theories is the idea
that some sentences can be \emph{truer} than others. But I say very
different things to the ordering over sentences this relation generates.
I say it is not a linear ordering, so it cannot be represented by the
real numbers. I also argue that since there is higher-order vagueness,
any mapping between sentences and mathematical objects is bound to be
inappropriate. This is no cause for regret; we can say all we want to
say by using the comparative \emph{truer than} without mapping it onto
some mathematical objects. From supervaluationism I take the idea that
we can keep classical logic without keeping the familiar bivalent
semantics for classical logic. But my preservation of classical logic is
more comprehensive than is normally permitted by supervaluationism, for
I preserve classical inference rules as well as classical sequents. And
I do this without relying on the concept of acceptable precisifications
as an unexplained explainer.

The world does not need another guide to varieties of theories of
vagueness, especially since Timothy Williamson
(\citeproc{ref-Williamson1994-WILV}{1994}) and Rosanna Keefe
(\citeproc{ref-Keefe2000}{2000}) have already provided quite good
guides. I assume throughout familiarity with popular theories of
vagueness.

\section{Truer}\label{truer}

The core of my theory is that some sentences involving vague terms are
truer than others. I won't give an analysis of \emph{truer}, instead I
will argue that we already tacitly understand this relation. The main
argument for this will turn on a consideration of two `many-valued'
theories of vagueness, one of which will play a central role (as the
primary villain) in what follows.

The most familiar many-valued theory, call it \emph{M}, says there are
continuum many truth values, and they can be felicitously represented by
the interval {[}0,~1{]}. The four main logical connectives: \emph{and},
\emph{or}, \emph{if} and \emph{not} are truth-functional. The functions
are:

\hfill\break
~~~~\noindent *V\emph{(}A\emph{~∧~}B\emph{) = min(}V\emph{(}A\emph{),
}V\emph{(}B*))\\
\strut ~~~~\emph{V}(\emph{A}~∨~\emph{B}) = max(\emph{V}(\emph{A}),
\emph{V}(\emph{B}))\\
\strut ~~~~\emph{V}(\emph{A}~→~\emph{B}) = max(1, 1 - \emph{V}(\emph{A})
+ \emph{V}(\emph{B}))\\
\strut ~~~~~~~~~~\emph{V}(¬\emph{A}) = 1 - V(\emph{A})\\
\strut ~~~

\noindent where \emph{V} is the valuation function on sentences,
min(\emph{x},~\emph{y}) is the smaller of \emph{y} and \emph{y} and
max(\emph{x},~\emph{y}) is the larger of \emph{y} and \emph{y}.

Adopting these rules for the connectives commits us to adopting the
logic Ł\textsubscript{C}. \emph{M} is the theory that this semantic
model, under its most natural interpretation, is appropriate for vague
natural languages. (We'll discuss less natural interpretations
presently.)

\emph{M} tells a particularly nice story about the Sorites. A premise
like \emph{If she's rich, someone with just a little less money is also
rich} will have a very high truth value. If we make the difference in
money between the two subjects small enough, this conditional will have
a truth value arbitrarily close to 1.

\emph{M} also tells a nice story about borderline cases and
determinateness. An object \emph{a} is a borderline case of being an
\emph{F} just in case the sentence \emph{a is F} has a truth value
between 0 and 1 exclusive. Similarly, \emph{a} is a determinate \emph{F}
just in case the truth value of \emph{a is F} is 1. (It is worthwhile
comparing how simple this analysis of determinateness is to the
difficulties supervaluationists have in providing an analysis of
determinateness. On this topic, see Williamson
(\citeproc{ref-Williamson1995}{1995}), McGee and McLaughlin
(\citeproc{ref-McGee1998}{1998}) and Williamson
(\citeproc{ref-Williamson2004-WILRTM}{2004}).)

But \emph{M} tells a particularly implausible story about
contradictions. Here is how Timothy Williamson
(\citeproc{ref-Williamson1994-WILV}{1994}) makes this problem vivid.

\begin{quote}
More disturbing is that the law of non-contradiction fails \ldots.
¬(\emph{p}~∧¬\emph{p}) always has the same degree of truth as
\emph{p}~∨¬\emph{p}, and this is perfectly true only when \emph{p} is
perfectly true or perfectly false. When \emph{p} is half-true, so are
both \emph{p}~∨¬\emph{p} and ¬(\emph{p}~∧¬\emph{p}).
(\citeproc{ref-Williamson1994-WILV}{Williamson 1994, 118})

At some point {[}in waking up{]} `He is awake' is supposed to be
half-true, so `He is not awake' will be half-true too. Then `He is awake
and he is not awake' will count as half-true. How can an explicit
contradiction be true to any degree other than 0?
(\citeproc{ref-Williamson1994-WILV}{Williamson 1994, 136})
\end{quote}

There is a way to keep the semantic engine behind \emph{M} while
avoiding this consequence. (The following few paragraphs are indebted
pretty heavily to the criticisms of Strawson's theory of descriptions in
Dummett (\citeproc{ref-Dummett1959}{1959}))

Consider an interpretation of the above semantics on which there are
only two truth values: True and False. Any sentence that gets truth
value 1 is true, all the others are false. The numbers in {[}0, 1)
represent different ways of being false. (As Tolstoy might have put it,
all true sentences are alike, but every false sentence is false in its
own unique way.) Which way a sentence is false can affect the truth
value of compounds containing that sentence. In particular, if \emph{A}
and \emph{B} are false, then the truth values of \emph{Not A} and
\emph{If A then B} will depend on the ways \emph{A} and \emph{B} take
their truth values. If \emph{V}(\emph{A}) = 0 and \emph{V}(\emph{B}) =
0.3, then \emph{Not A} and \emph{If A then B} will be true, but if
\emph{V}(\emph{A}) becomes 0.6, and remember this is just another way of
being false, both \emph{Not A} and \emph{If A then B} will be false.

The new theory we get, one I'll call \emph{M\textsubscript{D}}, is
similar to \emph{M} in some respects. For example, it agrees about what
the axioms should be for a logic for natural language. But it has
several philosophical differences. In particular, it has none of the
three characteristics of \emph{M} we noted above.

It cannot tell as plausible story as \emph{M} does about the Sorites. If
any sentence with truth value below 1 is false, then many of the
premises in a Sorites argument are \emph{false}. This is terrible -- it
was bad enough to be told that one of the premises were false, but now
we find many thousands of them are false. I doubt that being told they
are false in a distinctive way will improve our estimation of the
theory. Similarly, it is hard to see just how the new theory has
anything interesting to say about the concept of a borderline case.

On the other hand, according to \emph{M\textsubscript{D}},
contradictions are always false. To be sure, a contradiction might be
false in some obscure new way, but it is still false. Recall
Williamson's objection that an explicit contradiction should be true to
degree 0 and nothing more. This objection only works if being true to
degree 0.5 is meant to be semantically significant. If being `true to
degree 0.5' is just another way of being false, then there is presumably
nothing wrong with contradictions are true to degree 0.5. This is not to
say Williamson's objection is no good, since he intended it as an
objection to \emph{M}, but just to say that re-interpreting the semantic
significance of the numbers in \emph{M} makes a philosophical
difference.

Despite \emph{M\textsubscript{D}}'s preferable treatment of
contradictions, I think \emph{M} is overall a better theory because it
has a much better account of borderline cases. But for now I want to
stress a simpler point: \emph{M} and \emph{M\textsubscript{D}} are
\emph{different} theories of vagueness, and that we grasp the difference
between these theories. One crucial difference between the two theories
is that in \emph{M}, but not \emph{M\textsubscript{D}},
\emph{S}\textsubscript{1} is truer than \emph{S}\textsubscript{2} if
\emph{V}(\emph{S}\textsubscript{1}) is greater than
\emph{V}(\emph{S}\textsubscript{2}). In \emph{M\textsubscript{D}}, if
\emph{S}\textsubscript{1}is truer than \emph{S}\textsubscript{2},
\emph{V}(\emph{S}\textsubscript{1}) must be one and
\emph{V}(\emph{S}\textsubscript{2}) less than one. And that is the
\emph{only} difference between the two theories. So if we understand
this difference, we must grasp this concept \emph{truer than}. Indeed,
it is in virtue of grasping this concept that we understand why saying
each of the Sorites conditionals is almost true is a \emph{prima facie}
plausible response to the Sorites, and why having a theory that implies
contradictions are truer than many other sentences is a rather
embarrassing thing.

I have implicitly defined \emph{truer} by noting its theoretical role.
As David Lewis (\citeproc{ref-Lewis1972a}{1972}) showed, terms can be
implicitly defined by their theoretical role. There is one unfortunate
twist here in that \emph{truer} is defined by its role in a \emph{false}
theory, but that does not block the implicit definition story. We know
what \emph{phlogiston} and \emph{ether} mean because of their role in
some false theories. The meaning of \emph{truer} can be extracted in the
same way from the false theory \emph{M}.

\section{\texorpdfstring{Further Reflections on
\emph{Truer}}{Further Reflections on Truer}}\label{further-reflections-on-truer}

As noted, I won't give a reductive analysis of \emph{truer}. The hopes
for doing that are no better than the hopes of giving a reductive
analysis of \emph{true}. But I will show that we pre-theoretically
understand the concept.

My primary argument for this has already been given. Intuitively we do
understand the difference between \emph{M} and its
\emph{M\textsubscript{D}}, and this is only explicable by our
understanding \emph{truer}. Hence we understand \emph{truer}.

Second, it's noteworthy that \emph{truer} is morphologically complex. If
we understand \emph{true}, and understand the modifier -\emph{er}, then
we know enough in principle to know how they combine. Not every
predicate can be turned into a comparative. But most can, and our
default assumption should be that \emph{true} is like the majority.

I have heard two arguments against that assumption. First, it could be
argued that most comparatives in English generate linear orderings, but
\emph{truer} generates a non-linear ordering. I reject the premise of
this argument. \emph{Cuter}, \emph{Smarter}, \emph{Smellier}, and
\emph{Tougher} all generate non-linear orderings over their respective
domains, and they seem fairly indicative of large classes. Second, it
could be argued that it's crucial to understanding comparatives that we
understand the interaction of the underlying adjectives with comparison
classes. Robin Jeshion and Mike Nelson made this objection in their
comments on my paper at BSPC 2003. Again, the premise is not obviously
true. We can talk about some objects being \emph{straighter} or
\emph{rounder} despite the fact that it's hard to understand \emph{round
for an office building} or \emph{straight for a line drive}. (Jonathan
Bennett made this point in discussion at BSPC.) \emph{Straight} and
\emph{round} either don't have or don't need comparison classes, but
they form comparatives. So \emph{true}, which also does not take
comparison classes, could also form a comparative.

Finally, if understanding the inferential role of a logical operator
helps know its meaning, then it is notable that \emph{truer} has a very
clear inferential role. It is the same as a strict material implication
\(\square\)(\emph{q}~⊃~\emph{p}) defined using a necessity operator
whose logic is KT. Since many operators have just this logic, this
doesn't individuate \emph{truer}, but it helps with inferential role
semantics aficionados.

I claim that the concept \emph{truer}, and the associated concept
\emph{as true as}, are the only theoretical tools we need to provide a
complete theory of vagueness. It is simplest to state the important
features of my theory by contrasting it with \emph{M}. I keep the
following good features of \emph{M}.

\begin{description}
\tightlist
\item[G1]
There are \emph{intermediate} sentences, i.e.~sentences that are truer
than some sentences and less true than others. For definiteness, I will
say \emph{S} is intermediate iff \emph{S} is truer than 0=1 and less
true than 0=0.
\item[G2]
\emph{a} is a borderline \emph{F} iff \emph{a is F} is intermediate, and
\emph{a} is determinately \emph{F} iff \emph{a} is \emph{F} and \emph{a}
is not a borderline \emph{F}.
\end{description}

I won't repeat the arguments here, but I take G1 to be a large advantage
of theories like \emph{M} over epistemicist theories. (See J. Burgess
(\citeproc{ref-Burgess2001}{2001}), Sider
(\citeproc{ref-Sider2001}{2001}) and Weatherson
(\citeproc{ref-Weatherson2003-WEAEPA}{2003a}) for more detailed
arguments to this effect.) And as noted G2 is a much simpler analysis of
determinacy and borderline than supervaluationists have been able to
offer.

I drop the following bad features of \emph{M}.

\begin{description}
\tightlist
\item[B1]
\emph{Some contradictions are intermediate sentences}.\\
On my theory all contradictions are determinately false, and
determinately determinately false, and so on. The argument for this has
been given above.
\item[B2]
\emph{Some classical tautologies are intermediate sentences}.\\
On my theory all classical tautologies are determinately true, and
determinately determinately true, and so on. We will note three
arguments for this being an improvement in the next section.
\item[B3]
\emph{Some classical inference rules are inadmissible}.\\
On my theory all classical inference rules are admissible. As Williamson
(\citeproc{ref-Williamson1994-WILV}{1994}) showed, the most prominent
version of supervaluationism is like \emph{M} in ruling some classical
rules to be inadmissible, and this is clearly a cost of those theories.
\item[B4]
\emph{Sentences of the form} S is intermediate \emph{are never
intermediate}\\
I will argue below this is a consequence of \emph{M}, and it means it is
impossible to provide a plausible theory of higher-order vagueness
within \emph{M}. In my theory we can say that there is higher-order
vagueness by treating \emph{truer} as an iterable operator, so we can
say that \emph{S is intermediate} is intermediate. If \emph{S} is
\emph{a is F}, that's equivalent to saying that \emph{a} is a borderline
case of a borderline case of an \emph{F}. Essentially we get out theory
of higher-order vagueness by simply iterating our theory of first-order
vagueness, which is what Williamson does in his justly celebrated
treatment of higher-order vagueness. Note it's not just \emph{M} that
has troubles with higher-order vagueness. See Williamson
(\citeproc{ref-Williamson1994-WILV}{1994}) and Weatherson
(\citeproc{ref-Weatherson2003-Keefe}{2003b}) for the difficulties
supervaluationists have with higher-order vagueness. The treatment of
higher-order vagueness here is a substantial advantage of my theory over
supervaluationism.
\item[B5]
\emph{Truer is a linear relation}.\\
On my theory it need not be the case that \emph{S}\textsubscript{1}is
truer than \emph{S}\textsubscript{2}, or \emph{S}\textsubscript{2} is
truer than \emph{S}\textsubscript{1}, or they are as true as each other.
In the last section I will argue that this is a substantial advantage of
my theory. I claim that \emph{truer} generates a Boolean lattice on
possible sentences of the language. (For a familiar example of a Boolean
lattice, think of the subsets of \(\mathbb{R}\) ordered by the subset
relation.)
\end{description}

I also provide a very different, and much more general, treatment of the
Sorites than is available within \emph{M}. The biggest technical
difference between my theory and \emph{M} concerns the relationship
between the semantics and the logic. In \emph{M} the logic falls out
from the truth-tables. Since I do not have the concept of an
intermediate truth value in my theory, I could not provide anything like
a truth-table. Instead I posit several constraints on the interaction of
\emph{truer} with familiar connectives, posit an analysis of validity in
terms of \emph{truer}, and note that those two posits imply that all and
only classically admissible inference rules are admissible.

\section{Constraints on Truer and Classical
Logic}\label{constraints-on-truer-and-classical-logic}

The following ten constraints on \emph{truer} seem intuitively
compelling. I've listed here both the philosophically important informal
claim, and the formal interpretation of that claim. (I use
\emph{A}~⩾\textsubscript{T}~\emph{B} as shorthand for \emph{A} \emph{is
at least as} \emph{true as B}. Note all the quantifiers over sentences
here are \emph{possibilist} quantifiers, we quantify over all possible
sentences in the language.)

\begin{description}
\item[(A1)]
⩾\textsubscript{T} is a weak ordering (i.e.~reflexive and transitive)\\
If \emph{A}~⩾\textsubscript{T}~\emph{B} and
\emph{B}~⩾\textsubscript{T}~\emph{C} then
\emph{A}~⩾\textsubscript{T}~\emph{C}\\
\emph{A}~⩾\textsubscript{T}~\emph{A}
\item[(A2)]
~∧~is a greatest lower bound with respect to ⩾\textsubscript{T}\\
\emph{A}~∧~\emph{B}~⩾\textsubscript{T}~\emph{C} iff
\emph{A}~⩾\textsubscript{T}~\emph{C} and
\emph{B}~⩾\textsubscript{T}~\emph{C}\\
\emph{C}~⩾\textsubscript{T}~\emph{A}~∧~\emph{B} iff for all \emph{S}
such that \emph{A}~⩾\textsubscript{T}~\emph{S} and
\emph{B}~⩾\textsubscript{T}~\emph{S} it is also the case that
\emph{C}~⩾\textsubscript{T}~\emph{S}
\item[(A3)]
∨ is a least upper bound with respect to ⩾\textsubscript{T}\\
\emph{A}~∨~\emph{B}~⩾\textsubscript{T}~\emph{C} iff for all \emph{S}
such that \emph{S}~⩾\textsubscript{T}~\emph{A}~and
\emph{S}~⩾\textsubscript{T}~\emph{B}, it is also the case that
\emph{S}~⩾\textsubscript{T}~\emph{C}\\
\emph{C}~⩾\textsubscript{T}~\emph{A}~∨~\emph{B} iff
\emph{C}~⩾\textsubscript{T}~\emph{A} and
\emph{B}~⩾\textsubscript{T}~\emph{C}
\item[(A4)]
¬ is ordering inverting with respect to ⩾\textsubscript{T}\\
\emph{A}~⩾\textsubscript{T}~\emph{B} iff
¬\emph{B}~⩾\textsubscript{T}~¬\emph{A}
\item[(A5)]
Double negation is redundant\\
¬¬\emph{A}~=\textsubscript{T}~\emph{A}
\item[(A6)]
There is an absolutely false sentence \emph{S\textsubscript{F}} and an
absolutely true sentence \emph{S\textsubscript{T}}\\
There are sentences \emph{S\textsubscript{F}} and
\emph{S\textsubscript{T}} such that
\emph{S\textsubscript{F}}~=\textsubscript{T}~¬\emph{S\textsubscript{T}}
and
¬\emph{S\textsubscript{F}}~=\textsubscript{T}~\emph{S\textsubscript{T}}
and for all \emph{S}:
\emph{S\textsubscript{T}}~⩾\textsubscript{T}~S~⩾\textsubscript{T}~\emph{S\textsubscript{F}}
\item[(A7)]
Contradictions are absolutely false\\
A~∧~¬A =\textsubscript{T} \emph{S\textsubscript{F}}
\item[(A8)]
∀ is a greatest lower bound with respect Φ to ⩾\textsubscript{T}\\
\emph{A}~⩾\textsubscript{T}~∀\emph{x}(Φ\emph{x})~iff for all \emph{S}
such that for all \emph{o}, if \emph{n} is a name of \emph{o} then
Φ\emph{n}~⩾\textsubscript{T}~\emph{S}, it is the case that
\emph{A}~⩾\textsubscript{T}~\emph{S}\\
∀\emph{x}(Φ\emph{x})~⩾\textsubscript{T}~\emph{A} iff for all \emph{o},
if \emph{n} is a name of \emph{o} then
Φ\emph{n}~⩾\textsubscript{T}~\emph{A}
\item[(A9)]
∃ is a least upper bound with respect to ⩾\textsubscript{T}\\
\emph{A}~⩾\textsubscript{T}~∃\emph{x}(Φ\emph{x})~iff for all \emph{o},
if \emph{n} is a name of \emph{o} then
\emph{A}~⩾\textsubscript{T}~Φ\emph{n}\\
∃\emph{x}(Φ\emph{x})~⩾\textsubscript{T}~\emph{A} iff for all \emph{S}
such that for all \emph{o}, if \emph{n} is a name of \emph{o} then
\emph{S}~⩾\textsubscript{T}~Φ\emph{n},
\emph{S}~⩾\textsubscript{T}~\emph{A}
\item[(A10)]
A material implication with respect to ⩾\textsubscript{T} can be
defined.\\
There is an operator~→~such that

\begin{enumerate}
\def\labelenumi{\arabic{enumi}.}
\tightlist
\item
  \emph{B}~→~\emph{A}~⩾\textsubscript{T}~\emph{S\textsubscript{T}} iff
  \emph{A}~⩾\textsubscript{T}~\emph{B}
\item
  (\emph{A}~∧~\emph{B})~→~\emph{C}~=\textsubscript{T}~A~→~(\emph{B}~→~\emph{C})
\end{enumerate}
\end{description}

Apart from (A10) these are fairly straightforward. We can't argue for
(A10) by saying English \emph{if\ldots then} is a material implication,
because that leads directly to the paradoxes of material implication.
Assuming that ¬\emph{A}~∨~\emph{B} is a material implication is
equivalent to assuming (inter alia) that \emph{A}~∨~¬\emph{A} is
perfectly true. I believe this, but since it is denied by many I want
that to be a conclusion, not a premise. So the argument for (A10) must
be a little indirect. In particular, we will appeal to the behaviour of
quantifiers. We can formally represent \emph{All Fs are Gs} in two ways:
using restricted or unrestricted quantifiers. In the first case the
formal representation will look like:

\begin{quote}
∀\emph{x}(\emph{Fx}~?~\emph{Gx})
\end{quote}

with some connective in place of `?'. But it seems clear that whatever
connective goes in there must be a material implication. In the second
case, the formal representation will look like:

\begin{quote}
{[}∀\emph{x}:\emph{Fx}{]}~\emph{Gx}
\end{quote}

In that case, we can define a connective ∇ that satisfies the definition
of a material implication:

\begin{quote}
\emph{A}~∇~\emph{B}~=\textsubscript{df}
{[}∀\emph{x}:~\emph{A}~∧~\emph{x}=\emph{y}{]}(\emph{B}~∧~\emph{x}=\emph{y})
\end{quote}

This is equivalent to the odd (but intelligible) sentence
\emph{Everything such that A is such that B}. Again, considerations
about what should be logical truths involving quantifiers suggests that
∇ must be a material implication. So either way there should be a
material implication present in the language, as (A10) says.

Given (A1) to (A10) it follows that this material implication is
equivalent to ¬\emph{A}~∨~\emph{B}, and hence \emph{A}~∨~¬\emph{A} is a
logical truth. This is a surprising \emph{conclusion}, since intuitively
vagueness poses problems for excluded middle, but I think it is more
plausible that vague instances of excluded middle are problematic for
\emph{pragmatic} reasons than that any of (A1) to (A10) are false.

What is interesting about these ten constraints is that they suffice for
classical logic, with just one more supposition. I assume that an
argument is \emph{valid} iff it is impossible for the premises taken
collectively to be truer than the conclusion, i.e.~iff it is impossible
for the conjunction of the premises to be truer than the conclusion.
Given that, we get:

\begin{quote}
∀\emph{A}\textsubscript{1}, \ldots,\emph{A\textsubscript{n}}, \emph{B}:
\emph{A}\textsubscript{1}, \ldots,
\emph{A\textsubscript{n}}~\(\vdash\)\textsubscript{T}~\emph{B} iff,
according to classical logic, \emph{A}\textsubscript{1}, \ldots,
\emph{A\textsubscript{n}}~\(\vdash\)\textsubscript{T}~\emph{B}
\end{quote}

(I use Γ~\(\vdash\)\textsubscript{T}~\emph{A} to mean that in all models
for ⩾\textsubscript{T} that satisfy the constraints, here (A1) to (A10),
the conclusion is at least as true as the greatest lower bound of the
premises.) I won't prove this result, but the idea is that (A1) to (A10)
imply that ⩾\textsubscript{T} defines a Boolean lattice over equivalence
classes of sentences with respect to =\textsubscript{T}. And all Boolean
lattices are models for classical logic, from which our result follows.
Indeed, Boolean lattices are models for classical logic in the strong
sense that classical inference rules, such as conditional proof and
\emph{reductio}, are admissible in logics defined on them, so we also
get the admissibility of classical inference rules in this theory. (Note
that this result only holds in the right-to-left direction for languages
that contain the ⩾\textsubscript{T} operator. Once this operator is
added, some arguments that are not classically valid, such as
\emph{B}~⩾\textsubscript{T}~\emph{A},
\emph{A}~\(\vdash\)\textsubscript{T}~\emph{B}, will valid. But the
addition of this operator is conservative: if we look at the
⩾\textsubscript{T}-free fragment of such languages, the above result
still holds in both directions.)

There are three reasons for wanting to keep classical logic in a theory
of vagueness. First, as Williamson has stressed, classical logic is at
the heart of many successful research programs. Second, non-classical
theories of vagueness tend to abandon too much of classical logic. For
instance, \emph{M} abandons the very plausible schema
(\emph{A}~∧~(A~→~B))~→~\emph{B}. The third reason is the one given here
- these ten independently plausible constraints on \emph{truer} entail
that the logic for a language containing \emph{truer} should be
classical. These three arguments add up to a powerful case that
non-classical theories like \emph{M} are mistaken, and we should prefer
a theory that preserved classical logic.

\section{Semantics and Proof Theory}\label{semantics-and-proof-theory}

In this section I will describe a semantics and proof theory for a
language containing \emph{truer} as an iterable operator. This is
important for the theory of higher-order vagueness. I say that \emph{a}
is a borderline borderline \emph{F} just in case the sentence \emph{a is
a borderline F} is intermediate, where `borderline' is analysed as in
section 2. It might not be obvious that it is consistent with (A1) to
(A10) that any sentence \emph{a is a borderline F} could be consistent.
One virtue of the model theory outlined here is that it shows this is
consistent.

For comparison, note that \emph{M} as it stands has no way of dealing
with higher-order vagueness, i.e.~with borderline cases of borderline
cases of \emph{F}-ness. If every sentence \emph{a is a borderline F}
either does or does not receive an integer truth value, then this
intuitive possibility is ruled out. We cannot solve the problem simply
by iterating \emph{M}. (This is a point stressed by
(\citeproc{ref-Williamson1994-WILV}{Williamson 1994} Ch. 4).) We cannot
say that it is true to degree 0.5 than (2) is true to degree 1, and true
to degree 0.5 that it is true to degree 0.8. For then it is only true to
degree 0.5 that (2) has some truth value or other. And the use of
truth-tables to generate a logic presupposes that every sentence has
some truth values or other. If this is not determinately true, \emph{M}
is not a complete theory. So the model theory will show that our theory
is substantially better than \emph{M} in this respect.

Consider the following (minor) variant on KT. Vary the syntax so
\(\square\)\emph{A} is only well-formed if \emph{A} is of the form
\emph{B}~→~\emph{C}. Call the resulting logic KT\textsubscript{R}, with
the R indicating the syntactic restriction. The restriction makes very
little difference. Since \emph{A} is equivalent to
(\emph{A}~→~\emph{A})~→~\emph{A}, even if \(\square\)\emph{A} is not
well-formed in KT\textsubscript{R}, the KT-equivalent sentence
\(\square\)((\emph{A}~→~\emph{A})~→~\emph{A}) will be well-formed. The
Kripke models for KT\textsubscript{R} are quite natural.
\(\square\)(\emph{B}~→~\emph{C}) is true at a point iff all accessible
points at which \emph{B} is true are points at which \emph{C} is true.
(There is no restriction on the accessibility relation other than
reflexivity.)

Since KT\textsubscript{R} is so similar to KT, we can derive most of its
formal properties by looking at the derivations of similar properties
for KT. (The next few paragraphs owe a lot to
(\citeproc{ref-Goldblatt1992}{Goldblatt 1992, Chs.1--3}).) Let's start
with an axiomatic proof theory. The axioms for KT\textsubscript{R} are:

\begin{itemize}
\tightlist
\item
  All classical tautologies
\item
  All well-formed instances of K:
  \(\square\)(\emph{A}~→~\emph{B})~→~(\(\square\)\emph{A}~→~\(\square\)\emph{B})
\item
  All well-formed instances of T: \(\square\)\emph{A}~→~\emph{A}
\end{itemize}

The rules for KT\textsubscript{R} are

\begin{itemize}
\tightlist
\item
  \textbf{Modus Ponens}: If \emph{A}~→~\emph{B} is a theorem and
  \emph{A} is a theorem, then \emph{B} is a theorem
\item
  \textbf{Restricted Necessitation}: If \emph{A} is a theorem and
  \(\square\)\emph{A} is well-formed, then \(\square\)\emph{A} is a
  theorem.
\end{itemize}

Given these, we can now define a maximal consistent set for
KT\textsubscript{R}. It is a set \emph{S} of sentences with the
following three properties:

\begin{itemize}
\tightlist
\item
  All theorems of KT\textsubscript{R} are in \emph{S}.
\item
  For all \emph{A}, either \emph{A} is in \emph{S} or ¬\emph{A} is in
  \emph{S}.
\item
  \emph{S} is closed under modus ponens.
\end{itemize}

The existence of Kripke models for KT\textsubscript{R} show that some
maximal consistent sets exist: the set of truths at any point will be a
maximal consistent set. The canonical model for KT\textsubscript{R} is
\(\langle\)\emph{W},~\emph{R},~\emph{V}\(\rangle\) where

\begin{itemize}
\tightlist
\item
  \emph{W} is the set of maximal consistent sets for KT\textsubscript{R}
\item
  \emph{R} is a subset of \emph{W}~⨉~\emph{W} such that
  \emph{w}\textsubscript{1}\emph{Rw}\textsubscript{2} iff for all
  \emph{A} such that \(\square\)\emph{A}~∈~\emph{w}\textsubscript{1},
  \emph{A}~∈~\emph{w}\textsubscript{2}
\item
  \emph{V} is the valuation such that \emph{V}(\emph{A})~= \{\emph{w}:
  \emph{A}~∈~\emph{w}\}
\end{itemize}

Since all instances of T are theorems, it can be easily shown that
\emph{R} is reflexive, and hence that this is a frame for
KT\textsubscript{R} and hence that KT\textsubscript{R} is canonically
complete.

We can translate all sentences of KT\textsubscript{R} into a language
that contains ⩾\textsubscript{T} but not \(\square\). Just replace
\(\square\)(\emph{B}~→~\emph{A}) with
\emph{A}~⩾\textsubscript{T}~\emph{B} wherever \(\square\) occurs
including inside sentences. (We appeal here and here alone to the
restriction in KT\textsubscript{R}.) Translating the axioms for
KT\textsubscript{R}, we get the following axioms for the logic of
⩾\textsubscript{T}.

\begin{itemize}
\tightlist
\item
  All classical tautologies
\item
  All instances of:
  (\emph{B}~~⩾\textsubscript{T}~\emph{A})~→~((\emph{A}~⩾\textsubscript{T}~(\emph{A}~→~\emph{A}))~→
  (\emph{B}~⩾\textsubscript{T}\$~(\emph{B}~→~\emph{B})))
\item
  All instances of:
  (\emph{B}~⩾\textsubscript{T}~\emph{A})~→~(\emph{A}~→~\emph{B})
\end{itemize}

The rules are

\begin{itemize}
\tightlist
\item
  \textbf{Modus ponens}: If \emph{A}~→~\emph{B} is a theorem and
  \emph{A} is a theorem, then \emph{B} is a theorem.
\item
  \textbf{Determination}: If \emph{A}~→~\emph{B} is a theorem, then
  \emph{B}~⩾\textsubscript{T}~\emph{A} is a theorem.
\end{itemize}

We can simplify somewhat by replacing the second axiom schema with

\begin{itemize}
\tightlist
\item
  All instances of: \emph{A}~⩾\textsubscript{T}~B
  →~(\emph{B}~⩾\textsubscript{T}~C~→~A~⩾\textsubscript{T}~\emph{C})
\end{itemize}

A Kripke model for this logic is just a Kripke model for KT, except we
say \emph{B}~⩾\textsubscript{T}~\emph{A} is true at a point iff \emph{B}
is true at all accessible points at which \emph{A} is true. This leads
to a semantic definition of validity. An argument is valid iff it
preserves truth at any point in all such models.

Maximal consistent sets with respect to ⩾\textsubscript{T} and a
canonical model for ⩾\textsubscript{T} can be easily constructed by
parallel with the maximal consistent sets and canonical models for
KT\textsubscript{R}. These constructions show that if \emph{A} is a
theorem of the logic for ⩾\textsubscript{T}, then it is true at all
points in all models. More generally, they can be used to show that this
logic is canonically complete, though the details of the proof are
omitted. The maximal consistent sets for ⩾\textsubscript{T}, i.e.~the
points in the canonical model, just are the results of applying the
translation rule \(\square\)(\emph{B}~→~\emph{A})~→
A~⩾\textsubscript{T}~\emph{B} to the (sentences in the) maximal
consistent sets for KT\textsubscript{R}.

That's important because the points in the canonical model for
⩾\textsubscript{T} are useful for understanding the relationship between
\emph{truer} and \emph{true}, and for understanding what languages are.
The set of true sentences in English is one of the points in the
canonical model for ⩾\textsubscript{T}. For semantic purposes, languages
just are points in this canonical model. It is indeterminate just which
such point English is, but it is one of them. For many purposes it is
useful to think of the theory based on \emph{truer} as a variant on
\emph{M}. But considering the canonical model for ⩾\textsubscript{T}
highlights the similarities with supervaluationism rather than the
similarities with \emph{M}, for the points in the canonical model look a
lot like precisifications. It is, however, worth noting the many
differences between my theory and supervaluationism. I identify
languages with a single point rather than with a set of points, which
leads to the smoother treatment of higher-order vagueness on my account.
Also, I don't start with a set of acceptable points/precisifications.
The canonical model contains all the points that are formally
consistent, and I identify particular languages, like English, by
vaguely saying that the point that represents English is (roughly)
there. (Imagine my vaguely pointing at some part of the model when
saying this.) The most important difference is that I take the points,
with the truer than relation already defined, to be primitive, and the
accessibility/acceptability relation to be defined in terms of them.
This reflects the fact that I take the \emph{truer} relation to be
primitive, and determinacy to be defined in terms of it, whereas
typically supervaluationists do things the other way around. None of
these differences are huge, but they all favour my theory over
supervaluationism.

To return to the point about higher order vagueness, note that all of
the following sentences are consistent in KT, and hence their
`equivalents' using \textgreater{}\textsubscript{T} are also consistent.

And obviously this pattern can be extended indefinitely. In general, any
claim of the form that \emph{a} is an \emph{n}-th order borderline case
of an \emph{F} is consistent in this theory, as can be seen by
comparison with KT.

To close this section, I will note that we can also provide a fairly
straightforward natural deduction system for the logic of
⩾\textsubscript{T}. There are two philosophical benefits to doing this.
First, it proves my earlier claim that I can keep all inference rules of
classical logic. Second, it helps justify (A1) to (A10). Most rules
correspond directly to one of the constraints. For that reason I've set
all the rules, even though you've probably seen most of them
before.\footnote{Thanks to Gabriel Uzquiano for several probing
  questions that led to this section being written.}

~~\textbf{(∧ In)}: Γ \(\vdash\) \emph{A}, Δ \(\vdash\) \emph{B}~→~Γ ∪ Δ
\(\vdash\) \emph{A}~∧~\emph{B}\\
\strut ~~\textbf{(∧ Out-left)}: Γ \(\vdash\) \emph{A}~∧~\emph{B} → Γ
\(\vdash\) \emph{A}\\
\strut ~~\textbf{(∧ Out-right)}: Γ \(\vdash\) \emph{A}~∧~\emph{B} → Γ
\(\vdash\) \emph{B}\\
\strut ~~\textbf{(∨ In-left)}: Γ \(\vdash\) \emph{B} →~Γ \(\vdash\)
\emph{A}~∨~\emph{B}\\
\strut ~~\textbf{(∨ In-right)}: Γ \(\vdash\) \emph{A} →~Γ \(\vdash\)
\emph{A}~∨~\emph{B}\\
\strut ~~\textbf{(∨ Out)}: Γ ∪ \{\emph{A}\} \(\vdash\) \emph{C}, Δ ∪
\{\emph{B}\} \(\vdash\) \emph{C}, Λ \(\vdash\)~\emph{A}~∨~B → Γ ∪ Δ ∪ Λ
\(\vdash\) \emph{C}\\
\strut ~~\textbf{(→ In)}: Γ ∪ \{\emph{A}\} \(\vdash\) B →~Γ \(\vdash\)
\emph{A}~→~\emph{B}\\
\strut ~~\textbf{(→ Out)}: Γ \(\vdash\) \emph{A}~→~\emph{B}, Δ
\(\vdash\)~\emph{A}~→~Γ ∪ Δ \(\vdash\) \emph{B}\\
\strut ~~\textbf{(¬ In)}: Γ ∪~\{\emph{A}\} \(\vdash\)
\emph{B}~∧~¬\emph{B} →~Γ \(\vdash\) ¬\emph{A}\\
\strut ~~\textbf{(¬ Out)}: Γ \(\vdash\) ¬¬\emph{A} → Γ \(\vdash\)
\emph{A}\\
\strut ~~\textbf{(⩾\textsubscript{T} In)}: Γ \(\vdash\) \emph{A} →
\{\emph{B}~⩾\textsubscript{T}~\emph{C}: \emph{B}~∈~ Γ\}
\(\vdash\)~\emph{A}~⩾\textsubscript{T}~\emph{C}\\
\strut ~~\textbf{(⩾\textsubscript{T} Convert)}: Γ
\(\vdash\)~\emph{A}~⩾\textsubscript{T}~B →~Γ \(\vdash\)
(\emph{B}~→~\emph{A})~⩾\textsubscript{T}~\emph{C}\\
\strut ~~\textbf{(⩾\textsubscript{T} Out)}: Γ \(\vdash\)
\emph{A}~⩾\textsubscript{T}~B →~Γ \(\vdash\)~\emph{B}~→~\emph{A}

\section{Sexy Sorites}\label{sexy-sorites}

A good theory of vagueness should tell us two things about the Sorites.
The easy part is to say what is wrong with Sorites arguments: not all
premises are perfectly true. The hard part is to say why the premises
looked plausible to start with. The \emph{M} theorist has the beginnings
of a story, though not the end of a story. The beginning is that all the
premises in a typical Sorites argument are nearly true, and they look
plausible because we confuse near truth for truth. Can I say the same
thing, since my theory is like \emph{M}? No, for two reasons. First,
since my theory explicitly gets rid of numerical representations of
intermediate truth values, I don't have any way to analyse \emph{almost
true}. Second, since I say that one of the Sorites premises is false,
I'd be committed to the odd view that some false sentence is almost
perfectly true. Thanks to Cian Dorr for pointing out this consequence.

The story the \emph{M} theorist tells does not generalise. The problem
is that not all Sorites arguments involve conditionals. A typical
Sorites situation involves a chain from a definite \emph{F} to a
definite not-\emph{F}. Let ´ denote the successor relation in this
sequence, so if \emph{F} is \emph{is tall} and \emph{a} is 178cm tall,
then \emph{a}´ will be 177.99cm tall, assuming the sequence progresses
0.1mm at a time. According to \emph{M}, every premise like (SI) is
almost true.

\begin{description}
\item[(SI)]
If \emph{a} is tall, then \emph{a}´ is tall.
\end{description}

But we could have built a Sorites argument with premises like (SA).

\begin{description}
\item[(SA)]
It is not the case that \emph{a} is tall and \emph{a}´ is not tall.
\end{description}

And premises of this form are not, in general, almost true. Indeed, some
will have a truth value not much about 0.5. So \emph{M} has no
explanation for why premises like (SA) look persuasive. This is quite
bad, because (SA) is \emph{more} plausible than (SI) as I'll now show.
Consider the following thought experiment. You are trying to get a group
of (typically non-responsive) undergraduates to appreciate the force of
the Sorites paradox. If they don't feel the force of (SI), how do you
persuade them? My first instinct is to appeal to something like (SA). If
that doesn't work, I appeal to theoretical considerations about how our
use of \emph{tall} couldn't possibly pick a boundary between \emph{a}
and \emph{a}´. I think I find (SI) plausible \emph{because} I find (SA)
plausible, and I would try to get the students to feel likewise. There's
an asymmetry here. I wouldn't defend (SA) by appealing to (SI), and I
don't find (SA) plausible because it follows from (SI). (This is not to
endorse universally quantified versions of either (SA) or (SI). They are
like Axiom V - claims that remain intuitively plausible even when we
know they are false.)

Sadly, many theories have little to say about why (SA) seems true. The
official epistemicist story is that speakers only accept sentences that
are determinately, i.e.~knowably, true. But some instances of (SA) are
actually \emph{false}, and many many more are not knowably true. The
supervaluationist story about (SA) is no better.

Here's a surprising fact about the Sorites that puts an unexpected
constraint on explanations of why (SA) is plausible. In the history of
debates about it, I don't think anyone has put forward a Sorites
argument where the major premises are like (SO).

\begin{description}
\item[(SO)]
Either \emph{a} is not tall, or \emph{a}´ is tall.
\end{description}

(This point is also noticed in Braun and Sider
(\citeproc{ref-SiderBraun}{2007}).) There's a good reason for this: (SO)
is \emph{not} intuitively true, unless perhaps one sees it as a
roundabout way of saying (SA). In this respect it conflicts quite
sharply with (SA), which \emph{is} intuitively true. But hardly any
theory of vagueness (certainly not \emph{M} or supervaluationism or
epistemicism) provide grounds for distinguishing (SA) from (SO), since
most theories of vagueness endorse DeMorgan's laws. Further, none of the
many and varied recent solutions to the Sorites that do not rely on
varying the underlying logic (e.g. (\citeproc{ref-Fara2000}{Fara 2000};
\citeproc{ref-Sorensen2001}{Sorensen 2001};
\citeproc{ref-Eklund2002}{Eklund 2002})) seem to do any better at
distinguishing (SA) from (SO). As far as I can tell none of these
theories \emph{could}, given their current conceptual resources, tell a
story about why (SA) is intuitively plausible that does not falsely
predict (SO) is intuitively plausible. That is, none of these theories
could solve the Sorites paradox with their current resources.

There is, however, a simple theory that does predict that (SA) will look
plausible while (SO) will not. Kit Fine (\citeproc{ref-Fine1975a}{1975})
noted that if we assume that speakers systematically confuse \emph{p}
for \emph{Determinately p}, even when \emph{p} occurs as a constituent
of larger sentences rather than as a standalone sentence, then we can
explain why speakers may accept vague instances of the law of
non-contradiction, but not vague instances of the law of excluded
middle. (That speakers do have these differing reactions to the two laws
has been noted in a few places, most prominently J. A. Burgess and
Humberstone (\citeproc{ref-Burgess1987}{1987}) and Tappenden
(\citeproc{ref-Tappenden1993}{1993}).) It's actually rather remarkable
how many true predictions one can make using Fine's hypothesis. It
correctly predicts that (5) should sound acceptable.

Now (5) is a contradiction, so both the fact that it sounds acceptable
if I am a borderline case of vagueness, and the fact that some theory
predicts this, are quite remarkable. This is about as good as it gets in
terms of evidence for a philosophical claim.

(We might wonder just why Fine's hypothesis is true. One idea is that
there really isn't any difference in truth value between \emph{p} and
\emph{Determinately p}. This leads to the absurd position that some
contradictions, like (5), are literally true. I prefer the following
two-part explanation. The first part is that when one utters a simple
subject-predicate sentence, one implicates that the subject
\emph{determinately} satisfies the predicate. This is a much stronger
implicature than conversational implicature, since it is not
cancellable. And it does not seem to be a conventional implicature.
Rather, it falls into the category of nonconventional nonconversational
implicatures Grice suggests exists on pg. 41 of his
(\citeproc{ref-Grice1989}{1989}). The second part is that some
implicatures, including determinacy implicatures, are computed locally
and the results of the computations passed up to whatever system
computes the intuitive content of the whole sentence. This implies that
constituents of sentences can have implicatures. This theme has been
studied quite a bit recently; see Levinson
(\citeproc{ref-Levinson2000}{2000}) for a survey of the linguistic data
and Sedivy et al. (\citeproc{ref-Sedivy1999}{1999}) for some empirical
evidence supporting up this claim. Just which, if any, implicatures are
computed locally is a major research question, but there is \emph{some}
evidence that Fine's hypothesis is the consequence of a relatively deep
fact about linguistic processing. This isn't essential to the current
project - really all that matters is that Fine's hypothesis is true -
but it does suggest some interesting further lines of research and
connections to ongoing research projects.)

If Fine's hypothesis is true, then we have a simple explanation for the
attractiveness of (SA). Speakers regularly confuse (SA) for (6), which
is true, while they confuse (SO) for (7), which is false.

This explanation cannot \emph{directly} explain why speakers find (SI)
attractive. My explanation for this, however, has already been given.
The intuitive force behind (SI) comes from the fact that it follows, or
at least appears to follow, from (SA), which looks practically
undeniable.

So Fine's hypothesis gives us an explanation of what's going on in
Sorites arguments that is available in principle to a wide variety of
theorists. Fine proposed it in part to defend a supervaluationist
theory, and Keefe (\citeproc{ref-Keefe2000}{2000}) adopts it for a
similar purpose. Patrick Greenough (\citeproc{ref-Greenough2003}{2003})
has recently adopted a similar looking proposal to provide an
epistemicist explanation of similar data. (Nothing in the explanation of
the attractiveness of Sorites premises turns on any \emph{analysis} of
determinacy, so the story can be told by epistemicists and
supervaluationists alike.) And the story can be added to the theory of
\emph{truer} sketched here. It might be regretted that we don't have a
\emph{distinctive} story about the Sorites in terms of \emph{truer}. But
the hypothesis that some sentences are truer than others is basically a
\emph{semantic} hypothesis, and if the reason Sorites premises look
attractive is anything like the reason (5) looks \emph{prima facie}
attractive, then that attractiveness should receive a \emph{pragmatic}
explanation. What is really important is that there be some story about
the Sorites we can tell.

\section{Linearity Intuitions}\label{linearity-intuitions}

The assumption that \emph{truer} is a non-linear relation is the basis
for most of the distinctive features of my theory, so it should be
defended. There are two reasons to believe it.

One is that we can't simultaneously accept all of the following five
principles.

\begin{itemize}
\tightlist
\item
  \emph{Truer} is a linear relation.
\item
  (A2), that conjunction is a greatest lower bound.
\item
  (A4), that negation is order inverting.
\item
  (A7), that contradictions are determinately false.
\item
  There are indeterminate sentences.
\end{itemize}

I think by far the least plausible of these is the first, so it must go.

Linearity (or at least determinate linearity) also makes it difficult to
tell a plausible story about higher order vagueness. Linearity is the
claim that for any two sentences \emph{A} and \emph{B}, the following
disjunction holds. Either
\emph{A}~\textgreater{}\textsubscript{T}~\emph{B}, or
\emph{B}~\textgreater{}\textsubscript{T}~\emph{A}, or
\emph{A}~=\textsubscript{T}~\emph{B}. If truer is determinately linear,
that disjunction is determinately true. And if \emph{truer} is linear,
and if that disjunction is determinately true, then one of its disjuncts
must be determinately true, for linearity rules out the possibility of a
determinately true disjunction with no determinately true disjunct. Now
take a special case of that disjunction, where \emph{B} is 0=0. In that
case we can rule out \emph{A}~\textgreater{}\textsubscript{T}~\emph{B}.
So the only options are
\emph{B}~\textgreater{}\textsubscript{T}~\emph{A} or
\emph{A}~=\textsubscript{T}~\emph{B}. We have concluded that given
linearity, one of these disjuncts must be determinately true. That is,
\emph{A} is either determinately intermediate or determinately
determinate. But intuitively neither of these need be true, for \emph{A}
might be in the `penumbra' between the determinately intermediate and
the determinately determinate. This argument is only a problem if we
assume determinate linearity, but it's hard to see the theoretical
motivation for believing in linearity but not determinate linearity.

Still, it is very easy to believe in linearity. Even for comparatives
that are clearly non-linear, like \emph{more intelligent than}, there is
a strong temptation to treat them as linear. (Numerical measurements of
intelligence are obviously inappropriate given that \emph{more
intelligent than} is non-linear, but there's a large industry involved
in producing such measurements.) And this temptation leads to some prima
facie plausible objections to my theory. (All of these objections arose
in the discussion of the paper at BSPC.)

\subsection*{True and Truer (due to Cian
Dorr)}\label{true-and-truer-due-to-cian-dorr}
\addcontentsline{toc}{subsection}{True and Truer (due to Cian Dorr)}

Here's an odd consequence of my theory plus the plausible assumption
that \emph{If S then S is true} is axiomatic. We can't infer from
\emph{A} is true and \emph{B} is false that \emph{A} is truer than
\emph{B}. But this looks like a reasonably plausible inference.

If we added this as inference rule, we would rule out all intermediate
sentences. To prove this assume, for \emph{reductio}, that \emph{A} is
intermediate. Since we keep classical logic, we know
\emph{A}~∨~¬\emph{A} is true. If \emph{A}, then \emph{A} is true, and
hence ¬\emph{A} is false. Then the new this inference rule implies
\emph{A}~⩾\textsubscript{T}~¬\emph{A}, hence
\emph{A}~∧~¬\emph{A}~⩾\textsubscript{T}~¬\emph{A}, since
¬\emph{A}~⩾\textsubscript{T}~¬\emph{A}, and hence
0=1~⩾\textsubscript{T}~¬\emph{A}, since
0=1~⩾\textsubscript{T}~\emph{A}~∧~¬\emph{A}, and ⩾\textsubscript{T} is
transitive. So \emph{A} is determinately true, not intermediate. A
converse proof shows that if ¬\emph{A}, then \emph{A} is determinately
false, not intermediate. So by (∨-Out) it follows that \emph{A} is not
intermediate, but since \emph{A} was arbitrary, there are no
intermediate truths. So this rule is unacceptable, despite its
plausibility.

\subsection*{Comparing Negative and Positive (due to Jonathan
Schaffer)}\label{comparing-negative-and-positive-due-to-jonathan-schaffer}
\addcontentsline{toc}{subsection}{Comparing Negative and Positive (due
to Jonathan Schaffer)}

Let \emph{a} be a regular borderline case of genius, somewhere near the
middle of the penumbra. Let \emph{b} be someone who is not a determinate
case of genius, but is very close. Let \emph{A} be \emph{a is a genius}
and \emph{B} be \emph{b is a genius}. It seems plausible that
\emph{A}~⩾\textsubscript{T}~¬\emph{B}, since \emph{a} is right around
the middle of the borderline cases of genius, but \emph{b} is only a
smidgen short of clear genius. But since \emph{b} is closer to being a
genius than \emph{a}, we definitely have
\emph{B}~⩾\textsubscript{T}~\emph{A}. By transitivity, it follows that
\emph{B}~⩾\textsubscript{T}~¬\emph{B}, hence \emph{B} is determinately
true (by the reasoning of the last paragraph). Since ¬\emph{B} is not
determinately false, it follows that \emph{B}~∧~¬\emph{B}~is not
determinately false, contradicting (A7).

Since I accept (A7) I must reject the initial assumption that
\emph{A}~⩾\textsubscript{T}~¬\emph{B}. But it's worth noting that this
case is quite general. Similar reasoning could be used to show that for
any indeterminate propositions of the form \emph{y} \emph{is a genius}
and \emph{y} \emph{is not a genius}, the first is not truer than the
second. This seems odd, since intuitively these could both be
indeterminate while the first is very nearly true and the second very
nearly false.

\subsection*{Comparing Different Predicates (due to Elizabeth
Harman)}\label{comparing-different-predicates-due-to-elizabeth-harman}
\addcontentsline{toc}{subsection}{Comparing Different Predicates (due to
Elizabeth Harman)}

One intuitive way to understand the behaviour of \emph{truer} is that
\emph{A} is truer than \emph{B} iff \emph{A} is true on every admissible
precisification on which \emph{B} is true and the converse does not
hold. This can't be an analysis of \emph{truer}, since it assumes we can
independently define what is an admissible precisification, and this
seems impossible. But it's a useful heuristic. And reflecting on it
brings up a surprising consequence of my theory. If we assume that
precisifications of predicates from different subject areas
(e.g.~\emph{hexagonal} and \emph{honest}) are independent, it follows
that subject-predicate sentences involving those predicates and
indeterminate instances of them are incomparable with respect to truth.
But this seems implausible. If France is a borderline case of being
hexagonal that is close to the lower bound, and George Washington is a
borderline case of being honest who is close to the upper bound, then we
should think \emph{George Washington is honest} is truer than
\emph{France is hexagonal}.

All three of these objections seem to me to turn on an underlying
intuition that \emph{truer} should be a linear relation. If we are given
this, then the inference principle Dorr suggests looks unimpeachable,
and the comparisons Schaffer and Harman suggested look right. But once
we drop the idea that truer is linear, I think the plausibility of these
claims falls away. So the arguments against linearity are ipso facto
arguments that we should simply drop the intuitions Dorr, Schaffer and
Harman are relying upon.

To conclude, it's worth noting that a very similar inferential rule to
the rule Dorr suggests is admissible. From the fact that \emph{A} is
determinately true, and \emph{B} is determinately false, it follows that
\emph{A} is truer than \emph{B}. If we assume, as seems reasonable, that
we're only in a position to \emph{say} that \emph{A} is true when it is
determinately true, then whenever we're in a position to say \emph{A} is
true and \emph{B} is false, it will be true that \emph{A} is truer than
\emph{B}. This line of defence is obviously similar to the explanation I
gave in the previous section of why Sorites premises look plausible, and
to the argument Rosanna Keefe gives that the failure of classical
inference rules is no difficulty for supervaluationism because it admits
very similar inference rules (\citeproc{ref-Keefe2000}{Keefe 2000}).

\subsection*{References}\label{references}
\addcontentsline{toc}{subsection}{References}

\phantomsection\label{refs}
\begin{CSLReferences}{1}{0}
\bibitem[\citeproctext]{ref-SiderBraun}
Braun, David, and Theodore Sider. 2007. {``Vague, so Untrue.''}
\emph{No{û}s} 41 (2): 133--56. doi:
\href{https://doi.org/10.1111/j.1468-0068.2007.00641.x}{10.1111/j.1468-0068.2007.00641.x}.

\bibitem[\citeproctext]{ref-Burgess1987}
Burgess, J. A., and I. L. Humberstone. 1987. {``Natural Deduction Rules
for a Logic of Vagueness.''} \emph{Erkenntnis} 27 (2): 197--229. doi:
\href{https://doi.org/10.1007/bf00175369}{10.1007/bf00175369}.

\bibitem[\citeproctext]{ref-Burgess2001}
Burgess, John. 2001. {``Vagueness, Epistemicism and
Response-Dependence.''} \emph{Australasian Journal of Philosophy} 79
(4): 507--24. doi:
\href{https://doi.org/10.1080/713659306}{10.1080/713659306}.

\bibitem[\citeproctext]{ref-Dummett1959}
Dummett, Michael. 1959. {``Truth.''} \emph{Proceedings of the
Aristotelian Society} 59 (1): 141--62. doi:
\href{https://doi.org/10.1093/aristotelian/59.1.141}{10.1093/aristotelian/59.1.141}.

\bibitem[\citeproctext]{ref-Eklund2002}
Eklund, Matti. 2002. {``Inconsistent Languages.''} \emph{Philosophy and
Phenomenological Research} 64 (2): 251--75. doi:
\href{https://doi.org/10.1111/j.1933-1592.2002.tb00001.x}{10.1111/j.1933-1592.2002.tb00001.x}.

\bibitem[\citeproctext]{ref-Fara2000}
Fara, Delia Graff. 2000. {``Shifting Sands: An Interest-Relative Theory
of Vagueness.''} \emph{Philosophical Topics} 28 (1): 45--81. doi:
\href{https://doi.org/10.5840/philtopics20002816}{10.5840/philtopics20002816}.
This paper was published under the name {``Delia Graff.''}

\bibitem[\citeproctext]{ref-Fine1975a}
Fine, Kit. 1975. {``Vagueness, Truth and Logic.''} \emph{Synthese} 30
(3-4): 265--300. doi:
\href{https://doi.org/10.1007/bf00485047}{10.1007/bf00485047}.

\bibitem[\citeproctext]{ref-Goldblatt1992}
Goldblatt, Robert. 1992. \emph{Logics of Time and Computation}. Palo
Alto: CSLI.

\bibitem[\citeproctext]{ref-Greenough2003}
Greenough, Patrick. 2003. {``Vagueness: A Minimal Theory.''} \emph{Mind}
112 (446): 235--81. doi:
\href{https://doi.org/10.1093/mind/112.446.235}{10.1093/mind/112.446.235}.

\bibitem[\citeproctext]{ref-Grice1989}
Grice, H. Paul. 1989. \emph{Studies in the Way of Words}. Cambridge,
MA.: Harvard University Press.

\bibitem[\citeproctext]{ref-Keefe2000}
Keefe, Rosanna. 2000. \emph{Theories of Vagueness}. Cambridge: Cambridge
University Press.

\bibitem[\citeproctext]{ref-Levinson2000}
Levinson, Stephen. 2000. \emph{Presumptive Meanings}. Cambridge, MA: MIT
Press.

\bibitem[\citeproctext]{ref-Lewis1972a}
Lewis, David. 1972. {``Psychophysical and Theoretical
Identifications.''} \emph{Australasian Journal of Philosophy} 50 (3):
249--58. doi:
\href{https://doi.org/10.1080/00048407212341301}{10.1080/00048407212341301}.
Reprinted in his \emph{Papers in Metaphysics and Epistemology},
Cambridge: Cambridge University Press, 1999, 248-261. References to
reprint.

\bibitem[\citeproctext]{ref-McGee1998}
McGee, Vann, and Brian McLaughlin. 1998. {``Review of Timothy
Williamson's \emph{Vagueness}.''} \emph{Linguistics and Philosophy} 21:
221--31.

\bibitem[\citeproctext]{ref-Restall2000}
Restall, Greg. 1999. \emph{An Introduction to Substructural Logics}.
London: Routledge.

\bibitem[\citeproctext]{ref-Sedivy1999}
Sedivy, Julie, Michael. Tanenhaus, Craig Chambers, and Gregory Carlson.
1999. {``Achieving Incremental Semantic Interpretation Through
Contextual Representation.''} \emph{Cognition} 71 (2): 109--47. doi:
\href{https://doi.org/10.1016/s0010-0277(99)00025-6}{10.1016/s0010-0277(99)00025-6}.

\bibitem[\citeproctext]{ref-Sider2001}
Sider, Theodore. 2001. {``Maximality and Intrinsic Properties.''}
\emph{Philosophy and Phenomenological Research} 63 (2): 357--64. doi:
\href{https://doi.org/10.1111/j.1933-1592.2001.tb00109.x}{10.1111/j.1933-1592.2001.tb00109.x}.

\bibitem[\citeproctext]{ref-Sorensen2001}
Sorensen, Roy. 2001. \emph{Vagueness and Contradiction}. Oxford: Oxford
University Press.

\bibitem[\citeproctext]{ref-Tappenden1993}
Tappenden, Jamie. 1993. {``The Liar and Sorites Paradoxes: Toward a
Unified Treatment.''} \emph{Journal of Philosophy} 90 (11): 551--77.
doi: \href{https://doi.org/10.2307/2940846}{10.2307/2940846}.

\bibitem[\citeproctext]{ref-Weatherson2003-WEAEPA}
Weatherson, Brian. 2003a. {``Epistemicism, Parasites, and Vague
Names.''} \emph{Australasian Journal of Philosophy} 81 (2): 276--79.
doi: \href{https://doi.org/10.1093/ajp/jag209}{10.1093/ajp/jag209}.

\bibitem[\citeproctext]{ref-Weatherson2003-Keefe}
---------. 2003b. {``Review of Rosanna Keefe, \emph{Theories of
Vagueness}.''} \emph{Philosophy and Phenomenological Research} 67:
491--94.

\bibitem[\citeproctext]{ref-Williamson1994-WILV}
Williamson, Timothy. 1994. \emph{{Vagueness}}. Routledge.

\bibitem[\citeproctext]{ref-Williamson1995}
---------. 1995. {``Definiteness and Knowability.''} \emph{Southern
Journal of Philosophy} 33 (Supp) (S1): 171--91. doi:
\href{https://doi.org/10.1111/j.2041-6962.1995.tb00769.x}{10.1111/j.2041-6962.1995.tb00769.x}.

\bibitem[\citeproctext]{ref-Williamson2004-WILRTM}
---------. 2004. {``{Reply to McGee and McLaughlin}.''}
\emph{Linguistics and Philosophy} 27 (1): 113--22. doi:
\href{https://doi.org/10.1023/b:ling.0000010847.78827.d0}{10.1023/b:ling.0000010847.78827.d0}.

\end{CSLReferences}



\noindent Published in\emph{
Philosophical Studies}, 2005, pp. 47-70.

\end{document}
