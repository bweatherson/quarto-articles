% Options for packages loaded elsewhere
% Options for packages loaded elsewhere
\PassOptionsToPackage{unicode}{hyperref}
\PassOptionsToPackage{hyphens}{url}
%
\documentclass[
  11pt,
  letterpaper,
  DIV=11,
  numbers=noendperiod,
  twoside]{scrartcl}
\usepackage{xcolor}
\usepackage[left=1.1in, right=1in, top=0.8in, bottom=0.8in,
paperheight=9.5in, paperwidth=7in, includemp=TRUE, marginparwidth=0in,
marginparsep=0in]{geometry}
\usepackage{amsmath,amssymb}
\setcounter{secnumdepth}{3}
\usepackage{iftex}
\ifPDFTeX
  \usepackage[T1]{fontenc}
  \usepackage[utf8]{inputenc}
  \usepackage{textcomp} % provide euro and other symbols
\else % if luatex or xetex
  \usepackage{unicode-math} % this also loads fontspec
  \defaultfontfeatures{Scale=MatchLowercase}
  \defaultfontfeatures[\rmfamily]{Ligatures=TeX,Scale=1}
\fi
\usepackage{lmodern}
\ifPDFTeX\else
  % xetex/luatex font selection
  \setmainfont[ItalicFont=EB Garamond Italic,BoldFont=EB Garamond
SemiBold]{EB Garamond Math}
  \setsansfont[]{EB Garamond}
  \setmathfont[]{Garamond-Math}
\fi
% Use upquote if available, for straight quotes in verbatim environments
\IfFileExists{upquote.sty}{\usepackage{upquote}}{}
\IfFileExists{microtype.sty}{% use microtype if available
  \usepackage[]{microtype}
  \UseMicrotypeSet[protrusion]{basicmath} % disable protrusion for tt fonts
}{}
\usepackage{setspace}
% Make \paragraph and \subparagraph free-standing
\makeatletter
\ifx\paragraph\undefined\else
  \let\oldparagraph\paragraph
  \renewcommand{\paragraph}{
    \@ifstar
      \xxxParagraphStar
      \xxxParagraphNoStar
  }
  \newcommand{\xxxParagraphStar}[1]{\oldparagraph*{#1}\mbox{}}
  \newcommand{\xxxParagraphNoStar}[1]{\oldparagraph{#1}\mbox{}}
\fi
\ifx\subparagraph\undefined\else
  \let\oldsubparagraph\subparagraph
  \renewcommand{\subparagraph}{
    \@ifstar
      \xxxSubParagraphStar
      \xxxSubParagraphNoStar
  }
  \newcommand{\xxxSubParagraphStar}[1]{\oldsubparagraph*{#1}\mbox{}}
  \newcommand{\xxxSubParagraphNoStar}[1]{\oldsubparagraph{#1}\mbox{}}
\fi
\makeatother


\usepackage{longtable,booktabs,array}
\usepackage{calc} % for calculating minipage widths
% Correct order of tables after \paragraph or \subparagraph
\usepackage{etoolbox}
\makeatletter
\patchcmd\longtable{\par}{\if@noskipsec\mbox{}\fi\par}{}{}
\makeatother
% Allow footnotes in longtable head/foot
\IfFileExists{footnotehyper.sty}{\usepackage{footnotehyper}}{\usepackage{footnote}}
\makesavenoteenv{longtable}
\usepackage{graphicx}
\makeatletter
\newsavebox\pandoc@box
\newcommand*\pandocbounded[1]{% scales image to fit in text height/width
  \sbox\pandoc@box{#1}%
  \Gscale@div\@tempa{\textheight}{\dimexpr\ht\pandoc@box+\dp\pandoc@box\relax}%
  \Gscale@div\@tempb{\linewidth}{\wd\pandoc@box}%
  \ifdim\@tempb\p@<\@tempa\p@\let\@tempa\@tempb\fi% select the smaller of both
  \ifdim\@tempa\p@<\p@\scalebox{\@tempa}{\usebox\pandoc@box}%
  \else\usebox{\pandoc@box}%
  \fi%
}
% Set default figure placement to htbp
\def\fps@figure{htbp}
\makeatother


% definitions for citeproc citations
\NewDocumentCommand\citeproctext{}{}
\NewDocumentCommand\citeproc{mm}{%
  \begingroup\def\citeproctext{#2}\cite{#1}\endgroup}
\makeatletter
 % allow citations to break across lines
 \let\@cite@ofmt\@firstofone
 % avoid brackets around text for \cite:
 \def\@biblabel#1{}
 \def\@cite#1#2{{#1\if@tempswa , #2\fi}}
\makeatother
\newlength{\cslhangindent}
\setlength{\cslhangindent}{1.5em}
\newlength{\csllabelwidth}
\setlength{\csllabelwidth}{3em}
\newenvironment{CSLReferences}[2] % #1 hanging-indent, #2 entry-spacing
 {\begin{list}{}{%
  \setlength{\itemindent}{0pt}
  \setlength{\leftmargin}{0pt}
  \setlength{\parsep}{0pt}
  % turn on hanging indent if param 1 is 1
  \ifodd #1
   \setlength{\leftmargin}{\cslhangindent}
   \setlength{\itemindent}{-1\cslhangindent}
  \fi
  % set entry spacing
  \setlength{\itemsep}{#2\baselineskip}}}
 {\end{list}}
\usepackage{calc}
\newcommand{\CSLBlock}[1]{\hfill\break\parbox[t]{\linewidth}{\strut\ignorespaces#1\strut}}
\newcommand{\CSLLeftMargin}[1]{\parbox[t]{\csllabelwidth}{\strut#1\strut}}
\newcommand{\CSLRightInline}[1]{\parbox[t]{\linewidth - \csllabelwidth}{\strut#1\strut}}
\newcommand{\CSLIndent}[1]{\hspace{\cslhangindent}#1}



\setlength{\emergencystretch}{3em} % prevent overfull lines

\providecommand{\tightlist}{%
  \setlength{\itemsep}{0pt}\setlength{\parskip}{0pt}}



 


\setlength\heavyrulewidth{0ex}
\setlength\lightrulewidth{0ex}
\usepackage[automark]{scrlayer-scrpage}
\clearpairofpagestyles
\cehead{
  Brian Weatherson
  }
\cohead{
  Attitudes and Relativism
  }
\ohead{\bfseries \pagemark}
\cfoot{}
\makeatletter
\newcommand*\NoIndentAfterEnv[1]{%
  \AfterEndEnvironment{#1}{\par\@afterindentfalse\@afterheading}}
\makeatother
\NoIndentAfterEnv{itemize}
\NoIndentAfterEnv{enumerate}
\NoIndentAfterEnv{description}
\NoIndentAfterEnv{quote}
\NoIndentAfterEnv{equation}
\NoIndentAfterEnv{longtable}
\NoIndentAfterEnv{abstract}
\renewenvironment{abstract}
 {\vspace{-1.25cm}
 \quotation\small\noindent\emph{Abstract}:}
 {\endquotation}
\newfontfamily\tfont{EB Garamond}
\addtokomafont{disposition}{\rmfamily}
\addtokomafont{title}{\normalfont\itshape}
\let\footnoterule\relax

\makeatletter
\renewcommand{\@maketitle}{%
  \newpage
  \null
  \vskip 2em%
  \begin{center}%
  \let \footnote \thanks
    {\itshape\huge\@title \par}%
    \vskip 0.5em%  % Reduced from default
    {\large
      \lineskip 0.3em%  % Reduced from default 0.5em
      \begin{tabular}[t]{c}%
        \@author
      \end{tabular}\par}%
    \vskip 0.5em%  % Reduced from default
    {\large \@date}%
  \end{center}%
  \par
  }
\makeatother
\RequirePackage{lettrine}

\renewenvironment{abstract}
 {\quotation\small\noindent\emph{Abstract}:}
 {\endquotation\vspace{-0.02cm}}
\KOMAoption{captions}{tableheading}
\makeatletter
\@ifpackageloaded{caption}{}{\usepackage{caption}}
\AtBeginDocument{%
\ifdefined\contentsname
  \renewcommand*\contentsname{Table of contents}
\else
  \newcommand\contentsname{Table of contents}
\fi
\ifdefined\listfigurename
  \renewcommand*\listfigurename{List of Figures}
\else
  \newcommand\listfigurename{List of Figures}
\fi
\ifdefined\listtablename
  \renewcommand*\listtablename{List of Tables}
\else
  \newcommand\listtablename{List of Tables}
\fi
\ifdefined\figurename
  \renewcommand*\figurename{Figure}
\else
  \newcommand\figurename{Figure}
\fi
\ifdefined\tablename
  \renewcommand*\tablename{Table}
\else
  \newcommand\tablename{Table}
\fi
}
\@ifpackageloaded{float}{}{\usepackage{float}}
\floatstyle{ruled}
\@ifundefined{c@chapter}{\newfloat{codelisting}{h}{lop}}{\newfloat{codelisting}{h}{lop}[chapter]}
\floatname{codelisting}{Listing}
\newcommand*\listoflistings{\listof{codelisting}{List of Listings}}
\makeatother
\makeatletter
\makeatother
\makeatletter
\@ifpackageloaded{caption}{}{\usepackage{caption}}
\@ifpackageloaded{subcaption}{}{\usepackage{subcaption}}
\makeatother
\usepackage{bookmark}
\IfFileExists{xurl.sty}{\usepackage{xurl}}{} % add URL line breaks if available
\urlstyle{same}
\hypersetup{
  pdftitle={Attitudes and Relativism},
  pdfauthor={Brian Weatherson},
  hidelinks,
  pdfcreator={LaTeX via pandoc}}


\title{Attitudes and Relativism}
\author{Brian Weatherson}
\date{2008}
\begin{document}
\maketitle
\begin{abstract}
Data about attitude reports provide some of the most interesting
arguments for, and against, various theses of semantic relativism. This
paper is a short survey of three such arguments. First, I'll argue
(against recent work by von Fintel and Gillies) that relativists can
explain the behaviour of relativistic terms in factive attitude reports.
Second, I'll argue (against Glanzberg) that looking at attitude reports
suggests that relativists have a \emph{more} plausible story to tell
than contextualists about the division of labour between semantics and
meta-semantics. Finally, I'll offer a new argument for invariantism
(i.e.~against both relativism and contextualism) about moral terms. The
argument will turn on the observation that the behaviour of normative
terms in factive and non-factive attitude reports is quite unlike the
behaviour of any other plausibly context-sensitive term.
\end{abstract}


\setstretch{1.1}
Data about attitude reports provide some of the most interesting
arguments for, and against, various theses of semantic relativism. This
paper is a short survey of three such arguments. First, I'll argue
(against recent work by von Fintel and Gillies) that relativists can
explain the behaviour of relativistic terms in factive attitude reports.
Second, I'll argue (against Glanzberg) that looking at attitude reports
suggests that relativists have a \emph{more} plausible story to tell
than contextualists about the division of labour between semantics and
meta-semantics. Finally, I'll offer a new argument for invariantism
(i.e.~against both relativism and contextualism) about moral terms. The
argument will turn on the observation that the behaviour of normative
terms in factive and non-factive attitude reports is quite unlike the
behaviour of any other plausibly context-sensitive term. Before that,
I'll start with some taxonomy, just so as it's clear what the intended
conclusions below are supposed to be.

\section{How Not to be a Strawman}\label{how-not-to-be-a-strawman}

Here are three mappings that we, as theorists about language, might be
interested in.

\begin{itemize}
\item
  \textbf{Physical Movements} → \textbf{Speech Acts}. I'm currently
  moving my fingers across a keyboard. In doing so, I'm making various
  assertions. It's a hard question to say just how I manage to make an
  assertion by moving my fingers in this way. Relatedly, it's a hard
  question to say just which assertions, requests, questions, commands
  etc people make by making physical movements. This mapping is the
  answer to that question.
\item
  \textbf{Speech Acts} → \textbf{Contents}. For some speech acts, their
  content is clear. If I assert that grass is green, the content of my
  assertion is that grass is green. For other speech acts, including
  perhaps other assertions, it is not immediately obvious what their
  content is. This mapping answers that question.
\item
  \textbf{Contents} → \textbf{Truth Values}. Given a content, it isn't
  always clear whether it is true or false (or perhaps something else).
  This mapping provides the truth values of every content of every
  speech act.
\end{itemize}

The details of each of these mappings is interesting, to say the least.
Indeed, a full description of those mappings would arguably include an
answer to every question ever asked. We're not likely to know that any
time soon. But we can ask, and perhaps answer, interesting questions
about the topology of the mappings. Here are two distinct questions we
can ask about each of the three mappings.

\begin{itemize}
\item
  \textbf{Is it one-one or one-many?} The mapping from a natural number
  to its natural predecessor is one-one. That's to say, every number is
  mapped to at most one other number. The mapping from a person to the
  children they have is one-many. That's not to say that everyone has
  many children, or indeed that everyone has any children. It's merely
  to say that some things in the domain are mapped to many things in the
  range.
\item
  \textbf{Is the mapping absolute or relative, and if relative, relative
  to what?} Consider a mapping that takes a person A as input, and
  relative to any person B, outputs the first child that A has with B.
  This mapping is relative; there's no such thing as the output of the
  function for any particular input A. All there is, is the output
  relative to B1, relative to B2, and so on. But note that in a good
  sense the mapping is one-one. Relative to any B, A is mapped to at
  most one person.
\end{itemize}

A very simple model of context-sensitivity in language says that all
three of these mappings are one-one and absolute. It will be helpful to
have a character who accepts that, so imagine we have a person, called
Strawman, who does. There are six basic ways to reject Strawman's views.
For each of the three mappings, we can say that it is one-many or we can
say that it is relative.

This way of thinking about Strawman's opponents gives us a nice
taxonomy. The first mapping is about \emph{speech acts}, the second
about \emph{contents} and the third about \emph{truth}. Someone who
disagrees with Strawman disagrees about one of these three mappings.
They might disagree by being a \emph{pluralist}, i.e.~thinking that the
mapping is one-many. Or they might disagree by being a
\emph{relativist}, i.e.~thinking the mapping is relative to some thing
external. The respective defenders of Strawman on these questions are
the \emph{monists} and the \emph{absolutists}.

So the \emph{speech act pluralist} is the person who thinks the first
mapping is one-many. The \emph{content relativist}, is the person who
thinks the second is relative to some other variable (perhaps an
assessor). And the \emph{truth monist} is the person who thinks that
contents have (at most) one truth value. All these terms are a little
bit stipulative, but I think it the terminology here somewhat matches up
with regular use. And it's the terminology I'll use throughout this
paper.

One other nice advantage of this taxonomy is that it helps clarify just
what is at issue between various opponents of Strawman. So Andy Egan
(\citeproc{ref-Egan2009-EGABBA}{2009}) has some data about uses of
``you'' in group settings that suggest such utterances pose a problem
for Strawman. But it's one thing to have evidence that Strawman is
wrong, another altogether to know which of his views is, on a particular
occasion, wrong. I think separating out Strawman's various commitments
helps clarify what is needed to isolate Strawman's mistake on an
occasion.

It is, I think, more or less common ground that the first of Strawman's
commitments, speech act monism, is false. The King can, by uttering
``It's cold in here'', both assert that it's cold in here, and command
his lackey to close the window. Those look like two distinct speech acts
that he's made with the one physical movement. Herman Cappellen and
Ernest Lepore (\citeproc{ref-Cappelen2005}{Cappelen and Lepore 2005})
have many more examples to show that Strawman is wrong here. Once we go
beyond that though, it's less clear that Strawman is mistaken. Perhaps
by thinking about cases where, by the one physical movement, we intend
to communicate \emph{p} to one audience member, and \emph{q} to another,
we can try to motivate speech-act relativism. That's an issue I'll leave
for another day. In contrast to what he says about speech acts, what
Strawman says about content and truth is, if not universally accepted,
at least popular. So I'll call \emph{orthodox contextualism} the view
that Strawman is right about the content mapping and the truth mapping;
each mapping is both one-one and absolute.

It is worthwhile noting two very separate models for content that lead
to two quite distinct ways in which we can reject Strawman's last two
absolutist views. John MacFarlane's paper on Non-Indexical Contextualism
MacFarlane (\citeproc{ref-MacFarlane2009-MACNC}{2009}) was extremely
useful in setting up the relevant distinctions here, but the particular
models for content I'm describing here are both set out in greatest
detail in recent work by Andy Egan.

The first is the centred worlds model for content. This is the idea that
for some utterance types, any token of that type expresses the same
content. But that content is a set of centred worlds, that is true at
some centres and false at other centres in the same world. So we might
think that the content of ``Beer is tasty'' is, roughly, the set of
possibilia who have pro-attitudes to the taste of beer. More precisely,
it is the set of world-centre pairs such that the agent at (or perhaps
closest to) the centre has pro-attitudes towards the taste of beer. On
this view, content monism will be maintained -- what an utterance of
``Beer is tasty'' says is invariant across assessors. (I'm assuming here
that assessor-relativity is the only relativity we're interested in.)
But truth absolutism will fail, since whether that content is true for
\emph{a}\textsubscript{1} and \emph{a}\textsubscript{2} will depend on
what their attitudes are towards beer. This kind of centred worlds model
for content is what Egan has developed in
(\citeproc{ref-Egan2007-EGAEMR}{Egan 2007}).

The second model lets assessors get into the content-fixing mechanism,
but says the content that is fixed is a familiar proposition whose truth
is not assessor relative. This is easiest to explain with an example
involving second-person pronouns. For some utterances of ``You are a
fool'', the content of that utterance, relative to \emph{x}, is that
\emph{x} is a fool. Now whether \emph{x} is indeed a fool is a simple
factual question, and whether it is true isn't assessor relative. But if
some people are fools and others are not, whether the utterance is true
or false depends on who is assessing it. So content relativism is true,
while truth absolutism is preserved. This is a view Egan has defended
for some tokens of second person pronouns
(\citeproc{ref-Egan2009-EGABBA}{Egan 2009}) .

In ``Conditionals and Indexical Relativism''
(\citeproc{ref-Weatherson2009-WEACAI}{Weatherson 2009}), I called the
combination of content relativism and truth absolutism ``indexical
relativism'', and defended such a view about indicative conditionals. I
called something similar to the combination of truth relativism and
content absolutism ``non-indexical contextualism''. More precisely, I
followed MacFarlane in using that phrase for the combination of truth
relativism and content absolutism and the view that whether a speaker's
utterance is true (relative to an assessor) is a matter of whether the
proposition they express is true relative to their context. I like the
name ``indexical relativism'', but it has also been used for theories
that aren't even heterodox in the above sense, so perhaps persisting
with it would just invite confusion. (And the name implies a particular
view about how the relativity works; namely that there is something like
an indexical element in what's asserted that gets its value from the
context of assessment.) In other contexts I've used ``relativism'' as
the label for all and only heterodox views, but this label is
potentially quite confusing. Indeed, it's a possible worry about the
arguments in my ``Conditionals and Indexical Relativism'' that they
really just support heterodoxy; a separate argument would be needed (and
might not be easy to supply) against pluralist alternatives to content
relativism about indicative conditionals.

\section{Factive Verbs and
Relativism}\label{factive-verbs-and-relativism}

In ``CIA Leaks'' (\citeproc{ref-vonFintel2008}{Fintel and Gillies
2008}), Kai von Fintel and Thony Gillies raise a problem for heterodox
theories about `might'. (Actually they raise several, but I'm only going
to deal with one of them here.) Their primary target is what I called
truth relativist theories, but the argument they raise is interesting to
consider from all heterodox perspectives. The problem concerns embedding
of `might'-clauses under factive attitude verbs. They argue as follows:

\begin{enumerate}
\def\labelenumi{\arabic{enumi}.}
\tightlist
\item
  \emph{S realises that p} presupposes that \emph{p}.
\item
  This presupposition is carried over when the sentence is used as the
  antecedent of a conditional. So, for instance, \emph{If S realises
  that p, then q} presupposes that \emph{p}.
\item
  But, on standard heterodox proposals, we can properly say \emph{If S
  realises that it might be that p, then q}, even though it isn't true
  that it might be that \emph{p}.
\item
  So heterodox proposals are false.
\end{enumerate}

Here is the example they use to make the case.

\begin{quote}
Bond planted a bug and some misleading evidence pointing to his being in
Zuurich and slipped out. Now he and Leiter are listening in from London.
As they listen, Leiter is getting a bit worried: Blofeld hasn't yet
found the misleading evidence that points to Bond's being in Zurich.
Leiter turns to Bond and says:

\begin{enumerate}
\def\labelenumi{\arabic{enumi}.}
\tightlist
\item
  If Blofeld realizes you might be in Zurich, you can breathe
  easy---he'll send his henchman to Zurich to find you. (93)
\end{enumerate}

Now the problem is that for the heterodox theorist, ``You might be in
Zurich'', as uttered by Leiter to Bond, expresses (relative to Bond), a
proposition that is true iff for all Bond knows, Bond might be in
Zurich. Just how it does this will differ for different heterodox
theorists, but so far they all agree. But that isn't the case; since
Bond knows he is in London. So (34) should sound defective, since it
contains a presupposition failure. But it isn't defective, so heterodoxy
is mistaken.
\end{quote}

Before we look at how heterodox theorists might respond to this case,
it's worth looking thinking about how Strawman might respond to it. The
simplest idea is to say that in \emph{It might be that p}, there is a
hidden variable \emph{X}. The value of \emph{X} is set by context. And
the sentence expresses the proposition that for all \emph{X} knows,
\emph{p} is true. (Perhaps we might use some epistemic relation other
than `knows', but that's not relevant here.)

Now, and this is crucial, the variable \emph{X} might be either free or
bound. If there is nothing around to bind it, as in a simple utterance
of \emph{It might be that p}, then it will be free. And typically if it
is free, \emph{X} denotes a group consisting of the speaker and perhaps
those in the same conversation. But when the might-clause is embedded
under a propositional attitude ascription (factive or not), the variable
\emph{X} will be bound to the subject of the attitude ascription. So in
\emph{Y believes that it might be that p}, the value of \emph{X} will
simply be \emph{Y}. So in \emph{Blofeld realises you might be in
Zurich}, the value of \emph{X} is Blofeld. And hence the embedded might
claim is true, since that claim is simply that for all Blofeld knows,
Bond is in Zurich. Which, in the story, is true.

The reason for going through all of this is that the theorist who
accepts truth absolutism, but rejects content absolutism, can say
exactly the same thing. There is a variable \emph{X} in the structure of
what's asserted. Strawman thinks that you only get a determinate
assertion when you fill in the value of \emph{X}. We can disagree with
that; we can say that an assertion can literally contain a variable, one
that potentially gets its value from assessors. That way the content of
a particular assertion can be different for different assessors. Once
we've made this move, we can rejoin Strawman's story. This variable is
either free or bound. If it is bound, we tell exactly the same story as
Strawman. But, we insist, if it is free, the value of \emph{X} is
sometimes set by contextual features of the hearer as well as of the
assessor. In the standard case, \emph{X} is a group consisting of the
speaker, the hearer, and perhaps some people who get in the group in
virtue of their proximity to the speaker or hearer.

So we end up saying the same thing about the acceptability of (34) as
Strawman. The content of \emph{You might be in Zurich,} as embedded in
(34), is quite different to the content those words would have if
uttered as a standalone sentence, because the value that a key variable
takes is different. For us, the value that variable takes differs for
different assessors, but that's completely irrelevant to the explanation
of the acceptability of (34).

For the truth relativist (who is also a content absolutist) things are a
little more interesting. Such a theorist will typically reject the
presence of a variable like X in the structure of what is said. So they
cannot appeal to the kind of explanation that we've offered (twice over)
for how (34) may be acceptable. The solution is to simply reject the
generalisation about factive verbs. Let's start with some seemingly
distant examples, in particular examples about fiction. It seems that
(5) doesn't have any false presuppositions.

\begin{enumerate}
\def\labelenumi{\arabic{enumi}.}
\setcounter{enumi}{4}
\tightlist
\item
  Watson realised that private detectives were (in late 19th Century
  London) better at solving murder mysteries than police.
\item
  Had Watson realised earlier that private detectives were better at
  solving murder mysteries than police, he would have liked Holmes more
  than he did.
\item
  Had Watson realised earlier that private detectives were better at
  solving murder mysteries than police, the early chapters of the book
  would have been more interesting.
\end{enumerate}

Note that this isn't anything particular to do with subjunctive
conditionals. Imagine that we are settling down to watch a new
adaptation of the Holmes stories that we are told won't be particularly
faithful to the books in detail. I might properly say (8).

\begin{enumerate}
\def\labelenumi{\arabic{enumi}.}
\setcounter{enumi}{7}
\tightlist
\item
  If Watson realises that private detectives were better at solving
  murder mysteries than police, the early scenes will be more
  interesting.
\end{enumerate}

But now it is easy to see the way out of the argument for the proponent
of the centred world view. The crucial thing about Watson isn't, such a
theorist will say, that he's in another possible world. The crucial
thing is that some propositions that are false relative to us (e.g.~the
proposition that private detectives were better at solving murder
mysteries than police) are true relative to him. The true generalisation
seems to be that S Vs that p, where V is factive, presupposes that p is
true relative to S. And that's true in the cases that von Fintel and
Gillies describe. So it's not true that the centred world theorist
should predict that these utterances have false presuppositions. And
that's all to the good, because of course they don't.

\section{Glanzberg on Metasemantics}\label{glanzberg-on-metasemantics}

Papers arguing for relativism about some term \emph{t} frequently,
perhaps typically, start with a survey of reasons why (orthodox)
contextualism about \emph{t} cannot be correct. And such a survey will
frequently include a sojourn through some quite specific contextualist
theories, with some fairly obvious counterexamples. Egan, Hawthorne, and
Weatherson (\citeproc{ref-Egan2005-EGAEMI}{2005}) sticks to the script
as far as this goes. We note that \emph{a} \emph{might be F} can't, for
instance, mean that \emph{For all S knows, a is F}, where \emph{S} is
the speaker. And we note that some other simple theories along the same
lines can't be true.

It's interesting to think through what kind of force such tours through
the possible contextualist theories could have. We might think that
there's a tacit argument that if some contextualist theory were true, it
would be one of these simple ones, and none of the simple ones is true,
so no contextualist theory is true. I'm not going to take a stand on
exegetical debates here, so I'm not going to consider who may or may not
have been making such an argument. I don't think that it was the
intended argument in Egan, Hawthorne, and Weatherson
(\citeproc{ref-Egan2005-EGAEMI}{2005}), but that's beside the point,
because it is an argument that's now being debated. The argument under
consideration isn't, or at least isn't directly, that the
contextualist's \emph{semantic} proposal is mistaken. Rather, the
argument is that the accompanying \emph{metasemantic} theory, i.e.~the
theory of how semantic values get fixed, is intolerably complicated.
Slightly more formally, we can argue as follows.

\begin{enumerate}
\def\labelenumi{\arabic{enumi}.}
\tightlist
\item
  If contextualism is true, the metasemantic theory of how a particular
  use of ``might'' gets its semantic value is hideously complicated.
\item
  Metasemantic theories about how context-sensitive terms get their
  values on particular occasions are never hideously complicated.
\item
  So, contextualism is false.
\end{enumerate}

The problem with this argument, as Glanzberg
(\citeproc{ref-Glanzberg2007}{2007}) has argued, is that premise 2 seems
to be false. If we just look at some so-called `automatic' indexicals,
like ``I'' or ``here'' or ``now'', it just may be plausible. (But even
in those cases things are tricky when we look at recordings, as in
Weatherson (\citeproc{ref-Weatherson2002-WEAMI}{2002}).) Once we widen
our gaze though, we see that there are examples of uncontroversially
context-sensitive terms, like ``that'', for which the accompanying
metasemantic theory is, by any standard, hideously complicated. So the
prospects of getting to relativism from metasemantic complexity are not,
I think, promising.

That isn't the only kind of metasemantic argument for relativism though.
A better argument for relativism turns on the fact that metasemantics
\emph{is} generally complicated. The contextualist, I think, has to make
metasemantics too systematic at a key point. (Again, I'm not doing
exegesis, but I do think something like this argument was intended in
Egan et al (2005). I'm largely here highlighting something that I think
has been thus far overlooked in the commentaries on that piece.)
Consider the following pair of sentences.

\begin{enumerate}
\def\labelenumi{\arabic{enumi}.}
\setcounter{enumi}{8}
\tightlist
\item
  Those guys are in trouble, but they don't know that they are.
\item
  ??Those guys are in trouble, but they might not be.
\end{enumerate}

Something has gone wrong in (10). I conclude that (10) can't be used to
express (9). That is, there's no good interpretation of (10) where those
guys can be the denotation of \emph{X} in the theory I attributed to
Strawman in the previous last section. That's to say, the value of
\emph{X} can't just be the most salient individual(s) in the context,
since the guys are being made rather salient. And nor can it be
anaphoric on previously mentioned people, unless they are the subjects
of a propositional attitude ascription. We'll investigate this exception
in what follows.

A natural move at this stage is to adopt what in Egan, Hawthorne, and
Weatherson (\citeproc{ref-Egan2005-EGAEMI}{2005}) we called the
\emph{Speaker Inclusion Constraint} (hereafter SIC). That is, in
unembedded uses of ``might'' the group \emph{X} always includes the
speaker. Now the explanation of the problem with (10) is that for the
speaker to assert the first clause, she must know that the guys are in
trouble, but if that's the case, and she's in group \emph{X}, then the
second clause is false.

If the SIC really holds, it looks like it should hold in virtue of the
meaning (in some sense of ``meaning'') of ``might''. As a rule, tidy
generalisations like this should be part of semantics, not
metasemantics. Compare two possible theories about ``we''. Both theories
say that ``we'' is a plural pronoun. One theory goes on to say that it
is part of the meaning of ``we'' that it picks out a group that includes
the speaker. That is, it puts this version of the SIC into the
semantics. Another theory says that the SIC for ``we'' is just a nice
metasemantic generalisation. I take it that the second position is very
unattractive; it's part of the meaning of ``we'' that it picks out a
group including the speaker. And I think the relevant point generalises.
At least it generalises as far as another SIC, namely the one that holds
for ``might''.

Semantic constraints on indexical terms hold, as a rule, for both
embedded and unembedded uses of those indexicals. You can't use ``she'',
even as a bound variable, to denote (relative to any variable
assignment) a male human. There's something badly wrong with \emph{Every
student thinks she will win}, if some students are female and others
male. As Michael Glanzberg pointed out to me, complex demonstratives
headed by ``this'' have to pick out an object that is in some way
proximal, and this applies to bound complex demonstratives as well. So
we have to use ``that'' rather than ``this'' in sentences like
\emph{Every connoisseur remembers that/*this first great wine they
drank}. For a more familiar example, you can't interpret \emph{O'Leary
thinks that I am a fool}, as uttered by Daniels, to mean that O'Leary
self-ascribes foolishness. In short, it seems that there are three
related conclusions we can draw here. First, there are semantic
constraints on the possible values of context-sensitive expressions.
Second, any interesting generalisation about the possible value of a
context-sensitive expression is traceable to such a semantic constraint.
Third, these constraints remain in force when the expression is used in
an attitude report, or as a bound variable.

The problem for contextualists about ``might'' is that it doesn't behave
at all this way. The SIC holds for unembedded uses. That implies that it
is part of the meaning of ``might''. So it should hold for embedded
uses. But it does not. Indeed, for many embedded uses of ``might'', a
reading compatible with the SIC is simply unavailable. For instance, we
can't interpret \emph{Every student thinks that they might have failed}
as meaning that every student thinks that, for all I know, they failed.
My knowledge just doesn't matter; we're talking about those students'
fears. This all suggests the following argument.

\begin{enumerate}
\def\labelenumi{\arabic{enumi}.}
\tightlist
\item
  If contextualism is true, then the explanation of the SIC is that it
  is part of the meaning of ``might'' that the relevant group X includes
  the speaker.
\item
  If it is part of the meaning of ``might'' that the relevant group X
  includes the speaker, then this must be true for all uses of
  ``might'', included embedded uses.
\item
  When ``might'' is used inside the scope of an attitude ascription, the
  relevant group need not include the speaker.
\item
  So, contextualism is not true.
\end{enumerate}

Glanzberg argued, correctly, that it's no problem for the contextualist
if, according to their theory, metasemantics was complicated and messy.
It's not a problem because, well, metasemantics is complicated and
messy. But this cuts both ways. And it is a problem for the
contextualist that they have to put the SIC into metasemantics. It's
just not messy enough to go there.

There are two objections to this argument (both of which were pressed
when this paper was presented at Arché) that are worth considering
together.

\begin{quote}
\emph{Objection One: There are other generalisations that do go into
metasemantics}

It's very odd, to say the least, to use third-person pronouns to denote
oneself. But this doesn't seem to go into the meaning of the pronoun.
Relatedly, it is possible to use bound third-person pronouns that take
(relative to some variable assignments) oneself as value. For instance,
an Australian boy can say ``Whenever an Australian boy goes to the
cricket, he cheers for Australia.'' So probably premise 1 of the above
argument, requiring that generalisations be semantic, is false. If not,
premise 2, requiring that semantic constraints on unbound pronouns also
constrain bound pronouns is false.
\end{quote}

\begin{quote}
\emph{Objection Two: The SIC is false}

Egan, Hawthorne, and Weatherson (\citeproc{ref-Egan2005-EGAEMI}{2005})
note that the SIC seems to fail in `game-playing' and `advice' contexts.
So in a game of hide and seek, where Billy is looking for something Suzy
has hidden, if he asks ``Is it in the basement?'', Suzy can truly say
``It might be'', even if she knows it isn't true. And a parent can tell
their child ``Wash that strawberry before you eat it; it might be
contaminated'' even if the parent knows that the strawberry has been
washed.
\end{quote}

The simplest response to the first objection is that the purported
generalisation, a third-person pronoun does not denote the speaker,
isn't really a universal generalisation at all. It's possible to refer
to oneself by name; certain people in the news do it on a regular basis.
For example, a famous footballer Smith might say ``Smith deserves a pay
raise''. In such a context, it isn't at all odd (or at least any odder
than it already is) to use third-person pronouns, e.g.~by continuing the
above utterance with ``and if he doesn't get one, he's not going to
play''.

The second objection is a little trickier, but I think it's possible to
understand these utterances as a kind of projection. The speaker is
speaking from the perspective of the hearer. This isn't an unattested
phenomenon. Something like it must be going on when speakers use ``we''
to denote the hearer. Imagine, for instance, a policeman coming across a
person staggering out of a pub and saying ``We've had a bit much to
drink it seems''. The policeman certainly isn't confessing to
dereliction of duty. Nor is this case sufficient to throw out the idea
that ``we'' is a \emph{first person} plural pronoun. Rather, the
policeman is speaking from the drunk's perspective. I suspect the same
thing is going on in both of the examples above.

So I think both objections can be answered. But neither answer is
completely convincing. And the two responses undermine each other. If we
accept projection is a widespread phenomenon, then perhaps
self-denotation with a third person pronoun is a kind of projection. We
should then restate the argument, without assuming there's a response to
this pair of objections.

To do so, let's step back from the details of the SIC. What we started
with was a simple fact, that (10) can't be used to express (9). That's
not threatened by counterexamples to the SIC, and it still needs
explanation. The SIC is a proposed semantic explanation, and perhaps, if
it has counterexamples, it fails. I suspect something like it is
correct, but I don't plan to rely on that suspicion here. That's because
we can be independently confident that the explanation here will be
semantic, not pragmatic. We can be confident of this because there just
doesn't look to be anything like a pragmatic explanation available.

Compare the discussion of third-person pronouns. Even if we can't use
third-person pronouns to pick out ourselves (when not speaking
projectively), there is an obvious pragmatic explanation for this. We
have first-person pronouns available, and if we mean to denote
ourselves, using a first-person pronoun will do so in the clearest
possible way. Since it is good to be clear, when we pass up the chance
to use a first-person pronoun, the obvious conclusion for a hearer to
draw is that we don't mean to denote ourselves. The details of this
explanation could use some filling out, but it at least has the form of
an explanation. It simply doesn't seem that any such explanation will be
available for why (10) can't be used to express (9). It's not that (9)
isn't a thought that we might be interested in expressing. And it's not
that if we wanted to express it, we would have had an obviously
preferable form of words to (10). It's true that we have the words in
(9) itself, but (a) they are longer, and (b) on the contextualist view
whenever we use an epistemic modal there is some such paraphrase
available, a paraphrase that typically does not defeat the acceptability
of epistemic modals.

If there isn't a pragmatic explanation of why (10) can't be used to
express (9), then there must be a semantic explanation. But the only
semantic explanation that looks plausible from a contextualist
perspective, is a semantic restriction on \emph{X}. And we know, from
considering the data about embedded epistemic modals, that there is no
such restriction. So we have a problem for contextualism. Slightly more
formally, we can offer the following argument against orthodox
contextualism about epistemic modals.

\begin{enumerate}
\def\labelenumi{\arabic{enumi}.}
\tightlist
\item
  Whatever acceptability data can't be explained pragmatically must be
  explained semantically.
\item
  There is no pragmatic explanation for why (10) can't be used to
  express (9).
\item
  If 1 and 2 then the meaning of ``might'' must explain why (10) can't
  be used to express (9).
\item
  If the meaning of ``might'' must explain why (10) can't be used to
  express (9), and contextualism is true, there must be a restriction,
  provided by the meaning of ``might'' on the relevant group \emph{X}
  that excludes groups like those guys.
\item
  If there is a restriction, provided by the meaning of ``might'' on the
  relevant group \emph{X} that excludes groups like those guys, then
  when ``might'' is embedded under an attitude verb, the group \emph{X}
  still can't be those guys.
\item
  In ``Those guys believe that they might be in trouble'', the relevant
  group \emph{X} just is those guys.
\item
  So, contextualism is false.
\end{enumerate}

This argument is just an argument against contextualism about ``might''.
It doesn't obviously generalise very far. It's crucial to the argument
that (10) can't be used to express (9), even when the relevant guys are
made pretty salient. A similar argument against contextualism about,
say, taste claims, would have to start with the premise that a clause
like ``but it's tasty'', at the end of a sentence about \emph{a},
couldn't be used to express the thought that it is tasty to \emph{a}.
And such a premise wouldn't be true. As Tamina Stephenson
(\citeproc{ref-Stephenson2007}{2007}) points out, make a particular dog
salient and ``It's tasty'' can express the thought that it is tasty to
the dog. So I'm rather sceptical that the considerations here generalise
to an argument against contextualism about predicates of personal taste.

\section{Against Moral Relativism}\label{against-moral-relativism}

At first glance there seems to be very little pattern to the way that
contextually sensitive terms behave in attitude ascriptions. We can find
terms, like we, whose denotation inside a belief ascription is not
particularly sensitive to the context of the ascribee. So in (11), we is
naturally interpreted as denoting the speaker and those around her.

\begin{enumerate}
\def\labelenumi{\arabic{enumi}.}
\setcounter{enumi}{10}
\tightlist
\item
  Otto believes that we are fools.
\item
  Vinny the Vulture believes that rotting carcasses are tasty.
\item
  Suzy believes that this kind of dog food is tasty.
\end{enumerate}

On the other hand, when we use epistemic modals in belief reports, the
relevant `knower' is always the ascribee. Consider, for example, (14).

\begin{enumerate}
\def\labelenumi{\arabic{enumi}.}
\setcounter{enumi}{13}
\tightlist
\item
  Jack believes that Smith might be happy.
\end{enumerate}

So we have a progression of cases, where in (11) the contextually
sensitive term `we' has to get its denotation from the context of
utterance, in (14) the contextually sensitive term `might' gets its
denotation from the context of the ascribee, and (12) and (13) show that
`tasty' can behave in either of these ways. I've been putting this all
in terms that will make most sense if we are accepting truth absolutism,
but the same points can be made without that assumption if we so desire.

As I said, at first it might look like there is no pattern here at all.
But if we look at other attitudes, we see that there is an interesting
pattern. The way that `we', `tasty' and `might' behave in belief reports
is just the same as they behave in knowledge reports. We can see this in
the following examples.

\begin{itemize}
\tightlist
\item
  Otto knows that we are fools.
\item
  Vinny the Vulture knows that rotting carcasses are tasty.
\item
  Suzy knows that this dog food is tasty.
\item
  Jack knows that Smith might be happy.
\end{itemize}

\emph{t} behaved quite differently in belief and knowledge reports. If
that were the case it would be possible in principle to find a passage
of the form of (15) that's true.

\begin{enumerate}
\def\labelenumi{\arabic{enumi}.}
\setcounter{enumi}{14}
\tightlist
\item
  S believes that \ldots{} t \ldots. Indeed S knows it. But S doesn't
  know that \ldots{} t \ldots{} .
\end{enumerate}

The problem for contextualism about ``wrong'' is that it is forced to
violate this principle. Assume that X is wrong means that X is wrong
relative to the standards of some salient group G. We'll leave aside for
now the question of whether G is determined by the speaker's context or
the assessor's context, as well as the question of whether the sentence
expresses a proposition involving G, or a proposition that is true or
false relative to groups. We'll also leave aside the question of just
what it means for something to be wrong relative to the standards. (Does
it mean that G actually disapproves of it, or would disapprove of it
under reflection, or that it doesn't have properties that G wants to
promote, or something else?) We'll simply assume that there have been
people whose standards are different to ours in ways that make a
difference for the wrongness of actions. If that isn't the case, we
hardly have a relativism worthy of the name. It's obviously
controversial just what could be an example of this, but I'll take as my
example Jefferson Davis's belief that helping fugitive slaves was wrong.
It seems true to say Davis had this belief, so (16) is true.

\begin{enumerate}
\def\labelenumi{\arabic{enumi}.}
\setcounter{enumi}{15}
\tightlist
\item
  Davis believed that helping fugitive slaves was wrong.
\item
  Davis knew that helping fugitive slaves was wrong.
\end{enumerate}

Now neither the truth of (16), nor the falsity of (17) is, on its own,
sufficient to undermine contextualism about ``wrong''. The truth of (16)
is consistent with the claim that ``wrong'' behaves like ``might''. So
in attitude ascriptions, what matters is the ascribees context. And the
falsity of (17) is consistent with the claim that ``wrong'' behaves like
``we'', and (17) is false because what helping fugitive slaves was wrong
expresses in our context is false.

Rather, the problem is that an adequate account of ``wrong'' has to
account both for the truth of (16) and the falsity of (17). And that
doesn't seem to be possible. At least it isn't possible without
supposing that ``wrong'' behaves differently in knowledge reports and
belief reports. And we've seen some reasons above to believe that that's
not how context-sensitive terms behave, whether the term is one like
``we'', for which an orthodox theory seems best, or like ``might'', for
which a heterodox theory seems best.

I'll end with some objections that I've encountered since I've started
discussing this argument, with my replies to each of them.

\begin{quote}
\emph{Objection: This is question-begging against the moral relativist}.
\end{quote}

This is the most common reaction I've heard, but I do find it hard to
make sense of. It is hard to see just which premise is question-begging.
Nothing in moral relativism as such prevents us accepting the truth of
\emph{Davis believed that helping fugitive slaves was wrong}, and
nothing in moral relativism prevents us from rejecting \emph{Davis knew
that helping fugitive slaves was wrong}. There is, I say, a problem with
doing both of these things, as we should want to do. But if an argument
is going to be rejected as question-begging because the other side can't
simultaneously accept \emph{all} of its premises, well we won't have
many arguments left to work with.

A little more seriously, the relativist theories that I'm opposing here
are semantic theories. If we can't reject them because they commit us to
endorsing sentences that we (the opponents of the view) can see to be
false, then it is hard to know what could count as an argument in
semantics. It's no defence of a view to say that its proponents cannot
see it is false, if the rest of us can see it.

\begin{quote}
\emph{Objection: We would see the knowledge claim (17) is true, if only
we didn't have anti-relativist prejudices}.
\end{quote}

This might well be right; it's certainly hard to know when one is
prejudice free. Perhaps all that's going on is that we don't want to be
committed in any way to saying that it's wrong to help fugitive slaves,
and we're worried that accepting (17) would, in some way, so commit us.

But note how much I've done to stack the deck in favour of
pro-relativist intuitions, and to dissipate this worry. The argument is
coming at the end of a whole paper defending relativism. Earlier in this
very paper I defended some relativist views from arguments using factive
attitude verbs by noting that it is tricky to state just what factivity
comes to. In particular, I noted that we can sometimes say \emph{A knows
that S} in circumstances where we would not, indeed could not
truthfully, utter \emph{S}. And I repeated this observation at the start
of this section. I think I've done as much as possible to (a) overcome
anti-relativist prejudice, and (b) frame the argument in such a way as
to make it as easy as possible to accept (17). But even in this frame, I
still say we can see that it is false.

\begin{quote}
\emph{Objection: This is only an objection to one kind of
context-sensitivity in moral terms, the kind we associate with
traditional moral relativism. But it doesn't show that moral terms are
in no way context-sensitive. We'd expect that they are, since some moral
terms are gradable adjectives, and gradable adjectives are
context-sensitive}.
\end{quote}

There's a really interesting philosophical position around here. Start
with the idea that we should be invariantists, perhaps realists in some
quite strong sense, about moral comparatives. Perhaps this could be tied
to the fairly intuitive idea that comparatives are what's crucial to
morality. Then say that terms like ``right'' and ``wrong'' pick out, in
a context-sensitive way, points on the moral scale. So some kind of
contextualism, presumably orthodox, is right for those terms. This
position is immune to the objection given above, because (16) turns out
to be true, and (17) false, if we interpret ``wrong'' to mean
\emph{above the actually contextually-salient level of wrongness, on the
objectively correct wrongness scale}.

But I think a similar pair of examples show that this won't work. Assume
that we're talking about various people's charitable giving in a context
where we don't hold people to super-high standards. So the charitable
actions of, say, Bill Gates count as laudable in our circumstances. (I
assume that on the merits Gates's donations are for the good;
determining whether this is true is well outside the scope of this
paper.) Now consider a philosopher, call him Peter, who doesn't believe
in the moral supererogatory, so he thinks anything less than the best
you can do is morally wrong. It seems to me that, as uttered in our
context, (18) expresses a truth and (19) a falsehood.

\begin{enumerate}
\def\labelenumi{\arabic{enumi}.}
\setcounter{enumi}{17}
\tightlist
\item
  Peter believes that Bill Gates's level of charitable donation is
  wrong.
\item
  Peter knows that Bill Gates's level of charitable donation is wrong.
\end{enumerate}

And I don't think there's a good contextualist explanation for this pair
of judgments. If ``wrong'' was just a simple context-sensitive term in
the way suggested, then (18) should be false, because Peter doesn't
believe that Bill Gates's level of charitable donations do rise to the
level of wrongness that is, as it happens, is salient in our context.
But intuitively, (18) is true.

The same kind of objection can be raised to a more prominent kind of
theory that takes a certain kind of normative standard to be set by
context, namely classical contextualism about ``knows''. Assume it is
common ground that George has excellent, but not irrefutable, evidence
that he has hands. Assume also that we're in a low standards context for
knowledge. And assume that René thinks knowledge requires objective
certainty. Then it seems that we can truly say (20), but not (21).

\begin{enumerate}
\def\labelenumi{\arabic{enumi}.}
\setcounter{enumi}{19}
\tightlist
\item
  René believes that George doesn't know he has hands.
\item
  René knows that George doesn't know he has hands.
\end{enumerate}

Again, the pattern here is very hard to explain on any kind of
contextualist theory, be it orthodox or heterodox.

\begin{quote}
\emph{Objection: Normative terms might be sui generis. Perhaps they are
the only counterexamples to the pattern in (15)}.
\end{quote}

Anything could be the exception to any rule we like. But it's bad
practice to assume that we have an exception on our hands. If we
heterodox theorists simply responded to von Fintel and Gillies' argument
from factive verbs by saying that we had an exception to a general
pattern here, we would, quite rightly, not be taken seriously.
Contextualists and relativists about normative terms should be held to
the same standard.

\begin{quote}
\emph{Objection: Relativism does have counterintuitive consequences, but
all theories have some counterintuitive consequences. Arguably everyone
is going to have to accept some kind of error theory, and this is a
relatively harmless kind of error to attribute to the folk}.
\end{quote}

If we were convinced, perhaps by one or other of the contemporary
developments of Mackie's argument from queerness
(\citeproc{ref-Mackie1977}{Mackie 1977}), that no non-relativistic moral
theory is possible (apart from a Mackie-style moral error theory), that
would be an interesting argument for moral relativism. Certainly I'd be
more willing to accept that (16) and (17) don't have the same kind of
context-sensitivity than I would be to accept that, say, it's not wrong
to torture babies.

It is well beyond the scope of this paper to adjudicate such debates in
any detail, but I am a little sceptical that we will ever face such a
choice. Generalising wildly, most of the time our choice is between (a)
accepting a moral error theory, (b) accepting some odd semantic
consequences, as outlined here, or (c) rejecting some somewhat plausible
claim about the nature of moral judgment, such as motivational
internalism. (Without internalism there's nothing to make moral
properties ``queer'' in Mackie's sense, for example.) Arguments that
present a trilemma such as this deserve to be judged on their merits,
but my feeling is that we should normally take option (c). That's not to
say this objection is obviously flawed, or that the argument I've
offered is a knock-down refutation of relativism. It clearly is not. But
I think it is a challenge that moral relativists and contextualists
should face up to.

\subsection*{References}\label{references}
\addcontentsline{toc}{subsection}{References}

\phantomsection\label{refs}
\begin{CSLReferences}{1}{0}
\bibitem[\citeproctext]{ref-Cappelen2005}
Cappelen, Herman, and Ernest Lepore. 2005. \emph{Insensitive Semantics:
A Defence of Semantic Minimalism and Speech Act Pluralism}. Oxford:
Blackwell.

\bibitem[\citeproctext]{ref-Egan2007-EGAEMR}
Egan, Andy. 2007. {``{Epistemic Modals, Relativism and Assertion}.''}
\emph{Philosophical Studies} 133 (1): 1--22. doi:
\href{https://doi.org/10.1007/s11098-006-9003-x}{10.1007/s11098-006-9003-x}.

\bibitem[\citeproctext]{ref-Egan2009-EGABBA}
---------. 2009. {``{Billboards, Bombs and Shotgun Weddings}.''}
\emph{Synthese} 166 (2): 251--79. doi:
\href{https://doi.org/10.1007/s11229-007-9284-4}{10.1007/s11229-007-9284-4}.

\bibitem[\citeproctext]{ref-Egan2005-EGAEMI}
Egan, Andy, John Hawthorne, and Brian Weatherson. 2005. {``{Epistemic
Modals in Context}.''} In \emph{Contextualism in Philosophy: Knowledge,
Meaning, and Truth}, edited by Gerhard Preyer and Georg Peter, 131--70.
Oxford: Oxford University Press.

\bibitem[\citeproctext]{ref-vonFintel2008}
Fintel, Kai von, and Anthony S. Gillies. 2008. {``CIA Leaks.''}
\emph{Philosophical Review} 117 (1): 77--98. doi:
\href{https://doi.org/10.1215/00318108-2007-025}{10.1215/00318108-2007-025}.

\bibitem[\citeproctext]{ref-Glanzberg2007}
Glanzberg, Michael. 2007. {``Context, Content and Relativism.''}
\emph{Philosophical Studies} 136 (1): 1--29. doi:
\href{https://doi.org/10.1007/s11098-007-9145-5}{10.1007/s11098-007-9145-5}.

\bibitem[\citeproctext]{ref-MacFarlane2009-MACNC}
MacFarlane, John. 2009. {``{Nonindexical Contextualism}.''}
\emph{Synthese} 166 (2): 231--50. doi:
\href{https://doi.org/10.1007/s11229-007-9286-2}{10.1007/s11229-007-9286-2}.

\bibitem[\citeproctext]{ref-Mackie1977}
Mackie, John. 1977. \emph{Ethics: Inventing Right and Wrong}. London:
Penguin.

\bibitem[\citeproctext]{ref-Stephenson2007}
Stephenson, Tamina. 2007. {``Judge Dependence, Epistemic Modals, and
Predicates of Personal Taste.''} \emph{Linguistics and Philosophy} 30
(4): 487--525. doi:
\href{https://doi.org/10.1007/s10988-008-9023-4}{10.1007/s10988-008-9023-4}.

\bibitem[\citeproctext]{ref-Weatherson2002-WEAMI}
Weatherson, Brian. 2002. {``Misleading Indexicals.''} \emph{Analysis} 62
(4): 308--10. doi:
\href{https://doi.org/10.1093/analys/62.4.308}{10.1093/analys/62.4.308}.

\bibitem[\citeproctext]{ref-Weatherson2009-WEACAI}
---------. 2009. {``{Conditionals and Indexical Relativism}.''}
\emph{Synthese} 166 (2): 333--57. doi:
\href{https://doi.org/10.1007/s11229-007-9283-5}{10.1007/s11229-007-9283-5}.

\end{CSLReferences}



\noindent Published in\emph{
Philosophical Perspectives}, 2008, pp. 527-544.


\end{document}
