% Options for packages loaded elsewhere
\PassOptionsToPackage{unicode}{hyperref}
\PassOptionsToPackage{hyphens}{url}
%
\documentclass[
  11pt,
  letterpaper,
  DIV=11,
  numbers=noendperiod,
  twoside]{scrartcl}

\usepackage{amsmath,amssymb}
\usepackage{setspace}
\usepackage{iftex}
\ifPDFTeX
  \usepackage[T1]{fontenc}
  \usepackage[utf8]{inputenc}
  \usepackage{textcomp} % provide euro and other symbols
\else % if luatex or xetex
  \usepackage{unicode-math}
  \defaultfontfeatures{Scale=MatchLowercase}
  \defaultfontfeatures[\rmfamily]{Ligatures=TeX,Scale=1}
\fi
\usepackage{lmodern}
\ifPDFTeX\else  
    % xetex/luatex font selection
    \setmainfont[ItalicFont=EB Garamond Italic,BoldFont=EB Garamond
Bold]{EB Garamond Math}
    \setsansfont[]{EB Garamond}
  \setmathfont[]{Garamond-Math}
\fi
% Use upquote if available, for straight quotes in verbatim environments
\IfFileExists{upquote.sty}{\usepackage{upquote}}{}
\IfFileExists{microtype.sty}{% use microtype if available
  \usepackage[]{microtype}
  \UseMicrotypeSet[protrusion]{basicmath} % disable protrusion for tt fonts
}{}
\usepackage{xcolor}
\usepackage[left=1.1in, right=1in, top=0.8in, bottom=0.8in,
paperheight=9.5in, paperwidth=7in, includemp=TRUE, marginparwidth=0in,
marginparsep=0in]{geometry}
\setlength{\emergencystretch}{3em} % prevent overfull lines
\setcounter{secnumdepth}{3}
% Make \paragraph and \subparagraph free-standing
\makeatletter
\ifx\paragraph\undefined\else
  \let\oldparagraph\paragraph
  \renewcommand{\paragraph}{
    \@ifstar
      \xxxParagraphStar
      \xxxParagraphNoStar
  }
  \newcommand{\xxxParagraphStar}[1]{\oldparagraph*{#1}\mbox{}}
  \newcommand{\xxxParagraphNoStar}[1]{\oldparagraph{#1}\mbox{}}
\fi
\ifx\subparagraph\undefined\else
  \let\oldsubparagraph\subparagraph
  \renewcommand{\subparagraph}{
    \@ifstar
      \xxxSubParagraphStar
      \xxxSubParagraphNoStar
  }
  \newcommand{\xxxSubParagraphStar}[1]{\oldsubparagraph*{#1}\mbox{}}
  \newcommand{\xxxSubParagraphNoStar}[1]{\oldsubparagraph{#1}\mbox{}}
\fi
\makeatother


\providecommand{\tightlist}{%
  \setlength{\itemsep}{0pt}\setlength{\parskip}{0pt}}\usepackage{longtable,booktabs,array}
\usepackage{calc} % for calculating minipage widths
% Correct order of tables after \paragraph or \subparagraph
\usepackage{etoolbox}
\makeatletter
\patchcmd\longtable{\par}{\if@noskipsec\mbox{}\fi\par}{}{}
\makeatother
% Allow footnotes in longtable head/foot
\IfFileExists{footnotehyper.sty}{\usepackage{footnotehyper}}{\usepackage{footnote}}
\makesavenoteenv{longtable}
\usepackage{graphicx}
\makeatletter
\newsavebox\pandoc@box
\newcommand*\pandocbounded[1]{% scales image to fit in text height/width
  \sbox\pandoc@box{#1}%
  \Gscale@div\@tempa{\textheight}{\dimexpr\ht\pandoc@box+\dp\pandoc@box\relax}%
  \Gscale@div\@tempb{\linewidth}{\wd\pandoc@box}%
  \ifdim\@tempb\p@<\@tempa\p@\let\@tempa\@tempb\fi% select the smaller of both
  \ifdim\@tempa\p@<\p@\scalebox{\@tempa}{\usebox\pandoc@box}%
  \else\usebox{\pandoc@box}%
  \fi%
}
% Set default figure placement to htbp
\def\fps@figure{htbp}
\makeatother
% definitions for citeproc citations
\NewDocumentCommand\citeproctext{}{}
\NewDocumentCommand\citeproc{mm}{%
  \begingroup\def\citeproctext{#2}\cite{#1}\endgroup}
\makeatletter
 % allow citations to break across lines
 \let\@cite@ofmt\@firstofone
 % avoid brackets around text for \cite:
 \def\@biblabel#1{}
 \def\@cite#1#2{{#1\if@tempswa , #2\fi}}
\makeatother
\newlength{\cslhangindent}
\setlength{\cslhangindent}{1.5em}
\newlength{\csllabelwidth}
\setlength{\csllabelwidth}{3em}
\newenvironment{CSLReferences}[2] % #1 hanging-indent, #2 entry-spacing
 {\begin{list}{}{%
  \setlength{\itemindent}{0pt}
  \setlength{\leftmargin}{0pt}
  \setlength{\parsep}{0pt}
  % turn on hanging indent if param 1 is 1
  \ifodd #1
   \setlength{\leftmargin}{\cslhangindent}
   \setlength{\itemindent}{-1\cslhangindent}
  \fi
  % set entry spacing
  \setlength{\itemsep}{#2\baselineskip}}}
 {\end{list}}
\usepackage{calc}
\newcommand{\CSLBlock}[1]{\hfill\break\parbox[t]{\linewidth}{\strut\ignorespaces#1\strut}}
\newcommand{\CSLLeftMargin}[1]{\parbox[t]{\csllabelwidth}{\strut#1\strut}}
\newcommand{\CSLRightInline}[1]{\parbox[t]{\linewidth - \csllabelwidth}{\strut#1\strut}}
\newcommand{\CSLIndent}[1]{\hspace{\cslhangindent}#1}

\setlength\heavyrulewidth{0ex}
\setlength\lightrulewidth{0ex}
\usepackage[automark]{scrlayer-scrpage}
\clearpairofpagestyles
\cehead{
  Brian Weatherson
  }
\cohead{
  Induction and Supposition
  }
\ohead{\bfseries \pagemark}
\cfoot{}
\makeatletter
\newcommand*\NoIndentAfterEnv[1]{%
  \AfterEndEnvironment{#1}{\par\@afterindentfalse\@afterheading}}
\makeatother
\NoIndentAfterEnv{itemize}
\NoIndentAfterEnv{enumerate}
\NoIndentAfterEnv{description}
\NoIndentAfterEnv{quote}
\NoIndentAfterEnv{equation}
\NoIndentAfterEnv{longtable}
\NoIndentAfterEnv{abstract}
\renewenvironment{abstract}
 {\vspace{-1.25cm}
 \quotation\small\noindent\emph{Abstract}:}
 {\endquotation}
\newfontfamily\tfont{EB Garamond}
\addtokomafont{disposition}{\rmfamily}
\addtokomafont{title}{\normalfont\itshape}
\let\footnoterule\relax
\KOMAoption{captions}{tableheading}
\makeatletter
\@ifpackageloaded{caption}{}{\usepackage{caption}}
\AtBeginDocument{%
\ifdefined\contentsname
  \renewcommand*\contentsname{Table of contents}
\else
  \newcommand\contentsname{Table of contents}
\fi
\ifdefined\listfigurename
  \renewcommand*\listfigurename{List of Figures}
\else
  \newcommand\listfigurename{List of Figures}
\fi
\ifdefined\listtablename
  \renewcommand*\listtablename{List of Tables}
\else
  \newcommand\listtablename{List of Tables}
\fi
\ifdefined\figurename
  \renewcommand*\figurename{Figure}
\else
  \newcommand\figurename{Figure}
\fi
\ifdefined\tablename
  \renewcommand*\tablename{Table}
\else
  \newcommand\tablename{Table}
\fi
}
\@ifpackageloaded{float}{}{\usepackage{float}}
\floatstyle{ruled}
\@ifundefined{c@chapter}{\newfloat{codelisting}{h}{lop}}{\newfloat{codelisting}{h}{lop}[chapter]}
\floatname{codelisting}{Listing}
\newcommand*\listoflistings{\listof{codelisting}{List of Listings}}
\makeatother
\makeatletter
\makeatother
\makeatletter
\@ifpackageloaded{caption}{}{\usepackage{caption}}
\@ifpackageloaded{subcaption}{}{\usepackage{subcaption}}
\makeatother

\usepackage{bookmark}

\IfFileExists{xurl.sty}{\usepackage{xurl}}{} % add URL line breaks if available
\urlstyle{same} % disable monospaced font for URLs
\hypersetup{
  pdftitle={Induction and Supposition},
  pdfauthor={Brian Weatherson},
  hidelinks,
  pdfcreator={LaTeX via pandoc}}


\title{Induction and Supposition}
\author{Brian Weatherson}
\date{2012}

\begin{document}
\maketitle
\begin{abstract}
An argument that we should not treat rules of inductive inference in
ordinary life as being anything like the inference rules in natural
deduction systems.
\end{abstract}


\setstretch{1.1}
Here's a fairly quick argument that there is contingent a priori
knowledge. Assume there are some ampliative inference rules. Since the
alternative appears to be inductive scepticism, this seems like a safe
enough assumption. Such a rule will, since it is ampliative, licence
some particular inference \emph{From} \emph{A} infer \emph{B} where
\emph{A} does not entail \emph{B}. That's just what it is for the rule
to be ampliative. Now run that rule inside suppositional reasoning. In
particular, first assume \emph{A}, then via this rule infer \emph{B}.
Now do a step of →-introduction, inferring \emph{A}~→~\emph{B} and
discharging the assumption \emph{A}. Since \emph{A} does not entail
\emph{B}, this will be contingent, and since it rests on a sound
inference with no (undischarged) assumptions, it is a priori knowledge.

This argument is hardly new; John Hawthorne
(\citeproc{ref-Hawthorne2002}{2002}) suggested a similar argument ten
years ago. But it is a quick argument for a striking conclusion, and
deserves close scrutiny. I'm going to argue that it fails because it
falsely assumes that we can treat rules of ampliative inference like
rules in a natural deduction system, and hence as rules that we can
apply inside the scope of a supposition. That assumption has recently
been defended by Stewart Cohen (\citeproc{ref-Cohen2010}{2010}) and
Sinan Dogramaci (\citeproc{ref-Dogramaci2010}{2010}), but I'm going to
argue, using a construction similar to one found in Dogramaci, that it
leads to absurdity given other plausible premises.

Here's the main argument. If any ampliative inference is justified, I
think the following rule, called `IR', is justified, since this is a
very weak form of an inductive inference.

\begin{description}
\tightlist
\item[IR]
From \emph{There are infinitely many Fs, and at most one is not G} and
\emph{x is F} infer \emph{x is G} unless there is some \emph{H} such
that it is provable from the undischarged assumptions that \emph{x is F
and H} and \emph{There are finitely many things that are both F and H,
and one of them is not G}.
\end{description}

Note that the rule doesn't say that merely one \emph{F}~∧~¬\emph{G} has
been observed; it requires that just one such thing exists. So this
seems like a very plausible inference; it really is just making an
inference within a known distribution, not outside it. And it is
explicitly qualified to deal with defeaters. And yet even this rule,
when applied inside the scope of suppositions, can lead to absurdity.

In the following proof, we'll let \emph{N} be the predicate `is a
natural number', and \emph{p} be the predicate `is the predecessor of',
and I'll appeal to the fact that there are infinitely many natural
numbers, and each number has at most one predecessor. I'll use a version
of the proof system in E. J. Lemmon's \emph{Beginning Logic}, but it
should be easy to transform the proof into any other proof system.

\[
\begin{aligned}
1 && (1) && &Na && \text{assumption} \\
2 && (2) && &Nb && \text{assumption} \\
1, 2 && (3) && &\neg Pab && \text{(1), (2), IR} \\
1 && (4)  && &Nb \rightarrow \neg Pab && \text{(2), (3), CP} \\
1 && (5)  && &\forall y (Ny \rightarrow \neg Pay) && \text{(4), UI} \\
 && (6)  && &Na \rightarrow \forall y (Ny \rightarrow \neg Pay) && \text{(1), (5), CP} \\
 && (7) && &\forall x (Nx \rightarrow \forall y (Ny \rightarrow \neg Pxy)) && \text{(6), UI} \\
 && (8) && &N2 \rightarrow \forall y (Ny \rightarrow \neg P2y) && \text{(7), UE}
\end{aligned}
\]

So we get the absurd result that if 2 is a number (which it is!), then
it is the predecessor of no number. But that's absurd, since obviously 3
is a number and 2 is the predecessor of it. Note that at step 3 we use
rule IR with \emph{F} being the predicate \emph{is a natural number},
\emph{G} being the predicate \emph{does not have a as a predecessor},
and \emph{b} being \emph{x}.

What could have gone wrong? I think the problem is using IR in the
context of a suppositional proof, as we've done here. But let's check if
there is another guilty suspect.

If the problem is Conditional Proof (CP in Lemmon's system), then that's
about as bad for the proof in the first paragraph that there are
contingent a priori truths as if the problem is IR. Since we're
interested in whether that proof works, we won't investigate this option
further. In any case, if~→~is material implication, that rule seems
unobjectionable. A referee suggested that if we've used an ampliative
rule earlier, then~→~should be weaker than material implication, and
under that interpretation (5) through (8) may be plausible. I think that
claim is basically right, but note that if we do this the argument for
contingent a priori knowledge with which I started will fail, since the
contingency of \emph{A}~⊃~\emph{B} will not imply the contingency of
\emph{A}~→~\emph{B} if~→~is weaker than ⊃.

It is hard to imagine that Universal Elimination (UE) is the problem. In
any case, line (7) is obviously bad anyway, so something must have gone
wrong in the proof before that.

Perhaps the problem is with Universal Introduction (UI); this is what
Dogramaci suggests. One objection he offers is that although we can
prove every instance of the universal quantifier, inferring the
universal version creates an undue aggregation of risks. Even if line
(4) is very probable, and it would still be probable if \emph{a} were
replaced with \emph{c}, \emph{d} or any other name, it doesn't follow
that the universal at line (5) is very probable. But I think this is to
confuse defeasible reasoning with probabilistic reasoning. The only way
to implement this restriction on making inferences that aggregate risk
would be to prevent us making any inference where the conclusion was
less probable than the premises. That will rule out uses of
∀-introduction as at (5). But it will also rule ∧-introduction, and
indeed any other inference with more than one input step. To impose such
a restriction would be to cripple natural deduction.

Another objection he offers (UI) is simply that it is the least
plausible, or least intuitive, of the rules used here. But in fact (UI)
is extremely intuitive. If we can prove every instance of a schema, we
should be able to prove its universal closure. On the other hand,
allowing ampliative rules to be used inside the scope of a supposition
allows a quick proof of contingent a priori knowledge, as shown in the
first paragraph. Now maybe there is such knowledge, but its existence is
hardly intuitive.

So I conclude the weakest link in the argument is step (3). Although IR
is a good rule, it can't be used inside the scope of a supposition. And
since IR is about as weak an inductive rule as we can imagine, I
conclude that ampliative inference rules can't in general be used inside
the scope of suppositions.

The general lesson here is that, as was made clear many years ago by
Gilbert Harman (\citeproc{ref-Harman1986}{1986}) is that there is a
difference between rules of inference and rules of implication. The
quick proof that there's contingent a priori knowledge uses a rule of
inference as if it is a rule of implication. Not respecting this
distinction between inference and implication leads to disaster, as
we've shown here, and should be shunned.

\section*{References}\label{references}
\addcontentsline{toc}{section}{References}

\phantomsection\label{refs}
\begin{CSLReferences}{1}{0}
\bibitem[\citeproctext]{ref-Cohen2010}
Cohen, Stewart. 2010. {``Bootstrapping, Defeasible Reasoning and \emph{a
Priori} Justification.''} \emph{Philosophical Perspectives} 24 (1):
141--59. doi:
\href{https://doi.org/10.1111/j.1520-8583.2010.00188.x}{10.1111/j.1520-8583.2010.00188.x}.

\bibitem[\citeproctext]{ref-Dogramaci2010}
Dogramaci, Sinan. 2010. {``Knowledge of Validity.''} \emph{No{û}s} 44
(3): 403--32. doi:
\href{https://doi.org/0.1111/j.1468-0068.2010.00746.x}{0.1111/j.1468-0068.2010.00746.x}.

\bibitem[\citeproctext]{ref-Harman1986}
Harman, Gilbert. 1986. \emph{Change in View}. Cambridge, MA: Bradford.

\bibitem[\citeproctext]{ref-Hawthorne2002}
Hawthorne, John. 2002. {``Deeply Contingent a Priori Knowledge.''}
\emph{Philosophy and Phenomenological Research} 65 (2): 247--69. doi:
\href{https://doi.org/10.1111/j.1933-1592.2002.tb00201.x}{10.1111/j.1933-1592.2002.tb00201.x}.

\end{CSLReferences}



\noindent Published in\emph{
The Reasoner}, 2012, pp. 78-80.


\end{document}
