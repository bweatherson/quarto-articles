% Options for packages loaded elsewhere
\PassOptionsToPackage{unicode}{hyperref}
\PassOptionsToPackage{hyphens}{url}
%
\documentclass[
  10pt,
  letterpaper,
  DIV=11,
  numbers=noendperiod,
  twoside]{scrartcl}

\usepackage{amsmath,amssymb}
\usepackage{setspace}
\usepackage{iftex}
\ifPDFTeX
  \usepackage[T1]{fontenc}
  \usepackage[utf8]{inputenc}
  \usepackage{textcomp} % provide euro and other symbols
\else % if luatex or xetex
  \usepackage{unicode-math}
  \defaultfontfeatures{Scale=MatchLowercase}
  \defaultfontfeatures[\rmfamily]{Ligatures=TeX,Scale=1}
\fi
\usepackage{lmodern}
\ifPDFTeX\else  
    % xetex/luatex font selection
    \setmainfont[ItalicFont=EB Garamond Italic,BoldFont=EB Garamond
Bold]{EB Garamond Math}
    \setsansfont[]{Europa-Bold}
  \setmathfont[]{Garamond-Math}
\fi
% Use upquote if available, for straight quotes in verbatim environments
\IfFileExists{upquote.sty}{\usepackage{upquote}}{}
\IfFileExists{microtype.sty}{% use microtype if available
  \usepackage[]{microtype}
  \UseMicrotypeSet[protrusion]{basicmath} % disable protrusion for tt fonts
}{}
\usepackage{xcolor}
\usepackage[left=1in, right=1in, top=0.8in, bottom=0.8in,
paperheight=9.5in, paperwidth=6.5in, includemp=TRUE, marginparwidth=0in,
marginparsep=0in]{geometry}
\setlength{\emergencystretch}{3em} % prevent overfull lines
\setcounter{secnumdepth}{3}
% Make \paragraph and \subparagraph free-standing
\makeatletter
\ifx\paragraph\undefined\else
  \let\oldparagraph\paragraph
  \renewcommand{\paragraph}{
    \@ifstar
      \xxxParagraphStar
      \xxxParagraphNoStar
  }
  \newcommand{\xxxParagraphStar}[1]{\oldparagraph*{#1}\mbox{}}
  \newcommand{\xxxParagraphNoStar}[1]{\oldparagraph{#1}\mbox{}}
\fi
\ifx\subparagraph\undefined\else
  \let\oldsubparagraph\subparagraph
  \renewcommand{\subparagraph}{
    \@ifstar
      \xxxSubParagraphStar
      \xxxSubParagraphNoStar
  }
  \newcommand{\xxxSubParagraphStar}[1]{\oldsubparagraph*{#1}\mbox{}}
  \newcommand{\xxxSubParagraphNoStar}[1]{\oldsubparagraph{#1}\mbox{}}
\fi
\makeatother


\providecommand{\tightlist}{%
  \setlength{\itemsep}{0pt}\setlength{\parskip}{0pt}}\usepackage{longtable,booktabs,array}
\usepackage{calc} % for calculating minipage widths
% Correct order of tables after \paragraph or \subparagraph
\usepackage{etoolbox}
\makeatletter
\patchcmd\longtable{\par}{\if@noskipsec\mbox{}\fi\par}{}{}
\makeatother
% Allow footnotes in longtable head/foot
\IfFileExists{footnotehyper.sty}{\usepackage{footnotehyper}}{\usepackage{footnote}}
\makesavenoteenv{longtable}
\usepackage{graphicx}
\makeatletter
\def\maxwidth{\ifdim\Gin@nat@width>\linewidth\linewidth\else\Gin@nat@width\fi}
\def\maxheight{\ifdim\Gin@nat@height>\textheight\textheight\else\Gin@nat@height\fi}
\makeatother
% Scale images if necessary, so that they will not overflow the page
% margins by default, and it is still possible to overwrite the defaults
% using explicit options in \includegraphics[width, height, ...]{}
\setkeys{Gin}{width=\maxwidth,height=\maxheight,keepaspectratio}
% Set default figure placement to htbp
\makeatletter
\def\fps@figure{htbp}
\makeatother
% definitions for citeproc citations
\NewDocumentCommand\citeproctext{}{}
\NewDocumentCommand\citeproc{mm}{%
  \begingroup\def\citeproctext{#2}\cite{#1}\endgroup}
\makeatletter
 % allow citations to break across lines
 \let\@cite@ofmt\@firstofone
 % avoid brackets around text for \cite:
 \def\@biblabel#1{}
 \def\@cite#1#2{{#1\if@tempswa , #2\fi}}
\makeatother
\newlength{\cslhangindent}
\setlength{\cslhangindent}{1.5em}
\newlength{\csllabelwidth}
\setlength{\csllabelwidth}{3em}
\newenvironment{CSLReferences}[2] % #1 hanging-indent, #2 entry-spacing
 {\begin{list}{}{%
  \setlength{\itemindent}{0pt}
  \setlength{\leftmargin}{0pt}
  \setlength{\parsep}{0pt}
  % turn on hanging indent if param 1 is 1
  \ifodd #1
   \setlength{\leftmargin}{\cslhangindent}
   \setlength{\itemindent}{-1\cslhangindent}
  \fi
  % set entry spacing
  \setlength{\itemsep}{#2\baselineskip}}}
 {\end{list}}
\usepackage{calc}
\newcommand{\CSLBlock}[1]{\hfill\break\parbox[t]{\linewidth}{\strut\ignorespaces#1\strut}}
\newcommand{\CSLLeftMargin}[1]{\parbox[t]{\csllabelwidth}{\strut#1\strut}}
\newcommand{\CSLRightInline}[1]{\parbox[t]{\linewidth - \csllabelwidth}{\strut#1\strut}}
\newcommand{\CSLIndent}[1]{\hspace{\cslhangindent}#1}

\setlength\heavyrulewidth{0ex}
\setlength\lightrulewidth{0ex}
\usepackage[automark]{scrlayer-scrpage}
\clearpairofpagestyles
\cehead{
  Brian Weatherson
  }
\cohead{
  Stalnaker on Sleeping Beauty
  }
\ohead{\bfseries \pagemark}
\cfoot{}
\makeatletter
\newcommand*\NoIndentAfterEnv[1]{%
  \AfterEndEnvironment{#1}{\par\@afterindentfalse\@afterheading}}
\makeatother
\NoIndentAfterEnv{itemize}
\NoIndentAfterEnv{enumerate}
\NoIndentAfterEnv{description}
\NoIndentAfterEnv{quote}
\NoIndentAfterEnv{equation}
\NoIndentAfterEnv{longtable}
\NoIndentAfterEnv{abstract}
\renewenvironment{abstract}
 {\vspace{-1.25cm}
 \quotation\small\noindent\rule{\linewidth}{.5pt}\par\smallskip
 \noindent }
 {\par\noindent\rule{\linewidth}{.5pt}\endquotation}
\KOMAoption{captions}{tableheading}
\makeatletter
\@ifpackageloaded{caption}{}{\usepackage{caption}}
\AtBeginDocument{%
\ifdefined\contentsname
  \renewcommand*\contentsname{Table of contents}
\else
  \newcommand\contentsname{Table of contents}
\fi
\ifdefined\listfigurename
  \renewcommand*\listfigurename{List of Figures}
\else
  \newcommand\listfigurename{List of Figures}
\fi
\ifdefined\listtablename
  \renewcommand*\listtablename{List of Tables}
\else
  \newcommand\listtablename{List of Tables}
\fi
\ifdefined\figurename
  \renewcommand*\figurename{Figure}
\else
  \newcommand\figurename{Figure}
\fi
\ifdefined\tablename
  \renewcommand*\tablename{Table}
\else
  \newcommand\tablename{Table}
\fi
}
\@ifpackageloaded{float}{}{\usepackage{float}}
\floatstyle{ruled}
\@ifundefined{c@chapter}{\newfloat{codelisting}{h}{lop}}{\newfloat{codelisting}{h}{lop}[chapter]}
\floatname{codelisting}{Listing}
\newcommand*\listoflistings{\listof{codelisting}{List of Listings}}
\makeatother
\makeatletter
\makeatother
\makeatletter
\@ifpackageloaded{caption}{}{\usepackage{caption}}
\@ifpackageloaded{subcaption}{}{\usepackage{subcaption}}
\makeatother

\ifLuaTeX
  \usepackage{selnolig}  % disable illegal ligatures
\fi
\usepackage{bookmark}

\IfFileExists{xurl.sty}{\usepackage{xurl}}{} % add URL line breaks if available
\urlstyle{same} % disable monospaced font for URLs
\hypersetup{
  pdftitle={Stalnaker on Sleeping Beauty},
  pdfauthor={Brian Weatherson},
  hidelinks,
  pdfcreator={LaTeX via pandoc}}


\title{Stalnaker on Sleeping Beauty}
\author{Brian Weatherson}
\date{2011}

\begin{document}
\maketitle
\begin{abstract}
A contribution to a book symposium on Stalnaker's Our Knowledge of the
Internal World, focussing on the way his framework helps cast new light
on the Sleeping Beauty problem.
\end{abstract}


\setstretch{1.1}
The Sleeping Beauty puzzle provides a nice illustration of the approach
to self-locating belief defended by Robert Stalnaker in \emph{Our
Knowledge of the Internal World} (\citeproc{ref-Stalnaker2008}{Stalnaker
2008}), as well as a test of the utility of that method. The setup of
the Sleeping Beauty puzzle is by now fairly familiar. On Sunday Sleeping
Beauty is told the rules of the game, and a (known to be) fair coin is
flipped. On Monday, Sleeping Beauty is woken, and then put back to
sleep. If, and only if, the coin landed tails, she is woken again on
Tuesday after having her memory of the Monday awakening
erased.\footnote{Note that I'm not assuming that Beauty's memories are
  erased in other cases. This makes the particular version of the case
  I'm discussing a little different to the version popularised in Elga
  (\citeproc{ref-Elga2000}{2000}). This shouldn't make any difference to
  most analyses of the puzzle, but it helps to clarify some issues.} On
Wednesday she is woken again and the game ends. There are a few
questions we can ask about Beauty's attitudes as the game progresses.
We'd like to know what her credence that the coin landed heads should be

\begin{enumerate}
\def\labelenumi{\arabic{enumi}.}
\tightlist
\item
  Before she goes to sleep Sunday;
\item
  When she wakes on Monday;
\item
  When she wakes on Tuesday; and
\item
  When she wakes on Wednesday?
\end{enumerate}

Standard treatments of the Sleeping Beauty puzzle ignore (d), run
together (b) and (c) into one (somewhat ill-formed) question, and then
divide theorists into `halfers' or `thirders' depending on how they
answer it. Following Stalnaker, I'm going to focus on (b) here, though
I'll have a little to say about (c) and (d) as well. I'll be following
orthodoxy in taking ½ to be the clear answer to (a), and in taking the
correct answers to (b) and (c) to be independent of how the coin lands,
though I'll briefly question that assumption at the end.

An answer to these four questions should respect two different kinds of
constraints. The answer for day \emph{n} should make sense `statically'.
It should be a sensible answer to the question of what Beauty should do
given what information she then has. And the answer should make sense
`dynamically'. It should be a sensible answer to the question of how
Beauty should have updated her credences from some earlier day, given
rational credences on the earlier day.

As has been fairly clear since the discussion of the problem in Elga
(\citeproc{ref-Elga2000}{2000}), Sleeping Beauty is puzzling because
static and dynamic considerations appear to push in different
directions. The static considerations apparently favour a ⅓ answer to
(b). When Beauty wakes, there are three options available to her: It is
Monday and the coin landed heads; It is Monday and the coin landed
tails; It is Tuesday and the coin landed tails. If we can argue that
each of those are equally probable given her evidence, we get the answer
⅓. The dynamic considerations apparently favour a ½ answer to (b). The
right answer to (a) is ½. Nothing happens on Monday or Tuesday that
surprises Beauty. And credences should only change if we are surprised.
So the right answer to (b) is ½.

Since we must have harmony between dynamic and static considerations,
one of these arguments must be misguided. (In fact, I think both are, to
some degree.) These days there is a cottage industry of `thirders'
developing accounts of credal dynamics that accord with the ⅓ answer to
(b).\footnote{See, for instance, Titlebaum
  (\citeproc{ref-Titlebaum2008}{2008}) and the references therein.} But
all of these accounts are considerably more complex than the
traditional, conditionalisation-based, dynamic theory that we all grew
up with.

Three of the many attractions of Robert Stalnaker's new account of
self-locating knowledge are (i) that it offers a way to answer all four
of our questions about Sleeping Beauty, (ii) that it does so while
remaining both statically and dynamically plausible, and (iii) that the
dynamic theory involved is, in large part, traditional
conditionalisation. I spend most of this note setting out Stalnaker's
account, and setting out his derivation of a ⅓ answer to (b). I conclude
with some reasons for preferring a slightly different solution of the
Sleeping Beauty puzzle within the broad framework Stalnaker suggests.

\section{Stalnaker on Self-Location}\label{stalnaker-on-self-location}

The picture of self-locating belief that we get from Lewis's ``Attitudes
\emph{De Dicto} and \emph{De Se}'' (\citeproc{ref-Lewis1979b}{Lewis
1979}) has been widely adopted in recent years.\footnote{Including by
  me. See Egan, Hawthorne, and Weatherson
  (\citeproc{ref-Egan2005-EGAEMI}{2005}).} On Lewis's picture, the
content of an attitude is a set of centered worlds. For current purposes
we'll take to centered worlds to be ⟨\{world, agent, time\}⟩ triples. To
believe that \emph{S}, where \emph{S} is a set of centered worlds, is to
believe that the triple ⟨\{your world, you, now\}⟩ ∈ \emph{S}.

The motivation for this picture comes from reflection on how to
represent locational uncertainty. If you're sure where in New York City
you are, you can pick out a point on a map and say ``I'm there''. If
you're not sure exactly where you are, but you have some information,
you can pick out a region on the map and say ``I'm somewhere in that
region''. If you're not sure who you are, but you know where everyone
is, you can do the same kind of thing. And it's plausible that this is a
(somewhat) realistic situation. As one modern-day Lewisian, Andy Egan,
puts it `I can believe that my pants are on fire without believing that
Egan's pants are on fire, and I can hope that someone turns a fire
extinguisher on me right now without hoping that someone turns a fire
extinguisher on Egan at 5:41pm.'\,'
(\citeproc{ref-Egan2004-JACSPA-2}{Egan 2004, 64}) There is an important
puzzle here that needs to be addressed, and can't obviously be addressed
in the framework Lewis accepted before 1979, where the content of a
propositional attitude is a set of Lewisian concreta. If possible worlds
are Lewisian concreta, then Lewisians like Egan are correct to respond
to puzzles about location by saying, ``sometimes (as when we want to
know who or where we are) \textbf{the world is not enough}''.
(\citeproc{ref-Egan2004-JACSPA-2}{Egan 2004, 64})

But this response is too self-centered. Not all locational thoughts are
self-locational thoughts. I can be just as uncertain about where
\emph{that} is as about where \emph{this} is, or as uncertain about who
\emph{you} are as about who \emph{I} am. Imagine I'm watching Egan's
unfortunate adventures with his infernal trousers on a delayed video
tape. I can believe \emph{his} pants are on fire without believing
Egan's pants are on fire, and hope that someone turns a fire
extinguisher on him \emph{then} without hoping some turns a fire
extinguisher on Egan at 5:41pm. Or, at least, that way of putting things
sounds just as good as Egan's original description of the case.

For a different example, imagine I wake at night and come to believe it
is midnight. As Lewis would represent it, I believe
⟨\emph{w},~me,~now⟩~∈~\{⟨\emph{w},~\emph{s},~\emph{t}⟩: \emph{t} =
midnight\}. When I wake, I think back to that belief, and judge that I
may have been mistaken. How should we represent this? Not that I now
believe ⟨\emph{w}, me, now⟩ ∉ \{⟨\emph{w}, \emph{s}, \emph{t}⟩: \emph{t}
= midnight\}. That's obviously true - I know the sun is up. We want to
represent something more contentious.

The best, I think, the Lewisian can do is to pick out some description
\emph{D} of my earlier belief and say what I believe is
⟨\emph{w},~me,~now⟩~∉~\{⟨\emph{w},~\emph{s},~\emph{t}⟩ : (ɩ\emph{x}:
\emph{Dx})\emph{x} happens at midnight\}. That is, I believe the belief
that satisfies \emph{D} doesn't happen at midnight. Is that good enough?
Well, we might imagine the debate continuing with the anti-Lewisian
proposing cases where \emph{D} will not be unique (because of forgotten
similar beliefs) or will not be satisfied (because of a misrecollection
of the circumstances of the belief), and so this approach will fail. And
we might imagine the Lewisian responding by complicating \emph{D}, or by
denying that in these cases we really do have beliefs about our earlier
beliefs. In other words, we can imagine the familiar debates about
descriptivism about names being replayed as debates about descriptivism
about prior beliefs. As enjoyable as that may be, it's interesting to
consider a different approach.

There's a more philosophical reason to worry about Lewis's model. If we
model uncertainty as a class of relationships to possible worlds, it
looks like there's a lot of actual uncertainty we won't be able to
model. Indeed, there are three kinds of uncertainty that we can't model
in this framework. First, we can't model uncertainty about logic and
mathematics. Second, if we accept the necessity of identity, we can't
model uncertainty about identity claims. Whatever it is to be uncertain
about whether \emph{a} is \emph{b}, it won't be a distinctive relation
to the set of worlds in which \emph{a} is \emph{b}, since that's all the
worlds. Third, we can't model uncertainty about claims about
self-identity, like \emph{I'm that guy}. Lewis's framework is an
improvement on the sets of possible worlds approach because it helps
with this third class of cases. But it doesn't help with the first or,
more importantly, with the second. We might think that a solution to
puzzles about self-identity should generalise to solve puzzles about
identity more broadly. Lewis's model doesn't. One of Stalnaker's key
insights is that we should, and can, have a model that addresses both
kinds of puzzles about identity.

On Stalnaker's model, a belief is just a distinction between worlds. The
content of a belief is a set of worlds, not a set of centered worlds.
But worlds have more structure than we thought they had. The formal
model is a bit more subtle than what I'll sketch here, but I think I'll
include enough detail to cover the Sleeping Beauty case. In each world,
each center, in Lewis's sense, has a haecceity. A world is the Cartesian
product of a Lewisian world, i.e.~a world without haecceities, and a
function from each contextually salient haecceity to a location. If we
see a kiss, and wonder who \emph{she} is, who \emph{he} is, and
\emph{when} they are kissing, then we can think of the worlds as
quadruples consisting of a haecceity-free world (perhaps a Lewisian
concreta), a woman, a man and a time. So we can represent three kinds of
locational doubts, not just self-locational doubt.\footnote{Stalnaker
  thinks we have independent reason to treat these structured entities
  as simply worlds. The main point of the last few sentences was that we
  can adopt Stalnaker's model while staying neutral on this metaphysical
  question.}

When an agent at center \emph{c} believes something self-locating,
e.g.~that it is Monday, the content of their belief is that \emph{c}'s
haecceity is on a Monday. If they don't know what day it is, there's a
sense in which they don't know what they believe, since they don't know
whether what they are believing is that \emph{c}'s center is on Monday,
or that some other center's haecceity is on Monday.\footnote{Perhaps it
  would be better to say that individuals and times have haecceities,
  rather than saying centers do. I have little idea what could tell
  between these options, or even if there is a substantive issue here.}
But their belief, the belief they would express on Monday by saying ``It
is Monday'', has two nice features. First, it is neither trivial, like
the belief that \emph{Monday is Monday}, nor changing in value over
time, since \emph{c}'s center is always on Monday. Second, it is the
kind of belief that people on days other than Monday can share, or
dispute. And this belief can be shared by others who have the capacity
to think \emph{de re} about \emph{c}, even if they can't uniquely
describe it. It's this last fact that lets Stalnaker handle the cases
that proved problematic for Lewis and the neo-Lewisians. For instance,
it lets Stalnaker model shared uncertainty about identity claims.

With all that in place, it's time to return to Sleeping Beauty. Let's
consider two propositions. The first, \emph{H}, is that the coin landed
heads. The second, \emph{M}, is what Beauty can express when she wakes
on Monday by saying ``It is Monday''. That is, it is a singular
proposition about a wakening experience that Beauty can now have
singular thoughts about (since she is now undergoing it), but which she
didn't previously have the capacity to determinately pick out. We'll
call this wakening \emph{a}. (Beauty might undergo multiple wakenings,
but we're going to focus on one for now, and call it \emph{a}.) Given
these three propositions, we can describe four possibilities. Or, as
we'll somewhat inaccurately describe them, four worlds.\footnote{Of
  course worlds are considerably more detailed than this, but the extra
  detail is an unnecessary confusion for the current storyline.}

\begin{enumerate}
\def\labelenumi{\arabic{enumi}.}
\tightlist
\item
  H ∧ \emph{M}
\item
  \emph{H} ∧ ¬\emph{M}
\item
  ¬\emph{H} ∧ \emph{M}
\item
  ¬\emph{H} ∧ ¬\emph{M}
\end{enumerate}

On Sunday, Beauty's credences are distributed over the algebra generated
by the partition \{\emph{H}, ¬\emph{H}\}, i.e.,
\{\{\emph{w}\textsubscript{1}, \emph{w}\textsubscript{2}\},
\{\emph{w}\textsubscript{3}, \emph{w}\textsubscript{4}\}\}. The algebra
is that course-grained because she doesn't have the capacity to think
\emph{M} thoughts. And that's because she's not acquainted with the
relevant haecceities. So she can't distinguish between worlds that
differ only on whether \emph{M} is true. On Sunday then, Beauty's
credences are given by Pr(\emph{H}) = Pr(¬H) = ½.

When she wakes on Monday, two things happen. First, she becomes
acquainted with \emph{a}. So she can now think about whether \emph{a} is
on Monday. That is, she can now think about whether \emph{M} is true. So
she can now carve the possibility space more finely. Indeed, now her
credences can be distributed over all propositions built out of the four
possibilities noted above. The second thing that happens is that Beauty
rules out one of these possibilities. In particular, she now knows that
\emph{H} ∧ ¬\emph{M}, a proposition she couldn't so much as think
before, is actually false. That's because if the coin landed heads, this
very wakening could not have taken place on Tuesday.

Stalnaker's position on Beauty's credences uses these two facts. First
Beauty `recalibrates' her credences to the new algebra, then she updates
by conditionalising on ¬\emph{H} ∨ \emph{M}. If after recalibration, her
credences are equally distributed over the four cells of the partition,
the conditionalising on ¬\emph{H} ∨ \emph{M} will move Pr(\emph{H}) to
⅓. That is, the thirders win!

But we might wonder why we use just this calibration, the one where all
four cells get equal credence. We're going to come back to this question
below. But first, I want to use Stalnaker's framework to respond to an
interesting objection to the thirder position.

\section{Monty Hall}\label{monty-hall}

Both C. S. Jenkins (\citeproc{ref-Jenkins2005}{2005}) and Joseph Halpern
(\citeproc{ref-Halpern2004}{2004}) have argued that the `thirder'
solution is undermined by its similarity to fallacious reasoning in the
Monty Hall case. The idea is easy enough to understand if we simply
recall the Monty Hall problem. The agent is in one of three states
\emph{s}\textsubscript{1}, \emph{s}\textsubscript{2} or
\emph{s}\textsubscript{3}, and has reason to believe each is equally
likely. She guesses which one she is in. An experimenter then selects a
state that is neither the state she is in, nor the state she guessed,
and tells her that she is not in that state. If she simply
conditionalises on the content of the experimenter's report, then her
credence that she guessed correctly will go from ⅓ to ½. This is a
bizarre failure of Reflection, so something must have gone
wrong.\footnote{The standard response is to say that the agent shouldn't
  just conditionalise on the content of the experimenter's utterance,
  but on the fact that the experimenter is making just that utterance.
  We'll return to this idea below.} Both Jenkins and Halpern suggest
that the violation of Reflection that `thirders' endorse in Sleeping
Beauty is just as bizarre.

But the Sleeping Beauty puzzle is not analogous to the Monty Hall
problem. That's because in Sleeping Beauty we seem forced to have a
violation of Reflection somewhere. Let's think a bit again about
Beauty's credences on Wednesday, and let's assume that we're trying to
avoid Reflection violations. Then on Monday (and Tuesday) her credence
in \emph{H} is ½. Now when Beauty awakes on those days, there are three
possibilities open to her. (Hopefully it won't lead to ambiguity if I
re-use the name \emph{a} for the awakening Beauty is undergoing when
thinking about \emph{H}.)

\begin{itemize}
\tightlist
\item
  \emph{a} is Monday and \emph{H}
\item
  \emph{a} is Monday and ¬\emph{H}
\item
  \emph{a} is Tuesday and ¬\emph{H}
\end{itemize}

When she wakes on Wednesday, she's in a position to reflect on these
possibilities. And she can rule out the second of them. That's what she
learns when she wakes and learns it is Wednesday; that if ¬\emph{H},
then that last awakening was on Tuesday. Now since that last awakening,
nothing odd has happened to Beauty. She hasn't had her memories erased.
She might have had her memories erased between Monday and Tuesday, but
that's not relevant to the time period she's considering. Moreover, she
knows that she hasn't had her memories erased. So I think she's in a
position to simply conditionalise on her new evidence. And that new
evidence is simply that whatever else was going on when she was thinking
about those three possibilities, she wasn't in the second possibility.

But now we face a challenge. Beauty knows that Wednesday will come. So
if her credence in \emph{H} on Wednesday isn't ½, then we'll have a
violation of Reflection. The violation is that on Sunday her credence in
\emph{H} is ½, but she knows it will go up on Wednesday. And that
violation is just as bad as the violation of Reflection that `thirders'
endorse. But if she conditionalises when she wakes up on Wednesday, then
the only way her updated credence in \emph{H} can be ½ is if her prior
credence in the first and third options above were equal. And the only
way that can happen is for her credence, when \emph{a} is happening, in
the proposition that \emph{a} is Monday and ¬\emph{H} is 0. But that's
bizarre. Whether or not the thirders are right to think that she should
give equal credence to that possibility as to the two others, she can't
give it credence 0. So Reflection will fail somewhere.

To see why Reflection is failing in these cases, it helps to look back
at the requirements we need in order to get from conditionalisation to
Reflection. In Rachael Briggs's careful analysis of when Reflection
holds, in Briggs (\citeproc{ref-Briggs2009}{2009}), Reflection is only
guaranteed to hold when agents know what their evidence is. In other
cases, even perfect conditionalisers may violate Reflection.

This assumption, namely that agents know what their evidence is, is a
kind of luminosity assumption. And not surprisingly, it has been
challenged by Timothy Williamson
(\citeproc{ref-Williamson2000-WILKAI}{Williamson 2000, 230--33}). What
is a little more surprising is that we only need a relatively weak
failure of luminosity in order to get problems for reflection. The
assumption that agents know what their evidence is can be broken into
two parts.

\begin{itemize}
\tightlist
\item
  If \emph{p} is part of \emph{S}'s evidence, then \emph{S} knows that
  \emph{p} is part of her evidence.
\item
  If \emph{p} is not part of \emph{S}'s evidence, then \emph{S} knows
  that \emph{p} is not part of her evidence.
\end{itemize}

The first part is, I think, implausible for reasons familiar from
Williamson's work. But the second is implausible even if one doesn't
like Williamson's style of reasoning. If we think \emph{p} must be true
to be part of \emph{S}'s evidence (as I think we should), and we think
that rational agent's can have false beliefs about anything, as also
seems plausible by simple observation of how easy it is to be misled,
then even a rational agent can fail to realise that \emph{p} is not part
of her evidence. The easiest way that can happen is if she falsely, but
reasonably, believes \emph{p}, and hence does not realise that due to
its falsity, it is not part of her evidence.

Williamson provides an interesting model, based on a discussion in Shin
(\citeproc{ref-Shin1989}{1989}), of a case where an agent does not know
that something is not part of her evidence. There are currently three
possible states the agent could be in: \emph{s}\textsubscript{1},
\emph{s}\textsubscript{2} or \emph{s}\textsubscript{3}. An experiment
will be run, and after the experiment the agent will get some evidence
depending on which state she's in.

\begin{itemize}
\tightlist
\item
  If she's in \emph{s}\textsubscript{1}, her evidence will rule out
  \emph{s}\textsubscript{3}.
\item
  If she's in \emph{s}\textsubscript{2}, her evidence will rule out
  \emph{s}\textsubscript{1} and \emph{s}\textsubscript{3}.
\item
  If she's in \emph{s}\textsubscript{3}, her evidence will rule out
  \emph{s}\textsubscript{1}.
\end{itemize}

Assume the agent knows these conditionals before the experiment is run,
and now let's assume the experiment has been run. Let \emph{xRy} mean
that \emph{y} is possible given the evidence \emph{S} gets in \emph{x}.
Then we can see that \emph{R} is transitive. That means that if \emph{p}
is part of \emph{S}'s evidence, then her evidence settles that \emph{p}
is part of her evidence. But \emph{R} is not Euclidean. So it is
possible that \emph{p} is not part of her evidence, even though her
evidence does not settle that \emph{p} is not part of her evidence. In
particular, if she is in \emph{s}\textsubscript{1}, that she isn't in
\emph{s}\textsubscript{1} is not part of her evidence. But for all she
can tell, she's in \emph{s}\textsubscript{2}. And if she's in
\emph{s}\textsubscript{2}, her evidence does rule out her being in
\emph{s}\textsubscript{1}. So her evidence doesn't settle that this is
not part of her evidence.

The model is obviously an abstraction from any kind of real-world case.
But as we argued above, it is plausible that there are cases where an
agent doesn't know what evidence she \emph{lacks}. And this kind of case
makes for Reflection failure. Assume that the agent's prior credences
are (and should be) that each state is equally likely. And assume the
agent conditionalises on the evidence she gets. Then her credence that
she's in \emph{s}\textsubscript{2} will go up no matter what state she's
in. And she knows in advance this will happen. But there's no obvious
irrationality here; it's not at all clear what kind of
reflection-friendly credal dynamics would be preferably to updating by
conditionalisation.\footnote{The idea that we should update by
  conditionalisation on our evidence, even when we don't know what the
  evidence is, has an amusing consequence in the Monty Hall problem. The
  agent guesses that she's in \emph{s\textsubscript{i}}, and comes to
  know she's not in \emph{s\textsubscript{j}}, where \emph{i} ≠
  \emph{j}. If she only comes to know that she's not in
  \emph{s\textsubscript{j}}, and not something stronger, such as knowing
  that she knows she's not in \emph{s\textsubscript{j}}, then she really
  should conditionalise on this, and her credence that her guess was
  correct will go up. This is the `mistaken' response to the puzzle that
  is frequently deprecated in the literature. But since the orthodox
  solutions to the puzzle rely on the agent reflecting on how she came
  to know ¬\emph{s\textsubscript{j}}, it seems that it is the right
  solution if she doesn't know that she knows
  ¬\emph{s\textsubscript{j}}.}

So when an agent doesn't know what evidence she lacks, Reflection can
fail. One way to think about the Sleeping Beauty case is that something
like this is going on, although it isn't quite analogous to the
Shin-Williamson example discussed above. In that example, the agent
doesn't know what evidence she lacks at the \emph{later} time. In the
Sleeping Beauty case, we can reasonably model Beauty as knowing exactly
what her evidence is when she wakes up. Her evidence does nothing more
or less than rule out \emph{w}\textsubscript{2}. That's something she
didn't know before waking up. But in a good sense she didn't know that
she didn't know that. That's because she was not in a position to even
think about \emph{w}\textsubscript{2} as such. Since she wasn't in a
position to think about \emph{a}, couldn't distinguish, even in thought,
between \emph{w}\textsubscript{1} and \emph{w}\textsubscript{2}. So any
proposition she could think about, and investigate whether she knew or
not, had to include either both \emph{w}\textsubscript{1} and
\emph{w}\textsubscript{2}, or include neither of them. So the only way
she could know that she didn't know \{\emph{w}\textsubscript{1},
\emph{w}\textsubscript{3}, \emph{w}\textsubscript{4}\} is if she tacitly
knew she didn't know that in virtue of knowing that she didn't know
\{\emph{w}\textsubscript{1}, \emph{w}\textsubscript{2},
\emph{w}\textsubscript{3}, \emph{w}\textsubscript{4}\}. But she didn't
know that she didn't know that for the simple reason that she did know
that \{\emph{w}\textsubscript{1}, \emph{w}\textsubscript{2},
\emph{w}\textsubscript{3}, \emph{w}\textsubscript{4}\}, i.e.~the
universal proposition, is true. So we have a case where Beauty doesn't
know what it is she doesn't know at the earlier time. And like cases
where the agent doesn't know what she doesn't know at the later time,
this is a case where reflection fails.

So there are two reasons to be sceptical of reflection-based arguments
against the `thirder' solution to the Sleeping Beauty puzzle.

\begin{itemize}
\tightlist
\item
  There is no plausible way for Beauty's credence in \emph{H} to be ½ on
  both Monday and Wednesday, but reflection requires this.
\item
  Reflection is only plausible when agents know both what evidence they
  have, and what evidence they lack, throughout the story. And it is
  implausible that Beauty satisfies this constraint, since she gains
  conceptual capacities during the story.
\end{itemize}

But this isn't a positive argument for the ⅓ solution. I'll conclude
with a discussion of two arguments for the ⅓ solution. Both arguments
are suggested by Stalnaker's framework, but only one of them is
ultimately defensible.

\section{Stalnaker on Sleeping Ugly}\label{stalnaker-on-sleeping-ugly}

When we left Stalnaker's discussion of the Sleeping Beauty case, we had
just noticed that there was a question about why Beauty should respond
to being able to more finely discriminate between states by
`recalibrating' to a credal state where each of
\emph{w}\textsubscript{1} through \emph{w}\textsubscript{4} receive
equal credence. This question about calibration is crucial to the
Sleeping Beauty puzzle because there are other post-calibration
distributions of credence are are \emph{prima facie} viable. Perhaps,
given what Beauty knows about the setup, she should never have assigned
any credence to \emph{H} ∧ ¬\emph{M}. Rather, she should have made it so
Pr(¬\emph{H} ∧ \emph{M}) = Pr(¬\emph{H} ∧ ¬\emph{M}) = ¼, and
Pr(\emph{H} ∧ \emph{M}) = ½. If she does that, the conditionalising on
¬(\emph{H} ∧ ¬\emph{M}) won't change a thing, and Pr(\emph{H}) will
still be ½. That is, the halfers win!

One argument against this, and in favour of the equally weighted
calibration, is suggested by Stalnaker's `Sleeping Ugly' example.
Sleeping Ugly is woken up on Monday and again (with erased memories) on
Tuesday however the coin lands. So when Ugly awakes, he has the capacity
to think new singular thoughts, but he doesn't get much evidence about
them. In particular, he can't share the knowledge Beauty would express
by saying, ``If the coin landed Heads, this is Monday.''\footnote{Stalnaker
  notes that this is a reason for thinking Beauty does learn something
  when she wakes up, and so there's a reason her credence in \emph{H}
  changes.} Now we might think it is intuitive that Ugly's credences
when he wakes up and reflects on his situation should be equal over the
four possibilities. Moreover, \emph{all} Ugly does is recalibrate; since
he doesn't learn anything about which day it is, his post-awakening
credence just is his recalibration. If all this is correct, and if
Beauty should recalibrate in the same way as Ugly, then Beauty should
recalibrate to the `equally weighted calibration'. And now we're back to
victory for the thirders!

But there's little reason to believe the crucial premise about how Ugly
should recalibrate his credences. What we know is that Ugly doesn't have
any reason to give any more credence to any one of the four
possibilities than to the others. It doesn't at all follow that he has
reason to give equal credence to each, any more than in general an
absence of reasons to treat one of the \emph{x}s differently to the
others is a reason to treat them all the same.\footnote{Compare this
  argument for giving nothing to charity. There are thousands of
  worthwhile charities, and I have no reason to give more to one than
  any of the others. But I can't afford to give large equal amounts to
  each, and if I gave small equal amounts to each, the administrative
  costs would mean my donation has no effect. So I should treat each
  equally, and the only sensible practical way to do this is to give
  none to each. Note that you really don't have to think one charity is
  more worthy than the others to think this is a bad argument; sometimes
  we just have to make arbitrary choices.}

The argument I'm considering here is similar to reasoning Adam Elga has
employed Elga (\citeproc{ref-Elga2004}{2004}), and which I have
criticised Weatherson (\citeproc{ref-Weatherson2005-WEACWD}{2005}). A
central focus of my criticism was that this kind of reasoning has a
tendency to lead to countable additivity violations. In an important
recent paper, Jacob Ross (\citeproc{ref-Ross2010}{2010}) has shown that
many thirder arguments similarly lead to countable additivity
violations. He shows this by deriving what he calls the `Generalised
Thirder Principle' (hereafter, GTP) from the premises of these
arguments. The GTP is a principle concerning a generalised version of
the Sleeping Beauty problem. Here is Ross's description of this class of
problems.

\begin{quote}
Let us define a \emph{Sleeping Beauty problem} as a problem in which a
fully rational agent, Beauty, will undergo one or more mutually
indistinguishable awakenings, and in which the number of awakenings she
will undergo is determined by the outcome of a random process. Let
\emph{S} be a partition of alternative hypotheses concerning the outcome
of this random process. Beauty knows the objective chances of each
hypothesis in \emph{S}, and she also knows how many time she will awaken
conditional on each of these hypotheses, but she has no other relevant
information. The problem is to determine how her credence should be
divided among the hypotheses in \emph{S} when she first awakens. (Ross
ms, 2-3)
\end{quote}

The GTP is a principle about this general class of problem. Here's how
Ross states it.

\begin{description}
\tightlist
\item[Generalized Thirder Principle]
In any standard Sleeping Beauty problem, upon first awakening, Beauty's
credence in any given hypothesis in \emph{S} must be proportional to the
product of the hypothesis' objective chance and the number of times
Beauty will awaken conditional on this hypothesis. \ldots{} {[}We can{]}
express this principle formally. For any hypothesis \emph{i} ∈ \emph{S},
let \emph{Ch}(\emph{i}) be the objective chance that hypothesis \emph{i}
is true, and let \emph{N}(\emph{i}) be the number of times Beauty
awakens if \emph{i} is true. Let \emph{p} be the Beauty's credence
function upon first awakening. The GTP states \ldots{}
\end{description}

\[
\text{For all }i, j ∈ S, \frac{P(i)}{P(j)} = \frac{N(i)Ch(i)}{N(j)Ch(j)} \text{ whenever }Ch(j) >  0. \text{ (Ross ms, 6-7)}
\]

The argument I'm considering seems to be committed to the GTP. In a
generalised Sleeping Beauty problem, we can imagine a version of
Sleeping Ugly who will awake every day that Beauty might awake. The
reasoning that leads one to think that Ugly should give equal credence
to each of the two days in the original Sleeping Beauty case seems to
generalise to imply that Ugly should give equal credence to each day in
this more general case. But if in the general example Beauty calibrates
to match these credences of Ugly, then conditionalises on the
information she receives, then she'll end up endorsing the GTP. That's
an unhappy outcome. It would be better to have an argument for the ⅓
solution that doesn't imply the GTP.

I'm going to argue that when Beauty wakes up her credences should
satisfy the following two premises. (As always, I use \emph{a} to name
the awakening that Beauty is now undergoing, and I'm using \emph{Cr} for
her credence function on waking.)

\begin{enumerate}
\def\labelenumi{\arabic{enumi}.}
\tightlist
\item
  \emph{Cr}(\emph{a} is Monday and \emph{H}) = \emph{Cr}(\emph{a} is
  Tuesday and ¬\emph{H})
\item
  \emph{Cr}(\emph{a} is Monday and \emph{H}) = \emph{Cr}(\emph{a} is
  Monday and ¬\emph{H})
\end{enumerate}

These constraints imply, given what Beauty knows about the setup, that
\emph{Cr}(\emph{H}) = ⅓. The arguments for each premise are quite
different.

The argument for P1 is one I mentioned above, so I'll just sketch it
quickly here. On Wednesday, Beauty's credence in \emph{H} should be back
to ½. But what she learns on Wednesday is ¬(\emph{a} is on Monday and
¬\emph{H}). So on Monday, her credence in \emph{H} conditional on
¬(\emph{a} is on Monday and ¬\emph{H}) should be ½. But given what
Beauty knows about the setup of the problem, this immediately implies
P1.

The argument for P2 requires a slightly more fanciful version of the
example. Imagine that on Sunday night, Beauty is visited by a time
traveller from Monday who comes back with a videotape of her waking on
Monday, and tells her that it was taken on Monday. So Beauty now has the
capacity to think about this very awakening, i.e., \emph{a}. This
doesn't seem to affect her credences in \emph{H}, it should still be ½.
Now imagine that her memory of this visit is erased overnight, so when
she wakes up on Monday her situation is just like in the original
Sleeping Beauty problem.

Call \emph{Cr}\_1 her credence function on Sunday after meeting the time
traveller. And call Cr\textsubscript{2} her credence function on Monday
after she wakes up and reflects on her situation. It seems the only
relevant difference between the situation on Sunday and the situation on
Monday is that Beauty has \emph{lost} the information that \emph{a} is
on Monday. The following principle about situations where an agent loses
information seems plausible. If \emph{Cr}\textsubscript{old} is the
pre-loss credence function, and \emph{Cr}\textsubscript{new} is the
post-loss credence function, and \emph{E} is the information lost, then

\begin{quote}
\emph{Cr}\textsubscript{old}(\emph{p}) =
\emph{Cr}\textsubscript{new}(\emph{p}~\textbar~\emph{E})
\end{quote}

The idea here is that information loss is a sort of reverse
conditionalisation. Applying this, we get that
\emph{Cr}\textsubscript{1}(\emph{H}) =
\emph{Cr}\textsubscript{2}(\emph{H}~\textbar~\emph{a} is Monday), so
\emph{Cr}\textsubscript{2}((\emph{H}~\textbar~\emph{a} is Monday\}) = ½,
so \emph{Cr}\textsubscript{2}(\emph{a} is Monday and \emph{H}) =
\emph{Cr}\textsubscript{2}(\emph{a} is Monday and ¬\emph{H}). And since
the situation on Monday in the revised problem, i.e., the situation when
Beauty's credence function is \emph{Cr}\textsubscript{2} is just like
the situation in the original Sleeping Beauty problem on Monday, it
follows that P1 is true in the original problem. And from P1 and P2, it
follows that the thirder solution is right.

But note a limitation of this solution. When Beauty wakes on
\emph{Tuesday} her credence function is defined over a different algebra
of propositions to what it was defined over after meeting the time
traveller. So there's no time travel based argument that her credences
on Tuesday should satisfy P2, or indeed that on Tuesday her credence in
\emph{H} should be ⅓. (For similar reasons, this kind of reason does not
support the GTP.)

One might try and argue that Beauty's situation on Tuesday is
indistinguishable from her situation on Monday, and so she should have
the same credences on Tuesday. Both the premise and the inference here
seem dubious. On Tuesday, Beauty knows different singular propositions,
so the situation isn't clearly indistinguishable. But more importantly,
it is implausible that indistinguishability implies same credences. The
relation \emph{should have the same credences in} is a transitive and
symmetric relation between states. The relation \emph{is
indistinguishable from} is neither transitive nor symmetric. So I
suspect that the kind of arguments developed here leave it an open
question what Beauty's credences should be on Tuesday, and indeed
whether there is a unique value for what her credences then should be.

\subsection*{References}\label{references}
\addcontentsline{toc}{subsection}{References}

\phantomsection\label{refs}
\begin{CSLReferences}{1}{0}
\bibitem[\citeproctext]{ref-Briggs2009}
Briggs, Ray. 2009. {``Distorted Reflection.''} \emph{Philosophical
Review} 118 (1): 59--85. doi:
\href{https://doi.org/10.1215/00318108-2008-029}{10.1215/00318108-2008-029}.

\bibitem[\citeproctext]{ref-Egan2004-JACSPA-2}
Egan, Andy. 2004. {``Second-Order Predication and the Metaphysics of
Properties.''} \emph{Australasian Journal of Philosophy} 82 (1): 48--66.
doi: \href{https://doi.org/10.1080/713659803}{10.1080/713659803}.

\bibitem[\citeproctext]{ref-Egan2005-EGAEMI}
Egan, Andy, John Hawthorne, and Brian Weatherson. 2005. {``{Epistemic
Modals in Context}.''} In \emph{Contextualism in Philosophy: Knowledge,
Meaning, and Truth}, edited by Gerhard Preyer and Georg Peter, 131--70.
Oxford: Oxford University Press.

\bibitem[\citeproctext]{ref-Elga2000}
Elga, Adam. 2000. {``Self-Locating Belief and the Sleeping Beauty
Problem.''} \emph{Analysis} 60 (4): 143--47. doi:
\href{https://doi.org/10.1093/analys/60.2.143}{10.1093/analys/60.2.143}.

\bibitem[\citeproctext]{ref-Elga2004}
---------. 2004. {``Defeating Dr. Evil with Self-Locating Belief.''}
\emph{Philosophy and Phenomenological Research} 69 (2): 383--96. doi:
\href{https://doi.org/10.1111/j.1933-1592.2004.tb00400.x}{10.1111/j.1933-1592.2004.tb00400.x}.

\bibitem[\citeproctext]{ref-Halpern2004}
Halpern, Joseph. 2004. {``Sleeping Beauty Reconsidered: Conditioning and
Reflection in Asynchronous Systems.''} In \emph{Oxford Studies in
Epistemology}, 1:111--42. Oxford: Oxford University Press.

\bibitem[\citeproctext]{ref-Jenkins2005}
Jenkins, C. S. 2005. {``Sleeping Beauty: A Wake-up Call.''}
\emph{Philosophica Mathematica} 13 (2): 194--201. doi:
\href{https://doi.org/10.1093/philmat/nki015}{10.1093/philmat/nki015}.

\bibitem[\citeproctext]{ref-Lewis1979b}
Lewis, David. 1979. {``Attitudes \emph{de Dicto} and \emph{de Se}.''}
\emph{Philosophical Review} 88 (4): 513--43. doi:
\href{https://doi.org/10.2307/2184646}{10.2307/2184646}. Reprinted in
his \emph{Philosophical Papers}, Volume 1, Oxford: Oxford University
Press, 1983, 133-156. References to reprint.

\bibitem[\citeproctext]{ref-Ross2010}
Ross, Jacob. 2010. {``Sleeping Beauty, Countable Additivity, and
Rational Dilemmas.''} \emph{Philosophical Review} 119 (4): 411--47. doi:
\href{https://doi.org/10.1215/00318108-2010-010}{10.1215/00318108-2010-010}.

\bibitem[\citeproctext]{ref-Shin1989}
Shin, Hyun Song. 1989. {``Non-Partitional Information on Dynamic State
Spaces and the Possibility of Speculation.''} Working Paper 90-11.
Univesity of Michigan: Center for Research on Economic; Social Theory.

\bibitem[\citeproctext]{ref-Stalnaker2008}
Stalnaker, Robert. 2008. \emph{Our Knowledge of the Internal World}.
Oxford: Oxford University Press.

\bibitem[\citeproctext]{ref-Titlebaum2008}
Titlebaum, Michael. 2008. {``The Relevance of Self-Locating Beliefs.''}
\emph{Philosophical Review} 117 (4): 555--605. doi:
\href{https://doi.org/10.1215/00318108-2008-016}{10.1215/00318108-2008-016}.

\bibitem[\citeproctext]{ref-Weatherson2005-WEACWD}
Weatherson, Brian. 2005. {``{Can We Do Without Pragmatic
Encroachment?}''} \emph{Philosophical Perspectives} 19 (1): 417--43.
doi:
\href{https://doi.org/10.1111/j.1520-8583.2005.00068.x}{10.1111/j.1520-8583.2005.00068.x}.

\bibitem[\citeproctext]{ref-Williamson2000-WILKAI}
Williamson, Timothy. 2000. \emph{{Knowledge and its Limits}}. Oxford
University Press.

\end{CSLReferences}



\noindent Published in\emph{
Philosophical Studies}, 2011, pp. 445–456.


\end{document}
