% Options for packages loaded elsewhere
\PassOptionsToPackage{unicode}{hyperref}
\PassOptionsToPackage{hyphens}{url}
%
\documentclass[
  10pt,
  letterpaper,
  DIV=11,
  numbers=noendperiod,
  twoside]{scrartcl}

\usepackage{amsmath,amssymb}
\usepackage{setspace}
\usepackage{iftex}
\ifPDFTeX
  \usepackage[T1]{fontenc}
  \usepackage[utf8]{inputenc}
  \usepackage{textcomp} % provide euro and other symbols
\else % if luatex or xetex
  \usepackage{unicode-math}
  \defaultfontfeatures{Scale=MatchLowercase}
  \defaultfontfeatures[\rmfamily]{Ligatures=TeX,Scale=1}
\fi
\usepackage{lmodern}
\ifPDFTeX\else  
    % xetex/luatex font selection
  \setmainfont[ItalicFont=EB Garamond Italic,BoldFont=EB Garamond
Bold]{EB Garamond Math}
  \setsansfont[]{Europa-Bold}
  \setmathfont[]{Garamond-Math}
\fi
% Use upquote if available, for straight quotes in verbatim environments
\IfFileExists{upquote.sty}{\usepackage{upquote}}{}
\IfFileExists{microtype.sty}{% use microtype if available
  \usepackage[]{microtype}
  \UseMicrotypeSet[protrusion]{basicmath} % disable protrusion for tt fonts
}{}
\usepackage{xcolor}
\usepackage[left=1in, right=1in, top=0.8in, bottom=0.8in,
paperheight=9.5in, paperwidth=6.5in, includemp=TRUE, marginparwidth=0in,
marginparsep=0in]{geometry}
\setlength{\emergencystretch}{3em} % prevent overfull lines
\setcounter{secnumdepth}{3}
% Make \paragraph and \subparagraph free-standing
\ifx\paragraph\undefined\else
  \let\oldparagraph\paragraph
  \renewcommand{\paragraph}[1]{\oldparagraph{#1}\mbox{}}
\fi
\ifx\subparagraph\undefined\else
  \let\oldsubparagraph\subparagraph
  \renewcommand{\subparagraph}[1]{\oldsubparagraph{#1}\mbox{}}
\fi


\providecommand{\tightlist}{%
  \setlength{\itemsep}{0pt}\setlength{\parskip}{0pt}}\usepackage{longtable,booktabs,array}
\usepackage{calc} % for calculating minipage widths
% Correct order of tables after \paragraph or \subparagraph
\usepackage{etoolbox}
\makeatletter
\patchcmd\longtable{\par}{\if@noskipsec\mbox{}\fi\par}{}{}
\makeatother
% Allow footnotes in longtable head/foot
\IfFileExists{footnotehyper.sty}{\usepackage{footnotehyper}}{\usepackage{footnote}}
\makesavenoteenv{longtable}
\usepackage{graphicx}
\makeatletter
\def\maxwidth{\ifdim\Gin@nat@width>\linewidth\linewidth\else\Gin@nat@width\fi}
\def\maxheight{\ifdim\Gin@nat@height>\textheight\textheight\else\Gin@nat@height\fi}
\makeatother
% Scale images if necessary, so that they will not overflow the page
% margins by default, and it is still possible to overwrite the defaults
% using explicit options in \includegraphics[width, height, ...]{}
\setkeys{Gin}{width=\maxwidth,height=\maxheight,keepaspectratio}
% Set default figure placement to htbp
\makeatletter
\def\fps@figure{htbp}
\makeatother
% definitions for citeproc citations
\NewDocumentCommand\citeproctext{}{}
\NewDocumentCommand\citeproc{mm}{%
  \begingroup\def\citeproctext{#2}\cite{#1}\endgroup}
\makeatletter
 % allow citations to break across lines
 \let\@cite@ofmt\@firstofone
 % avoid brackets around text for \cite:
 \def\@biblabel#1{}
 \def\@cite#1#2{{#1\if@tempswa , #2\fi}}
\makeatother
\newlength{\cslhangindent}
\setlength{\cslhangindent}{1.5em}
\newlength{\csllabelwidth}
\setlength{\csllabelwidth}{3em}
\newenvironment{CSLReferences}[2] % #1 hanging-indent, #2 entry-spacing
 {\begin{list}{}{%
  \setlength{\itemindent}{0pt}
  \setlength{\leftmargin}{0pt}
  \setlength{\parsep}{0pt}
  % turn on hanging indent if param 1 is 1
  \ifodd #1
   \setlength{\leftmargin}{\cslhangindent}
   \setlength{\itemindent}{-1\cslhangindent}
  \fi
  % set entry spacing
  \setlength{\itemsep}{#2\baselineskip}}}
 {\end{list}}
\usepackage{calc}
\newcommand{\CSLBlock}[1]{\hfill\break\parbox[t]{\linewidth}{\strut\ignorespaces#1\strut}}
\newcommand{\CSLLeftMargin}[1]{\parbox[t]{\csllabelwidth}{\strut#1\strut}}
\newcommand{\CSLRightInline}[1]{\parbox[t]{\linewidth - \csllabelwidth}{\strut#1\strut}}
\newcommand{\CSLIndent}[1]{\hspace{\cslhangindent}#1}

\setlength\heavyrulewidth{0ex}
\setlength\lightrulewidth{0ex}
\usepackage[automark]{scrlayer-scrpage}
\clearpairofpagestyles
\cehead{
  Brian Weatherson
  }
\cohead{
  Centrality and Marginalisation
  }
\ohead{\bfseries \pagemark}
\cfoot{}
\makeatletter
\newcommand*\NoIndentAfterEnv[1]{%
  \AfterEndEnvironment{#1}{\par\@afterindentfalse\@afterheading}}
\makeatother
\NoIndentAfterEnv{itemize}
\NoIndentAfterEnv{enumerate}
\NoIndentAfterEnv{description}
\NoIndentAfterEnv{quote}
\NoIndentAfterEnv{equation}
\NoIndentAfterEnv{longtable}
\NoIndentAfterEnv{abstract}
\renewenvironment{abstract}
 {\vspace{-1.25cm}
 \quotation\small\noindent\rule{\linewidth}{.5pt}\par\smallskip
 \noindent }
 {\par\noindent\rule{\linewidth}{.5pt}\endquotation}
\KOMAoption{captions}{tableheading}
\makeatletter
\@ifpackageloaded{caption}{}{\usepackage{caption}}
\AtBeginDocument{%
\ifdefined\contentsname
  \renewcommand*\contentsname{Table of contents}
\else
  \newcommand\contentsname{Table of contents}
\fi
\ifdefined\listfigurename
  \renewcommand*\listfigurename{List of Figures}
\else
  \newcommand\listfigurename{List of Figures}
\fi
\ifdefined\listtablename
  \renewcommand*\listtablename{List of Tables}
\else
  \newcommand\listtablename{List of Tables}
\fi
\ifdefined\figurename
  \renewcommand*\figurename{Figure}
\else
  \newcommand\figurename{Figure}
\fi
\ifdefined\tablename
  \renewcommand*\tablename{Table}
\else
  \newcommand\tablename{Table}
\fi
}
\@ifpackageloaded{float}{}{\usepackage{float}}
\floatstyle{ruled}
\@ifundefined{c@chapter}{\newfloat{codelisting}{h}{lop}}{\newfloat{codelisting}{h}{lop}[chapter]}
\floatname{codelisting}{Listing}
\newcommand*\listoflistings{\listof{codelisting}{List of Listings}}
\makeatother
\makeatletter
\makeatother
\makeatletter
\@ifpackageloaded{caption}{}{\usepackage{caption}}
\@ifpackageloaded{subcaption}{}{\usepackage{subcaption}}
\makeatother
\ifLuaTeX
  \usepackage{selnolig}  % disable illegal ligatures
\fi
\usepackage{bookmark}

\IfFileExists{xurl.sty}{\usepackage{xurl}}{} % add URL line breaks if available
\urlstyle{same} % disable monospaced font for URLs
\hypersetup{
  pdftitle={Centrality and Marginalisation},
  pdfauthor={Brian Weatherson},
  hidelinks,
  pdfcreator={LaTeX via pandoc}}

\title{Centrality and Marginalisation\thanks{Thanks to Herman Cappelen
and Ishani Maitra for many discussions about the material in this
paper.}}
\author{Brian Weatherson}
\date{2014}

\begin{document}
\maketitle
\begin{abstract}
A commentary on Herman Cappelen's ``Philosophy without Intuitions''.
\end{abstract}

\setstretch{1.1}
\section{Welcome to the History of Late Analytic
Philosophy}\label{sec-Intro}

It's a good time to be doing history of late analytic philosophy. There
is flurry of new and exciting work on how philosophy got from the death
pangs of positivism and ordinary language philosophy to where it is
today. Some may see this as a much needed gap in the literature. Indeed,
there are a couple of reasons for scepticism about there being such a
field as history of late analytic philosophy, both of which are
plausible but wrong.

One reason is that it is too recent. But it can't be too recent for
general historical study; there are courses in history departments on
September 11, so it's not like looking at philosophy from thirty to
forty years ago is rushing in where historians fear to tread. And
indeed, if logical positivism could be treated historically in the
1960s, and ordinary language philosophy could be treated historically at
the turn of the century, it seems a reasonable time to look back at the
important works of the 1970s that established the contemporary era in
philosophy.

Another reason is that we all know it so well. We are still so engaged
with the key works by Kripke, Lewis, Burge, Perry, Thomson and so on
that we don't need to also look at them the way we look at Descartes,
Locke and Hume. But this, it turns out, is not true. Books by Daniel
Nolan (\citeproc{ref-Nolan2005}{2005}) and Wolfgang Schwarz
(\citeproc{ref-Schwarz2009}{2009}) changed the way that some
philosophers, even those who knew the Lewisian corpus fairly well,
changed the way they read Lewis. There has also been a minor flurry of
work on how important the Gödel/Schmidt case is to the argument of
\emph{Naming and Necessity} (\citeproc{ref-Devitt2010}{Devitt 2011};
\citeproc{ref-Ichikawa2010}{Ichikawa, Maitra, and Weatherson 2012};
\citeproc{ref-Machery2012}{Machery et al. 2012}).

But that's nothing compared to the bombshell that is \emph{Philosophy
Without Intuitions}. (Cappelen (\citeproc{ref-Cappelen2012}{2012}); all
page citations, unless otherwise noted, to this book.) Herman Cappelen
shows, extremely convincingly to my eyes at least, that intuitions play
a much smaller role in late analytic philosophy than many philosophers
thought. Indeed, there is a lot of textual evidence both for the claim
that intuitions don't do much philosophical work, and for the claim that
many people have said that they do. The first of these claims is all to
the good, says Cappelen, since there isn't a particularly good
epistemological defence of the use of intuitions.

The evidence for Cappelen's claims comes in two parts. The first part,
which I won't discuss much here, is an extended argument that words like
`intuitively', or `counterintuitive', as they appear in philosophical
discourse, don't in general function to pick out, or even draw attention
to, any distinctive kind of mental state we could call an `intuition'.
The second part argues that when we look at the actual introduction of
thought experiments into late analytic philosophy, we don't see the
appeal to intuitions that many philosophers seem to think go along with
thought experiments. Rather, we see a whole host of interesting
philosophical moves. Sometimes a thought experiment functions to
highlight an explanandum. Sometimes it gives us a \emph{prima facie}
plausible thesis that we then argue for (or against) at great length.
Sometimes it just raises a puzzle.

One upshot of this historical work, one that Cappelen I think does a
good job highlighting, is that contemporary philosophy is much more
\emph{interesting} than its practitioners sometimes take it to be.
Philosophy is a way of investigating hard questions about the world,
often at great expense in terms of human capital, but with thankfully
little in the way of other expenses. It isn't a matter of tidying up
conceptual space. Thinking of philosophy this way should, I think, help
us see why so many different kinds of projects are philosophically
important.

\section{Centrality and Its Discontents}\label{sec-Centrality}

The big goal of Cappelen's book is to refute the view, which he dubs
Centrality, that intuitions (of a certain kind) are central to analytic
philosophy, and in particular that they are a primary source of evidence
for analytic philosophers. The intuitions that he has in mind have these
three characteristics. (The quotes are from pages 112-3, where these
features are articulated.)

\begin{description}
\tightlist
\item[F1: Phenomenology]
``An Intuitive Judgment has a distinctive phenomenology'' .
\item[F2: Rock]
``An intuitive judgment has a special epistemic status \ldots Intuitive
judgments justify, but they need no justification''.
\item[F3: Conceptual]
A judgment is an intuition ``only if it is justified solely by the
subjects' conceptual competence''.
\end{description}

There's some more detail on F2, but we'll get to that in
Section~\ref{sec-Fragile}. And there's a fourth characteristic of
intuitions that I want to add.

\begin{description}
\tightlist
\item[F4: Speed]
Intuitions are rapid reactions..\footnote{My own views about the
  importance of this, as well as much else in this paper, owe a lot to
  Jennifer Nagel (\citeproc{ref-Nagel2007}{2007},
  \citeproc{ref-Nagel2013}{2013}).}
\end{description}

I'm going to spend much of this paper defending a view that intuitions
characterised by F2 and F4 do play a role, though perhaps not a
\emph{central} role, in philosophy. But I do think that intuitions
characterised by F1 and F3 are just not important to philosophy. Indeed,
I think it's a very important fact that they are not that important.

The claim that intuitions have a distinctive phenomenology is mostly
harmless but, it seems to me, false. I certainly don't find anything in
common when I introspect my judgments that, say, no set is a member of
itself, or that losing a limb would seriously reduce my happiness, or
that the only language I think in is English. It will fall out of the
view I'm defending that the best intuitions have no phenomenology, but I
don't think that's a particularly important fact about them.

But the claim that intuitions derive solely from conceptual
competencies, plus the claim that these are the central source of
evidence in philosophy, is both wrong and dangerous. If that conjunction
were true, we'd expect most philosophical conclusions to be conceptual
truths (whatever those are). I'm not going to take a stand on whether
there are conceptual truths, but I think it is pretty obvious that
conceptual truths won't help much resolve the following debates.
(Compare the list E1-E6 on pages 200-201, which I'm basically just
extending.)

\begin{itemize}
\tightlist
\item
  Do bans on pornography involve trading off speech rights versus
  welfare considerations, or do they just involve evaluating the free
  speech interests of different groups?
\item
  Is it permissible to eat whales?
\item
  Under what circumstances is it permissible to end a terminally ill
  patient's life, or to withhold life-saving treatment?
\item
  Is all context dependency in language traceable to the presence of
  bindable variables?
\item
  Does belief have a phenomenology?
\item
  Which animals (and which non-animals) have beliefs?
\end{itemize}

If philosophy uses largely conceptual evidence, these aren't
philosophical questions. More generally, if Centrality (in Cappelen's
sense) is true of philosophy, then feminist philosophy, legal
philosophy, political philosophy, bioethics, philosophy of language and
(most of) philosophy of mind are not part of philosophy. (This list is
far from exhaustive; making philosophy Centrality-friendly would involve
writing out huge swathes of the discipline.)

Modus tollens obviously beckons. But as Cappelen notes (213), one
occasional reaction to this is to identify certain parts of philosophy
as the `Core' of the discipline, and say Centrality is true of those. If
Centrality is true of the core of philosophy, then feminist philosophy
\emph{etc}., are not part of the core of the field. Maybe now some
people would be disposed to use modus ponens not modus tollens.

That would be a large mistake. It would have shocked Plato, and Locke,
and Hume, and practically every other major figure in the history of
philosophy to learn that political philosophy wasn't central to the
field. I do think (contra some of what Cappelen says) that some
philosophy involves a priori and conceptual investigation. Indeed, I
even do some of it. But it's not true that when I'm doing that I'm doing
work that's deeper, or more philosophical, or more central to philosophy
than the work that, for example, Rae Langton or Susan Moller Okin or
Tamar Szabó Gendler or Sarah-Jane Leslie do.

This reason alone suffices for me to hope that Cappelen's book has a
very wide readership. Centrality isn't true, but it is I think widely
believed to true of at least some parts of the field. (Cappelen quotes
many people endorsing this view.) I suspect that on the basis of this
mistake, the parts of philosophy about which Centrality is not obviously
false (especially metaphysics and epistemology) have been seen as more
central to the discipline than they really ought to be. That's not a bad
state of affairs for metaphysicians and epistemologists, but it's not
good for philosophy, and I hope that Cappelen's book helps put a stop to
it.

\section{Intuitions in Detective Work}\label{sec-Detective}

Despite my very broad sympathy with Cappelen's project, I do think
there's a role for intuitions of some kind in philosophy. Just what this
kind is, and what this role is, will take some spelling out to avoid
Cappelen's arguments. So that's what I'll do for the next few pages.

The intuitions I have in mind are characterised by F2 and F4; they are
default justified, and they are fast. Here's how I think these kinds of
intuitions could matter philosophically.

When humans are growing up, they develop a lot of cognitive skills. Some
of these skills are grounded in specific bits of propositional
knowledge. We learn to count in part by learning that 2 comes after 1,
and 3 comes after 2, and so on. But not all of them are. We learn how to
tell causation from correlation, at least in simple cases, by developing
various heuristics, none of which come close to a full theory of
causation. Indeed, none of these heuristics would even be true, if
stated as universal generalisations. But this ability to pick out which
of the many predecessors of an event is its cause is one we develop very
early (\citeproc{ref-Gopnik2009}{Gopnik 2009, 33--44}), and it is vital
to navigating the world.

I think we develop a lot of skills like that; skills which either go
beyond our propositional knowledge, or at the very least are hard to
articulate in terms of propositions. That we have these kinds of skills
should hardly be news to philosophers; under the label `heuristics' they
have become quite familiar thanks to the work of, among others, Daniel
Kahneman. They occasionally get a bad press, because one central way in
which psychologists detect them is by seeing where they lead to errors
that careful thought would correct. (For instance, our heuristics
sometimes say that a conjunction is more probable than one of the
conjuncts, and careful thinking would correct this.) But this should not
blind us to the fact that these incredibly fast heuristics are often
very reliable; reliable enough to be an independent check on our
theorising.

The use of the term `intuition' to pick out these heuristics isn't
particularly idiosyncratic; Kahneman (\citeproc{ref-Kahneman2011}{2011})
himself moves back and forth freely between the two terms. He
approvingly cites Herbert Simon's remark that ``intuition is nothing
more and nothing less than recognition'', which I think is basically
right. We intuit that \emph{a} is \emph{F} by recognising that it has
the tell-tale signs of \emph{F}hood. Of course we're a million miles
from conceptual or \emph{a priori} reasoning here; as I said, I agree
entirely with Cappelen that F3 is not a feature of any philosophically
significant source of evidence. Here are a couple of cases, one real
life and one fictional, that draw out far removed intuitive thinking can
be from a priori or conceptual thinking. The first is from Kahneman's
description of a case reported by Gary Klein
(\citeproc{ref-Klein1999}{1999}); the second is from (Norwegian) crime
novelist Jo Nesbø (\citeproc{ref-Nesbo2009}{2009}). First Kahneman,

\begin{quote}
A team of firefighters entered a house in which the kitchen was on fire.
Soon after they started hosing down the kitchen, the commander heard
himself shout ``Let's get out of here!'' without realizing why. The
floor collapsed almost immediately after the firefighters escaped. Only
after the fact did the commander realize that the fire had been
unusually quiet and that his ears had been unusually hot \ldots{} He had
no idea what was wrong, but he knew something was wrong.
(\citeproc{ref-Kahneman2011}{Kahneman 2011, 11})
\end{quote}

Now Nesbø. In the story, Harry is the hero, Harry Hole, and Beate is a
talented forensic detective.

\begin{quote}
`Forget what you have or haven't got,' Harry said. `What was your first
impression? Don't think, speak.'

Beate smiled. She knew Harry now. First, intuition, then the facts.
Because intuition provides facts too; it's all the information the crime
scene gives you, but which the brain cannot articulate straight off.
(\citeproc{ref-Nesbo2009}{Nesbø 2009, 126})
\end{quote}

There's at least a family resemblance between Harry Hole's instruction
here and Lewis's instruction to his readers at the start of ``Elusive
Knowledge'' (\citeproc{ref-Lewis1996b}{Lewis 1996}).

\begin{quote}
If you are a contented fallibilist, I implore you to be honest, be
naive, hear it afresh. `He knows, yet he has not eliminated all
possibilities of error.' Even if you've numbed your ears, doesn't this
overt, explicit fallibilism still sound wrong?
(\citeproc{ref-Lewis1996b}{Lewis 1996, 550})
\end{quote}

Reviewers of Nesbø's books often describe his hero as `intuitive'.
That's a little misleading; Harry Hole thinks intuition has a key role
to play in detective work, but the adjective suggests that he relies
heavily on his own intuition. That's not right; he's just as often
badgering his colleagues to give him their impressions of a crime scene,
or an interview subject. In these scenes he reminds me of no one so much
as a colleague constantly wanting to know what one thinks about some
thought experiment or variation on a familiar case. (These are often the
best kind of colleague - full of inspiring ideas!)

So I think a lot of philosophical progress is made by drawing on, and
drawing out, these skills. But isn't this just to say something
uncontroversial and uninteresting, namely that philosophy relies on
implicit knowledge? As Cappelen puts it,

\begin{quote}
It is not controversial that conversations have propositions in the
common ground. Nor is it controversial that all arguments start with
premises that are not argued for. (155)
\end{quote}

Well, there's something a bit interesting here, namely that the `common
ground' and the `not argued for' premises have much greater overlap in
philosophy than in other fields. A book starting with observations about
the Galápagos Islands starts with premises that are not argued for, but
are asserted on the basis of observations. These premises surely weren't
in the common ground before the `conversation' starts. I'll say more
about this in the next section.

Because first I want to fuss a little about just what `common ground'
is. We'll start with an observation Cappelen makes about the
Ginet/Goldman case of Henry and the fake barns
(\citeproc{ref-Goldman1976}{Goldman 1976}). Many philosophers take it to
be an interesting fact that in one scenario, Henry knows there's a barn,
while in another he does not. Cappelen says that these facts are
``presented as being pre-theoretically in the common ground'' (172).
That seems false at first blush. Before reading Goldman's paper, it's
not clear philosophers are in a position to form singular thoughts about
Henry. That's an uncharitable reading though. A more plausible claim is
to say that we are pre-theoretically disposed to accept some long
sentence that roughly says that an agent in such-and-such scenario knows
there is a barn, while an agent in a slightly different scenario does
not.

We might gloss that last claim as saying that we implicitly knew
something about these scenarios. I'm not sure that's right though. We do
surely have lots of implicit knowledge. I know, and so do you, that the
Sydney Opera House is south of the Royal Albert Hall, even if you'd
never articulated that thought to yourself or another. But do our
dispositions to respond to quite finely drawn, and often reasonably
long, vignettes count as implicit beliefs, or should they count as
things we were in a position to know, but only learned once a
philosopher had done the work of drawing the vignette? I can see merit
in both positions, and don't see firm grounds for preferring one.

Let's introduce some terminology to avoid taking a stance on this
question. Say that a subject has \emph{Socratic knowledge} that \emph{p}
when the following two conditions are met:

\begin{enumerate}
\def\labelenumi{\arabic{enumi}.}
\tightlist
\item
  Once the agent is asked to consider \emph{p} in the right way, they
  will come to know \emph{p}.
\item
  The evidential basis for this knowledge that \emph{p} is not the
  asking itself.
\end{enumerate}

The first clause says that anyone who reacts to a Gettier case with
``Oh, of course that's justified true belief without knowledge'' has
Socratic knowledge that such a case is a counterexample to the JTB
theory of knowledge. And they have this Socratic knowledge before the
case is even raised. The second clause says that if the person reacts
instead with ``Oh, some philosophers use thought experiments that don't
make sense unless you know which cars come from which countries'', that
\emph{won't} count as Socratic knowledge. They would be expressing some
knowledge, to be sure, but the telling of the example would play an
evidential role.

If you are very liberal about which dispositions count as implicit
beliefs, and implicit knowledge, then Socratic knowledge will just be a
special kind of implicit knowledge. But if you think considering
examples can lead to learning new facts, not just drawing out
dispositions, then you will think `Socratic' is like `alleged', a
non-factive modifier. As I've defined it, once you hear Gettier cases
once, that they are counterexamples to the JTB theory ceases to be
Socratic knowledge, and becomes regular knowledge. Note also that we can
make sense of some implicit states being more or less Socratic than
others; some dispositions to assent require very careful work to
trigger.

Why is the class of propositions that we Socratically know so rich and
fertile? It's because of the central role of heuristics in our cognitive
lives. Our interactions with the world don't just furnish us with a set
of truths about the world. They also furnish us with skills that we can
apply to generate more truths. I suspect that something like this
observation is at the heart of the endorsement of F3, that intuitions
reveal conceptual truths. When we intuit that \emph{p}, we don't always
merely recall a prior belief that \emph{p}, or infer \emph{p} from what
we antecedently explicitly knew. But nor do we observe that \emph{p}. So
what is it? It must be something internal, but not memory or inference.
Conceptual competence isn't a bad first guess, but Cappelen shows that
isn't the right answer. I think the right answer has to do with
cognitive skills, i.e., heuristics.

\section{Philosophy: A Negative Characterisation}\label{sec-Negative}

So intuitions matter because they reveal Socratic knowledge, and
Socratic knowledge, when made explicit, is a very good guide to the
world. That implies that intuitions should not be confined to
philosophy. And, indeed, they are not. If an economic theorist claimed
the standard of living among English men was higher in 1915 than in
1935, it would be perfectly reasonable to reply that intuitively that
cannot be right, because in 1915 a rather large number of English men
were living on the Western Front in catastrophically poor conditions.
What is distinctive of philosophy then?

We need to clarify this question before we can answer it. Philosophy is
both a discipline with a history over many millennia, and an
organisational unit inside modern universities. These two things overlap
well, but not perfectly. Once we note that they are distinct, we can
separate out the following three questions.

\begin{enumerate}
\def\labelenumi{\arabic{enumi}.}
\tightlist
\item
  What questions are philosophical questions?
\item
  What questions are, within the academy, primarily addressed by
  researchers in philosophy departments?
\item
  What questions should be, at least within the academy, primarily
  addressed by researchers in philosophy departments?
\end{enumerate}

The three questions don't overlap. When Milton Friedman
(\citeproc{ref-Friedman1953}{1953}) writes about economic methodology, I
think he's addressing a philosophical question, but work like this is,
and probably should be, carried out in economics departments. Questions
about professional ethics are philosophical questions that I think
should be researched in philosophy departments, but in the United States
at least typically receive more attention in professional schools. Let's
focus on the third question; what should a philosophy department do?

My colleagues at Michigan and St Andrews work on an incredibly wide
range of questions, from the interpretation of quantum physics through
history of logic through moral psychology and so on. And I think
philosophy departments should have this range of interests. But what do
all these questions have in common?

It's not anything to do with necessity or a priority. Those categories
seriously cross cut philosophy, as Cappelen points out. Historical
investigations into disputes about the parentage of various
might-have-been-royals, or mathematical investigation into the nature of
the primes are not philosophical, but have to do with necessity and a
priority. Whether there's a language of thought is contingent, a
posteriori, and almost paradigmatically philosophical.

It's not really anything to do with \emph{depth}, at least on a natural
understanding of that. Why pandas have thumbs, and humans have
appendices, turn out to be reasonably deep questions, but they are for
biologists, not philosophers. Under what circumstances is democracy
compatible with a strong executive is, at least to me, an incredibly
deep and important question, but it's a question to be answered,
primarily, in history and political science departments.\footnote{This
  is not to say that political philosophers couldn't help with this
  question. There are lots of questions that should have as their
  research centre some other department, but to which philosophers can
  usefully help. Indeed, the examples from economic methodology and
  evolutionary explanation I just mentioned are two more such questions.}
On the other hand, whether we can tell a plausible supervaluationist
story about belief reports is not particularly deep, but a perfectly
good subject for a philosophical inquiry as in Weatherson
(\citeproc{ref-Weatherson2003-WEAMMP}{2003a}).

Better, I think, is to say that philosophical questions are those where
implicit or Socratic knowledge, including crucially intuitions, can
plausibly play a large role in getting to an answer. Philosophy is a
little recursive, so it includes investigations into its own
investigations, including historical work and metaphilosophical work.
(Two fields which, prior to Cappelen's book, had surprisingly little
interaction.) That's not to say we're always right that Socratic
knowledge can answer the questions philosophy sets. Maybe some questions
in mind and language are best answered with the aid of neurological or
phonological work that requires powerful measuring devices. But the
questions are ones where starting with the knowledge and skills we
already have seems like a plausible starting point, or at least not
entirely crazy. This makes philosophy distinct from, say, history. We
use intuitions in history too, especially intuitions about what explains
what. But we need more; intuition won't help if you want to know how
many troops Henry had at Agincourt.

This hypothesis explains, I think, one of the historically important
facts about philosophy. Philosophy gives birth to disciplines. Physics,
economics, psychology and cognitive science were all, at one time, part
of philosophy. In some cases, the split was very recent. The economics
tripos at Cambridge only split from philosophy in 1903
(\citeproc{ref-Tribe2002}{Tribe 2002}). The \emph{Australasian Journal
of Philosophy} was the \emph{Australasian Journal of Psychology and
Philosophy} until 1946. Why does philosophy give rise to disciplines
like these?

I think having a negative characterisation of philosophy helps explain
it. Philosophy has a lot in common, methodologically, with physics,
economics, psychology and so on. All those fields use intuitions and
other forms of Socratic knowledge. But the other fields use other things
too, especially observation. It's when it becomes clear that armchair
methods play too small a role in the research that the field leaves
philosophy.

Of course, philosophers care more about their questions than their
methods, so when the need for non-armchair methods becomes pressing,
some of the individual philosophers will go along, picking up more and
more observational knowledge and experimental skills. Note how much more
empirical research informs the recent work by (for example) Gilbert
Harman, Kim Sterelny and Peter Carruthers, compared to their earlier
work (\citeproc{ref-Harman1973}{Harman 1973};
\citeproc{ref-Harman2011}{Kilkarni and Harman 2011};
\citeproc{ref-DevittSterelny1987}{Devitt and Sterelny 1987};
\citeproc{ref-Sterelny2012}{Sterelny 2012};
\citeproc{ref-Carruthers1990}{Carruthers 1990},
\citeproc{ref-Carruthers2011}{2011}). From the other direction, our
armchairs come with more knowledge now than they used to, which is
partially why engaging with Laura Ruetsche's work in philosophy of
science requires more empirical knowledge engaging with William
Whewell's (\citeproc{ref-Ruetsche2011}{Ruetsche 2011};
\citeproc{ref-Whewell1840}{Whewell 1840}). But still I think the general
picture holds; a question is fit for philosophy iff it is plausible that
the intuitive, armchair methods which are part of every academic's
toolkit can, on their own, generate serious progress on the question.

\section{Letting Go}\label{sec-Go}

I've said that Lewis's instruction at the start of ``Elusive Knowledge''
is to look to intuitions, not to theoretical beliefs. But that might
involve reading more into Lewis than is really there. What he literally
asks the reader is to not appeal to their preferred theory of knowledge.
Is that the same as an appeal to intuitions?

It need not always be. Sometimes, asking people to let go of their prior
theory involves asking them to engage in a complex cognitive task. In
Meditation One, Descartes has us go through quite a lot of thoughts
before we can be pre-theoretical in the way he wants us to be.

But I don't think that's what's going on with Lewis. For one thing, he
doesn't guide us back to a pre-theoretic naı̈veté the way Descartes does.
But more generally, I think getting snap judgments is a way of letting
go of some prior theories.

The picture I have here, and it is nothing more than a picture, is that
intuitions are judgments delivered by heuristics, heuristics are
deployed by Fodorian modules, and Fodorian modules are informationally
encapsulated (\citeproc{ref-Fodor1983}{J. A. Fodor 1983};
\citeproc{ref-Fodor2000}{J. Fodor 2000}). That is, when we rely on a
heuristic, we don't use all of the information at our disposal. The
classic example of this is eyesight; we may know that there are no
elephants on Market Street in St Andrews, but given the right visual
stimuli, our eyes will still insist that there is an elephant
\emph{right there}. The background theory about the spatial distribution
of elephants isn't encoded into the visual module. More generally, to
rely on a heuristic just is to make a judgment using a part of our mind
that doesn't believe some of the things that we do. And that's good,
because it is a kind of independent check on the beliefs we
have.\footnote{Philosophers sometimes understate the importance of
  independent checks. We can know a scale is working, but if we want to
  check its reliability we don't use it, we use something else. I
  suspect that a certain amount of theory-independence is part of the
  explanation of the value of intuitions.}

But isn't the idea that snap judgments are essential to philosophy
inconsistent with the fact that we work very hard on getting our
examples just right, and (as Cappelen shows), argue at great length over
what to say about various examples? I think it isn't, because there are
two respects in which our practice reveals a sensitivity to snap
judgments, and a respect for their use as a check on theorising.

Let me tell you a small secret. I haven't heard anything that even
sounds like a counterexample to the broadly Stalnakerian theory of
indicative conditionals that I like for about a decade. That's not
because there aren't any intuitive counterexamples. It's just because my
intuitions have been trained to accord with this kind of
theory.\footnote{Relatedly, I haven't seen Liverpool get awarded an
  undeserved free kick for about that long.} So what do I do? Do I give
up on the use of intuitions as a test of theory? No, I ask colleagues
for their intuitions. Sometimes I ask them a lot of different questions,
and sometimes I work rather hard on refining the question, or (when they
sadly disagree with my theory) finding ways to undermine their
intuitions. Given the number of similar questions I get from other
colleagues, I don't think my methodology here is distinctive. In short,
we can work very hard before and after getting the snap judgments, while
giving those judgments a role.

This might be more idiosyncratic, but I also do a bunch of things in
papers to draw out snap judgments. The main idea is to distract the
reader from the fact that they are about to be prompted for an
intuition, one that may not accord with their preferred theory. So I'll
use deliberately absurd props (like Vinny the talking vulture), or start
an example without flagging that it is an example. My favourite move
along these lines is to set up an example in such a way that the example
doesn't make sense unless some theoretical claim I want to argue for is
true. Then, after much discussion of the correct verdict on the case, I
can announce that the very sensibility of the prior discussion is proof
that, at least intuitively, the theory I'm pushing must be true.

We're going to come back to this theme a bit later, because I think it's
rather important. The cases you can remember from papers are probably
not the ones where intuition mattered. The big role for intuition in
philosophy (and in many other disciplines) is in checking the small
steps along the way. That's why I join Cappelen in opposing the
methodological rationalists; I don't think intuitions are distinctive to
philosophy, and these small steps don't have much of a phenomenology.
But that doesn't mean they are unimportant.

\section{Strength and Fragility}\label{sec-Fragile}

One of the big trends in late 20th Century epistemology has been the
separation of two senses of \emph{strength of evidence}. This might mean

\begin{enumerate}
\def\labelenumi{\arabic{enumi}.}
\tightlist
\item
  How strong a doxastic state is supported by the evidence.
\item
  How resilient the force of the evidence is in the face of
  counterevidence.
\end{enumerate}

One thing that conservative epistemologies (e.g., Harman
(\citeproc{ref-Harman1986}{1986})) and dogmatic epistemologies (e.g.,
Pryor (\citeproc{ref-Pryor2000}{2000})) have in common is that sources
which might be very strong in the first sense might be very weak in the
second sense. In particular, there can be sources of evidence that
ground knowledge, and hence be rather strong in tthe first sense, but
easily overturned by conflicting evidence. I prefer to reserve the terms
`strong' and `weak' for the first sense, and use the terms `resilient'
or `fragile' for the presence or absence of the second property. In that
language, the important insight of the conservatives and dogmatists is
that evidence can be strong but fragile.

That's roughly how I think of intuitions -- they are strong but rather
fragile. So they can be unjustified justifiers, which is how I read
Cappelen's feature F2 (i.e., Rock).\footnote{There's an ambiguity in
  Cappelen's text that I'm not sure I'm interpreting the right way.
  Let's that someone intuits that in a particular case, \emph{c} doesn't
  cause \emph{e}. Call the content of that intuition, i.e., what is
  \emph{intuited}, \emph{p\textsubscript{d}}. And call the proposition
  that the person has this intuition, i.e., the event of the intuiting,
  \emph{p\textsubscript{g}}. Plausibly both \emph{p\textsubscript{d}}
  and \emph{p\textsubscript{g}} could be evidence in the right cases,
  though most of the time the salient evidence will be
  \emph{p\textsubscript{d}}. I think \emph{p\textsubscript{d}} can be an
  unjustified justifier in the sense that other beliefs, e.g., that a
  particular theory of causation is false, can be justified on the basis
  of \emph{p\textsubscript{d}}, but no other beliefs the agent has
  justify \emph{p\textsubscript{d}}. But you might want a stronger sense
  of `unjustified', where it means not just not justified by anything
  else, but not justified \emph{at all}. I think in these kinds of
  cases, \emph{p\textsubscript{d}} is justified, just not justified by
  anything else. And the justification is, as I'll get to below, strong
  but fragile. If when Cappelen says that intuitions, according to
  Centrality, are unjustified justifiers he means that the belief that
  \emph{p\textsubscript{d}} is unjustified, then I'm not defending
  Centrality. I just mean that the agent need not have any other mental
  states which justify the belief \emph{p\textsubscript{d}}, or indeed
  any access to anything that justifies \emph{p\textsubscript{d}}. But
  for all that it might be that the belief that
  \emph{p\textsubscript{d}} is justified, and the grounds for the
  justification include what the agent learned about causation as a
  child, plus perhaps her competence in distinguishing causes from
  non-causes.}

Cappelen notes it is hard to tell whether something is being used as a
starting point, or an unjustified justifier, so he gives three
diagnostics for this. I mostly agree with one, and disagree with the
other two. I agree that intuitions are non-inferential, and they aren't
based on any particular experience, which is his criteria F2.1. (Though
they usually are based on experiences taken collectively.) But I would
alter the following suggestion, which he gives as a second diagnostic.

\begin{quote}
\begin{description}
\tightlist
\item[F2.2 Evidence Recalcitrance]
Intuitions are evidence recalcitrant; i.e., holders of them are not
disposed to give them up even when their best arguments for those
intuitions are shown to fail. (Compare pg 112)
\end{description}
\end{quote}

I would rather offer something normative here. What's true of intuitions
is that they might provide a stronger ground for belief than the best
evidence we can offer for them. Compare the case of Gettier. As Cappelen
carefully notes (194n3), Gettier doesn't appeal to a raw intuition. He
gives an argument that his subjects don't know. Unfortunately, it isn't
a compelling argument, since it takes as a premise that we can't get
knowledge from a false belief, and that isn't quite right
(\citeproc{ref-Warfield2005}{Warfield 2005}). But Gettier was, to some
extent, justified in believing these subjects didn't know to a greater
degree than he was justified in believing this argument was sound. And
that, I think, is not uncommon.

This is why I don't think Cappelen's `Rough Guide to Rock Detection'
(121), the third of the diagnostics, is perfectly reliable. He says that
if evidence is given for \emph{p} in a context, that's evidence that
\emph{p} isn't an unjustified justifier in that context. But sometimes
we give arguments for judgments that we think could rest without them.
Compare this little dialogue.

\begin{quote}
A: Is `John happiness' a well-formed sentence?~\\
B: No; it doesn't have a verb.
\end{quote}

Here B gives a judgment, then offers a little argument for it. The
argument has a strong premise, namely that all sentences have verbs.
That's debatable; `Lo, gavagai!' may be a counterexample. But B's
judgment isn't undermined by examples that undermine her argument. As in
the Gettier case, we may give an argument that doesn't capture the full
normative force of the judgment.

To say that intuitions are unjustified justifiers is not to say they are
particularly special. If some conservative or dogmatic epistemology is
true, there will be other unjustified justifiers. And if not, then this
story about intuitions will be pretty implausible.

This picture of intuitions as strong but fragile meshes well, I think,
with the picture from section Section~\ref{sec-Go}. There I said the
important intuitions are the ones you barely notice or remember. That's
because the intuitions are fragile; if you remembered them enough to
argue about them (or experimentally test them), the fragility conditions
had probably been triggered, and the intuition probably wasn't doing
much argumentative work.\footnote{I'm simplifying a little here. My
  preferred position is that intuit\emph{ed}s provide strong but fragile
  evidence, while intuit\emph{ing}s provide weak but resilient evidence.
  The reason this is relevant is related to footnote 7.}

But why not think that intuitions are so fragile that they have no use
in any philosophical debate? This question deserves more space than I
can give it, but here are three sketches of answers.

\begin{enumerate}
\def\labelenumi{\arabic{enumi}.}
\tightlist
\item
  Intuitions might be valuable checks on theory, and might be resilient
  enough to perform a valuable checking role.
\item
  Just like heuristics have characteristic errors, it might be that
  careful reasoning has characteristic errors, and there are cases where
  our first impressions are more reliable. See Gladwell
  (\citeproc{ref-GladwellBlink}{2005}) for a summary of some relevant
  evidence.
\item
  Somewhat surprisingly, there may be cases when it is best to trust the
  less reliable source. The case for this is a bit detailed, and not
  original to me, so I'll just include a brief footnote for those
  interested.\footnote{At one point in Ben Levinstein's doctoral
    dissertation, he considers whether there's a general rule for
    deciding which of two conflicting sources we can trust. There turns
    out to be very little in general one can say. In particular,
    \emph{trust the more reliable source} turns out not in general to be
    good advice. If sources have characteristic errors, it might be that
    given what the two sources have said, it is better on this occasion
    to trust the less reliable source, because the verdicts the sources
    deliver provide evidence that we are seeing one of the
    characteristic errors of the more reliable source. It takes more
    space than I have here to fill in the details of this argument, and
    most of the details I'd include would be Levinstein's not mine. But
    here's the big conclusion. Assume that intuitions are often wrong,
    but rarely dramatically wrong. The reason for that is that
    heuristics are bad at getting things exactly right, and good at
    getting in the ballpark. And that careful reasoning is often right,
    but sometimes dramatically wrong. This is trickier to motivate, but
    I think true. Then when intuition dramatically diverges from theory,
    and we don't have independent reason to think that intuition is
    mistaken about the kind of case that's in question, we should trust
    the intuition more than the theory.}
\end{enumerate}

\section{Some Lewisian Case Studies}\label{sec-Lewis}

I've described one kind of mental state that deserves the name
`intuition', and which could play a role in philosophical activity. But,
as Cappelen presses, we have to work to convert that `could' to a
`does'. Do we really rely in intuitive, or heuristic-driven, judgments
about cases in analytic philosophy?

As Cappelen shows, the answer is ``A lot less than you may have
guessed.'' We argue a lot more than we intuit, especially about the
famous cases.\footnote{There is interesting work to be done on the
  relative role of intuitions and arguments about \emph{principles}, but
  I'm going to leave that for another day, and focus here on cases. The
  principles/cases distinction can be a bit slippery, but paradigm cases
  are easy to identify, and we'll be working with fairly paradigmatic
  cases here.} The bit of analytic philosophy I'm most familiar with is
David Lewis's corpus, and since that doesn't play much of a role in
Cappelen's story, I'll illustrate his point with some examples from it.

Going from memory, I would have guessed the clearest example of a case
refuting a theory was the use of finkish dispositions to refute the
conditional analysis of dispositions. But go to the opening pages of
``Finkish Dispositions'' (\citeproc{ref-Lewis1997b}{Lewis 1997a}), and
you find not an intuition about a case, but an argument that finks are
possible. And even though that argument is followed up with more cases,
Lewis rather explicitly \emph{argues} for his conclusions about each
one. See, for example, the glass loving sorcerer on page 147. Lewis
doesn't avert to an intuition that the loved glass is fragile, rather he
``wield{[}s{]} an assumption that dispositions are an intrinsic
matter.'' (\citeproc{ref-Lewis1997b}{Lewis 1997a, 147})

The discussion of causation turns out to be a little more fertile. From
(the longer version of) ``Causation as Influence'', I count the
following appeals to intuitions about cases.

\begin{itemize}
\tightlist
\item
  The chancy bomb example which shows simple probabilistic analyses of
  indeterministic causation won't work (\citeproc{ref-Lewis2004a}{Lewis
  2004a, 79}).
\item
  The Merlin and Morgana example which shows that trumping is possible,
  and matters for what is the cause (\citeproc{ref-Lewis2004a}{Lewis
  2004a, 81}).
\item
  The variant on Billy and Suzy that raises problems for
  quasi-dependence (\citeproc{ref-Lewis2004a}{Lewis 2004a, 83}).
\item
  The crazed President example which shows that causation by double
  prevention is possible, and that causation is not an intrinsic
  relation (\citeproc{ref-Lewis2004a}{Lewis 2004a, 84}).
\item
  The Frankfurt example which shows we can have causation without
  dependence (\citeproc{ref-Lewis2004a}{Lewis 2004a, 95}).
\end{itemize}

There's a strong sense, I think, in which none of the judgments in these
cases are argued for. Indeed, they arise as \emph{problems} for theories
that are otherwise doing rather well. If there was an argument around,
it would be for the negation of the intuited judgment. So I think
there's a role for intuition here.

But we should not imagine that this is normal for philosophers, or even
for Lewis. Cases, it is true, play a large role in Lewis's writing. But
they are very rarely simple refutations of existing theories. We could
perhaps distinguish four roles that cases play, or perhaps four types of
philosophical cases.

\begin{enumerate}
\def\labelenumi{\arabic{enumi}.}
\tightlist
\item
  Refutation of theories, as in these causation cases.
\item
  Illustrations that help explain what's going on in an argument, as in
  the examples from ``Finkish Dispositions''. For a more extensive
  version of this, see Lewis's version of Puzzling Pierre
  (\citeproc{ref-Lewis1981d}{Lewis 1981}).
\item
  Tools for showing that we must distinguish various concepts, such as
  the discussion of Ned Kelly's proof that there's no honest cop
  (\citeproc{ref-Lewis1988f}{Lewis 1988}).
\item
  Simplified versions of the real world, on which we can test various
  explanatory hypotheses, such as the footy and rugby people in ``Naming
  the Colors'' (\citeproc{ref-Lewis1997c}{Lewis 1997b}).
\end{enumerate}

And that list is probably incomplete. The last is fairly fascinating as
a case study actually.\footnote{See Sugden
  (\citeproc{ref-Sugden2000}{2000}, \citeproc{ref-Sugden2009}{2009}) for
  much more on this use of thought experiments.} Some of you may have
had the following experience when programming, or indeed doing anything
that looks like working with code (such as writing in LaTeX). A bug
arises. It helps to find a minimal example in which the bug arises,
i.e., a smallest program that produces the same bug. This helps you spot
what's going on, and if you still need help, it helps your interlocutors
focus on the central problem. It's important that you haven't changed
the problem; the example must be of the same kind as what you started
with. But the example could be much simpler than the case you're most
interested in. Some philosophy examples are, I suspect, like that. Their
value lies in revealing that some striking feature of reality would
persist even if the world were simpler. So, probably, the explanation of
the feature lies in some respect the real messy world shares with the
simple example world. (Compare Cappelen's discussion of Perry's messy
shopper in section 8.1.)

It is perhaps no coincidence that the easiest place to find examples of
type 1 in Lewis's work is in the papers on causation. Lewis thinks there
is no such thing as causation (\citeproc{ref-Lewis2004a}{Lewis 2004a},
\citeproc{ref-Lewis2004d}{2004b}). Whatever our theory of `causes'
should be, it shouldn't match that verb with a binary property. Rather,
the aim of philosophical work on causation is to give a reductive
analysis of causal thought and talk. In such a project, judgments about
how we use `causes' are more likely to be central.

It's also not coincidental that when an example is central to a paper,
such as the `dishonest' cop and Puzzled Pierre, they really don't look
like type 1. That's one big and important lesson from Cappelen's work.
Philosophers do use examples to refute theories, but they are rarely the
big famous examples. If an example is central to a philosophy paper, it
typically plays one of the other three roles.

\section{Summary}\label{sec-Summary}

Let's take stock. I've argued for the following theses:

\begin{enumerate}
\def\labelenumi{\arabic{enumi}.}
\tightlist
\item
  Socratic knowledge is important to philosophy.
\item
  The distinctive feature of philosophy is that it addresses questions
  that can, at least \emph{prima facie}, be productively worked on while
  relying primarily on Socratic knowledge.
\item
  Intuitions are manifestations of cognitive skills, and much Socratic
  knowledge is constituted by the possession of such cognitive skills.
\item
  Like other forms of Socratic knowledge, intuitions are mostly \emph{a
  posteriori}, and have roles outside philosophy as well as inside it.
\item
  Intuitions are default justified; that is, they can be unjustified
  justifiers.
\item
  This default is very weak; intuitions can easily be overridden by
  other considerations.
\item
  Relatedly, it is rare for any one intuition to be central to a
  philosophical work; philosophical intuitions mostly concern the little
  cases we see along the way to larger projects.
\end{enumerate}

I also hinted at, without developing, an argument for

\begin{enumerate}
\def\labelenumi{\arabic{enumi}.}
\setcounter{enumi}{7}
\tightlist
\item
  The right intuition can stop even a plausible theory dead in its
  tracks; and we have (thanks to Ben Levinstein) a mathematical model
  for why this can be so even if intuitions are much less reliable than
  theories.
\end{enumerate}

I opened with a discussion of why it matters to philosophy's
self-conception that point 4 is correct. Since Cappelen also endorses 4,
I probably don't need to say more about that here. But I think there is
more to say about 7.

The first thing I want to note is that 7 is of course consistent with
Cappelen's textual research on important work in late analytic
philosophy. In just about any thought experiment that you can remember,
the intuitions about it don't carry much philosophical weight in the
work in which it is introduced. The intuitions that matter are the
little ones, the ones that go by so quickly that no one questions them
and are largely forgotten by all but the \emph{cognoscenti} in that
field. Even these intuitions aren't \emph{that} common. There are less
of them in Lewis than I would have guessed.

Still, I disagree with Cappelen that philosophy is without these
intuitions. And so I disagree that there's no role for double checking,
experimentally if need be, whether these intuitions are really
intuitive. If a well run survey showed that most people disagree with
Lewis's judgment about, say, the chancy bombs example, I'd reconsider my
views about probabilistic causation. But I'd be really surprised to see
this.

The second thing to note is that while 7 is true, it's not the case that
intuitions about one case are never central to a philosophical project.
There is one big counterexample: the Gettier literature. Like Cappelen
(194n3), I think this literature is incredibly unrepresentative of
philosophy. And I think that's in part because it was methodologically
flawed. I tried to make this point in an earlier paper
(\citeproc{ref-Weatherson2003-WEAWGA}{Weatherson 2003b}), but I didn't
get it quite right. (What I should have said was more like what Elijah
Chudnoff (\citeproc{ref-Chudnoff2011}{2011}) does say.) When we saw the
Gettier example, this should have been an invitation to try and find out
what feature of knowledge was driving the fact that the belief in the
main examples didn't amount to knowledge. Gettier suggested it was
inference from a false premise, but that doesn't quite work
(\citeproc{ref-Warfield2005}{Warfield 2005}). You might think it is
insensitivity, but that doesn't quite work. At this point there should
have been one of two paths taken - attempts to find some other
explanation of the data, or a reconsideration of whether our initial
judgment about the case was wrong. That's what the picture of philosophy
sketched here would have predicted, and (this is the point I was trying
but failing to make in the earlier paper) that's what reflection on our
successes in other areas of philosophy would have recommended. But the
first kind of project ended up intertwined with attempts to analyse
knowledge, and stalled for decades. And the second project wasn't
seriously undertaken, with some honorable exceptions such as Sartwell
(\citeproc{ref-Sartwell1992}{1992}) and Hetherington
(\citeproc{ref-Hetherington2001}{2001}). Now eventually this didn't
matter, because we discovered that safety based explanations of the
Gettier case would work, even if there is no safety based analysis of
knowledge, and even if there is some work to be done in getting the
safety condition just right
(\citeproc{ref-Williamson1994-WILV}{Williamson 1994},
\citeproc{ref-Williamson2000-WILKAI}{2000};
\citeproc{ref-Sainsbury1996}{Sainsbury 1995};
\citeproc{ref-Lewis1996b}{Lewis 1996};
\citeproc{ref-Weatherson2004-WEALMT}{Weatherson 2004}). So if we
strengthened 7 into a universal claim it would be false -- thirty years
of epistemological struggle attest to this. But it was really when
epistemology fell into line with practice in other fields of philosophy
that it made progress on the Gettier case.

\subsection*{References}\label{references}
\addcontentsline{toc}{subsection}{References}

\phantomsection\label{refs}
\begin{CSLReferences}{1}{0}
\bibitem[\citeproctext]{ref-Cappelen2012}
Cappelen, Herman. 2012. \emph{Philosophy Without Intuitions}. Oxford:
Oxford University Press.

\bibitem[\citeproctext]{ref-Carruthers1990}
Carruthers, Peter. 1990. \emph{The Metaphysics of the Tractatus}.
Cambridge: Cambridge University Press.

\bibitem[\citeproctext]{ref-Carruthers2011}
---------. 2011. \emph{The Opacity of Mind: An Integrative Theory of
Self-Knowledge}. Oxford: Oxford University Press.

\bibitem[\citeproctext]{ref-Chudnoff2011}
Chudnoff, Elijah. 2011. {``What Should a Theory of Knowledge Do?''}
\emph{Dialectica} 65 (4): 561--79. doi:
\href{https://doi.org/10.1111/j.1746-8361.2011.01285.x}{10.1111/j.1746-8361.2011.01285.x}.

\bibitem[\citeproctext]{ref-Devitt2010}
Devitt, Michael. 2011. {``Experimental Semantics.''} \emph{Philosophy
and Phenomenological Research} 82 (2): 418--35. doi:
\href{https://doi.org/ppr201182222}{ppr201182222}.

\bibitem[\citeproctext]{ref-DevittSterelny1987}
Devitt, Michael, and Kim Sterelny. 1987. \emph{Language and Reality: An
Introduction to the Philosophy of Language}. Cambridge, MA.: {MIT}
Press.

\bibitem[\citeproctext]{ref-Fodor2000}
Fodor, Jerry. 2000. \emph{The Mind Doesn't Work That Way}. Cambridge,
MA: {MIT} Press.

\bibitem[\citeproctext]{ref-Fodor1983}
Fodor, Jerry A. 1983. \emph{The Modularity of Mind}. Cambridge, MA: MIT
Press.

\bibitem[\citeproctext]{ref-Friedman1953}
Friedman, Milton. 1953. {``The Methodology of Positive Economics.''} In
\emph{Essays in Positive Economics}, 3--43. Chicago: University of
Chicago Press.

\bibitem[\citeproctext]{ref-GladwellBlink}
Gladwell, Malcolm. 2005. \emph{Blink: The Power of Thinking Without
Thinking}. New York: Little, Brown.

\bibitem[\citeproctext]{ref-Goldman1976}
Goldman, Alvin I. 1976. {``Discrimination and Perceptual Knowledge.''}
\emph{The Journal of Philosophy} 73 (20): 771--91. doi:
\href{https://doi.org/10.2307/2025679}{10.2307/2025679}.

\bibitem[\citeproctext]{ref-Gopnik2009}
Gopnik, Alison. 2009. \emph{The Philosophical Baby: What Children's
Minds Tell Us about Truth, Love, and the Meaning of Life}. New York:
Farrar, Straus; Giroux.

\bibitem[\citeproctext]{ref-Harman1973}
Harman, Gilbert. 1973. \emph{Thought}. Princeton: Princeton University
Press.

\bibitem[\citeproctext]{ref-Harman1986}
---------. 1986. \emph{Change in View}. Cambridge, MA: Bradford.

\bibitem[\citeproctext]{ref-Hetherington2001}
Hetherington, Stephen. 2001. \emph{Good Knowledge, Bad Knowledge: On Two
Dogmas of Epistemology}. Oxford: Oxford University Press.

\bibitem[\citeproctext]{ref-Ichikawa2010}
Ichikawa, Jonathan, Ishani Maitra, and Brian Weatherson. 2012. {``In
Defence of a Kripkean Dogma.''} \emph{Philosophy and Phenomenological
Research} 85 (1): 56--68. doi:
\href{https://doi.org/10.1111/j.1933-1592.2010.00478.x}{10.1111/j.1933-1592.2010.00478.x}.

\bibitem[\citeproctext]{ref-Kahneman2011}
Kahneman, Daniel. 2011. \emph{Thinking Fast and Slow}. New York: Farrar,
Straus; Giroux.

\bibitem[\citeproctext]{ref-Harman2011}
Kilkarni, Sanjeev, and Gilbert Harman. 2011. \emph{An Elementary
Introduction to Statistical Learning Theory}. Hoboken, NJ: Wiley.

\bibitem[\citeproctext]{ref-Klein1999}
Klein, Gary A. 1999. \emph{Sources of Power}. Cambridge, MA.: {MIT}
Press.

\bibitem[\citeproctext]{ref-Lewis1981d}
Lewis, David. 1981. {``What Puzzling {P}ierre Does Not Believe.''}
\emph{Australasian Journal of Philosophy} 59 (3): 283--89. doi:
\href{https://doi.org/10.1080/00048408112340241}{10.1080/00048408112340241}.
Reprinted in his \emph{Papers in Metaphysics and Epistemology},
Cambridge: Cambridge University Press, 1999, 408-417. References to
reprint.

\bibitem[\citeproctext]{ref-Lewis1988f}
---------. 1988. {``The Trap's Dilemma.''} \emph{Australasian Journal of
Philosophy} 66 (2): 220--23. doi:
\href{https://doi.org/10.1080/00048408812343301}{10.1080/00048408812343301}.
Reprinted in his \emph{Papers in Ethics and Social Philosophy},
Cambridge: Cambridge University Press, 2000, 95-100. References to
reprint.

\bibitem[\citeproctext]{ref-Lewis1996b}
---------. 1996. {``Elusive Knowledge.''} \emph{Australasian Journal of
Philosophy} 74 (4): 549--67. doi:
\href{https://doi.org/10.1080/00048409612347521}{10.1080/00048409612347521}.
Reprinted in his \emph{Papers in Metaphysics and Epistemology},
Cambridge: Cambridge University Press, 1999, 418-446. References to
reprint.

\bibitem[\citeproctext]{ref-Lewis1997b}
---------. 1997a. {``Finkish Dispositions.''} \emph{The Philosophical
Quarterly} 47 (187): 143--58. doi:
\href{https://doi.org/10.1111/1467-9213.00052}{10.1111/1467-9213.00052}.
Reprinted in his \emph{Papers in Metaphysics and Epistemology},
Cambridge: Cambridge University Press, 1999, 133-151. References to
reprint.

\bibitem[\citeproctext]{ref-Lewis1997c}
---------. 1997b. {``Naming the Colours.''} \emph{Australasian Journal
of Philosophy} 75 (3): 325--42. doi:
\href{https://doi.org/10.1080/00048409712347931}{10.1080/00048409712347931}.
Reprinted in his \emph{Papers in Metaphysics and Epistemology},
Cambridge: Cambridge University Press, 1999, 332-358. References to
reprint.

\bibitem[\citeproctext]{ref-Lewis2004a}
---------. 2004a. {``Causation as Influence.''} In \emph{Causation and
Counterfactuals}, edited by John Collins, Ned Hall, and L. A. Paul,
75--106. Cambridge: {MIT} Press.

\bibitem[\citeproctext]{ref-Lewis2004d}
---------. 2004b. {``Void and Object.''} In \emph{Causation and
Counterfactuals}, edited by John Collins, Ned Hall, and L. A. Paul,
277--90. Cambridge: {MIT} Press.

\bibitem[\citeproctext]{ref-Machery2012}
Machery, Eduoard, Ron Mallon, Shaun Nichols, and Stephen Stich. 2012.
{``If Folk Intuitions Vary, Then What?''} \emph{Philosophy and
Phenomenological Research} 86 (3): 618--35. doi:
\href{https://doi.org/10.1111/j.1933-1592.2011.00555.x}{10.1111/j.1933-1592.2011.00555.x}.

\bibitem[\citeproctext]{ref-Nagel2007}
Nagel, Jennifer. 2007. {``Epistemic Intuitions.''} \emph{Philosophy
Compass} 2 (6): 792--819. doi:
\href{https://doi.org/10.1111/j.1747-9991.2007.00104.x}{10.1111/j.1747-9991.2007.00104.x}.

\bibitem[\citeproctext]{ref-Nagel2013}
---------. 2013. {``Defending the Evidential Value of Epistemic
Intuitions: A Reply to Stich.''} \emph{{P}hilosophy and
{P}henomenological {R}esearch} 86 (1): 179--99. doi:
\href{https://doi.org/10.1111/phpr.12008}{10.1111/phpr.12008}.

\bibitem[\citeproctext]{ref-Nesbo2009}
Nesbø, Jo. 2009. \emph{The Redeemer}. London: Vintage Books.

\bibitem[\citeproctext]{ref-Nolan2005}
Nolan, Daniel. 2005. \emph{David Lewis}. Chesham: Acumen Publishing.

\bibitem[\citeproctext]{ref-Pryor2000}
Pryor, James. 2000. {``The Sceptic and the Dogmatist.''} \emph{No{û}s}
34 (4): 517--49. doi:
\href{https://doi.org/10.1111/0029-4624.00277}{10.1111/0029-4624.00277}.

\bibitem[\citeproctext]{ref-Ruetsche2011}
Ruetsche, Laura. 2011. \emph{Interpreting Quantum Theories}. Oxford:
Oxford University Press.

\bibitem[\citeproctext]{ref-Sainsbury1996}
Sainsbury, Mark. 1995. {``Vagueness, Ignorance and Margin for Error.''}
\emph{British Journal for the Philosophy of Science} 46: 589--601. doi:
\href{https://doi.org/10.1093/bjps/46.4.589}{10.1093/bjps/46.4.589}.

\bibitem[\citeproctext]{ref-Sartwell1992}
Sartwell, Crispin. 1992. {``Why Knowledge Is Merely True Belief.''}
\emph{Journal of Philosophy} 89 (4): 167--80. doi:
\href{https://doi.org/10.2307/2026639}{10.2307/2026639}.

\bibitem[\citeproctext]{ref-Schwarz2009}
Schwarz, Wolfgang. 2009. \emph{David Lewis: Metaphysik Und Analyse}.
Paderborn: Mentis-Verlag.

\bibitem[\citeproctext]{ref-Sterelny2012}
Sterelny, Kim. 2012. \emph{The Evolved Apprentice: How Evolution Made
Humans Unique}. Cambridge, MA.: Bradford.

\bibitem[\citeproctext]{ref-Sugden2000}
Sugden, Robert. 2000. {``Credible Worlds: The Status of Theoretical
Models in Economics.''} \emph{Journal of Economic Methodology} 7 (1):
1--31. doi:
\href{https://doi.org/10.1080/135017800362220}{10.1080/135017800362220}.

\bibitem[\citeproctext]{ref-Sugden2009}
---------. 2009. {``Credible Worlds, Capacities and Mechanisms.''}
\emph{Erkenntnis} 70 (1): 3--27. doi:
\href{https://doi.org/10.1007/s10670-008-9134-x}{10.1007/s10670-008-9134-x}.

\bibitem[\citeproctext]{ref-Tribe2002}
Tribe, Kevin. 2002. {``The Cambridge Economics Tripos 1903--55 and the
Training of Economists.''} \emph{The Manchester School} 68 (2): 222--48.
doi:
\href{https://doi.org/10.1111/1467-9957.00191}{10.1111/1467-9957.00191}.

\bibitem[\citeproctext]{ref-Warfield2005}
Warfield, Ted A. 2005. {``Knowledge from Falsehood.''}
\emph{Philosophical Perspectives} 19: 405--16. doi:
\href{https://doi.org/10.1111/j.1520-8583.2005.00067.x}{10.1111/j.1520-8583.2005.00067.x}.

\bibitem[\citeproctext]{ref-Weatherson2003-WEAMMP}
Weatherson, Brian. 2003a. {``{Many Many Problems}.''} \emph{The
Philosophical Quarterly} 53 (213): 481--501. doi:
\href{https://doi.org/10.1111/1467-9213.00327}{10.1111/1467-9213.00327}.

\bibitem[\citeproctext]{ref-Weatherson2003-WEAWGA}
---------. 2003b. {``{What Good Are Counterexamples?}''}
\emph{Philosophical Studies} 115 (1): 1--31. doi:
\href{https://doi.org/10.1023/A:1024961917413}{10.1023/A:1024961917413}.

\bibitem[\citeproctext]{ref-Weatherson2004-WEALMT}
---------. 2004. {``Luminous Margins.''} \emph{Australasian Journal of
Philosophy} 82 (3): 373--83. doi:
\href{https://doi.org/10.1080/713659874}{10.1080/713659874}.

\bibitem[\citeproctext]{ref-Whewell1840}
Whewell, William. 1840. \emph{The Philosophy of the Inductive Sciences,
Founded Upon Their History}. London: John W. Parker.

\bibitem[\citeproctext]{ref-Williamson1994-WILV}
Williamson, Timothy. 1994. \emph{{Vagueness}}. Routledge.

\bibitem[\citeproctext]{ref-Williamson2000-WILKAI}
---------. 2000. \emph{{Knowledge and its Limits}}. Oxford University
Press.

\end{CSLReferences}



\noindent Published in\emph{
Philosophical Studies}, 2014, pp. 517-533.

\end{document}
