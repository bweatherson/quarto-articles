% Options for packages loaded elsewhere
% Options for packages loaded elsewhere
\PassOptionsToPackage{unicode}{hyperref}
\PassOptionsToPackage{hyphens}{url}
%
\documentclass[
  11pt,
  letterpaper,
  DIV=11,
  numbers=noendperiod,
  twoside]{scrartcl}
\usepackage{xcolor}
\usepackage[left=1.1in, right=1in, top=0.8in, bottom=0.8in,
paperheight=9.5in, paperwidth=7in, includemp=TRUE, marginparwidth=0in,
marginparsep=0in]{geometry}
\usepackage{amsmath,amssymb}
\setcounter{secnumdepth}{3}
\usepackage{iftex}
\ifPDFTeX
  \usepackage[T1]{fontenc}
  \usepackage[utf8]{inputenc}
  \usepackage{textcomp} % provide euro and other symbols
\else % if luatex or xetex
  \usepackage{unicode-math} % this also loads fontspec
  \defaultfontfeatures{Scale=MatchLowercase}
  \defaultfontfeatures[\rmfamily]{Ligatures=TeX,Scale=1}
\fi
\usepackage{lmodern}
\ifPDFTeX\else
  % xetex/luatex font selection
  \setmainfont[ItalicFont=EB Garamond Italic,BoldFont=EB Garamond
SemiBold]{EB Garamond Math}
  \setsansfont[]{EB Garamond}
  \setmathfont[]{Garamond-Math}
\fi
% Use upquote if available, for straight quotes in verbatim environments
\IfFileExists{upquote.sty}{\usepackage{upquote}}{}
\IfFileExists{microtype.sty}{% use microtype if available
  \usepackage[]{microtype}
  \UseMicrotypeSet[protrusion]{basicmath} % disable protrusion for tt fonts
}{}
\usepackage{setspace}
% Make \paragraph and \subparagraph free-standing
\makeatletter
\ifx\paragraph\undefined\else
  \let\oldparagraph\paragraph
  \renewcommand{\paragraph}{
    \@ifstar
      \xxxParagraphStar
      \xxxParagraphNoStar
  }
  \newcommand{\xxxParagraphStar}[1]{\oldparagraph*{#1}\mbox{}}
  \newcommand{\xxxParagraphNoStar}[1]{\oldparagraph{#1}\mbox{}}
\fi
\ifx\subparagraph\undefined\else
  \let\oldsubparagraph\subparagraph
  \renewcommand{\subparagraph}{
    \@ifstar
      \xxxSubParagraphStar
      \xxxSubParagraphNoStar
  }
  \newcommand{\xxxSubParagraphStar}[1]{\oldsubparagraph*{#1}\mbox{}}
  \newcommand{\xxxSubParagraphNoStar}[1]{\oldsubparagraph{#1}\mbox{}}
\fi
\makeatother


\usepackage{longtable,booktabs,array}
\usepackage{calc} % for calculating minipage widths
% Correct order of tables after \paragraph or \subparagraph
\usepackage{etoolbox}
\makeatletter
\patchcmd\longtable{\par}{\if@noskipsec\mbox{}\fi\par}{}{}
\makeatother
% Allow footnotes in longtable head/foot
\IfFileExists{footnotehyper.sty}{\usepackage{footnotehyper}}{\usepackage{footnote}}
\makesavenoteenv{longtable}
\usepackage{graphicx}
\makeatletter
\newsavebox\pandoc@box
\newcommand*\pandocbounded[1]{% scales image to fit in text height/width
  \sbox\pandoc@box{#1}%
  \Gscale@div\@tempa{\textheight}{\dimexpr\ht\pandoc@box+\dp\pandoc@box\relax}%
  \Gscale@div\@tempb{\linewidth}{\wd\pandoc@box}%
  \ifdim\@tempb\p@<\@tempa\p@\let\@tempa\@tempb\fi% select the smaller of both
  \ifdim\@tempa\p@<\p@\scalebox{\@tempa}{\usebox\pandoc@box}%
  \else\usebox{\pandoc@box}%
  \fi%
}
% Set default figure placement to htbp
\def\fps@figure{htbp}
\makeatother





\setlength{\emergencystretch}{3em} % prevent overfull lines

\providecommand{\tightlist}{%
  \setlength{\itemsep}{0pt}\setlength{\parskip}{0pt}}



 


\setlength\heavyrulewidth{0ex}
\setlength\lightrulewidth{0ex}
\usepackage[automark]{scrlayer-scrpage}
\clearpairofpagestyles
\cehead{
  Brian Weatherson
  }
\cohead{
  Review of “Lewisian Themes”
  }
\ohead{\bfseries \pagemark}
\cfoot{}
\makeatletter
\newcommand*\NoIndentAfterEnv[1]{%
  \AfterEndEnvironment{#1}{\par\@afterindentfalse\@afterheading}}
\makeatother
\NoIndentAfterEnv{itemize}
\NoIndentAfterEnv{enumerate}
\NoIndentAfterEnv{description}
\NoIndentAfterEnv{quote}
\NoIndentAfterEnv{equation}
\NoIndentAfterEnv{longtable}
\NoIndentAfterEnv{abstract}
\renewenvironment{abstract}
 {\vspace{-1.25cm}
 \quotation\small\noindent\emph{Abstract}:}
 {\endquotation}
\newfontfamily\tfont{EB Garamond}
\addtokomafont{disposition}{\rmfamily}
\addtokomafont{title}{\normalfont\itshape}
\let\footnoterule\relax

\makeatletter
\renewcommand{\@maketitle}{%
  \newpage
  \null
  \vskip 2em%
  \begin{center}%
  \let \footnote \thanks
    {\itshape\huge\@title \par}%
    \vskip 0.5em%  % Reduced from default
    {\large
      \lineskip 0.3em%  % Reduced from default 0.5em
      \begin{tabular}[t]{c}%
        \@author
      \end{tabular}\par}%
    \vskip 0.5em%  % Reduced from default
    {\large \@date}%
  \end{center}%
  \par
  }
\makeatother
\RequirePackage{lettrine}

\renewenvironment{abstract}
 {\quotation\small\noindent\emph{Abstract}:}
 {\endquotation\vspace{-0.02cm}}
\KOMAoption{captions}{tableheading}
\makeatletter
\@ifpackageloaded{caption}{}{\usepackage{caption}}
\AtBeginDocument{%
\ifdefined\contentsname
  \renewcommand*\contentsname{Table of contents}
\else
  \newcommand\contentsname{Table of contents}
\fi
\ifdefined\listfigurename
  \renewcommand*\listfigurename{List of Figures}
\else
  \newcommand\listfigurename{List of Figures}
\fi
\ifdefined\listtablename
  \renewcommand*\listtablename{List of Tables}
\else
  \newcommand\listtablename{List of Tables}
\fi
\ifdefined\figurename
  \renewcommand*\figurename{Figure}
\else
  \newcommand\figurename{Figure}
\fi
\ifdefined\tablename
  \renewcommand*\tablename{Table}
\else
  \newcommand\tablename{Table}
\fi
}
\@ifpackageloaded{float}{}{\usepackage{float}}
\floatstyle{ruled}
\@ifundefined{c@chapter}{\newfloat{codelisting}{h}{lop}}{\newfloat{codelisting}{h}{lop}[chapter]}
\floatname{codelisting}{Listing}
\newcommand*\listoflistings{\listof{codelisting}{List of Listings}}
\makeatother
\makeatletter
\makeatother
\makeatletter
\@ifpackageloaded{caption}{}{\usepackage{caption}}
\@ifpackageloaded{subcaption}{}{\usepackage{subcaption}}
\makeatother
\usepackage{bookmark}
\IfFileExists{xurl.sty}{\usepackage{xurl}}{} % add URL line breaks if available
\urlstyle{same}
\hypersetup{
  pdftitle={Review of ``Lewisian Themes''},
  pdfauthor={Brian Weatherson},
  hidelinks,
  pdfcreator={LaTeX via pandoc}}


\title{Review of ``Lewisian Themes''}
\author{Brian Weatherson}
\date{2005}
\begin{document}
\maketitle
\begin{abstract}
A review of Frank Jackson and Graham Priest (eds), ``Lewisian Themes:
The Philosophy of David K. Lewis'', Oxford University Press, 2004.
\end{abstract}


\setstretch{1.1}
David Lewis didn't seem particularly fond of festschrifts. He didn't
participate in the one volume of papers on his work published in his
lifetime (\emph{Reality and Humean Supervenience}, Rowman and
Littlefield 2001). He rarely contributed to festschrifts for others, in
recent decades only contributing papers to volumes in honour of his very
closest friends (David Armstrong and Hugh Mellor). I haven't done a
systematic study of this, but my casual reading suggests he didn't often
refer to papers published in festschrifts either. More generally, he
strongly favoured publishing in journals with blind review to publishing
in edited volumes of solicited papers.

So if there was to be a festschrift for David Lewis, as there surely
should be, it should be one put together just the way \emph{Lewisian
Themes} was. Most of the papers for it were selected by blind
refereeing, not by inviting the usual suspects. (And the result was that
several young philosophers, even graduate students, are represented in
the collection.) The festschrift did not start life as this volume, but
as an edition of a journal. And not just any journal, but the journal
that Lewis's work most often graced: the \emph{Australasian Journal of
Philosophy}. This volume reprints the papers from that issue, and adds
four more papers that were left out for space considerations.

The result is a marvellous collection that reflects Lewis's work in both
its quality and its breadth. Given the amount of material herein, I can
do little more than summarise the papers and present a few reflections
on what some of them tell us about the future directions for Lewisian
philosophy. Here is the list of topics covered, each of which is
followed in parentheses by the names of the authors who address that
topic.

\begin{itemize}
\tightlist
\item
  Quantum Mechanics (David Lewis; David Papineau)
\item
  Properties (Andy Egan; Josh Parsons)
\item
  Truth in Fiction (Richard Hanley; Jim Mackenzie)
\item
  Ramseyan Humility (Rae Langton; Jonathan Schaffer)
\item
  Chance (Adam Elga; Ned Hall)
\item
  Mental Content (Robert Stalnaker; Barry Taylor; Alan Hájek and Philip
  Pettit)
\item
  Modal Realism (Kris McDaniel; Barry Taylor; Max Cresswell)
\item
  Counterpart Theory and Essentialism (Max Cresswell; L. A. Paul)
\item
  Similarity (Barry Taylor; David Vander Laan)
\item
  Holes (Roberto Casati and Achille Varzi)
\item
  Truth (Marian David)
\item
  Vagueness (Gideon Rosen and Nicholas J. J. Smith)
\end{itemize}

The breadth of the papers here is astounding. Three of the papers
(Papineau, Langton and Schaffer) respond to material of Lewis's that was
only published posthumously. Summaries of David Lewis's work normally
start by mentioning his modal realism, but this presents a somewhat
distorted picture of his interests and importance. The list here is a
better guide to the range of Lewis's work. (Though as the editors note,
it is quite incomplete because it leaves out Lewis's work on materialism
and on set theory and mereology, not to mention his work on conditionals
and on value theory and many other topics.)

The only paper directly about modal realism is McDaniel's interesting
paper on how we might formulate and defend a version of modal realism
that allows worlds to overlap. Barry Taylor's paper develops some
considerations about simplicity that suggest a worry for Lewis's
argument against linguistic ersatzism: in particular he argues against
the existence of alien universals. And Max Cresswell formulates his
discussion of counterpart theory using modal realist language. And
that's it for the discussion of modal realism, which seems about right
given its relative importance in Lewis's overall picture.

Many of the papers accept the broad outlines of Lewis's views, but
suggest alternative ways of developing the details. For instance, Andy
Egan's paper ``Second Order Predication and the Metaphysics of
Properties'' discusses Lewis's view that properties are sets of
individuals. Egan argues that Lewis should instead have said that
properties are functions from worlds to sets of individuals in that
world. You might think that these two views are equivalent. Let \emph{f}
be a function that Egan calls a property, and consider the set
\{\emph{x}: for some world \emph{w}, \emph{x}~∈~\emph{f}(\emph{w})\}.
Why wouldn't that set be a Lewis-style property? Well, it would be if no
object is in more than one world. Isn't that true on Lewis's
metaphysics? Not quite. Most ordinary individuals are in one world only,
but \emph{properties} are in multiple worlds. If the property
\emph{being red} has some property, say \emph{being instantiated}, in
some worlds but not others, should it be a member of the property
\emph{being instantiated} or not? Egan argues, persuasively, that Lewis
can give no good answer to this question. But on the view that
properties are functions, we can say that the function that we identify
with the property \emph{being instantiated} maps worlds onto a set that
includes \emph{being red} in worlds where there are red things, and a
set that does not include it in worlds where there are no red things. So
this looks like a nice fix to a technical problem for Lewis. Egan goes
on to argue that the change undermines some of Lewis's arguments against
endurantism and modal overlap.

Ned Hall discusses Lewis's widely influential theory of objective
chance. There are two disputes running through Hall's paper: the dispute
between reductionists and anti-reductionists about chance, and the
dispute between the old and new versions of the Principal Principle,
presented here as (O) and (N).

\begin{description}
\tightlist
\item[(O)]
C\textsubscript{0}(A \textbar{} E \& ch\textsubscript{t}(A) = x) = x
\item[(N)]
C\textsubscript{0}(A \textbar{} HL) = ch\textsubscript{t}(A \textbar{}
L)
\end{description}

Here C\textsubscript{0} is an initial credence function of a rational
agent, ch\textsubscript{t} is the chance function at t, H is the history
of the world to t, L is the conjunction of the laws, and E is any
evidence that is `admissible' with respect to A at t. (Evidence is
admissible, roughly, if it doesn't provide any more information about A
than we get from knowing its chance.) Lewis thought it was a cost of his
reductionist position that he was forced to give up (O) and retreat to
(N). Hall suggests that it isn't a cost, it is really something we could
have derived all along. Hall's reasoning is a little hard to follow
here. He shows elegantly that if the reductionist agrees to treat the
chance function as an expert function, then she should accept (N) rather
than (O). (Roughly speaking, to treat a function as an expert function
is, as Hall says on page 101, to consider its opinion so epistemically
superior to yours that you are disposed to defer to it. Before we can
give a formal statement of this we need to make it precise, which may
not be trivial. One of the important innovations of Hall's paper is to
clarify some importantly distinct ways of making it precise, in
particular distinguishing the cases where the expert is better informed
than you from the case where the expert is a better judge of evidence
than you.) But it is not clear how showing that the reductionist is
forced to accept (N) rather than (O), shows that it isn't a \emph{cost}
of reductionism that it is incompatible with (O).

Hall makes two nice points about how (N) stands with respect to the
reductionist/anti‑reductionist debate. First, he notes that it isn't
clear in general how reductionists can derive (N), i.e.~derive the claim
that the chance function \emph{is} an expert function, from their
general principles. Lewis often touted it as a virtue of reductionism
that such a derivation seems possible, but Hall suggests that there will
be problems with the details here. Lewis's optimism rested on cases
where there was a close connection between chance and limiting relative
frequency, but this is not the only kind of case. Hall argues,
persuasively, that this part of the reductionist program will have
difficulties with cases where the chance of an F being a G varies
continuously with some magnitude \emph{m} of the F (say its mass), and
the proposition A concerns whether a particular object \emph{a}, which
is an \emph{F} such that \emph{m}(\emph{a}) is unique, is also a G. (The
general project of trying to incorporate continuously variable
quantities, like masses, into Lewis's metaphysics appears to be very
difficult, and there could be some very interesting philosophical
discoveries to be made in this field.) Second, Hall notes that it is not
as hard for the anti‑reductionist to explain (N) as Lewis suggests. To
be sure, the anti-reductionist can't derive (N) from, say, metaphysical
necessities and logic. But we know that reasonable principles of
inductive rationality won't follow merely from metaphysical necessities
and logic; in addition, we need a number of substantive synthetic
principles. There is no reason why (N), or something that entails (N),
cannot be among these.

The difficulty Lewis imagines for the anti-reductionist here arises
largely from an insufficient attention to epistemology. Lewis never took
epistemology (at least informal epistemology) as seriously as he took
metaphysics, philosophy of mind, or philosophy of language, and
occasionally this shows up in the papers here. Jonathan Schaffer's paper
``Quiddistic Knowledge'' argues that there is no plausible
epistemological theory that grounds Lewis's claim in ``Ramseyan
Humility'' that ``we are irredemiably ignorant about the identities of
the fundamental properties.'' And Robert Stalnaker in ``Lewis on
Intentionality'' argues that Lewis's content internalism derives from an
implausible commitment to epistemological internalism.

There are of course many more good ideas throughout this volume. For
example, Marian David has a thorough, and thoroughly convincing,
response to Lewis's argument that the correspondence theory of truth is
not a competitor to the redundancy theory of truth. Max Cresswell
provides a few reasons for thinking that the counterpart relation has to
be an equivalence relation, in which case we may as well do without
counterpart theory. (Cresswell's paper and the recent paper on a similar
topic by Michael Fara and Timothy Williamson, ``Counterparts and
Actuality'', \emph{Mind} 114: 1-30, complement each other nicely, and
should be required reading for anyone interested in debates about
counterpart theory.) Alan Hájek and Philip Pettit argue that Lewis's
argument against the view that desires are beliefs about what is good
assumes that goodness is not an `indexical' property, and that this
might fail if what is good is partially a function of, say, our
evidence, as on the view that what is good is what maximises
\emph{expected} utility. And there are many more gems like these.

It is a great compliment to Lewis's style as a philosopher that, as we
see in this collection, Lewis amassed a great number of admirers who
thought the best way to honour his work was to criticise it in every way
imaginable. It is sad to stop and think how much we would have learned
from hearing Lewis's responses.

\vspace{1cm}



\noindent Published in \emph{Notre Dame Philosophical Reviews}, 2005 \\
\url{https://ndpr.nd.edu/reviews/lewisian-themes-the-philosophy-of-david-k-lewis/}


\end{document}
