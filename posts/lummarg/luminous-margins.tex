% Options for packages loaded elsewhere
% Options for packages loaded elsewhere
\PassOptionsToPackage{unicode}{hyperref}
\PassOptionsToPackage{hyphens}{url}
%
\documentclass[
  11pt,
  letterpaper,
  DIV=11,
  numbers=noendperiod,
  twoside]{scrartcl}
\usepackage{xcolor}
\usepackage[left=1.1in, right=1in, top=0.8in, bottom=0.8in,
paperheight=9.5in, paperwidth=7in, includemp=TRUE, marginparwidth=0in,
marginparsep=0in]{geometry}
\usepackage{amsmath,amssymb}
\setcounter{secnumdepth}{3}
\usepackage{iftex}
\ifPDFTeX
  \usepackage[T1]{fontenc}
  \usepackage[utf8]{inputenc}
  \usepackage{textcomp} % provide euro and other symbols
\else % if luatex or xetex
  \usepackage{unicode-math} % this also loads fontspec
  \defaultfontfeatures{Scale=MatchLowercase}
  \defaultfontfeatures[\rmfamily]{Ligatures=TeX,Scale=1}
\fi
\usepackage{lmodern}
\ifPDFTeX\else
  % xetex/luatex font selection
  \setmainfont[ItalicFont=EB Garamond Italic,BoldFont=EB Garamond
Bold]{EB Garamond Math}
  \setsansfont[]{EB Garamond}
  \setmathfont[]{Garamond-Math}
\fi
% Use upquote if available, for straight quotes in verbatim environments
\IfFileExists{upquote.sty}{\usepackage{upquote}}{}
\IfFileExists{microtype.sty}{% use microtype if available
  \usepackage[]{microtype}
  \UseMicrotypeSet[protrusion]{basicmath} % disable protrusion for tt fonts
}{}
\usepackage{setspace}
% Make \paragraph and \subparagraph free-standing
\makeatletter
\ifx\paragraph\undefined\else
  \let\oldparagraph\paragraph
  \renewcommand{\paragraph}{
    \@ifstar
      \xxxParagraphStar
      \xxxParagraphNoStar
  }
  \newcommand{\xxxParagraphStar}[1]{\oldparagraph*{#1}\mbox{}}
  \newcommand{\xxxParagraphNoStar}[1]{\oldparagraph{#1}\mbox{}}
\fi
\ifx\subparagraph\undefined\else
  \let\oldsubparagraph\subparagraph
  \renewcommand{\subparagraph}{
    \@ifstar
      \xxxSubParagraphStar
      \xxxSubParagraphNoStar
  }
  \newcommand{\xxxSubParagraphStar}[1]{\oldsubparagraph*{#1}\mbox{}}
  \newcommand{\xxxSubParagraphNoStar}[1]{\oldsubparagraph{#1}\mbox{}}
\fi
\makeatother


\usepackage{longtable,booktabs,array}
\usepackage{calc} % for calculating minipage widths
% Correct order of tables after \paragraph or \subparagraph
\usepackage{etoolbox}
\makeatletter
\patchcmd\longtable{\par}{\if@noskipsec\mbox{}\fi\par}{}{}
\makeatother
% Allow footnotes in longtable head/foot
\IfFileExists{footnotehyper.sty}{\usepackage{footnotehyper}}{\usepackage{footnote}}
\makesavenoteenv{longtable}
\usepackage{graphicx}
\makeatletter
\newsavebox\pandoc@box
\newcommand*\pandocbounded[1]{% scales image to fit in text height/width
  \sbox\pandoc@box{#1}%
  \Gscale@div\@tempa{\textheight}{\dimexpr\ht\pandoc@box+\dp\pandoc@box\relax}%
  \Gscale@div\@tempb{\linewidth}{\wd\pandoc@box}%
  \ifdim\@tempb\p@<\@tempa\p@\let\@tempa\@tempb\fi% select the smaller of both
  \ifdim\@tempa\p@<\p@\scalebox{\@tempa}{\usebox\pandoc@box}%
  \else\usebox{\pandoc@box}%
  \fi%
}
% Set default figure placement to htbp
\def\fps@figure{htbp}
\makeatother


% definitions for citeproc citations
\NewDocumentCommand\citeproctext{}{}
\NewDocumentCommand\citeproc{mm}{%
  \begingroup\def\citeproctext{#2}\cite{#1}\endgroup}
\makeatletter
 % allow citations to break across lines
 \let\@cite@ofmt\@firstofone
 % avoid brackets around text for \cite:
 \def\@biblabel#1{}
 \def\@cite#1#2{{#1\if@tempswa , #2\fi}}
\makeatother
\newlength{\cslhangindent}
\setlength{\cslhangindent}{1.5em}
\newlength{\csllabelwidth}
\setlength{\csllabelwidth}{3em}
\newenvironment{CSLReferences}[2] % #1 hanging-indent, #2 entry-spacing
 {\begin{list}{}{%
  \setlength{\itemindent}{0pt}
  \setlength{\leftmargin}{0pt}
  \setlength{\parsep}{0pt}
  % turn on hanging indent if param 1 is 1
  \ifodd #1
   \setlength{\leftmargin}{\cslhangindent}
   \setlength{\itemindent}{-1\cslhangindent}
  \fi
  % set entry spacing
  \setlength{\itemsep}{#2\baselineskip}}}
 {\end{list}}
\usepackage{calc}
\newcommand{\CSLBlock}[1]{\hfill\break\parbox[t]{\linewidth}{\strut\ignorespaces#1\strut}}
\newcommand{\CSLLeftMargin}[1]{\parbox[t]{\csllabelwidth}{\strut#1\strut}}
\newcommand{\CSLRightInline}[1]{\parbox[t]{\linewidth - \csllabelwidth}{\strut#1\strut}}
\newcommand{\CSLIndent}[1]{\hspace{\cslhangindent}#1}



\setlength{\emergencystretch}{3em} % prevent overfull lines

\providecommand{\tightlist}{%
  \setlength{\itemsep}{0pt}\setlength{\parskip}{0pt}}



 


\setlength\heavyrulewidth{0ex}
\setlength\lightrulewidth{0ex}
\usepackage[automark]{scrlayer-scrpage}
\clearpairofpagestyles
\cehead{
  Brian Weatherson
  }
\cohead{
  Luminous Margins
  }
\ohead{\bfseries \pagemark}
\cfoot{}
\makeatletter
\newcommand*\NoIndentAfterEnv[1]{%
  \AfterEndEnvironment{#1}{\par\@afterindentfalse\@afterheading}}
\makeatother
\NoIndentAfterEnv{itemize}
\NoIndentAfterEnv{enumerate}
\NoIndentAfterEnv{description}
\NoIndentAfterEnv{quote}
\NoIndentAfterEnv{equation}
\NoIndentAfterEnv{longtable}
\NoIndentAfterEnv{abstract}
\renewenvironment{abstract}
 {\vspace{-1.25cm}
 \quotation\small\noindent\emph{Abstract}:}
 {\endquotation}
\newfontfamily\tfont{EB Garamond}
\addtokomafont{disposition}{\rmfamily}
\addtokomafont{title}{\normalfont\itshape}
\let\footnoterule\relax
\KOMAoption{captions}{tableheading}
\makeatletter
\@ifpackageloaded{caption}{}{\usepackage{caption}}
\AtBeginDocument{%
\ifdefined\contentsname
  \renewcommand*\contentsname{Table of contents}
\else
  \newcommand\contentsname{Table of contents}
\fi
\ifdefined\listfigurename
  \renewcommand*\listfigurename{List of Figures}
\else
  \newcommand\listfigurename{List of Figures}
\fi
\ifdefined\listtablename
  \renewcommand*\listtablename{List of Tables}
\else
  \newcommand\listtablename{List of Tables}
\fi
\ifdefined\figurename
  \renewcommand*\figurename{Figure}
\else
  \newcommand\figurename{Figure}
\fi
\ifdefined\tablename
  \renewcommand*\tablename{Table}
\else
  \newcommand\tablename{Table}
\fi
}
\@ifpackageloaded{float}{}{\usepackage{float}}
\floatstyle{ruled}
\@ifundefined{c@chapter}{\newfloat{codelisting}{h}{lop}}{\newfloat{codelisting}{h}{lop}[chapter]}
\floatname{codelisting}{Listing}
\newcommand*\listoflistings{\listof{codelisting}{List of Listings}}
\makeatother
\makeatletter
\makeatother
\makeatletter
\@ifpackageloaded{caption}{}{\usepackage{caption}}
\@ifpackageloaded{subcaption}{}{\usepackage{subcaption}}
\makeatother
\usepackage{bookmark}
\IfFileExists{xurl.sty}{\usepackage{xurl}}{} % add URL line breaks if available
\urlstyle{same}
\hypersetup{
  pdftitle={Luminous Margins},
  pdfauthor={Brian Weatherson},
  hidelinks,
  pdfcreator={LaTeX via pandoc}}


\title{Luminous Margins\thanks{Thanks to Tamar Szabó Gendler, John
Hawthorne, Chris Hill, Ernest Sosa and the \emph{AJP}'s referees.}}
\author{Brian Weatherson}
\date{2004}
\begin{document}
\maketitle
\begin{abstract}
Timothy Williamson has recently argued that few mental states are
luminous, meaning that to be in that state is to be in a position to
know that you are in the state. His argument rests on the plausible
principle that beliefs only count as knowledge if they are safely true.
That is, any belief that could easily have been false is not a piece of
knowledge. I argue that the form of the safety rule Williamson uses is
inappropriate, and the correct safety rule might not conflict with
luminosity.
\end{abstract}


\setstretch{1.1}
\section{Luminosity}\label{luminosity}

In \emph{Knowledge and Its Limits} Timothy Williamson argues that few
conditions are \emph{luminous}. \footnote{Williamson
  (\citeproc{ref-Williamson2000-WILKAI}{2000}) Ch. 4; all references to
  this book unless otherwise specified.} A condition is luminous iff we
know we are in it whenever we are. Slightly more formally, Williamson
defines

\begin{quote}
A condition C is defined to be \emph{luminous} if and only if (L) holds:

\begin{description}
\tightlist
\item[(L)]
For every case α, if in α C obtains, then in α one is in a position to
know that C obtains (95).
\end{description}
\end{quote}

Intuitively, the argument against this is as follows. The following
three conditions are incompatible.

\begin{description}
\tightlist
\item[Gradual Change]
There is a series of cases, each very similar to adjacent cases, that
starts with a case where C clearly obtains, and ends with a case where C
clearly doesn't obtain.
\item[Luminosity]
Whenever C obtains you can know it does.
\item[Safety]
Only safe beliefs count as knowledge, so whenever you can know that C
obtains, C obtains in all very similar cases.
\end{description}

Luminosity and Safety entail

\begin{description}
\tightlist
\item[Tolerance]
Whenever C obtains, it obtains in all very similar cases.
\end{description}

But Tolerance is incompatible with Gradual Change, since Tolerance
entails that if the first member of the series is a case where C
obtains, then every successive member is also a case where C obtains.
Williamson argues that for any interesting epistemic condition, Gradual
Change is a clear possibility. And he argues that Safety is a general
principle about knowledge. So Luminosity must be scrapped. The
counterexamples to Luminosity we get from following this proof through
are always borderline cases of C obtaining. In these cases Luminosity
fails because any belief that C did obtain would be unsafe, and hence
not knowledge.

I will argue, following Sainsbury (\citeproc{ref-Sainsbury1996}{1995}),
that Williamson has misinterpreted the requirement that knowledge be
safe. The most plausible safety condition might be compatible with
Gradual Change and Luminosity, if we make certain plausible assumptions
about the structure of phenomenal beliefs.

One consequence of the failure of Luminosity is that a certain
historically important kind foundationalist analysis of knowledge fails.
This kind of foundationalist takes the foundations to be luminous.
Although I think Williamson's argument against Luminosity does not work,
my objections are no help to the foundationalist. As I said, my
objection to Williamson rests on certain assumptions about the structure
of phenomenal beliefs. It is a wide open empirical and philosophical
question whether these assumptions are true. If this kind of
foundationalism provided a plausible \emph{analysis} of knowledge, then
it would be a wide open question whether our purported knowledge rested
on any foundations, and hence a wide open question whether we really had
any knowledge. But this is a closed question. It is a Moorean fact that
we know many things. So while I object to Williamson's claim that we
have no luminous mental states, I do not object to the weaker claim that
we \emph{might} not have any luminous mental states, and this claim is
enough to do much of the philosophical work to which Williamson puts
Luminosity.

\section{Williamson's Example}\label{williamsons-example}

Williamson suggests that (L), the formal rendition of Luminosity, fails
for all interesting conditions even if we restrict the quantifier to
those that are `physically and psychologically feasible' (94), and I
will assume that is what we are quantifying over. To argue that (L)
fails for any interesting C, Williamson first argues that it fails in a
special case, when C is the condition \emph{feeling cold}, and then
argues that the conditions that lead to failure here are met for any
other interesting C. So I will also focus on the special case.

Mr Davis's apartment faces southwest, so while it is often cold in the
mornings it always warms up as the midday and afternoon sun streams in.
This morning Mr Davis felt cold when he awoke, but now at noon he is
quite warm, almost hot. But the change from wake-up time to the present
is rather gradual. Mr Davis does not take a hot bath that morning, nor
cook a hot breakfast, but sits reading by the window until the sun does
its daily magic. Assume, for the sake of the argument, that
\emph{feeling cold} is luminous, so whenever Mr Davis feels cold, he
knows he feels cold. Williamson argues this leads to a contradiction as
follows. (I've changed names and pronouns to conform with my example.)

\begin{quote}
Let \emph{t}\textsubscript{0}, \emph{t}\textsubscript{1}, \ldots,
\emph{t\textsubscript{n}} be a series of times at one millisecond
intervals from dawn to noon. Let α\textsubscript{\emph{i}} be the case
at \emph{t\textsubscript{i}} (0 ⩽~\emph{i}~⩽~\emph{n}). Consider a time
\emph{t\textsubscript{i}} between \emph{t}\textsubscript{0} and
\emph{t\textsubscript{n}}, and suppose that at \emph{t\textsubscript{n}}
Mr Davis knows that he feels cold. \ldots{} Now at
\emph{t\textsubscript{i}}\textsubscript{+1} he is almost equally
confident that he feels cold, by the description of the case. So if he
does not feel cold at \emph{t\textsubscript{i}}\textsubscript{+1}, then
his confidence at \emph{t\textsubscript{i}} that he feels cold is not
reliably based, for his almost equal confidence on a similar basis one
millisecond earlier that he felt cold was misplaced \ldots{} His
confidence at \emph{t\textsubscript{i}} was reliably based in the way
required for knowledge only if he feels cold at
\emph{t\textsubscript{i}}\textsubscript{+1}. In the terminology of
cases\ldots:

(\textsc{i}\textsubscript{\emph{i}}) If in α\textsubscript{\emph{i}} he
knows that he feels cold, then in α\textsubscript{\emph{i}+1} he feels
cold. (97)
\end{quote}

Given (L), all instances of (\textsc{i}\textsubscript{\emph{i}}), and
the fact that Mr Davis feels cold when he awakes, we get the false
conclusion that he now feels cold. So if we accept all instances of
(\textsc{i}\textsubscript{\emph{i}}), we must conclude that (L) is false
when C is \emph{feeling cold} and `one' denotes Mr Davis. Why, then,
accept (\textsc{i}\textsubscript{\emph{i}})? One move Williamson makes
here is purely defensive. He notes that
(\textsc{i}\textsubscript{\emph{i}}) is different from the conditionals
that lead to paradox in the Sorites argument. The antecedent of
(\textsc{i}\textsubscript{\emph{i}}) contains the modal operator
\emph{knows that} absent from its consequent, so we cannot chain
together instances of (\textsc{i}\textsubscript{\emph{i}}) to produce an
implausible conditional claim. If that operator were absent then from
all the instances of (\textsc{i}\textsubscript{\emph{i}}) it would
follow that if Mr Davis feels cold at dawn he feels cold at noon, which
is false. But by strengthening the antecedent, Williamson weakens
(\textsc{i}\textsubscript{\emph{i}}) to avoid that conclusion. But the
fact that (\textsc{i}\textsubscript{\emph{i}}) is not paradoxical is not
sufficient reason to accept it.

\section{Reliability}\label{reliability}

It is useful to separate out two distinct strands in Williamson's
argument for (\textsc{i}\textsubscript{\emph{i}}). One strand sees
Williamson arguing for (\textsc{i}\textsubscript{\emph{i}}) by resting
on the principle that beliefs constitute knowledge only if they are
reliably based. The idea is that if Mr Davis's belief that he feels cold
is a bit of knowledge, it is reliable, and if it is reliable it is true
in all similar situations, and hence it is true in
α\textsubscript{\emph{i}+1}. The other strand sees him appealing to a
vague but undoubtedly real requirement that beliefs must be
\emph{safely} true in order to be knowledge. Neither argument is
successful, though the second kind of argument is better than the first.

Williamson acknowledges Conee and Feldman's arguments that no
reliabilist epistemologist has yet solved the generality problem (100).
But he takes this to be reason to abandon not the concept of
reliability, but the hope of providing a reductive analysis of it.
Williamson thinks we can get a long way by just resting on the intuitive
concept of reliability. This seems to be a mistake. There are two
ordinary ways of using `reliable' in the context of discussing beliefs,
and neither provides support for (\textsc{i}\textsubscript{\emph{i}}).

First, and this is clearly not what is needed, sometimes `reliable' just
means true. This is the sense of the word in which we can consistently
say, ``It turned out the information that old Ronnie provided us about
where the gov'nor was eating tonight was reliable, which was plenty
surprising since Ronnie hadn't been right about anything since the Nixon
administration.'' This is the sense in which reliable means just what
the etymology suggests it means, something that can be relied upon. And
that means, in practice, \emph{true}. But that won't help at all, for if
`reliable' just means true, then nothing follows from the fact that
knowledge is reliable that does not follow from the fact that it is
factive.

Second, there is a distinctively philosophical sense in which reliable
means something more like true in a wide range of circumstances. This is
the sense in which a stopped clock is not even reliable twice a day. At
first, this might look to help Williamson a little more. But if
philosophical usage is to be key, the second look is more discouraging.
For in its philosophical usage, reliability does not even entail truth.
And if reliability does not entail truth in the actual situation, it
surely does not entail truth in nearby situations. But Williamson's
argument for (\textsc{i}\textsubscript{\emph{i}}) requires that
reliability in α\textsubscript{\emph{i}} entails truth in
α\textsubscript{\emph{i}+1}. So on neither of its natural readings does
the concept of reliability seal the argument here, and since we have no
unnatural reading to fall back upon, the argument from reliability for
(\textsc{i}\textsubscript{\emph{i}}) fails. To be fair, by chapter 5 of
Williamson's book the concept of reliability that seems to be employed
is little distinguishable from the concept of safety. So let us turn to
those arguments.

\section{Safety}\label{safety}

Williamson at times suggests that the core argument for
(\textsc{i}\textsubscript{\emph{i}}) is a straight appeal to intuition.
``{[}E{]}ven when we can appeal to rigorous rules, they only postpone
the moment at which we must apply concepts in particular cases on the
basis of good judgement. \ldots{} The argument for
(\textsc{i}\textsubscript{\emph{i}}) appeals to such judgement.'' (101)
The appeal to intuition is the royal road to scepticism, so we would be
justified in being a little wary of it. Weinberg, Stich, and Nichols
(\citeproc{ref-Weinberg2001}{2001}) discovered that undergraduates from
the same social class as Williamson, Mr Davis and I would frequently
judge that a subject could not know that mule was a mule unless he could
tell it apart from a cleverly painted zebra. The judgements of that
class are not obviously the basis for a sane epistemology.

Williamson undersells his argument by making it an appeal to judgement.
For there is a principle here, if not a rigorous rule, that grounds the
judgement. The principle is something like Ernest Sosa's safety
principle. The idea is that a belief does not constitute knowledge if it
is false in similar situations. ``{[}N{]}ot easily would S believe that
\emph{p} without it being the case that \emph{p}.''
(\citeproc{ref-Sosa1999}{Sosa 1999, 142}) There is much to be said here
about what is a \emph{similar} situation. (David Lewis
(\citeproc{ref-Lewis1996b}{1996}) discusses a concept of similarity in
the context of saying that worlds can be \emph{salient}, in his sense,
in virtue of being similar to salient worlds.) It might turn out that
there is no account of similarity that makes it plausible that this is a
constraint on knowledge. But for present purposes I am prepared to grant
(a) that only safe beliefs count as knowledge, and (b) that
α\textsubscript{\emph{i}+1} is a similar situation to
α\textsubscript{\emph{i}}.

This might seem like too much of a concession to Williamson, for it
already conflicts with some platitudes about knowledge. Consider a case
that satisfies the following three conditions. Some light reflects off a
leopard some distance away and strikes our eyes. The impact of that
light causes, by the normal processes, a belief that a leopard is nearby
to appear in our belief box. Beliefs, including leopard-related beliefs,
that we form by this kind of process are on the whole very reliable. You
might think these conditions are sufficient for our belief to count as
\emph{knowledge} that a tiger is present. The proponent of Safety denies
this. She says that if, for example, there are several cheetahs with a
particularly rare mutation that make the look much like leopards around,
and if we saw them at similar distance we would have mistaken them for
leopards. Since we could easily have had the belief that a leopard is
nearby while there were no leopards, only cheetahs, nearby, the belief
is not safe and so does not count as knowledge.

There are two reasons to think that safety is too strong here, neither
of which strike me as completely compelling. (I'm still conceding things
to Williamson here. If there's a general objection to Safety then his
argument against Luminosity does not get off the ground. That's not my
position. As I'll soon argue, I think Williamson has misinterpreted
Safety.) The first reason is a worry that if we deny knowledge in a case
of reliable veridical perception, we are conceding too much to the
sceptic. But the proponent of Safety has a very good reason to
distinguish this case from my current veridical perception of a table -
my perception is safe and the perception of a leopard is not. So there
is no slippery slope to scepticism here. The second is that the
allegedly similar case is not really that similar, because in that case
the belief is caused by a \emph{cheetah}, not a \emph{leopard}. But to
regard cases where the evidence is different in this way as being
dissimilar is to make the safety condition impotent, and Sosa has shown
that we need \emph{some} version of Safety to account for our intuitions
about different cases.\footnote{I assume here a relatively conservative
  epistemological methodology, one that says we should place a high
  priority on having our theories agree with our intuitive judgments.
  I'm in favour of a more radical methodology that makes theoretical
  virtues as important as agreement with particular intuitions
  Weatherson (\citeproc{ref-Weatherson2003-WEAWGA}{2003}). On the
  radical view Safety might well be abandoned. But on that view
  knowledge might be merely true belief, or merely justified true
  belief, so the argument for Luminosity will be a non-starter. But the
  argument of this paper does not rest on these radical methodological
  principles. The position I'm defending is that, supposing a standard
  methodological approach, we should accept a Safety principle. But as
  I'll argue, the version of Safety Williamson adopts is not
  appropriate, and the appropriate version does not necessarily support
  the argument against Luminosity.}

So I think some version of Safety should be adopted. I don't think this
gives us (\textsc{i}\textsubscript{\emph{i}}), for reasons related to
some concerns first raised by Mark Sainsbury
(\citeproc{ref-Sainsbury1996}{1995}). The role for Safety condition in a
theory of knowledge is to rule out knowledge by lucky guesses. This
includes lucky guesses in mathematics. If Mr Davis guesses that 193 plus
245 is 438, he does not thereby know what 193 plus 245 is. Can Safety
show why this is so? Yes, but only if we phrase it in a certain way.
Assume that we have a certain belief \emph{B} with content \emph{p}. (As
it might be, Mr Davis's belief with content 193 + 245 = 438.) Then the
following two conditions both have claims to being the correct analysis
of `safe' as it appears in Safety.

\begin{description}
\tightlist
\item[Content-safety]
\emph{B} is safe iff \emph{p} is true in all similar worlds.
\item[Belief-safety]
\emph{B} is safe iff \emph{B} is true in all similar worlds.
\end{description}

If we rest with content-safety, then we cannot explain why Mr Davis's
lucky guess does not count as knowledge. For in all nearby worlds, the
content of the belief he actually has is true. If we use belief-safety
as our condition though, I think we can show why Mr Davis has not just
got some mathematical knowledge. The story requires following Marian
David's good advice for token physicalists and rejecting content
essentialism about belief (David (\citeproc{ref-David2002}{2002}); see
also Gibbons (\citeproc{ref-Gibbons1993}{1993})). The part of Mr Davis's
brain that currently instantiates a belief that 193 plus 245 is 438
could easily have instantiated a belief that 193 plus 245 is 338, for Mr
Davis is not very good at carrying hundreds while guessing. If, as good
physicalists, we identify his belief with the part of the brain that
instantiates it, we get the conclusion that this very belief could have
had the \emph{false} content that 193 plus 245 is 338. So the belief is
not safe, and hence it is not knowledge.

This lends some credence to the idea that it's belief-safety, not
content-safety, that's the important safety criteria. When talking about
Mr Davis's mathematical hunches, belief-safety is a stronger condition
than content-safety. But when talking about his feelings, things may be
reversed.

Let me tell you a little story about how Mr Davis's mind is
instantiated. Mr Davis's phenomenal beliefs do not arise from one part
of his brain, his belief box or mind's eye, tracking another part, the
part whose states constitute his feeling cold. Rather, when he is in
some phenomenal state, the very same brain states constitute both the
phenomena and a belief about the phenomena. Mr Davis's brain is so wired
that he could not have any sensation of radiant heat (or lack thereof)
without his \emph{thereby} believing that he is having just that
sensation, because he could not have felt cold without that feeling
itself being a belief that he felt cold. In that case, belief-safety
will not entail (\textsc{i}\textsubscript{\emph{i}}). Imagine that at
α\textsubscript{\emph{i}} Mr Davis feels cold, but at
α\textsubscript{\emph{i}+1} he does not. (I assume here, with
Williamson, that there is such an \emph{i}.) At
α\textsubscript{\emph{i}} he thereby believes that he feels cold. The
content of that belief is a \emph{de se} proposition that is false at
α\textsubscript{\emph{i}+1}, so it violates content-safety. But in
α\textsubscript{\emph{t}+1} that part of his brain does not constitute
his feeling cold (for he does not feel cold), and thereby does not
constitute his believing that he feels cold. By hypothesis, by that time
no part of his brain constitutes feeling cold. So the belief in
α\textsubscript{\emph{i}} that he feels cold is not false in
α\textsubscript{\emph{i}+1}; it either no longer exists, or now has the
true content that Mr Davis does not feel cold. So belief-safety does not
prevent this belief of Mr Davis's from being knowledge. And indeed, it
seems rather plausible that it \emph{is} knowledge, for he could not
have had just \emph{this} belief without it being true. This belief
violates content-safety but not belief-safety, and since we have no
reason to think that content-safety rather than belief-safety is the
right form of the safety constraint, we have no reason to reject the
intuition that this belief, this more or less infallible belief, counts
as a bit of knowledge.

This story about Mr Davis's psychology might seem unbelievable, so let
me clear up some details. Mr Davis has both phenomenal and judgemental
beliefs about his phenomenal states. The phenomenal beliefs are present
when and only when the phenomenal states are present. The judgemental
beliefs are much more flexible, they are nomically independent of the
phenomena they describe. The judgemental beliefs are grounded in `inner
perceptions' of his phenomenal states. The phenomenal beliefs are not,
they just are the phenomenal states. The judgemental beliefs can be
complex, as in a belief that \emph{I feel cold iff it is Monday}, while
the phenomenal beliefs are always simple. It is logically possible that
Mr Davis be wired so that he feel cold without believing he feels cold,
but it is not an accident that \emph{he} is so wired. Most of his
conspecifics are similarly set up. It is possible that at a particular
time Mr Davis has \emph{both} a phenomenal belief and a judgemental
belief that he feels cold, with the beliefs being instantiated in
different parts of his brain. If he has both of these beliefs in
α\textsubscript{\emph{i}}, then Williamson's argument may well show that
the judgemental belief does not count as knowledge, for it could be
false in α\textsubscript{\emph{i}+1}. If he has the judgemental belief
that he is not cold in α\textsubscript{\emph{i}}, then the phenomenal
belief that he is cold may not be knowledge, for it is plausible that
the existence of a contrary belief defeats a particular belief's claim
to knowledge. But that does not mean that he is not \emph{in a position
to know} that he is cold in α\textsubscript{\emph{i}}.

Some may object that it is conceptually impossible that a brain state
that instantiates a phenomenal feel should also instantiate a belief.
And it is true that Mr Davis's phenomenal states do not have some of the
features that we typically associate with beliefs. These states are
relatively unstructured, for example. Anyone who thinks that it is a
conceptual truth that mental representations are structured like
linguistic representations will think that Mr Davis could not have the
phenomenal beliefs I have ascribed to him. But it is very implausible
that this is a \emph{conceptual} truth. The best arguments for the
language of thought hypothesis rest on empirical facts about believers,
especially the facts that mental representation is typically productive
and systematic. If there are limits to how productive and systematic Mr
Davis's phenomenal representations are, then it is possible that his
phenomenal states are beliefs. Certainly those states are correlated
with inputs (external states of affairs) and outputs (bodily movements,
if not actions) to count as beliefs on some functionalist conceptions of
belief.

A referee noted that we don't \emph{need} the strong assumption that
phenomenal states can be beliefs to make the argument here, though it
probably is the most illumination example. Either of the following
stories about Mr Davis's mind could have done. First, Mr Davis's
phenomenal belief may be of the form ``I feel ϕ'', where ``I'' and
``feel'' are words in Mr Davis's language of thought, and ϕ is the
phenomenal state, functioning as a name for itself. As long as the
belief arises whenever Mr Davis is ϕ, and it has the phenomenal state as
a constituent, it can satisfy belief-safety even when content-safety
fails. The second option involves some more contentious assumptions. The
phenomenal belief may be of the form ``I feel thus'', where the
demonstrative picks out the phenomenal state. As long as it is essential
to the belief that it includes a demonstrative reference to that
phenomenal state, it will satisfy belief-safety. This is more
contentious because it might seem plausible that a particular
demonstrative belief could have picked out a different state. What won't
work, of course, is if the phenomenal belief is ``I feel \emph{F}'',
where \emph{F} is an attempted description of the phenomenal state. That
certainly violates every kind of safety requirement. I think it is
plausible that phenomenal states could be belief states, but if you do
not believe that it is worth noting the argument could possibly go
through without it, as illustrated in this paragraph.

Mr Davis is an interesting case because he shows just how strong a
safety assumption we need to ground
(\textsc{i}\textsubscript{\emph{i}}). For Mr Davis is a counterexample
to (\textsc{i}\textsubscript{\emph{i}}), but his coldness beliefs
satisfy many plausible safety-like constraints. For example, his beliefs
about whether he feels cold are sensitive to whether he feels cold.
Williamson (Ch. 7) shows fairly conclusively that knowledge does not
entail sensitivity, so one might have thought that in interesting cases
sensitivity would be too strong for what is needed, not too weak as it
is here. From this it follows that any safety condition that is strictly
weaker than sensitivity, such as the condition that the subject could
not easily believe \emph{p} and be wrong, is not sufficient to support
(\textsc{i}\textsubscript{\emph{i}}). Williamson slides over this point
by assuming that the subject will be almost as confident that he feels
cold at α\textsubscript{\emph{i}+1} as he is at
α\textsubscript{\emph{i}}. This is no part of the description of the
case, as Mr Davis shows.

My argument above rests on the denial of content essentialism, which
might look like a relatively unsafe premise. So to conclude this
section, let's see how far the argument can go without that assumption.
Sainsbury responds to his example, the lucky arithmetic guess, by
proposing a different version of safety: mechanism-safety.

\begin{description}
\tightlist
\item[Mechanism-safety]
\emph{B} is safe iff the mechanism that produced \emph{B} produces true
beliefs in all similar worlds.
\end{description}

I didn't want to rest on this too much because I think it's rather hard
to say exactly what the mechanism is that produces Mr Davis's belief
that he feels cold. But if it's just his sensory system, then I think it
is clear that even at α\textsubscript{\emph{i}}, Mr Davis's belief that
he feels cold satisfies mechanism-safety. The bigger point here is that
content-safety is a very distinctive kind of safety claim, but it's the
only kind that justifies (\textsc{i}\textsubscript{\emph{i}}).

\section{Retractions}\label{retractions}

To close, let me stress how limited my criticisms of Williamson here
are. Very briefly, the argument is that there can be some
self-presenting mental states, states that are either token identical
with the belief that they exist or are constituents of (the contents of)
beliefs that they exist, and these beliefs will satisfy all the safety
requirements we should want, even in borderline cases. If some
conditions are invariably instantiated by self-presenting states, then
those conditions will be luminous. And I think it is a live possibility,
relative at least to the assumptions Williamson makes, that there are
such self-presenting states. But there aren't very many of them. There
is a reason I picked \emph{feels cold} as my illustration. It's not
laughable that it is self-presenting.

On the other hand, it is quite implausible that, say, \emph{knowing
where to buy the best Guinness} is self-presenting. And for states that
are not self-presenting, I think Williamson's anti-luminosity argument
works. That's because it is very plausible (a) that for a belief to be
knowledge it must satisfy either belief-safety or mechanism-safety, (b)
a non-self-presenting state satisfies belief-safety or mechanism-safety
only if it satisfies content-safety, and (c) as Williamson showed, if
beliefs about a state must satisfy content-safety to count as knowledge,
then that state is not luminous. So epistemic states, like the state of
knowing where to buy the best Guinness, are not luminous. That is to
say, one can know where to buy the best Guinness without knowing that
one knows this. And saying that (for these reasons) is to just endorse
Williamson's arguments against the KK principle. Those arguments are an
important special case of the argument against luminosity, and I don't
see how any of my criticisms of the general argument touch the special
case.

Williamson describes his attacks on luminosity as an argument for
cognitive homelessness. If a state was luminous, that state would be a
cognitive home. Williamson thinks we are homeless. I think we \emph{may}
have a small home in our phenomenal states. This home is not a mansion,
perhaps just a small apartment with some afternoon sun, but it may be a
home.

Don't be fooled into thinking this supports any kind of foundationalism
about knowledge, however. It is true that \emph{if} we have the kind of
self-presenting states that Mr Davis has (under one of the three
descriptions I've offered), then we have the self-justifying beliefs
that foundationalism needs to get started. But it is at best a wide-open
philosophical and scientific question whether we have any such states,
while it is not a wide-open question whether we have any knowledge, or
any justified beliefs. If these states are the only things that could
serve as foundations, it would be at least conceptually possible that we
could have knowledge without self-justifying foundations. So the kind of
possibility exemplified by Mr Davis cannot, on its own, prop up
foundationalism.

\subsection*{References}\label{references}
\addcontentsline{toc}{subsection}{References}

\phantomsection\label{refs}
\begin{CSLReferences}{1}{0}
\bibitem[\citeproctext]{ref-David2002}
David, Marian. 2002. {``Content Essentialism.''} \emph{Acta Analytica}
17: 103--14. doi:
\href{https://doi.org/10.1007/bf03177510}{10.1007/bf03177510}.

\bibitem[\citeproctext]{ref-Gibbons1993}
Gibbons, John. 1993. {``Identity Without Supervenience.''}
\emph{Philosophical Studies} 70 (1): 59--79. doi:
\href{https://doi.org/10.1007/bf00989662}{10.1007/bf00989662}.

\bibitem[\citeproctext]{ref-Lewis1996b}
Lewis, David. 1996. {``Elusive Knowledge.''} \emph{Australasian Journal
of Philosophy} 74 (4): 549--67. doi:
\href{https://doi.org/10.1080/00048409612347521}{10.1080/00048409612347521}.
Reprinted in his \emph{Papers in Metaphysics and Epistemology},
Cambridge: Cambridge University Press, 1999, 418-446. References to
reprint.

\bibitem[\citeproctext]{ref-Sainsbury1996}
Sainsbury, Mark. 1995. {``Vagueness, Ignorance and Margin for Error.''}
\emph{British Journal for the Philosophy of Science} 46: 589--601. doi:
\href{https://doi.org/10.1093/bjps/46.4.589}{10.1093/bjps/46.4.589}.

\bibitem[\citeproctext]{ref-Sosa1999}
Sosa, Ernest. 1999. {``How to Defeat Opposition to Moore.''}
\emph{Philosophical Perspectives} 13: 141--53. doi:
\href{https://doi.org/10.1111/0029-4624.33.s13.7}{10.1111/0029-4624.33.s13.7}.

\bibitem[\citeproctext]{ref-Weatherson2003-WEAWGA}
Weatherson, Brian. 2003. {``{What Good Are Counterexamples?}''}
\emph{Philosophical Studies} 115 (1): 1--31. doi:
\href{https://doi.org/10.1023/A:1024961917413}{10.1023/A:1024961917413}.

\bibitem[\citeproctext]{ref-Weinberg2001}
Weinberg, Jonathan, Stephen Stich, and Shaun Nichols. 2001.
{``Normativity and Epistemic Intuitions.''} \emph{Philosophical Topics}
29 (1): 429--60. doi:
\href{https://doi.org/10.5840/philtopics2001291/217}{10.5840/philtopics2001291/217}.

\bibitem[\citeproctext]{ref-Williamson2000-WILKAI}
Williamson, Timothy. 2000. \emph{{Knowledge and its Limits}}. Oxford
University Press.

\end{CSLReferences}



\noindent Published in\emph{
Australasian Journal of Philosophy}, 2004, pp. 373-383.


\end{document}
