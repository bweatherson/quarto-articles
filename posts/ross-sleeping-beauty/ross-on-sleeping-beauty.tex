% Options for packages loaded elsewhere
% Options for packages loaded elsewhere
\PassOptionsToPackage{unicode}{hyperref}
\PassOptionsToPackage{hyphens}{url}
%
\documentclass[
  11pt,
  letterpaper,
  DIV=11,
  numbers=noendperiod,
  twoside]{scrartcl}
\usepackage{xcolor}
\usepackage[left=1.1in, right=1in, top=0.8in, bottom=0.8in,
paperheight=9.5in, paperwidth=7in, includemp=TRUE, marginparwidth=0in,
marginparsep=0in]{geometry}
\usepackage{amsmath,amssymb}
\setcounter{secnumdepth}{3}
\usepackage{iftex}
\ifPDFTeX
  \usepackage[T1]{fontenc}
  \usepackage[utf8]{inputenc}
  \usepackage{textcomp} % provide euro and other symbols
\else % if luatex or xetex
  \usepackage{unicode-math} % this also loads fontspec
  \defaultfontfeatures{Scale=MatchLowercase}
  \defaultfontfeatures[\rmfamily]{Ligatures=TeX,Scale=1}
\fi
\usepackage{lmodern}
\ifPDFTeX\else
  % xetex/luatex font selection
  \setmainfont[ItalicFont=EB Garamond Italic,BoldFont=EB Garamond
Bold]{EB Garamond Math}
  \setsansfont[]{EB Garamond}
  \setmathfont[]{Garamond-Math}
\fi
% Use upquote if available, for straight quotes in verbatim environments
\IfFileExists{upquote.sty}{\usepackage{upquote}}{}
\IfFileExists{microtype.sty}{% use microtype if available
  \usepackage[]{microtype}
  \UseMicrotypeSet[protrusion]{basicmath} % disable protrusion for tt fonts
}{}
\usepackage{setspace}
% Make \paragraph and \subparagraph free-standing
\makeatletter
\ifx\paragraph\undefined\else
  \let\oldparagraph\paragraph
  \renewcommand{\paragraph}{
    \@ifstar
      \xxxParagraphStar
      \xxxParagraphNoStar
  }
  \newcommand{\xxxParagraphStar}[1]{\oldparagraph*{#1}\mbox{}}
  \newcommand{\xxxParagraphNoStar}[1]{\oldparagraph{#1}\mbox{}}
\fi
\ifx\subparagraph\undefined\else
  \let\oldsubparagraph\subparagraph
  \renewcommand{\subparagraph}{
    \@ifstar
      \xxxSubParagraphStar
      \xxxSubParagraphNoStar
  }
  \newcommand{\xxxSubParagraphStar}[1]{\oldsubparagraph*{#1}\mbox{}}
  \newcommand{\xxxSubParagraphNoStar}[1]{\oldsubparagraph{#1}\mbox{}}
\fi
\makeatother


\usepackage{longtable,booktabs,array}
\usepackage{calc} % for calculating minipage widths
% Correct order of tables after \paragraph or \subparagraph
\usepackage{etoolbox}
\makeatletter
\patchcmd\longtable{\par}{\if@noskipsec\mbox{}\fi\par}{}{}
\makeatother
% Allow footnotes in longtable head/foot
\IfFileExists{footnotehyper.sty}{\usepackage{footnotehyper}}{\usepackage{footnote}}
\makesavenoteenv{longtable}
\usepackage{graphicx}
\makeatletter
\newsavebox\pandoc@box
\newcommand*\pandocbounded[1]{% scales image to fit in text height/width
  \sbox\pandoc@box{#1}%
  \Gscale@div\@tempa{\textheight}{\dimexpr\ht\pandoc@box+\dp\pandoc@box\relax}%
  \Gscale@div\@tempb{\linewidth}{\wd\pandoc@box}%
  \ifdim\@tempb\p@<\@tempa\p@\let\@tempa\@tempb\fi% select the smaller of both
  \ifdim\@tempa\p@<\p@\scalebox{\@tempa}{\usebox\pandoc@box}%
  \else\usebox{\pandoc@box}%
  \fi%
}
% Set default figure placement to htbp
\def\fps@figure{htbp}
\makeatother


% definitions for citeproc citations
\NewDocumentCommand\citeproctext{}{}
\NewDocumentCommand\citeproc{mm}{%
  \begingroup\def\citeproctext{#2}\cite{#1}\endgroup}
\makeatletter
 % allow citations to break across lines
 \let\@cite@ofmt\@firstofone
 % avoid brackets around text for \cite:
 \def\@biblabel#1{}
 \def\@cite#1#2{{#1\if@tempswa , #2\fi}}
\makeatother
\newlength{\cslhangindent}
\setlength{\cslhangindent}{1.5em}
\newlength{\csllabelwidth}
\setlength{\csllabelwidth}{3em}
\newenvironment{CSLReferences}[2] % #1 hanging-indent, #2 entry-spacing
 {\begin{list}{}{%
  \setlength{\itemindent}{0pt}
  \setlength{\leftmargin}{0pt}
  \setlength{\parsep}{0pt}
  % turn on hanging indent if param 1 is 1
  \ifodd #1
   \setlength{\leftmargin}{\cslhangindent}
   \setlength{\itemindent}{-1\cslhangindent}
  \fi
  % set entry spacing
  \setlength{\itemsep}{#2\baselineskip}}}
 {\end{list}}
\usepackage{calc}
\newcommand{\CSLBlock}[1]{\hfill\break\parbox[t]{\linewidth}{\strut\ignorespaces#1\strut}}
\newcommand{\CSLLeftMargin}[1]{\parbox[t]{\csllabelwidth}{\strut#1\strut}}
\newcommand{\CSLRightInline}[1]{\parbox[t]{\linewidth - \csllabelwidth}{\strut#1\strut}}
\newcommand{\CSLIndent}[1]{\hspace{\cslhangindent}#1}



\setlength{\emergencystretch}{3em} % prevent overfull lines

\providecommand{\tightlist}{%
  \setlength{\itemsep}{0pt}\setlength{\parskip}{0pt}}



 


\setlength\heavyrulewidth{0ex}
\setlength\lightrulewidth{0ex}
\usepackage[automark]{scrlayer-scrpage}
\clearpairofpagestyles
\cehead{
  Brian Weatherson
  }
\cohead{
  Ross on Sleeping Beauty
  }
\ohead{\bfseries \pagemark}
\cfoot{}
\makeatletter
\newcommand*\NoIndentAfterEnv[1]{%
  \AfterEndEnvironment{#1}{\par\@afterindentfalse\@afterheading}}
\makeatother
\NoIndentAfterEnv{itemize}
\NoIndentAfterEnv{enumerate}
\NoIndentAfterEnv{description}
\NoIndentAfterEnv{quote}
\NoIndentAfterEnv{equation}
\NoIndentAfterEnv{longtable}
\NoIndentAfterEnv{abstract}
\renewenvironment{abstract}
 {\vspace{-1.25cm}
 \quotation\small\noindent\emph{Abstract}:}
 {\endquotation}
\newfontfamily\tfont{EB Garamond}
\addtokomafont{disposition}{\rmfamily}
\addtokomafont{title}{\normalfont\itshape}
\let\footnoterule\relax
\cehead{
       Ishani Maitra and Brian Weatherson
        }
\usepackage{nicefrac}
\KOMAoption{captions}{tableheading}
\makeatletter
\@ifpackageloaded{caption}{}{\usepackage{caption}}
\AtBeginDocument{%
\ifdefined\contentsname
  \renewcommand*\contentsname{Table of contents}
\else
  \newcommand\contentsname{Table of contents}
\fi
\ifdefined\listfigurename
  \renewcommand*\listfigurename{List of Figures}
\else
  \newcommand\listfigurename{List of Figures}
\fi
\ifdefined\listtablename
  \renewcommand*\listtablename{List of Tables}
\else
  \newcommand\listtablename{List of Tables}
\fi
\ifdefined\figurename
  \renewcommand*\figurename{Figure}
\else
  \newcommand\figurename{Figure}
\fi
\ifdefined\tablename
  \renewcommand*\tablename{Table}
\else
  \newcommand\tablename{Table}
\fi
}
\@ifpackageloaded{float}{}{\usepackage{float}}
\floatstyle{ruled}
\@ifundefined{c@chapter}{\newfloat{codelisting}{h}{lop}}{\newfloat{codelisting}{h}{lop}[chapter]}
\floatname{codelisting}{Listing}
\newcommand*\listoflistings{\listof{codelisting}{List of Listings}}
\makeatother
\makeatletter
\makeatother
\makeatletter
\@ifpackageloaded{caption}{}{\usepackage{caption}}
\@ifpackageloaded{subcaption}{}{\usepackage{subcaption}}
\makeatother
\usepackage{bookmark}
\IfFileExists{xurl.sty}{\usepackage{xurl}}{} % add URL line breaks if available
\urlstyle{same}
\hypersetup{
  pdftitle={Ross on Sleeping Beauty},
  pdfauthor={Brian Weatherson},
  hidelinks,
  pdfcreator={LaTeX via pandoc}}


\title{Ross on Sleeping Beauty}
\author{Brian Weatherson}
\date{2013}
\begin{document}
\maketitle
\begin{abstract}
In two excellent recent papers, Jacob Ross has argued that the standard
arguments for the `thirder' answer to the Sleeping Beauty puzzle lead to
violations of countable additivity. The problem is that most arguments
for that answer generalise in awkward ways when he looks at the whole
class of what he calls Sleeping Beauty problems. In this note I develop
a new argument for the thirder answer that doesn't generalise in this
way.
\end{abstract}


\setstretch{1.1}
In two excellent recent papers, Jacob Ross
(\citeproc{ref-Ross2010}{2010}, \citeproc{ref-Ross2012}{2012}) has
argued that the standard arguments for the ⅓~answer to the Sleeping
Beauty puzzle lead to violations of countable additivity. The problem is
that most arguments for the ⅓~answer generalise in awkward ways when he
look at the whole class of what he calls \emph{Sleeping Beauty
problems}.

\begin{quote}
Let us define a \emph{Sleeping Beauty problem} as any problem in which a
fully rational agent, Beauty, will undergo one or more mutually
indistinguishable awakenings and in which the number of awakenings she
will undergo is determined by the outcome of a random process. Let
\emph{S} be a partition of alternative hypotheses concerning the outcome
of this process. Beauty knows the objective chance of each hypothesis in
\emph{S}, and she also knows how many times she will awaken conditional
on each of these hypotheses, but she has no other relevant information.
The problem is to determine how her credence should be divided among the
hypotheses in \emph{S} when she first awakens. The original Sleeping
Beauty problem is the instance of this class of problems in which the
random process that determines how many times Beauty awakens is a fair
coin toss and in which Beauty awakens once given Heads and twice given
Tails. (\citeproc{ref-Ross2010}{Ross 2010, 413--14})
\end{quote}

The `Generalised Thirder Principle' says

\begin{quote}
In any Sleeping Beauty problem, defined by a partition \emph{S}, upon
first awakening, Beauty's credence in any given hypothesis in \emph{S}
should be proportional to the product of its objective chance and the
number of times Beauty awakens if this hypothesis is true.
(\citeproc{ref-Ross2010}{Ross 2010, 414})
\end{quote}

Ross shows that this Principle leads to violations of countable
additivity, and argues (convincingly to my mind) that this is a serious
problem for the Principle. In Ross (\citeproc{ref-Ross2010}{2010}) he
argues that many arguments for the ⅓~answer imply the Generalised
Thirder Principle, and in Ross (\citeproc{ref-Ross2012}{2012}) he argues
that the same is true of the argument for the ⅓~answer in Weatherson
(\citeproc{ref-Weatherson-SoSB}{2011}). He suggests that we have some
inductive reason to think that \emph{all} arguments for the ⅓~answer
will imply the Generalised Thirder Principle. In reply I want to make
three points.

First, Ross offers a much more careful presentation and analysis of the
argument for the ⅓~answer than appears in Weatherson
(\citeproc{ref-Weatherson-SoSB}{2011}), and that analysis does really
suggest that the argument overgenerates. But second, the careful setting
out suggests that the argument fails by its own standards; it commits a
fallacy of equivocation. So I don't think it's as much inductive
evidence for Ross's general conclusion about all arguments for the
⅓~answer as he suggests. And third, the inductive conclusion Ross draws
is not true. There are arguments for the ⅓~answer that don't imply the
Generalised Thirder principle. Below I outline two ways to fix the
argument in Weatherson (\citeproc{ref-Weatherson-SoSB}{2011}) so that it
becomes such an argument.

It will be worthwhile having a minimal example of where the Generalised
Thirder principle goes beyond the arguments for the ⅓~answer, so
consider the following minimal version.

\begin{quote}
\textbf{Three-Day Sleeping Beauty}

A coin will be flipped. If it comes up Heads, Beauty will wake up
Monday, and go back to sleep until Thursday. If it comes up Tails,
Beauty will wake Monday, then have her memories of that waking erased,
then wake again Tuesday, and have her memories of that waking erased,
then wake up Wednesday, and go back to sleep until Thursday. The
different possible wakings will be indistinguishable.\footnote{I have
  some concerns about what `indistinguishable' means in this context.
  We'll come back to that issue a lot in what follows.} When she wakes
on Monday, what should her credence be that the coin landed Heads?
\end{quote}

The Generalised Thirder says that it should be ¼. One of our aims here
will be to come up with an argument for the ⅓~answer in the original
puzzle that doesn't imply the answer to the Three-Day Sleeping Beauty
puzzle is ¼.

Returning to the original puzzle, Let \emph{Cr}\textsubscript{1} be
Beauty's credence function when she wakes on Monday. Let \emph{M} be the
proposition Beauty would express on waking with ``Today is Monday''. And
let \emph{H} be the proposition that the coin landed heads. The ⅓~answer
entails that the following claims are all true.

\begin{enumerate}
\def\labelenumi{\arabic{enumi}.}
\tightlist
\item
  \emph{Cr}\textsubscript{1}(\emph{M}~∧~\emph{H})~=~\emph{Cr}\textsubscript{1}(\emph{M}~∧~¬\emph{H})
\item
  \emph{Cr}\textsubscript{1}(¬\emph{M}~∧~¬\emph{H})~=~\emph{Cr}\textsubscript{1}(¬\emph{M}~∧~¬\emph{H})
\item
  \emph{Cr}\textsubscript{1}(\emph{M}~∧~¬\emph{H})~=~\emph{Cr}\textsubscript{1}(¬\emph{M}~∧~¬\emph{H})
\end{enumerate}

The usual argument for the ⅓~answer argues for (1) and (2), and then
derives (3). And the arguments for (2) typically generalise into
arguments for the Generalised Thirder Principle, and hence to violations
of countable additivity.

To get a bit better feel for what's going on with these principles,
consider one interesting sub-class of Ross's large category of Sleeping
Beauty problems. These are problems where there are \emph{n} chance
hypotheses, and for any \emph{i} such that 1 ⩽ \emph{i} ⩽ \emph{n},
Beauty is woken \emph{i} times if chance hypothesis
\emph{h\textsubscript{i}} is true. When she wakes up, we can let the
proposition \emph{p\textsubscript{ji}}, where 1 ⩽ \emph{j} ⩽ \emph{i} ⩽
\emph{n}, be the proposition that this is the \emph{i}'th day, and
chance hypothesis \emph{h\textsubscript{j}} is true. The following
table, where the chance hypotheses are on the rows, and days are on the
columns, represents the possibilities as they strike Beauty.

\begin{longtable}[]{@{}lcccc@{}}
\toprule\noalign{}
~ & ~ Day 1 ~ & ~ Day 2 ~ & ~\ldots~ & ~Day \emph{n} \\
\midrule\noalign{}
\endhead
\bottomrule\noalign{}
\endlastfoot
\emph{h}\textsubscript{1}~ & ~\emph{p}\textsubscript{11} ~ & ~ ~ & ~ ~ &
~ \\
\emph{h}\textsubscript{2}~ & ~\emph{p}\textsubscript{12} ~ &
~\emph{p}\textsubscript{22} ~ & ~ ~ & ~ \\
\ldots{} ~ & ~ ~ & ~ ~ & ~ ~ & ~ \\
\emph{h\textsubscript{n}}~ & ~\emph{p}\textsubscript{1\emph{n}}~ &
~\emph{p}\textsubscript{2\emph{n}}~ & ~\ldots~ &
~\emph{p\textsubscript{nn}} \\
\end{longtable}

The usual argument for the ⅓~answer to Sleeping Beauty includes a
`vertical' argument and a `horizontal' argument. The `vertical' argument
attempts to show that
\emph{Cr}\textsubscript{1}(\emph{p}\textsubscript{1\emph{i}}~\textbar~\emph{p}\textsubscript{11}~∨~\ldots~∨~\emph{p}\textasciitilde1\emph{n})~=~Ch(\emph{h\textsubscript{i}}).
The `horizontal' argument attempts to show that
\emph{Cr}\textsubscript{1}(\emph{p\textsubscript{ji}}~\textbar~\emph{p}\textsubscript{11}~∨~\ldots~∨~\emph{p}\textasciitilde1\emph{i})~=~1/\emph{i}.
Or, at least, it attempts to show that those claims are true for the
special case where \emph{n}~=~2. but as Ross shows, the arguments
offered seem to work in the \emph{n}~=~2 case iff they work in the
general case. And these vertical and horizontal arguments together do
imply the Generalised Thirder Principle.

The argument in Weatherson (\citeproc{ref-Weatherson-SoSB}{2011}) took a
different tack. It argued for (1) and (3), and derived (2) Cr. In terms
of the table above, the idea was to replace the `horizontal' argument,
and indeed to reject its conclusion in the general case, with a
`diagonal' argument, which showed that
\emph{Cr}\textsubscript{1}(\emph{p\textsubscript{ii}}~\textbar~\emph{p}\textsubscript{11}~∨~\ldots~∨~\emph{p\textsubscript{nn}})~=~Ch(\emph{h\textsubscript{i}}).
If the vertical and diagonal arguments worked, and didn't
overgeneralise, then they wouldn't entail a solution for cases like
Three-Day Sleeping Beauty, or about any case from the above class where
\emph{n} \textgreater{} 2.

If we drop the restriction to Sleeping Beauty problems where Beauty is
woken \emph{i} times if chance hypothesis \emph{h\textasciitilde i} is
true, and return to Ross's more general class, the arguments in
Weatherson (\citeproc{ref-Weatherson-SoSB}{2011}) were meant to prove
the following two claims, and not a lot more.

\begin{description}
\tightlist
\item[First Day]
In any Sleeping Beauty problem where Beauty wakes at least one time in
every chance hypothesis, and exactly one time in at least one of them,
when she wakes the first time, her conditional credences in each
hypothesis in \emph{S}, conditional on this actually being the first
waking, equals the objective chance of each such hypothesis.
\item[Last Day]
In any Sleeping Beauty problem where Beauty wakes at least one time in
every chance hypothesis,, when she wakes the first time, her conditional
credences in each hypothesis in \emph{S}, conditional on this actually
being the last waking, equals the objective chance of each such
hypothesis.
\end{description}

Since \textbf{First Day} entails (1), and \textbf{Last Day} entails (3),
these principles entail the ⅓~answer. But they don't settle what to say
about the Three-Day Sleeping Beauty example. If
\emph{Cr}\textsubscript{3} is Beauty's credences when she wakes for the
first time in that example, these principles are consistent with Beauty
having the following credal distribution.

\begin{quote}
\noindent *Cr\emph{\textsubscript{3}(Today is Monday and Heads) =
\(\nicefrac{1}{5}\)~\\
}Cr\emph{\textsubscript{3}(Today is Monday and Tails) =
\(\nicefrac{1}{5}\)~\\
}Cr\emph{\textsubscript{3}(Today is Tuesday and Tails) =
\(\nicefrac{2}{5}\)~\\
}Cr*\textsubscript{3}(Today is Wednesday and Tails) =
\(\nicefrac{1}{5}\)~
\end{quote}

But those credences are incompatible with the Generalised Thirder
Principle, so \textbf{First Day} and \textbf{Last Day} do not entail
that principle.

Ross (\citeproc{ref-Ross2012}{2012}) argues that this isn't right, and
that the motivation for \textbf{First Day} offered in Weatherson
(\citeproc{ref-Weatherson-SoSB}{2011}) in fact does lead to the
Generalised Thirder Principle on its own. I think that's true in a
sense; the argument provides just as much support for the Generalised
Thirder Principle as it does for \textbf{First Day}. But that's because
it is a bad argument, and doesn't support \textbf{First Day}. We'll see
why that's true shortly. But since it is true, a new argument is needed
for (1) , one that supports \textbf{First Day}, but not the Generalised
Thirder principle.

Most discussions of Sleeping Beauty assume that the contents of
propositional attitudes are sets of centered worlds. Following Stalnaker
(\citeproc{ref-Stalnaker2008}{2008}), the argument in Weatherson
(\citeproc{ref-Weatherson-SoSB}{2011}) tried to get by with propositions
simply being sets of worlds. The key idea was that the worlds themselves
would be fine-grained enough that thoughts like \emph{Hesperus is
Phosophorus}, or \emph{Today is Monday} would be contingently true if
true at all. Very roughly, we associate singular terms with something
like Fregean senses.\footnote{In Weatherson
  (\citeproc{ref-Weatherson-SoSB}{2011}) these are described as
  haeceitties, but this is misleading at best. It is crucial that
  `Hesperus' and `Phosphorus' have different associations, but
  intuitively they have the same haecceity. Thinking of the associations
  as being with something like senses is better.} When Beauty wonders
whether today is Monday, she isn't wondering about whether an instance
of the law of identity is true. She has no more interest in that law
than does the first gentleman of Europe. She is wondering, in effect,
whether two senses, TODAY and MONDAY, have the same referent.

This way of looking at things implies that credal dynamics need to be
complicated. Sometimes our credences change because we acquire more
information, and we react accordingly. But sometimes our credences
change because we acquire new senses, and we can think new thoughts.
That will become crucial in what follows.

There is a quick argument for \textbf{Last Day}. Consider what happens
after Beauty wakes up at the end of the puzzle, and is told that the
game is over. (That is, on Wednesday in the original puzzle, or on
Thursday in the three-day variant.) Plausibly, her credence in \emph{H}
should be back to ½, or at least it is hard to see a good argument why
it should be anything else.\footnote{One might be tempted by a seemingly
  stronger argument. For instance, one could argue that on Wednesday,
  Beauty knows that the chance of \emph{H} was ½, and she has no
  inadmissible evidence, so by the Principal Principle her credence in
  \emph{H} should be ½. But it isn't clear that she has no inadmissible
  evidence; perhaps the evidence she would express by thinking back to
  her last waking and saying \emph{That waking happened} is
  inadmissible. Or one might be tempted by a Reflection Principle based
  argument against any alternative credence. But such arguments seem to
  lead to odd results in general around Sleeping Beauty. It's best, I
  think, to stick with the clear intuition that on Wednesday her
  credence in \emph{H} should be ½.} She can also think back to her last
waking, and think to herself \emph{If H, that was on Monday, and if}
\(¬\)H, that was on Tuesday. (She'll replace \emph{Tuesday} with
\emph{Wednesday} in the three-day variant.) Call this conditional
\emph{C}. Now think back to that last waking. At that time she is
wondering various thoughts about the waking she is currently undergoing,
a waking she will describe as ``this waking''. When she wakes on
Wednesday, and thinks about ``that waking'', it is plausible she is
thinking the very same kind of Fregean thought. She is thinking about
the same thing, and thinking about it in the same kind of way. If that's
right, then the big difference between her credences after the puzzle
ends and her credences on the last waking are just that she comes to
learn \emph{C}. So on the last waking, her credence in \emph{H}
conditional on \emph{C} should be ½, since when she learns \emph{C} and
nothing else, her credence in \emph{H} becomes ½. And that entails (3),
and similar reasoning generalises to all Sleeping Beauty problems to
entail \textbf{Last Day}.\footnote{Strictly speaking, all we've really
  shown is that Beauty's credences on the last day she wakes should
  satisfy \textbf{Last Day}. So if the focus of the puzzle is on her
  credences on the first day, all we've strictly speaking shown is that
  if \emph{H}, then \textbf{Last Day} is true. I think it is plausible
  that \textbf{Last Day} should be independent of how the coin lands,
  but I admit that I don't have an argument against someone who wants to
  dispute this.}

The argument in Weatherson (\citeproc{ref-Weatherson-SoSB}{2011}) for
(1) , and for \textbf{First Day}, involved a rather baroque variant of
the example involving time travelers. And, as we've noted already, it
also involves some fallacious equivocation. Before we get to that, it is
worth noting two other arguments for the same conclusion, neither of
which generalise to the Generalised Thirder Principle. Both arguments
have contentious premises, but they are somewhat independent, so I hope
that presenting both arguments will increase the number of people who
agree with \textbf{FirstDay}.

The first argument is an argument from the Principal Principle. Consider
the version of the Sleeping Beauty puzzle where the coin is tossed after
Beauty goes back to sleep on Monday. (We didn't say so far when the coin
is tossed, and it was consistent with everything we said that it is
Monday night.) And assume that Beauty knows that the coin is tossed
Monday. So when she wakes on Monday, she knows that the chance of
\emph{H} is ½~if it is Monday. That is,
\emph{Cr}\textsubscript{1}(\emph{Ch}(\emph{H})~=~½~\textbar~\emph{M})~=~1.
The Principal Principle says that unless Beauty has inadmissible
evidence,
\emph{Cr}\textsubscript{1}(\emph{H}~\textbar~\emph{Cr}(\emph{H})~=~½)~=~½.
And plausibly she doesn't have any inadmissible evidence conditional on
it being Monday. Putting these two together, we get
\emph{Cr}\textsubscript{1}(\emph{H}~\textbar~\emph{M})~=~½. And that
kind of reasoning generalises to support \textbf{First Day}.

The argument here is similar to the original argument given for (1) in
Elga (\citeproc{ref-Elga2000}{2000}). Elga imagines a variant of the
example where Beauty is told, sometime after she wakes up, that it is
Monday, and argues that after that she should have credence ½~in
\emph{H}, and derives (1) from that. Halpern
(\citeproc{ref-Halpern2004}{2004}) objects to this argument on the
grounds that the possibility of Beauty being told what day it is
undermines the indistinguishability of the wakings, and this undermines
Elga's own argument for (2). I'm not sure Halpern is right, but in any
case, this argument doesn't rely on any possibility of Beauty being told
what day it is. I imagine that some people will object to the claim that
Beauty has no inadmissible evidence. But it is hard to see what she
knows which is inadmissible, at least conditional on it being Monday.
She \emph{knows} that if it is Monday, the truth of \emph{H} rests on a
chance event that is yet to take place, and from which she is causally
isolated. That looks to me like knowledge that she has no inadmissible
evidence.

The second argument is a version of the the Technicolor Beauty argument
in Titlebaum (\citeproc{ref-Titlebaum2008}{2008}). It relies on a
variant of the example that drops the idea that the wakings are
indistinguishable in a strong sense. I'll set up first what the idea
behind the argument is, and then set out how it works. Assume that
Beauty can think about \emph{M} on Sunday. Follow Ross in using
\emph{Cr}\textsubscript{0} for Beauty's credences on Sunday. The same
Principal Principle style argument we used above suggests that
\emph{Cr}\textsubscript{0}(\emph{H}~\textbar~\emph{M})~=~½. Indeed in
this case the argument is even stronger, since everyone agrees that on
Sunday, Beauty has no inadmissible evidence. But nothing happens to
surprise Beauty between Sunday and Monday, so
\emph{Cr}\textsubscript{1}(\emph{H}~\textbar~\emph{M})~=~½ should be
½~as well, and we derive (1) from there.

There are a few problems with this argument. For one thing, the `no
surprise' premise goes by very quickly. More importantly, Beauty can't
actually have \emph{M} thoughts on Sunday. She can think to herself that
Monday is Monday, or at least she could if she cared to think about the
law of identity. But that's not the same thought as \emph{M}. Remember,
the guiding idea here is that contents are Fregean; it isn't easy to
have the same thought as someone who thinks \emph{This is Monday}.
Something dramatic needs to happen to let Beauty have such a thought on
Sunday, when she isn't in a position to make the same kind of
demonstration as she is on Monday.

Here's one way the dramatic thing might happen. Change the example so
that the wakings Beauty undergoes are not \emph{phenomenally}
indistinguishable. In fact, Beauty is told on Sunday that each waking
will be in a brightly coloured room, and the colours will be different
each day. As it happens, the room that will be used for Monday is red,
though Beauty doesn't know that. Let \emph{RE} be the proposition that
one of Beauty's wakings will be in a red room, and \emph{RM} be the
proposition that she wakes Monday in a red room. Now clearly \emph{RE},
on its own, is inadmissible evidence in the sense that
\emph{Cr}\textsubscript{0}(\emph{H}~\textbar~\emph{RE}) need not be ½.
After all, \emph{RE} is probabilistic evidence that Beauty has more than
one waking, since the more wakings she has, the more chance there is
that one of them will be in a red room. On the other hand, \emph{RM}
does not look like inadmissible evidence. She has to wake up in some
colour room or other on Monday; learning it is red doesn't change
anything. So \emph{Cr}\textsubscript{0}(\emph{H}~\textbar~\emph{RM})
should be ½. And that's true even though \emph{RM} obviously entails
\emph{RE}.

Now she wakes on Monday, and the room is red. What follows? Well, she
now knows \emph{RE}. And she can identify her current waking with the
red waking she imagined (or at least could have imagined) on Sunday. So
it is at least arguable that when on Monday she considers the thought
\emph{This waking is on Monday}, that's the very same thought she
considers on Sunday by saying to herself \emph{The waking in a red room
is on Monday}. Making that last claim more plausible would require
offering a more detailed theory of mental content than I have the space
(or ability) to do here. For now I'm just going to take as a premise
that there's a workable theory of mental content that types contents
more finely than does a purely referential theory, but on which it is
nevertheless the case that Beauty's demonstrative thought on Monday has
the same content as her descriptive thought (about the waking in the red
room) does on Sunday.

Here is one way of thinking about that claim about content that may make
it more plausible. (This idea is derived from the arguments in Jeshion
(\citeproc{ref-Jeshion2002}{2002}).) Imagine that on Sunday Beauty names
the waking in a red room. She calls it `Bluey'. She knows that `Bluey'
might not refer. That is, she knows that Bluey, like Vulcan and Sherlock
Holmes, might not exist. But she nevertheless entertains detailed
thoughts about Bluey. She wonders if Bluey will be on Monday, whether
she'll be happy when Bluey happens, and so on. Now she wakes up on
Monday, and sees the red walls. She says to herself, ``This is Bluey''.
From that point on, it seems that she'd express the same thought with
\emph{This waking is} φ and \emph{Bluey is} φ, and it seems she's
express the same thought with \emph{Bluey is} φ and \emph{The waking in
a red room is} φ. By appeal to transitivity of identity, and
substituting a particular value for φ, we get that she expresses the
same thought by saying \emph{This waking is on Monday} as by saying
\emph{The waking in a red room is on Monday}.

If that claim about content is right, then \emph{all} that happens on
Monday is that Beauty learns that \emph{RE} is true. She doesn't acquire
the ability to think new thoughts, or to make fresh divisions in
possibility space, the way that she does in the standard version of the
puzzle. In the standard version of the puzzle, the demonstrative thought
she considers on Monday, the one she would express by saying \emph{This
is Monday}, is not equivalent to anything she can think on Sunday. So
when she wakes, she not only acquires some evidence, she acquires a new
cognitive capacity. That doesn't happen here, which makes the
calculations easier.

In particular, it lets us appeal to the following key fact. If
\emph{E}\textsubscript{2} entails \emph{E}\textsubscript{1}, and a
particular update only involves conditionalising one's prior credences,
then learning \emph{E}\textsubscript{1} doesn't change the conditional
credence of anything given \emph{E}\textsubscript{2}. That's a
consequence of the following theorem. Let Pr be any probability
function, and let Pr\textsuperscript{+} be the result of
conditionalising that function on \emph{E}\textsubscript{1}. Then
Pr(\emph{H}~\textbar~\emph{E}\textsubscript{2})~=~Pr\textsuperscript{+}(\emph{H}~\textbar~\emph{E}\textsubscript{2}).
So if Beauty only learns \emph{RE}, that doesn't change the conditional
credence of anything given \emph{RM}. In particular, it doesn't change
the conditional credence of \emph{H} given \emph{RM}. So
\emph{Cr}\textsubscript{1}(\emph{H}~\textbar~\emph{RM})~=~½. And since
\emph{M} and \emph{RM} are trivially equivalent, since Beauty can see
the room is red, it follows that
\emph{Cr}\textsubscript{1}(\emph{H}~\textbar~\emph{M})~=~½, as required.

I suspect the main objection to this argument will be that adding the
room colours makes a substantial change to the problem. The fact that
⅓~is the correct answer in this technicolour version of Sleeping Beauty,
says the objector, is no reason to think that it is also the correct
answer in the version where the wakings are phenomenally
indistinguishable. But I think the objector will have a hard time making
the case that phenomenal indistinguishability is epistemically
significant unless they want to defend what Williamson
(\citeproc{ref-Williamson2007-WILTPO-17}{2007}) calls the `phenomenal
conception of evidence'. As has been pointed out in prior work on
Sleeping Beauty, the different wakings are not evidentially equivalent;
when she wakes up and sees that \emph{this} waking is happening, that's
a piece of evidence she gets in some but not all wakings. (This point is
made in Weintraub (\citeproc{ref-Weintraub2004}{2004}) and Stalnaker
(\citeproc{ref-Stalnaker2008}{2008}).) It might well be argued that this
demonstrative evidence is in a sense symmetric; although she gets
evidence that is different in some sense on the different wakings, the
force of that evidence is the same. But that's still true in the
technicolour version of the problem as well. So I think, contra the
objector, that this is a good argument for (1) .

But neither of those were the argument offered in Weatherson
(\citeproc{ref-Weatherson-SoSB}{2011}). That argument involved a rather
baroque modification to the puzzle. A time traveller films Beauty waking
on Monday and travels back to Sunday to show Beauty the film. After she
sees the film, and can think about the waking it depicts, the traveller
tells her that he took it on Monday. He then wipes Beauty's memories of
this telling, but not of the showing of the film. The argument then
proceeds as follows.

\begin{enumerate}
\def\labelenumi{\arabic{enumi}.}
\tightlist
\item
  On Sunday, she can think about her Monday waking in the same way as
  she thinks about it on Monday when she wakes, thanks to the time
  traveller's film.
\item
  On Sunday, the rational credence in \emph{H} is ½.
\item
  If premise 1 is true, then on Monday, after she wakes, the only
  difference between her epistemic state then and her epistemic state
  after being told that the film was of Monday is that she no longer
  knows \emph{M}.
\item
  If the only difference between two epistemic states is that in the
  first, an agent knows \emph{M} and in the second she does not, then
  the rational credence of \emph{H} given \emph{M} in the second state
  equals the rational credence of \emph{H} in the first state.
\end{enumerate}

And from that
\emph{Cr}\textsubscript{1}(\emph{H}~\textbar~\emph{M})~=~½~was claimed
to follow. Now there wasn't much of an argument for the first premise
offered, and it might well be thought objectionable. It certainly relies
on a liberal conception of sameness of `ways of thinking'. But let's set
that aside, because there is a much bigger problem with the argument. It
hopelessly equivocates on the phrase `On Sunday'. Let's distinguish the
following four times that are all on Sunday.

\begin{itemize}
\tightlist
\item
  \emph{t}\textsubscript{1} is before the time traveller turns up.
\item
  \emph{t}\textsubscript{2} is immediately after the time traveller
  shows Beauty the film.
\item
  \emph{t}\textsubscript{3} is immediately after the time traveller
  tells Beauty that the film is of Monday.
\item
  \emph{t}\textsubscript{4} is after the time traveller wipes Beauty's
  memories of that telling, but not of the showing of the film.
\end{itemize}

Now let's consider the first two premises in Weatherson's argument. The
first premise is clearly false if `On Sunday' refers to
\emph{t}\textsubscript{1}, but arguably true if it refers to
\emph{t}\textsubscript{2}, \emph{t}\textsubscript{3} or
\emph{t}\textsubscript{4}. The second premise is clearly true if `On
Sunday' refers to \emph{t}\textsubscript{1}, but much less plausible if
it refers to any later time. If it refers to \emph{t}\textsubscript{2}
or \emph{t}\textsubscript{4}, it is arguably equivalent to the ½~answer
to the original Sleeping Beauty problem, which makes it pretty useless
in an argument for the ⅓~answer! If it refers to
\emph{t}\textsubscript{3}, it is much too close to what we're trying to
prove in arguing for \textbf{FirstDay}, so it is still argumentatively
useless.

Ross argues that the style of argument we've been considering would, if
it showed that (1) , and indeed more generally supported \textbf{First
Day}, would show much more. In fact, it would show that for any two
chance hypotheses \emph{h\textsubscript{i}} and
\emph{h\textsubscript{j}}, and any \emph{k} such that
\emph{h\textsubscript{j}} is consistent with at least \emph{k} wakings,
that the following is true. (I'll again use \emph{p\textsubscript{kj}}
to mean that this is the \emph{k}'th waking and
\emph{h\textsubscript{j}} is true.)

\begin{quote}
\emph{Cr}\textsubscript{1}(\emph{p}\textsubscript{1\emph{i}}~\textbar~\emph{p}\textsubscript{1\emph{i}}~∨~\emph{p\textsubscript{kj}})
=
\emph{Ch}(\emph{h\textsubscript{i}})/(\emph{Ch}(\emph{h\textsubscript{i}})
+ \emph{Ch}(\emph{h\textsubscript{j}}))
\end{quote}

And from that we can derive the Generalised Thirder Principle, and hence
countable additivity violations. That wasn't what was intended; the
argument was only designed to work for the special case where
\emph{k}~=~1, i.e., \textbf{First Day}. Now I think Ross is right in the
following sense; there's just as good an argument in Weatherson
(\citeproc{ref-Weatherson-SoSB}{2011}) for the above equation as there
is for \textbf{First Day}. But that argument is no good for the reasons
described above. Let's see where the same equivocation comes into Ross's
telling of the story. I've changed Ross's notation a fair bit into
notation I find easier to work with. Hopefully I haven't lost anything
in the process. I'm also going to focus on the special case of Three-Day
Sleeping Beauty, where \emph{h}\textsubscript{1} is the coin lands heads
and \emph{h}\textsubscript{2\$}is the coin lands tails, and on the case
where \emph{k}~=~2. So what we're really going to look at is whether
there's an argument in Three-Day Sleeping Beauty for this equation.

\begin{quote}
\emph{Cr}\textsubscript{3}(\emph{p}\textsubscript{11}~\textbar~\emph{p}\textsubscript{11}~∨~\emph{p}\textsubscript{22}
) =
\emph{Ch}(\emph{h}\textsubscript{1})/(\emph{Ch}(\emph{h}\textsubscript{1})
+ \emph{Ch}(\emph{h}\textsubscript{2}))~=~½
\end{quote}

Now as noted above, I think that this equation need not hold in
Three-Day Sleeping Beauty; in the model I gave for it earlier,
\emph{Cr}\textsubscript{3}(p\textsubscript{11})~=~\(\nicefrac{1}{5}\)~,
and
\emph{Cr}\textsubscript{3}(p\textsubscript{22})~=~\(\nicefrac{2}{5}\)~,
so
\emph{Cr}\textsubscript{3}(\emph{p}\textsubscript{11}~\textbar~\emph{p}\textsubscript{11}~∨~\emph{p}\textsubscript{22})~=~⅓.
But let's see how Ross derives the ½~answer.

Again there is a time-traveller who shows Beauty a film. But this isn't
necessarily a film of the first waking; the time traveller films the
first waking if \emph{h}\textsubscript{1}, and the second waking if
\emph{h}\textsubscript{2}. After seeing the film, Beauty is told this.
So on Sunday, says Ross, Beauty's credences should satisfy these
constraints. (I'm following Ross in using \emph{Cr}\textsubscript{0} for
the Sunday credences.)

\begin{quote}
\noindent *Cr\emph{\textsubscript{0}(}h\emph{\textsubscript{1}) =
}Cr\emph{\textsubscript{0}(}h\emph{\textsubscript{1} ∧
}p\emph{\textsubscript{11})\\
}Cr\emph{\textsubscript{0}(}h\emph{\textsubscript{1}) =
}Ch\emph{(}h*\textsubscript{1})
\end{quote}

There are other premises used, but these will be the crucial ones.
They're crucial because there's no good reason to think that there's
\emph{any} time Sunday when Beauty's credences should satisfy both these
equations. Before she sees the film, she can't even think about
propositions like \emph{p}\textsubscript{11}, since she can't have
singular thoughts about the waking it depicts. After she sees the film,
there is no reason to think that her credences should align with
chances. Causal contact with a time traveller who brings information
that may well be about a time after the chance event, evidence whose
existence may depend on the outcome of the chance event, is pretty much
paradigmatically inadmissible evidence for the purposes of the Principal
Principle. So after the film, there is no reason to think
\emph{Cr}\textsubscript{0}(\emph{h}\textsubscript{1})~=~\emph{Ch}(\emph{h}\textsubscript{1}).

I should stress that Ross doesn't \emph{endorse} the equivocating
premises here; he merely attributes them to Weatherson
(\citeproc{ref-Weatherson-SoSB}{2011}), and fairly so. But I think once
we see the equivocation we can see there is no fear the kind of argument
used in Weatherson (\citeproc{ref-Weatherson-SoSB}{2011}) will lead to
the Generalised Thirder Principle. That argument is too flawed to lead
to anything. But there are plenty of other arguments for \textbf{First
Day}, such as the two arguments offered here. And both of those
arguments rely on distinctive features of the first day. Most notably,
they rely on the fact that for all we say in the setup of the problem,
the first waking is before the chance event. So there's no reason to
think they will have the problematic consequences that Ross finds in the
argument for \textbf{First Day} in Weatherson
(\citeproc{ref-Weatherson-SoSB}{2011}).

\section*{References}\label{references}
\addcontentsline{toc}{section}{References}

\phantomsection\label{refs}
\begin{CSLReferences}{1}{0}
\bibitem[\citeproctext]{ref-Elga2000}
Elga, Adam. 2000. {``Self-Locating Belief and the Sleeping Beauty
Problem.''} \emph{Analysis} 60 (4): 143--47. doi:
\href{https://doi.org/10.1093/analys/60.2.143}{10.1093/analys/60.2.143}.

\bibitem[\citeproctext]{ref-Halpern2004}
Halpern, Joseph. 2004. {``Sleeping Beauty Reconsidered: Conditioning and
Reflection in Asynchronous Systems.''} In \emph{Oxford Studies in
Epistemology}, 1:111--42. Oxford: Oxford University Press.

\bibitem[\citeproctext]{ref-Jeshion2002}
Jeshion, Robin. 2002. {``Acquiantanceless \emph{de Re} Belief'.''} In
\emph{Meaning and Truth: Investigations in Philosophical Semantics},
edited by Joseph Keim Campbell, Michael O'Rourke, and David Shier,
53--74. New York: Seven Bridges Press.

\bibitem[\citeproctext]{ref-Ross2010}
Ross, Jacob. 2010. {``Sleeping Beauty, Countable Additivity, and
Rational Dilemmas.''} \emph{Philosophical Review} 119 (4): 411--47. doi:
\href{https://doi.org/10.1215/00318108-2010-010}{10.1215/00318108-2010-010}.

\bibitem[\citeproctext]{ref-Ross2012}
---------. 2012. {``All Roads Lead to Violations of Countable
Additivity.''} \emph{Philosophical Studies} 161 (3): 381--90. doi:
\href{https://doi.org/10.1007/s11098-011-9744-z}{10.1007/s11098-011-9744-z}.

\bibitem[\citeproctext]{ref-Stalnaker2008}
Stalnaker, Robert. 2008. \emph{Our Knowledge of the Internal World}.
Oxford: Oxford University Press.

\bibitem[\citeproctext]{ref-Titlebaum2008}
Titlebaum, Michael. 2008. {``The Relevance of Self-Locating Beliefs.''}
\emph{Philosophical Review} 117 (4): 555--605. doi:
\href{https://doi.org/10.1215/00318108-2008-016}{10.1215/00318108-2008-016}.

\bibitem[\citeproctext]{ref-Weatherson-SoSB}
Weatherson, Brian. 2011. {``Stalnaker on Sleeping Beauty.''}
\emph{Philosophical Studies} 155 (3): 445--56. doi:
\href{https://doi.org/10.1007/s11098-010-9613-1}{10.1007/s11098-010-9613-1}.

\bibitem[\citeproctext]{ref-Weintraub2004}
Weintraub, Ruth. 2004. {``Sleeping Beauty: A Simple Solution.''}
\emph{Analysis} 64 (1): 8--10. doi:
\href{https://doi.org/10.1093/analys/64.1.8}{10.1093/analys/64.1.8}.

\bibitem[\citeproctext]{ref-Williamson2007-WILTPO-17}
Williamson, Timothy. 2007. \emph{{The Philosophy of Philosophy}}.
Blackwell.

\end{CSLReferences}



\noindent Published in\emph{
Philosophical Studies}, 2013, pp. 503-512.


\end{document}
