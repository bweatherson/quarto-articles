% Options for packages loaded elsewhere
\PassOptionsToPackage{unicode}{hyperref}
\PassOptionsToPackage{hyphens}{url}
%
\documentclass[
  10pt,
  letterpaper,
  DIV=11,
  numbers=noendperiod,
  twoside]{scrartcl}

\usepackage{amsmath,amssymb}
\usepackage{setspace}
\usepackage{iftex}
\ifPDFTeX
  \usepackage[T1]{fontenc}
  \usepackage[utf8]{inputenc}
  \usepackage{textcomp} % provide euro and other symbols
\else % if luatex or xetex
  \usepackage{unicode-math}
  \defaultfontfeatures{Scale=MatchLowercase}
  \defaultfontfeatures[\rmfamily]{Ligatures=TeX,Scale=1}
\fi
\usepackage{lmodern}
\ifPDFTeX\else  
    % xetex/luatex font selection
  \setmainfont[ItalicFont=EB Garamond Italic,BoldFont=EB Garamond
Bold]{EB Garamond Math}
  \setsansfont[]{Europa-Bold}
  \setmathfont[]{Garamond-Math}
\fi
% Use upquote if available, for straight quotes in verbatim environments
\IfFileExists{upquote.sty}{\usepackage{upquote}}{}
\IfFileExists{microtype.sty}{% use microtype if available
  \usepackage[]{microtype}
  \UseMicrotypeSet[protrusion]{basicmath} % disable protrusion for tt fonts
}{}
\usepackage{xcolor}
\usepackage[left=1in, right=1in, top=0.8in, bottom=0.8in,
paperheight=9.5in, paperwidth=6.5in, includemp=TRUE, marginparwidth=0in,
marginparsep=0in]{geometry}
\setlength{\emergencystretch}{3em} % prevent overfull lines
\setcounter{secnumdepth}{3}
% Make \paragraph and \subparagraph free-standing
\ifx\paragraph\undefined\else
  \let\oldparagraph\paragraph
  \renewcommand{\paragraph}[1]{\oldparagraph{#1}\mbox{}}
\fi
\ifx\subparagraph\undefined\else
  \let\oldsubparagraph\subparagraph
  \renewcommand{\subparagraph}[1]{\oldsubparagraph{#1}\mbox{}}
\fi


\providecommand{\tightlist}{%
  \setlength{\itemsep}{0pt}\setlength{\parskip}{0pt}}\usepackage{longtable,booktabs,array}
\usepackage{calc} % for calculating minipage widths
% Correct order of tables after \paragraph or \subparagraph
\usepackage{etoolbox}
\makeatletter
\patchcmd\longtable{\par}{\if@noskipsec\mbox{}\fi\par}{}{}
\makeatother
% Allow footnotes in longtable head/foot
\IfFileExists{footnotehyper.sty}{\usepackage{footnotehyper}}{\usepackage{footnote}}
\makesavenoteenv{longtable}
\usepackage{graphicx}
\makeatletter
\def\maxwidth{\ifdim\Gin@nat@width>\linewidth\linewidth\else\Gin@nat@width\fi}
\def\maxheight{\ifdim\Gin@nat@height>\textheight\textheight\else\Gin@nat@height\fi}
\makeatother
% Scale images if necessary, so that they will not overflow the page
% margins by default, and it is still possible to overwrite the defaults
% using explicit options in \includegraphics[width, height, ...]{}
\setkeys{Gin}{width=\maxwidth,height=\maxheight,keepaspectratio}
% Set default figure placement to htbp
\makeatletter
\def\fps@figure{htbp}
\makeatother
% definitions for citeproc citations
\NewDocumentCommand\citeproctext{}{}
\NewDocumentCommand\citeproc{mm}{%
  \begingroup\def\citeproctext{#2}\cite{#1}\endgroup}
\makeatletter
 % allow citations to break across lines
 \let\@cite@ofmt\@firstofone
 % avoid brackets around text for \cite:
 \def\@biblabel#1{}
 \def\@cite#1#2{{#1\if@tempswa , #2\fi}}
\makeatother
\newlength{\cslhangindent}
\setlength{\cslhangindent}{1.5em}
\newlength{\csllabelwidth}
\setlength{\csllabelwidth}{3em}
\newenvironment{CSLReferences}[2] % #1 hanging-indent, #2 entry-spacing
 {\begin{list}{}{%
  \setlength{\itemindent}{0pt}
  \setlength{\leftmargin}{0pt}
  \setlength{\parsep}{0pt}
  % turn on hanging indent if param 1 is 1
  \ifodd #1
   \setlength{\leftmargin}{\cslhangindent}
   \setlength{\itemindent}{-1\cslhangindent}
  \fi
  % set entry spacing
  \setlength{\itemsep}{#2\baselineskip}}}
 {\end{list}}
\usepackage{calc}
\newcommand{\CSLBlock}[1]{\hfill\break\parbox[t]{\linewidth}{\strut\ignorespaces#1\strut}}
\newcommand{\CSLLeftMargin}[1]{\parbox[t]{\csllabelwidth}{\strut#1\strut}}
\newcommand{\CSLRightInline}[1]{\parbox[t]{\linewidth - \csllabelwidth}{\strut#1\strut}}
\newcommand{\CSLIndent}[1]{\hspace{\cslhangindent}#1}

\setlength\heavyrulewidth{0ex}
\setlength\lightrulewidth{0ex}
\usepackage[automark]{scrlayer-scrpage}
\clearpairofpagestyles
\cehead{
  Brian Weatherson
  }
\cohead{
  Deontology and Descartes’s Demon
  }
\ohead{\bfseries \pagemark}
\cfoot{}
\makeatletter
\newcommand*\NoIndentAfterEnv[1]{%
  \AfterEndEnvironment{#1}{\par\@afterindentfalse\@afterheading}}
\makeatother
\NoIndentAfterEnv{itemize}
\NoIndentAfterEnv{enumerate}
\NoIndentAfterEnv{description}
\NoIndentAfterEnv{quote}
\NoIndentAfterEnv{equation}
\NoIndentAfterEnv{longtable}
\NoIndentAfterEnv{abstract}
\renewenvironment{abstract}
 {\vspace{-1.25cm}
 \quotation\small\noindent\rule{\linewidth}{.5pt}\par\smallskip
 \noindent }
 {\par\noindent\rule{\linewidth}{.5pt}\endquotation}
\KOMAoption{captions}{tableheading}
\makeatletter
\@ifpackageloaded{caption}{}{\usepackage{caption}}
\AtBeginDocument{%
\ifdefined\contentsname
  \renewcommand*\contentsname{Table of contents}
\else
  \newcommand\contentsname{Table of contents}
\fi
\ifdefined\listfigurename
  \renewcommand*\listfigurename{List of Figures}
\else
  \newcommand\listfigurename{List of Figures}
\fi
\ifdefined\listtablename
  \renewcommand*\listtablename{List of Tables}
\else
  \newcommand\listtablename{List of Tables}
\fi
\ifdefined\figurename
  \renewcommand*\figurename{Figure}
\else
  \newcommand\figurename{Figure}
\fi
\ifdefined\tablename
  \renewcommand*\tablename{Table}
\else
  \newcommand\tablename{Table}
\fi
}
\@ifpackageloaded{float}{}{\usepackage{float}}
\floatstyle{ruled}
\@ifundefined{c@chapter}{\newfloat{codelisting}{h}{lop}}{\newfloat{codelisting}{h}{lop}[chapter]}
\floatname{codelisting}{Listing}
\newcommand*\listoflistings{\listof{codelisting}{List of Listings}}
\makeatother
\makeatletter
\makeatother
\makeatletter
\@ifpackageloaded{caption}{}{\usepackage{caption}}
\@ifpackageloaded{subcaption}{}{\usepackage{subcaption}}
\makeatother
\ifLuaTeX
  \usepackage{selnolig}  % disable illegal ligatures
\fi
\usepackage{bookmark}

\IfFileExists{xurl.sty}{\usepackage{xurl}}{} % add URL line breaks if available
\urlstyle{same} % disable monospaced font for URLs
\hypersetup{
  pdftitle={Deontology and Descartes's Demon},
  pdfauthor={Brian Weatherson},
  hidelinks,
  pdfcreator={LaTeX via pandoc}}

\title{Deontology and Descartes's Demon}
\author{Brian Weatherson}
\date{2008}

\begin{document}
\maketitle
\begin{abstract}
In this paper, I defend a broadly Cartesian position about doxastic
freedom. At least some of our beliefs are freely formed, so we are
responsible for them. Moreover, this has consequences for epistemology.
But the some here is crucial. Some of our beliefs are not freely formed,
and we are not responsible for those. And that has epistemological
consequences too. Out of these considerations a concept of doxastic
responsibility arises that is useful to the externalist in responding to
several challenges. I will say at some length how it supports a familiar
style of externalism response to the New Evil Demon problem, and I will
note some difficulties in reconciling internalism with the idea that
justification is a kind of blamelessness. The internalist, I will argue,
has to say that justification is a kind of praiseworthiness, and this
idea that praise is more relevant to epistemic concepts than blame will
be a recurring theme of the paper.
\end{abstract}

\setstretch{1.1}
\section{Digesting Evidence}\label{digesting-evidence}

In his \emph{Principles of Philosophy}, Descartes says,

\begin{quote}
Finally, it is so manifest that we possess a free will, capable of
giving or withholding its assent, that this truth must be reckoned among
the first and most common notions which are born with us.
(\citeproc{ref-DescartesPrinciples}{Descartes 1644/2003, para. xxxix})
\end{quote}

In this paper, I am going to defend a broadly Cartesian position about
doxastic freedom. At least some of our beliefs are freely formed, so we
are responsible for them. Moreover, this has consequences for
epistemology. But the some here is crucial. Some of our beliefs are not
freely formed, and we are not responsible for those. And that has
epistemological consequences too. Out of these considerations a concept
of doxastic responsibility arises that is useful to the externalist in
responding to several challenges. I will say at some length how it
supports a familiar style of externalism response to the New Evil Demon
problem, and I will note some difficulties in reconciling internalism
with the idea that justification is a kind of blamelessness. The
internalist, I will argue, has to say that justification is a kind of
praiseworthiness, and this idea that praise is more relevant to
epistemic concepts than blame will be a recurring theme of the paper.

While the kind of position I am adopting has been gaining supporters in
recent years, it is still largely unpopular. The arguments of William
Alston (\citeproc{ref-Alston1988}{1988}) have convinced many that it is
a mistake to talk of doxastic freedom, or doxastic responsibility. The
short version of this argument is that our beliefs are involuntary, and
freedom and responsibility require voluntariness. The longer, and more
careful, argument involves drawing some distinctions between ways in
which we might come to be in a state. It helps to start with an example
where the normative facts are relatively uncontroversial, namely
digestion.

Imagine that Emma eats a meat pie, and due to a malfunction in her
stomach the pie is not properly digested, leading to some medical
complications. Is Emma responsible for her ill-health? Well, that
depends on the back-story. If Emma knew that she could not properly
digest meat pies, but ate one anyway, she is responsible for the illness
via her responsibility for eating the pie. Even if Emma did not know
this, she might be responsible for the state of her stomach. If her
stomach could not digest the pie because it had been damaged by Emma's
dietary habits, and say Emma knew that her diet could damage her
stomach, then Emma is responsible for the state of her stomach and hence
for the misdigestion of the pie and hence for her ill-health. But if
neither of these conditions obtain, if it just happens that her stomach
misdigests the pie, then Emma is not responsible for her ill-health.
Even though the cause of her ill-health is something that her stomach
does, he is not responsible for that since her stomach is not under her
voluntary control. Put another way, her responsibility for maintaining
her own health means that she is responsible for the type of digester
she is, but he is not responsible for this token digestion.

Simplifying a little, Alston thinks that the case of belief is similar.
Say that Emma has a false belief that \emph{p}. Is she responsible for
this piece of doxastic ill-health? Again, that depends on the back
story. If Emma believes that \emph{p} because she was careless in
gathering evidence, and the evidence would have pointed to
\textasciitilde{}\emph{p}, then she is responsible for being a bad
gatherer of evidence. If Emma has been negligent in maintaining her
doxastic health, or worse if she has been doing things she knows
endangers doxastic health, then she is responsible for being the type of
believer she is. But she is never responsible merely for the token
belief that is formed. Her mind simply digests the evidence she has, and
Emma's responsibility only extends to her duty to gather evidence for
it, and her duty to keep her mind in good working order. She is not
responsible for particular acts of evidential digestion.

But these particular acts of evidential digestion are the primary
subject matters of epistemology. When we say Emma's belief is justified
or unjustified, we frequently mean that it is a good or bad response to
the evidence in the circumstances. (I am obviously here glossing over
enormous disputes about what makes for a good response, what is
evidence, and what relevance the circumstances have. But most theories
of justification can be fit into this broad schema, provided we are
liberal enough in interpreting the terms `good', `evidence' and
`circumstances'.) If Emma is not responsible for her response to the
evidence, then either we have to divorce justification from
responsibility, or we have to say that the concept of justification
being used in these discussions is defective.

We can summarise these considerations as a short argument. The following
formulation is from Sharon Ryan (\citeproc{ref-Ryan2003}{2003, 49}).

\begin{enumerate}
\def\labelenumi{\arabic{enumi}.}
\tightlist
\item
  If we have any epistemic obligations, then doxastic attitudes must
  sometimes be under our voluntary control.
\item
  Doxastic attitudes are never under our voluntarily control.
\item
  We do not have any epistemic obligations.
\end{enumerate}

Ryan goes on to reject both premises. (And she does so while
interpreting ``voluntary control'' to mean ``direct voluntary control'';
the response is not meant to sidestep Alston's argument.) Matthias Steup
(\citeproc{ref-Steup2000}{2000}, \citeproc{ref-Steup2008}{2008}) also
rejects both premises of this argument. I am more sympathetic to premise
1, but I (tentatively) agree with them, against what sometimes seems to
be orthodoxy, that premise 2 fails. That is, I endorse a kind of
doxastic voluntarism. (Just what kind will become clearer as we go
along.) There are four questions that anyone who endorses voluntarism,
and wants to argue that this matters epistemologically, should I think
answer. These are:

\begin{description}
\tightlist
\item[(A)]
What is wrong with current arguments against voluntarism?
\item[(B)]
What does the voluntariness of (some) beliefs consist in?
\item[(C)]
Which kinds of beliefs are voluntary?
\item[(D)]
What difference does the distinction between these classes make for
epistemology?
\end{description}

My answer to (A) will be similar to Ryan's, and to Steup's, but with I
think enough differences in emphasis to be worth working through. My
answer to (B), however, will be a little more different. I am going to
draw on some work on self-control to argue that some beliefs are
voluntary because they are the result of exercises of, or failures to
exercise, self-control. My answer to (C) is that what I will call
inferential beliefs are voluntary, while perceptual beliefs are not.
Ryan and Steup sometimes seem to suggest that even perceptual beliefs
are voluntary, and I do not think this is true. The consequence for
this, I will argue in answering (D), is that inferential beliefs should
be judged by how well they respond to the evidence, while perceptual
beliefs should be judged by how well they reflect reality. When an agent
has misleading evidence, their inferential beliefs might be fully
justified, but their perceptual beliefs, being misleading, are not.

I will detail my answers to those four questions in sections 2, 4, 6 and
7. In between I will discuss recent work on self-control (section 3) and
the contrast between my answer to (B) and other voluntarist answers
(section 5). In section 8 I will say how my partially voluntarist
position gives the externalist a way to avoid the New Evil Demon
problem. And in section 9 I will make a direct argument for the idea
that justification is a kind of praiseworthiness, not a kind of
blamelessness.

Before we start, I want to note two ways, other than Ryan's, of
formulating an argument against doxastic responsibility. These are going
to seem quite similar to Ryan's formulation, but I think they hide
important differences. The first version uses the idea that some doings
(or states) are volitional. That is, we do them (or are in them) because
we formed a volition to do so, and this volition causes the doing (or
state) in the right kind of way.

\begin{enumerate}
\def\labelenumi{\arabic{enumi}.}
\tightlist
\item
  If we have any epistemic obligations, then either the formation or
  maintenance of doxastic attitudes must sometimes be volitional.
\item
  The formation or maintenance of doxastic attitudes is never
  volitional.
\item
  We do not have any epistemic obligations.
\end{enumerate}

I will not argue against premise 2 of this argument, though Carl Ginet
(\citeproc{ref-Ginet1985}{1985}, \citeproc{ref-Ginet2001}{2001}) has
done so. But I think there's little to be said for premise 1. The
principle behind it is that we are only responsible for volitional
doings. And that principle is very dubious. We could run the kind of
regress arguments against it that Gilbert Ryle
(\citeproc{ref-Ryle1949}{1949}) offers. But it is simpler to note some
everyday counterexamples. Borrowing an example from Angela A. M. Smith
(\citeproc{ref-AngelaSmith2005}{2005}), if I forget a friend's birthday,
that is something I am responsible and blameworthy for, but forgetting a
birthday is not volitional. (Below I will offer a Rylean argument that
we are sometimes praiseworthy for doings that are not volitional.) So
this argument fails. Alternatively, we could run the argument by appeal
to freedom.

\begin{enumerate}
\def\labelenumi{\arabic{enumi}.}
\tightlist
\item
  If we have any epistemic obligations, then doxastic attitudes must
  sometimes be free.
\item
  Doxastic attitudes are never free.
\item
  We do not have any epistemic obligations.
\end{enumerate}

Premise 1 of this argument is more plausible. But, as we'll see
presently, premise 2 is not very plausible. Whether Descartes was right
that premise 2 is obviously false, it does seem on reflection very hard
to defend. So this argument fails. Ryan's formulation is interesting
because it is not clear just which of the premises fails. As I said, I
am going to suggest that premise 2 fails, and that doxastic attitudes
are voluntary. But this will turn on some fine judgments about the
voluntary/involuntary boundary. If I am wrong about those judgments,
then the arguments below will suggest that premise 1, not premise 2, in
Ryan's formulation fails. Either way though, the argument is
unsuccessful.

\section{Responding to the
Involuntarists}\label{responding-to-the-involuntarists}

There are two kinds of argument against the idea that belief is
voluntary. One kind, tracing back to Bernard Williams
(\citeproc{ref-WilliamsDecidingToBelieve}{1976}), holds that the
possibility of voluntary belief can be shown to be incoherent by
reflection on the concept of belief. This argument is no longer widely
endorsed. Nishi Shah (\citeproc{ref-Shah2002}{2002}) provides an
excellent discussion of the problems with Williams' argument, and I have
nothing to add to his work. I will focus on the other kind, that claims
we can see that belief is involuntary by observing differences between
beliefs and paradigm cases of voluntary actions. I will make three
objections to these arguments. First, the argument looks much less
plausible once we distinguish between having a belief and forming a
belief. Second, the argument seems to rely on inferring from the fact
that we do not do something (in particular, believe something that we
have excellent evidence is false) to the conclusion that we can not do
it. As Sharon Ryan (\citeproc{ref-Ryan2003}{2003}) points out, this
little argument overlooks the possibility that we will not do it. Third,
the argument relies on too narrow a conception of what is voluntary, and
when we get a more accurate grasp on that concept, we'll give up the
argument. Here is a representative version of the argument from William
Alston.

\begin{quote}
Can you, at this moment, start to believe that the United States is
still a colony of Great Britain, just by deciding to do so? \ldots{}
{[}S{]}uppose that someone offers you \$500,000,000 to believe it, and
you are much more interested in the money than in believing the truth.
Could you do what it takes to get that reward? . . . Can you switch
propositional attitudes toward that proposition just by deciding to do
so? It seems clear to me that I have no such power. Volitions,
decisions, or choosings don't hook up with anything in the way of
propositional attitude inauguration, just as they don't hook up with the
secretion of gastric juices or cell metabolism.
(\citeproc{ref-Alston1988}{Alston 1988, 122})
\end{quote}

Now Alston does note, just one page earlier, that what is really
relevant is whether our being in a state of belief is voluntary, not
whether the activity of belief formation is voluntary. But he thinks
nevertheless that issues about whether we can form beliefs, any old
beliefs it seems, voluntarily matters to the question about the
voluntariness of belief states.

If we think about what it is to be in a state voluntarily, this all
seems beside the point. We can see this by considering what it is to be
in a political state voluntarily. Consider Shane, who was born into
Victoria. His coming to be in Victoria was hence not, in any way,
voluntary. Shane is now a grown man, and he has heard many travellers'
tales of far away lands. But the apparent attractions of Sydney and
other places have no pull on Shane; he has decided to stay in Victoria.
If he has the capacity to leave Victoria, then Shane's continued
presence in Victoria is voluntary. Similarly, we are voluntarily in a
belief state if we have the capacity to leave it, but choose not to
exercise this capacity. Whether the belief was formed voluntarily is
beside the point.

If Shane leaves a state, the natural place to leave is for another
state, perhaps New South Wales or South Australia. It might be thought
that if we leave a belief state, we have to move into another belief
state. So to have this capacity to leave, we need the ability to form
beliefs voluntarily. Not at all. The capacity to become uncertain,
i.e.~to not be in any relevant belief state, is capacity enough. (If
Shane has a boat, and the capacity to flourish at sea, then perhaps he
too can have the capacity to leave Victoria without the capacity to go
into another state.)

But do we have the capacity to become uncertain? Descartes appeared to
think so; arguably the point of the First Meditation is to show us how
to exercise this capacity. Moreover, this capacity need not be one that
we exercise in any particularly nearby possible worlds. We might
exercise our freedom by always doing the right thing. As Descartes goes
on to say in the Fourth Meditation.

\begin{quote}
For in order to be free, there is no need for me to be capable of going
in each of two directions; on the contrary, the more I incline in one
direction -- either because I clearly understand that reasons of truth
and goodness point that way, or because of a divinely produced
disposition of my inmost thoughts -- the freer is my choice.
(\citeproc{ref-DescartesMeditations}{Descartes 1641/1996, 40})
\end{quote}

This seems like an important truth. Someone who is so sure of their own
interests and values, and so strong-willed as to always aim to promote
them, cannot in a certain sense act against their own self-interest and
values. But this does not make their actions in defence of those
interests and values unfree. If it did, we might well wonder what the
value of freedom was. And note that even if there's a sense that our
character could not have done otherwise, this in no way suggests their
actions are outside their control. Indeed, a person who systematically
promotes the interests and values they have seems an exemplar of an
agent in control. The character I am imagining here is in important
respects unlike normal humans. We know we can, and do, act against our
interests and values. But we can become more or less like them, and it
is important to remember, as Descartes does, that in doing so we do not
sacrifice freedom for values or interests.

John Cottingham (\citeproc{ref-Cottingham2002}{2002}) interprets
Descartes here as suggesting that there is a gap between free action and
voluntary action, contrasting his ``strongly compatibilist notion of
human freedom'' (350) with the ``doxastic involuntarism'' (355)
suggested by the following lines of the Third Meditation.

\begin{quote}
Yet when I turn to the things themselves which I think I perceive very
clearly, I am so convinced by them that I spontaneously declare: let
whoever can do so deceive me, he will never bring it about that I am
nothing, so long as I continue to think that I am something \ldots{}
(\citeproc{ref-DescartesMeditations}{Descartes 1641/1996, 25})
\end{quote}

Now there are two questions here. The first is whether Descartes
intended to draw this distinction. That is, whether Descartes thought
that the kind of free actions that he discusses in the Fourth
Meditations, the free action where we are incapable of going in the
other directions, are nevertheless involuntary. I do not have any
informed opinions about this question. The second is whether this kind
of consideration supports the distinction between the free and the
voluntary. And it seems to me that it does not. Just as Descartes says
the free person will be moved by reasons in the right way, it seems
natural to say that a person who acts voluntarily will be responsive to
reasons. Voluntary action does require freedom from certain kinds of
coercion, but the world does not coerce us when it gives us reason to
believe one thing rather than another. If we have voluntary control over
our beliefs, then we should be compelled by the sight of rain to believe
it is raining.

In her discussion of the puzzle of imaginative resistance, Tamar Szabó
Gendler (\citeproc{ref-Gendler2000}{2000}) notes that philosophers have
a tendency to read too much into intuitions about certain cases. What we
can tell from various thought experiments is that in certain
circumstances we will not do a certain thing. But getting from what we
will not do to what we can not do is a tricky matter, and it is a bad
mistake to infer from will not to can not too quickly. Matthias Steup
(\citeproc{ref-Steup2000}{2000}) points out that if you or I try to
stick a knife into our hand, we similarly will not do it. (I assume a
somewhat restricted readership here.) But this is no evidence that we
cannot do it. And Sharon Ryan (\citeproc{ref-Ryan2003}{2003}) notes that
we will not bring ourselves to run over pedestrians for no reason. For
most of us, our moral sense prevents acting quite this destructively.
Yet our continued avoiding of pedestrians is a series of free, even
voluntary, actions. We could run over the pedestrians, but we will not.
Since forming false beliefs is a form of self-harm, it is not surprising
that it has a similar phenomenology, even if it is genuinely possible.

It might be argued that we will engage in small forms of self-harm that
we can do when the financial rewards are great enough. So we should be
able to form this belief about the United States for a large amount sum
of money. But I suspect that the only way to exercise the capacity to
believe the United States is still a colony is by first suspending my
belief that it is no longer a colony. And the only way I can do that is
by generally becoming more sceptical of what I have been told over the
years. Once I get into such a sceptical mood, I will be sceptical of
claims that I will get half a billion dollars should I have this wild
political belief. So I will not form the belief in part because the
`promisor' lacks the capacity to sufficiently convince me that I will be
richly rewarded for doing so. This looks like a lack of capacity on
their part, not my part.

The final point to make about this argument, and those like it, is that
if we are to conclude that belief formation is never voluntary, then we
need to compare it to all kinds of voluntary action. And Alston really
only ever compares belief formation to volitional action. If this does
not exhaust the range of voluntary action, then belief formation might
be properly analogous to some other voluntary action. Indeed, this turns
out to be the case. To see so, we need to make a small detour through
modern work on self-control.

\section{How to Control Your Temper}\label{how-to-control-your-temper}

To start, let's consider three examples of a person failing to keep a
commitment they have made about what the good life is. The three ways
will be familiar from Gary Watson's discussion of recklessness, weakness
and compulsion Watson (\citeproc{ref-Watson1977}{1977}), and the
discussion of these cases by Jeanette Kennett and Michael Smith
(\citeproc{ref-KennettSmith1996a}{1996b},
\citeproc{ref-KennettSmith1996b}{1996a}). My characterisation of the
cases will turn out to differ a little from theirs, but the cases are
similar. Each of the examples concerns a character Murray, who has
decided that he should not swear around his young son Red. He resolves
to do this, and has been working on curbing his tendency to swear
whenever anything bad happens. But three times over the course of the
day he breaks his commitment.\footnote{The cases, especially the second,
  were inspired by Richard Holton's discussion of resolutions to prevent
  `automatic' actions like smoking or sleeping in. See Holton
  (\citeproc{ref-Holton2003}{2003}, \citeproc{ref-Holton2004}{2004}).}

The first time comes when Murray puts his hand down on a hot plate that
he did not realise was on. The searing pain undermines his self-control,
and he is unable to stop himself from swearing loudly through the pain.

The second time comes when Murray drops and breaks a wine glass. Murray
does not lose his self-control, but he does not exercise the
self-control he has. He temporarily forgets his commitment and so, quite
literally, curses his misfortune. On doing so he immediately remembers
that Red is around, and the commitment he has made, and regrets what he
did.

The third time comes on the tram home, when Murray gets into a
disagreement with a political opponent. Murray can not find the words to
express what he feels about the opponent without breaking his
commitment. So he decides, without much reason, that his need to express
what he feels outweighs his commitment, and starts describing his
opponent using language he would, all things considered, not have used
around young Red.

The first and third cases are close to textbook cases of compulsion and
recklessness. Note in the first case that when Murray reflects back on
what happened, he might be irritated that his work on reducing his
tendency to swear has not been more successful. But he will not be upset
that he did not exercise more self-control on that occasion. He did not
have, no normal person would have, the amount of self-control he would
have needed to stop swearing then. All that would help is having the
disposition to say different things when his self-control is defeated.
And that is not a disposition he can acquire on the spot.

I have described the first case as one where Murray's self-control is
undermined. This is a term taken from recent work by Richard Holton and
Stephen Shute (\citeproc{ref-HoltonShute2007}{2007}), who carefully
distinguish between self-control being undermined by a provocation, and
it being overwhelmed by a provocation. Undermining occurs when the
provocation causes the agent to have less self-control than they usually
have; overwhelming occurs when the provocation is too much for the
agent's control. The difference is relevant to them, because they are
interested in what it is for an agent to lose control. That seems to be
what happens here. After all, the things one would naturally do
afterwards (jumping around, screaming, swearing if one's so disposed) do
not seem particularly controlled by any measure.

Similarly I have accepted Watson's description of cases like the third
as instances of recklessness, but we should not think this necessarily
contrasts with weakness. It might be that in this case Murray is both
weak and reckless. He is not akratic, if we stipulatively define akrasia
as acting against one's better judgment. But if we accept Richard
Holton's view that weakness of will consists in being ``too ready to
reconsider their intentions'' (\citeproc{ref-Holton1999}{Holton 1999,
241}), then in this case Murray is weak-willed.\footnote{Whether Murray
  is akratic is a slightly more complicated question than I have
  suggested in the text. If akrasia is acting against one's judgment,
  then he is not; if akrasia is acting against one's \emph{considered}
  judgment, then he is. `Akrasia' is a technical term, so I do not think
  a huge amount turns on what we say about this question. There is an
  interesting historical precedent for Holton's theory of weakness of
  will. Ryle hints at a similar position to Holton's when he says
  ``Strength of will is a propensity the exercise of which consist in
  sticking to tasks' that is, in not being deterred or diverted.
  Weakness of will is having too little of this propensity.''
  (\citeproc{ref-Ryle1949}{1949, 73}) But the idea is not well developed
  in Ryle. We'll return below to the differences between Ryle's and
  Holton's theories.}This seems to be the right way to talk about the
case to me. With these details in place, we can talk about what's
crucial to this essay, the contrast with the second case.

In the second case Murray fails to exercise self-control. He could have
prevented himself from swearing in front of his son. Breaking a wine
glass is irritating, but it neither undermines nor, necessarily,
overwhelms self-control. Murray had the capacity to think about his
resolution to not swear in front of Red. And if he had exercised this
capacity, he would not have sworn when he did.

In the first case, Murray will only regret his lack of prior work at
changing his dispositions in cases where his control fails. In the
second case he will regret that, but he will also regret what he did on
that occasion, for he could have kept his resolution, had only he
thought of it. This regret seems appropriate, for in the second case he
did something wrong at the time he swore, as well perhaps as having done
something wrong earlier. (Namely, not having worked hard enough on his
dispositions.) This difference in regret does not constitute the
difference between compulsion and a case where self-control fails, but
it is pretty good evidence that this is a failure of self-control.

So the second case is not one where Murray was compelled. He had the
capacity to keep his commitment, and nothing was stopping him exercising
this control, but he failed to do so. His failure was a failure of
self-control. Murray's self-control is, in this case, overwhelmed by the
provocation. But it need not have been. Within some fairly broad limits,
how much self-control we exercise is up to us.\footnote{Holton
  (\citeproc{ref-Holton2003}{2003}) compares self-control to a muscle
  that we can exercise. We can make a similar point to the one in the
  text about physical muscles. If I try to lift a box of books and fail,
  that does not show I lack the muscular capacity to lift the box; I
  might not have been trying hard enough.} Murray's failure of
self-control is culpable because anyone with the capacity for
self-control Murray has could have avoided breaking his commitment. I am
not going to try to offer an analysis of what it is to have a capacity,
but I suspect something like the complicated counterfactual analysis
Kennett and Smith offer, and that Smith offers elsewhere
(\citeproc{ref-Smith1997}{M. Smith 1997},
\citeproc{ref-Smith2003}{2003}), is broadly correct.\footnote{Ryle
  (\citeproc{ref-Ryle1949}{1949, 71ff}) also offers a counterfactual
  account of capacities that seems largely accurate.}

Kennett and Smith stress two things about this capacity that are worth
noting here. First, having this kind of capacity is part of what it is
to be rational. That is, being rational requires thinking of the right
thing at the right time. As Ryle says, ``Intelligently reflecting how to
act is, among other things, considering what is pertinent and
disregarding what is inappropriate.''(\citeproc{ref-Ryle1949}{Ryle 1949,
31}) Second, Kennett and Smith note that exercises of this capacity
cannot be volitional. Following Davidson
(\citeproc{ref-Davidson1963}{1963}), they say they cannot be actions. I
find this terminology somewhat strained. Catching a fast moving ball is
an action, I would say, but it does not seem to be volitional. So I will
use `volitional action' for this Davidsonian sense of action.

Many recent philosophers have endorsed the idea that some of the mental
states for which we hold people responsible are not voluntary, or at
least are not volitional. Adams (\citeproc{ref-Adams1985}{1985}), Heller
(\citeproc{ref-Heller2000}{2000}), Owens
(\citeproc{ref-Owens2000}{2000}), and Hieronymi
(\citeproc{ref-Hieronymi2008}{2008}) note ways in which we appropriately
blame people for being in certain states, where being in that state is
not volitional. Something like this idea seems to be behind Ryle's
several regress arguments against the intellectualist legend. It just is
not true that what we do divides cleanly into outcomes of conscious
thought on the one hand, and mere bodily movements (a la digestion) on
the other.\footnote{As I read him, Ryle takes this fact to reveal an
  important weakness in Descartes' theory of mind.} Rather there is a
spectrum of cases from pure ratiocination at one end to pure bodily
movement at the other. And some of the things in the middle of this
spectrum are proper subjects of reactive attitudes. The focus in this
literature has been on blame, but some states in the middle of this
spectrum are also praiseworthy.

Consider some action that is strikingly imaginative, e.g.~a writer's apt
metaphor or, say, a cricket captain's imaginative field placements. It
seems that, assuming the field settings are successful, the captain
deserves praise for being so imaginative. But of course the captain did
not, really could not, first intend to imagine such field settings, then
carry out that intention. So something for which the captain deserves
praise, his act of imagination, is not volitional. So not all
praiseworthy things we do are volitional.

There are two responses to this argument that I can imagine, neither of
them particularly plausible. First, we might think that the captain's
imagination is simply a remarkable feature of nature, as the Great
Barrier Reef is. It is God, or Mother Nature, who should be praised, not
the captain. Now it seems fair to react to some attributes of a person
this way. A person does not deserve praise for having great eyesight,
for example. But such a reaction seems grossly inappropriate, almost
dehumanising, in this case. To be sure, we might also praise God or
Mother Nature for yielding such an imaginative person, but we'll do that
as well as rather than instead of, praising the person. Second, we might
praise the captain for his work in studying the game, and thinking about
possible ways to dismiss batsmen, rather than this particular action.
But if that is what we praise the captain for, we should equally praise
the captain's opponent, a hard working dullard. And that does not seem
right. The hard-working dullard deserves praise for his hard work in the
lead up, but the hard-working imaginative skipper deserves praise for
what he does in the game too. So reactive attitudes, particularly
praise, are appropriately directed at things people do even if these
things are not volitional.

The key point of this section then is that responsibility outruns
volition. Some actions are blameworthy because they are failures of
self-control. Some actions are praiseworthy because they are wonderful
feats of imagination. But neither failing to exercise self-control, nor
exercising imagination, needs be volitional is order to be a locus of
responsibility. I will argue in the next section that these
considerations support the idea of responsibility for beliefs.

\section{Voluntariness about Belief}\label{voluntariness-about-belief}

Here is a situation that will seem familiar to anyone who has spent time
in a student household. Mark is writing out the shopping list for the
weekly grocery shop. He goes to the fridge and sees that there is a
carton of orange juice in the fridge. He forms the belief that there is
orange juice in the fridge, and hence that he does not need to buy
orange juice. As it turns out both of these beliefs are false. One of
his housemates finishes off the orange juice, but stupidly put the empty
carton back in the fridge. When Mark finds this out, he is irritated at
his housemate, but he is also irritated at himself. He did not have to
draw the conclusion that there was orange juice in the fridge. He was,
after all, living in a student house where people do all sorts of dumb
things. That his housemate might have returned an empty container to the
fridge was well within the range of live possibilities. Indeed had he
even considered the possibility he would have thought it was a live
possibility, and checked whether the container was empty before forming
beliefs about what was needed for the shopping.

Examples like this can be easily multiplied. There are all sorts of
beliefs that we form in haste, where we could have stopped to consider
the various realistic hypotheses consistent with the evidence, and doing
so would have stopped us forming the belief. Indeed, unless one is a
real master of belief formation, it should not be too hard to remember
such episodes frequently from one's everyday life. These conclusions
that we leap to are voluntary beliefs; we could have avoided forming
them. And not only could we have avoided these formations, but we would
have if we had followed the methods for belief formation that we approve
of. That seems enough, to me, to say the formation is voluntary. This is
not the only way that voluntary doings, like calling a relevant
possibility to mind, can matter to belief. The next example will be a
little more controversial, but it points at the importance of dismissing
irrelevant possibilities.

Later that evening, Mark is watching his team, Geelong, lose another
football game. Geelong are down by eight goals with fifteen minutes to
go. His housemates are leaving to go see a movie, and want to know if
Mark wants to come along. He says that he is watching the end of the
game because Geelong might come back. One of his housemates replies, ``I
guess it is possible they'll win. Like it is possible they'll call you
up next week to see if you want a game with them.'' Mark replies,
``Yeah, you are right. This one's over. So, which movie?'' Mark does
nott just give in to his housemates, he forms the belief that Geelong
will lose. Later that night, when asked what the result of the game was,
he says that he did nott see the final score, but that Geelong lost by a
fair bit. (In a recent paper
(\citeproc{ref-Weatherson2005-WEACWD}{Weatherson 2005}) I go into a lot
more detail on the relation between not taking possibilities seriously,
and having beliefs. The upshot is that what Mark does can count as
belief formation, even if his credence that Geelong will lose does not
rise.)

Now it is tempting, or perhaps I should say that I am tempted, to view
the housemate as offering Mark a reason to believe that Geelong will
lose. We could view the housemate's comments as shorthand for the
argument that Geelong's winning is as likely as Mark's playing for
Geelong, and since the latter will not happen, neither will the former.
And maybe that is part of what the housemate is doing. But the larger
part is that she is mocking Mark for his misplaced confidence. And the
point of mocking someone, at least the point of constructive mockery
like this, is to get them to change their attitudes. Mark does so, by
ceasing to take seriously the possibility that Geelong will come back.
In doing so, he exercises a capacity he had for a while, the capacity to
cease taking this unserious possibility seriously, but needed to be
prompted to use.

In both cases I say Mark's belief formation is voluntary. In the first
case he forms the belief because he does not exercise his doxastic
self-control. He should have hesitated and not formed a belief until he
checked the orange juice. And he would have done so if only he'd thought
of the possibility that the container was empty. But he did not. And
just as things we do because we do not bring the right thing to mind,
like Murray's swearing in the second case, are voluntary and
blameworthy, Mark's belief is voluntary and blameworthy. In the second
case, he forms the belief by ceasing to take an unserious possibility
seriously. In most cases of non-perceptual, non-testimonial belief
formation, there is a counter-possibility that we could have taken
seriously. Skill at being a believer involves not taking extreme
possibilities, from Cartesian sceptical scenarios to unlikely
footballing heroics, seriously. Exercises of such skill are rarely, if
ever, volitional. But just like other mental activities that are not
volitional can be voluntary and praiseworthy, not taking an extreme
possibility seriously can be voluntary and praiseworthy.\footnote{Ryle
  (\citeproc{ref-Ryle1949}{1949, 29ff}) stresses the importance of
  calling the right things to mind to rational thought and action. I am
  using a case here where Mark deliberately casts an option from his
  mind, but the more general point is that what possibilities we call to
  mind is a crucial part of rational action, and can be praiseworthy or
  blameworthy, whether or not it is volitional.}

I have made two claims for Mark's beliefs in the above two cases. First,
they are instances of voluntary belief formation. In each case he could
have done otherwise, either by exercising or failing to exercise his
capacity to take various hypotheses seriously. Second, they are
appropriate subjects of praise and blame. I imagine some people will
agree with the second point but not the first. They will say that only
volitional actions are voluntary, even though things we do like bringing
relevant considerations to mind are praiseworthy or blameworthy. Such
people will agree with most of what I say in this paper. In particularly
they'll agree that the examples involving Mark undermine Alston's
argument against the applicability of deontological concepts in
epistemology. So I am not going to die in a ditch over just what we call
voluntary. That is, I won't fuss too much over whether we want to say
premise 2 in Ryan's formulation of the argument is shown to be false by
these examples (as I say) or premise 1 is shown to be false (as such an
objector will say.) I will just note that it is hard for such people to
say intuitive things about the second instance of Murray's swearing, and
this seems like a strong reason to not adopt their position.\footnote{Ryle
  seems to have taken an intermediate position. He holds, I think, the
  view that voluntary acts are culpable acts where we had the capacity
  to do otherwise (71). So Mark's belief about the orange juice is
  voluntary because he had the capacity to retain doubt, and nothing
  prevented him exercising it. But the belief about the football is not
  voluntary because we should not talk about praiseworthy acts being
  voluntary or involuntary. The last point is the kind of error that
  Grice (\citeproc{ref-Grice1989}{1989, Ch.1}) showed us how to avoid.}

\section{Ryan and Steup}\label{ryan-and-steup}

Sharon Ryan has a slightly different view. She thinks that the truth of
voluntarism consists in the fact that we hold certain beliefs
intentionally. She does not offer an analysis of what it is to do
something intentionally, except to say that consciously deciding to do
something is not necessary for doing it intentionally, but doing it
purposefully is (\citeproc{ref-Ryan2003}{Ryan 2003, 70--71}) In a
similar vein, she says ``When there's a car zooming toward me and I
believe that there is, I'm believing freely because I'm believing what I
mean to believe.'' (\citeproc{ref-Ryan2003}{Ryan 2003, 74}) This is said
to be an intentional, and I take it a voluntary, belief.

It seems to me that there's a large difference between things we
voluntarily do, and things we mean to do, or do purposefully. There are
several things we do voluntarily without meaning to do them. Murray's
swearing in the second example above is one instance. When we misspeak,
or (as I frequently do) mistype, we do things voluntarily without
meaning to do them. I do not mean by mistype cases where we simply hit
the wrong key, but such cases as where I write in one more negation than
I meant to, or, as I did earlier this evening, write ``S is justified in
believing that \emph{p}'' when I meant to write ``S is justified in
believing that she is justified in believing that \emph{p}.'' These are
voluntary actions because I had the capacity to get it right, but did
not exercise the capacity. But they are not things I meant to do. (I
suspect there are also cases where we do things because we mean to do
them, but they are not voluntary. These include cases where we train
ourselves to produce a reflexive response. But I will not stress such
cases here.)

Matthias Steup (\citeproc{ref-Steup2008}{2008}) argues that if
compatibilism is true about free action, then our beliefs are free. His
argument consists in running through the most plausible candidates to be
compatibilist notions of freedom, and for each candidate that is
plausible, showing that at least some of our beliefs satisfy the
purported conditions on free actions. I agree with a lot of what Steup
says, indeed this paper has been heavily influenced by what he says. But
one crucial analogy fails I think. Steup is concerned to reject the
premise that if φ-ing is free, one φs because one has formed the
intention to φ. His response centres around `automatic' actions, such as
the things we do when starting our drive to work: inserting the key,
shifting into reverse, etc.

\begin{quote}
The question is whether they are caused by any antecedently formed
intentions. I don't think they are. \ldots{} I didn't form an intention
to \ldots{} shift into reverse\ldots. I do things like that
automatically, without thinking about them, and I assume you do too. But
one can't form an intention to φ without thinking about φing \ldots{}
Just one more example: I'd like to see the person who, just before
brushing her teeth, forms the intention to unscrew the cap of the
toothpaste tube. (\citeproc{ref-Steup2008}{Steup 2008, 383})
\end{quote}

I suspect that Steup simply has to look in the mirror. It is true that
we do not usually form conscious intentions to shift into reverse, or
unscrew the cap, but not all intentions are conscious. If we were asked
later, perhaps by someone who thought we'd acted wrongly, whether we
intended to do these things, the natural answer is \emph{yes}. The best
explanation of this is that we really did have an intention to do them,
albeit an unconscious one. (I am indebted here to Ishani Maitra.)

Steup is right that free actions do not require a prior intention, but
his examples do not quite work. The examples I have used above are the
Rylean regress stoppers, such as acts of imagination, and actions that
we do because we did not think, like Murray's swearing. If asked later
whether he intended to say what he said, Murray would say yes in the
third example, but (I think) no in the first and second. Intuitively, I
think, he did not have such an intention.\footnote{If so, Murray is not
  weak-willed according to Holton's theory of will, but, since he does
  not keep his resolution, he is weak-willed according to Ryle's
  otherwise similar theory. This seems to be an advantage of Holton's
  theory over Ryle's. Murray's problem is not that his will was weak, it
  is that it was not called on. More generally, Ryle's identification of
  weakness of will with irresoluteness seems to fail for people who
  frequently \emph{forget} their resolutions. These people are surely
  irresolute, but (in agreement with Holton's theory) I think they are
  not weak-willed.}

\section{Involuntarism about Perceptual
Beliefs}\label{involuntarism-about-perceptual-beliefs}

In some early 1990s papers, Daniel Gilbert and colleagues defended a
rather startling thesis concerning the relation of comprehension and
belief (\citeproc{ref-GilbertKrullMalone1990}{Gilbert, Krull, and Malone
1990}; \citeproc{ref-Gilbert1991}{Gilbert 1991};
\citeproc{ref-GilbertTafarodiMalone1993}{Gilbert, Tafarodi, and Malone
1993}) Casual introspection suggests that when one reads or hears
something, one first comprehends it and then, if it is backed by
sufficient reasons, believes it. Gilbert
(\citeproc{ref-Gilbert1991}{1991}) argues against this seeming
separation of comprehension and belief, and in favour of a view said to
derive from Spinoza. When we comprehend a sentence, we add it to our
stock of beliefs. If the new belief is implausible given our old
beliefs, then we ``unbelieve'' it.\footnote{The evidence for this view
  is set out in Gilbert, Krull, and Malone
  (\citeproc{ref-GilbertKrullMalone1990}{1990}) and Gilbert, Tafarodi,
  and Malone (\citeproc{ref-GilbertTafarodiMalone1993}{1993}).}

We may picturesquely compare the two models of belief and comprehension
to two models for security. The way security works at a nightclub is
that anyone can turn up at the door, but only those cleared by the
guards are allowed in. On the other hand, the way security works at a
shopping mall is that anyone is allowed in, but security might remove
those it regards as undesirable. Intuitively, our minds work on the
nightclub model. A hypothesis can turn up and ask for admission, but it
has to be approved by our cognitive security before we adopt it as a
belief. Gilbert's position is that we work on the shopping mall model.
Any hypothesis put in front of us is allowed in, as a belief, and the
role of security is to remove troublemakers once they have been brought
inside.

Now I do not want to insist Gilbert's theory is correct. The
experimental evidence for it is challenged in a recent paper
(\citeproc{ref-HassonSimmonsTodorov2005}{Hasson, Simmons, and Todorov
2005}). But I do want to argue that if it is correct, then there is a
kind of belief that is clearly involuntary. We do not have much control
over what claims pass in front of our eyes, or to our ears. (We have
some indirect control over this -- we could wear eye shades and ear
plugs -- but no direct control, which is what's relevant.) If all such
claims are believed, these are involuntary beliefs. To be sure, nothing
Gilbert says implies that we can not quickly regain voluntary control
over our beliefs as we unbelieve the unwanted inputs. But in the time it
takes to do this, our beliefs are out of our control.

Gilbert's theory is rather contentious, but there are other kinds of
mental representations that it seems clear we can not help forming. In
\emph{The Modularity of Mind}, Jerry Fodor
(\citeproc{ref-Fodor1983}{1983}) has a long discussion of how the
various input modules that he believes to exist are not under our
voluntary control.\footnote{As he says, they have a mandatory operation.
  See pages 52-55 in particular, but the theme is central to the book.}
If I am sitting on a train opposite some people who are chatting away, I
can not help but hear what they say. (Unless, perhaps, I put my fingers
in my ear.) This is true not just in the sense that I can not help
receive the sound waves generated by their vocalisations. I also can not
help interpreting and comprehending what they are saying. Much as I
might like to not be bothered with the details of their lives, I can not
help but hear what they say as a string of English sentences. Not just
hearing, but hearing as happens automatically.

This automatic `hearing as' is not under my voluntary control. I do not
do it because I want to do it, or as part of a general plan that I
endorse or have chosen to undertake. It does not reflect any deep
features of my character. (Frankly I would much rather that I just heard
most of these conversations as meaningless noise, like the train's
sound.) But I do it, involuntarily, nonetheless. This involuntariness is
reflected in some of our practices. A friend tells me not to listen to
X, because X is so often wrong about everything. Next I see the friend I
say that I now believe that \emph{p}, and when the friend asks why, I
say it is because X said that \emph{p}. The friend might admonish me.
They will not admonish me for being within hearing range of X; that
might have been unavoidable. And, crucially, they will not admonish me
for interpreting X's utterances. Taken literally, that might be what
they were asking me not to do. But they'll know it was unavoidable. What
they were really asking me not to do was the one relevant thing that I
had control over, namely believe what X said.

As Fodor points out at length, both seeing as and hearing as are
generally outside voluntary control. Our perceptual systems, and by this
I am including verbal processing systems, quickly produce
representations that are outside voluntary control in any sense. If any
of these representations amount to beliefs, then there are some
involuntary beliefs that we have. So we might think that in the case
above, although it was up to me to believe that \emph{p}, it was not up
to me to believe that, say, X said that \emph{p}, because this belief
was produced by a modular system over which I have no control.

This is not the position that Fodor takes. He thinks that beliefs are
not produced by input modules. Rather, the non-modular part of the mind,
the central processor, is solely responsible for forming and fixing
beliefs. And the operation of this central processor is generally not
mandatory, at least not in the sense that the operation of the modules
is mandatory. Whether this is right seems to turn (in part) on a hard
question to do with the analysis of belief.

Let us quickly review Fodor's views on the behaviour of input modules.
The purpose of each module is to, within a specified domain, quickly and
automatically produce representations of the world. These are, as on the
nightclub model, then presented to cognition to be allowed in as beliefs
or not. Here is how Fodor puts it.

\begin{quote}
I am supposing that input systems offer central processes hypotheses
about the world, such hypotheses being responsive to the current, local
distribution of proximal stimulations. The evaluation of these
hypotheses in light of the rest of what one knows is one of the things
that central processes are for; indeed, it \emph{is} the fixation of
perceptual belief.(\citeproc{ref-Fodor1983}{Fodor 1983, 136})
\end{quote}

But these representations do not just offer hypotheses. They can also
guide action prior to being `approved' by the central processes. That,
at least, seems to be the point of Fodor's discussion of the
evolutionary advantages of having fast modules
(\citeproc{ref-Fodor1983}{Fodor 1983, 70--71}). The core idea is that
when one is at risk of being eaten by a panther, there is much to be
said for a quick, automatic, panther recognition device. But there is
just as much to be said for acting immediately on one's panther
recognition capacities rather than, say, searching for possible reasons
why this panther appearance might be deceptive. And browsing reason
space for such evidence of deceptions is just what central processes, in
Fodor's sense, do. So it seems the natural reaction to seeing a panther
should be, and is, guided more-or-less directly by the input modules not
central processes.

So these `hypotheses' are representations with belief-like direction of
fit, i.e.~they are responsive to the world, that guide action in the way
that beliefs do. These are starting to sound a lot like beliefs. Perhaps
we should take a Gilbert-style line and say that we automatically
believe what we perceive, and the role of Fodorian central processes is
not to accept or reject mere hypotheses, but to unbelieve undesirable
inputs.\footnote{To be clear, the position being considered here is not
  that we automatically believe \emph{p} when someone says \emph{p} to
  us, but that we automatically believe that they said that \emph{p}.}
There are a number of considerations that can be raised for and against
this idea, and perhaps our concept of belief is not fine enough to
settle the matter. But let's first look at three reasons for thinking
these inputs are not beliefs.

First, if they are beliefs then we are often led into inconsistency. If
we are looking at a scene we know to be illusory, then we might see
something as an \emph{F} when we know it is not an \emph{F}. If the
outputs of visual modules are beliefs, then we inconsistently believe
both that it is and is not \emph{F}. Perhaps this inconsistency is not
troubling, however. After all, one of the two inconsistent beliefs is
involuntary, so we are not responsible for it. So this inconsistency is
not a sign of irrationality, just a sign of defective perception. And
that is not something we should be surprised by; the case by definition
is one where perception misfires.

Second, the inputs do not, qua inputs, interact with other beliefs in
the right kind of way. Even if we believe that \emph{if p then q}, and
perceive that \emph{p}, we will not even be disposed to infer that
\emph{q} unless and until \emph{p} gets processed centrally. On this
point, see Stich (\citeproc{ref-Stich1978}{1978}) and Fodor
(\citeproc{ref-Fodor1983}{1983, 83--86}). The above considerations in
favour of treating inputs as beliefs turned heavily on the idea that
they have the same functional characteristics as paradigm beliefs. But
as David Braddon-Mitchell and Frank Jackson
(\citeproc{ref-DBMJackson2007}{2007, 114--23}) stress, functionalism can
only be saved from counterexamples if we include these inferential
connections between belief states in the functional charactisation of
belief. So from a functionalist point of view, the encapsulation of
input states counts heavily against their being beliefs.

Finally, if Fodor is right, then the belief-like representation of the
central processes form something like a natural kind. On the other hand,
the class consisting of these representations plus the representations
of the input modules looks much more like a disjunctive kind. Even if
all members of the class play the characteristic role of beliefs, we
might think it is central to our concept of belief that belief is a
natural kind. So these inputs should not count as beliefs.

On the other hand, we should not overestimate the role of central
processes, even if Fodor is right that central processes are quite
different to input systems. There are two related features of the way we
process inputs that point towards counting some inputs as beliefs, and
hence as involuntary beliefs. The first feature is that we do not have
to put any effort into believing what we see. On the contrary, as both
Descartes and Hume were well aware, we believe what we see by default,
and have to put effort into being sceptical. The second feature is that,
dramatic efforts aside, we can only be so sceptical. Perhaps sustained
reflection on the possibility of an evil demon can make us doubt all of
our perceptions at once. But in all probability, at least most of the
time, we can not doubt everything we see and hear.\footnote{As noted in
  the last footnote, when I talk here about what we hear, I mean to
  include propositions of the form \emph{S said that p}, not necessarily
  the \emph{p} that \emph{S} says.} We can perhaps doubt any perceptual
input we receive, but we can not doubt them all.

In the picturesque terms from above, we might think our security system
is less like a nightclub and more like the way customs appears to work
at many airports. (Heathrow Airport is especially like this, but I think
it is not that unusual.) Everyone gets a cursory glance from the customs
officials, but most people walk through the customs hall without even
being held up for an instant, and there are not enough officials to stop
everyone even if they wanted to. Our central processes, faced with the
overwhelming stream of perceptual inputs, are less the all-powerful
nightclub bouncer and more the overworked customs official, looking for
the occasional smuggler who should not be getting through.

The fact that inputs turn into fully fledged beliefs by default is some
reason to say that they are beliefs as they stand. It is noteworthy that
what Gilbert et al's experiments primarily tested was whether sentences
presented to subjects under cognitive load ended up as beliefs of the
subjects. Now this could be because comprehending a sentence implies, at
least temporarily, believing it. But perhaps a more natural reading in
the first instance is that inputted sentences turn into beliefs unless
we do something about it. Gilbert et al are happy inferring that in this
case, the inputs are beliefs until and unless we do that something. This
seems to be evidence that the concept of belief philosophers and
psychologists use include states that need to be actively rejected if
they are not to acquire all the paradigm features of belief. And that
includes the inputs from Fodorian modules.

That argument is fairly speculative, but we can make more of the fact
that subjects can not stop everything coming through. This implies that
there will be some long disjunctions of perceptual inputs that they will
end up believing no matter how hard they try. Any given input can be
rejected, but subjects only have so much capacity to block the flow of
perceptual inputs. So some long disjunctions will turn up in their
beliefs no matter how hard they try to keep them out. I think these are
involuntary beliefs.

So I conclude tentatively that perceptual inputs are involuntary
beliefs, at least for the time it would take the central processes to
evaluate them were it disposed to do so. And I conclude less tentatively
that subjects involuntarily believe long disjunctions of perceptual
inputs. So some beliefs are involuntary.

Space considerations prevent a full investigation of this, but there is
an interesting connection here to some late medieval ideas about
evidence. In a discussion of how Descartes differed from his medieval
influences, Matthew L. Jones writes ``For Descartes, the realignment of
one's life came about by training oneself to assent only to the evident;
for the scholastics, assenting to the evident required no exercise, as
it was automatic.''\footnote{Jones attributes this view to Scotus and
  Ockham, and quotes Pedro Fonseca as saying almost explicitly this in
  his commentary on Aristotle's \emph{Metaphysics}.}
(\citeproc{ref-Jones2006}{Jones 2006, 84}) There is much contemporary
interest in the analysis of evidence, with Timothy Williamson's proposal
that our evidence is all of our knowledge being a central focus
(\citeproc{ref-Williamson2000-WILKAI}{Williamson 2000} Ch. 9). I think
there's much to be said for using Fodor's work on automatic input
systems to revive the medieval idea that the evident is that which we
believe automatically, or perhaps it is those pieces of knowledge that
we came to believe automatically. As I said though, space prevents a
full investigation of these interesting issues.

\section{Epistemological
Consequences}\label{epistemological-consequences}

So some of our beliefs, loosely speaking the perceptual beliefs, are
spontaneous and involuntary, while other beliefs, the inferential
beliefs, are voluntary in that we have the capacity to check them by
paying greater heed to counter-possibilities. (In what follows it will
not matter much whether we take the spontaneous beliefs to include all
the perceptual inputs, or just the long disjunctions of perceptual
inputs that are beyond our capacity to reject. I will note the few
points where it matters significantly.) This has some epistemological
consequences, for the appropriate standards for spontaneous, involuntary
beliefs are different to the appropriate standards for considered,
reflective beliefs. I include in the latter category beliefs that were
formed when considered reflection was possible, but was not undertaken.

To think about the standards for spontaneous beliefs, start by
considering the criteria we could use to say that one kind of animal has
a better visual system than another. One dimension along which we could
compare the two animals concerns discriminatory capacity -- can one
animal distinguish between two things that the other cannot distinguish?
But we would also distinguish between two animals with equally
fine-grained visual representations, and the way we would distinguish is
in terms of the accuracy of those representations. Some broadly
externalist, indeed broadly reliabilist, approach has to be right when
it comes to evaluating the visual systems of different animals.

Things are a little more complicated when it comes to evaluating
individual visual beliefs of different animals, but it is still clear
that we will use externalist considerations. So imagine we are looking
for standards for evaluating particular visual beliefs of again fairly
basic animals. One very crude externalist standard we might use is that
a belief is good iff it is true. Alternatively, we might say that the
belief is good iff the process that produces it satisfied some
externalist standard, e.g.~it is generally reliable. Or we might, in a
way, combine these and say that the belief is good iff it amounts to
knowledge, incorporating both the truth and reliability standards. It is
not clear which of these is best. Nor is it even clear which, if any,
animals without sophisticated cognitive systems can be properly said to
have perceptual beliefs. (I will not pretend to be able to evaluate the
conceptual and empirical considerations that have been brought to bear
on this question.) But what is implausible is to say that these animals
have beliefs, and the relevant epistemic standards for evaluating these
beliefs are broadly internal.

This matters to debates about the justificatory standards for our
beliefs because we too have perceptual beliefs. And the way we form
perceptual beliefs is not that different from the way simple animals do.
(If the representations of input processes are beliefs, then it does not
differ in any significant way.) When we form beliefs in ways that
resemble those simple believers, most notably when we form perceptual
beliefs, we too are best evaluated using externalist standards. The
quality of our visual beliefs, that is, seems to directly track the
quality of our visual systems. And the quality of our visual system is
sensitive to external matters. So the quality of our visual beliefs is
sensitive to external matters.

On the other hand, when we reason, we are doing something quite
different to what a simple animal can do. A belief that is the product
of considered reflection should be assessed, inter alia, by assessing
the standards of the reflection that produced it. To a first
approximation, such a belief seems to be justified if it is well
supported by reasons. Some reasoners will be in reasonable worlds, and
their beliefs will be mostly true. Some reasoners will be in deceptive
worlds, and many of their beliefs will be false. But this does not seem
to change what we say about the quality of their reasoning. This, I take
it, is the core intuition behind the New Evil Demon problem, that we'll
address much more below.

So we're naturally led to a view where epistemic justification has a
bifurcated structure. A belief that is the product of perception is
justified iff the perception is reliable; a belief that is (or could
have been) the product of reflection is justified iff it is
well-supported by reasons.\footnote{There is a delicate matter here
  about individuating beliefs. If I look up, see, and hence believe it
  is raining outside, that is a perceptual belief. I could have recalled
  that it was raining hard a couple of minutes ago, and around here that
  kind of rain does not stop quickly, and formed an inferential belief
  that it was raining outside. I want to say that that would have been a
  different belief, although it has the same content. If I do not say
  that, it is hard to defend the position suggested here when it comes
  to the justificatory status of perceptual beliefs whose contents I
  could have otherwise inferred.} This position will remind many of
Ernest Sosa's view that there is animal knowledge, and higher knowledge,
or \emph{scientia} (\citeproc{ref-Sosa1991}{Sosa 1991},
\citeproc{ref-Sosa1997}{1997}). And the position is intentionally
similar to Sosa's. But there is one crucial difference. On my view,
there is just one kind of knowledge, and the two types of justification
kick in depending on the kind of knower, or the kind of knowing, that is
in question. If we simply form perceptual beliefs, without the
possibility of reconsidering them (in a timely manner), then if all goes
well, our beliefs are knowledge. Not some lesser grade of animal
knowledge, but simply knowledge. To put it more bluntly, if you're an
animal, knowledge just is animal knowledge. On the other hand, someone
who has the capacity (and time) to reflect on their perceptions, and
fails to do so even though they had good evidence that their perceptions
were unreliable, does not have knowledge. Their indolence defeats their
knowledge. Put more prosaically, the more you are capable of doing, the
more that is expected of you.

\section{The New Evil Demon Problem}\label{the-new-evil-demon-problem}

The primary virtue of the above account, apart from its intuitive
plausibility, is that it offers a satisfactory response to the New Evil
Demon argument. The response in question is not new; it follows fairly
closely the recent response due to Clayton Littlejohn
(\citeproc{ref-Littlejohn2009}{2009}), who in turn builds on responses
due to Kent Bach (\citeproc{ref-Bach1985}{1985}) and Mylan Engel
(\citeproc{ref-Engel1992}{1992}). But I think it is an attractive
feature of the view defended in this paper that it coheres so nicely
with a familiar and attractive response to the argument.

The New Evil Demon argument concerns victims of deception who satisfy
all the internal standards we can imagine for being a good epistemic
agent. So they are always careful to avoid making fallacious inferences,
they respect the canons of good inductive and statistical practice, they
do not engage in wishful thinking, and so on. The core intuition of the
New Evil Demon argument is that although these victims do not have
knowledge (because their beliefs are false), they do have justified
beliefs. Since the beliefs do not satisfy any plausible externalist
criteria of justification, we conclude that no externalist criteria can
be correct. The argument is set out by Stewart Cohen
(\citeproc{ref-Cohen1984}{1984}).

A fairly common response is to note that even according to externalist
epistemology there will be some favourable epistemic property that the
victim's beliefs have, and this can explain our intuition that there is
something epistemically praiseworthy about the victim's beliefs. My
approach is a version of this, one that is invulnerable to recent
criticisms of the move. For both this response and the criticism to it,
see James Pryor (\citeproc{ref-Pryor2001-PRYHOR}{2001}). I am going to
call my approach the agency approach, because the core idea is that the
victim of the demon is in some sense a good doxastic agent, in that all
their exercises of doxastic agency are appropriate, although their
perception is quite poor and this undermines their beliefs.

As was noted above, the quality of our visual beliefs is sensitive to
external matters. This is true even for the clear-thinking victim of
massive deception. Denying that the victim's visual beliefs are as good
as ours is not at all implausible; indeed intuition strongly supports
the idea that they are not as good. What they are as good at as we are
is exercising their epistemic agency. That is to say, they are excellent
epistemic agents. But since there is more to being a good believer than
being a good epistemic agent, there is also for example the matter of
being a good perceiver, they are not as good at believing as we are.

So the short version of my response to the New Evil Demon problem is
this. There are two things we assess when evaluating someone's beliefs.
We evaluate how good an epistemic agent they are. And we evaluate how
good they are at getting evidence from the world. Even shorter, we
evaluate both their collection and processing of evidence. Externalist
standards for evidence collection are very plausible, as is made clear
when we consider creatures that do little more than collect evidence.
The intuitions that the New Evil Demon argument draws on come from
considering how we process evidence. When we consider beliefs that are
the products of agency, such as beliefs that can only be arrived at by
extensive reflection, we naturally consider the quality of the agency
that led to those beliefs. In that respect a victim might do as well as
we do, or even better. But that is no threat to the externalist
conclusion that they are not, all things considered, as good at
believing as we are.

As I mentioned earlier, this is similar to a familiar response to the
argument that James Pryor considers and rejects. He considers someone
who says that what is in common to us and the clear-thinking victim is
that we are both epistemically blameless. The objection he considers
says that the intuitions behind the argument come from confusing this
notion of being blameless with the more general notion of being
justified. This is similar to my idea that the victim might be a good
epistemic agent while still arriving at unjustified beliefs because they
are so bad at evidence collection. But Pryor argues that this kind of
deontological approach cannot capture all of the intuitions around the
problem.

Pryor considers three victims of massive deception. Victim A uses all
sorts of faulty reasoning practices to form beliefs, practices that A
could, if they were more careful, could see were faulty. Victim B was
badly `brought up', so although they use methods that are subtly
fallacious, there is no way we could expect B to notice these mistakes.
Victim C is our paradigm of good reasoning, though of course C still has
mostly false beliefs because all of their apparent perceptions are
misleading. Pryor says that both B and C are epistemically blameless; C
because they are a perfect reasoner and B because they cannot be blamed
for their epistemic flaws. But we intuit that C is better, in some
epistemic respects, than B. So there is some internalist friendly kind
of evaluation that is stronger than being blameless. Pryor suggests that
it might be \emph{being justified}, which he takes to be an internalist
but non-deontological concept.

The agency approach has several resources that might be brought to bear
on this case. For one thing, even sticking to deontological concepts we
can make some distinctions between B and C. We can, in particular, say
that C is epistemically praiseworthy in ways that B is not. Even if B
cannot be blamed for their flaws, C can be praised for not exemplifying
those flaws. It is consistent with the agency approach to say that C can
be praised for many of their epistemic practices while saying that,
sadly, most of C's beliefs are unjustified because they are based on
faulty evidence, or on merely apparent evidence.

The merits of this kind of approach can be brought out by considering
how we judge agents who are misled about the nature of the good. Many
philosophers think that it is far from obvious which character traits
are virtues and which are vices. Any particular example is bound to be
controversial, but I think it should be uncontroversial that there are
some such examples. So I will assume that, as Simon Keller
(\citeproc{ref-Keller2005}{2005}) suggests, it is true but unobvious
that patriotism is not a virtue but a vice.

Now consider three agents D, E and F. D takes patriotism to extremes,
developing a quite hostile strand of nationalism, which leads to
unprovoked attacks on non-compatriots. E is brought up to be patriotic,
and lives this way without acting with any particular hostility to
foreigners. F is brought up the same way, but comes to realise that
patriotism is not at all virtuous, and comes to live according to purely
cosmopolitan norms. Now it is natural to say that D is blameworthy in a
way that E and F are not. As long as it seems implausible to blame E for
not working through the careful philosophical arguments that tell
against following patriotic norms, we should not blame E for being
somewhat patriotic. But it is also natural to say that F is a better
agent than either D or E. That is because F exemplifies a virtue,
cosmopolitanism, that D and E do not, and does not exemplify a vice,
patriotism, that D and E do exemplify. F is in this way praiseworthy,
while D and E are not.

This rather strongly suggests that when agents are misled about norms, a
gap will open up between blamelessness and praiseworthiness. We can say
that Pryor's victim C is a better epistemic agent than A or B, because
they are praiseworthy in a way that A and B are not. And we can say this
even though we do not say that B is blameworthy and we do not say that
being a good epistemic agent is all there is to being a good believer.

At this point the internalist might respond with a new form of the
argument. A victim of deception is, they might intuit, just as
praiseworthy as a regular person, if they perform the same inferential
moves. I think at this point the externalist can simply deny the
intuitions. In general, praiseworthiness is subject to a degree of luck.
(Arguably blameworthiness is as well, but saying so sounds somewhat more
counterintuitive than saying praiseworthiness is a matter of luck.) For
example, imagine two people dive into ponds in which they believe there
are drowning children. The first saves two children. The second was
mistaken; there are no children to be rescued in the pond they dive
into. Both are praiseworthy for their efforts, but they are not equally
praiseworthy. The first, in particular, is praiseworthy for rescuing two
children. As we saw in the examples of the writer and the good cricket
captain above, praiseworthiness depends on outputs as well as inputs,
and if the victim of deception produces beliefs that are defective,
i.e.~false, then through no fault of their own they are less
praiseworthy.

\section{Praise and Blame}\label{praise-and-blame}

As Pryor notes, many philosophers have thought that a deontological
conception of justification supports an internalist theory of
justification. I rather think that is mistaken, and that at least one
common deontological understanding of what justification is entails a
very strong kind of externalism. This is probably a reason to not adopt
that deontological understanding.

Assume, for reductio, that S's belief that \emph{p} is justified iff S
is blameless in believing that \emph{p}. I will call this principle J=B
to note the close connection it posits between justification and
blamelessness. Alston (\citeproc{ref-Alston1988}{1988}) seems to
identify the deontological conception of justification with J=B, or at
least to slide between the two when offering critiques. But one of
Alston's own examples, the `culturally isolated tribesman', suggests a
principle that can be used to pull these two ideas apart. The example,
along with Pryor's three brains case, suggests that A1 is true.

\begin{description}
\item[A1]
It is possible for S to have a justified but false belief that her
belief in \emph{p} is justified.
\end{description}

A1 is a special instance of the principle that justification does not
entail truth. Some externalists about justification will want to reject
the general principle, but all internalists (and indeed most
externalists) will accept it. Now some may think that the general
principle is right, but that beliefs about what we are justified in
believing are special, and if they are justified they are true. But such
an exception seems intolerably ad hoc. If we can have false but
justified beliefs about some things, then presumably we can have false
but justified beliefs about our evidence, since in principle our
evidence could be practically anything. So the following situation seems
possible; indeed it seems likely that something of this form happens
frequently in real life. S has a false but justified belief that
\emph{e} is part of her evidence. S knows both that anyone with evidence
\emph{e} is justified in believing \emph{p} in the absence of defeaters,
and that there are no defeaters present. So S comes to believe, quite
reasonably, that she is justified in believing that \emph{p}. But S does
not have this evidence, and in fact all of her evidence points towards
\textasciitilde{}\emph{p}.\footnote{I am assuming here that evidence of
  evidence need not be evidence. This seems likely to be true. In
  Bayesian terms, something can raise the probability of \emph{e}, while
  lowering the probability of \emph{p}, even though the probability of
  \emph{p} given \emph{e} is greater than the probability of \emph{p}.
  Bayesian models are not fully general, but usually things that are
  possible in Bayesian models are possible in real life.} So it is false
that she is justified in believing \emph{p}.

The following principle seems to be a reasonable principle concerning
blameless inference.

\begin{description}
\item[A2]
If S blamelessly believes that she is justified in believing that
\emph{p}, and on the basis of that belief comes to believe that
\emph{p}, then she is blameless in believing that \emph{p}.
\end{description}

This is just a principle of transfer of blameworthiness. The quite
natural thought is that you do not become blameworthy by inferring from
\emph{I am justified in believing p} to \emph{p}. This inference is
clearly not necessarily truth-preserving, but that is not a constraint
on inferences that transfer blameworthiness, since not all ampliative
inferences are blameworthy. (Indeed, many are praiseworthy.) And it is
hard to imagine a less blameworthy ampliative inference schema than this
one.

We can see this more clearly with an example of A2. Suzy sees a lot of
\emph{Fs} and observes they are all \emph{Gs}. She infers that it is
justified for her to conclude that all \emph{Fs} are \emph{Gs}. Now it
turns out this is a bad inference. In fact, \emph{G} is a gruesome
predicate in her world, so that is not a justified inference. But Suzy,
like many people, does not have the concept of gruesomeness, and without
it had no reason to suspect that this would be a bad inference. So she
is blameless. If all that is correct, it is hard to imagine that she
becomes blameworthy by actually inferring from what she has so far that
all \emph{Fs} are in fact \emph{Gs}. Perhaps you might think her
original inference, that it is justified to believe all \emph{Fs} are
\emph{Gs}, was blameworthy, but blame can not kick in for the first time
when she moves to the first order belief.

I am now going to derive a contradiction from A1, A2 and J=B, and a
clearly consistent set of assumptions about a possible case of belief.

\begin{enumerate}
\def\labelenumi{\arabic{enumi}.}
\tightlist
\item
  S justifiedly, but falsely, believes that she is justified in
  believing \emph{p}. (Assumption - A1)
\item
  On the basis of this belief, S comes to believe that \emph{p}.
  (Assumption)
\item
  S blamelessly believes that she is justified in believing that
  \emph{p}. (1, J=B)
\item
  S blamelessly believes that \emph{p}. (2, 3, A2)
\item
  S is justified in believing that \emph{p}. (4, J=B)
\item
  It is false that S is justified in believing that \emph{p}. (1)
\end{enumerate}

One of A1, A2 and J=B has to go. If you accept J=B, I think it has got
to be A1, since A2 is extremely plausible. But A1 only fails if we
accept quite a strong externalist principle of justification, namely
that justification entails truth. More precisely, we're led to the view
that justification entails truth when it comes to propositions about our
own justification. But as we saw above, that pretty directly implies
that justification entails truth when it comes to propositions about our
own evidence. And, on the plausible assumption that evidence can be
practically anything, that leads to there being a very wide range of
cases where justification entails truth. So J=B entails this strong form
of externalism.

This does not mean that internalists cannot accept a deontological
conception of justification. But the kind of deontological conception of
justification that is left standing by this argument is quite different
to J=B, and I think to existing deontological conceptions of
justification. Here's what it would look like. First, we say that a
belief's being justified is not a matter of it being blameless, but a
matter of it being in a certain way praiseworthy. Second, we say that
the inference from \emph{I am justified in believing that p} to \emph{p}
is not praiseworthy if the premise is false. So if we tried to run the
above argument against J=P (the premise that justified beliefs are
praiseworthy) it would fail at step 4. So anyone who wants to hold that
justification is (even in large part) deontological, and wants to accept
that justification can come apart from truth, should hold that
justification is a kind of praiseworthiness, not a kind of
blamelessness.

\subsection*{References}\label{references}
\addcontentsline{toc}{subsection}{References}

\phantomsection\label{refs}
\begin{CSLReferences}{1}{0}
\bibitem[\citeproctext]{ref-Adams1985}
Adams, Robert Merrihew. 1985. {``Involuntary Sins.''}
\emph{Philosophical Review} 94 (1): 3--31. doi:
\href{https://doi.org/10.2307/2184713}{10.2307/2184713}.

\bibitem[\citeproctext]{ref-Alston1988}
Alston, William. 1988. {``The Deontological Conception of Epistemic
Justification.''} \emph{Philosophical Perspectives} 2: 257--99. doi:
\href{https://doi.org/10.2307/2214077}{10.2307/2214077}.

\bibitem[\citeproctext]{ref-Bach1985}
Bach, Kent. 1985. {``A Rationale for Reliabilism.''} \emph{Monist} 68
(2): 246--63. doi:
\href{https://doi.org/10.5840/monist198568224}{10.5840/monist198568224}.

\bibitem[\citeproctext]{ref-DBMJackson2007}
Braddon-Mitchell, David, and Frank Jackson. 2007. \emph{The Philosophy
of Mind and Cognition, {Second Edition}}. Malden, MA: Blackwell.

\bibitem[\citeproctext]{ref-Cohen1984}
Cohen, Stewart. 1984. {``Justification and Truth.''} \emph{Philosophical
Studies} 46 (3): 279--95. doi:
\href{https://doi.org/10.1007/bf00372907}{10.1007/bf00372907}.

\bibitem[\citeproctext]{ref-Cottingham2002}
Cottingham, John. 2002. {``Descartes and the Voluntariness of Belief.''}
\emph{Monist} 85 (3): 343--60. doi:
\href{https://doi.org/10.5840/monist200285323}{10.5840/monist200285323}.

\bibitem[\citeproctext]{ref-Davidson1963}
Davidson, Donald. 1963. {``Actions, Reasons and Causes.''} \emph{Journal
of Philosophy} 60 (23): 685--700. doi:
\href{https://doi.org/10.2307/2023177}{10.2307/2023177}.

\bibitem[\citeproctext]{ref-DescartesMeditations}
Descartes, René. 1641/1996. \emph{Meditations on First Philosophy, {Tr.
John Cottingham}}. Cambridge: Cambridge University Press.

\bibitem[\citeproctext]{ref-DescartesPrinciples}
---------. 1644/2003. \emph{The Principles of Philosophy, {Tr. John
Veitch}}. Champaign, IL: Project Gutenberg.

\bibitem[\citeproctext]{ref-Engel1992}
Engel, Mylan. 1992. {``Personal and Doxastic Justification in
Epistemology.''} \emph{Philosophical Studies} 67 (2): 133--50. doi:
\href{https://doi.org/10.1007/bf00373694}{10.1007/bf00373694}.

\bibitem[\citeproctext]{ref-Fodor1983}
Fodor, Jerry A. 1983. \emph{The Modularity of Mind}. Cambridge, MA: MIT
Press.

\bibitem[\citeproctext]{ref-Gendler2000}
Gendler, Tamar Szabó. 2000. {``The Puzzle of Imaginative Resistance.''}
\emph{Journal of Philosophy} 97 (2): 55--81. doi:
\href{https://doi.org/10.2307/2678446}{10.2307/2678446}.

\bibitem[\citeproctext]{ref-Gilbert1991}
Gilbert, Daniel T. 1991. {``How Mental Systems Believe.''}
\emph{American Psychologist} 46 (2): 107--19. doi:
\href{https://doi.org/10.1037//0003-066x.46.2.107}{10.1037//0003-066x.46.2.107}.

\bibitem[\citeproctext]{ref-GilbertKrullMalone1990}
Gilbert, Daniel T., Douglas S. Krull, and Patrick S. Malone. 1990.
{``Unbelieving the Unbelievable: Some Problems in the Rejection of False
Information.''} \emph{Journal of Personality and Social Psychology} 59
(4): 601--13. doi:
\href{https://doi.org/10.1037//0022-3514.59.4.601}{10.1037//0022-3514.59.4.601}.

\bibitem[\citeproctext]{ref-GilbertTafarodiMalone1993}
Gilbert, Daniel T., Romin W. Tafarodi, and Patrick S. Malone. 1993.
{``You Can't Not Believe Everything You Read.''} \emph{Journal of
Personality and Social Psychology} 65 (2): 221--33. doi:
\href{https://doi.org/10.1037//0022-3514.65.2.221}{10.1037//0022-3514.65.2.221}.

\bibitem[\citeproctext]{ref-Ginet1985}
Ginet, Carl. 1985. {``Contra Reliabilism.''} \emph{Monist} 68 (2):
175--87. doi:
\href{https://doi.org/10.5840/monist198568218}{10.5840/monist198568218}.

\bibitem[\citeproctext]{ref-Ginet2001}
---------. 2001. {``Deciding to Believe.''} In \emph{Knowledge, Truth
and Duty}, edited by Matthias Steup, 63--76. Oxford: Oxford University
Press.

\bibitem[\citeproctext]{ref-Grice1989}
Grice, H. Paul. 1989. \emph{Studies in the Way of Words}. Cambridge,
MA.: Harvard University Press.

\bibitem[\citeproctext]{ref-HassonSimmonsTodorov2005}
Hasson, Uri, Joseph P. Simmons, and Alexander Todorov. 2005. {``Believe
It or Not: On the Possibility of Suspending Belief.''}
\emph{Psychological Science} 16 (7): 566--71. doi:
\href{https://doi.org/10.1111/j.0956-7976.2005.01576.x}{10.1111/j.0956-7976.2005.01576.x}.

\bibitem[\citeproctext]{ref-Heller2000}
Heller, Mark. 2000. {``Hobartian Voluntarism: Grounding a Deontological
Conception of Epistemological Justification.''} \emph{Pacific
Philosophical Quarterly} 81 (2): 130--41. doi:
\href{https://doi.org/10.1111/1468-0114.00099}{10.1111/1468-0114.00099}.

\bibitem[\citeproctext]{ref-Hieronymi2008}
Hieronymi, Pamela. 2008. {``Responsibility for Believing.''}
\emph{Synthese} 161 (3): 357--73. doi:
\href{https://doi.org/10.1007/s11229-006-9089-x}{10.1007/s11229-006-9089-x}.

\bibitem[\citeproctext]{ref-Holton1999}
Holton, Richard. 1999. {``Intention and Weakness of Will.''} \emph{The
Journal of Philosophy} 96 (5): 241--62. doi:
\href{https://doi.org/10.2307/2564667}{10.2307/2564667}.

\bibitem[\citeproctext]{ref-Holton2003}
---------. 2003. {``How Is Strength of Will Possible?''} In
\emph{Weakness of Will and Varities of Practical Irrationality}, edited
by Sarah Stroud and Christine Tappolet, 39--67. Oxford: Oxford
University Press.

\bibitem[\citeproctext]{ref-Holton2004}
---------. 2004. {``Rational Resolve.''} \emph{Philosophical Review} 113
(4): 507--35. doi:
\href{https://doi.org/10.1215/00318108-113-4-507}{10.1215/00318108-113-4-507}.

\bibitem[\citeproctext]{ref-HoltonShute2007}
Holton, Richard, and Stephen Shute. 2007. {``Self-Control in the Modern
Provocation Defence.''} \emph{Oxford Journal of Legal Studies} 27 (1):
49--73. doi:
\href{https://doi.org/10.1093/ojls/gql034}{10.1093/ojls/gql034}.

\bibitem[\citeproctext]{ref-Jones2006}
Jones, Matthew L. 2006. \emph{The Good Life in the Scientific
Revolution: Descartes, Pascal, Leibniz and the Cultivation of Virtue}.
Chicago: University of Chicago Press.

\bibitem[\citeproctext]{ref-Keller2005}
Keller, Simon. 2005. {``Patriotism as Bad Faith.''} \emph{Ethics} 115
(3): 563--92. doi:
\href{https://doi.org/10.1086/428458}{10.1086/428458}.

\bibitem[\citeproctext]{ref-KennettSmith1996b}
Kennett, Jeanette, and Michael Smith. 1996a. {``Frog and Toad Lose
Control.''} \emph{Analysis} 56 (2): 63--73. doi:
\href{https://doi.org/10.1111/j.0003-2638.1996.00063.x}{10.1111/j.0003-2638.1996.00063.x}.

\bibitem[\citeproctext]{ref-KennettSmith1996a}
---------. 1996b. {``Philosophy and Commonsense: The Case of Weakness of
Will.''} In \emph{The Place of Philosophy in the Study of Mind}, edited
by Michaelis Michael and John O'Leary-Hawthorne, 141--57. Norwell, MA:
Kluwer. doi:
\href{https://doi.org/10.1017/CBO9780511606977.005}{10.1017/CBO9780511606977.005}.

\bibitem[\citeproctext]{ref-Littlejohn2009}
Littlejohn, Clayton. 2009. {``The Externalist's Demon.''} \emph{Canadian
Journal of Philosophy} 39 (3): 399--434. doi:
\href{https://doi.org/10.1353/cjp.0.0054}{10.1353/cjp.0.0054}.

\bibitem[\citeproctext]{ref-Owens2000}
Owens, David. 2000. \emph{Reason Without Freedom: The Problem of
Epistemic Responsibility}. New York: Routledge.

\bibitem[\citeproctext]{ref-Pryor2001-PRYHOR}
Pryor, James. 2001. {``Highlights of Recent Epistemology.''}
\emph{British Journal for the Philosophy of Science} 52 (1): 95--124.
doi: \href{https://doi.org/10.1093/bjps/52.1.95}{10.1093/bjps/52.1.95}.

\bibitem[\citeproctext]{ref-Ryan2003}
Ryan, Sharon. 2003. {``Doxastic Compatibilism and the Ethics of
Belief.''} \emph{Philosophical Studies} 114 (1-2): 47--79. doi:
\href{https://doi.org/10.1023/A:1024409201289}{10.1023/A:1024409201289}.

\bibitem[\citeproctext]{ref-Ryle1949}
Ryle, Gilbert. 1949. \emph{The Concept of Mind}. New York: Barnes;
Noble.

\bibitem[\citeproctext]{ref-Shah2002}
Shah, Nishi. 2002. {``Clearing Space for Doxastic Voluntarism.''}
\emph{The Monist} 85 (3): 436--45. doi:
\href{https://doi.org/10.5840/monist200285326}{10.5840/monist200285326}.

\bibitem[\citeproctext]{ref-AngelaSmith2005}
Smith, Angela M. 2005. {``Responsibility for Attitudes: Activity and
Passivity in Mental Life.''} \emph{Ethics} 115 (2): 236--71. doi:
\href{https://doi.org/10.1086/426957}{10.1086/426957}.

\bibitem[\citeproctext]{ref-Smith1997}
Smith, Michael. 1997. {``A Theory of Freedom and Responsibility.''} In
\emph{Ethics and Practical Reason}, edited by Garrett Cullity and Berys
Gaut, 293--317. Oxford: Oxford University Press.

\bibitem[\citeproctext]{ref-Smith2003}
---------. 2003. {``Rational Capacities.''} In \emph{Weakness of Will
and Varities of Practical Irrationality}, edited by Sarah Stroud and
Christine Tappolet, 17--38. Oxford: Oxford University Press.

\bibitem[\citeproctext]{ref-Sosa1991}
Sosa, Ernest. 1991. \emph{Knowledge in Perspective}. New York: Cambridge
University Press.

\bibitem[\citeproctext]{ref-Sosa1997}
---------. 1997. {``Reflective Knowledge in the Best Circles.''}
\emph{Journal of Philosophy} 94 (8): 410--30. doi:
\href{https://doi.org/10.2307/2564607}{10.2307/2564607}.

\bibitem[\citeproctext]{ref-Steup2000}
Steup, Matthias. 2000. {``Doxastic Voluntarism and Epistemic
Deontology.''} \emph{Acta Analytica} 15 (1): 25--56.

\bibitem[\citeproctext]{ref-Steup2008}
---------. 2008. {``Doxastic Freedom.''} \emph{Synthese} 161 (3):
375--92. doi:
\href{https://doi.org/10.1007/s11229-006-9090-4}{10.1007/s11229-006-9090-4}.

\bibitem[\citeproctext]{ref-Stich1978}
Stich, Stephen. 1978. {``Beliefs and Subdoxastic States.''}
\emph{Philosophy of Science} 45 (4): 499--518. doi:
\href{https://doi.org/10.1086/288832}{10.1086/288832}.

\bibitem[\citeproctext]{ref-Watson1977}
Watson, Gary. 1977. {``Skepticism about Weakness of Will.''}
\emph{Philosophical Review} 86 (3): 316--39. doi:
\href{https://doi.org/10.2307/2183785}{10.2307/2183785}.

\bibitem[\citeproctext]{ref-Weatherson2005-WEACWD}
Weatherson, Brian. 2005. {``{Can We Do Without Pragmatic
Encroachment?}''} \emph{Philosophical Perspectives} 19 (1): 417--43.
doi:
\href{https://doi.org/10.1111/j.1520-8583.2005.00068.x}{10.1111/j.1520-8583.2005.00068.x}.

\bibitem[\citeproctext]{ref-WilliamsDecidingToBelieve}
Williams, Bernard. 1976. {``Deciding to Believe.''} In \emph{Problems of
the Self}, 136--51. Cambridge: Cambridge University Press.

\bibitem[\citeproctext]{ref-Williamson2000-WILKAI}
Williamson, Timothy. 2000. \emph{{Knowledge and its Limits}}. Oxford
University Press.

\end{CSLReferences}



\noindent Published in\emph{
Journal of Philosophy}, 2008, pp. 540-569.

\end{document}
