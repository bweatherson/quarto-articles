% Options for packages loaded elsewhere
% Options for packages loaded elsewhere
\PassOptionsToPackage{unicode}{hyperref}
\PassOptionsToPackage{hyphens}{url}
%
\documentclass[
  11pt,
  letterpaper,
  DIV=11,
  numbers=noendperiod,
  twoside]{scrartcl}
\usepackage{xcolor}
\usepackage[left=1.1in, right=1in, top=0.8in, bottom=0.8in,
paperheight=9.5in, paperwidth=7in, includemp=TRUE, marginparwidth=0in,
marginparsep=0in]{geometry}
\usepackage{amsmath,amssymb}
\setcounter{secnumdepth}{3}
\usepackage{iftex}
\ifPDFTeX
  \usepackage[T1]{fontenc}
  \usepackage[utf8]{inputenc}
  \usepackage{textcomp} % provide euro and other symbols
\else % if luatex or xetex
  \usepackage{unicode-math} % this also loads fontspec
  \defaultfontfeatures{Scale=MatchLowercase}
  \defaultfontfeatures[\rmfamily]{Ligatures=TeX,Scale=1}
\fi
\usepackage{lmodern}
\ifPDFTeX\else
  % xetex/luatex font selection
  \setmathfont[]{Garamond-Math}
\fi
% Use upquote if available, for straight quotes in verbatim environments
\IfFileExists{upquote.sty}{\usepackage{upquote}}{}
\IfFileExists{microtype.sty}{% use microtype if available
  \usepackage[]{microtype}
  \UseMicrotypeSet[protrusion]{basicmath} % disable protrusion for tt fonts
}{}
\usepackage{setspace}
% Make \paragraph and \subparagraph free-standing
\makeatletter
\ifx\paragraph\undefined\else
  \let\oldparagraph\paragraph
  \renewcommand{\paragraph}{
    \@ifstar
      \xxxParagraphStar
      \xxxParagraphNoStar
  }
  \newcommand{\xxxParagraphStar}[1]{\oldparagraph*{#1}\mbox{}}
  \newcommand{\xxxParagraphNoStar}[1]{\oldparagraph{#1}\mbox{}}
\fi
\ifx\subparagraph\undefined\else
  \let\oldsubparagraph\subparagraph
  \renewcommand{\subparagraph}{
    \@ifstar
      \xxxSubParagraphStar
      \xxxSubParagraphNoStar
  }
  \newcommand{\xxxSubParagraphStar}[1]{\oldsubparagraph*{#1}\mbox{}}
  \newcommand{\xxxSubParagraphNoStar}[1]{\oldsubparagraph{#1}\mbox{}}
\fi
\makeatother


\usepackage{longtable,booktabs,array}
\usepackage{calc} % for calculating minipage widths
% Correct order of tables after \paragraph or \subparagraph
\usepackage{etoolbox}
\makeatletter
\patchcmd\longtable{\par}{\if@noskipsec\mbox{}\fi\par}{}{}
\makeatother
% Allow footnotes in longtable head/foot
\IfFileExists{footnotehyper.sty}{\usepackage{footnotehyper}}{\usepackage{footnote}}
\makesavenoteenv{longtable}
\usepackage{graphicx}
\makeatletter
\newsavebox\pandoc@box
\newcommand*\pandocbounded[1]{% scales image to fit in text height/width
  \sbox\pandoc@box{#1}%
  \Gscale@div\@tempa{\textheight}{\dimexpr\ht\pandoc@box+\dp\pandoc@box\relax}%
  \Gscale@div\@tempb{\linewidth}{\wd\pandoc@box}%
  \ifdim\@tempb\p@<\@tempa\p@\let\@tempa\@tempb\fi% select the smaller of both
  \ifdim\@tempa\p@<\p@\scalebox{\@tempa}{\usebox\pandoc@box}%
  \else\usebox{\pandoc@box}%
  \fi%
}
% Set default figure placement to htbp
\def\fps@figure{htbp}
\makeatother


% definitions for citeproc citations
\NewDocumentCommand\citeproctext{}{}
\NewDocumentCommand\citeproc{mm}{%
  \begingroup\def\citeproctext{#2}\cite{#1}\endgroup}
\makeatletter
 % allow citations to break across lines
 \let\@cite@ofmt\@firstofone
 % avoid brackets around text for \cite:
 \def\@biblabel#1{}
 \def\@cite#1#2{{#1\if@tempswa , #2\fi}}
\makeatother
\newlength{\cslhangindent}
\setlength{\cslhangindent}{1.5em}
\newlength{\csllabelwidth}
\setlength{\csllabelwidth}{3em}
\newenvironment{CSLReferences}[2] % #1 hanging-indent, #2 entry-spacing
 {\begin{list}{}{%
  \setlength{\itemindent}{0pt}
  \setlength{\leftmargin}{0pt}
  \setlength{\parsep}{0pt}
  % turn on hanging indent if param 1 is 1
  \ifodd #1
   \setlength{\leftmargin}{\cslhangindent}
   \setlength{\itemindent}{-1\cslhangindent}
  \fi
  % set entry spacing
  \setlength{\itemsep}{#2\baselineskip}}}
 {\end{list}}
\usepackage{calc}
\newcommand{\CSLBlock}[1]{\hfill\break\parbox[t]{\linewidth}{\strut\ignorespaces#1\strut}}
\newcommand{\CSLLeftMargin}[1]{\parbox[t]{\csllabelwidth}{\strut#1\strut}}
\newcommand{\CSLRightInline}[1]{\parbox[t]{\linewidth - \csllabelwidth}{\strut#1\strut}}
\newcommand{\CSLIndent}[1]{\hspace{\cslhangindent}#1}



\setlength{\emergencystretch}{3em} % prevent overfull lines

\providecommand{\tightlist}{%
  \setlength{\itemsep}{0pt}\setlength{\parskip}{0pt}}



 


\setlength\heavyrulewidth{0ex}
\setlength\lightrulewidth{0ex}
\usepackage[automark]{scrlayer-scrpage}
\clearpairofpagestyles
\cehead{
  Brian Weatherson
  }
\cohead{
  Indecisive Decision Theory
  }
\ohead{\bfseries \pagemark}
\cfoot{}
\makeatletter
\newcommand*\NoIndentAfterEnv[1]{%
  \AfterEndEnvironment{#1}{\par\@afterindentfalse\@afterheading}}
\makeatother
\NoIndentAfterEnv{itemize}
\NoIndentAfterEnv{enumerate}
\NoIndentAfterEnv{description}
\NoIndentAfterEnv{quote}
\NoIndentAfterEnv{equation}
\NoIndentAfterEnv{longtable}
\NoIndentAfterEnv{abstract}
\renewenvironment{abstract}
 {\vspace{-1.25cm}
 \quotation\small\noindent\emph{Abstract}:}
 {\endquotation}
\newfontfamily\tfont{EB Garamond}
\addtokomafont{disposition}{\rmfamily}
\addtokomafont{title}{\normalfont\itshape}
\let\footnoterule\relax

\makeatletter
\renewcommand{\@maketitle}{%
  \newpage
  \null
  \vskip 2em%
  \begin{center}%
  \let \footnote \thanks
    {\itshape\huge\@title \par}%
    \vskip 0.5em%  % Reduced from default
    {\large
      \lineskip 0.3em%  % Reduced from default 0.5em
      \begin{tabular}[t]{c}%
        \@author
      \end{tabular}\par}%
    \vskip 0.5em%  % Reduced from default
    {\large \@date}%
  \end{center}%
  \par
  }
\makeatother
\RequirePackage{lettrine}

\renewenvironment{abstract}
 {\quotation\small\noindent\emph{Abstract}:}
 {\endquotation\vspace{-0.02cm}}

\setmainfont{EB Garamond Math}[
  BoldFont = {EB Garamond SemiBold},
  ItalicFont = {EB Garamond Italic},
  RawFeature = {+smcp},
]

\newfontfamily\scfont{EB Garamond Regular}[RawFeature=+smcp]
\renewcommand{\textsc}[1]{{\scfont #1}}

\renewcommand{\LettrineTextFont}{\scfont}
\KOMAoption{captions}{tableheading}
\makeatletter
\@ifpackageloaded{caption}{}{\usepackage{caption}}
\AtBeginDocument{%
\ifdefined\contentsname
  \renewcommand*\contentsname{Table of contents}
\else
  \newcommand\contentsname{Table of contents}
\fi
\ifdefined\listfigurename
  \renewcommand*\listfigurename{List of Figures}
\else
  \newcommand\listfigurename{List of Figures}
\fi
\ifdefined\listtablename
  \renewcommand*\listtablename{List of Tables}
\else
  \newcommand\listtablename{List of Tables}
\fi
\ifdefined\figurename
  \renewcommand*\figurename{Figure}
\else
  \newcommand\figurename{Figure}
\fi
\ifdefined\tablename
  \renewcommand*\tablename{Table}
\else
  \newcommand\tablename{Table}
\fi
}
\@ifpackageloaded{float}{}{\usepackage{float}}
\floatstyle{ruled}
\@ifundefined{c@chapter}{\newfloat{codelisting}{h}{lop}}{\newfloat{codelisting}{h}{lop}[chapter]}
\floatname{codelisting}{Listing}
\newcommand*\listoflistings{\listof{codelisting}{List of Listings}}
\makeatother
\makeatletter
\makeatother
\makeatletter
\@ifpackageloaded{caption}{}{\usepackage{caption}}
\@ifpackageloaded{subcaption}{}{\usepackage{subcaption}}
\makeatother
\usepackage{bookmark}
\IfFileExists{xurl.sty}{\usepackage{xurl}}{} % add URL line breaks if available
\urlstyle{same}
\hypersetup{
  pdftitle={Indecisive Decision Theory},
  pdfauthor={Brian Weatherson},
  hidelinks,
  pdfcreator={LaTeX via pandoc}}


\title{Indecisive Decision Theory}
\author{Brian Weatherson}
\date{2021}
\begin{document}
\maketitle
\begin{abstract}
A decisive decision theory says that in any given decision problem,
either one choice is best, or all the choices are equally good. I argue
against this, and in favor of indecisive decision theories. The main
example that is used is a game with a demon (who is good at predicting
others' moves) that has multiple equilibria. It is argued that all the
plausible decisive theories violate a principle of dynamic consistency
that we should accept.
\end{abstract}


\setstretch{1.1}
\section{Decisiveness}\label{decisiveness}

Say a decision theory is \textbf{decisive} iff for any decision problem,
it says either:

\begin{enumerate}
\def\labelenumi{\arabic{enumi}.}
\tightlist
\item
  There is a uniquely best choice, and rationality requires choosing
  it.; or
\item
  There is a non-singleton set of choices each of which is tied for
  being best, and each of which can be permissibly chosen.
\end{enumerate}

A decision theory is \textbf{decisive over binary choices} iff it
satisfies this condition for all decision problems where there are just
two choices. Most decision theories in the literature are decisive, and
of those that are not, most of them are at least decisive over binary
choices. I'm going to argue that the correct decision theory, whatever
it is, is indecisive. It is not, I'll argue, even decisive over binary
choices.

The argument turns on a pair of very similar decision problems. Each
problem has the following structure. There is a human Player, and a
predictor, who I'll call Doctor. Doctor is very good, as good as the
demon in Newcomb's problem, at predicting Player's behavior. Doctor will
make two decisions. First, they will opt-in (which I'll write as I for
In), or opt-out (which I'll write as O for out). If they opt-out, Doctor
gets \$1, and Player gets \$100. (Assume both Doctor and Player prefer
more money to less, and indeed that over these small sums there is more
or less no declining marginal utility of money.) If they opt-in, this
will be publicly announced, an another game will be played. Each of
Doctor and Player will (independently) pick a letter: A or B. Doctor
will aim to predict Player's choice, and will be rewarded iff that
prediction is correct. Table~\ref{tbl-main-game-state} is the payout
table for the possible outcomes of this game.

\begin{longtable}[]{@{}cccc@{}}
\caption{Main game, in state description
form}\label{tbl-main-game-state}\tabularnewline
\toprule\noalign{}
Human Pick & Doctor Pick & Human Reward & Doctor Reward \\
\midrule\noalign{}
\endfirsthead
\toprule\noalign{}
Human Pick & Doctor Pick & Human Reward & Doctor Reward \\
\midrule\noalign{}
\endhead
\bottomrule\noalign{}
\endlastfoot
A & A & \$6 & \$4 \\
A & B & \$0 & \$0 \\
B & A & \$3 & \$0 \\
B & B & \$4 & \$1 \\
\end{longtable}

If you prefer this in the way we standardly present games in normal
form, it looks like Table~\ref{tbl-main-game-table}, with the human as
Row, doctor as Column, and in each cell the human's payout is listed
first. (All payouts are in dollars)

\begin{longtable}[]{@{}lcc@{}}
\caption{Main game, in table description
form}\label{tbl-main-game-table}\tabularnewline
\toprule\noalign{}
& A & B \\
\midrule\noalign{}
\endfirsthead
\toprule\noalign{}
& A & B \\
\midrule\noalign{}
\endhead
\bottomrule\noalign{}
\endlastfoot
\textbf{A} & 6, 4 & 0, 0 \\
\textbf{B} & 3, 0 & 4, 1 \\
\end{longtable}

Doctor is good at predictions, and prefers more money to less. So if
Doctor predicts Player will choose A, they will opt-in, and also play A,
getting \$4. But what if they predict Player will choose B. They will
get \$1 either by opting-out, or by opting-in and choosing B. Since they
are indifferent in this case, let's say they will flip a coin to decide
which way to go. And Player knows that this is how Doctor will decide,
should Doctor predict Player will choose B. Doctor will also flip a coin
to decide what prediction to make if they think Player is completely
indifferent between the choices, and Player also knows this.

Now I said there were going to be two problems. Here's how they are
created. Two Players, Amsterdam and Brussels, will play the game. They
have identical utility functions over money, and identical prior
probability distributions about what Doctor will do conditional on each
of their choices. That's to say, they both think the probability that
Doctor will be wrong is vanishingly small. But Brussels is busy and has
to run to the bank, so they have to write their decision in an envelope
that will be revealed, after Doctor chooses A or B, iff Doctor opts-in
in the game with Brussels. Amsterdam, on the other hand, gets to see
Doctor's decision about whether to opt-in or opt-out, and then (if
Doctor opts-in) has to write their decision in an envelope that will be
revealed after Doctor chooses A or B. For ease of reference, call
Brussels's decision the \emph{early} decision and Amsterdam's decision
the \emph{late} decision.

Here is the core philosophical premise in the argument to follow.

\begin{description}
\item[The Core Premise]
If there is is precisely one permissible choice for Amsterdam, and one
permissible choice for Brussels, then it must be the same choice. That
is, if each of them is obliged to choose a particular letter, it must be
the same letter. It can't be that one is obliged to choose A, and the
other obliged to choose B.
\end{description}

I'm calling \textbf{The Core Premise} a premise, though in the next
section I'll offer a few arguments for it, in case you don't think it is
obviously correct. (I sort of think it is obviously correct, but I'm
really not going to rely on you sharing that intuition.) And I'll argue
that any decisive theory (that meets minimal coherence standards) has to
violate this constraint. Some decisive theories say Amsterdam should
choose A and Brussels should choose B, but a few say the reverse. And a
few (otherwise implausible) decisive theories say that Amsterdam and
Brussels should do the same thing in this game, but different things in
games with the same structure but slightly different payouts. In every
case, a decisive theory will make some incoherent pair of
recommendations, and so is mistaken.

\textbf{The Core Premise} is a conditional, but any decisive theory that
denies that the choices are tied will meet the condition. So from now on
in arguing against decisive theories I'll mostly just interpret
\textbf{The Core Premise} as saying Amsterdam and Brussels must make the
same choice. The main thing to check for is that a theory doesn't say
the choices are tied. None of the theories I'll look at will say that,
but it's an important thing to check.

Before we start there are four pieces of important housekeeping.

First, the definition of decisiveness referred to options being tied.
For the definition to be interesting, it can't just be that options are
tied if each is rationally permissible. Then a decisive theory would
just be one that either says one option is mandatory or many options are
permissible. To solve this problem, I'll borrow a technique from Ruth
Chang (\citeproc{ref-Chang2002}{2002}). Some options are \textbf{tied}
iff either is permissible, but this permissibility is sensitive to
sweetening. That is, if options \emph{X} and \emph{Y} are tied, then for
any positive ε, the agent prefers \emph{X}~+~ε to \emph{Y}. If either
choice is permissible even if \emph{X} is `sweetened', i.e.., replaced
in the list of choices by X~+~ε, we'll say they aren't tied. My thesis
then is that the correct decision theory says that sometimes there are
multiple permissible options, and each of them would still be
permissible if one of them was sweetened.

Second, there is an important term in the definition of decisiveness
that I haven't clarified: \textbf{decision problem}. Informally, the
argument assumes that in setting out the proble facing Amsterdam and
Brussels is indeed a decision problem. More formally, I'm assuming it
suffices to specify a decision problem to describe the following four
values.

\begin{itemize}
\tightlist
\item
  What choices Player has;
\item
  What possible states of the world there are (where it is understood
  that the choices of Player make no causal impact on which state is
  actual)\footnote{My personal preference is to understand states
    historically. For any proposition relevant to the decision, a state
    determines its truth value if it is about the past, or its chance at
    the start of deliberation if it is about the future. And then causal
    independence comes in from a separate presupposition that there is
    no backwards causation. But I definitely won't assume this picture
    of states here.};
\item
  What the probability is of being in any state conditional on making
  each choice; and
\item
  What return Player gets for each choice-state pair.
\end{itemize}

Most recent papers on decision theory do not precisely specify what they
count as a decision problem, but they seem to implicitly share this
assumption, since they will often describe a vignette that settles
nothing beyond these four things as a decision problem. And that's what
I did as well! You should understand this as being part of the
definition of decisiveness. This implies that there are two ways to
reject decisiveness.

First, a theory could say that these four conditions underspecify a real
decision problem. In any real situation, decision theory has a decisive
verdict, but it rests on information, typically information about
Player, not settled by these four values. I'll say a theory that goes
this route is \emph{intrapersonally} decisive, but not
\emph{interpersonally} decisive.

Second, a theory could say that no matter how much one adds to the
specification, there will be cases where the correct decision theory
does not issue a verdict. Such a theory is not \emph{intrapersonally}
decisive. There is nothing you could add about the person to the
specification of a decision problem which decides what they should do,
or even which options are tied. I want to ultimately defend such a view,
and this paper is a part of the defence. But it's a proper part. Nothing
I say here rules out mere interpersonal indecisiveness. That's an
argument for another day. Today, we have enough to be getting on with.

Third, I have set up this problem quite explicitly as a game, with
another player - Doctor. I don't think this is particularly big deal,
though I gather not everyone agrees. In this respect (among others) I'm
following William Harper (\citeproc{ref-Harper1986}{1986}), who
recommended treating Newcomb's Problem as a game. It's not clear why it
wouldn't be a game. The Newcomb demon makes a choice, and if you assume
the demon gets utility 1 from correct predictions and utility 0 from
incorrect predictions, they make the choice that maximises their return
given their beliefs about what the other (human) player will do. So game
theoretic techniques can and should apply. I think any problem with a
predictive demon is best thought of as a game where the demonic player
gets utility 1 or 0 depending on whether their prediction is right. Here
the predictive player, Doctor, has a utility function with slightly more
structure. But if we think decision theory should apply in cases where
there is a predictor around, it should still apply when that predictor
has preferences with slightly more structure.

Fourth, this project was inspired by reading David Pearce's argument
against `Single Solution Concepts' (\citeproc{ref-Pearce1983}{Pearce
1983}). My initial plan was to simply translate his argument into
decision theoretic terms and use it as an argument for indecisiveness. I
ended up with a somewhat different argument to his, one that draws
heavily on Brian Skyrms's work on Stag Hunts
(\citeproc{ref-Skyrms1990}{1990}, \citeproc{ref-Skyrms2001}{2001},
\citeproc{ref-Skyrms2004}{2004}). But the project started out from an
idea of Pearce's. And as you'll see in the conclusion, it will end with
a related idea of is.

\section{Defending The Core Premise}\label{defending-the-core-premise}

In this section I'll offer three arguments in defence of \textbf{The
Core Premise}. The first will be to argue that Amsterdam and Brussels
have in a key sense the same choice, the second will argue that
violations of \textbf{The Core Premise} will violate the Sure Thing
Principle, and the third is that violations of \textbf{The Core Premise}
lead to people being willing to pay to avoid information. The arguments
will make frequent use of the following equivalence. A player must
choose a letter iff the player prefers choosing that letter. So I'll
move freely from saying that a theory says Amsterdam should choose X to
saying Amsterdam should prefer X. I don't think this should be
controversial, but it's worth noting. Onto the three arguments.

Think about what Brussels is doing when writing in the envelope. They
know that the envelope will only be opened if Doctor opts-in. If Doctor
does, then what they play will determine their payout. So they should
imagine that Doctor has opted-in, and act accordingly. But in that
imaginative situation, they will do the same thing as if they knew that
Doctor had opted-in. That is, they'll do the same thing Amsterdam will
do. There isn't any difference, for purposes of choice, between
supposing that Doctor has opted-in, and learning that Doctor has
opted-in. And Brussels should suppose that Doctor has opted-in. After
all, they are being asked what to contribute to the game if, and only
if, Doctor opts-in. So the two choices are effectively the same, and
they should get the same verdict. That's the first argument.

Assume that a theory says Amsterdam should do X, but Brussels should do
Y, where X and Y are distinct. Now ask the theory, what should Brussels
prefer conditional on Doctor opting-in, and conditional on Doctor
opting-out. Since Amsterdam should choose X, conditional on Doctor
opting-in, Brussels thinks X is better than Y. And conditional on Doctor
opting-out, Brussels is indifferent between X and Y, so thinks X is as
at least as good as Y. The sure thing principle (or at least the version
that matters here) says that if Brussels knows that precisely one of a
set of outcomes obtains, and X is at least as good as Y conditional on
each member of the set, then X is at least as good as Y overall. But
that contradicts the assumption that Brussels should choose Y. So that's
the second argument.

Assume again that a theory says Amsterdam should do X, but Brussels
should do Y, where X and Y are distinct. Now imagine a third player,
Cardiff. Cardiff isn't busy, like Brussels. But Cardiff hasn't yet found
out whether Doctor has opted-in. They are offered the chance to buy ear
plugs, so they won't hear the announcement of whether Doctor opts-in or
opts-out. They should like to get those, since right now they prefer Y
to X, but there's a chance that they'll hear Doctor has opted-in, which
will leave them in the same situation as Amsterdam, and hence they'll
choose X. And there is nothing they can gain from hearing the
announcement. But this is absurd - a player should not pay to avoid
relevant information about the game. So that's the third argument.

Now this last argument has one caveat. I didn't calculate how much
Player should pay for the ear plugs. That turns out to vary a little
depending on just which decisive theory we are looking at. A theory may
say that Cardiff should not pay anything for the ear plugs, since they
are certain Doctor will opt-out. This third argument isn't particularly
effective, I think, against those theories. But it works well, and I
feel is the strongest argument, against some other theories. But we'll
have to look case by case at just how much a theory would recommend
Cardiff pay for the ear plugs.

So that's the defence of \textbf{The Core Premise}. What I'll now show
is that a wide range of decisive theories violate it, and so we can
conclude they are false.

\section{Early and Late Choices}\label{early-and-late-choices}

To see why theories might violate \textbf{The Core Premise}, it's
helpful to set out explicitly the choices that Amsterdam and Brussels
face. And we'll treat Doctor largely as a non-player character, just as
the demon is typically treated in Newcomb's problem. So from now on the
columns will not be Doctor's choices, but what Doctor predicts the human
player chooses. And we'll assume Doctor maximises their financial return
given a correct prediction. It's easy to set out the choice Amsterdam
faces; it's just the embedded game with some notational differences.
I've presented it in Table~\ref{tbl-main-game-ams}.

\begin{longtable}[]{@{}lcc@{}}
\caption{The decision problem Amsterdam
faces}\label{tbl-main-game-ams}\tabularnewline
\toprule\noalign{}
& PA & PB \\
\midrule\noalign{}
\endfirsthead
\toprule\noalign{}
& PA & PB \\
\midrule\noalign{}
\endhead
\bottomrule\noalign{}
\endlastfoot
\textbf{A} & 6 & 0 \\
\textbf{B} & 3 & 4 \\
\end{longtable}

In Table~\ref{tbl-main-game-ams} I've written \textbf{PA} and
\textbf{PB} in the columns to indicated that A or B is Predicted. But in
this game that makes little difference, since Doctor will do whatever
they predict Amsterdam will do. Things are a little different for
Brussels. If Doctor predicts that Brussels has written B, they will flip
a coin to decide whether to opt-out, or opt-in. So we can't write
Brussels's return in actual dollars, since we don't know how the coin
lands. But we can write the return in expected dollars, and we assume
that Brussels is after all trying to maximise expected dollars. (We'll
come back to this assumption in the next section.) So the table Brussels
faces looks like this.

\begin{longtable}[]{@{}lcc@{}}
\caption{The decision problem Brussels
faces}\label{tbl-main-game-bru}\tabularnewline
\toprule\noalign{}
& PA & PB \\
\midrule\noalign{}
\endfirsthead
\toprule\noalign{}
& PA & PB \\
\midrule\noalign{}
\endhead
\bottomrule\noalign{}
\endlastfoot
\textbf{A} & 6 & 50 \\
\textbf{B} & 3 & 52 \\
\end{longtable}

If Doctor predicts B, then Player has a 1 in 2 chance of getting \$100,
and a 1 in 2 chance of getting the payout from the previous game. So
their average payout is \$50 if they play A, and \$52 if they play B.
Hence the values in the right hand column here.

So what \textbf{The Core Premise} says is that if each of these games
has a uniquely rational choice, it must be the same choice. As we'll
see, a lot of theories do not satisfy this constraint.

\section{Evidential Decision Theory}\label{evidential-decision-theory}

Given a perfect predictor, Evidential Decision Theory says that the only
payout values that matter are those in the main diagonal, running from
northwest to southeast. So Amsterdam should choose A, since they'll
expect to get \$6 from A and \$4 from B. But Brussels should choose B,
since they'll expect to get \$6 from A and \$52 from B. So Evidential
Decision Theory violates \textbf{The Core Premise}, and hence is
mistaken.

When I introduced Cardiff's case, I said we had to check what a
particular theory said about what Cardiff would pay for the earplugs. So
let's do that for Evidential Decision Theory. If Cardiff thinks they
would be told that Doctor has opted-in, they would pay up to \$46 to
avoid that information, since they think they will get \$52 without the
information and \$6 with it. But maybe they would be told that Doctor
opted-out. It turns out the assumption they would be told that is
incoherent, and Cardiff knows it. If they are told Doctor has opted-out,
Doctor will know they are indifferent between the options. And in that
case, Doctor will flip a coin to decide what `prediction' to make. But
if Doctor thinks it is 50/50 what Cardiff will do, they have an expected
return of \$2 from opting-in and choosing A, but an expected return of
\$1 from opting-out. So they will opt-in and choose A, contradicting the
assumption that Cardiff will be told they opted-out. And Cardiff can do
all this reasoning. So Cardiff can predict that if they are told
anything, it will be that Doctor has opted-in, putting them in the same
position as Amsterdam. But if they are told nothing, they are in the
same position as Brussels. And they prefer, by \$46, being in the same
position as Brussels.

Now this is not really a new objection to Evidential Decision Theory.
You can find similar points being made about the strange behaviour of
Evidential Decision Theory in dynamic choice settings as far back as
Gibbard and Harper (\citeproc{ref-GibbardHarper1978}{1978}). The details
of my argument are a bit different, but ultimately they rest on the same
foundations. I think those are perfectly solid foundations, but given
how long the arguments have been around, clearly not everyone agrees. So
I want to note one internal tension within Evidential Decision Theory
this case brings up.

As Edward Elliott (\citeproc{ref-Elliot2019}{2019}) notes, within
contemporary decision theory there is little overlap between work on
what to do when a predictor is around, and work on the nature of risk.
All parties to the former dispute take for granted the orthodox view
that when there is no predictor, one should maximise expected utility.
But that's very controversial within the debates about risk. There the
big question is whether the heterodox risk-weighed utility theory
developed by Quiggin (\citeproc{ref-Quiggin1982}{1982}) and Buchak
(\citeproc{ref-BuchakRisk}{2013}) is preferable to orthodoxy.

The Quiggin/Buchak view raises a dilemma for Evidential Decision Theory.
If they reject the view and stick with orthodoxy, as most do, they
should have an argument against the risk-weighted view. But the
strongest such arguments turn on the fact that the risk-weighted view
violates the Sure Thing Principle, and leads to people paying to avoid
information. Evidential Decision Theorists can't complain about it on
those grounds, since their theory does the same thing. Alternatively,
they can modify their theory to incorporate the Quiggin/Buchak view. But
then they wouldn't have a decisive decision theory, since on that view
what to do in a decision problem depends on something not typically
specified in the problem, namely the chooser's attitude towards risk. So
even if the Evidential Decision Theorist rejects the arguments behind
\textbf{The Core Premise}, as I suspect most will, they need to either
find a new objection to risk-weighted theories of choice, or modify
their theory in a way that abandons decisiveness. I think the arguments
for \textbf{The Core Premise} are sound, but even if they aren't, it
seems unlikely that there is a plausible \emph{decisive} theory that can
be derived from Evidential Decision Theory.

\section{Stag Hunt}\label{stag-hunt}

It's possible to transform any decision problem involving a predictor
into a game. David Lewis (\citeproc{ref-Lewis1979e}{1979}) already noted
the relationship between Prisoners' Dilemma and Newcomb's Problem. And
William Harper (\citeproc{ref-Harper1986}{1986}) noted that you could
turn any problem involving a predictor into a game by assuming the
predictor wants to make correct predictions and acts in their own
interest.

It's also frequently possible to do the reverse transformation, to turn
a game into a decision problem involving a predictor. Start with any
one-shot two person game, where each player has the same number of
choices in front of them. Then change the payout for Column so that they
get 1 if Row and Column make the `same' choice (for some mapping between
Row's and Column's choices), and 0 otherwise. Then just treat the Column
player as a known to be accurate predictor, either an agent making
predictive choices or a state of the world that tracks the agent's
choices in some way. Now Row's choice is just a familiar kind of
decision problem.

If you plug various famous games into the recipe from the previous
paragraph, you get some familiar examples from modern decision theory.
If you start with Prisoners' Dilemma and apply this recipe, you get
Newcomb's Problem. If you start with Matching Pennies \footnote{You can
  see examples of all these games, and all the game theoretic machinery
  I use throughout this paper, in any standard game theory textbook. My
  favorite such textbook is Bonanno (\citeproc{ref-Bonanno2018}{2018}),
  which has the two advantages of being philosophically sophisticated
  and open access. I'm not going to include citations for every bit of
  textbook game theory I use; that seems about as appropriate as citing
  an undergrad logic textbook every time I use logic. But if you want
  more details on anything unfamiliar in this paper, that's where to
  look.} , you get Death in Damascus
(\citeproc{ref-GibbardHarper1978}{Gibbard and Harper 1978}). If you
start with Battle of the Sexes, you get Asymmetric Death in Damascus
(\citeproc{ref-Richter1984}{Richter 1984}). If you start with Chicken,
you get the Psychopath Button (\citeproc{ref-Egan2007-EGASCT}{Egan
2007}). But there hasn't been quite as much attention paid to what
happens if you start with Stag Hunt and run this recipe. The game you
get turns out to be very useful for classifying decisive decision
theories that choose two boxes in Newcomb's Problem.

Table~\ref{tbl-generic} is an abstract form of a Stag Hunt game, where
the options are G/g for Gather or H/h for Hunt. Actually, it is a table
for a generic symmetric game; what makes it a Stag Hunt are the four
constraints listed below.\footnote{I've listed the constraints as strict
  inequalities, but that might be over the top. Sometimes you'll see one
  or other of these constraints weakened to an inclusive inequality.
  This difference won't matter for current purposes.}

\begin{longtable}[]{@{}lcc@{}}
\caption{A generic symmetric game}\label{tbl-generic}\tabularnewline
\toprule\noalign{}
& g & h \\
\midrule\noalign{}
\endfirsthead
\toprule\noalign{}
& g & h \\
\midrule\noalign{}
\endhead
\bottomrule\noalign{}
\endlastfoot
\textbf{G} & \emph{x}, \emph{x} & \emph{y}, \emph{z} \\
\textbf{H} & \emph{z}, \emph{y} & \emph{w}, \emph{w} \\
\end{longtable}

\begin{itemize}
\tightlist
\item
  \emph{x}~\textgreater~\emph{z}
\item
  \emph{w}~\textgreater~\emph{y}
\item
  \emph{w}~\textgreater~\emph{x}
\item
  \emph{x}~+~\emph{y}~\textgreater~\emph{z}~+~\emph{w}
\end{itemize}

The first two constraints imply that ⟨\emph{G},~\emph{g}⟩ and
⟨\emph{H},~\emph{h}⟩ are both equilibria. This isn't like Prisoners'
Dilemma, that only has one equilibrium. But it is like Prisoners'
Dilemma in that there is a cooperative solution, in this case
⟨\emph{H},~\emph{h}⟩, but it isn't always easy to get to it. It isn't
easy because there are at least two kinds of reasons to play \emph{G}.

First, one might play \emph{G} because one wants to minimise regret.
Each play is a guess that the other player will do the same thing. If
one plays \emph{G} and guesses wrong, one loses \emph{w}~-~\emph{y}
compared to what one could have received. If one plays \emph{H} and
guesses wrong, one loses \emph{x}~-~\emph{z}. And the last constraint
entails that \emph{x}~-~\emph{z}~\textgreater~\emph{w}~-~\emph{y}. So
playing \emph{G} minimises possible regret.

Second, one might want to maximise expected utility, given uncertainty
about what the other player will do. Since one has no reason to think
the other player will prefer \emph{g} to \emph{h} or vice versa - both
are equilibria - maybe one should give each of them equal probability.
And then it will turn out that \emph{G} is the option with highest
expected utility. Intuitively, \emph{H} is a risky option and \emph{G}
is a safe option, and when in doubt, perhaps one should go for the safe
option.

What I'll call a \emph{Stag Decision} is basically a Stag Hunt game
where the other player is a predictor. So the decision looks like
Table~\ref{tbl-stag-decision}, where the above four constraints on the
values still hold, and \textbf{PX} means the predictor predicts
\textbf{X} will be chosen.

\begin{longtable}[]{@{}lcc@{}}
\caption{A Stag Problem (when the in text constraints are
applied)}\label{tbl-stag-decision}\tabularnewline
\toprule\noalign{}
& PG & PH \\
\midrule\noalign{}
\endfirsthead
\toprule\noalign{}
& PG & PH \\
\midrule\noalign{}
\endhead
\bottomrule\noalign{}
\endlastfoot
\textbf{G} & \emph{x} & \emph{y} \\
\textbf{H} & \emph{z} & \emph{w} \\
\end{longtable}

These kinds of decisions are important in the history of game theory
because they illustrate in the one game the two most prominent theories
of equilibrium selection: risk dominance and payoff dominance
(\citeproc{ref-HarsanyiSelten1988}{Harsanyi and Selten 1988}). Risk
dominance recommends gathering; payoff dominance recommends hunting. And
most contemporary proponents of decisive decision theories in philosophy
fall into one of these two camps.

In principle, there are three different views that a decisive theory
could have about Stag Decisions: always Hunt, always Gather, or
sometimes do one and sometimes the other. A decisive theory has to give
a particular recommendation on any given Stag Decision, but it could say
that the four constraints don't settle what that decision should be.
Still, in practice all existing decisive theories fall into one or other
of the first two categories.

One approach, endorsed for rather different reasons by Richard Jeffrey
(\citeproc{ref-Jeffrey1983}{1983}) and Frank Arntzenius
(\citeproc{ref-Arntzenius2008}{2008}), says to hunt because it says in
decisions with multiple equilibria, one should choose the equilibria
with the best payout. This approach will end up agreeing with everything
the Evidential Decision Theorist says about the choices facing Amsterdam
and Brussels, and should be rejected for the same reason. It treats
differently choices that are fundamentally the same, it violates Sure
Thing, and it says Cardiff should pay \$46 to avoid finding out what
Doctor selected. And the same will be true for any decisive theory that
says to always Hunt in Stag Decisions.

Another family of approaches says to always Gather in Stag Decisions.
For very different reasons, this kind of view is endorsed by Ralph
Wedgwood (\citeproc{ref-Wedgwood2013}{2013}), Dmitri Gallow
(\citeproc{ref-Gallow2020}{2020}) and Abelard Podgorski
(\citeproc{ref-Podgorski2022}{2022}). These three views differ from each
other in how they motivate Gathering, and in how they extend the view to
other choices, but they all agree that one should Gather in any Stag
Decision. And this leads to the reverse problem to that facing the
always Hunt view.

Both Amsterdam and Brussels are facing Stag Decisions. But for
Amsterdam, choosing A is Hunting and choosing B is Gathering, while for
Brussels, choosing A is Gathering and choosing B is Hunting. So any view
which says to always Gather will say that Amsterdam should choose B, and
Brussels should choose A. Again, this treats differently choices that
are fundamentally the same, and violates Sure Thing. But does it mean
Cardiff will pay to avoid information? Here things are a little
trickier, because Cardiff has four possible choices: Receive information
or pay to decline it, and then choose A or B. And the different
approaches to Gathering say different things about how to make decisions
in four-way choices. So let's set that argument for \textbf{The Core
Premise} aside - the first two arguments for it still seem like decisive
objections to any view that one should always Gather.

What about views that deny that all Stag Decisions should be treated
alike? As I've said, I don't think any such view is in the literature,
but it's good to think about other views. Let's drop the assumption that
we're even looking at a Stag Decision (though it will turn out that we
are), and think about what to do in general in cases where there are two
strict equilibria. That is, think about what our imaginary decisive
decision theory will say about Table~\ref{tbl-two-eqm}, where we just
have the constraints \emph{x}~\textgreater~\emph{z} and
\emph{w}~\textgreater~\emph{y}, and again \emph{PX} means the predictor
predicts \emph{X}.

\begin{longtable}[]{@{}lcc@{}}
\caption{A decision problem with multiple
equilibria}\label{tbl-two-eqm}\tabularnewline
\toprule\noalign{}
& PE & PF \\
\midrule\noalign{}
\endfirsthead
\toprule\noalign{}
& PE & PF \\
\midrule\noalign{}
\endhead
\bottomrule\noalign{}
\endlastfoot
\textbf{E} & \emph{x} & \emph{y} \\
\textbf{F} & \emph{z} & \emph{w} \\
\end{longtable}

Any coherent solution must be invariant under redescriptions of the
problem. So if you take a real world example that fits this category,
and relabel which option is E and which is F, the recommendation should
flip. And if you rescale the utilities by multiplying by a positive
constant or adding a constant, the verdict should be unchanged, since
utilities are only defined up to positive affine transformation. The
only theories that meet these constraints say that a choice has a
`score' \emph{x}~+~\emph{my}, where \emph{x} is the equilibrium payoff,
and \emph{y} is the other possible payoff, and \emph{m} is a free
variable the theory sets which reflects how much it cares about the
value of the non-equilibrium payoff. The theory then says to pick the
option with the higher score, or to be indifferent otherwise. So it says
to strictly prefer E to F iff \emph{x}~+~\emph{my}~\textgreater{}
\emph{w}~+~\emph{mz} and to be indifferent between the choices if that's
an equality not an inequality. Setting \emph{m} to 0 gives you the view
that says one should always Hunt, since one should always pick the
equilibrium with the highest equilibrium value. Setting \emph{m} to 1
gives you the view that you should always Gather, since you should
maximise the sum (or, equivalently, the average) of the two payouts you
might get with the choice. And both of these views violate \textbf{The
Core Premise}. But what should we say about views that give \emph{m}
other values?

The first thing to say is that it is very hard to see any good
philosophical motivation for values of \emph{m} other than 0 or 1. Both
these values make a certain amount of sense, but the reasons behind any
other value are harder to understand. Still, if coherence required some
other value for \emph{m}, I'm sure someone would come up with a
motivation.

The second thing to say is that we have done more already than object
just to the theories that set \emph{m} to 1 or 0. Any theory that has
\emph{m}~\textless~⅔ will say that Amsterdam should choose A and
Brussels should choose B, violating \textbf{The Core Premise}. And any
view that has \emph{m}~\textgreater~46/47 will say that Amsterdam should
choose B and Brussels should choose A, also violating \textbf{The Core
Premise}. But we don't yet have an objection to theories on which
⅔~⩽~\emph{m}~⩽~46/47.

To see what's wrong with those theories, keep the structure of the game
the same, but change the rewards as in Table~\ref{tbl-revised-game} (all
rewards are in dollars).

\begin{longtable}[]{@{}llll@{}}
\caption{State description of the game with an opt-out
included}\label{tbl-revised-game}\tabularnewline
\toprule\noalign{}
Human Pick & Doctor Pick & Human Reward & Doctor Reward \\
\midrule\noalign{}
\endfirsthead
\toprule\noalign{}
Human Pick & Doctor Pick & Human Reward & Doctor Reward \\
\midrule\noalign{}
\endhead
\bottomrule\noalign{}
\endlastfoot
None & Opt-out & 0 & 1 \\
A & A & 4 & 4 \\
A & B & 0 & 0 \\
B & A & 2+(1/\emph{m}) & 0 \\
B & B & 2 & 1 \\
\end{longtable}

Then the `late game' that Amsterdam faces will look like
Table~\ref{tbl-revised-late}:

\begin{longtable}[]{@{}lcc@{}}
\caption{Table of the revised version of the late
game}\label{tbl-revised-late}\tabularnewline
\toprule\noalign{}
& PA & PB \\
\midrule\noalign{}
\endfirsthead
\toprule\noalign{}
& PA & PB \\
\midrule\noalign{}
\endhead
\bottomrule\noalign{}
\endlastfoot
\textbf{A} & 4 & 0 \\
\textbf{B} & 2+(1/\emph{m}) & 2 \\
\end{longtable}

Since 2~+~m(2+(1/\emph{m}))~\textgreater~4~+~0m for any value of
\emph{m} satisfying ⅔~⩽~\emph{m}~⩽~46/47, the theory will say Amsterdam
should choose B.\footnote{More slowly, we can use the formula to work
  out the score of each option. The score of A is the value in the
  top-left, 4, plus \emph{m} times the value in the top-right, 0. And
  that's 4, no matter the value of \emph{m}. The score of B is the value
  in the bottom-left, 2, plus \emph{m} times the value in the
  bottom-left, 2+(1/\emph{m}). That is, the score is
  2~+~(1~+~2\emph{m})~=~3~+~2\emph{m}. Since \emph{m}~\textgreater~½,
  this value is greater than 4, which was the score of A.}

The `early game' that Brussels faces will look like
Table~\ref{tbl-revised-early}.

\begin{longtable}[]{@{}lcc@{}}
\caption{Table of the revised version of the early
game}\label{tbl-revised-early}\tabularnewline
\toprule\noalign{}
& PA & PB \\
\midrule\noalign{}
\endfirsthead
\toprule\noalign{}
& PA & PB \\
\midrule\noalign{}
\endhead
\bottomrule\noalign{}
\endlastfoot
\textbf{A} & 4 & 0 \\
\textbf{B} & 2+(1/\emph{m}) & 1 \\
\end{longtable}

Since in this game Player gets nothing if Doctor opts-out, and there is
a 50/50 chance the Doctor will opt-out if they predict B, the returns in
the right-hand column are half what they are in the late game. Since
4~+~0\emph{m}~\textgreater~1~+~\emph{m}(2+(1/\emph{m})) for any value of
\emph{m} satisfying ⅔~⩽~\emph{m}~⩽~46/47, the theory will say Brussels
should choose A.\footnote{More slowly, we can use the formula to work
  out the score of each option. The score of A is the value in the
  top-left, 4, plus \emph{m} times the value in the top-right, 0. And
  that's 4, no matter the value of \emph{m}. The score of B is the value
  in the bottom-left, 1, plus \emph{m} times the value in the
  bottom-left, 2~+~(1/\emph{m}). That is, the score is
  1~+~(1~+~2\emph{m})~=~2~+~2\emph{m}. Since \emph{m}~\textless~1, this
  value is less than 4, which was the score of A.}

So any decisive theory will violate \textbf{The Core Premise} for some
choice pair or other. Hence all decisive theories are mistaken.

\section{Where To Next?}\label{where-to-next}

Decision theory cannot be everything that some of its proponents want it
to be. It cannot be a guide that tells us what to do in every situation,
even if we allow it to sometimes say that options are tied. So what can
decision theory be? A natural answer is that it can tell us which
options are rationally permissible, knowing that there will often be a
plurality of options that are permissible. I think the way to finding a
plausible indecisive theory goes via answering the following five
questions.

First, does decision theory start with what the chooser believes, or
with what they should believe? If Player is certain that the red box has
more money, but they have conclusive evidence that the blue box has more
money, which box does decision theory say that they should choose? If
decision theory is the theory of which actions ``most effectively serve
one's desires according to one's beliefs''
(\citeproc{ref-Lewis-Price-17051988}{Lewis 2020c, 465}), then it is the
red box. If it is the theory of which choices are rational, then it is
the blue box. I'm sympathetic to the arguments that Nomy Arpaly
(\citeproc{ref-Arpaly2002}{2002}) makes that the theory of rational
choice should not pay any special attention to the agent's beliefs.
What's rational to choose in a situation is a function of what's
rational to believe in that situation, not what one actually
believes.\footnote{See also Lewis
  (\citeproc{ref-Lewis-Mellor-14101981}{2020a}) where Lewis sketches a
  view that would say that each choice is rational in a way, and there
  need not be anything more to say about which is rational
  all-things-considered. I take it he means decision theory is the
  theory of the part of rationality that the red box chooser does well
  on. A similar point is suggested in Lewis
  (\citeproc{ref-Lewis-Talbott-22061984}{2020d}).}

Second, in a given situation, how many different beliefs are rational?
The Uniqueness thesis says the answer is one. Permissivism says that
Uniqueness is false, and for some propositions in some situations, there
are multiple rational attitudes to have. See Kopec and Titelbaum
(\citeproc{ref-KopecTitelbaum2016}{2016}) for a good survey of the
issues, Schultheis (\citeproc{ref-Schultheis2018}{2018}) for a recent
argument for Uniqueness, and Callahan
(\citeproc{ref-Callahan2021}{2021}) for a recent argument for
Permissivism.\footnote{Interestingly, Callahan connects Permissivism to
  existentialism. I suspect there are deep and unexplored connections
  between existentialism and decision theory, especially concerning the
  questions about the priority of strategies or individual choices. But
  that's for another paper.} I'm on the Permissivist side of this
debate.

Now if you think decision theory should be sensitive to rational beliefs
rather than actual beliefs, and you think Permissivism is true, you're
committed to indecisiveness. You won't even need demons. After all, any
situation where any credence in \emph{p} between \emph{x} and \emph{y}
is permissible will mean there are multiple bets at distinct odds on
\emph{p} that rationality neither requires taking nor requires passing.
I think this is a perfectly sound argument for indecisiveness, but I
didn't lean on it here because the premises are considerably less secure
than the ones I've appealed to.

But there is a third question that needs answering before we can offer a
plausible indecisive theory: what is a mixed strategy? Relatedly, what
role do mixed strategies have in the correct decision theory? This is a
rather vexed question, and an important one. Almost all recent arguments
against causal decision theory seem, to my eyes at least, to turn on
attributing a bad theory of mixed strategies to the causal decision
theorist. You can see this from the fact that almost all recent papers
on decision theory involve problems that, when converted into games,
have no pure strategy equilibria, but do have mixed strategy equilibria.
We can't offer a full decision theory, even an indecisive one, without
resolving these problems, and that means having a theory of mixed
strategies. And that's very much a theory for another paper.\footnote{For
  what it's worth, I think that theory must include the following two
  factors. First, playing a mixed strategy is just what Lewis
  (\citeproc{ref-Lewis-Kavka-10071979}{2020b}) calls using a
  tie-breaking procedure. Second, the output of such a tie-breaking
  procedure is in principle unpredictable by anything that doesn't time
  travel.}

Note one thing I haven't said so far, and won't say in what follows. I
don't say that the way to find the correct indecisive theory is to come
up with a bunch of cases, consult our intuitions about them, and then
see which theory can match at least 80\% of those intuitions. (Or
whatever percentage we are working with this week.) That is a dubious
approach in general, but around here it is close to incoherent.

Most contemporary work in decision theory starts with the assumption
that when there are no demons around (or anything else vaguely demonic),
expected utility maximisation is the correct decision theory. And then
theorists will start rolling out fantastic cases involving demons or
predictors or lesions or genes or twins or triplets or whatever is in
fashion. And they will ask what extension of expected utility theory
best tracks intuitions about these cases. But this seems like a very
dubious strategy, since intuitions about cases will not lead one to
expected utility theory in the first place. Trying to match intuitions
about cases like the Allais or Ellsberg paradoxes will lead one to
prefer some non-standard theory like the one developed by John Quiggin
(\citeproc{ref-Quiggin1982}{1982}) or Lara Buchak
(\citeproc{ref-BuchakRisk}{2013}). It seems very unlikely that the best
way to extend a counterintuitive theory like expected utility
maximisation is by consulting intuitions about puzzle cases. It is much
better to ask what principles we want our theory to endorse, and work
towards a theory that satisfies those principles. And that is the
methodology I have adopted here.

I've relied heavily in this paper on two such principles: The Sure Thing
principle, and the principle that information has non-negative value.
I'll end by describing one more principle, and noting two questions it
raises. The principle is that a decider should be a probabilist, and
that they should maximise expected utility. More precisely, it says that
if the states are
\{\emph{S}\textsubscript{1},~\ldots,~\emph{S\textsubscript{m}}\}, and
the choices are
\{\emph{O}\textsubscript{1},~\ldots,~\emph{O\textsubscript{n}}\}, then
\emph{O\textsubscript{i}} is a permissible choice just in case there is
some probability function \emph{Pr} such that

\[
\sum_{k = 1}^m V(S_k \wedge O_i)Pr(S_k) \geq \sum_{k = 1}^m V(S_k \wedge O_j)Pr(S_k)
\]

for all \emph{j}~∈~\{1,~\ldots,~\emph{n}\}. Even if the subjective
probability of the state is affected by the choice one makes, there
should be some probability function that the chooser ends up with, and
their choice should make sense by the lights of that probability
function. Note that if we assume that the chooser can select any mixed
strategy from among their choices, there is guaranteed to be at least
one strategy that satisfies this requirement, even if one thinks the
states are choices of a demon who can predict one's strategy.\footnote{If
  the demon can predict what one will do on a given occasion while
  playing a mixed strategy, this guarantee may fail. But assuming what I
  said in the last footnote about mixed strategies, that would mean
  we're in the realm of backwards causation, and the states are not
  causally independent of the actions.}

So that seems to me like a minimal constraint on choices. As Pearce
(\citeproc{ref-Pearce1984}{1984}) shows, it is equivalent to the
requirement that one not make a choice that is strictly dominated by
some other choice, or by some mixture of other choices. (This result is
hardly obvious, but it turns out to be a reasonably straightforward
consequence of the existence of Nash equilibria for all finite zero-sum
games.) That's hardly an uncontroversial principle, but it is also one
I'm happy to adopt. If you're still on board, there are two more
questions that we need to answer before we finish our decision theory.

Are all further constraints on rational decisions representable as
constraints on the \emph{Pr} in this principle? There surely are some
further constraints on rational decisions. If you're offered a bet at
even money on whether I will become Canadian President next week, the
only rational thing to do is to decline it. And that's true even though
there is a \emph{Pr} such that taking the bet maximises expected
utility. But that \emph{Pr} is completely irrational given your
evidence. So \emph{Do something that maximises expected utility given
some probability} is too liberal a rule; we need to say something about
the \emph{Pr}. Do we need to say more than that? My answer is no, though
I'm not even going to start defending that here.\footnote{Note that if
  you say no to this question, and you think that probabilities have to
  be real-valued, then you're committed to weak dominance not having a
  role to play in decision theory. So this is a non-trivial question.}

The Canadian Presidency examples suggests that there are constraints on
\emph{Pr} that are external to decision theory. You shouldn't take that
bet because you shouldn't have probability above 0.5 that I'll become
Canadian President next week. The order of explanation runs from the
(ir)rationality of the credal state to the (ir)rationality of the
decision. Our fifth and final question is, are there any cases where the
order of explanation goes the other way? Arntzenius
(\citeproc{ref-Arntzenius2008}{2008}) argued that one should have
credences such that the highest value equilibrium was also the choice
that maximised expected utility. That's an example of a constraint on
\emph{Pr} where the order of explanation runs from decisions to beliefs.
I argued against that principle, but not because of a systematic reason
to think that the order of explanation can't run that way. Instead I
argued that this particular principle was dynamically incoherent. That
leaves open the general question of whether any such principles, where
constraints on decisions explain constraints on belief, are right.

The long term goal of the project behind this paper is to argue that
there are no such principles. The only constraints on rational decision
are that one should maximise expected utility given some \emph{Pr}, and
this \emph{Pr} should satisfy independently motivated epistemic
requirements. Now I haven't come close to arguing for that here, and
it's a very strong claim. Given everything else I've said, it basically
amounts to the claim that the theory of equilibrium selection has no
role to play in normative decision theory. It may have a central role to
play in descriptive decision theory, in explaining why people end up at
a certain equilibrium. But it can't justify that equilibrium, since any
equilibrium could be rationally justified.\footnote{But note here that
  what I'm calling an equilibrium is just a coherent set of beliefs that
  is grounded in the evidence. It doesn't include the requirement,
  typical in game-theory, that the chooser has true beliefs about some
  aspect of the world around them.}

But all of this is for future work. The aim of this paper has been to
open up the possibility of an indecisive, i.e., permissive, decision
theory. Decisive decision theories have to take a stand on Stag
Decisions, and there is no coherent way for them to do that. So no
decisive theory is correct, and the correct decision theory is
indecisive.

\subsection*{References}\label{references}
\addcontentsline{toc}{subsection}{References}

\phantomsection\label{refs}
\begin{CSLReferences}{1}{0}
\bibitem[\citeproctext]{ref-Arntzenius2008}
Arntzenius, Frank. 2008. {``No Regrets; or, Edith Piaf Revamps Decision
Theory.''} \emph{Erkenntnis} 68 (2): 277--97. doi:
\href{https://doi.org/10.1007/s10670-007-9084-8}{10.1007/s10670-007-9084-8}.

\bibitem[\citeproctext]{ref-Arpaly2002}
Arpaly, Nomy. 2002. {``Moral Worth.''} \emph{Journal of Philosophy} 99
(5): 223--45. doi:
\href{https://doi.org/10.2307/3655647}{10.2307/3655647}.

\bibitem[\citeproctext]{ref-Bonanno2018}
Bonanno, Giacomo. 2018. {``Game Theory.''} Davis, CA: Kindle Direct
Publishing. 2018.
\url{http://faculty.econ.ucdavis.edu/faculty/bonanno/GT_Book.html}.

\bibitem[\citeproctext]{ref-BuchakRisk}
Buchak, Lara. 2013. \emph{Risk and Rationality}. Oxford: Oxford
University Press.

\bibitem[\citeproctext]{ref-Callahan2021}
Callahan, Laura Frances. 2021. {``Epistemic Existentialism.''}
\emph{Episteme} 18 (4): 539--54. doi:
\href{https://doi.org/10.1017/epi.2019.25}{10.1017/epi.2019.25}.

\bibitem[\citeproctext]{ref-Chang2002}
Chang, Ruth. 2002. {``The Possibility of Parity.''} \emph{Ethics} 112
(4): 659--88. doi:
\href{https://doi.org/10.1086/339673}{10.1086/339673}.

\bibitem[\citeproctext]{ref-Egan2007-EGASCT}
Egan, Andy. 2007. {``{Some Counterexamples to Causal Decision
Theory}.''} \emph{Philosophical Review} 116 (1): 93--114. doi:
\href{https://doi.org/10.1215/00318108-2006-023}{10.1215/00318108-2006-023}.

\bibitem[\citeproctext]{ref-Elliot2019}
Elliott, Edward. 2019. {``Normative Decision Theory.''} \emph{Analysis}
79 (4): 755--72. doi:
\href{https://doi.org/10.1093/analys/anz059}{10.1093/analys/anz059}.

\bibitem[\citeproctext]{ref-Gallow2020}
Gallow, J. Dmitri. 2020. {``The Causal Decision Theorist's Guide to
Managing the News.''} \emph{The Journal of Philosophy} 117 (3): 117--49.
doi:
\href{https://doi.org/10.5840/jphil202011739}{10.5840/jphil202011739}.

\bibitem[\citeproctext]{ref-GibbardHarper1978}
Gibbard, Allan, and William Harper. 1978. {``Counterfactuals and Two
Kinds of Expected Utility.''} In \emph{Foundations and Applications of
Decision Theory}, edited by C. A. Hooker, J. J. Leach, and E. F.
McClennen, 125--62. Dordrecht: Reidel.

\bibitem[\citeproctext]{ref-Harper1986}
Harper, William. 1986. {``Mixed Strategies and Ratifiability in Causal
Decision Theory.''} \emph{Erkenntnis} 24 (1): 25--36. doi:
\href{https://doi.org/10.1007/BF00183199}{10.1007/BF00183199}.

\bibitem[\citeproctext]{ref-HarsanyiSelten1988}
Harsanyi, John C., and Reinhard Selten. 1988. \emph{A General Theory of
Equilibrium Selection in Games}. Cambridge, MA: {MIT} Press.

\bibitem[\citeproctext]{ref-Jeffrey1983}
Jeffrey, Richard. 1983. {``Bayesianism with a Human Face.''} In
\emph{Testing Scientific Theories}, edited by J. Earman (ed.).
Minneapolis: University of Minnesota Press.

\bibitem[\citeproctext]{ref-KopecTitelbaum2016}
Kopec, Matthew, and Michael G. Titelbaum. 2016. {``The Uniqueness
Thesis.''} \emph{Philosophy Compass} 11 (4): 189--200. doi:
\href{https://doi.org/10.1111/phc3.12318}{10.1111/phc3.12318}.

\bibitem[\citeproctext]{ref-Lewis1979e}
Lewis, David. 1979. {``Prisoners' Dilemma Is a {N}ewcomb Problem.''}
\emph{Philosophy and Public Affairs} 8 (3): 235--40. Reprinted in his
\emph{Philosophical Papers}, Volume 2, Oxford: Oxford University Press,
1986, 299-304. References to reprint.

\bibitem[\citeproctext]{ref-Lewis-Mellor-14101981}
---------. 2020a. {``Letter to {D}. H. Mellor, 14 October 1981.''} In
\emph{Philosophical Letters of David {K}. Lewis}, edited by Helen Beebee
and A. R. J. Fisher, 1:432--34. Oxford: Oxford University Press.

\bibitem[\citeproctext]{ref-Lewis-Kavka-10071979}
---------. 2020b. {``Letter to Gregory Kavka, 10 July 1979.''} In
\emph{Philosophical Letters of David {K}. Lewis}, edited by Helen Beebee
and A. R. J. Fisher, 1:423--24. Oxford: Oxford University Press.

\bibitem[\citeproctext]{ref-Lewis-Price-17051988}
---------. 2020c. {``Letter to Huw Price, 17 May 1988.''} In
\emph{Philosophical Letters of David k. Lewis}, edited by Helen Beebee
and A. R. J. Fisher, 2:464--66. Oxford: Oxford University Press.

\bibitem[\citeproctext]{ref-Lewis-Talbott-22061984}
---------. 2020d. {``Letter to William j. Talbott, 22 June 1984.''} In
\emph{Philosophical Letters of David k. Lewis}, edited by Helen Beebee
and A. R. J. Fisher, 1:448--49. Oxford: Oxford University Press.

\bibitem[\citeproctext]{ref-Pearce1983}
Pearce, David G. 1983. {``A Problem with Single Valued Solution
Concepts.''} 1983.
\url{https://sites.google.com/a/nyu.edu/davidpearce/}.

\bibitem[\citeproctext]{ref-Pearce1984}
---------. 1984. {``Rationalizable Strategic Behavior and the Problem of
Perfection.''} \emph{Econometrica} 52 (4): 1029--50. doi:
\href{https://doi.org/10.2307/1911197}{10.2307/1911197}.

\bibitem[\citeproctext]{ref-Podgorski2022}
Podgorski, Aberlard. 2022. {``Tournament Decision Theory.''}
\emph{No{û}s} 56 (1): 176--203. doi:
\href{https://doi.org/10.1111/nous.12353}{10.1111/nous.12353}.

\bibitem[\citeproctext]{ref-Quiggin1982}
Quiggin, John. 1982. {``A Theory of Anticipated Utility.''}
\emph{Journal of Economic Behavior \& Organization} 3 (4): 323--43. doi:
\href{https://doi.org/10.1016/0167-2681(82)90008-7}{10.1016/0167-2681(82)90008-7}.

\bibitem[\citeproctext]{ref-Richter1984}
Richter, Reed. 1984. {``Rationality Revisited.''} \emph{Australasian
Journal of Philosophy} 62 (4): 393--404. doi:
\href{https://doi.org/10.1080/00048408412341601}{10.1080/00048408412341601}.

\bibitem[\citeproctext]{ref-Schultheis2018}
Schultheis, Ginger. 2018. {``Living on the Edge: Against Epistemic
Permissivism.''} \emph{Mind} 127 (507): 863--79. doi:
\href{https://doi.org/10.1093/mind/fzw065}{10.1093/mind/fzw065}.

\bibitem[\citeproctext]{ref-Skyrms1990}
Skyrms, Brian. 1990. \emph{The Dynamics of Rational Deliberation}.
Cambridge, MA: Harvard University Press.

\bibitem[\citeproctext]{ref-Skyrms2001}
---------. 2001. {``The Stag Hunt.''} \emph{Proceedings and Addresses of
the American Philosophical Association} 75 (2): 31--41. doi:
\href{https://doi.org/10.2307/3218711}{10.2307/3218711}.

\bibitem[\citeproctext]{ref-Skyrms2004}
---------. 2004. \emph{The Stag Hunt and the Evolution of Social
Structure}. Cambridge: {C}ambridge {U}niversity {P}ress.

\bibitem[\citeproctext]{ref-Wedgwood2013}
Wedgwood, Ralph. 2013. {``A Priori Bootstrapping.''} In \emph{The a
Priori in Philosophy}, edited by Albert Casullo and Joshua C. Thurow,
225--46. Oxford: Oxford University Press.

\end{CSLReferences}



Unpublished. First posted in 2021.


\end{document}
