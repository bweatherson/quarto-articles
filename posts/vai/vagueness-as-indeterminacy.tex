% Options for packages loaded elsewhere
% Options for packages loaded elsewhere
\PassOptionsToPackage{unicode}{hyperref}
\PassOptionsToPackage{hyphens}{url}
%
\documentclass[
  11pt,
  letterpaper,
  DIV=11,
  numbers=noendperiod,
  twoside]{scrartcl}
\usepackage{xcolor}
\usepackage[left=1.1in, right=1in, top=0.8in, bottom=0.8in,
paperheight=9.5in, paperwidth=7in, includemp=TRUE, marginparwidth=0in,
marginparsep=0in]{geometry}
\usepackage{amsmath,amssymb}
\setcounter{secnumdepth}{3}
\usepackage{iftex}
\ifPDFTeX
  \usepackage[T1]{fontenc}
  \usepackage[utf8]{inputenc}
  \usepackage{textcomp} % provide euro and other symbols
\else % if luatex or xetex
  \usepackage{unicode-math} % this also loads fontspec
  \defaultfontfeatures{Scale=MatchLowercase}
  \defaultfontfeatures[\rmfamily]{Ligatures=TeX,Scale=1}
\fi
\usepackage{lmodern}
\ifPDFTeX\else
  % xetex/luatex font selection
  \setmainfont[ItalicFont=EB Garamond Italic,BoldFont=EB Garamond
Bold]{EB Garamond Math}
  \setsansfont[]{EB Garamond}
  \setmathfont[]{Garamond-Math}
\fi
% Use upquote if available, for straight quotes in verbatim environments
\IfFileExists{upquote.sty}{\usepackage{upquote}}{}
\IfFileExists{microtype.sty}{% use microtype if available
  \usepackage[]{microtype}
  \UseMicrotypeSet[protrusion]{basicmath} % disable protrusion for tt fonts
}{}
\usepackage{setspace}
% Make \paragraph and \subparagraph free-standing
\makeatletter
\ifx\paragraph\undefined\else
  \let\oldparagraph\paragraph
  \renewcommand{\paragraph}{
    \@ifstar
      \xxxParagraphStar
      \xxxParagraphNoStar
  }
  \newcommand{\xxxParagraphStar}[1]{\oldparagraph*{#1}\mbox{}}
  \newcommand{\xxxParagraphNoStar}[1]{\oldparagraph{#1}\mbox{}}
\fi
\ifx\subparagraph\undefined\else
  \let\oldsubparagraph\subparagraph
  \renewcommand{\subparagraph}{
    \@ifstar
      \xxxSubParagraphStar
      \xxxSubParagraphNoStar
  }
  \newcommand{\xxxSubParagraphStar}[1]{\oldsubparagraph*{#1}\mbox{}}
  \newcommand{\xxxSubParagraphNoStar}[1]{\oldsubparagraph{#1}\mbox{}}
\fi
\makeatother


\usepackage{longtable,booktabs,array}
\usepackage{calc} % for calculating minipage widths
% Correct order of tables after \paragraph or \subparagraph
\usepackage{etoolbox}
\makeatletter
\patchcmd\longtable{\par}{\if@noskipsec\mbox{}\fi\par}{}{}
\makeatother
% Allow footnotes in longtable head/foot
\IfFileExists{footnotehyper.sty}{\usepackage{footnotehyper}}{\usepackage{footnote}}
\makesavenoteenv{longtable}
\usepackage{graphicx}
\makeatletter
\newsavebox\pandoc@box
\newcommand*\pandocbounded[1]{% scales image to fit in text height/width
  \sbox\pandoc@box{#1}%
  \Gscale@div\@tempa{\textheight}{\dimexpr\ht\pandoc@box+\dp\pandoc@box\relax}%
  \Gscale@div\@tempb{\linewidth}{\wd\pandoc@box}%
  \ifdim\@tempb\p@<\@tempa\p@\let\@tempa\@tempb\fi% select the smaller of both
  \ifdim\@tempa\p@<\p@\scalebox{\@tempa}{\usebox\pandoc@box}%
  \else\usebox{\pandoc@box}%
  \fi%
}
% Set default figure placement to htbp
\def\fps@figure{htbp}
\makeatother


% definitions for citeproc citations
\NewDocumentCommand\citeproctext{}{}
\NewDocumentCommand\citeproc{mm}{%
  \begingroup\def\citeproctext{#2}\cite{#1}\endgroup}
\makeatletter
 % allow citations to break across lines
 \let\@cite@ofmt\@firstofone
 % avoid brackets around text for \cite:
 \def\@biblabel#1{}
 \def\@cite#1#2{{#1\if@tempswa , #2\fi}}
\makeatother
\newlength{\cslhangindent}
\setlength{\cslhangindent}{1.5em}
\newlength{\csllabelwidth}
\setlength{\csllabelwidth}{3em}
\newenvironment{CSLReferences}[2] % #1 hanging-indent, #2 entry-spacing
 {\begin{list}{}{%
  \setlength{\itemindent}{0pt}
  \setlength{\leftmargin}{0pt}
  \setlength{\parsep}{0pt}
  % turn on hanging indent if param 1 is 1
  \ifodd #1
   \setlength{\leftmargin}{\cslhangindent}
   \setlength{\itemindent}{-1\cslhangindent}
  \fi
  % set entry spacing
  \setlength{\itemsep}{#2\baselineskip}}}
 {\end{list}}
\usepackage{calc}
\newcommand{\CSLBlock}[1]{\hfill\break\parbox[t]{\linewidth}{\strut\ignorespaces#1\strut}}
\newcommand{\CSLLeftMargin}[1]{\parbox[t]{\csllabelwidth}{\strut#1\strut}}
\newcommand{\CSLRightInline}[1]{\parbox[t]{\linewidth - \csllabelwidth}{\strut#1\strut}}
\newcommand{\CSLIndent}[1]{\hspace{\cslhangindent}#1}



\setlength{\emergencystretch}{3em} % prevent overfull lines

\providecommand{\tightlist}{%
  \setlength{\itemsep}{0pt}\setlength{\parskip}{0pt}}



 


\setlength\heavyrulewidth{0ex}
\setlength\lightrulewidth{0ex}
\usepackage[automark]{scrlayer-scrpage}
\clearpairofpagestyles
\cehead{
  Brian Weatherson
  }
\cohead{
  Vagueness as Indeterminacy
  }
\ohead{\bfseries \pagemark}
\cfoot{}
\makeatletter
\newcommand*\NoIndentAfterEnv[1]{%
  \AfterEndEnvironment{#1}{\par\@afterindentfalse\@afterheading}}
\makeatother
\NoIndentAfterEnv{itemize}
\NoIndentAfterEnv{enumerate}
\NoIndentAfterEnv{description}
\NoIndentAfterEnv{quote}
\NoIndentAfterEnv{equation}
\NoIndentAfterEnv{longtable}
\NoIndentAfterEnv{abstract}
\renewenvironment{abstract}
 {\vspace{-1.25cm}
 \quotation\small\noindent\emph{Abstract}:}
 {\endquotation}
\newfontfamily\tfont{EB Garamond}
\addtokomafont{disposition}{\rmfamily}
\addtokomafont{title}{\normalfont\itshape}
\let\footnoterule\relax
\KOMAoption{captions}{tableheading}
\makeatletter
\@ifpackageloaded{caption}{}{\usepackage{caption}}
\AtBeginDocument{%
\ifdefined\contentsname
  \renewcommand*\contentsname{Table of contents}
\else
  \newcommand\contentsname{Table of contents}
\fi
\ifdefined\listfigurename
  \renewcommand*\listfigurename{List of Figures}
\else
  \newcommand\listfigurename{List of Figures}
\fi
\ifdefined\listtablename
  \renewcommand*\listtablename{List of Tables}
\else
  \newcommand\listtablename{List of Tables}
\fi
\ifdefined\figurename
  \renewcommand*\figurename{Figure}
\else
  \newcommand\figurename{Figure}
\fi
\ifdefined\tablename
  \renewcommand*\tablename{Table}
\else
  \newcommand\tablename{Table}
\fi
}
\@ifpackageloaded{float}{}{\usepackage{float}}
\floatstyle{ruled}
\@ifundefined{c@chapter}{\newfloat{codelisting}{h}{lop}}{\newfloat{codelisting}{h}{lop}[chapter]}
\floatname{codelisting}{Listing}
\newcommand*\listoflistings{\listof{codelisting}{List of Listings}}
\makeatother
\makeatletter
\makeatother
\makeatletter
\@ifpackageloaded{caption}{}{\usepackage{caption}}
\@ifpackageloaded{subcaption}{}{\usepackage{subcaption}}
\makeatother
\usepackage{bookmark}
\IfFileExists{xurl.sty}{\usepackage{xurl}}{} % add URL line breaks if available
\urlstyle{same}
\hypersetup{
  pdftitle={Vagueness as Indeterminacy},
  pdfauthor={Brian Weatherson},
  hidelinks,
  pdfcreator={LaTeX via pandoc}}


\title{Vagueness as Indeterminacy}
\author{Brian Weatherson}
\date{2018}
\begin{document}
\maketitle
\begin{abstract}
Traditionally, we thought vague predicates were predicates with
borderline cases. In recent years traditional wisdom has come under
attack from several leading theorists. They are motivated by a common
idea, that terms with borderline cases, but sharp boundaries around the
borderline cases, are not vague. I argue for a return to tradition. Part
of the argument is that the alternatives that have been proposed are
themselves subject to intuitive counterexample. And part of the argument
is that we need a theory of what vagueness is that applies to
non-predicates. The traditional picture can be smoothly generalised to
non-predicates if we identify vagueness generally with indeterminacy.
Modern rivals to tradition do not admit of such smooth generalisation.
\end{abstract}


\setstretch{1.1}
Recently there has been a flurry of proposals on how to `define'
vagueness. These proposals are not meant to amount to \emph{theories} of
vagueness as, for instance, epistemic or supervaluational theories of
vagueness are. That is, they are not meant to provide solutions to the
raft of puzzles and paradoxes traditionally associated with vagueness.
Rather, they are meant to give us a sense of which terms in the language
are vague, and to use Matti Eklund's phrase, in what their vagueness
consists. Doing this \emph{might} be a prelude to a successful theory of
vagueness, or it might just be an interesting classificatory question in
its own right.

When this activity started, most notably with Patrick Greenough's ``A
Minimal Theory of Vagueness'', I suspected that it would be a hopeless
project. Imagine, I thought, trying to give a definition of what
causation is that didn't amount to a theory of causation. That project
seems hopeless, and I didn't think the prospects for a definition of
vagueness were much better. I now think I was wrong, and we can learn a
lot from thinking about which terms are vague, independent of our theory
of vagueness. (As we'll get to below, Greenough's theory isn't marketed
as a definition of vagueness, but rather a `minimal theory' to which all
parties can agree. But it has been taken, e.g.~by Eklund and Nicholas
Smith, to be providing a rival to genuine definitions of vagueness, and
I'll follow Eklund and Smith in this respect.)

The point of this exercise is not to give an analysis of how the man on
the Clapham omnibus uses `vague' and its cognates. As is widely
recognised, `vague' is often used in ordinary language as a predicate
that applies to claims like \emph{The Grand Canyon is between 2 and 2
trillion years old}, i.e.~claims that are consistent with a wide range
of possible worlds. That's not the sense of `vague' which philosophers
use, nor the sense we are trying to define. But nor should we think we
are just trying to analyse philosophical use of `vague'. The
philosophers' usage may be our starting point, but if we find
philosophers have traditionally being ignoring theoretically important
commonalities, or blurring theoretically important distinction, our best
definition may well amount to a revision of philosophical usage.

The game, I think, is one of setting goals for what a theory of
vagueness should do. It is a legitimate objection to a theory of
vagueness that it isn't comprehensive, that it doesn't cover the field.
If supervaluationism was only a theory of how vague words that started
with consonants behaved, for example, that would be a problem for
supervaluationism. But to press objections of this form we must have an
antecedent answer to the question of which words are in the field, and
hence should be covered. That's the good question which these
definitions of vagueness address. Because I take this to be the
important issue, I'm going to start this paper with a bunch of examples
of apparently vague, and apparently non-vague, terms. We'll then look at
which theories do the best job at systematising intuitions about these
cases. I'll then argue that the best way to systematise our intuitions
about these cases while respecting theoretically important commonalities
and distinctions is to take vagueness to be indeterminacy, while staying
silent for now on whether the indeterminacy is semantic or epistemic. In
doing so I'm returning to a traditional view of vagueness, one that is
discussed in such classic works as Kit Fine's statement of
supervaluationism (\citeproc{ref-Fine1975a}{Fine 1975}). So I make no
claim to originality in my conclusions here, though I hope at least
\emph{some} of the arguments are original.

\section{Examples}\label{examples}

I'm going to introduce five classes of examples, which will serve as our
data in what follows. I'll give a fairly tendentious description of each
class to orient us before starting. Our five classes are (a) words that
are indeterminate but not vague, (b) vague words that are not
predicates, especially predicate modifiers, (c) vague predicates whose
conditions of application are contentious, (d) vague predicates whose
application depends on discrete states of the world, and (e) vague
predicates that do not determine boundaries.

\subsection{Indeterminacy without
Vagueness}\label{indeterminacy-without-vagueness}

Many philosophers, if asked, would say that vague words are those that
have borderline cases. As noted above, Fine
(\citeproc{ref-Fine1975a}{1975}) takes exactly this view. My preferred
view, that vagueness is indeterminacy, is a simple generalisation of
this view to non-predicates. But it is a commonplace of the literature
on definitions of vagueness that this won't do because of examples of
indeterminacy without vagueness. Two examples are commonly used. One of
these is Sainsbury's example \emph{child}
(\citeproc{ref-Sainsbury1991}{Sainsbury 1991}). By definition, the
extension of \emph{child} is the set of persons under sixteen years old,
and its anti-extension is the set of persons eighteen years old or
older. Sixteen and seventeen year olds are borderline cases. The
intuition is that even though \emph{child} has borderline cases it is
not vague, because there are sharp boundaries to its borderline.

A similar case arrives with \emph{mass} as it is used by a Newtonian
physicist. (I'm grateful to Delia Graff Fara for pointing out the
connection here.) As Field (\citeproc{ref-Field1973}{1973}) showed,
\emph{mass} is indeterminate between two meanings, rest mass and proper
mass. But it is intuitively \emph{not} vague, because it is determinate
that it means either rest mass or proper mass. These cases are well
discussed in the existing literature, and I won't say much more about
them here, save to note that one of the examples that is usually taken
to be very problematic for the vagueness as indeterminacy view,
\emph{child}, is not synonymous with any term in any natural language.
This is not a reason that it could not serve as a counterexample,
because a definition should cover all terms actual and possible.

\subsection{Non-Predicate Vagueness}\label{non-predicate-vagueness}

Not only predicates are vague. There is an extensive literature on vague
singular terms. Arguably many determiners are vague. And, as I'll stress
here, many predicate modifiers are vague.

We can make an intuitive distinction between vague and precise predicate
modifiers. Compare the following two (obviously artificial) predicate
modifiers. (I owe these examples to David Chalmers.) Where \emph{F} is a
predicate such that \emph{Fa} is true iff for some variable \emph{v},
\emph{v}(\emph{a})~\textgreater~\emph{x} and \emph{v} has a natural zero
value (e.g.~like height and unlike utility) then we can define
\emph{doubly F} and \emph{bigly F}. It is true that \emph{a} is doubly
\emph{F} iff \emph{v}(\emph{a})~\textgreater~2\emph{x} and \emph{a} is
bigly \emph{F} iff \emph{v}(\emph{a})/\emph{x} is big. Now there's a
good sense in which \emph{doubly} is a precise modifier, for the
modification it makes to its attached predicate can be precisely
defined, while \emph{bigly} is a vague modifier. That's the sense in
which I mean some modifiers are vague and others are precise. Note that
even though \emph{doubly} is precise it can be a constituent of a vague
predicate, such as \emph{doubly tall}. That makes sense; just as a vague
sentence need only contain one vague word, so need a vague complex
predicate need only contain one vague word.

Now we might well ask whether natural language modifiers like
\emph{very} are vague or precise. I'm sad to say that I really don't
have an answer to that question, but I think it's an excellent question.
To get a sense of how hard it is, note one awkward feature of
\emph{very} -- it is most comfortable attaching to words that are
themselves vague. For instance (1a) is a sentence of English while (1b)
is not.

\begin{description}
\tightlist
\item[(1)]
\hfill
\begin{enumerate}
\def\labelenumi{(\alph{enumi})}
\tightlist
\item
  Jack is very old.
\end{enumerate}

\begin{enumerate}
\def\labelenumi{(\alph{enumi})}
\setcounter{enumi}{1}
\tightlist
\item
  Jack is very forty-seven years old.
\end{enumerate}
\end{description}

I don't know whether this is a universal feature of \emph{very}. My best
guess is that it is though in conversation some people have proposed
interesting putative counterexamples. (I'm grateful here to Daniel
Nolan.) But to avoid that complication, I'll introduce a new word
\emph{very}. This modifier is defined such that if \emph{F} is vague
then \emph{very}* \emph{F} means the same thing as \emph{very F}, and if
\emph{F} is not vague then \emph{very* F} is meaningless, like
\emph{very forty-seven years old}. It's an excellent question whether
\emph{very} is vague, and I think it's a requirement on a definition of
vagueness that it allow this question to be asked. As we'll see, this is
sadly not true of most proposed definitions of vagueness on the market.

\subsection{Philosophically Interesting Vague
Terms}\label{philosophically-interesting-vague-terms}

It's morally obligatory that someone with my standard of living donate
1\% of their income to charity. It's not morally obligatory that someone
with my standard of living donate 100\% of their income to charity. What
is the largest \emph{x} such that it's morally obligatory that someone
with my standard of living donate \emph{x}\% of their income to charity?
(As a moralistic cheapskate I'd rather like to know.) Arguably this is
vague. But perhaps only arguably. On some divine command theories it is
precise, because there's a fact about what God wants me to do, however
hard this is to figure out. (It's even a knowable fact, since God knows
it.) But on more standard secular moral theories this may indeed be
vague.

There are two lessons to draw from this case. First, if two philosophers
can debate what the correct theory of morality is while one thinks it is
vague and the other thinks is precise, as I think could happen in a
dispute between a divine command theorist and a virtue ethicist, then
knowing that a vague term is vague is not required for understanding the
term. (I assume here the divine command theorist is not so confused that
she's not really talking about \emph{goodness}.)

Second, it is important to remember that for some vague terms competent
users of the term need not know in virtue of what they apply. Much of
the literature on vagueness focuses on words like \emph{tall},
\emph{thin} and \emph{bald} where all competent users know which kinds
of underlying facts are relevant to their application. But not all vague
terms are like that, as \emph{good} illustrates. And this phenomena
extends beyond the normative, at least narrowly conceived. If you
believe Tom Wolfe (\citeproc{ref-Wolfe2000}{2000}) then among the youth
of America \emph{going out with} is vague and many do not know exactly
in virtue of what it applies. It's a familiar point in philosophy of
mind that competent users can disagree about what kinds of features a
thing must have to satisfy \emph{is thinking}. And we can multiply
instances of this by considering any area of philosophy we like.

\subsection{Discrete Vague Terms}\label{discrete-vague-terms}

An academic with one child has few children for an academic. An academic
with five children does not have few children for an academic. (I'll
omit the comparison class `for an academic' from now on.) Where is the
borderline between those with few children and those not with few
children? (I don't ask out of personal interest this time.) This
question, like the question of how much giving is morally obligatory,
feels vague. But note that we cannot generate a compelling Sorites
paradox using \emph{has few children}. Let's see how badly this Sorites
argument fails.

\begin{description}
\tightlist
\item[(2)]
\hfill
\begin{enumerate}
\def\labelenumi{(\alph{enumi})}
\tightlist
\item
  An academic with one child has few children.
\end{enumerate}

\begin{enumerate}
\def\labelenumi{(\alph{enumi})}
\setcounter{enumi}{1}
\tightlist
\item
  If an academic with one child has few children, then an academic with
  two children has few children.
\end{enumerate}

\begin{enumerate}
\def\labelenumi{(\alph{enumi})}
\setcounter{enumi}{2}
\tightlist
\item
  If an academic with two children has few children, then an academic
  with three children has few children.
\end{enumerate}

\begin{enumerate}
\def\labelenumi{(\alph{enumi})}
\setcounter{enumi}{3}
\tightlist
\item
  If an academic with three children has few children, then an academic
  with four children has few children.
\end{enumerate}

\begin{enumerate}
\def\labelenumi{(\alph{enumi})}
\setcounter{enumi}{4}
\tightlist
\item
  If an academic with four children has few children, then an academic
  with five children has few children.
\end{enumerate}
\end{description}

Arguably premise (2e) is plausible because as a material conditional it
can be seen to be true via the falsity of the antecedent. And at a pinch
I can see (2d) as compellingly true for the same reason. But neither
(2b) nor (2c) strike me as at all compelling. If someone presents this
argument as a Sorites paradox, I simply deny that the paradox-mongerer
can know these premises to be true, or that I have a reason to believe
they are true. To be sure, I don't know which premise is false. (If you
think you know b to be false replace academics in the example with a
more fertile professional group.) But just because I don't know where
the argument fails doesn't mean it presents any kind of paradox. When I
have no reason to accept two, maybe three, of the premises, the argument
falls well short of being paradoxical.

A small note on terminology. Contemporary scientific theories imply that
many familiar vague predicates apply in virtue of facts about the world
that are, at some level, discrete. What I'm interested in under this
heading are predicates where the differences between salient adjacent
cases are easily observable, such as the difference between having two
and three children.

\subsection{Vagueness without
Boundaries}\label{vagueness-without-boundaries}

The letter of Patrick Greenough's proposal (to be discussed in
Section~\ref{sec-greenough} below) suggests that every vague term has
only vague boundaries. This is not true. The predicate \emph{in one's
early thirties} has a sharp boundary at the lower end and a precise
boundary at the upper end. But it isn't too hard to amend his theory to
allow for such cases, by saying (in effect) that a vague term is a term
with at least one vague boundary. Nicholas Smith makes basically that
move in his paper. But such a move won't work, because some vague
predicates don't have boundaries. Indeed, some predicates can be vague
even though they are satisfied by every object in the domain. The
examples here are a little more complicated than in the rest of the
paper, but I think they are important enough to warrant the complexity.

For the next several paragraphs the domain will be adult Australian
women, and when I use \emph{tall} I'll mean tall for an adult Australian
woman. I don't know enough facts to know where the boundaries are for
\emph{tall} in this context, but I'll stipulate that a woman shorter
than 170cm is determinately not tall, and a woman taller than 180cm is
determinately tall. I claim here neither that I know where these
boundaries are nor that I could know where they are. But I assume there
are boundaries. I'm making these stipulations because it is easier to
follow the examples if I use 170 and 180 rather than variables like
\emph{y} and \emph{z}. It will become obvious that the particular
numbers won't matter, as long as there's separation between them. It
also doesn't matter whether we use a semantic or epistemic account of
determinacy here. It will matter that we use classical logic at various
points (e.g.~in assuming there are boundaries), but I think that's
perfectly reasonable in this context. (Here I follow the arguments in
section 2 of Greenough's paper.)

Consider the class of predicates defined by the following schema.

\begin{quote}
\emph{tall\textsubscript{x}} =\textsubscript{df} tall or shorter than
\emph{x} cm
\end{quote}

For \emph{x} \textless{} 170, \emph{tall\textsubscript{x}} has all the
same borderline cases as \emph{tall}, and is presumably vague in
anyone's book. For \emph{x}~\textgreater~180,
\emph{tall\textsubscript{x}} determinately applies to everyone in the
domain, and for now we'll say that makes it not vague. (Though note it
need not determinately determinately apply to everyone in the domain,
and we'll see below that might be a reason to group it with the vague
predicates.)

When \emph{x} is between 170 and 180, \emph{tall\textsubscript{x}} has
some very odd properties. The borderline cases are those women whose
height is between \emph{x} and 180cm. When \emph{x} is close to 180,
this might be a very small border. While we're assuming classical logic,
we can assume that there is a value \emph{y} such that women taller than
\emph{y} cm are tall and those shorter than \emph{y} cm are not tall. We
need not here assume the value of \emph{y} is either epistemically
\emph{or semantically} determinate. Consider a value of \emph{x}, say
179, such that \emph{x}~\textgreater~\emph{y}. (Again it's not a
necessary assumption that 179~\textgreater~\emph{y}, but it makes the
example easier to understand if I use a particular number.) Now
\emph{tall}\textsubscript{179} has some interesting properties. It has
borderline cases, those women between 179 and 180cm tall. But it is
satisfied by every woman, since every woman is either tall or shorter
than 179cm. I think the existence of the borderline cases is sufficient
to make \emph{tall}\textsubscript{179} vague. Note that these cases are
quite different to \emph{child}, because at the upper boundary there is
no sharp jump from borderline cases to clear cases -- the two blur
together in just the way borderline cases and clear cases of \emph{tall}
blur together, so whatever reasons we had to worry about \emph{child}
being vague are not applicable here. Still the `borderline cases' are
mislabelled here for there is no border they fall on. Every woman
satisfies the predicate. So no definition of vagueness in terms of
having a vague boundary, indeed of having a boundary at all, can work.

One might object here that a definition of vagueness is only meant to
apply to words not phrases. But just as we can worry about a possible
word \emph{child}, we can worry about a possible atomic word \emph{gish}
that means the same thing as \emph{tall}\textsubscript{179}, so that
move won't help here.

We now have enough data on the table. In the next section I argue that
treating vagueness as being indeterminacy provides a \emph{satisfactory}
treatment of the data. In the third section I argue that none of the
live alternatives is so satisfactory. So I conclude, somewhat
tentatively, that we should define vagueness as indeterminacy.

\section{Vagueness as Indeterminacy}\label{vagueness-as-indeterminacy}

Back when I was a supervaluationist, I thought that what it was for a
term to be vague was for it to refer to different things on different
precisifications. That won't do as a theory-neutral definition, for it
presupposes supervaluationism, which is not only a theory but a false
theory. But we can capture the essential idea in slightly less loaded
language.

I will have to make three possibly controversial assumptions. First, I
assume a broadly Montagovian perspective, on which we can talk about the
referent of an arbitrary term. (See
(\citeproc{ref-Montague1970}{Montague 1970},
\citeproc{ref-Montague1973}{1973}) for more details.) That referent
might be an object, or a truth value, or a function from objects to
truth values, or a more complicated function built out of these. Second,
I assume we can sensibly use an expanded Lagadonian language where
objects can be names for themselves, truth-values can be names for
themselves, functions from objects to truth-values can be names for
themselves, and so on. (See Lewis 1986 for more on Lagadonian
languages.) Third, I assume there is no metaphysical vagueness, so each
of these Lagadonian names is not vague.

Those assumptions let us make a first pass at a definition of vagueness,
as follows. A term \emph{t} is vague iff there is some object,
truth-value or function \emph{l} which can serve as its own name such
that the following sentence is neither determinately true nor
determinately false.

That delivers the intuitively correct account in four of the five cases
we discuss above, all except the cases like \emph{child}*. I'll say much
more about that case below. But it is in one respect slightly too
liberal, and we need to make a small adjustment or two to fix this.
Consider a predicate \emph{F} that is defined over a vague domain, but
which is determinately satisfied by every object in the domain.
Intuitively it is a partial function, which maps every member of its
domain to \emph{true}. And assume for sake of argument that it is
determinate that it maps every member the domain to \emph{true}. (Say,
for example, it means \emph{is self-identical} when applied to a member
of the domain.) Such a predicate is not, I think, vague. But since it is
indeterminate which partial function it denotes, the above theory
suggests it is vague. We need to make a small adjustment. To state the
corrected theory, we will stipulate that \emph{every} term denotes a
function. What were previously thought of as terms denoting constants
will be treated as terms denoting constant functions. So instead of a
name like \emph{Scott Soames} denoting Scott Soames, we'll take it to
denote the function that takes anything whatsoever as input, and
returning Scott Soames as output. Given that, our second take at a
definition of vagueness is as follows.

\begin{quote}
\emph{t} is vague iff ∃\emph{x}, \emph{y}\textsubscript{1},
\emph{y}\textsubscript{2} such that
\emph{y}\textsubscript{1}~≠~\emph{y}\textsubscript{2} and it is
indeterminate whether ∃\emph{l} such that \emph{t} denotes \emph{l} and
\emph{l}(\emph{x})~=~\emph{y}\textsubscript{1}, and it is indeterminate
whether ∃\emph{l} such that \emph{t} denotes \emph{l} and
\emph{l}(\emph{x})~=~\emph{y}\textsubscript{2}.
\end{quote}

To get a sense of the definition, it helps to translate it back into
supervaluational talk, and look at the special case where \emph{t} is a
predicate. Then the definition comes to the claim that there is some
object that is in the extension of \emph{t} on one precisification, and
in the anti-extension of \emph{t} on another, which seems like what was
intended.

Arguably even that is not enough of a correction. (I'm indebted in the
following three paragraphs to Mark Johnston.) Frequently there are
debates in semantics over the appropriate \emph{type} of various
terms.\footnote{In what follows I'll refer to functions of type ⟨X,~Y⟩.
  These are functions from things of type X to things of type Y, where
  the basic types are entities, represented by e, and truth values,
  represented by t. So a function of type ⟨e,t⟩ is a function from
  objects to truth values, or, equivalently, the characteristic function
  of a set. A function of type ⟨⟨e, t⟩, ⟨e,t⟩⟩ is a function from
  characteristic functions of sets to characteristic functions of sets.
  This is plausibly the semantic value of a predicate modifier like
  \emph{very}.} For instance, a straightforward account would say that
in \emph{She ran yesterday}, \emph{yesterday} modifies the intransitive
verb \emph{run}, so it denotes a function of type ⟨⟨e,t⟩,~⟨e,t⟩⟩. But on
a Davidsonian semantics, \emph{yesterday} denotes a property of the
running event being discussed, so its type is simply ⟨e,t⟩. Now it is at
least a philosophical possibility that there should be no fact of the
matter which of these theories is correct.

There are two things we might say about such a possibility. On the one
hand, it doesn't at all seem right that a word should count as vague
because it is indeterminate what its type should be. That suggests the
above definition needs modification. On the other hand, the above
definition doesn't imply \emph{t} is vague whenever there are two
distinct functions that could be the denotation of \emph{t}; it must
also be the case that these functions have overlapping domains. The most
natural cases of syntactic indeterminacy are cases where the two
possible denotations are functions of quite different types. That
suggests the above definition needs no modification.

I think the case for modification is a little stronger. That's partially
because the possibility of type-shifting suggests there's a possibility,
perhaps a distant one but a possibility, that the second suggestion
could fail. And partially because even if there are no uncontroversial
cases of syntactic indeterminacy that will mistakenly be treated as
cases of vagueness by this theory, the mere possibility of classifying a
case of syntactic indeterminacy as a case of vagueness should be enough
to warrant concern. And there is a way to modify the definition that
does not look like it will lead to mistakenly ruling out any cases of
vagueness that should be ruled in, as follows.

\begin{quote}
\emph{t} is vague iff ∃\emph{x}, \emph{y}\textsubscript{1},
\emph{y}\textsubscript{2} such that
\emph{y}\textsubscript{1}~≠~\emph{y}\textsubscript{2} and
\emph{y}\textsubscript{1} is of the same type as
\emph{y}\textsubscript{2}, and it is indeterminate whether ∃\emph{l}
such that \emph{t} denotes \emph{l} and
\emph{l}(\emph{x})~=~\emph{y}\textsubscript{1}, and it is indeterminate
whether ∃\emph{l} such that \emph{t} denotes \emph{l} and
\emph{l}(\emph{x})~=~\emph{y}\textsubscript{2}.
\end{quote}

That implies that if \emph{yesterday} is indeterminate merely because it
is indeterminate what type of function it denotes, it won't count as
vague, and that's all to the good. So this is our final definition of
vagueness.

Still there's a problem with \emph{child}. Many people have thought that
it should not be considered vague for one reason or another. Sometimes
this is just asserted as a raw intuition, as in Smith and Eklund.
There's no arguing with an intuition, so I won't try arguing with it.
Rather I'll just repeat a point I made at the start. We aren't here in
the business simply of summarising ordinary or philosophical intuitions.
Rather we are looking for a definition that captures all the cases that
fall into the most theoretically important categories. And intuitions
about theoretical importance are less impressive than demonstrations of
theoretical importance.

Patrick Greenough (\citeproc{ref-Greenough2003}{2003}) suggests that the
problem with terms like \emph{child} is that they aren't vague, but
rather that they are simply undefined for the alleged borderline cases.
If that's true, and perhaps for some of the examples people had in mind
in this area it is, then our definition agrees that they are not vague.
For a term that carves a precise division out of part of the domain, and
then stays silent, is precise not vague on my account.

Greenough also suggests that the problem with \emph{child} is that it is
not higher-order vague.\footnote{When I say a term is higher-order
  vague, I mean that it is subject to higher-order vagueness, not that
  it is vague whether the term is vague.} But as he says this can hardly
be the entirety of the problem. For it does not seem to be
\emph{definitional} that the vague terms are also higher-order vague.
True, there is a theoretically important category of terms that are
vague and higher-order vague. But it is not a category that we cannot
represent. A term \emph{t} is in this category just in case \emph{t} is
vague, and \emph{definitely t} is vague, and \emph{definitely definitely
t} is vague, and so on. So we can capture that category, even if we
don't call only members of that category the vague terms. And this
doesn't seem to diminish the theoretical importance of the category of
terms I called vague.

It might be thought that what is wrong with \emph{child} is that it
cannot be used to generate a Sorites argument. If you think that's what
is centrally important to vague terms, then there's a theoretical reason
to separate \emph{child} from the genuinely vague. But we should have
seen enough by now to show that that can't be right. It's hard to know
what it is for a predicate modifier to be Sorites-susceptible, and our
last two example predicates, \emph{has few children} and
\emph{tall}\textsubscript{179} cannot be used to set up Sorites
arguments. So that \emph{child} does not generate a Sorites paradox is
no reason to classify it outside the vague.

So I take it there is no compelling reason to classify \emph{child} and
similar terms as precise rather than vague. Admittedly there is an
intuition that they are not vague, and perhaps that should be respected.
But if the cost of respecting that intuition is that we misclassify
several other terms, we should reject the intuition. That's what I'll
argue in the next section.

\section{Rival Definitions}\label{sec-greenough}

I just mentioned the idea that a vague predicate could be defined as one
that is susceptible to a Sorites argument. This account is sometimes
attributed to Delia Graff Fara (\citeproc{ref-Fara2000}{2000}), but it
seems quite a widespread view. For instance, Terence Horgan
(\citeproc{ref-Horgan1995}{1995}) says that it is distinctive of vague
predicates that they can be used to generate \emph{inconsistency}
because the Sorites premises attaching to them are \emph{true}. As I
mentioned, such views are vulnerable to a wide variety of
counterexamples. Many of these counterexamples also apply to rival
definitions of vagueness.

Matti Eklund (\citeproc{ref-Eklund2005}{2005}) develops a similar kind
of definition. He starts with Crispin Wright's
(\citeproc{ref-Wright1975}{1975}) famous definition of what it is for a
predicate \emph{F} to be \emph{tolerant}.

\begin{quote}
Whereas large enough differences in \emph{F}'s parameter of application
sometimes matter to the justice with which it is applied, some small
enough difference never thus matters.
\end{quote}

Eklund's position then is that \emph{F} is vague iff it is part of
semantic competence with respect to \emph{F} to be disposed to accept
that \emph{F} is tolerant. Eklund agrees that it is inconsistent to
assert that \emph{F} is indeed tolerant. But as he has argued
extensively elsewhere, the falsity of the tolerance principle is
compatible with it being part of competence that one is disposed to
accept it. (A view in the same family is put forward in Sorensen
(\citeproc{ref-Sorensen2001}{2001}).) I have no wish to dispute this
part of Eklund's theory. Indeed that meaning principles can be false,
even inconsistent, it seems to have been a fairly fruitful idea in a
variety of areas of Eklund's philosophy. But I don't think it helps with
vague terms.

Three of the problems with this have already been given. It is not clear
what a parameter of application for a non-predicate like \emph{very}
even is, so it isn't clear what it means to say that \emph{very} is
tolerant. It surely is not required of competent users of \emph{few
children} that they are disposed to accept the premises in our earlier
Sorites argument. And for some vague predicates, like
\emph{tall}\textsubscript{179}, the tolerance principle is not plausible
to a competent speaker because it is not plausible that a ``large
enough'' difference in the parameter of application (presumably height)
matters. These problems all seem to carry over from the problems
associated with Sorites based definitions.

I suspect, though I'm less certain here, that the philosophically
interesting cases also pose a problem for Eklund's view. When we look at
philosophically interesting cases, like \emph{being good}, there are two
distinct ways to read Eklund's claim that competent speakers are
disposed to accept the tolerance principle. These are the wide scope and
the narrow scope reading. To see the ambiguity, let's write out Eklund's
principle in full.

\begin{quote}
Competent speakers are disposed to accept that whereas large enough
differences in \emph{F}'s parameter of application sometimes matter to
the justice with which it is applied, some small enough difference never
thus matters.
\end{quote}

Here's the wide scope reading of this.

\begin{quote}
\emph{F}'s parameter of application is such that whereas competent
speakers are disposed to accept that large enough differences in it
sometimes matter to the justice with which \emph{F} is applied, some
small enough difference never thus matters.
\end{quote}

And here is the narrow scope reading, with a phrase added for emphasis.

\begin{quote}
Competent speakers are disposed to accept that whereas large enough
differences in \emph{F}'s parameter of application, \emph{whatever it
is}, sometimes matter to the justice with which it is applied, some
small enough difference never thus matters.
\end{quote}

To see the difference between the two cases, assume for the sake of
argument that a competent speaker thinks that to be good is to do
actions whose consequences have a high enough utility, whereas in
reality to be good is to obey enough of God's commands. In each case
\emph{being good} is vague, because we are using satisficing versions of
consequentialism and divine command theory. So the parameter of
application for \emph{being good} is the number of God's commands you
obey. The competent speaker will not accept the wide scope version of
tolerance with respect to \emph{being good}, because they don't think
that large differences with respect to how many of God's commands you
obey matter to the justice with which \emph{being good} is applied. Such
cases can be multiplied endlessly to show that the wide scope version of
Eklund's principle cannot generally be true, because it makes it the
case that competent speakers have correct views on contentious
philosophical matters the resolution of which goes beyond semantic
competence. For these reasons Eklund has said (personal communication)
that he intends the narrow scope version.

But the narrow scope version also faces some difficulties. The most
direct problem is that one can be a competent user of a term like
\emph{food} or \emph{dangerous} or \emph{beautiful} without having any
thoughts about parameters of application. I suspect I was a competent
user of these terms before I even had the concept of a parameter of
application. Even bracketing this concern, there is a worry that
competence requires knowing of a term whether it is vague or not. But
this seems to be a mistake. It is not a requirement of competence with
moral terms like \emph{good} that one know whether they are maximising
or satisficing terms. Tom Wolfe and the students he observed while
writing \emph{I Am Charlotte Simmons} seemed to disagree about whether
\emph{going out with} is vague, but they were both competent users, they
simply disagreed on something like a normative question. (See Wolfe
(\citeproc{ref-Wolfe2000}{2000}) for more on his take on matters.) And
it seems that two users of language could disagree over whether \emph{is
thinking} is vague without disagreeing over whether either is a
competent semanticist. They may well disagree over whether either is a
competent philosopher of mind, but such disagreements are neither here
nor there with respect to our present purposes. So I don't think that
either disambiguation of Eklund's principle can properly account for
vagueness in philosophically interesting terms.

Nicholas Smith argues for a definition of vagueness that uses some
heavier duty assumptions about the foundations of semantics. In
particular, he sets out the following definition,

\begin{quote}
\textbf{Closeness}\\
If \emph{a} and \emph{b} are very similar in \emph{F}-relevant respects,
then \emph{Fa} and \emph{Fb} are very similar in respect of truth.
\end{quote}

and goes on to say that vague predicates are those that satisfy
non-vacuously satisfy Closeness over some part of their domain. For this
to work there must be, as Smith acknowledges, both degrees of truth and
something like a distance metric defined on them. (These are separate
assumptions; in the theory of Weatherson 2005 the first is true but not
the second.) I won't question those assumptions, but rather focus on the
problems the definition has even granting the assumptions.

As with the two definitions considered so far, it is hard to see how
this could possibly be generalised to cover vagueness in non-predicates.
It's true (given our assumptions) that if \emph{a} and \emph{b} are
similar in \emph{very tall}-relevant respects, then `\emph{a} is very
tall' and `\emph{b} is very tall' will be similar in respect of truth.
But that doesn't show \emph{very} is vague, for the same condition is
satisfied when we replace \emph{very} with the precise modifier
\emph{doubly}. This isn't an argument that Smith's definition
\emph{couldn't} be extended to cover modifiers, but a claim that it is
hard to see how this will work.

The definition also has trouble with \emph{tall}\textsubscript{179} for
this satisfies Closeness vacuously. Though to be fair given the logical
assumptions Smith makes, it is possible that no predicate with the
properties I've associated with \emph{tall}\textsubscript{179} can be
defined.

More seriously, there is a problem with predicates like \emph{has few
children}. It just isn't true that ``An academic with two children has
few children'' is close in truth value to ``An academic with one child
has few children''. In general Smith's theory has trouble with,
i.e.~rules out by definition, vague terms where the underlying `relevant
respects' are highly discrete. Note that the problem here extends to
some predicates where the underlying facts are continuous. Consider the
predicate \emph{is very late for the meeting}. At least where I come
from, a person who is roughly ten minutes late is a borderline case of
this predicate. But which side of ten minutes late they are matters. (In
what follows I make some wild guesses about how numerical degrees of
truth, which aren't part of my preferred theory, should operate. But I
think the guesses are defensible given the empirical data.) If Alice is
nine and three-quarters minutes late, and Bob is ten and a quarter
minutes late, then the degree of truth of ``Alice is very late'' will be
much smaller than the degree of truth of ``Bob is very late''. The later
you are the truer ``you are very late'' gets, but crossing
conventionally salient barriers like the ten minutes barrier matter much
more to the degree of truth than crossing other barriers like the nine
minutes thirty-three seconds barrier. Smith (in conversation) has
suggested that he's prepared to accept that \emph{is very late for the
meeting} is only partially vague if the truth-values `jump' at the ten
minute mark as I'm suggesting. But this seems improper, for this is as
clear a case of a vague predicate as we have. Still, it's worth
remembering as always that every definition has its costs, and this may
be a cost one chooses to live with. Personally I think it is excessive.

Patrick Greenough did not put forward his theory as a \emph{definition}
of vagueness, but rather as a \emph{minimal theory} to which all
partisans could agree. Like Eklund, Greenough plays off Crispin Wright's
idea of tolerance. Roughly, a vague predicate is one that is
epistemically tolerant -- it's one where you can't know that a small
difference makes a difference. Here's a less rough statement of it,
though note this is heavily paraphrased.

Let τ be a variable that ranges over truth states (e.g.~true,
determinately true, not determinately determinately not determinately
true, etc.) \emph{v} a function from objects to real numbers such that
whether \emph{x} is \emph{F} depends only on the value of
\emph{v}(\emph{x}) (i.e.~\emph{v} is \emph{F}'s parameter of
application) and \emph{c} a suitably small number. Then \emph{F} is
vague iff the following claim non-vacuously holds.

\begin{quote}
∀τ∀⍺∀β∀\emph{a}∀\emph{b}, if \emph{v}(α) -
\emph{v}(β)~\textless~\emph{c} and \emph{a} names α and \emph{b} names β
and it is knowable that \emph{Fa} is τ then it is not knowable that
\emph{Fb} is not τ.
\end{quote}

Less formally, we can't know where any boundary at any order of
definiteness for \emph{F} lies. (It isn't clear in Greenough's
presentation exactly what the non-vacuous condition comes to. He only
explicitly says that for the special case where τ is \emph{is true}
there must be an \emph{a} and a \emph{b} such it is knowable that
\emph{Fa} is τ and \emph{Fb} is not τ, but maybe that should be extended
to all τ.) Because of cases like \emph{in one's early thirties} this
cannot do as a general definition, but it is easy enough to repair it by
restricting the quantifier attaching to \emph{a} and \emph{b} to a range
over which \emph{F} has only vague boundaries. Doing this amounts to
weakening Greenough's claim from the view that vague terms have only
vague boundaries to the view that they have some vague boundaries, which
seems plausible. But still there are problems.

Most obviously, \emph{tall}\textsubscript{179} does not non-vacuously
satisfy the tolerance requirement. And like all the tolerance-based
theories it is far from clear how it should be extended to vagueness in
non-predicates. On the other hand, Greenough's theory might well handle
the discrete cases like \emph{has few children}. I say \emph{might}
rather than \emph{does} because it is rather hard to work out how the
higher-orders of vagueness work for such terms. I'll simply note that
there are some plausible enough epistemic models on which \emph{has few
children} satisfies his requirement.

There is a problem which is distinctive to Greenough's view of his
theory as a minimal theory. As Smith notes, Greenough makes it a
requirement that vague boundaries are unknown. But this is controverted
in some mainstream theories, for example the version of
supervaluationism in Dorr (\citeproc{ref-Dorr2003}{2003}). Since Dorr's
theory should not be ruled out by a minimal theory or a definition, this
is a weakness in Greenough's theory.

The more philosophically interesting problems concern, appropriately
enough, the philosophically interesting terms. Greenough has a proof
that his definition is equivalent to a definition in terms of borderline
cases. The proof has several assumptions, one of which being that we
know what the parameter of application of a vague term is. More
precisely, he assumes that we know everyone older than an old person is
old, which is unproblematic, but he also assumes that the proof
generalises to all vague cases, and this amounts to the assumption that
we know parameters of application. As we've seen, this isn't true of
philosophically interesting vague terms. This leaves open the
possibility that Greenough's theory, unlike Smith's and Eklund's
theories, overgenerates. The following is probably not a live
possibility in any interesting sense, but it isn't I think the kind of
thing a definition (or minimal theory) should rule out by definition.

It is possible that a kind of mysterianism about ethics is true, and we
cannot know whether \emph{good} is vague or precise. For a concrete
example, let's assume it is knowable that some kind of divine command
theory is true, but it is unknowable whether to be good one must obey
all of God's commands or merely enough of them, where it is vague what
counts as enough of them. In fact morality requires obeying all God's
commands, but this is not knowable -- for all we know the satisficing
version is the true moral theory. If this is the case then \emph{good}
will be epistemically tolerant, for we cannot know that a small
difference in how many of God's commands you obey makes a difference to
whether you are good, or determinately good etc. But in fact \emph{good}
is precise, for it precisely means obeying all of God's commands.
Earlier I objected to Eklund's theory because semantic competence does
not require knowing parameters of application, especially as such. This
is the converse objection -- I claim that a term's being precise does
not imply that we know, or even \emph{could know}, that it applies in
virtue of a precise condition. All that matters is that it \emph{does}
apply in virtue of a precise condition.

It's a constant danger in philosophy that one infer from the falsity of
all extant rivals that one's preferred theory is correct. I certainly
don't want to argue that because Eklund's, Smith's and Greenough's
definitions are incorrect that the traditionalist definition I have
offered must be right. But we can make that conclusion more plausible by
noting how widely the arguments levelled here generalise. The
philosophically interesting cases seem to tell against \emph{any}
definition of vagueness in terms of semantic competence, for they show
that competent users can have exactly the same attitude towards vague
terms as they have towards precise terms. And our moral example suggests
that any definition in terms of epistemic properties will be in trouble
for it might not be knowable whether a particular term is vague or
precise. Finally, the cases of vague predicate modifiers raise
difficulties for any attempt to define the vagueness of a term in terms
of properties of sentences in which it is used rather than mentioned.
For it seems that as long as \emph{very} attaches only to vague
predicates, then whether \emph{very}* is vague or precise will make no
salient differences to the sentences in which it appears. So we have to
look at sentences in which the allegedly vague term is mentioned. And
while I don't have a definitive argument here, I think looking at the
range of cases we want to cover, and in particular at the range of cases
where tolerance-type principles fail to be non-vacuously satisfied, our
best option for completing these sentences is to look whether the term
has a determinate or indeterminate denotation. We can then pass the
questions of what determinacy consists in, and in particular the
question of whether it is an epistemic or semantic feature, to the
theorist of vagueness.

\subsection*{References}\label{references}
\addcontentsline{toc}{subsection}{References}

\phantomsection\label{refs}
\begin{CSLReferences}{1}{0}
\bibitem[\citeproctext]{ref-Dorr2003}
Dorr, Cian. 2003. {``Vagueness Without Ignorance.''} \emph{Philosophical
Perspectives} 17: 83--113. doi:
\href{https://doi.org/10.1111/j.1520-8583.2003.00004.x}{10.1111/j.1520-8583.2003.00004.x}.

\bibitem[\citeproctext]{ref-Eklund2005}
Eklund, Matti. 2005. {``What Vagueness Consists In.''}
\emph{Philosophical Studies} 125 (1): 27--60. doi:
\href{https://doi.org/10.1007/s11098-005-7773-1}{10.1007/s11098-005-7773-1}.

\bibitem[\citeproctext]{ref-Fara2000}
Fara, Delia Graff. 2000. {``Shifting Sands: An Interest-Relative Theory
of Vagueness.''} \emph{Philosophical Topics} 28 (1): 45--81. doi:
\href{https://doi.org/10.5840/philtopics20002816}{10.5840/philtopics20002816}.
This paper was published under the name {``Delia Graff.''}

\bibitem[\citeproctext]{ref-Field1973}
Field, Hartry. 1973. {``Theory Change and the Indeterminacy of
Reference.''} \emph{Journal of Philosophy} 70 (14): 462--81. doi:
\href{https://doi.org/10.2307/2025110}{10.2307/2025110}.

\bibitem[\citeproctext]{ref-Fine1975a}
Fine, Kit. 1975. {``Vagueness, Truth and Logic.''} \emph{Synthese} 30
(3-4): 265--300. doi:
\href{https://doi.org/10.1007/bf00485047}{10.1007/bf00485047}.

\bibitem[\citeproctext]{ref-Greenough2003}
Greenough, Patrick. 2003. {``Vagueness: A Minimal Theory.''} \emph{Mind}
112 (446): 235--81. doi:
\href{https://doi.org/10.1093/mind/112.446.235}{10.1093/mind/112.446.235}.

\bibitem[\citeproctext]{ref-Horgan1995}
Horgan, Terrence. 1995. {``Transvaluationism: A Dionysian Approach to
Vagueness.''} \emph{Southern Journal of Philosophy} 33: Spindel
Conference Supplement: 97--125. doi:
\href{https://doi.org/10.1111/j.2041-6962.1995.tb00765.x}{10.1111/j.2041-6962.1995.tb00765.x}.

\bibitem[\citeproctext]{ref-Montague1970}
Montague, Richard. 1970. {``Universal Grammar.''} \emph{Theoria} 36 (3):
373--98. doi:
\href{https://doi.org/10.1111/j.1755-2567.1970.tb00434.x}{10.1111/j.1755-2567.1970.tb00434.x}.

\bibitem[\citeproctext]{ref-Montague1973}
---------. 1973. {``The Proper Treatment of Quantification in Ordinary
English.''} In \emph{Approaches to Ordinary Language}, edited by K. J.
J. Hintikka, J. M. E. Moravcsik, and P. Suppes, 221--42. Dordrecht:
Reidel.

\bibitem[\citeproctext]{ref-Sainsbury1991}
Sainsbury, Mark. 1991. {``Is There Higher-Order Vagueness?''} \emph{The
Philosophical Quarterly} 41 (163): 167--82. doi:
\href{https://doi.org/10.2307/2219591}{10.2307/2219591}.

\bibitem[\citeproctext]{ref-Sorensen2001}
Sorensen, Roy. 2001. \emph{Vagueness and Contradiction}. Oxford: Oxford
University Press.

\bibitem[\citeproctext]{ref-Wolfe2000}
Wolfe, Tom. 2000. \emph{Hooking up}. Farrar, Strauss; Giroux: New York.

\bibitem[\citeproctext]{ref-Wright1975}
Wright, Crispin. 1975. {``On the Coherence of Vague Predicates.''}
\emph{Synthese} 30 (3-4): 325--65. doi:
\href{https://doi.org/10.1007/bf00485049}{10.1007/bf00485049}.

\end{CSLReferences}



\noindent Published in\emph{
Cuts and Clouds: Vagueness, its Nature, and its Logic}, 2018, pp. 77-90.


\end{document}
