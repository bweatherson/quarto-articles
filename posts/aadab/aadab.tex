% Options for packages loaded elsewhere
\PassOptionsToPackage{unicode}{hyperref}
\PassOptionsToPackage{hyphens}{url}
%
\documentclass[
  10pt,
  letterpaper,
  DIV=11,
  numbers=noendperiod,
  twoside]{scrartcl}

\usepackage{amsmath,amssymb}
\usepackage{setspace}
\usepackage{iftex}
\ifPDFTeX
  \usepackage[T1]{fontenc}
  \usepackage[utf8]{inputenc}
  \usepackage{textcomp} % provide euro and other symbols
\else % if luatex or xetex
  \usepackage{unicode-math}
  \defaultfontfeatures{Scale=MatchLowercase}
  \defaultfontfeatures[\rmfamily]{Ligatures=TeX,Scale=1}
\fi
\usepackage{lmodern}
\ifPDFTeX\else  
    % xetex/luatex font selection
  \setmainfont[ItalicFont=EB Garamond Italic,BoldFont=EB Garamond
Bold]{EB Garamond Math}
  \setsansfont[]{Europa-Bold}
  \setmathfont[]{Garamond-Math}
\fi
% Use upquote if available, for straight quotes in verbatim environments
\IfFileExists{upquote.sty}{\usepackage{upquote}}{}
\IfFileExists{microtype.sty}{% use microtype if available
  \usepackage[]{microtype}
  \UseMicrotypeSet[protrusion]{basicmath} % disable protrusion for tt fonts
}{}
\usepackage{xcolor}
\usepackage[left=1in, right=1in, top=0.8in, bottom=0.8in,
paperheight=9.5in, paperwidth=6.5in, includemp=TRUE, marginparwidth=0in,
marginparsep=0in]{geometry}
\setlength{\emergencystretch}{3em} % prevent overfull lines
\setcounter{secnumdepth}{3}
% Make \paragraph and \subparagraph free-standing
\ifx\paragraph\undefined\else
  \let\oldparagraph\paragraph
  \renewcommand{\paragraph}[1]{\oldparagraph{#1}\mbox{}}
\fi
\ifx\subparagraph\undefined\else
  \let\oldsubparagraph\subparagraph
  \renewcommand{\subparagraph}[1]{\oldsubparagraph{#1}\mbox{}}
\fi


\providecommand{\tightlist}{%
  \setlength{\itemsep}{0pt}\setlength{\parskip}{0pt}}\usepackage{longtable,booktabs,array}
\usepackage{calc} % for calculating minipage widths
% Correct order of tables after \paragraph or \subparagraph
\usepackage{etoolbox}
\makeatletter
\patchcmd\longtable{\par}{\if@noskipsec\mbox{}\fi\par}{}{}
\makeatother
% Allow footnotes in longtable head/foot
\IfFileExists{footnotehyper.sty}{\usepackage{footnotehyper}}{\usepackage{footnote}}
\makesavenoteenv{longtable}
\usepackage{graphicx}
\makeatletter
\def\maxwidth{\ifdim\Gin@nat@width>\linewidth\linewidth\else\Gin@nat@width\fi}
\def\maxheight{\ifdim\Gin@nat@height>\textheight\textheight\else\Gin@nat@height\fi}
\makeatother
% Scale images if necessary, so that they will not overflow the page
% margins by default, and it is still possible to overwrite the defaults
% using explicit options in \includegraphics[width, height, ...]{}
\setkeys{Gin}{width=\maxwidth,height=\maxheight,keepaspectratio}
% Set default figure placement to htbp
\makeatletter
\def\fps@figure{htbp}
\makeatother
% definitions for citeproc citations
\NewDocumentCommand\citeproctext{}{}
\NewDocumentCommand\citeproc{mm}{%
  \begingroup\def\citeproctext{#2}\cite{#1}\endgroup}
\makeatletter
 % allow citations to break across lines
 \let\@cite@ofmt\@firstofone
 % avoid brackets around text for \cite:
 \def\@biblabel#1{}
 \def\@cite#1#2{{#1\if@tempswa , #2\fi}}
\makeatother
\newlength{\cslhangindent}
\setlength{\cslhangindent}{1.5em}
\newlength{\csllabelwidth}
\setlength{\csllabelwidth}{3em}
\newenvironment{CSLReferences}[2] % #1 hanging-indent, #2 entry-spacing
 {\begin{list}{}{%
  \setlength{\itemindent}{0pt}
  \setlength{\leftmargin}{0pt}
  \setlength{\parsep}{0pt}
  % turn on hanging indent if param 1 is 1
  \ifodd #1
   \setlength{\leftmargin}{\cslhangindent}
   \setlength{\itemindent}{-1\cslhangindent}
  \fi
  % set entry spacing
  \setlength{\itemsep}{#2\baselineskip}}}
 {\end{list}}
\usepackage{calc}
\newcommand{\CSLBlock}[1]{\hfill\break\parbox[t]{\linewidth}{\strut\ignorespaces#1\strut}}
\newcommand{\CSLLeftMargin}[1]{\parbox[t]{\csllabelwidth}{\strut#1\strut}}
\newcommand{\CSLRightInline}[1]{\parbox[t]{\linewidth - \csllabelwidth}{\strut#1\strut}}
\newcommand{\CSLIndent}[1]{\hspace{\cslhangindent}#1}

\setlength\heavyrulewidth{0ex}
\setlength\lightrulewidth{0ex}
\usepackage[automark]{scrlayer-scrpage}
\clearpairofpagestyles
\cehead{
  Brian Weatherson
  }
\cohead{
  Anti-Anti-Desire-As-Belief
  }
\ohead{\bfseries \pagemark}
\cfoot{}
\makeatletter
\newcommand*\NoIndentAfterEnv[1]{%
  \AfterEndEnvironment{#1}{\par\@afterindentfalse\@afterheading}}
\makeatother
\NoIndentAfterEnv{itemize}
\NoIndentAfterEnv{enumerate}
\NoIndentAfterEnv{description}
\NoIndentAfterEnv{quote}
\NoIndentAfterEnv{equation}
\NoIndentAfterEnv{longtable}
\NoIndentAfterEnv{abstract}
\renewenvironment{abstract}
 {\vspace{-1.25cm}
 \quotation\small\noindent\rule{\linewidth}{.5pt}\par\smallskip
 \noindent }
 {\par\noindent\rule{\linewidth}{.5pt}\endquotation}
\KOMAoption{captions}{tableheading}
\makeatletter
\@ifpackageloaded{caption}{}{\usepackage{caption}}
\AtBeginDocument{%
\ifdefined\contentsname
  \renewcommand*\contentsname{Table of contents}
\else
  \newcommand\contentsname{Table of contents}
\fi
\ifdefined\listfigurename
  \renewcommand*\listfigurename{List of Figures}
\else
  \newcommand\listfigurename{List of Figures}
\fi
\ifdefined\listtablename
  \renewcommand*\listtablename{List of Tables}
\else
  \newcommand\listtablename{List of Tables}
\fi
\ifdefined\figurename
  \renewcommand*\figurename{Figure}
\else
  \newcommand\figurename{Figure}
\fi
\ifdefined\tablename
  \renewcommand*\tablename{Table}
\else
  \newcommand\tablename{Table}
\fi
}
\@ifpackageloaded{float}{}{\usepackage{float}}
\floatstyle{ruled}
\@ifundefined{c@chapter}{\newfloat{codelisting}{h}{lop}}{\newfloat{codelisting}{h}{lop}[chapter]}
\floatname{codelisting}{Listing}
\newcommand*\listoflistings{\listof{codelisting}{List of Listings}}
\makeatother
\makeatletter
\makeatother
\makeatletter
\@ifpackageloaded{caption}{}{\usepackage{caption}}
\@ifpackageloaded{subcaption}{}{\usepackage{subcaption}}
\makeatother
\ifLuaTeX
  \usepackage{selnolig}  % disable illegal ligatures
\fi
\usepackage{bookmark}

\IfFileExists{xurl.sty}{\usepackage{xurl}}{} % add URL line breaks if available
\urlstyle{same} % disable monospaced font for URLs
\hypersetup{
  pdftitle={Anti-Anti-Desire-As-Belief},
  pdfauthor={Brian Weatherson},
  hidelinks,
  pdfcreator={LaTeX via pandoc}}

\title{Anti-Anti-Desire-As-Belief}
\author{Brian Weatherson}
\date{2024}

\begin{document}
\maketitle
\begin{abstract}
David Lewis put forward a decision theoretic argument against there
being a tight connection between desires and beliefs about the good. I
argue that his argument fails twice over. It makes inconsistent
background assumptions about his opponents' views, and it over-generates
so broadly that if it worked, it would also rule out some standard
economic models. I end with a puzzle that arises from the response to
Lewis. If one responds to moral uncertainty by saying one should
maximise expected moral value, how does one treat cases where one's
action is evidence for or against the goodness of different actions?
\end{abstract}

\setstretch{1.1}
A particular anti-Humean, call her Auntie, believes there is a tight
connection between wanting something and believing that it is good.
David Lewis (\citeproc{ref-Lewis1988}{1988},
\citeproc{ref-Lewis1996}{1996}) has a famous argument that Auntie's view
is incoherent. The point of this note is to respond on Auntie's behalf.

This is not because I agree with Auntie. On the broader question I think
Lewis is right and Auntie is wrong. But Lewis's argument doesn't show
that Auntie is wrong, and it's useful to see why it does not.

The point of this paper is also not to stick up for Auntie when no one
has stuck up for her before. There are lots of replies to Lewis on
Auntie's behalf from all sorts of directions. What's new here is that I
show how two criticisms in particular, one by Huw Price
(\citeproc{ref-Price1989}{1989}) and one by Jessica Collins
(\citeproc{ref-Collins2015}{2015}), fit together. Both of them say that
Auntie should reject one of the assumptions that Lewis attributes to
her. My primary contribution is to show that the two assumptions they
reject are inconsistent. That is, Lewis's argument must fail because it
requires attributing inconsistent background assumptions to Auntie, and
that's certainly unfair.

\section{The Ludovician Argument}\label{the-ludovician-argument}

I'm not going to go over the argument in Lewis's 1988 paper, which
relies on some really controversial assumptions. Instead I'll go over
the 1996 argument, relying heavily on the presentation of it by Collins
and by Ittay Nissan-Rozen (\citeproc{ref-NissanRozen2013}{2013}).

Assume that we have a finite set of worlds. We will use \emph{w} as a
variable over worlds. A world, in this sense, is a specification of the
truth value of all the truth-apt things that are relevant to a
particular decision. The worlds in this sense are more coarse grained
than Ludovician concreta in that they only specify truth values of
relevant propositions, not of all propositions. That's why we can assume
that there are finitely many of them. But these worlds are more fine
grained than Ludovician concreta in a different sense. They will be used
to represent moral uncertainty. So there can be pairs of them that are
descriptively alike but evaluatively distinct. Given the supervenience
of the evaluative on the descriptive, this is impossible for Ludovician
worlds.

For any descriptive proposition A, assume there is a distinct
proposition Å, meaning that A is good. Let V be an agent's value
function, and Pr their credence function, with subscripts representing
what those functions are like after updating. So V\textsubscript{A} and
Pr\textsubscript{A} are the values of the value and credence functions
after updating on A. Strictly speaking given how I've set this up, it is
sets of worlds not individual worlds that get values. But I'll sometimes
write V(\emph{w}) when strictly it should be V(\{\emph{w}\}); I don't
think this can lead to any confusion. (Later I'll also write
Pr(\emph{w}) for the probability of Pr(\{\emph{w}\}); again it shouldn't
result in confusion.)

Lewis's argument against Auntie uses five assumptions. In these
assumptions B is an arbitrary proposition, and A is an arbitrary
\emph{descriptive} proposition.

\begin{description}
\tightlist
\item[Equation]
The way to represent Auntie's anti-Humean view is V(A)~=~Pr(Å).
\item[Invariance]
V\textsubscript{A}(\emph{w})~=~V(\emph{w})
\item[Additivity]
V(A)~=~Σ\textsubscript{\emph{w}}V(\emph{w})Pr(\emph{w}~\textbar~A)
\item[Restricted Conditionalisation]
Pr\textsubscript{A}(B)~=~Pr(B~\textbar~A)
\item[Good-Bad]
All worlds are either GOOD or BAD. If \emph{w} is GOOD, then
V(\emph{w})~=~1, and otherwise V(\emph{w})~=~0.
\end{description}

The last assumption is obviously absurd, but it is useful for setting
out the argument. In any case, if the first four assumptions are true,
then they should be consistent with \textbf{Good-Bad}. Given those
assumptions, here is Lewis's argument.

\begin{longtable}[]{@{}
  >{\raggedleft\arraybackslash}p{(\columnwidth - 4\tabcolsep) * \real{0.1071}}
  >{\raggedright\arraybackslash}p{(\columnwidth - 4\tabcolsep) * \real{0.3929}}
  >{\raggedright\arraybackslash}p{(\columnwidth - 4\tabcolsep) * \real{0.5000}}@{}}
\toprule\noalign{}
\endhead
\bottomrule\noalign{}
\endlastfoot
Pr(Å) & ~=~V(A) & \\
& ~=~Σ\textsubscript{\emph{w}}V(\emph{w}) Pr(\emph{w}~\textbar~A) &
(\textbf{Additivity}) \\
& ~=~Σ\textsubscript{\emph{w}}V\textsubscript{A}(\emph{w})
Pr(\emph{w}~\textbar~A) & (\textbf{Invariance}) \\
& ~=~Σ\textsubscript{\emph{w}}V\textsubscript{A}(\emph{w})
Pr\textsubscript{A}(\emph{w} \textbar~A) & (\textbf{Restricted
Conditionalisation}) \\
& ~=~V\textsubscript{A}(A) & (\textbf{Additivity}, applied to updated
values) \\
& ~=~Pr\textsubscript{A}(Å) & (\textbf{Equation}, again after
updating) \\
& ~=~Pr(Å~\textbar~A) & (\textbf{Restriced Conditionalisation}) \\
\end{longtable}

But it is absurd that A and Å are independent. At least, it's absurd if
evaluative uncertainty is coherent. The following situation seems
perfectly possible. The agent knows someone, call them Peter, who they
greatly admire. Peter faces a difficult decision; let A be that he takes
one option, and B that he takes the other salient option. Right now
agent thinks it is 60\% likely that A is good, and 40\% likely that B is
good. But agent \emph{really} admires Peter. They are sure that whatever
Peter does, it will be good. So conditional on A, their credence in Å is
100\%. This all seems coherent, so the conclusion of Lewis's argument
must be mistaken. Lewis himself argues that independence leads to
incoherence, so the last line of the argument is a reductio.

Now the argument I've given might not exactly look like the argument
Lewis gives. He spends a lot of time in each paper spelling out a
different reason that the independence conclusion is absurd. But despite
the amount of attention Lewis gives to this issue, whether it is
plausible to say A and Å are always independent is not at issue. All of
Auntie's defenders in the literature (at least all the ones I've read)
agree that it is not plausible. They don't try to accept the conclusion
of this reductio; rather they reject one of the premises. The premises
I've presented here are all ones that Lewis makes at some point or other
in setting out the argument for independence. So I think the version
I've given here, following Collins and Ittay-Rozen, is faithful enough
to Lewis.

\section{Questioning the Assumptions}\label{questioning-the-assumptions}

Let's think a bit harder about what Auntie says about this agent who is
waiting to see what admirable Peter will do. Given Auntie's view, should
agent prefer that Peter makes A true, or should they be indifferent
about whether Peter makes A or B true? Both options have some
plausibility. On the one hand, right now the agent thinks that A is a
little likelier to be good. On the other hand, whatever Peter does, the
agent will be completely happy with it, because they will then think
that Peter has done the right thing.

I'm not going to resolve this question for Auntie; at the end I'll come
back to the question and place it in a broader philosophical context.
What I do want to stress is that precisifying Auntie's view requires
saying what she thinks about this question. And the assumptions one
attributes to Auntie should be consistent with her answer. Lewis's
argument does not satisfy this constraint, whatever Auntie says about
agent and Peter.

Assume, first, that agent wants Peter to do A, because it's right now
what is thought to be better. Then agent has to reject
\textbf{Additivity}. \textbf{Additivity} is the rule for people who
evaluate choices conditional on those things being done, not for people
who evaluate choices given their current values. As Collins points out,
it's the rule for one-boxing in Newcomb's Problem, and it is weird that
a two-boxer like Lewis should appeal to it.

So assume, alternatively, that Auntie thinks the agent should be
indifferent between Peter's choices. Then Auntie will reject
\textbf{Equation}. According to \textbf{Equation}, the agent should
value propositions according to their current evaluations of the
goodness of the proposition. But on this assumption, the agent evaluates
propositions like A and B according to how good they are thought to be
conditional on obtaining. That is, Auntie's view is not
\textbf{Equation}, but V(A)~=~Pr(Å~\textbar~A). This is exactly what
Price recommended Auntie adopt immediately after Lewis's first paper
came out.

In the second paper Lewis has a response to Price's suggestion, but as
Hàjek (\citeproc{ref-Hajek2015}{2015}) observes, it is very hard to
understand what the response really is. Lewis states the kind of view
Price endorses, makes a couple of observations about it, and then ends
as if the question is settled. If it's meant to be a reductio, it's
really not clear what the implausible conclusion is. Hàjek speculates
that a paragraph or more simply went missing; the text is puzzling
enough to take such speculations seriously.\footnote{Since Hàjek's paper
  was published, we've had two volumes of correspondance by Lewis
  published (\citeproc{ref-LewisLetters1}{Beebee and Fisher 2020a},
  \citeproc{ref-LewisLetters2}{2020b}). But unfortunately nothing in
  them sheds light on this interpretative question.}

Whatever Auntie says about agent and Peter, she has grounds to reject
one of the assumptions Lewis attributes to her. If she says agent
prefers Peter to make A true, the \textbf{Additivity} assumption should
be rejected; as indeed Collins rejects it. If she says agent is
indifferent between Peter's actions, the \textbf{Equation} should be
rejected; as indeed Price rejects it. Attributing both
\textbf{Additivity} and \textbf{Equation} to Auntie implies that Auntie
inconsistently holds that agent both prefers and does not prefer that
Peter makes A true. Auntie certainly has grounds to reject this
attribution of inconsistent assumptions.

That is to say, while Lewis did succeed in deriving an implausible
result from Auntie's view plus some auxiliary hypotheses, it is
perfectly reasonable to say that the auxiliary hypotheses are to blame
rather than Auntie's view. Once Auntie decides what to say about Peter
and his admirer, she has a conclusive reason to reject one or other of
these auxiliary hypothesis. Whatever other flaws Auntie's view has, it
isn't to blame for the implausible conclusion Lewis derives from it.

\section{Auntie the Capitalist}\label{auntie-the-capitalist}

There is another assumption which Auntie should obviously reject:
\textbf{Good-Bad}. Lewis acknowledges that this is a simplifying
assumption, but says that we can restate Auntie's view without it. The
real assumption is that there for any \emph{w}, there is a numerical
value for \emph{w}, which measures how good it is. Let g be the function
from worlds to goodness, so g(\emph{w})~=~\emph{x} means that \emph{w}
has \emph{x} units of goodness. The assumption that g is a function into
the reals isn't completely trivial, but let's assume Auntie is happy to
live with it. Then really what Lewis needs is \textbf{Corrected
Equation}.

\begin{description}
\tightlist
\item[Corrected Equation]
The way to represent Auntie's anti-Humean view is
\newline V(A)~=~Σ\textsubscript{\emph{w}}g(\emph{w})~Pr(\emph{w}). That
is, agent values A according to its expected goodness.
\end{description}

Lewis shows, using the same assumptions as before, that given this
understanding of Auntie's view, it also leads to absurdity. And as
before, I think his argument requires attributing views to Auntie that
she would surely reject as soon as she makes her mind up about the agent
who admires Peter. But let's say that I'm wrong about that. There is
something else problematic about Lewis's argument at this point.

Nothing else in Lewis's argument turns on the fact that g is a measure
of goodness. The argument goes through just as well (or just as badly)
for any numerical function that g could be. That function could be a
measure of anything. It could, for instance, be a measure of how much
profit agent makes in \emph{w}. In that case, \textbf{Corrected
Equation} says that agent values propositions according to their
expected profitability. That's just the standard theory of the firm from
basic economics. If Lewis's argument shows that Auntie's view is
inconsistent, it also shows that the standard theory of the firm is
inconsistent.

To be sure, there is a lot wrong with the `standard theory of the firm'
as a theory of either real or idealised firms. But it's rather
implausible that it's trivial, and particularly implausible that it
could be shown to be trivial by some simple decision theory. That would
be particularly ironic given how much of decision theory was developed
to explain decision making by idealised firms.

The lesson here is that Lewis's argument over-generates. If it shows
anything, it shows that having one's valuation track the expected value
of any numerical measure is inconsistent. But that can't be right.
Having no priority in the world other than maximising expected profits
might be morally abhorrent, but it isn't inconsistent with decision
theory. So Lewis's argument must be wrong.

\section{A Puzzle}\label{a-puzzle}

That completes my objection to Lewis's argument. I'll end with a puzzle
that arises from the discussion here.

Go back to agent the agent who thinks that whatever Peter does will be
correct. Change the case so that agent is in fact Peter. That is, in
this version of the case, Peter isn't sure what's right, and isn't sure
what he'll do, but is sure that whatever he does will be good. Assume
that Peter only cares about maximising the good, even when he doesn't
know what is in fact good.\footnote{Perhaps Peter was convinced by the
  arguments from MacAskill, Bykvist, and Ord
  (\citeproc{ref-MacAskillEtAl2020}{2020}) that this is what he should
  do.} Question: What should Peter want to have happen, assuming all
this?

I can think of at least four coherent responses to this question.

First, one might think that since Peter thinks A is more likely to be
good, and he wants to do good, he should make A true.

Second, one might think that since Peter will be sure he does the good
thing whatever he does, he should be indifferent between making A true
and making B true.

Third, one might think that this is really just a special case of
Newcomb's Problem, where maximising expected utility according to
unconditional probabilities (over states causally independenrt of one's
action) gives a different recommendation to maximising expected utility
according to conditional probability. This answer says that whatever you
say about Newcomb's Problem, whether you say conditional probabilities
are to be used (as most one-boxers say), or unconditional probabilities
are to be used (as some two-boxers say), the same goes here.

The third answer isn't inconsistent with the previous two. But it is a
distinct answer. It is an answer that people who disagree about
Newcomb's Problem can agree is correct. But it's also a substantive
claim, since there is a coherent way to deny it. In particular, it is
coherent to adopt the version of causal decision theory that Lewis
(\citeproc{ref-Lewis1981bn}{1981}) defends for descriptive uncertainty,
and something that looks like evidential decision theory for moral
uncertainty.

Here is how that might go. Following Bradley and List
(\citeproc{ref-BradleyList2009}{2009}), let worlds be ordered pairs
⟨\emph{d},~\emph{v}⟩, such that \emph{d} settles the (relevant)
descriptive facts, and \emph{v} is a numerical measure of the goodness
of the world.\footnote{This way of thinking about worlds helps explain
  some terminology that I left undefined earlier. A descriptive
  proposition is a proposition \emph{p} such that for any
  \emph{d},~\emph{v}, and \emph{v}ʹ, if ⟨\emph{d},~\emph{v}⟩ and
  ⟨\emph{d},~\emph{v}ʹ⟩ are worlds, then ⟨\emph{d},~\emph{v}⟩~∈~\emph{p}
  iff ⟨\emph{d},~\emph{v}ʹ⟩~∈~\emph{p}.} In the terminology used earlier
g((\emph{d},~\emph{v}))~=~\emph{v}; the second term is how good the
world is.

Let K be a partition of the worlds such that whatever the agent does
makes no causal difference to which member of the partition is actual.
Intuitively, the true element of K is the conjunction of all the facts
that are outside the causal control of the chooser. Crucially, K settles
the \emph{facts} outside the agent's causal control; it does not settle
anything evaluative.\footnote{That is, all the cells of the partition
  are descriptive propositions.} For any proposition A, the value to the
agent of A is given by this equation:

\begin{quote}
V(A) = Σ\textsubscript{\emph{k}∈K}Pr(\emph{k})
Σ\textsubscript{⟨\emph{d},~\emph{v}⟩∈\emph{k}}
\emph{v}Pr(⟨\emph{d},~\emph{v}⟩~\textbar~A~∧~\emph{k})
\end{quote}

The agent should then make true the proposition with the highest value
that it is within their power to make true. The inner sum in this
equation looks like preferred definition of decision-theoretic value for
Evidential Decision Theorists. In this respect I'm following Lewis
closely. As he says, ``Within a single dependency hypothesis, so to
speak, V-maximising {[}i.e., Evidential Decision Theory{]} is right.''
(\citeproc{ref-Lewis1981bn}{Lewis 1981, 7}). The idea here is that if
Lewis could be right about this claim, and all moral uncertainty takes
place within dependency hypotheses, then the puzzle here will not be
just like Newcomb's Problem.

Finally, one might think this version of the Peter example is
incoherent. Couldn't one simply think about what to do, and then having
made a decision, learn what the admirable person thinks is right, and
hence what is right? Well, maybe it's not so simple. Maybe one thinks
that one always acts in the right way, even if one's thoughts are not
always right. Perhaps one has an inner voice, a la Socrates, that
prevents one from \emph{acting} the wrong way, but which only kicks in
at the moment of action.

I'm inclined to think that last possibility, where one is somewhat
confident that one will somehow find oneself unable to act wrongly, is
just conceivable enough for the example to be coherent. Just like with
Newcomb's Problem, all that's needed to get the problem going is that
the action is some evidence of some underlying fact. In Newcomb's
Problem we can get a difference between Evidential Decision Theory and
Causal Decision Theory even with an imperfect demon, as long as their
predictions are known to be better than chance. In this case, we can get
a difference between maximising conditional expected goodness and
unconditional expected goodness as long as the decider thinks their
action is some evidence that they did the right thing. Is that a
coherent assumption to make? I think it probably is, and if so, it
raises an interesting question about the details of views on moral
uncertainty.

\subsection*{References}\label{references}
\addcontentsline{toc}{subsection}{References}

\phantomsection\label{refs}
\begin{CSLReferences}{1}{0}
\bibitem[\citeproctext]{ref-LewisLetters1}
Beebee, Helen, and A. R. J. Fisher, eds. 2020a. \emph{Philosophical
Letters of David k. Lewis}. Vol. 1. Oxford: Oxford University Press.

\bibitem[\citeproctext]{ref-LewisLetters2}
---------, eds. 2020b. \emph{Philosophical Letters of David k. Lewis}.
Vol. 2. Oxford: Oxford University Press.

\bibitem[\citeproctext]{ref-BradleyList2009}
Bradley, Richard, and Christian List. 2009. {``Desire-as-Belief
Revisited.''} \emph{Analysis} 69 (1): 31--37. doi:
\href{https://doi.org/10.1093/analys/ann005}{10.1093/analys/ann005}.

\bibitem[\citeproctext]{ref-Collins2015}
Collins, Jessica. 2015. {``Decision Theory After Lewis.''} In \emph{A
Companion to David Lewis}, edited by Barry Loewer and Jonathan Schaffer,
446--58. John Wiley; Sons.

\bibitem[\citeproctext]{ref-Hajek2015}
Hàjek, Alan. 2015. {``On the Plurality of Lewis's Triviality Results.''}
In \emph{A Companion to David Lewis}, edited by Barry Loewer and
Jonathan Schaffer, 425--45. John Wiley; Sons.

\bibitem[\citeproctext]{ref-Lewis1981bn}
Lewis, David. 1981. {``Causal Decision Theory.''} \emph{Australasian
Journal of Philosophy} 59 (1): 5--30. doi:
\href{https://doi.org/10.1080/00048408112340011}{10.1080/00048408112340011}.

\bibitem[\citeproctext]{ref-Lewis1988}
---------. 1988. {``Desire as Belief.''} \emph{Mind} 97 (387): 323--32.
doi:
\href{https://doi.org/10.1093/mind/xcvii.387.323}{10.1093/mind/xcvii.387.323}.

\bibitem[\citeproctext]{ref-Lewis1996}
---------. 1996. {``Desire as Belief {II}.''} \emph{Mind} 105 (418):
303--13. doi:
\href{https://doi.org/10.1093/mind/105.418.303}{10.1093/mind/105.418.303}.

\bibitem[\citeproctext]{ref-MacAskillEtAl2020}
MacAskill, William, Krister Bykvist, and Toby Ord. 2020. \emph{Moral
Uncertainty}. Oxford: {O}xford {U}niversity {P}ress.

\bibitem[\citeproctext]{ref-NissanRozen2013}
Nissan-Rozen, Ittay. 2013. {``Jeffrey Conditionalization, the Principal
Principle, the Desire as Belief Thesis, and Adams's Thesis.''}
\emph{British Journal for the Philosophy of Science} 64 (4): 837--50.
doi: \href{https://doi.org/10.1093/bjps/axs039}{10.1093/bjps/axs039}.

\bibitem[\citeproctext]{ref-Price1989}
Price, Huw. 1989. {``Defending Desire-as-Belief.''} \emph{Mind} 98
(389): 119--27. doi:
\href{https://doi.org/10.1093/mind/XCVIII.389.119}{10.1093/mind/XCVIII.389.119}.

\end{CSLReferences}



\noindent Unpublished. Posted online in 2024.

\end{document}
