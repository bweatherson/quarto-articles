% Options for packages loaded elsewhere
% Options for packages loaded elsewhere
\PassOptionsToPackage{unicode}{hyperref}
\PassOptionsToPackage{hyphens}{url}
%
\documentclass[
  11pt,
  letterpaper,
  DIV=11,
  numbers=noendperiod,
  twoside]{scrartcl}
\usepackage{xcolor}
\usepackage[left=1.55in, right=1.55in, top=1.45in, bottom=1.45in,
includemp=TRUE, marginparwidth=0in, marginparsep=0in]{geometry}
\usepackage{amsmath,amssymb}
\setcounter{secnumdepth}{3}
\usepackage{iftex}
\ifPDFTeX
  \usepackage[T1]{fontenc}
  \usepackage[utf8]{inputenc}
  \usepackage{textcomp} % provide euro and other symbols
\else % if luatex or xetex
  \usepackage{unicode-math} % this also loads fontspec
  \defaultfontfeatures{Scale=MatchLowercase}
  \defaultfontfeatures[\rmfamily]{Ligatures=TeX,Scale=1}
\fi
\usepackage{lmodern}
\ifPDFTeX\else
  % xetex/luatex font selection
  \setmainfont[ItalicFont=EB Garamond Italic,BoldFont=EB Garamond
SemiBold]{EB Garamond Math}
  \setsansfont[]{EB Garamond}
  \setmathfont[]{Garamond-Math}
\fi
% Use upquote if available, for straight quotes in verbatim environments
\IfFileExists{upquote.sty}{\usepackage{upquote}}{}
\IfFileExists{microtype.sty}{% use microtype if available
  \usepackage[]{microtype}
  \UseMicrotypeSet[protrusion]{basicmath} % disable protrusion for tt fonts
}{}
\usepackage{setspace}
% Make \paragraph and \subparagraph free-standing
\makeatletter
\ifx\paragraph\undefined\else
  \let\oldparagraph\paragraph
  \renewcommand{\paragraph}{
    \@ifstar
      \xxxParagraphStar
      \xxxParagraphNoStar
  }
  \newcommand{\xxxParagraphStar}[1]{\oldparagraph*{#1}\mbox{}}
  \newcommand{\xxxParagraphNoStar}[1]{\oldparagraph{#1}\mbox{}}
\fi
\ifx\subparagraph\undefined\else
  \let\oldsubparagraph\subparagraph
  \renewcommand{\subparagraph}{
    \@ifstar
      \xxxSubParagraphStar
      \xxxSubParagraphNoStar
  }
  \newcommand{\xxxSubParagraphStar}[1]{\oldsubparagraph*{#1}\mbox{}}
  \newcommand{\xxxSubParagraphNoStar}[1]{\oldsubparagraph{#1}\mbox{}}
\fi
\makeatother


\usepackage{longtable,booktabs,array}
\usepackage{calc} % for calculating minipage widths
% Correct order of tables after \paragraph or \subparagraph
\usepackage{etoolbox}
\makeatletter
\patchcmd\longtable{\par}{\if@noskipsec\mbox{}\fi\par}{}{}
\makeatother
% Allow footnotes in longtable head/foot
\IfFileExists{footnotehyper.sty}{\usepackage{footnotehyper}}{\usepackage{footnote}}
\makesavenoteenv{longtable}
\usepackage{graphicx}
\makeatletter
\newsavebox\pandoc@box
\newcommand*\pandocbounded[1]{% scales image to fit in text height/width
  \sbox\pandoc@box{#1}%
  \Gscale@div\@tempa{\textheight}{\dimexpr\ht\pandoc@box+\dp\pandoc@box\relax}%
  \Gscale@div\@tempb{\linewidth}{\wd\pandoc@box}%
  \ifdim\@tempb\p@<\@tempa\p@\let\@tempa\@tempb\fi% select the smaller of both
  \ifdim\@tempa\p@<\p@\scalebox{\@tempa}{\usebox\pandoc@box}%
  \else\usebox{\pandoc@box}%
  \fi%
}
% Set default figure placement to htbp
\def\fps@figure{htbp}
\makeatother


% definitions for citeproc citations
\NewDocumentCommand\citeproctext{}{}
\NewDocumentCommand\citeproc{mm}{%
  \begingroup\def\citeproctext{#2}\cite{#1}\endgroup}
\makeatletter
 % allow citations to break across lines
 \let\@cite@ofmt\@firstofone
 % avoid brackets around text for \cite:
 \def\@biblabel#1{}
 \def\@cite#1#2{{#1\if@tempswa , #2\fi}}
\makeatother
\newlength{\cslhangindent}
\setlength{\cslhangindent}{1.5em}
\newlength{\csllabelwidth}
\setlength{\csllabelwidth}{3em}
\newenvironment{CSLReferences}[2] % #1 hanging-indent, #2 entry-spacing
 {\begin{list}{}{%
  \setlength{\itemindent}{0pt}
  \setlength{\leftmargin}{0pt}
  \setlength{\parsep}{0pt}
  % turn on hanging indent if param 1 is 1
  \ifodd #1
   \setlength{\leftmargin}{\cslhangindent}
   \setlength{\itemindent}{-1\cslhangindent}
  \fi
  % set entry spacing
  \setlength{\itemsep}{#2\baselineskip}}}
 {\end{list}}
\usepackage{calc}
\newcommand{\CSLBlock}[1]{\hfill\break\parbox[t]{\linewidth}{\strut\ignorespaces#1\strut}}
\newcommand{\CSLLeftMargin}[1]{\parbox[t]{\csllabelwidth}{\strut#1\strut}}
\newcommand{\CSLRightInline}[1]{\parbox[t]{\linewidth - \csllabelwidth}{\strut#1\strut}}
\newcommand{\CSLIndent}[1]{\hspace{\cslhangindent}#1}



\setlength{\emergencystretch}{3em} % prevent overfull lines

\providecommand{\tightlist}{%
  \setlength{\itemsep}{0pt}\setlength{\parskip}{0pt}}



 


\setlength\heavyrulewidth{0ex}
\setlength\lightrulewidth{0ex}
\usepackage[automark]{scrlayer-scrpage}
\clearpairofpagestyles
\cehead{
  Brian Weatherson
  }
\cohead{
  Anti-Anti-Desire-As-Belief
  }
\ohead{\bfseries \pagemark}
\cfoot{}
\makeatletter
\newcommand*\NoIndentAfterEnv[1]{%
  \AfterEndEnvironment{#1}{\par\@afterindentfalse\@afterheading}}
\makeatother
\NoIndentAfterEnv{itemize}
\NoIndentAfterEnv{enumerate}
\NoIndentAfterEnv{description}
\NoIndentAfterEnv{quote}
\NoIndentAfterEnv{equation}
\NoIndentAfterEnv{longtable}
\NoIndentAfterEnv{abstract}
\renewenvironment{abstract}
 {\vspace{-1.25cm}
 \quotation\small\noindent\emph{Abstract}:}
 {\endquotation}
\newfontfamily\tfont{EB Garamond}
\addtokomafont{disposition}{\rmfamily}
\addtokomafont{title}{\normalfont\itshape}
\let\footnoterule\relax

\makeatletter
\renewcommand{\@maketitle}{%
  \newpage
  \null
  \vskip 2em%
  \begin{center}%
  \let \footnote \thanks
    {\itshape\huge\@title \par}%
    \vskip 0.5em%  % Reduced from default
    {\large
      \lineskip 0.3em%  % Reduced from default 0.5em
      \begin{tabular}[t]{c}%
        \@author
      \end{tabular}\par}%
    \vskip 0.5em%  % Reduced from default
    {\large \@date}%
  \end{center}%
  \par
  }
\makeatother
\RequirePackage{lettrine}

\renewenvironment{abstract}
 {\quotation\small\noindent\emph{Abstract}:}
 {\endquotation\vspace{-0.02cm}}
\cehead{
      Anon
      }
\KOMAoption{captions}{tableheading}
\makeatletter
\@ifpackageloaded{caption}{}{\usepackage{caption}}
\AtBeginDocument{%
\ifdefined\contentsname
  \renewcommand*\contentsname{Table of contents}
\else
  \newcommand\contentsname{Table of contents}
\fi
\ifdefined\listfigurename
  \renewcommand*\listfigurename{List of Figures}
\else
  \newcommand\listfigurename{List of Figures}
\fi
\ifdefined\listtablename
  \renewcommand*\listtablename{List of Tables}
\else
  \newcommand\listtablename{List of Tables}
\fi
\ifdefined\figurename
  \renewcommand*\figurename{Figure}
\else
  \newcommand\figurename{Figure}
\fi
\ifdefined\tablename
  \renewcommand*\tablename{Table}
\else
  \newcommand\tablename{Table}
\fi
}
\@ifpackageloaded{float}{}{\usepackage{float}}
\floatstyle{ruled}
\@ifundefined{c@chapter}{\newfloat{codelisting}{h}{lop}}{\newfloat{codelisting}{h}{lop}[chapter]}
\floatname{codelisting}{Listing}
\newcommand*\listoflistings{\listof{codelisting}{List of Listings}}
\makeatother
\makeatletter
\makeatother
\makeatletter
\@ifpackageloaded{caption}{}{\usepackage{caption}}
\@ifpackageloaded{subcaption}{}{\usepackage{subcaption}}
\makeatother
\usepackage{bookmark}
\IfFileExists{xurl.sty}{\usepackage{xurl}}{} % add URL line breaks if available
\urlstyle{same}
\hypersetup{
  pdftitle={Anti-Anti-Desire-As-Belief},
  pdfauthor={Brian Weatherson},
  hidelinks,
  pdfcreator={LaTeX via pandoc}}


\title{Anti-Anti-Desire-As-Belief}
\author{Brian Weatherson}
\date{2025}
\begin{document}
\maketitle
\begin{abstract}
David Lewis put forward a decision theoretic argument against there
being a tight connection between desires and beliefs about the good. I
reply on behalf of the targets of his arguments. I first present a
puzzle that anyone who believes in a connection between desire and
belief has to answer. I then argue that there are two plausible answers
to this puzzle, and whichever answer one gives, one has a prior reason
to reject a premise in Lewis's argument.
\end{abstract}


\setstretch{1.1}
\section{Lewis's Argument}\label{sec-lewis}

David Lewis (\citeproc{ref-Lewis1988}{1988},
\citeproc{ref-Lewis1996}{1996}) has an argument against the view that
desires can be reduced to beliefs. I'm going to respond on behalf of the
Desire-as-Belief (hereafter, DAB) thesis. In particular, I'm going to
offer two responses. The responses themselves will not be particularly
original; one is due to Huw Price (\citeproc{ref-Price1989}{1989}), and
the other to Jessica Collins (\citeproc{ref-Collins2015}{2015}). What
will I hope be original is showing how they fit together. I'll argue
that given the other assumptions Lewis makes, the DAB theorist has
independent reason to reject one or other of the premises Lewis offers.
The reason is that there is a tricky puzzle that defenders of DAB must
have an answer to, and there are two viable answers, one of which leads
to (and motivates) Price's response to Lewis, the other of which leads
to (and motivates) Collins's.

I'll start with a presentation of Lewis's argument, shorn of what seem
to me to be extraneous details. This will look a little different to
Lewis's own presentation, but all the premises here are ones he uses,
and its gets to the conclusion he wants, so it seems to be a fair
representation.\footnote{I'm drawing here on presentations of the
  argument by Collins (\citeproc{ref-Collins2015}{2015}), Hàjek
  (\citeproc{ref-Hajek2015}{2015}), and Nissan-Rozen
  (\citeproc{ref-Nissan-Rozen2015-NISATR}{2015}).}

Assume that we have a finite set of worlds. We will use \emph{w} as a
variable over worlds. A world, in this sense, is a specification of the
truth value of all the truth-apt things that are relevant to a
particular decision. They are `small worlds' in the sense of Savage
(\citeproc{ref-Savage1954}{1954}), not possible worlds in the sense of
Lewis (\citeproc{ref-Lewis1986a}{1986}). For the most part they are more
coarse-grained than the concrete worlds of \emph{Plurality}, but in one
respect they are more fine-grained: they might differ from each other
purely in normative features.\footnote{This is a consequence of the
  point Chalmers (\citeproc{ref-Chalmers2011b}{2011}) makes that it is
  epistemic, not metaphysical, possibility that matters here, plus the
  assumption that our characters might be normatively uncertain.}

For any descriptive proposition A, assume there is a distinct
proposition Å, meaning that A is desirable.\footnote{It's more common
  here to let Å be that A is good, but I want to separate questions
  about the desire-belief relationship from questions about moral
  motivation, so my characters will be motivated to do something just to
  the extent they believe it desirable, leaving it as a further question
  whether all and only good things are desirable.} Let V be an agent's
value function, and Pr their credence function, with subscripts
representing what those functions are like after updating. So
V\textsubscript{A} and Pr\textsubscript{A} are the values of the value
and credence functions after updating on A. Strictly speaking given how
I've set this up, propositions not worlds have values. But it is
convenient to talk about the value of a world, so I'll sometimes write
V(\emph{w}) when strictly it should be V(\{\emph{w}\}); I don't think
this can lead to any confusion. (Later I'll also write Pr(\emph{w}) for
the probability of Pr(\{\emph{w}\}); again it shouldn't result in
confusion.)

Lewis's argument against DAB uses six assumptions. In these assumptions
B is an arbitrary proposition, and A is an arbitrary \emph{descriptive}
proposition.

\begin{description}
\tightlist
\item[Binary Desirability]
All worlds are either desirable or undesirable. If \emph{w} is
desirable, then V(\emph{w})~=~1, and otherwise V(\emph{w})~=~0.
\item[Equation]
Given \textbf{Binary Desirability}, the correct way to represent DAB is
V(A)~=~Pr(Å).
\item[Additivity]
V(A)~=~Σ\textsubscript{\emph{w}}V(\emph{w})Pr(\emph{w}~\textbar~A)
\item[Worldly Invariance]
V\textsubscript{A}(\emph{w})~=~V(\emph{w})
\item[Restricted Conditionalisation]
Pr\textsubscript{A}(B)~=~Pr(B~\textbar~A)
\item[Possible Independence]
For some Å, Pr(Å) ≠ Pr(Å \textbar{} A)
\end{description}

The first assumption is obviously absurd, but it is useful for setting
out the argument. In any case, if the last five assumptions are true,
then they should be consistent with \textbf{Binary Desirability}. Given
those assumptions, here is Lewis's argument. By \textbf{Possible
Independence} there is an Å such that Pr(Å) ≠ Pr(Å \textbar{} A). But
given the other assumptions we can reason as follows.

\begin{longtable}[]{@{}
  >{\raggedleft\arraybackslash}p{(\linewidth - 4\tabcolsep) * \real{0.1071}}
  >{\raggedright\arraybackslash}p{(\linewidth - 4\tabcolsep) * \real{0.3929}}
  >{\raggedright\arraybackslash}p{(\linewidth - 4\tabcolsep) * \real{0.5000}}@{}}
\toprule\noalign{}
\endhead
\bottomrule\noalign{}
\endlastfoot
Pr(Å) & ~=~V(A) & (\textbf{Binary Desirability} + \textbf{Equation}) \\
& ~=~Σ\textsubscript{\emph{w}}V(\emph{w}) Pr(\emph{w}~\textbar~A) &
(\textbf{Additivity}) \\
& ~=~Σ\textsubscript{\emph{w}}V\textsubscript{A}(\emph{w})
Pr(\emph{w}~\textbar~A) & (\textbf{Worldly Invariance}) \\
& ~=~Σ\textsubscript{\emph{w}}V\textsubscript{A}(\emph{w})
Pr\textsubscript{A}(\emph{w} \textbar~A) & (\textbf{Restricted
Conditionalisation}) \\
& ~=~V\textsubscript{A}(A) & (\textbf{Additivity}, applied to updated
values) \\
& ~=~Pr\textsubscript{A}(Å) & (\textbf{Equation}, again after
updating) \\
& ~=~Pr(Å~\textbar~A) & (\textbf{Restriced Conditionalisation}) \\
\end{longtable}

And the last line contradicts our assumption. So not all six of these
assumptions can be correct, if DAB is true. Since Lewis thinks they are
all correct, he concludes DAB is false.

As I said, this presentation doesn't look a lot like Lewis's argument.
Most notably, \textbf{Possible Independence} isn't an assumption for
Lewis, it is something he derives from yet further premises. Much of the
1988 paper is devoted to spelling out the absurd consequences of denying
\textbf{Possible Independence}. These arguments haven't convinced
everyone. They assume that conditionalisation is the right way to update
on any new information, descriptive or normative. If normative
propositions are centered worlds, as the picture in Lewis
(\citeproc{ref-Lewis1989b}{1989}) suggests, that seems like the wrong
way to update. If the picture of self-locating belief that Lewis
(\citeproc{ref-Lewis1979b}{1979}) offers is correct, we can't update our
beliefs about what time it is by conditionalisation on the noise the
alarm clock makes.\footnote{The point that conditionalisation isn't the
  right way to update beliefs with centered worlds contents, and this
  raises a problem for Lewis, is made by Graham Oddie
  (\citeproc{ref-Oddie1994}{1994}).}

But \textbf{Possible Independence} is surely true. Assume that A is a
proposition about someone, call him Peter, might do. And assume that we
desire that the morally good thing is done, that we don't know whether A
is good or not, but we are very confidence in Peter's moral judgment. If
Peter does A, that's good evidence A is good. That all seems coherent,
which is enough to support \textbf{Possible Independence}.

While unrestricted conditionalisation is questionable,
\textbf{Restricted Conditionalisation} seems fairly secure. In some
presentations of the argument, Lewis uses a version of invariance that
says the value of any proposition does not change on learning A. This is
questionable, but \textbf{Worldly Invariance} seems fairly secure. It's
just the view that in a decision tree, the value of a terminal node
doesn't depend on where we are in the tree. That's normally taken for
granted in formal models of dynamic choice, and I think rightly so.

If we treat \textbf{Binary Desirability} as a harmless simplification,
that means the only substantive assumptions left are \textbf{Equation}
and \textbf{Additivity}. Given those, the argument against DAB goes
through. I'm going to argue that anyone sympathetic to DAB has
independent reason to reject one or other of those claims. Part of the
argument that this is an \emph{independent} reason is that different
sympathisers will reject one rather than the other. To show this, I'll
start with a short story.

\section{Auntie and Auntie}\label{sec-auntie}

Our heroes are two anti-Humeans, called Auntie E and Auntie C, who both
endorse a version of DAB. Both of them are aunts of Peter, the moral
exemplar from Section~\ref{sec-lewis}. Both Aunties E and Auntie C think
that if Peter does something, it's very likely to be the right thing to
do. Indeed, they are fairly deferential to Peter in this respect; if
Peter does something they thought was wrong, they take that as some
(strong but inconclusive) reason to change their belief about the
morality of the action. They are both moralists, and think something is
desirable iff it is good.

Let A be the Proposition that Peter does some action \emph{a}, and Å
that it is deisrable/good. Like Lewis does with \textbf{Binary
Desirability}, I'll make a simplifying assumption: it's common knowledge
that not doing \emph{a} is good iff doing \emph{a} is not good. That
means  can be read as either of the epistemically equivalent
propositions \emph{a is not good} and \emph{not a is good}.

Before Peter acts, both Auntie's have the same credal distribution,
satisfying these constraints.

\begin{itemize}
\tightlist
\item
  C(Å~\textbar~A) = 0.8
\item
  C(¬Å \textbar{} ¬A) = 0.9
\item
  C(A) = 0.7
\end{itemize}

Table~\ref{tbl-credence} shows the credence each Auntie has in each of
the four possibilities from crossing A with Å.

\begin{longtable}[]{@{}rcc@{}}
\caption{Auntie's credence that Peter will do A, and that it will be
right.}\label{tbl-credence}\tabularnewline
\toprule\noalign{}
& Å & ¬Å \\
\midrule\noalign{}
\endfirsthead
\toprule\noalign{}
& Å & ¬Å \\
\midrule\noalign{}
\endhead
\bottomrule\noalign{}
\endlastfoot
A & 0.56 & 0.14 \\
¬A & 0.03 & 0.27 \\
\end{longtable}

Now you might think at this point that I've said enough to tell you what
each Auntie hopes Peter will do. After all, I've told you everything
relevant about each Auntie's credence in Å, and I've told you that their
credences in propositions about goodness determine their values. But I
haven't told you one thing extra - I haven't told you what decision
theory the two Aunties follow.

Auntie E is an evidential decision theorist. For her, the value of an
arbitrary action \emph{x} is given by \textbf{Auntie E's Value}. In this
formula, where X is the proposition that \emph{x} is performed, and D is
the propositions that things are desirable.

\begin{description}
\tightlist
\item[Auntie E's Value]
V(\emph{x}) = C(D~\textbar~X)
\end{description}

That is, she looks at Peter's options, and hopes that he does the one
that she is most confident is good, conditional on Peter doing it. That
means she hopes Peter does not do \emph{a}, since then she'll have
credence 0.9 that Peter has done the right thing. If Peter does
\emph{a}, she'll only have credence 0.8 that he'll have done the right
thing, which isn't as good.

Auntie C endorses a version of causal decision theory, in particular
something like the version supported by David Lewis
(\citeproc{ref-Lewis1981bn}{1981}).\footnote{Here I'm following Collins,
  who notes that it is odd that Lewis attributes to Auntie a form of
  evidential decision theory, which Lewis himself does not endorse.} In
particular, Auntie's values are given by \textbf{Auntie C's Value}. In
the formula, C\textsubscript{\emph{x}} be the result of \emph{imaging}
the credence function C on the proposition \emph{x} is performed. Auntie
C believes changing the moral facts is a bigger change to the world than
changing any descriptive facts, so imaging always moves credences up or
down in Table~\ref{tbl-credence}, never left or right.\footnote{Recall
  here that the worlds are epistemic possibilities, not metaphysical
  ones, so it makes sense to talk about merely changing the moral facts.}

\begin{description}
\tightlist
\item[Auntie C's Value]
V(\emph{x}) = C\textsubscript{\emph{x}}(D)
\end{description}

This resembles equation (11) in ``Causal Decision Theory''. Indeed,
after the first character, it just is the special case of that equation
where the only possible values are 1 and 0. But the first character
matters. Lewis is presenting a theory of usefulness, not of value. His
formula is meant to measure the thing that a rational actor maximises.
It is not measuring the thing an altruistic friend hopes is maximised.
We'll come back to this point in Section~\ref{sec-auntie-c}. For now, I
just want to note the similarities to Lewis's own theory.

Using this formula, Auntie C hopes that Peter does a iff C(Å)
\textgreater{} C(¬Å). Since C(Å)~=~0.59, and C(¬Å)~=~0.41, that means
she does hope that Peter does A.

The next two steps are to see why the Aunties reject Lewis's argument.

\section{DAB and EDT}\label{sec-auntie-e}

The easier case is Auntie E. She rejects \textbf{Equation}. The
relationship between desire and belief is not V(A)~=~Pr(Å), but
V(A)~=~Pr(Å \textbar{} A). Lewis is aware of this response, it's
developed by Price (\citeproc{ref-Price1989}{1989}), and his response is
that this isn't a form of desire as \emph{belief}. His thought, and this
comes out a little more clearly in a recently published letter than in
the papers (\citeproc{ref-Lewis-McDermott-06121993}{Lewis {[}1993{]}
2020}), is that belief isn't load-bearing in Auntie E's view.

Everyone agrees that beliefs play a role in instrumental desires. If
desires to take a pill because one believes it will cure one's disease,
that desire will go away if one loses either the belief that it's a
cure, or the belief that one is diseased. Lewis notes that Auntie E is
committed to the existence of a proposition D consisting of all and only
the desirable worlds, and to the claim that by necessity any agent with
desires will desire it. Moreover, he argues, on Auntie E's view this is
the only non-instrumental desire an agent has. Beliefs don't affect
non-instrumental desires on this view, since everyone has the same
non-instrumental desires. They just affect instrumental desires. But
Humeans and anti-Humeans agree about the connection between belief and
non-instrumental desires.

I'll offer three replies on Auntie E's behalf. I'm not defending her
view in general; I'm not an evidential decision theorist. I'm just
defending the claim that this is a kind of desire as beleif.

First, the simplifying assumption \textbf{Binary Desirability} looks
less benign here. The construction of the necessarily desired
proposition D requires this assumption. If some worlds are more
desirable than others there are still preferences that Auntie E thinks
are necessary (i.e., that a more desirable world is preferred to a less
desirable one), but no desires.

Second, the necessity claim Auntie E seems committed to looks less
worrying once remember what kinds of things worlds are in this context.
They are the elements of the contents of attitudes. That is, on Lewis's
view, they are centered worlds. Auntie E's view is really that by
necessity an agent desires that one of the centered worlds found
desirable by the current center, i.e., that very agent, is actualised.
That is, agents desire what they believe desirable. That's not a
surprising necessity claim. It's arguably just a kind of verbal
necessity; Auntie E could simply say that a belief doesn't count as a
belief about \emph{desirabilty} unless it is matched by a desire. Lewis
is committed to the view that anyone who believes \emph{p and q}
believes \emph{p}; this isn't a worrying necessity because a belief
wouldn't count as a belief in a conjunction unless it had this property.
Auntie E is making a version of the same move.

Third, the objection that this isn't really the right kind of reduction
is a very odd complaint from \emph{David Lewis}. It's agreed on all
sides that, on Auntie E's view, what an agent desires supervenes on what
they believe. Moreover, the reverse supervenience, of belief on desire,
in general doesn't hold. Lewis's complaint here is that Auntie E hasn't
made desire depend on belief in the right way, even though she has
ensured these supervenience claims hold. The only way to make this
complaint work is to understand dependence using some hyperintensional
notion like grounding. But there's no sign of such a notion in Lewis's
work, and there are good reasons to think it couldn't be added to his
view (\citeproc{ref-MacBrideJL2022}{MacBride and Janssen-Lauret 2022}).

So EDT offers a way out of Lewis's argument. That's not totally
surprising; it's not Lewis's view. Things are trickier with Auntie C.

\section{DAB and CDT}\label{sec-auntie-c}

Auntie C says that the misstep in Lewis's argument is
\textbf{Additivity}. Following Collins, she says that this is something
that only an adherent of EDT should accept. She's mostly mystified about
why Lewis, famously an enemy of EDT, would have accepted it in the first
place.

In the second DAB paper, Lewis has a response to this. Oddly, it's in a
parenthetical paragraph. After describing an action that's essentially
two-boxing in Newcomb's problem, he writes

\begin{quote}
Should you take that actions?---Yes \ldots{} Do you desire to perform
it?---No \ldots{} {[}O{]}ur topic here is not choiceworthiness but
desire \ldots{} {[}so{]} we adopt an ``evidential'' conception of
expected value \ldots{} Choiceworthiness is governed by a different
``causal'' conception. (\citeproc{ref-Lewis1996}{Lewis 1996, 303})
\end{quote}

So Lewis thinks that Auntie C is confusing the two notions, and that she
only rejects \textbf{Aditivity} because of this confusion. But there's
something important that can be said on Auntie C's behalf.

In the first DAB paper, Lewis sets out the notion of desire that he
thinks is at issue. There, he is most concerned to distinguish it from a
notion that is intuitively used in actions taken from duty. We might
intuitively say that we did X from duty, though we desired to do Y.
Lewis resists; he thinks that in any such case we also desire X. As he
puts it,

\begin{quote}
We are within our rights to construe `desire' inclusively, to cover the
entire range of states that move us. (\citeproc{ref-Lewis1988}{Lewis
1988, 323})
\end{quote}

This isn't an optional move on Lewis's part. If he doesn't make this
move, the Humean picture of action he wants to defend fails immediately.

Now Auntie C says that if we're modelling ``the entire range of states
that move us'', and we are, as Lewis recommends, moved to take two boxes
in Newcomb's Problem, we really better not insist that our model
satisfies \textbf{Additivity}. We'd be left saying that the two-boxer
acts against all desire, merely motivated by the duty to do what's
rational. This would be a violation of Lewis's Humeanism. It would also
be phenomenologically implausible. The two-boxer isn't moved by duty,
but by the desire for the extra \$1000. So for both theoretical and
phenomenological reasons, we need a notion of desire that violates
\textbf{Additivity}.

Indeed, she insists that it isn't just in first-personal cases like
Newcomb's Problem that \textbf{Additivity} fails. Imagine that Peter
faces the choice in Table~\ref{tbl-peter-second}, where the values in
the box now represent the moral value of the choice, and Auntie C thinks
Peter choosing Up is excellent evidence for Left, while his choosing
Down is excellent evidence for Right.

\begin{longtable}[]{@{}rcc@{}}
\caption{Peter's second choice}\label{tbl-peter-second}\tabularnewline
\toprule\noalign{}
& Left & Right \\
\midrule\noalign{}
\endfirsthead
\toprule\noalign{}
& Left & Right \\
\midrule\noalign{}
\endhead
\bottomrule\noalign{}
\endlastfoot
Up & 3 & 0 \\
Down & 5 & 1 \\
\end{longtable}

The first quote of Lewis's suggests that both Auntie's should desire
that Peter choose Up, since that will be evidence that an outcome of
value 3 will obtain. But it seems coherent to hope that he chooses Down.
Several of the arguments for two-boxing seem replicable here. It's weird
to hope that Peter chooses Down conditional on Left, and hope he chooses
Down conditional on Right, and unconditionally hope that he chooses Up.

This is not to deny that there is a theoretical role for something like
news-value. Joyce (\citeproc{ref-Joyce1999}{1999}) makes extensive use
of that notion in developing a causal decision theory. It's just that in
the context of defending Humeanism about action, two-boxers like Lewis
can't equate news-value with desirability. So \textbf{Addition} has to
fail, since as Lewis agrees, \textbf{Addition} entails that desirability
goes with news value.

\section{Conclusion}\label{sec-conclusion}

Any defender of DAB has to pick whether to take Auntie E's side or
Auntie C's side in the original question about Peter. Whichever side
they pick, they will have an independent reason to reject one of the
premises Lewis puts forward in his argument against their view. So they
have independent reason to reject Lewis's argument.

I've been stressing \emph{independent} here because it's not news that
any defender of DAB has to reject some premise of Lewis's argument. That
follows from the fact that his argument is valid! For the response to be
more than table-thumping, the proponent of DAB has to say which premise
they reject, and why it is reasonable to reject it. The point of the
examples involving Peter is to identify the premise in question, which
will differ between different proponents, and say what that reason is.
In the second DAB paper, Lewis offers responses to each of these
reasons, and in the last two sections I've gone over why those responses
don't work. It's somewhat ironic that in each case my analysis has
rested in part on the view that Lewis's response is inconsistent with
what he says elsewhere: about the role of possible worlds in philosophy
in Section~\ref{sec-auntie-e}, and about the role of desire in producing
action in Section~\ref{sec-auntie-c}. I'll leave to another day whether
someone who was willing to abandon large parts of the Lewisian framework
might be able to rescue his argument against DAB.

\subsection*{References}\label{references}
\addcontentsline{toc}{subsection}{References}

\phantomsection\label{refs}
\begin{CSLReferences}{1}{0}
\bibitem[\citeproctext]{ref-Chalmers2011b}
Chalmers, David. 2011. {``Frege's Puzzle and the Objects of Credence.''}
\emph{Mind} 120 (479): 587--635. doi:
\href{https://doi.org/10.1093/mind/fzr046}{10.1093/mind/fzr046}.

\bibitem[\citeproctext]{ref-Collins2015}
Collins, Jessica. 2015. {``Decision Theory After Lewis.''} In \emph{A
Companion to David Lewis}, edited by Barry Loewer and Jonathan Schaffer,
446--58. John Wiley \& Sons.

\bibitem[\citeproctext]{ref-Hajek2015}
Hàjek, Alan. 2015. {``On the Plurality of Lewis's Triviality Results.''}
In \emph{A Companion to David Lewis}, edited by Barry Loewer and
Jonathan Schaffer, 425--45. John Wiley \& Sons.

\bibitem[\citeproctext]{ref-Joyce1999}
Joyce, James M. 1999. \emph{The Foundations of Causal Decision Theory}.
Cambridge: Cambridge University Press.

\bibitem[\citeproctext]{ref-Lewis1979b}
Lewis, David. 1979. {``Attitudes \emph{de Dicto} and \emph{de Se}.''}
\emph{Philosophical Review} 88 (4): 513--43. doi:
\href{https://doi.org/10.2307/2184646}{10.2307/2184646}. Reprinted in
his \emph{Philosophical Papers}, Volume 1, Oxford: Oxford University
Press, 1983, 133-156. References to reprint.

\bibitem[\citeproctext]{ref-Lewis1981bn}
---------. 1981. {``Causal Decision Theory.''} \emph{Australasian
Journal of Philosophy} 59 (1): 5--30. doi:
\href{https://doi.org/10.1080/00048408112340011}{10.1080/00048408112340011}.

\bibitem[\citeproctext]{ref-Lewis1986a}
---------. 1986. \emph{On the Plurality of Worlds}. Oxford: Blackwell
Publishers.

\bibitem[\citeproctext]{ref-Lewis1988}
---------. 1988. {``Desire as Belief.''} \emph{Mind} 97 (387): 323--32.
doi:
\href{https://doi.org/10.1093/mind/xcvii.387.323}{10.1093/mind/xcvii.387.323}.

\bibitem[\citeproctext]{ref-Lewis1989b}
---------. 1989. {``Dispositional Theories of Value.''}
\emph{Aristotelian Society Supplementary Volume} 63 (1): 113--37. doi:
\href{https://doi.org/10.1093/aristoteliansupp/63.1.89}{10.1093/aristoteliansupp/63.1.89}.
Reprinted in his \emph{Papers in Ethics and Social Philosophy},
Cambridge: Cambridge University Press, 2000, 68-94. References to
reprint.

\bibitem[\citeproctext]{ref-Lewis1996}
---------. 1996. {``Desire as Belief {II}.''} \emph{Mind} 105 (418):
303--13. doi:
\href{https://doi.org/10.1093/mind/105.418.303}{10.1093/mind/105.418.303}.

\bibitem[\citeproctext]{ref-Lewis-McDermott-06121993}
---------. (1993) 2020. {``Letter to Michael McDermott, 6 December
1993.''} In \emph{Philosophical Letters of David {K}. Lewis}, edited by
Helen Beebee and A. R. J. Fisher, 2:508. Oxford: Oxford University
Press.

\bibitem[\citeproctext]{ref-MacBrideJL2022}
MacBride, Fraser, and Frederique Janssen-Lauret. 2022. {``Why Lewis
Would Have Rejected Grounding.''} In \emph{Perspectives on the
Philosophy of {D}avid {K}. {L}ewis}, edited by Helen Beebee and A. R. J.
Fisher, 66--91. {O}xford {U}niversity {P}ress. doi:
\href{https://doi.org/10.1093/oso/9780192845443.003.0005}{10.1093/oso/9780192845443.003.0005}.

\bibitem[\citeproctext]{ref-Nissan-Rozen2015-NISATR}
Nissan-Rozen, Ittay. 2015. {``A Triviality Result for the ``Desire by
Necessity''thesis.''} \emph{Synthese} 192 (8): 2535--56.

\bibitem[\citeproctext]{ref-Oddie1994}
Oddie, Graham. 1994. {``Harmony, Purity, Truth.''} \emph{Mind} 103
(412): 451--72. doi:
\href{https://doi.org/10.1093/mind/103.412.451}{10.1093/mind/103.412.451}.

\bibitem[\citeproctext]{ref-Price1989}
Price, Huw. 1989. {``Defending Desire-as-Belief.''} \emph{Mind} 98
(389): 119--27. doi:
\href{https://doi.org/10.1093/mind/XCVIII.389.119}{10.1093/mind/XCVIII.389.119}.

\bibitem[\citeproctext]{ref-Savage1954}
Savage, Leonard. 1954. \emph{The Foundations of Statistics}. New York:
John Wiley.

\end{CSLReferences}



\noindent Draft for submission.


\end{document}
