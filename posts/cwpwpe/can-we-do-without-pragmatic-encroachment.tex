% Options for packages loaded elsewhere
\PassOptionsToPackage{unicode}{hyperref}
\PassOptionsToPackage{hyphens}{url}
%
\documentclass[
  10pt,
  letterpaper,
  DIV=11,
  numbers=noendperiod,
  twoside]{scrartcl}

\usepackage{amsmath,amssymb}
\usepackage{setspace}
\usepackage{iftex}
\ifPDFTeX
  \usepackage[T1]{fontenc}
  \usepackage[utf8]{inputenc}
  \usepackage{textcomp} % provide euro and other symbols
\else % if luatex or xetex
  \usepackage{unicode-math}
  \defaultfontfeatures{Scale=MatchLowercase}
  \defaultfontfeatures[\rmfamily]{Ligatures=TeX,Scale=1}
\fi
\usepackage{lmodern}
\ifPDFTeX\else  
    % xetex/luatex font selection
  \setmainfont[ItalicFont=EB Garamond Italic,BoldFont=EB Garamond
Bold]{EB Garamond Math}
  \setsansfont[]{Europa-Bold}
  \setmathfont[]{Garamond-Math}
\fi
% Use upquote if available, for straight quotes in verbatim environments
\IfFileExists{upquote.sty}{\usepackage{upquote}}{}
\IfFileExists{microtype.sty}{% use microtype if available
  \usepackage[]{microtype}
  \UseMicrotypeSet[protrusion]{basicmath} % disable protrusion for tt fonts
}{}
\usepackage{xcolor}
\usepackage[left=1in, right=1in, top=0.8in, bottom=0.8in,
paperheight=9.5in, paperwidth=6.5in, includemp=TRUE, marginparwidth=0in,
marginparsep=0in]{geometry}
\setlength{\emergencystretch}{3em} % prevent overfull lines
\setcounter{secnumdepth}{3}
% Make \paragraph and \subparagraph free-standing
\ifx\paragraph\undefined\else
  \let\oldparagraph\paragraph
  \renewcommand{\paragraph}[1]{\oldparagraph{#1}\mbox{}}
\fi
\ifx\subparagraph\undefined\else
  \let\oldsubparagraph\subparagraph
  \renewcommand{\subparagraph}[1]{\oldsubparagraph{#1}\mbox{}}
\fi


\providecommand{\tightlist}{%
  \setlength{\itemsep}{0pt}\setlength{\parskip}{0pt}}\usepackage{longtable,booktabs,array}
\usepackage{calc} % for calculating minipage widths
% Correct order of tables after \paragraph or \subparagraph
\usepackage{etoolbox}
\makeatletter
\patchcmd\longtable{\par}{\if@noskipsec\mbox{}\fi\par}{}{}
\makeatother
% Allow footnotes in longtable head/foot
\IfFileExists{footnotehyper.sty}{\usepackage{footnotehyper}}{\usepackage{footnote}}
\makesavenoteenv{longtable}
\usepackage{graphicx}
\makeatletter
\def\maxwidth{\ifdim\Gin@nat@width>\linewidth\linewidth\else\Gin@nat@width\fi}
\def\maxheight{\ifdim\Gin@nat@height>\textheight\textheight\else\Gin@nat@height\fi}
\makeatother
% Scale images if necessary, so that they will not overflow the page
% margins by default, and it is still possible to overwrite the defaults
% using explicit options in \includegraphics[width, height, ...]{}
\setkeys{Gin}{width=\maxwidth,height=\maxheight,keepaspectratio}
% Set default figure placement to htbp
\makeatletter
\def\fps@figure{htbp}
\makeatother
% definitions for citeproc citations
\NewDocumentCommand\citeproctext{}{}
\NewDocumentCommand\citeproc{mm}{%
  \begingroup\def\citeproctext{#2}\cite{#1}\endgroup}
\makeatletter
 % allow citations to break across lines
 \let\@cite@ofmt\@firstofone
 % avoid brackets around text for \cite:
 \def\@biblabel#1{}
 \def\@cite#1#2{{#1\if@tempswa , #2\fi}}
\makeatother
\newlength{\cslhangindent}
\setlength{\cslhangindent}{1.5em}
\newlength{\csllabelwidth}
\setlength{\csllabelwidth}{3em}
\newenvironment{CSLReferences}[2] % #1 hanging-indent, #2 entry-spacing
 {\begin{list}{}{%
  \setlength{\itemindent}{0pt}
  \setlength{\leftmargin}{0pt}
  \setlength{\parsep}{0pt}
  % turn on hanging indent if param 1 is 1
  \ifodd #1
   \setlength{\leftmargin}{\cslhangindent}
   \setlength{\itemindent}{-1\cslhangindent}
  \fi
  % set entry spacing
  \setlength{\itemsep}{#2\baselineskip}}}
 {\end{list}}
\usepackage{calc}
\newcommand{\CSLBlock}[1]{\hfill\break\parbox[t]{\linewidth}{\strut\ignorespaces#1\strut}}
\newcommand{\CSLLeftMargin}[1]{\parbox[t]{\csllabelwidth}{\strut#1\strut}}
\newcommand{\CSLRightInline}[1]{\parbox[t]{\linewidth - \csllabelwidth}{\strut#1\strut}}
\newcommand{\CSLIndent}[1]{\hspace{\cslhangindent}#1}

\setlength\heavyrulewidth{0ex}
\setlength\lightrulewidth{0ex}
\usepackage[automark]{scrlayer-scrpage}
\clearpairofpagestyles
\cehead{
  Brian Weatherson
  }
\cohead{
  Can We Do Without Pragmatic Encroachment?
  }
\ohead{\bfseries \pagemark}
\cfoot{}
\makeatletter
\newcommand*\NoIndentAfterEnv[1]{%
  \AfterEndEnvironment{#1}{\par\@afterindentfalse\@afterheading}}
\makeatother
\NoIndentAfterEnv{itemize}
\NoIndentAfterEnv{enumerate}
\NoIndentAfterEnv{description}
\NoIndentAfterEnv{quote}
\NoIndentAfterEnv{equation}
\NoIndentAfterEnv{longtable}
\NoIndentAfterEnv{abstract}
\renewenvironment{abstract}
 {\vspace{-1.25cm}
 \quotation\small\noindent\rule{\linewidth}{.5pt}\par\smallskip
 \noindent }
 {\par\noindent\rule{\linewidth}{.5pt}\endquotation}
\KOMAoption{captions}{tableheading}
\makeatletter
\@ifpackageloaded{caption}{}{\usepackage{caption}}
\AtBeginDocument{%
\ifdefined\contentsname
  \renewcommand*\contentsname{Table of contents}
\else
  \newcommand\contentsname{Table of contents}
\fi
\ifdefined\listfigurename
  \renewcommand*\listfigurename{List of Figures}
\else
  \newcommand\listfigurename{List of Figures}
\fi
\ifdefined\listtablename
  \renewcommand*\listtablename{List of Tables}
\else
  \newcommand\listtablename{List of Tables}
\fi
\ifdefined\figurename
  \renewcommand*\figurename{Figure}
\else
  \newcommand\figurename{Figure}
\fi
\ifdefined\tablename
  \renewcommand*\tablename{Table}
\else
  \newcommand\tablename{Table}
\fi
}
\@ifpackageloaded{float}{}{\usepackage{float}}
\floatstyle{ruled}
\@ifundefined{c@chapter}{\newfloat{codelisting}{h}{lop}}{\newfloat{codelisting}{h}{lop}[chapter]}
\floatname{codelisting}{Listing}
\newcommand*\listoflistings{\listof{codelisting}{List of Listings}}
\makeatother
\makeatletter
\makeatother
\makeatletter
\@ifpackageloaded{caption}{}{\usepackage{caption}}
\@ifpackageloaded{subcaption}{}{\usepackage{subcaption}}
\makeatother
\ifLuaTeX
  \usepackage{selnolig}  % disable illegal ligatures
\fi
\IfFileExists{bookmark.sty}{\usepackage{bookmark}}{\usepackage{hyperref}}
\IfFileExists{xurl.sty}{\usepackage{xurl}}{} % add URL line breaks if available
\urlstyle{same} % disable monospaced font for URLs
\hypersetup{
  pdftitle={Can We Do Without Pragmatic Encroachment?},
  pdfauthor={Brian Weatherson},
  hidelinks,
  pdfcreator={LaTeX via pandoc}}

\title{Can We Do Without Pragmatic Encroachment?}
\author{Brian Weatherson}
\date{2005}

\begin{document}
\maketitle
\begin{abstract}
I argue that interests primarily affect the relationship between
credence and belief. A view is set out and defended where evidence and
rational credence are not interest-relative, but belief, rational
belief, and knowledge are.
\end{abstract}

\setstretch{1.1}
\section{Introduction}\label{introduction}

Recently several authors have defended claims suggesting that there is a
closer connection between practical interests and epistemic
justification than has traditionally been countenanced. Jeremy Fantl and
Matthew McGrath (\citeproc{ref-Fantl2002}{2002}) argue that there is a
``pragmatic necessary condition on epistemic justification'' (77),
namely the following.

\begin{description}
\item[(PC)]
\emph{S} is justified in believing that \emph{p} only if \emph{S} is
rational to prefer as if \emph{p}. (77)
\end{description}

And John Hawthorne (\citeproc{ref-Hawthorne2004}{2004}) and Jason
Stanley (\citeproc{ref-Stanley2005-STAKAP}{2005}) have argued that what
it takes to turn true belief into knowledge is sensitive to the
practical environment the subject is in. These authors seem to be
suggesting there is, to use Jonathan Kvanvig's phrase ``pragmatic
encroachment'' in epistemology. In this paper I'll argue that their
arguments do not quite show this is true, and that concepts of
epistemological justification need not be pragmatically sensitive. The
aim here isn't to show that (PC) is false, but rather that it shouldn't
be described as a pragmatic condition on \emph{justification}. Rather,
it is best thought of as a pragmatic condition on \emph{belief}. There
are two ways to spell out the view I'm taking here. These are both
massive simplifications, but they are close enough to the truth to show
the kind of picture I'm aiming for.

First, imagine a philosopher who holds a very simplified version of
functionalism about belief, call it (B).

\begin{description}
\item[(B)]
\emph{S} believes that \emph{p} iff \emph{S} prefers as if \emph{p}
\end{description}

Our philosopher one day starts thinking about justification, and decides
that we can get a principle out of (B) by adding normative operators to
both sides, inferring (JB).

\begin{description}
\item[(JB)]
\emph{S} is justified in believing that \emph{p} only if \emph{S} is
justified to prefer as if \emph{p}
\end{description}

Now it would be a mistake to treat (JB) as a pragmatic condition on
\emph{justification} (rather than belief) if it was derived from (B) by
this simple means. And if our philosopher goes on to infer (PC) from
(JB), by replacing `justified' with `rational', and inferring the
conditional from the biconditional, we still don't get a pragmatic
condition on \emph{justification}.

Second, Fantl and McGrath focus their efforts on attacking the following
principle.

\textbf{Evidentialism}\\
For any two subjects \emph{S} and S′, necessarily, if \emph{S} and S′
have the same evidence for/against \emph{p}, then \emph{S} is justified
in believing that \emph{p} iff S′ is, too.

I agree, evidentialism is false. And I agree that there are
counterexamples to evidentialism from subjects who are in different
practical situations. What I don't agree is that we learn much about the
role of pragmatic factors in \emph{epistemology} properly defined from
these counterexamples to evidentialism. Evidentialism follows from the
following three principles.

\textbf{Probabilistic Evidentialism}\\
For any two subjects \emph{S} and S′, and any degree of belief
\(\alpha\) necessarily, if \emph{S} and S′ have the same evidence
for/against \emph{p}, then \emph{S} is justified in believing that
\emph{p} to degree \(\alpha\) iff S′ is, too.

\textbf{Threshold View}\\
For any two subjects \emph{S} and S′, and any degree of belief
\(\alpha\), if \emph{S} and S′ both believe \emph{p} to degree
\(\alpha\), then \emph{S} believes that \emph{p} iff S′ does too.

\textbf{Probabilistic Justification}\\
For any \(S, S\) is justified in believing \emph{p} iff there is some
degree of belief \(\alpha\) such that \emph{S} is justified in believing
\emph{p} to degree \(\alpha\), and in S's situation, believing \emph{p}
to degree \(\alpha\) suffices for believing \emph{p}.

(Degrees of belief here are meant to be the subjective correlates of
Keynesian probabilities. See Keynes (\citeproc{ref-Keynes1921}{1921})
for more details. They need not, and usually will not, be numerical
values. The Threshold View is so-called because given some other
plausible premises it implies that \(S\) believes that \emph{p} iff S's
degree of belief in \emph{p} is above a threshold.)

I endorse Probabilistic Justification, and for present purposes at least
I endorse Probabilistic Evidentialism. The reason I think Evidentialism
fails is because the Threshold View is false. It is plausible that
Probabilistic Justification and Probabilistic Evidentialism are
epistemological principles, while the Threshold View is a principle from
philosophy of mind. So this matches up with the earlier contention that
the failure of Evidentialism tells us something interesting about the
role of pragmatics in philosophy of mind, rather than something about
the role of pragmatics in epistemology.

As noted, Hawthorne and Stanley are both more interested in knowledge
than justification. So my discussion of their views will inevitably be
somewhat distorting. I think what I say about justification here should
carry over to a theory of knowledge, but space prevents a serious
examination of that question. The primary bit of `translation' I have to
do to make their works relevant to a discussion of justification is to
interpret their defences of the principle (KP) below as implying some
support for (JP), which is obviously similar to (PC).

\begin{description}
\item[(KP)]
If \emph{S} knows that \emph{p}, then \emph{S} is justified in using
\emph{p} as a premise in practical reasoning.
\item[(JP)]
If \emph{S} justifiably believes that \emph{p}, then \emph{S} is
justified in using \emph{p} as a premise in practical reasoning.
\end{description}

I think (JP) is just as plausible as (KP). In any case it is
independently plausible whether or not Hawthorne and Stanley are
committed to it. So I'll credit recognition of (JP)'s importance to a
theory of justification to them, and hope that in doing so I'm not
irrepairably damaging the public record.

The overall plan here is to use some philosophy of mind, specifically
functionalist analyses of belief to respond to some arguments in
epistemology. But, as you can see from the role the Threshold View plays
in the above argument, our starting point will be the question what is
the relation between the credences decision theory deals with, and our
traditional notion of a belief? I'll offer an analysis of this relation
that supports my above claim that we should work with a pragmatic notion
of belief rather than a pragmatic notion of justification. The analysis
I offer has a hole in it concerning propositions that are not relevant
to our current plans, and I'll fix the hold in section 3. Sections 4 and
5 concern the role that closure principles play in my theory, in
particular the relationship between having probabilistically coherent
degrees of belief and logically coherent beliefs. In this context, a
closure principle is a principle that says probabilistic coherence
implies logical coherence, at least in a certain domain. (It's called a
closure principle because we usually discuss it by working out
properties of probabilistically coherent agents, and show that their
beliefs are closed under entailment in the relevant domain.) In section
4 I'll defend the theory against the objection, most commonly heard from
those wielding the preface paradox, that we need not endorse as strong a
closure principle as I do. In section 5 I'll defend the theory against
those who would endorse an even stronger closure principle than is
defended here. Once we've got a handle on the relationship between
degrees of belief and belief \emph{tout court}, we'll use that to
examine the arguments for pragmatic encroachment. In section 6 I'll
argue that we can explain the intuitions behind the cases that seem to
support pragmatic encroachment, while actually keeping all of the
pragmatic factors in our theory of belief. In section 7 I'll discuss how
to endorse principles like (PC) and (JP) (as far as they can be
endorsed) while keeping a non-pragmatic theory of probabilistic
justification. The interesting cases here are ones where agents have
mistaken and/or irrational beliefs about their practical environment,
and intuitions in those cases are cloudy. But it seems the most natural
path in these cases is to keep a pragmatically sensitive notion of
belief, and a pragmatically insensitive notion of justification.

\section{Belief and Degree of Belief}\label{belief-and-degree-of-belief}

Traditional epistemology deals with beliefs and their justification.
Bayesian epistemology deals with degrees of belief and their
justification. In some sense they are both talking about the same thing,
namely epistemic justification. Two questions naturally arise. Do we
really have two subject matters here (degrees of belief and belief
\emph{tout court}) or two descriptions of the one subject matter? If
just one subject matter, what relationship is there between the two
modes of description of this subject matter?

The answer to the first question is I think rather easy. There is no
evidence to believe that the mind contains two representational systems,
one to represent things as being probable or improbable and the other to
represent things as being true or false. The mind probably does contain
a vast plurality of representational systems, but they don't divide up
the doxastic duties this way. If there are distinct visual and auditory
representational systems, they don't divide up duties between degrees of
belief and belief \emph{tout court}, for example. If there were two
distinct systems, then we should imagine that they could vary
independently, at least as much as is allowed by constitutive
rationality. But such variation is hard to fathom. So I'll infer that
the one representational system accounts for our credences and our
categorical beliefs. (It follows from this that the question Bovens and
Hawthorne (\citeproc{ref-Bovens1999}{1999}) ask, namely what beliefs
\emph{should} an agent have given her degrees of belief, doesn't have a
non-trivial answer. If fixing the degrees of belief in an environment
fixes all her doxastic attitudes, as I think it does, then there is no
further question of what she should believe given these are her degrees
of belief.)

The second question is much harder. It is tempting to say that \(S\)
believes that \emph{p} iff S's credence in \emph{p} is greater than some
salient number \(r\), where \(r\) is made salient either by the context
of belief ascription, or the context that \emph{S} is in. I'm following
Mark Kaplan (\citeproc{ref-Kaplan1996}{1996}) in calling this the
threshold view. There are two well-known problems with the threshold
view, both of which seem fatal to me.

As Robert Stalnaker (\citeproc{ref-Stalnaker1984}{1984, 91}) emphasised,
any number \(r\) is bound to seem arbitrary. Unless these numbers are
made salient by the environment, there is no special difference between
believing \emph{p} to degree 0.9786 and believing it to degree 0.9875.
But if \(r\) is 0.98755, this will be \emph{the difference} between
believing \emph{p} and not believing it, which is an important
difference. The usual response to this, as found in Foley
(\citeproc{ref-Foley1993}{1993} Ch. 4) and Hunter
(\citeproc{ref-Hunter1996}{1996}) is to say that the boundary is vague.
But it's not clear how this helps. On an epistemic theory of vagueness,
there is still a number such that degrees of belief above that count,
and degrees below that do not, and any such number is bound to seem
unimportant. On supervaluational theories, the same is true. There won't
be a \emph{determinate} number, to be sure, but there will a number, and
that seems false. My preferred degree of belief theory of vagueness, as
set out in Weatherson (\citeproc{ref-Weatherson2005-WEATTT}{2005}) has
the same consequence. Hunter defends a version of the threshold view
combined with a theory of vagueness based around fuzzy logic, which
seems to be the only theory that could avoid the arbitrariness
objection. But as Williamson (\citeproc{ref-Williamson1994-WILV}{1994})
showed, there are deep and probably insurmountable difficulties with
that position. So I think the vagueness response to the arbitrariness
objection is (a) the only prima facie plausible response and (b)
unsuccessful.

The second problem concerns conjunction. It is also set out clearly by
Stalnaker.

\begin{quote}
Reasoning in this way from accepted premises to their deductive
consequences (\emph{p}, also \(Q\), therefore \(R\)) does seem perfectly
straightforward. Someone may object to one of the premises, or to the
validity of the argument, but one could not intelligibly agree that the
premises are each acceptable and the argument valid, while objecting to
the acceptability of the conclusion.
(\citeproc{ref-Stalnaker1984}{Stalnaker 1984, 92})
\end{quote}

If categorical belief is having a credence above the threshold, then one
can coherently do exactly this. Let \emph{x} be a number between \(r\)
and than \(r\) \(\frac{1}{2}\), such that for an atom of type U has
probability \emph{x} of decaying within a time \emph{t}, for some
\emph{t} and U. Assume our agent knows this fact, and is faced with two
(isolated) atoms of U. Let \emph{p} be that the first decays within
\emph{t}, and \(q\) be that the second decays within \emph{t}. She
should, given her evidence, believe \emph{p} to degree \(x, q\) to
degree \emph{x}, and \(p \wedge q\) to degree \(x ^2\). If she believed
\(p \wedge q\) to a degree greater than \(r\), she'd have to either have
credences that were not supported by her evidence, or credences that
were incoherent. (Or, most likely, both.) So this theory violates the
platitude. This is a well-known argument, so there are many responses to
it, most of them involving something like appeal to the preface paradox.
I'll argue in section 4 that the preface paradox doesn't in fact offer
the threshold view proponent much support here. But even before we get
to there, we should note that the arbitrariness objection gives us
sufficient reason to reject the threshold view.

A better move is to start with the functionalist idea that to believe
that \emph{p} is to treat \emph{p} as true for the purposes of practical
reasoning. To believe \emph{p} is to have preferences that make sense,
by your own lights, in a world where \emph{p} is true. So, if you prefer
A to B and believe that \emph{p}, you prefer A to B given \emph{p}. For
reasons that will become apparent below, we'll work in this paper with a
notion of preference where \emph{conditional} preferences are
primary.\footnote{To say the agent prefers A to B given \(q\) is not to
  say that if the agent were to learn \(q\), she would prefer A to B.
  It's rather to say that she prefers the state of the world where she
  does A and \(q\) is true to the state of the world where she does B
  and \(q\) is true. These two will come apart in cases where learning
  \(q\) changes the agent's preferences. We'll return to this issue
  below.} So the core insight we'll work with is the following:

\begin{quote}
If you prefer A to B given \(q\), and you believe that \emph{p}, then
you prefer A to B given \(p \wedge q\)
\end{quote}

The bold suggestion here is that if that is true for all the A, B and
\emph{q} that matter, then you believe \emph{p}. Put formally, where
\emph{Bel}(\emph{p}) means that the agent believes that \emph{p}, and A
\(\geq _q\) B means that the agent thinks A is at least as good as B
given \(q\), we have the following

\begin{enumerate}
\def\labelenumi{\arabic{enumi}.}
\tightlist
\item
  \emph{Bel}(\emph{p})
  \(\leftrightarrow \forall\)A\(\forall\)B\(\forall q\) (A \(\geq _q\) B
  \(\leftrightarrow\) A \(\geq _{p \wedge q}\) B)
\end{enumerate}

In words, an agent believes that \emph{p} iff conditionalising on
\emph{p} doesn't change any conditional preferences over things that
matter.\footnote{This might seem \emph{much} too simple, especially when
  compared to all the bells and whistles that functionalists usually put
  in their theories to (further) distinguish themselves from crude
  versions of behaviourism. The reason we don't need to include those
  complications here is that they will all be included in the analysis
  of \emph{preference}. Indeed, the theory here is compatible with a
  thoroughly anti-functionalist treatment of preference. The claim is
  not that we can offer a functional analysis of belief in terms of
  non-mental concepts, just that we can offer a functionalist reduction
  of belief to other mental concepts. The threshold view is \emph{also}
  such a reduction, but it is such a crude reduction that it doesn't
  obviously fall into any category.} The left-to-right direction of this
seems trivial, and the right-to-left direction seems to be a plausible
way to operationalise the functionalist insight that belief is a
functional state. There is some work to be done if (1) is to be
interpreted as a truth though.

If we interpret the quantifiers in (1) as unrestricted, then we get the
(false) conclusion that just about no one believes no contingent
propositions. To prove this, consider a bet that wins iff the statue in
front of me waves back at me due to random quantum effects when I wave
at it. If I take the bet and win, I get to live forever in paradise. If
I take the bet and lose, I lose a penny. Letting A be that I take the
bet, B be that I decline the bet, \(q\) be a known tautology (so my
preferences given \(q\) are my preferences \emph{tout court}) and
\emph{p} be that the statue does not wave back, we have that I prefer A
to B, but not A to B given \emph{p}. So by this standard I don't believe
that \emph{p}. This is false -- right now I believe that statues won't
wave back at me when I wave at them.

This seems like a problem. But the solution to it is not to give up on
functionalism, but to insist on its pragmatic foundations. The
quantifiers in (1) should be restricted, with the restrictions motivated
pragmatically. What is crucial to the theory is to say what the
restrictions on A and B are, and what the restrictions on \(q\) are.
We'll deal with these in order.

For better or worse, I don't right now have the option taking that bet
and hence spending eternity in paradise if the statue waves back at me.
Taking or declining such unavailable bets are not open choices. For any
option that is open to me, assuming that statues do not in fact wave
does not change its utility. That's to say, I've already factored in the
non-waving behaviour of statues into my decision-making calculus. That's
to say, I believe statues don't wave.

An action A is a live option for the agent if it is really possible for
the agent to perform A. An action A is a salient option if it is an
option the agent takes seriously in deliberation. Most of the time
gambling large sums of money on internet gambling sites over my phone is
a live option, but not a salient option. I know this option is
suboptimal, and I don't have to recompute every time whether I should do
it. Whenever I'm making a decision, I don't have to add in to the list
of choices \emph{bet thousands of dollars on internet gambling sites},
and then rerule that out every time. I just don't consider that option,
and properly so. If I have a propensity to daydream, then becoming the
centrefielder for the Boston Red Sox might be a salient option to me,
but it certainly isn't a live option. We'll say the two initial
quantifiers range over the options that are live and salient options for
the agent.

Note that we \emph{don't} say that the quantifiers range over the
options that are live and salient for the person making the belief
ascription. That would lead us to a form of contextualism for which we
have little evidence. We also don't say that an option becomes salient
for the agent iff they \emph{should} be considering it. At this stage we
are just saying what the agent does believe, not what they should
believe, so we don't have any clauses involving normative concepts.

Now we'll look at the restrictions on the quantifier over propositions.
Say a proposition is \emph{relevant} if the agent is disposed to take
seriously the question of whether it is true (whether or not she is
currently considering that question) and conditionalising on that
proposition or its negation changes some of the agents
\emph{unconditional} preferences over live, salient options.\footnote{Conditionalising
  on the proposition \emph{There are space aliens about to come down and
  kill all the people writing epistemology papers} will make me prefer
  to stop writing this paper, and perhaps grab some old metaphysics
  papers I could be working on. So that proposition satisfies the second
  clause of the definition of relevance. But it clearly doesn't satisfy
  the first clause. This part of the definition of relevance won't do
  much work until the discussion of agents with mistaken environmental
  beliefs in section 7.} The first clause is designed to rule out wild
hypotheses that the agent does not take at all seriously. If \(q\) is
not such a proposition, if the agent is disposed to take it seriously,
then it is relevant if there are live, salient A and B such that A
\(\geq _q\) B \(\leftrightarrow\) A \(\geq\) B is false. Say a
proposition is \emph{salient} if the agent is currently considering
whether it is true. Finally, say a proposition is \emph{active} relative
to \emph{p} iff it is a (possibly degenerate) conjunction of
propositions such that each conjunct is either relevant or salient, and
such that the conjunction is consistent with \emph{p}. (By a degenerate
conjunction I mean a conjunction with just one conjunct. The consistency
requirement is there because it might be hard in some cases to make
sense of preferences given inconsistencies.) Then the propositional
quantifier in (1) ranges over active propositions.

We will expand and clarify this in the next section, but our current
solution to the relationship between beliefs and degrees of belief is
that degrees of belief determine an agent's preferences, and she
believes that \emph{p} iff the claim (1) about her preferences is true
when the quantifiers over options are restricted to live, salient
actions, and the quantifier over propositions is restricted to salient
propositions. The simple view would be to say that the agent believes
that \emph{p} iff conditioning on \emph{p} changes none of her
preferences. The more complicated view here is that the agent believes
that \emph{p} iff conditioning on \emph{p} changes none of her
conditional preferences over live, salient options, where the conditions
are also active relative to \emph{p}.

\section{Impractical Propositions}\label{impractical-propositions}

The theory sketched in the previous paragraph seems to me right in the
vast majority of cases. It fits in well with a broadly functionalist
view of the mind, and as we'll see it handles some otherwise difficult
cases with aplomb. But it needs to be supplemented a little to handle
beliefs about propositions that are practically irrelevant. I'll
illustrate the problem, then note how I prefer to solve it.

I don't know what Julius Caeser had for breakfast the morning he crossed
the Rubicon. But I think he would have had \emph{some} breakfast. It is
hard to be a good general without a good morning meal after all. Let
\emph{p} be the proposition that he had breakfast that morning. I
believe \emph{p}. But this makes remarkably little difference to my
practical choices in most situations. True, I wouldn't have written this
paragraph as I did without this belief, but it is rare that I have to
write about Caeser's dietary habits. In general whether \emph{p} is true
makes no practical difference to me. This makes it hard to give a
pragmatic account of whether I believe that \emph{p}. Let's apply (1) to
see whether I really believe that \emph{p}.

\begin{enumerate}
\def\labelenumi{\arabic{enumi}.}
\tightlist
\item
  \emph{Bel}(\emph{p})
  \(\leftrightarrow \forall\)A\(\forall\)B\(\forall q\) (A \(\geq _q\) B
  \(\leftrightarrow\) A \(\geq _{p \wedge q}\) B)
\end{enumerate}

Since \emph{p} makes no practical difference to any choice I have to
make, the right hand side is true. So the left hand side is true, as
desired. The problem is that the right hand side of (2) is also true
here.

\begin{enumerate}
\def\labelenumi{\arabic{enumi}.}
\setcounter{enumi}{1}
\tightlist
\item
  \emph{Bel}(\(\neg p\))
  \(\leftrightarrow \forall\)A\(\forall\)B\(\forall q\) (A \(\geq _q\) B
  \(\leftrightarrow\) A \(\geq _{\neg p \wedge q}\) B)
\end{enumerate}

Adding the assumption that Caeser had no breakfast that morning doesn't
change any of my practical choices either. So I now seem to
\emph{inconsistently} believe both \emph{p} and \(\neg p\). I have some
inconsistent beliefs, I'm sure, but those aren't among them. We need to
clarify what (1) claims.

To do so, I supplement the theory sketched in section 2 with the
following principles.

\begin{itemize}
\item
  A proposition \emph{p} is \emph{eligible} \emph{for belief} if it
  satisfies \(\forall\)A\(\forall\)B\(\forall q\) (A \(\geq _q\) B
  \(\leftrightarrow\) A \(\geq _{p \wedge q}\) B), where the first two
  quantifiers range over the open, salient actions in the sense
  described in section 2.
\item
  For any proposition \emph{p}, and any proposition \(q\) that is
  relevant or salient, among the actions that are (by stipulation!) open
  and salient with respect to \emph{p} are \emph{believing that p},
  \emph{believing that q}, \emph{not believing that p} and \emph{not
  believing that q}
\item
  For any proposition, the subject prefers believing it to not believing
  it iff (a) it is eligible for belief and (b) the agents degree of
  belief in the proposition is greater than \(\frac{1}{2}\).
\item
  The previous stipulation holds both unconditionally and conditional on
  \emph{p}, for any \emph{p}.
\item
  The agent believes that \emph{p} iff
  \(\forall\)A\(\forall\)B\(\forall q\) (A \(\geq _q\) B
  \(\leftrightarrow\) A \(\geq _{p \wedge q}\) B), where the first two
  quantifiers range over all actions that are either open and salient
  \emph{tout court} (i.e.~in the sense of section 2) or open and salient
  with respect to \emph{p} (as described above).
\end{itemize}

This all looks moderately complicated, but I'll explain how it works in
some detail as we go along. One simple consequence is that an agent only
believes that \emph{p} iff their degree of belief in \emph{p} is greater
than \(\frac{1}{2}\). Since my degree of belief in Caeser's foodless
morning is not greater than \(\frac{1}{2}\), in fact it is considerably
less, I don't believe \(\neg p\). On the other hand, since my degree of
belief in \emph{p} is considerably greater than \(\frac{1}{2}\), I
prefer to believe it than disbelieve it, so I believe it.

There are many possible objections to this position, which I'll address
sequentially.

\emph{Objection}: Even if I have a high degree of belief in \emph{p}, I
might prefer to not believe \emph{p} because I think that belief in
\emph{p} is bad for some other reason. Perhaps, if \emph{p} is a
proposition about my brilliance, it might be immodest to believe that
\emph{p}.

\emph{Reply}: Any of these kinds of considerations should be put into
the credences. If it is immodest to believe that you are a great
philosopher, it is equally immodest to believe to a high degree that you
are a great philosopher.

\emph{Objection}: Belief that \emph{p} is not an action in the ordinary
sense of the term.

\emph{Reply}: True, which is why this is described as a supplement to
the original theory, rather than just cashing out its consequences.

\emph{Objection}: It is impossible to choose to believe or not believe
something, so we shouldn't be applying these kinds of criteria.

\emph{Reply}: I'm not as convinced of the impossibility of belief by
choice as others are, but I won't push that for present purposes. Let's
grant that beliefs are always involuntary. So these `actions' aren't
open actions in any interesting sense, and the theory is section 2 was
really incomplete. As I said, this is a supplement to the theory in
section 2.

This doesn't prevent us using principles of constitutive rationality,
such as we prefer to believe \emph{p} iff our credence in \emph{p} is
over \(\frac{1}{2}\). Indeed, on most occasions where we use
constitutive rationality to infer that a person has some mental state,
the mental state we attribute to them is one they could not fail to
have. But functionalists are committed to constitutive rationality
(\citeproc{ref-Lewis1994b}{Lewis 1994}). So my approach here is
consistent with a broadly functionalist outlook.

\emph{Objection}: This just looks like a roundabout way of stipulating
that to believe that \emph{p}, your degree of belief in \emph{p} has to
be greater than \(\frac{1}{2}\). Why not just add that as an extra
clause than going through these little understood detours about
preferences about beliefs?

\emph{Reply}: There are three reasons for doing things this way rather
than adding such a clause.

First, it's nice to have a systematic theory rather than a theory with
an ad hoc clause like that.

Second, the effect of this constraint is much more than to restrict
belief to propositions whose credence is greater than \(\frac{1}{2}\).
Consider a case where \emph{p} and \(q\) and their conjunction are all
salient, \emph{p} and \(q\) are probabilistically independent, and the
agent's credence in each is 0.7. Assume also that \(p, q\) and
\(p \wedge q\) are completely irrelevant to any practical deliberation
the agent must make. Then the criteria above imply that the agent does
not believe that \emph{p} or that \(q\). The reason is that the agent's
credence in \(p \wedge q\) is 0.49, so she prefers to not believe
\(p \wedge q\). But conditional on \emph{p}, her credence in
\(p \wedge q\) is 0.7, so she prefers to believe it. So conditionalising
on \emph{p} does change her preferences with respect to believing
\(p \wedge q\), so she doesn't believe \emph{p}. So the effect of these
stipulations rules out much more than just belief in propositions whose
credence is below \(\frac{1}{2}\).

This suggests the third, and most important point. The problem with the
threshold view was that it led to violations of closure. Given the
theory as stated, we can prove the following theorem. Whenever \emph{p}
and \(q\) and their conjunction are all open or salient, and both are
believed, and the agent is probabilistically coherent, the agent also
believes \(p \wedge q\). This is a quite restricted closure principle,
but this is no reason to deny that it is \emph{true}, as it fails to be
true on the threshold view.

The proof of this theorem is a little complicated, but worth working
through. First we'll prove that if the agent believes \emph{p}, believes
\(q\), and \emph{p} and \(q\) are both salient, then the agent prefers
believing \(p \wedge q\) to not believing it, if \(p \wedge q\) is
eligible for belief. In what follows \emph{Pr}(\(x | y\)) is the agent's
conditional degree of belief in \emph{x} given \(y\). Since the agent is
coherent, we'll assume this is a probability function (hence the name).

\begin{enumerate}
\def\labelenumi{\arabic{enumi}.}
\item
  Since the agent believes that \(q\), they prefer believing that \(q\)
  to not believing that \(q\) (by the criteria for belief)
\item
  So the agent prefers believing that \(q\) to not believing that \(q\)
  given \emph{p} (From 1 and the fact that they believe that \emph{p},
  and that \(q\) is salient)
\item
  So \emph{Pr}(\(q | p\)) \(> \frac{1}{2}\) (from 2)
\item
  \emph{Pr}(\(q | p\)) = \emph{Pr}(\(p \wedge q | p\)) (by probability
  calculus)
\item
  So \emph{Pr}(\(p \wedge q | p\)) \(> \frac{1}{2}\) (from 3, 4)
\item
  So, if \(p \wedge q\) is eligible for belief, then the agent prefers
  believing that \(p \wedge q\) to not believing it, given \emph{p}
  (from 5)
\item
  So, if \(p \wedge q\) is eligible for belief, the agent prefers
  believing that \(p \wedge q\) to not believing it (from 6, and the
  fact that they believe that \emph{p}, and \(p \wedge q\) is salient)
\end{enumerate}

So whenever, \(p, q\) and \(p \wedge q\) are salient, and the agent
believes each conjunct, the agent prefers believing the conjunction
\(p \wedge q\) to not believing it, if \(p \wedge q\) is eligible. Now
we have to prove that \(p \wedge q\) is eligible for belief, to prove
that it is actually believed. That is, we have to prove that (5) follows
from (4) and (3), where the initial quantifiers range over actions that
are open and salient \emph{tout court}.

\begin{enumerate}
\def\labelenumi{\arabic{enumi}.}
\item
  \(\forall\)A\(\forall\)B\(\forall r\) (A \(\geq_r\) B
  \(\leftrightarrow\) A \(\geq _p \wedge r\) B)
\item
  \(\forall\)A\(\forall\)B\(\forall r\) (A \(\geq_r\) B
  \(\leftrightarrow\) A \(\geq _q \wedge r\) B)
\item
  \(\forall\)A\(\forall\)B\(\forall r\) (A \(\geq_r\) B
  \(\leftrightarrow\) A \(\geq _{p \wedge q \wedge r}\) B)
\end{enumerate}

Assume that (5) isn't true. That is, there are A, B and \emph{S} such
that \(\neg\)(A \(\geq_s\) B \(\leftrightarrow\) A
\(\geq _{p \wedge q \wedge s}\)B). By hypothesis \emph{S} is active, and
consistent with \(p \wedge q\). So it is the conjunction of relevant,
salient propositions. Since \(q\) is salient, this means \(q \wedge s\)
is also active. Since \emph{S} is consistent with \(p \wedge q\), it
follows that \(q \wedge s\) is consistent with \emph{p}. So
\(q \wedge s\) is a possible substitution instance for \(r\) in (3).
Since (3) is true, it follows that A \(\geq _{q \wedge s}\) B
\(\leftrightarrow\) A \(\geq _{p \wedge q \wedge s}\) B. By similar
reasoning, it follows that \emph{s} is a permissible substitution
instance in (4), giving us A \(\geq_s\) B \(\leftrightarrow\) A
\(\geq _{q \wedge s}\) B. Putting the last two biconditionals together
we get A \(\geq_s\) B \(\leftrightarrow\) A
\(\geq _{p \wedge q \wedge s}\)B, contradicting our hypothesis that
there is a counterexample to (5). So whenever (3) and (4) are true, (5)
is true as well, assuming \(p, q\) and \(p \wedge q\) are all salient.

\section{Defending Closure}\label{defending-closure}

So on my account of the connection between degrees of belief and belief
\emph{tout court}, probabilistic coherence implies logical coherence
amongst salient propositions. The last qualification is necessary. It is
possible for a probabilistically coherent agent to not believe the
\emph{non}-salient consequences of things they believe, and even for a
probabilistically coherent agent to have inconsistent beliefs as long as
not all the members of the inconsistent set are active. Some people
argue that even this weak a closure principle is implausible. David
Christensen (\citeproc{ref-Christensen2005}{2005}), for example, argues
that the preface paradox provides a reason for doubting that beliefs
must be closed under entailment, or even must be consistent. Here is his
description of the case.

\begin{quote}
We are to suppose that an apparently rational person has written a long
non-fiction book---say, on history. The body of the book, as is typical,
contains a large number of assertions. The author is highly confident in
each of these assertions; moreover, she has no hesitation in making them
unqualifiedly, and would describe herself (and be described by others)
as believing each of the book's many claims. But she knows enough about
the difficulties of historical scholarship to realize that it is almost
inevitable that at least a few of the claims she makes in the book are
mistaken. She modestly acknowledges this in her preface, by saying that
she believes the book will be found to contain some errors, and she
graciously invites those who discover the errors to set her straight.
(\citeproc{ref-Christensen2005}{Christensen 2005, 33--34})
\end{quote}

Christensen thinks such an author might be rational in every one of her
beliefs, even though these are all inconsistent. Although he does not
say this, nothing in his discussion suggests that he is using the
irrelevance of some of the propositions in the author's defence. So here
is an argument that we should abandon closure amongst relevant beliefs.

Christensen's discussion, like other discussions of the preface paradox,
makes frequent use of the fact that examples like these are quite
common. We don't have to go to fake barn country to find a
counterexample to closure. But it seems to me that we need two quite
strong idealisations in order to get a real counterexample here.

The first of these is discussed in forthcoming work by Ishani Maitra
(\citeproc{ref-MaitraANG}{2010}), and is briefly mentioned by
Christensen in setting out the problem. We only have a counterexample to
closure if the author \emph{believes} every thing she writes in her
book. (Indeed, we only have a counterexample if she reasonably believes
every one of them. But we'll assume a rational author who only believes
what she ought to believe.) This seems unlikely to be true to me. An
author of a historical book is like a detective who, when asked to put
forward her best guess about what explains the evidence, says ``If I had
to guess, I'd say \ldots{}'' and then launches into spelling out her
hypothesis. It seems clear that she need not \emph{believe} the truth of
her hypothesis. If she did that, she could not later learn it was true,
because you can't learn the truth of something you already believe. And
she wouldn't put any effort into investigating alternative suspects. But
she can come to learn her hypothesis was true, and it would be rational
to investigate other suspects. It seems to me (following here Maitra's
discussion) that we should understand scholarly assertions as being
governed by the same kind of rules that govern detectives making the
kind of speech being contemplated here. And those rules don't require
that the speaker believe the things they say without qualification. The
picture is that the little prelude the detective explicitly says is
implicit in all scholarly work.

There are three objections I know to this picture, none of them
particularly conclusive. First, Christensen says that the author doesn't
qualify their assertions. But neither does our detective qualify most
individual sentences. Second, Christensen says that most people would
describe our author as believing her assertions. But it is also natural
to describe our detective as believing the things she says in her
speech. It's natural to say things like ``She thinks it was the butler,
with the lead pipe,'' in reporting her hypothesis. Third, Timothy
Williamson (\citeproc{ref-Williamson2000-WILKAI}{2000}) has argued that
if speakers don't believe what they say, we won't have an explanation of
why Moore's paradoxical sentences, like ``The butler did it, but I don't
believe the butler did it,'' are always defective. Whatever the
explanation of the paradoxicality of these sentences might be, the
alleged requirement that speakers believe what they say can't be it. For
our detective cannot properly say ``The butler did it, but I don't
believe the butler did it'' in setting out her hypothesis, even though
\emph{believing} the butler did it is not necessary for her to say ``The
butler did it'' in setting out just that hypothesis.

It is plausible that for \emph{some} kinds of books, the author should
only say things they believe. This is probably true for travel guides,
for example. Interestingly, casual observation suggests that authors of
such books are much less likely to write modest prefaces. This makes
some sense if those books can only include statements their authors
believe, and the authors believe the conjunctions of what they believe.

The second idealisation is stressed by Simon Evnine
(\citeproc{ref-Evnine1999}{1999}) in his paper ``Believing
Conjunctions''. The following situation does not involve me believing
anything inconsistent.

\begin{itemize}
\tightlist
\item
  I believe that what Manny just said, whatever it was, is false.
\item
  Manny just said that the stands at Fenway Park are green.
\item
  I believe that the stands at Fenway Park are green.
\end{itemize}

If we read the first claim \emph{de dicto}, that I believe that Manny
just said something false, then there is no inconsistency. (Unless I
also believe that what Manny just said was that the stands in Fenway
Park are green.) But if we read it \emph{de re}, that the thing Manny
just said is one of the things I believe to be false, then the situation
does involve me being inconsistent. The same is true when the author
believes that one of the things she says in her book is mistaken. If we
understand what she says \emph{de dicto}, there is no contradiction in
her beliefs. It has to be understood \emph{de re} before we get a
logical problem. And the fact is that most authors do not have \emph{de
re} attitudes towards the claims made in their book. Most authors don't
even remember everything that's in their books. (I'm not sure I remember
how this section started, let alone this paper.) Some may argue that
authors don't even have the capacity to consider a proposition as long
and complicated as the conjunction of all the claims in their book.
Christensen considers this objection, but says it isn't a serious
problem.

\begin{quote}
It is undoubtedly true that ordinary humans cannot entertain book-length
conjunctions. But surely, agents who do not share this fairly
\emph{superficial} limitation are easily conceived. And it seems just as
wrong to say of such agents that they are rationally required to believe
in the inerrancy of the books they write. (38: my emphasis)
\end{quote}

I'm not sure this is undoubtedly true; it isn't clear that propositions
(as opposed to their representations) have lengths. And humans can
believe propositions that \emph{can} be represented by sentences as long
as books. But even without that point, Christensen is right that there
is an idealisation here, since ordinary humans do not know exactly what
is in a given book, and hence don't have \emph{de re} attitudes towards
the propositions expressed in the book.

I'm actually rather suspicious of the intuition that Christensen is
pushing here, that idealising in this way doesn't change intuitions
about the case. The preface paradox gets a lot of its (apparent) force
from intuitions about what attitude we should have towards real books.
Once we make it clear that the real life cases are not relevant to the
paradox, I find the intuitions become rather murky. But I won't press
this point.

A more important point is that we believers in closure don't think that
authors should think their books are inerrant. Rather, following
Stalnaker (\citeproc{ref-Stalnaker1984}{1984}), we think that authors
shouldn't unqualifiedly \emph{believe} the individual statements in
their book if they don't believe the conjunction of those statements.
Rather, their attitude towards those propositions (or at least some of
them) should be that they are probably true. (As Stalnaker puts it, they
accept the story without believing it.) Proponents of the preface
paradox know that this is a possible response, and tend to argue that it
is impractical. Here is Christensen on this point.

\begin{quote}
It is clear that our everyday binary way of talking about beliefs has
immense practical advantages over a system which insisted on some more
fine-grained reporting of degrees of confidence \ldots{} At a minimum,
talking about people as believing, disbelieving, or withholding belief
has at least as much point as do many of the imprecise ways we have of
talking about things that can be described more precisely. (96)
\end{quote}

Richard Foley makes a similar point.

\begin{quote}
There are \emph{deep} reasons for wanting an epistemology of beliefs,
reasons that epistemologies of degrees of belief by their very nature
cannot possibly accommodate. (\citeproc{ref-Foley1993}{Foley 1993, 170},
my emphasis)
\end{quote}

It's easy to make too much of this point. It's a lot easier to triage
propositions into TRUE, FALSE and NOT SURE and work with those
categories than it is to work assign precise numerical probabilities to
each proposition. But these are not the only options. Foley's discussion
subsequent to the above quote sometimes suggests they are, especially
when he contrasts the triage with ``indicat{[}ing{]} as accurately as I
can my degree of confidence in each assertion that I defend.'' (171) But
really it isn't \emph{much} harder to add two more categories, PROBABLY
TRUE and PROBABLY FALSE to those three, and work with that five-way
division rather than a three-way division. It's not clear that humans as
they are actually constructed have a \emph{strong} preference for the
three-way over the five-way division, and even if they do, I'm not sure
in what sense this is a `deep' fact about them.

Once we have the five-way division, it is clear what authors should do
if they want to respect closure. For any conjunction that they don't
believe (i.e.~classify as true), they should not believe one of the
conjuncts. But of course they can classify every conjunct as probably
true, even if they think the conjunction is false, or even certainly
false. Still, might it not be considered something of an idealisation to
say rational authors must make this five-way distinction amongst
propositions they consider? Yes, but it's no more of an idealisation
than we need to set up the preface paradox in the first place. To use
the preface paradox to find an example of someone who reasonably
violates closure, we need to insist on the following three constraints.

\begin{enumerate}
\def\labelenumi{\arabic{enumi}.}
\item
  They are part of a research community where only asserting
  propositions you believe is compatible with active scholarship;
\item
  They know exactly what is in their book, so they are able to believe
  that one of the propositions in the book is mistaken, where this is
  understood \emph{de re}; but
\item
  They are unable to effectively function if they have to effect a
  five-way, rather than a three-way, division amongst the propositions
  they consider.
\end{enumerate}

Put more graphically, to motivate the preface paradox we have to think
that our inability to have \emph{de re} thoughts about the contents of
books is a ``superficial constraint'', but our preference for working
with a three-way rather than a five-way division is a ``deep'' fact
about our cognitive system. Maybe each of these attitudes could be
plausible taken on its own (though I'm sceptical of that) but the
conjunction seems just absurd.

I'm not entirely sure an agent subject to exactly these constraints is
even fully conceivable. (Such an agent is negatively conceivable, in
David Chalmers's terminology, but I rather doubt they are positively
conceivable.) But even if they are a genuine possibility, why the norms
applicable to an agent satisfying that very gerrymandered set of
constraints should be considered relevant norms for our state is far
from clear. I'd go so far as to say it's clear that the applicability
(or otherwise) of a given norm to such an odd agent is no reason
whatsoever to say it applies to us. But since the preface paradox only
provides a reason for just these kinds of agents to violate closure, we
have no reason for ordinary humans to violate closure. So I see no
reason here to say that we can have probabilistic coherence without
logical coherence, as proponents of the threshold view insist we can
have, but which I say we can't have \emph{at least when the propositions
involved are salient}. The more pressing question, given the failure of
the preface paradox argument, is why I don't endorse a much stronger
closure principle, one that drops the restriction to salient
propositions. The next section will discuss that point.

I've used Christensen's book as a stalking horse in this section,
because it is the clearest and best statement of the preface paradox.
Since Christensen is a paradox-mongerer and I'm a paradox-denier, it
might be thought we have a deep disagreement about the relevant
epistemological issues. But actually I think our overall views are
fairly close despite this. I favour an epistemological outlook I call
``Probability First'', the view that getting the epistemology of partial
belief right is of the first importance, and everything else should flow
from that. Christensen's view, reduced to a slogan, is ``Probability
First and Last''. This section has been basically about the difference
between those two slogans. It's an important dispute, but it's worth
bearing in mind that it's a factional squabble within the Probability
Party, not an outbreak of partisan warfare.

\section{Too Little Closure?}\label{too-little-closure}

In the previous section I defended the view that a coherent agent has
beliefs that are deductively cogent with respect to salient
propositions. Here I want to defend the importance of the qualification.
Let's start with what I take to be the most important argument for
closure, the passage from Stalnaker's \emph{Inquiry} that I quoted
above.

\begin{quote}
Reasoning in this way from accepted premises to their deductive
consequences (\emph{p}, also \(Q\), therefore \(R\)) does seem perfectly
straightforward. Someone may object to one of the premises, or to the
validity of the argument, but one could not intelligibly agree that the
premises are each acceptable and the argument valid, while objecting to
the acceptability of the conclusion.
(\citeproc{ref-Stalnaker1984}{Stalnaker 1984, 92})
\end{quote}

Stalnaker's wording here is typically careful. The relevant question
isn't whether we can accept \emph{p}, accept \(q\), accept \emph{p} and
\(q\) entail \(r\), and reject \(r\). As Christensen
(\citeproc{ref-Christensen2005}{2005} Ch. 4) notes, this is impossible
even on the threshold view, as long as the threshold is above 2/3. The
real question is whether we can accept \emph{p}, accept \(q\), accept
\emph{p} and \(q\) entail \(r\), and \emph{fail} to accept \(r\). And
this is always a live possibility on any threshold view, though it seems
absurd at first that this could be coherent.

But it's important to note how \emph{active} the verbs in Stalnaker's
description are. When faced with a valid argument we have to
\emph{object} to one of the premises, or the validity of the argument.
What we can't do is \emph{agree} to the premises and the validity of the
argument, while \emph{objecting} to the conclusion. I agree. If we are
really \emph{agreeing} to some propositions, and \emph{objecting} to
others, then all those propositions are salient. And in that case
closure, deductive coherence, is mandatory. This doesn't tell us what we
have to do if we haven't previously made the propositions salient in the
first place.

The position I endorse here is very similar in its conclusions to that
endorsed by Gilbert Harman in \emph{Change in View}. There Harman
endorses the following principle. (At least he endorses it as true -- he
doesn't seem to think it is particularly explanatory because it is a
special case of a more general interesting principle.)

\textbf{Recognized Logical Implication Principle}\\
One has reason to believe \emph{p} if one \emph{recognizes} that
\emph{p} is logically implied by one's view.
(\citeproc{ref-Harman1986}{Harman 1986, 17})

This seems right to me, both what it says and its implicature that the
reason in question is not a conclusive reason. My main objection to
those who use the preface paradox to argue against closure is that they
give us a mistaken picture of what we have \emph{to do} epistemically.
When I have inconsistent beliefs, or I don't believe some consequence of
my beliefs, that is something I have a reason to deal with at some
stage, something I have to do. When we say that we have things to do, we
don't mean that we have to do them \emph{right now}, or instead of
everything else. My current list of things to do includes cleaning my
bathroom, yet here I am writing this paper, and (given the relevant
deadlines) rightly so. We can have the job of cleaning up our epistemic
house as something to do while recognising that we can quite rightly do
other things first. But it's a serious mistake to infer from the
permissibility of doing other things that cleaning up our epistemic
house (or our bathroom) isn't something to be done. The bathroom won't
clean itself after all, and eventually this becomes a problem.

There is a possible complication when it comes to tasks that are very
low priority. My attic is to be cleaned, or at least it could be
cleaner, but there are no imaginable circumstances under which something
else wouldn't be higher priority. Given that, should we really leave
\emph{clean the attic} on the list of things to be done? Similarly,
there might be implications I haven't followed through that it couldn't
possibly be worth my time to sort out. Are they things to be done? I
think it's worthwhile recording them as such, because otherwise we might
miss opportunities to deal with them in the process of doing something
else. I don't need to put off anything else in order to clean the attic,
but if I'm up there for independent reasons I should bring down some of
the garbage. Similarly, I don't need to follow through implications
mostly irrelevant to my interests, but if those propositions come up for
independent reasons, I should deal with the fact that some things I
believe imply something I don't believe. Having it be the case that all
implications from things we believe to things we don't believe
constitute jobs to do (possibly in the loose sense that cleaning my
attic is something to do) has the right implications for what epistemic
duties we do and don't have.

While waxing metaphorical, it seems time to pull out a rather helpful
metaphor that Gilbert Ryle (\citeproc{ref-Ryle1949}{1949}) develops in
\emph{The Concept of Mind} at a point where he's covering what we'd now
call the inference/implication distinction. (This is a large theme of
chapter 9, see particularly pages 292-309.) Ryle's point in these
passages, as it frequently is throughout the book, is to stress that
minds are fundamentally active, and the activity of a mind cannot be
easily recovered from its end state. Although Ryle doesn't use this
language, his point is that we shouldn't confuse the difficult activity
of drawing inferences with the smoothness and precision of a logical
implication. The language Ryle does use is more picturesque. He compares
the easy work a farmer does when sauntering down a path from the hard
work he did when building the path. A good argument, in philosophy or
mathematics or elsewhere, is like a well made path that permits
sauntering from the start to finish without undue strain. But from that
it doesn't follow that the task of coming up with that argument, of
building that path in Ryle's metaphor, was easy work. The easiest paths
to walk are often the hardest to build. Path-building, smoothing out our
beliefs so they are consistent and closed under implication, is hard
work, even when the finished results look clean and straightforward. Its
work that we shouldn't do unless we need to. But making sure our beliefs
are closed under entailment even with respect to irrelevant propositions
is suspiciously like the activity of buildings paths between points
without first checking you need to walk between them.

For a less metaphorical reason for doubting the wisdom of this unchecked
commitment to closure, we might notice the difficulties theorists tend
to get into all sorts of difficulties. Consider, for example, the view
put forward by Mark Kaplan in \emph{Decision Theory as Philosophy}. Here
is his definition of belief.

\begin{quote}
You count as believing P just if, were your sole aim to assert the truth
(as it pertains to P), and you only options were to assert that P,
assert that \(\neg\)P or make neither assertion, you would prefer to
assert that P. (109)
\end{quote}

Kaplan notes that conditional definitions like this are prone to Shope's
conditional fallacy. If my sole aim were to assert the truth, I might
have different beliefs to what I now have. He addresses one version of
this objection (namely that it appears to imply that everyone believes
their sole desire is to assert the truth) but as we'll see presently he
can't avoid all versions of it.

These arguments are making me thirsty. I'd like a beer. Or at least I
think I would. But wait! On Kaplan's theory I can't think that I'd like
a beer, for if my sole aim were to assert the truth as it pertains to my
beer-desires, I wouldn't have beer desires. And then I'd prefer to
assert that I wouldn't like a beer, I'd merely like to assert the truth
as it pertains to my beer desires.

Even bracketing this concern, Kaplan ends up being committed to the view
that I can (coherently!) believe that \emph{p} even while regarding
\emph{p} as highly improbable. This looks like a refutation of the view
to me, but Kaplan accepts it with some equanimity. He has two primary
reasons for saying we should live with this. First, he says that it only
looks like an absurd consequence if we are committed to the Threshold
View. To this all I can say is that \emph{I} don't believe the Threshold
View, but it still seems absurd to me. Second, he says that any view is
going to have to be revisionary to some extent, because our ordinary
concept of belief is not ``coherent'' (142). His view is that, ``Our
ordinary notion of belief both construes belief as a state of confidence
short of certainty and takes consistency of belief to be something that
is at least possible and, perhaps, even desirable'' and this is
impossible. I think the view here interprets belief as a state less than
confidence and allows for as much consistency as the folk view does
(i.e.~consistency amongst salient propositions), so this defence is
unsuccessful as well.

None of the arguments here in favour of our restrictions on closure are
completely conclusive. In part the argument at this stage rests on the
lack of a plausible rival theory that doesn't interpret belief as
certainty but implements a stronger closure principle. It's possible
that tomorrow someone will come up with a theory that does just this.
Until then, we'll stick with the account here, and see what its
epistemological implications might be.

\section{Examples of Pragmatic
Encroachment}\label{examples-of-pragmatic-encroachment}

Fantl and McGrath's case for pragmatic encroachment starts with cases
like the following. (The following case is not quite theirs, but is
similar enough to suit their plan, and easier to explain in my
framework.)

\begin{quote}
\emph{Local and Express}

There are two kinds of trains that run from the city to the suburbs: the
local, which stops at all stations, and the express, which skips the
first eight stations. Harry and Louise want to go to the fifth station,
so they shouldn't catch the Express. Though if they do it isn't too hard
to catch a local back the other way, so it isn't usually a large cost.
Unfortunately, the trains are not always clearly labelled. They see a
particular train about to leave. If it's a local they are better off
catching it, if it is an express they should wait for the next local,
which they can see is already boarding passengers and will leave in a
few minutes. While running towards the train, they hear a fellow
passenger say ``It's a local.'' This gives them good, but far from
overwhelming, reason to believe that the train is a local. Passengers
get this kind of thing wrong fairly frequently, but they don't have time
to get more information. So each of them face a gamble, which they can
take by getting on the train. If the train is a local, they will get
home a few minutes early. If it is an express they will get home a few
minutes later. For Louise, this is a low stakes gamble, as nothing much
turns on whether she is a few minutes early or late, but she does have a
weak preference for arriving earlier rather than later. But for Harry it
is a high stakes gamble, because if he is late he won't make the start
of his daughter's soccer game, which will highly upset her. There is no
large payoff for Harry arriving early.
\end{quote}

What should each of them do? What should each of them believe?

The first question is relatively easy. Louise should catch the train,
and Harry should wait for the next. For each of them that's the utility
maximising thing to do. The second one is harder. Fantl and McGrath
suggest that, despite being in the same epistemic position with respect
to everything except their interests, Louise is justified in believing
the train is a local and Harry is not. I agree. (If you don't think the
particular case fits this pattern, feel free to modify it so the
difference in interests grounds a difference in what they are justified
in believing.) Does this show that our notion of epistemic justification
has to be pragmatically sensitive? I'll argue that it does not.

The fundamental assumption I'm making is that what is primarily subject
to epistemic evaluation are degrees of belief, or what are more commonly
called states of confidence in ordinary language. When we think about
things this way, we see that Louise and Harry are justified in adopting
\emph{the very same degrees of belief}. Both of them should be
confident, but not absolutely certain, that the train is a local. We
don't have even the appearance of a counterxample to Probabilistic
Evidentialism here. If we like putting this in numerical terms, we could
say that each of them is justified in assigning a probability of around
0.9 to the proposition \emph{That train is a local}.\footnote{I think
  putting things numerically is misleading because it suggests that the
  kind of bets we usually use to measure degrees of belief are open,
  salient options for Louise and Harry. But if those bets were open and
  salient, they wouldn't \emph{believe} the train is a local. Using
  qualitative rather than quantitative language to describe them is just
  as accurate, and doesn't have misleading implications about their
  practical environment.} So as long as we adopt a Probability First
epistemology, where we in the first instance evaluate the probabilities
that agents assign to propositions, Harry and Louise are evaluated alike
iff they do the same thing.

How then can we say that Louise alone is justified in believing that the
train is a local? Because that state of confidence they are justified in
adopting, the state of being fairly confident but not absolutely certain
that the train is a local, counts as believing that the train is a local
given Louise's context but not Harry's context. Once Louise hears the
other passenger's comment, conditionalising on \emph{That's a local}
doesn't change any of her preferences over open, salient actions,
including such `actions' as believing or disbelieving propositions. But
conditional on the train being a local, Harry prefers catching the
train, which he actually does not prefer.

In cases like this, interests matter not because they affect the degree
of confidence that an agent can reasonably have in a proposition's
truth. (That is, not because they matter to epistemology.) Rather,
interests matter because they affect whether those reasonable degrees of
confidence amount to belief. (That is, because they matter to philosophy
of mind.) There is no reason here to let pragmatic concerns into
epistemology.

\section{Justification and Practical
Reasoning}\label{justification-and-practical-reasoning}

The discussion in the last section obviously didn't show that there is
no encroachment of pragmatics into epistemology. There are, in
particular, two kinds of concerns one might have about the prospects for
extending my style of argument to block all attempts at pragmatic
encroachment. The biggest concern is that it might turn out to be
impossible to defend a Probability First epistemology, particularly if
we do not allow ourselves pragmatic concerns. For instance, it is
crucial to this project that we have a notion of evidence that is not
defined in terms of traditional epistemic concepts (e.g.~as knowledge),
or in terms of interests. This is an enormous project, and I'm not going
to attempt to tackle it here. The second concern is that we won't be
able to generalise the discussion of that example to explain the
plausibility of (JP) without conceding something to the defenders of
pragmatic encroachment.

\begin{description}
\tightlist
\item[(JP)]
If \emph{S} justifiably believes that \emph{p}, then \emph{S} is
justified in using \emph{p} as a premise in practical reasoning.
\end{description}

And that's what we will look at in this section. To start, we need to
clarify exactly what (JP) means. Much of this discussion will be
indebted to Fantl and McGrath's discussion of various ways of making
(JP) more precise. To see some of the complications at issue, consider a
simple case of a bet on a reasonably well established historical
proposition. The agent has a lot of evidence that supports \emph{p}, and
is offered a bet that returns \$1 if \emph{p} is true, and loses \$500
if \emph{p} is false. Since her evidence doesn't support \emph{that}
much confidence in \emph{p}, she properly declines the bet. One might
try to reason intuitively as follows. Assume that she justifiably
believed that \emph{p}. Then she'd be in a position to make the
following argument.

\begin{quote}
\emph{p}\\
If \emph{p}, then I should take the bet\\
So, I should take the bet
\end{quote}

Since she isn't in a position to draw the conclusion, she must not be in
a position to endorse both of the premises. Hence (arguably) she isn't
justified in believing that \emph{p}. But we have to be careful here. If
we assume also that \emph{p} is true (as Fantl and McGrath do, because
they are mostly concerned with knowledge rather than justified belief),
then the second premise is clearly false, since it is a conditional with
a true antecedent and a false consequent. So the fact that she can't
draw the conclusion of this argument only shows that she can't endorse
\emph{both} of the premises, and that's not surprising since one of the
premises is most likely false. (I'm not assuming here that the
conditional is true iff it has a true antecendent or a false consequent,
just that it is only true if it has a false antecedent or a true
consequent.)

In order to get around this problem, Fantl and McGrath suggest a few
other ways that our agent might reason to the bet. They suggest each of
the following principles.

\begin{quote}
S knows that p only if, for any act A, if \emph{S} knows that if p, then
A is the best thing she can do, then \emph{S} is rational to do A. (72)

\emph{S} knows that p only if, for any states of affairs A and B, if
\(S\) knows that if p, then A is better for her than B, then \emph{S} is
rational to prefer A to B. (74)

\textbf{(PC)} \emph{S} is justified in believing that p only if \emph{S}
is rational to prefer as if p.~(77)
\end{quote}

Hawthorne (\citeproc{ref-Hawthorne2004}{2004, 174--81}) appears to
endorse the second of these principles. He considers an agent who
endorses the following implication concerning a proposed sell of a
lottery ticket for a cent, which is well below its actuarially fair
value.

\begin{quote}
I will lose the lottery.\\
If I keep the ticket, I will get nothing.\\
If I sell the ticket, I will get a cent.\\
So I ought to sell the ticket. (174)
\end{quote}

(To make this fully explicit, it helps to add the tacit premise that a
cent is better than nothing.) Hawthorne says that this is intuitively a
\emph{bad} argument, and concludes that the agent who attempts to use it
is not in a position to know its first premise. But that conclusion only
follows if we assume that the argument form is acceptable. So it is
plausible to conclude that he endorses Fantl and McGrath's second
principle.

The interesting question here is whether the theory endorsed in this
paper can validate the true principles that Fantl and McGrath
articulate. (Or, more precisely, we can validate the equivalent true
principles concerning justified belief, since knowledge is outside the
scope of the paper.) I'll argue that it can in the following way. First,
I'll just note that given the fact that the theory here implies the
closure principles we outlined in section 5, we can easily enough
endorse Fantl and McGrath's first two principles. This is good, since
they seem true. The longer part of the argument involves arguing that
their principle (PC), which doesn't hold on the theory endorsed here, is
in fact incorrect.

One might worry that the qualification on the closure principles in
section 5 mean that we can't fully endorse the principles Fantl and
McGrath endorse. In particular, it might be worried that there could be
an agent who believes that \emph{p}, believes that if \emph{p}, then A
is better than B, but doesn't put these two beliefs together to infer
that A is better than B. This is certainly a possibility given the
qualifications listed above. But note that in this position, if those
two beliefs were justified, the agent would certainly be \emph{rational}
to conclude that A is better than B, and hence rational to prefer A to
B. So the constraints on the closure principles don't affect our ability
to endorse these two principles.

The real issue is (PC). Fantl and McGrath offer a lot of cases where
(PC) holds, as well as arguing that it is plausibly true given the role
of implications in practical reasoning. What's at issue is that (PC) is
stronger than a deductive closure principle. It is, in effect,
equivalent to endorsing the following schema as a valid principle of
implication.

\begin{quote}
\emph{p}\\
Given \emph{p}, A is preferable to B\\
So, A is preferable to B
\end{quote}

I call this Practical Modus Ponens, or PMP. The middle premise in PMP is
\emph{not} a conditional. It is not to be read as \emph{If p, then A is
preferable to B}. Conditional valuations are not conditionals. To see
this, again consider the proposed bet on (true) \emph{p} at exorbitant
odds, where A is the act of taking the bet, and B the act of declining
the bet. It's true that given \emph{p}, A is preferable to B. But it's
not true that if \emph{p}, then A is preferable to B. Even if we
restrict our attention to cases where the preferences in question are
perfectly valid, this is a case where PMP is invalid. Both premises are
true, and the conclusion is false. It might nevertheless be true that
whenever an agent is justified in believing both of the premises, she is
justified in believing the conclusion. To argue against this, we need a
\emph{very} complicated case, involving embedded bets and three separate
agents, Quentin, Robby and Thom. All of them have received the same
evidence, and all of them are faced with the same complex bet, with the
following properties.

\begin{itemize}
\tightlist
\item
  \emph{p} is an historical proposition that is well (but not
  conclusively) supported by their evidence, and happens to be true. All
  the agents have a high credence in \emph{p}, which is exactly what the
  evidence supports.
\item
  The bet A, which they are offered, wins if \emph{p} is true, and loses
  if \emph{p} is false.
\item
  If they win the bet, the prize is the bet B.
\item
  \emph{S} is also an historical proposition, but the evidence tells
  equally for and against it. All the agents regard \emph{S} as being
  about as likely as not. Moreover, \emph{S} turns out to be false.
\item
  The bet B is worth \$2 if \emph{S} is true, and worth -\$1 if \emph{S}
  is false. Although it is actually a losing bet, the agents all
  rationally value it at around 50 cents.
\item
  How much A costs is determined by which proposition from the partition
  \{\(q, r, s\)\} is true.
\item
  If \(q\) is true, A costs \$2
\item
  If \(r\) is true, A costs \$500
\item
  If \emph{t} is true, A costs \$1
\item
  The evidence the agents has strongly supports \(r\), though \emph{t}
  is in fact true
\item
  Quentin believes \(q\)
\item
  Robby believes \(r\)
\item
  Thom believes \emph{t}
\end{itemize}

All of the agents make the utility calculations that their beliefs
support, so Quentin and Thom take the bet and lose a dollar, while Robby
declines it. Although Robby has a lot of evidence in favour of \emph{p},
he correctly decides that it would be unwise to bet on \emph{p} at
effective odds of 1000 to 1 against. I'll now argue that both Quentin
and Thom are potential counterexamples to (PC). There are three
possibilities for what we can say about those two.

First, we could say that they are justified in believing \emph{p}, and
rational to take the bet. The problem with this position is that if they
had rational beliefs about the partition \{\(q, r, t\)\} they would
realise that taking the bet does not maximise expected utility. If we
take rational decisions to be those that maximise expected utility given
a rational response to the evidence, then the decisions are clearly not
rational.

Second, we could say that although Quentin and Thom are not rational in
accepting the bet, nor are they justified in believing that \emph{p}.
This doesn't seem particularly plausible for several reasons. The
irrationality in their belief systems concerns whether \(q, r\) or
\emph{t} is true, not whether \emph{p} is true. If Thom suddenly got a
lot of evidence that \emph{t} is true, then all of his (salient) beliefs
would be well supported by the evidence. But it is bizarre to think that
whether his belief in \emph{p} is rational turns on how much evidence he
has for \emph{t}. Finally, even if we accept that agents in higher
stakes situations need more evidence to have justified beliefs, the fact
is that the agents are in a low-risk situation, since \emph{t} is
actually true, so the most they could lose is \$1.

So it seems like the natural thing to say is that Quentin and Thom
\emph{are} justified in believing that \emph{p}, and are justified in
believing that given \emph{p}, it maximises expected utility to take the
bet, but they are not rational to take the bet. (At least, in the
version of the story where they are thinking about which of \(q, r\) and
\emph{t} are correct given their evidence when thinking about whether to
take the bet they are counterexamples to (PC).) Against this, one might
respond that if belief in \emph{p} is justified, there are arguments one
might make to the conclusion that the bet should be taken. So it is
inconsistent to say that the belief is justified, but the decision to
take the bet is not rational. The problem is finding a premise that goes
along with \emph{p} to get the conclusion that taking the bet is
rational. Let's look at some of the premises the agent might use.

\begin{itemize}
\tightlist
\item
  If \emph{p}, then the best thing to do is to take the bet.
\end{itemize}

This isn't true (\emph{p} is true, but the best thing to do isn't to
take the bet). More importantly, the agents think this is only true if
\emph{S} is true, and they think \emph{S} is a 50/50 proposition. So
they don't believe this premise, and it would not be rational to believe
it.

\begin{itemize}
\tightlist
\item
  If \emph{p}, then probably the best thing to do is to take the bet.
\end{itemize}

Again this isn't true, and it isn't well supported, and it doesn't even
support the conclusion, for it doesn't follow from the fact that
\emph{x} is probably the best thing to do that \emph{x} should be done.

\begin{itemize}
\tightlist
\item
  If \emph{p}, then taking the bet maximises rational expected utility.
\end{itemize}

This isn't true -- it is a conditional with a true antecedent and a
false consequent. Moreover, if Quentin and Thom were rational, like
Robby, they would recognise this.

\begin{itemize}
\tightlist
\item
  If \emph{p}, then taking the bet maximises expected utility relative
  to their beliefs.
\end{itemize}

This is true, and even reasonable to believe, but it doesn't imply that
they should take the bet. It doesn't follow from the fact that doing
something maximises expected utility relative to my crazy beliefs that I
should do that thing.

\begin{itemize}
\tightlist
\item
  Given \emph{p}, taking the bet maximises rational expected utility.
\end{itemize}

This is true, and even reasonable to believe, but it isn't clear that it
supports the conclusion that the agents should take the bet. The
implication appealed to here is PMP, and in this context that's close
enough to equivalent to (PC). If we think that this case is a prima
facie problem for (PC), as I think is intuitively plausible, then we
can't use (PC) to show that it \emph{doesn't} post a problem. We could
obviously continue for a while, but it should be clear it will be very
hard to find a way to justify taking the bet even spotting the agents
\emph{p} as a premise they can use in rational deliberation. So it seems
to me that (PC) is not in general true, which is good because as we'll
see in cases like this one the theory outlined here does not support it.

The theory we have been working with says that belief that \emph{p} is
justified iff the agent's degree of belief in \emph{p} is sufficient to
amount to belief in their context, and they are justified in believing
\emph{p} to that degree. Since by hypothesis Quentin and Thom are
justified in believing \emph{p} to the degree that they do, the only
question left is whether this amounts to belief. This turns out not to
be settled by the details of the case as yet specified. At first glance,
assuming there are no other relevant decisions, we might think they
believe that \emph{p} because (a) they prefer (in the relevant sense)
believing \emph{p} to not believing \emph{p}, and (b) conditionalising
on \emph{p} doesn't change their attitude towards the bet. (They prefer
taking the bet to declining it, both unconditionally and conditional on
\emph{p}.)

But that isn't all there is to the definition of belief \emph{tout
court}. We must also ask whether conditionalising on \emph{p} changes
any preferences conditional on any active proposition. And that may well
be true. Conditional on \(r\), Quentin and Thom prefer not taking the
bet to taking it. But conditional on \(r\) and \emph{p}, they prefer
taking the bet to not taking it. So if \(r\) is an active proposition,
they don't believe that \emph{p}. If \(r\) is not active, they do
believe it. In more colloquial terms, if they are concerned about the
possible truth of \(r\) (if it is salient, or at least not taken for
granted to be false) then \emph{p} becomes a potentially high-stakes
proposition, so they don't believe it without extraordinary evidence
(which they don't have). Hence they are only a counterexample to (PC) if
\(r\) is not active. But if \(r\) is not active, our theory predicts
that they are a counterexample to (PC), which is what we argued above is
intuitively correct.

Still, the importance of \(r\) suggests a way of saving (PC). Above I
relied on the position that if Quentin and Thom are not maximising
rational expected utility, then they are being irrational. This is
perhaps too harsh. There is a position we could take, derived from some
suggestions made by Gilbert Harman in \emph{Change in View}, that an
agent can rationally rely on their beliefs, even if those beliefs were
not rationally formed, if they cannot be expected to have kept track of
the evidence they used to form that belief. If we adopt this view, then
we might be able to say that (PC) is compatible with the correct
normative judgments about this case.

To make this compatibility explicit, let's adjust the case so Quentin
takes \(q\) for granted, and cannot be reasonably expected to have
remembered the evidence for \(q\). Thom, on the other hand, forms the
belief that \emph{t} rather than \(r\) is true in the course of thinking
through his evidence that bears on the rationality of taking or
declining the bet. (In more familiar terms, \emph{t} is part of the
inference Thom uses in coming to conclude that he should take the bet,
though it is not part of the final implication he endorses whose
conclusion is that he should take the bet.) Neither Quentin nor Thom is
a counterexample to (PC) thus understood. (That is, with the notion of
rationality in (PC) understood as Harman suggests that it should be.)
Quentin is not a counterexample, because he is \emph{rational} in taking
the bet. And Thom is not a counterexample, because in his context, where
\(r\) is active, his credence in \emph{p} does not amount to belief in
\emph{p}, so he is not justified in believing \emph{p}.

We have now two readings of (PC). On the strict reading, where a
rational choice is one that maximises rational expected utility, the
principle is subject to counterexample, and seems generally to be
implausible. On the loose reading, where we allow agents to rely on
beliefs formed irrationally in the past in rational decision making,
(PC) \emph{is} plausible. Happily, the theory sketched here agrees with
(PC) on the plausible loose reading, but not on the implausible strict
reading. In the previous section I argued that the theory also accounts
for intuitions about particular cases like \emph{Local and Express}. And
now we've seen that the theory accounts for our considered opinions
about which principles connecting justified belief to rational decision
making we should endorse. So it seems at this stage that we can account
for the intuitions behind the pragmatic encroachment view while keeping
a concept of probabilistic epistemic justification that is free of
pragmatic considerations.

\section{Conclusions}\label{conclusions}

Given a pragmatic account of belief, we don't need to have a pragmatic
account of justification in order to explain the intuitions that whether
\(S\) justifiably believes that \emph{p} might depend on pragmatic
factors. My focus here has been on sketching a theory of belief on which
it is the belief part of the concept of a justified belief which is
pragmatically sensitive. I haven't said much about why we should prefer
to take that option than say that the notion of epistemic justification
is a pragmatic notion. I've mainly been aiming to show that a particular
position is an open possibility, namely that we can accept that whether
a particular agent is justified in believing \emph{p} can be sensitive
to their practical environment without thinking that the primary
epistemic concepts are themselves pragmatically sensitive.\footnote{Thanks
  to Michael Almeida, Tamar Szabó Gendler, Peter Gerdes, Jon Kvanvig,
  Barry Lam, Ishani Maitra, Robert Stalnaker, Jason Stanley, Matthew
  Weiner for helpful discussions, and especially to Matthew McGrath for
  correcting many mistakes in an earlier draft of this paper.}

\subsection*{References}\label{references}
\addcontentsline{toc}{subsection}{References}

\phantomsection\label{refs}
\begin{CSLReferences}{1}{0}
\bibitem[\citeproctext]{ref-Bovens1999}
Bovens, Luc, and James Hawthorne. 1999. {``The Preface, the Lottery, and
the Logic of Belief.''} \emph{Mind} 108 (430): 241--64. doi:
\href{https://doi.org/10.1093/mind/108.430.241}{10.1093/mind/108.430.241}.

\bibitem[\citeproctext]{ref-Christensen2005}
Christensen, David. 2005. \emph{Putting Logic in Its Place}. Oxford:
Oxford University Press.

\bibitem[\citeproctext]{ref-Evnine1999}
Evnine, Simon. 1999. {``Believing Conjunctions.''} \emph{Synthese} 118:
201--27. doi:
\href{https://doi.org/10.1023/A:1005114419965}{10.1023/A:1005114419965}.

\bibitem[\citeproctext]{ref-Fantl2002}
Fantl, Jeremy, and Matthew McGrath. 2002. {``Evidence, Pragmatics, and
Justification.''} \emph{Philosophical Review} 111: 67--94. doi:
\href{https://doi.org/10.2307/3182570}{10.2307/3182570}.

\bibitem[\citeproctext]{ref-Foley1993}
Foley, Richard. 1993. \emph{Working Without a Net}. Oxford: Oxford
University Press.

\bibitem[\citeproctext]{ref-Harman1986}
Harman, Gilbert. 1986. \emph{Change in View}. Cambridge, MA: Bradford.

\bibitem[\citeproctext]{ref-Hawthorne2004}
Hawthorne, John. 2004. \emph{Knowledge and Lotteries}. Oxford: Oxford
University Press.

\bibitem[\citeproctext]{ref-Hunter1996}
Hunter, Daniel. 1996. {``On the Relation Between Categorical and
Probabilistic Belief.''} \emph{No{û}s} 30: 75--98. doi:
\href{https://doi.org/10.2307/2216304}{10.2307/2216304}.

\bibitem[\citeproctext]{ref-Kaplan1996}
Kaplan, Mark. 1996. \emph{Decision Theory as Philosophy}. Cambridge:
Cambridge University Press.

\bibitem[\citeproctext]{ref-Keynes1921}
Keynes, John Maynard. 1921. \emph{Treatise on Probability}. London:
Macmillan.

\bibitem[\citeproctext]{ref-Lewis1994b}
Lewis, David. 1994. {``Reduction of Mind.''} In \emph{A Companion to the
Philosophy of Mind}, edited by Samuel Guttenplan, 412--31. Oxford:
Blackwell. doi:
\href{https://doi.org/10.1017/CBO9780511625343.019}{10.1017/CBO9780511625343.019}.
Reprinted in his \emph{Papers in Metaphysics and Epistemology}, 1999,
291-324, Cambridge: Cambridge University Press. References to reprint.

\bibitem[\citeproctext]{ref-MaitraANG}
Maitra, Ishani. 2010. {``Assertion, Norms and Games.''} In
\emph{Assertion: New Philosophical Essays}, edited by Jessica Brown and
Herman Cappelen, 277--96. Oxford: Oxford University Press.

\bibitem[\citeproctext]{ref-Ryle1949}
Ryle, Gilbert. 1949. \emph{The Concept of Mind}. New York: Barnes;
Noble.

\bibitem[\citeproctext]{ref-Stalnaker1984}
Stalnaker, Robert. 1984. \emph{Inquiry}. Cambridge, MA: MIT Press.

\bibitem[\citeproctext]{ref-Stanley2005-STAKAP}
Stanley, Jason. 2005. \emph{{Knowledge and Practical Interests}}. Oxford
University Press.

\bibitem[\citeproctext]{ref-Weatherson2005-WEATTT}
Weatherson, Brian. 2005. {``{True, Truer, Truest}.''}
\emph{Philosophical Studies} 123 (1-2): 47--70. doi:
\href{https://doi.org/10.1007/s11098-004-5218-x}{10.1007/s11098-004-5218-x}.

\bibitem[\citeproctext]{ref-Williamson1994-WILV}
Williamson, Timothy. 1994. \emph{{Vagueness}}. Routledge.

\bibitem[\citeproctext]{ref-Williamson2000-WILKAI}
---------. 2000. \emph{{Knowledge and its Limits}}. Oxford University
Press.

\end{CSLReferences}



\noindent Published in\emph{
Philosophical Perspectives}, 2005, pp. 417-443.

\end{document}
