% Options for packages loaded elsewhere
% Options for packages loaded elsewhere
\PassOptionsToPackage{unicode}{hyperref}
\PassOptionsToPackage{hyphens}{url}
%
\documentclass[
  11pt,
  letterpaper,
  DIV=11,
  numbers=noendperiod,
  twoside]{scrartcl}
\usepackage{xcolor}
\usepackage[left=1.1in, right=1in, top=0.8in, bottom=0.8in,
paperheight=9.5in, paperwidth=7in, includemp=TRUE, marginparwidth=0in,
marginparsep=0in]{geometry}
\usepackage{amsmath,amssymb}
\setcounter{secnumdepth}{3}
\usepackage{iftex}
\ifPDFTeX
  \usepackage[T1]{fontenc}
  \usepackage[utf8]{inputenc}
  \usepackage{textcomp} % provide euro and other symbols
\else % if luatex or xetex
  \usepackage{unicode-math} % this also loads fontspec
  \defaultfontfeatures{Scale=MatchLowercase}
  \defaultfontfeatures[\rmfamily]{Ligatures=TeX,Scale=1}
\fi
\usepackage{lmodern}
\ifPDFTeX\else
  % xetex/luatex font selection
  \setmainfont[ItalicFont=EB Garamond Italic,BoldFont=EB Garamond
Bold]{EB Garamond Math}
  \setsansfont[]{EB Garamond}
  \setmathfont[]{Garamond-Math}
\fi
% Use upquote if available, for straight quotes in verbatim environments
\IfFileExists{upquote.sty}{\usepackage{upquote}}{}
\IfFileExists{microtype.sty}{% use microtype if available
  \usepackage[]{microtype}
  \UseMicrotypeSet[protrusion]{basicmath} % disable protrusion for tt fonts
}{}
\usepackage{setspace}
% Make \paragraph and \subparagraph free-standing
\makeatletter
\ifx\paragraph\undefined\else
  \let\oldparagraph\paragraph
  \renewcommand{\paragraph}{
    \@ifstar
      \xxxParagraphStar
      \xxxParagraphNoStar
  }
  \newcommand{\xxxParagraphStar}[1]{\oldparagraph*{#1}\mbox{}}
  \newcommand{\xxxParagraphNoStar}[1]{\oldparagraph{#1}\mbox{}}
\fi
\ifx\subparagraph\undefined\else
  \let\oldsubparagraph\subparagraph
  \renewcommand{\subparagraph}{
    \@ifstar
      \xxxSubParagraphStar
      \xxxSubParagraphNoStar
  }
  \newcommand{\xxxSubParagraphStar}[1]{\oldsubparagraph*{#1}\mbox{}}
  \newcommand{\xxxSubParagraphNoStar}[1]{\oldsubparagraph{#1}\mbox{}}
\fi
\makeatother


\usepackage{longtable,booktabs,array}
\usepackage{calc} % for calculating minipage widths
% Correct order of tables after \paragraph or \subparagraph
\usepackage{etoolbox}
\makeatletter
\patchcmd\longtable{\par}{\if@noskipsec\mbox{}\fi\par}{}{}
\makeatother
% Allow footnotes in longtable head/foot
\IfFileExists{footnotehyper.sty}{\usepackage{footnotehyper}}{\usepackage{footnote}}
\makesavenoteenv{longtable}
\usepackage{graphicx}
\makeatletter
\newsavebox\pandoc@box
\newcommand*\pandocbounded[1]{% scales image to fit in text height/width
  \sbox\pandoc@box{#1}%
  \Gscale@div\@tempa{\textheight}{\dimexpr\ht\pandoc@box+\dp\pandoc@box\relax}%
  \Gscale@div\@tempb{\linewidth}{\wd\pandoc@box}%
  \ifdim\@tempb\p@<\@tempa\p@\let\@tempa\@tempb\fi% select the smaller of both
  \ifdim\@tempa\p@<\p@\scalebox{\@tempa}{\usebox\pandoc@box}%
  \else\usebox{\pandoc@box}%
  \fi%
}
% Set default figure placement to htbp
\def\fps@figure{htbp}
\makeatother


% definitions for citeproc citations
\NewDocumentCommand\citeproctext{}{}
\NewDocumentCommand\citeproc{mm}{%
  \begingroup\def\citeproctext{#2}\cite{#1}\endgroup}
\makeatletter
 % allow citations to break across lines
 \let\@cite@ofmt\@firstofone
 % avoid brackets around text for \cite:
 \def\@biblabel#1{}
 \def\@cite#1#2{{#1\if@tempswa , #2\fi}}
\makeatother
\newlength{\cslhangindent}
\setlength{\cslhangindent}{1.5em}
\newlength{\csllabelwidth}
\setlength{\csllabelwidth}{3em}
\newenvironment{CSLReferences}[2] % #1 hanging-indent, #2 entry-spacing
 {\begin{list}{}{%
  \setlength{\itemindent}{0pt}
  \setlength{\leftmargin}{0pt}
  \setlength{\parsep}{0pt}
  % turn on hanging indent if param 1 is 1
  \ifodd #1
   \setlength{\leftmargin}{\cslhangindent}
   \setlength{\itemindent}{-1\cslhangindent}
  \fi
  % set entry spacing
  \setlength{\itemsep}{#2\baselineskip}}}
 {\end{list}}
\usepackage{calc}
\newcommand{\CSLBlock}[1]{\hfill\break\parbox[t]{\linewidth}{\strut\ignorespaces#1\strut}}
\newcommand{\CSLLeftMargin}[1]{\parbox[t]{\csllabelwidth}{\strut#1\strut}}
\newcommand{\CSLRightInline}[1]{\parbox[t]{\linewidth - \csllabelwidth}{\strut#1\strut}}
\newcommand{\CSLIndent}[1]{\hspace{\cslhangindent}#1}



\setlength{\emergencystretch}{3em} % prevent overfull lines

\providecommand{\tightlist}{%
  \setlength{\itemsep}{0pt}\setlength{\parskip}{0pt}}



 


\setlength\heavyrulewidth{0ex}
\setlength\lightrulewidth{0ex}
\usepackage[automark]{scrlayer-scrpage}
\clearpairofpagestyles
\cehead{
  Brian Weatherson
  }
\cohead{
  Against Ex Ante Pareto
  }
\ohead{\bfseries \pagemark}
\cfoot{}
\makeatletter
\newcommand*\NoIndentAfterEnv[1]{%
  \AfterEndEnvironment{#1}{\par\@afterindentfalse\@afterheading}}
\makeatother
\NoIndentAfterEnv{itemize}
\NoIndentAfterEnv{enumerate}
\NoIndentAfterEnv{description}
\NoIndentAfterEnv{quote}
\NoIndentAfterEnv{equation}
\NoIndentAfterEnv{longtable}
\NoIndentAfterEnv{abstract}
\renewenvironment{abstract}
 {\vspace{-1.25cm}
 \quotation\small\noindent\emph{Abstract}:}
 {\endquotation}
\newfontfamily\tfont{EB Garamond}
\addtokomafont{disposition}{\rmfamily}
\addtokomafont{title}{\normalfont\itshape}
\let\footnoterule\relax
\KOMAoption{captions}{tableheading}
\makeatletter
\@ifpackageloaded{caption}{}{\usepackage{caption}}
\AtBeginDocument{%
\ifdefined\contentsname
  \renewcommand*\contentsname{Table of contents}
\else
  \newcommand\contentsname{Table of contents}
\fi
\ifdefined\listfigurename
  \renewcommand*\listfigurename{List of Figures}
\else
  \newcommand\listfigurename{List of Figures}
\fi
\ifdefined\listtablename
  \renewcommand*\listtablename{List of Tables}
\else
  \newcommand\listtablename{List of Tables}
\fi
\ifdefined\figurename
  \renewcommand*\figurename{Figure}
\else
  \newcommand\figurename{Figure}
\fi
\ifdefined\tablename
  \renewcommand*\tablename{Table}
\else
  \newcommand\tablename{Table}
\fi
}
\@ifpackageloaded{float}{}{\usepackage{float}}
\floatstyle{ruled}
\@ifundefined{c@chapter}{\newfloat{codelisting}{h}{lop}}{\newfloat{codelisting}{h}{lop}[chapter]}
\floatname{codelisting}{Listing}
\newcommand*\listoflistings{\listof{codelisting}{List of Listings}}
\makeatother
\makeatletter
\makeatother
\makeatletter
\@ifpackageloaded{caption}{}{\usepackage{caption}}
\@ifpackageloaded{subcaption}{}{\usepackage{subcaption}}
\makeatother
\usepackage{bookmark}
\IfFileExists{xurl.sty}{\usepackage{xurl}}{} % add URL line breaks if available
\urlstyle{same}
\hypersetup{
  pdftitle={Against Ex Ante Pareto},
  pdfauthor={Brian Weatherson},
  hidelinks,
  pdfcreator={LaTeX via pandoc}}


\title{Against Ex Ante Pareto}
\author{Brian Weatherson}
\date{2025}
\begin{document}
\maketitle
\begin{abstract}
Given two lotteries over social outcomes, does the fact that everyone in
the society thinks that one is at least as good as the other suffice to
conclude that the first is at least as good as the other? I'm going to
argue that, somewhat surprisingly, the answer is no. The principle that
it does always suffice is called the Ex Ante Pareto principle, and it's
been used to derive some striking consequences in recent work. The point
of this note is to suggest that rather than accept those consequences,
we should be sceptical of the Ex Ante Pareto principle. Or, more
precisely, we should be sceptical of the precisification of the
principle that has the most interesting consequences.
\end{abstract}


\setstretch{1.1}
Consider the kind of problem that Kenneth Arrow
(\citeproc{ref-Arrow1950}{1950}) was interested in when he developed his
famous impossibility result. We have some citizens:
\emph{c}\textsubscript{1}, \ldots, \emph{c\textsubscript{n}}. Each
citizen \emph{c\textsubscript{i}} has a preference has a preference
relation ≿\textsubscript{\emph{i}} over the possible outcomes, which is
assumed to be symmetric, transitive, and complete. Our task is to find a
social preference relation ≿\textsubscript{∀}, which is also symmetric,
transitive, and complete, as a function of those individual relations.
The problem is that some weak-ish looking desiderata concerning how the
social ordering relates to the individual ordering entail that the job
cannot be done. There are some familiar moves to make next, in my
opinion by far the best is to deny to desirability of the Independence
condition Arrow uses, but all I need for now is that it's a familiar
problem.

In recent philosophy there has been some interest in a variant on
Arrow's original problem. The variant folks are interested in differs
from Arrow's in three dimensions.

First, we assume that the citizens are self-interested, or at least
differentially interested enough that we can which of them have their
preferences satisfied can vary freely. This turns out be enough to get
out of Arrow's original puzzle; it's a big enough violation of his
Unrestricted Domain principle that it's possible to satisfy the rest of
the principles.

Second, we drop the assumption that ≿\textsubscript{∀} is complete.
Maybe for some social outcomes, neither is at least as good as the
other. Amartya Sen (\citeproc{ref-Sen1970sec}{{[}1970{]} 2017}) had
originally noted that was enough to avoid the impossibility theorem that
Arrow developed. Unfortunately, as Allan Gibbard
(\citeproc{ref-Gibbard2014}{2014}) quickly showed, there are other
plausible principles that you still have to violate even without this
assumption. Still, it's going to be important here.

Third, we make life much harder for ourselves by dropping the assumption
that there are finitely many citizens. We will assume that there are
countably many citizens, but there may not be finitely many. It turns
out that once we allow this all sorts of new impossibility results come
into play.

This note will be about the distinctive kind of impossibility results we
get when we assume that ≿\textsubscript{\emph{i}} and ≿\textsubscript{∀}
have in their domain not just social outcomes, but lotteries over social
outcomes. Just like with Arrow, there are some very plausible looking
principles that can't all be true in this setting. One of those
principles says that for any two lotteries \emph{X} and \emph{Y}, if
\emph{X}~≿\textsubscript{\emph{i}}~\emph{Y} for all \emph{i}, then
\emph{X}~≿\textsubscript{∀}~\emph{Y}. That looks pretty plausible! How
could \emph{X} not be better than \emph{Y} if everyone is just as happy
with it? And I agree, it is pretty plausible. But, I'll argue, the
things it clashes with are more plausible still.

I do not expect anything like agreement with that comparative
plausibility claim; there is a lot of room for reasonable disagreement
about which plausible principles should be given up when they are shown
to clash. But I hope the clash is interesting.

\section{Why Are We Doing This}\label{sec-why}

There are actually two questions here. Why should we care about
relations like ≿\textsubscript{∀}, and why should we care about how they
behave in worlds where there are infinite populations. Let's take those
in order.

Jake Nebel (\citeproc{ref-Nebel2025}{2025}), who developed one of the
impossibility results I'll be discussing here, starts by inquiring into
what preferences a \emph{benevolent} agent would have. This way of
thinking about the problem has a few consequences. (None of this
paragraph is meant to be new; Nebel notes all these points in his
paper.) For one thing, we shouldn't assume that ≿\textsubscript{∀} will
be unique. It's at least conceptually possible that there are different
ways to be maximally benevolent. For another, we \emph{possibly}
shouldn't be treating ≿\textsubscript{\emph{i}} as measuring each
person's preferences, as opposed to their welfare. Here we get into
quite tricky questions about the nature of benevolence, and these
questions get very hard when we're thinking about probabilistic
benevolence.

I'm going to sidestep all those worries by taking our task to not be
benevolence, but governance. Assume we have a basically pleasant
bureaucrat who has to make a decision that will affect many people in
various, not entirely predicatable, ways. To make it concrete, assume
there is a moderately busy intersection that isn't working as it stands,
and the bureaucrat's job is to redesign it. In theory we might say that
in a democracy what matters should be what people want the intersection
to look like. In practice, we've all settled on representative democracy
as the form of government, and approximately no one changes their vote
on the design of a moderately busy intersection, so really the
bureaucrat is unconstrained by the democratic system. (At least, they
are unconstrained as long as they don't do anything particularly
egregious.) That doesn't mean they do what they want. Most traffic
engineers care about safety, and traffic flow, and noise, and cost, and
everything else that might be relevant, and try really hard to balance
the competing interests when designing intersections. That is, they try
do turn the various preferences users have into a social ordering and
pick the best option. But how do they manage this alchemy, of turning
the many orderings into one? That's the kind of project we're interested
in here. It's got the same formal structure as Nebel's project of
creating a benevolent preference ranking, but the differences might
matter once we have to make hard choices about which plausible principle
to give up.

But bureaucrats don't have infinite populations to deal with. Well,
maybe they don't. We actually don't know how big the universe is, and
how large the population of human, or otherwise morally salient, people
is or will be. And the `will be' part matters. Our imagined traffic
engineer has to think about people who don't yet exist. If in ten years
a tree near the intersection will have grown in such a way that if they
use design \emph{d} then the intersection will be unsafe for young
children, that's a reason to not implement \emph{d}, even if none of the
people put at risk are yet born. Once we think about possible future
people, the task of our bureaucrats gets much trickier.

Still, it is pretty uncommon for any bureaucrat to make decisions that
affect infinitely many people. Those are the decisions we'd expect to be
made through the democratic process. But, and this is a point Frank Hong
and Jeffrey Sanford Russell (\citeproc{ref-HongRussell2025}{2025}) are
careful to stress, some of these principles we're interested in have
strange consequences about how we compare infinite worlds even if the
worlds being compared only differ in finitely many ways. (Hong and
Russell developed the other big impossibility result we'll be talking
about here.) It turns out that looking hard at the infinite cases has
consequences for how we build up ≿\textsubscript{∀}, even if we don't
think anything that might be done has infinitely large consequences.

\phantomsection\label{refs}
\begin{CSLReferences}{1}{0}
\bibitem[\citeproctext]{ref-Arrow1950}
Arrow, Kenneth J. 1950. {``A Difficulty in the Concept of Social
Welfare.''} \emph{Journal of Political Economy} 58 (4): 328--46. doi:
\href{https://doi.org/10.1086/256963}{10.1086/256963}.

\bibitem[\citeproctext]{ref-Gibbard2014}
Gibbard, Allan F. 2014. {``Social Choice and the Arrow Conditions.''}
\emph{Economics and Philosophy} 30 (3): 269--84. doi:
\href{https://doi.org/10.1017/S026626711400025X}{10.1017/S026626711400025X}.

\bibitem[\citeproctext]{ref-HongRussell2025}
Hong, Frank, and Jeffrey Sanford Russell. 2025. {``Paradoxes of Infinite
Aggregation.''} \emph{Noûs} 59 (3): 809--27. doi:
\href{https://doi.org/10.1111/nous.12535}{10.1111/nous.12535}.

\bibitem[\citeproctext]{ref-Nebel2025}
Nebel, Jacob M. 2025. {``Infinite Ethics and the Limits of
Impartiality.''} No{û}s. 2025. doi:
\href{https://doi.org/10.1111/nous.70010}{10.1111/nous.70010}.

\bibitem[\citeproctext]{ref-Sen1970sec}
Sen, Amartya. (1970) 2017. \emph{Collective Choice and Social Welfare:}
An expanded edition. Cambridge, MA: Harvard University Press. doi:
\href{https://doi.org/10.4159/9780674974616}{10.4159/9780674974616}.

\end{CSLReferences}



\noindent Published in\emph{
?meta:citation.container-title}, 2025, pp. ?meta:citation.page.


\end{document}
