% Options for packages loaded elsewhere
% Options for packages loaded elsewhere
\PassOptionsToPackage{unicode}{hyperref}
\PassOptionsToPackage{hyphens}{url}
%
\documentclass[
  11pt,
  letterpaper,
  DIV=11,
  numbers=noendperiod,
  twoside]{scrartcl}
\usepackage{xcolor}
\usepackage[left=1.1in, right=1in, top=0.8in, bottom=0.8in,
paperheight=9.5in, paperwidth=7in, includemp=TRUE, marginparwidth=0in,
marginparsep=0in]{geometry}
\usepackage{amsmath,amssymb}
\setcounter{secnumdepth}{3}
\usepackage{iftex}
\ifPDFTeX
  \usepackage[T1]{fontenc}
  \usepackage[utf8]{inputenc}
  \usepackage{textcomp} % provide euro and other symbols
\else % if luatex or xetex
  \usepackage{unicode-math} % this also loads fontspec
  \defaultfontfeatures{Scale=MatchLowercase}
  \defaultfontfeatures[\rmfamily]{Ligatures=TeX,Scale=1}
\fi
\usepackage{lmodern}
\ifPDFTeX\else
  % xetex/luatex font selection
  \setmainfont[ItalicFont=EB Garamond Italic,BoldFont=EB Garamond
Bold,BoldFont=EB Garamond SemiBold]{EB Garamond Math}
  \setsansfont[]{EB Garamond}
  \setmathfont[]{Garamond-Math}
\fi
% Use upquote if available, for straight quotes in verbatim environments
\IfFileExists{upquote.sty}{\usepackage{upquote}}{}
\IfFileExists{microtype.sty}{% use microtype if available
  \usepackage[]{microtype}
  \UseMicrotypeSet[protrusion]{basicmath} % disable protrusion for tt fonts
}{}
\usepackage{setspace}
% Make \paragraph and \subparagraph free-standing
\makeatletter
\ifx\paragraph\undefined\else
  \let\oldparagraph\paragraph
  \renewcommand{\paragraph}{
    \@ifstar
      \xxxParagraphStar
      \xxxParagraphNoStar
  }
  \newcommand{\xxxParagraphStar}[1]{\oldparagraph*{#1}\mbox{}}
  \newcommand{\xxxParagraphNoStar}[1]{\oldparagraph{#1}\mbox{}}
\fi
\ifx\subparagraph\undefined\else
  \let\oldsubparagraph\subparagraph
  \renewcommand{\subparagraph}{
    \@ifstar
      \xxxSubParagraphStar
      \xxxSubParagraphNoStar
  }
  \newcommand{\xxxSubParagraphStar}[1]{\oldsubparagraph*{#1}\mbox{}}
  \newcommand{\xxxSubParagraphNoStar}[1]{\oldsubparagraph{#1}\mbox{}}
\fi
\makeatother


\usepackage{longtable,booktabs,array}
\usepackage{calc} % for calculating minipage widths
% Correct order of tables after \paragraph or \subparagraph
\usepackage{etoolbox}
\makeatletter
\patchcmd\longtable{\par}{\if@noskipsec\mbox{}\fi\par}{}{}
\makeatother
% Allow footnotes in longtable head/foot
\IfFileExists{footnotehyper.sty}{\usepackage{footnotehyper}}{\usepackage{footnote}}
\makesavenoteenv{longtable}
\usepackage{graphicx}
\makeatletter
\newsavebox\pandoc@box
\newcommand*\pandocbounded[1]{% scales image to fit in text height/width
  \sbox\pandoc@box{#1}%
  \Gscale@div\@tempa{\textheight}{\dimexpr\ht\pandoc@box+\dp\pandoc@box\relax}%
  \Gscale@div\@tempb{\linewidth}{\wd\pandoc@box}%
  \ifdim\@tempb\p@<\@tempa\p@\let\@tempa\@tempb\fi% select the smaller of both
  \ifdim\@tempa\p@<\p@\scalebox{\@tempa}{\usebox\pandoc@box}%
  \else\usebox{\pandoc@box}%
  \fi%
}
% Set default figure placement to htbp
\def\fps@figure{htbp}
\makeatother


% definitions for citeproc citations
\NewDocumentCommand\citeproctext{}{}
\NewDocumentCommand\citeproc{mm}{%
  \begingroup\def\citeproctext{#2}\cite{#1}\endgroup}
\makeatletter
 % allow citations to break across lines
 \let\@cite@ofmt\@firstofone
 % avoid brackets around text for \cite:
 \def\@biblabel#1{}
 \def\@cite#1#2{{#1\if@tempswa , #2\fi}}
\makeatother
\newlength{\cslhangindent}
\setlength{\cslhangindent}{1.5em}
\newlength{\csllabelwidth}
\setlength{\csllabelwidth}{3em}
\newenvironment{CSLReferences}[2] % #1 hanging-indent, #2 entry-spacing
 {\begin{list}{}{%
  \setlength{\itemindent}{0pt}
  \setlength{\leftmargin}{0pt}
  \setlength{\parsep}{0pt}
  % turn on hanging indent if param 1 is 1
  \ifodd #1
   \setlength{\leftmargin}{\cslhangindent}
   \setlength{\itemindent}{-1\cslhangindent}
  \fi
  % set entry spacing
  \setlength{\itemsep}{#2\baselineskip}}}
 {\end{list}}
\usepackage{calc}
\newcommand{\CSLBlock}[1]{\hfill\break\parbox[t]{\linewidth}{\strut\ignorespaces#1\strut}}
\newcommand{\CSLLeftMargin}[1]{\parbox[t]{\csllabelwidth}{\strut#1\strut}}
\newcommand{\CSLRightInline}[1]{\parbox[t]{\linewidth - \csllabelwidth}{\strut#1\strut}}
\newcommand{\CSLIndent}[1]{\hspace{\cslhangindent}#1}



\setlength{\emergencystretch}{3em} % prevent overfull lines

\providecommand{\tightlist}{%
  \setlength{\itemsep}{0pt}\setlength{\parskip}{0pt}}



 


\setlength\heavyrulewidth{0ex}
\setlength\lightrulewidth{0ex}
\usepackage[automark]{scrlayer-scrpage}
\clearpairofpagestyles
\cehead{
  Brian Weatherson
  }
\cohead{
  Against Ex Ante Pareto
  }
\ohead{\bfseries \pagemark}
\cfoot{}
\makeatletter
\newcommand*\NoIndentAfterEnv[1]{%
  \AfterEndEnvironment{#1}{\par\@afterindentfalse\@afterheading}}
\makeatother
\NoIndentAfterEnv{itemize}
\NoIndentAfterEnv{enumerate}
\NoIndentAfterEnv{description}
\NoIndentAfterEnv{quote}
\NoIndentAfterEnv{equation}
\NoIndentAfterEnv{longtable}
\NoIndentAfterEnv{abstract}
\renewenvironment{abstract}
 {\vspace{-1.25cm}
 \quotation\small\noindent\emph{Abstract}:}
 {\endquotation}
\newfontfamily\tfont{EB Garamond}
\addtokomafont{disposition}{\rmfamily}
\addtokomafont{title}{\normalfont\itshape}
\let\footnoterule\relax
\KOMAoption{captions}{tableheading}
\makeatletter
\@ifpackageloaded{caption}{}{\usepackage{caption}}
\AtBeginDocument{%
\ifdefined\contentsname
  \renewcommand*\contentsname{Table of contents}
\else
  \newcommand\contentsname{Table of contents}
\fi
\ifdefined\listfigurename
  \renewcommand*\listfigurename{List of Figures}
\else
  \newcommand\listfigurename{List of Figures}
\fi
\ifdefined\listtablename
  \renewcommand*\listtablename{List of Tables}
\else
  \newcommand\listtablename{List of Tables}
\fi
\ifdefined\figurename
  \renewcommand*\figurename{Figure}
\else
  \newcommand\figurename{Figure}
\fi
\ifdefined\tablename
  \renewcommand*\tablename{Table}
\else
  \newcommand\tablename{Table}
\fi
}
\@ifpackageloaded{float}{}{\usepackage{float}}
\floatstyle{ruled}
\@ifundefined{c@chapter}{\newfloat{codelisting}{h}{lop}}{\newfloat{codelisting}{h}{lop}[chapter]}
\floatname{codelisting}{Listing}
\newcommand*\listoflistings{\listof{codelisting}{List of Listings}}
\makeatother
\makeatletter
\makeatother
\makeatletter
\@ifpackageloaded{caption}{}{\usepackage{caption}}
\@ifpackageloaded{subcaption}{}{\usepackage{subcaption}}
\makeatother
\usepackage{bookmark}
\IfFileExists{xurl.sty}{\usepackage{xurl}}{} % add URL line breaks if available
\urlstyle{same}
\hypersetup{
  pdftitle={Against Ex Ante Pareto},
  pdfauthor={Brian Weatherson},
  hidelinks,
  pdfcreator={LaTeX via pandoc}}


\title{Against Ex Ante Pareto}
\author{Brian Weatherson}
\date{2025}
\begin{document}
\maketitle
\begin{abstract}
Given two lotteries over social outcomes, does the fact that everyone in
the society thinks that one is at least as good as the other suffice to
conclude that the first is at least as good as the other? I'm going to
argue that, somewhat surprisingly, the answer is no. The principle that
it does always suffice is called the Ex Ante Pareto principle, and it's
been used to derive some striking consequences in recent work. The point
of this note is to suggest that rather than accept those consequences,
we should be sceptical of the Ex Ante Pareto principle. Or, more
precisely, we should be sceptical of the precisification of the
principle that has the most interesting consequences.
\end{abstract}


\setstretch{1.1}
Consider the kind of problem that Kenneth Arrow
(\citeproc{ref-Arrow1950}{1950}) was interested in when he developed his
famous impossibility result. We have some citizens:
\emph{c}\textsubscript{1}, \ldots,~\emph{c\textsubscript{n}}. Each
citizen \emph{c\textsubscript{i}} has a preference has a preference
relation ≿\textsubscript{\emph{i}} over the possible outcomes, which is
assumed to be symmetric, transitive, and complete. Our task is to find a
social preference relation ≿\textsubscript{∀}, which is also symmetric,
transitive, and complete, as a function of those individual relations.
The problem is that some weak-ish looking desiderata concerning how the
social ordering relates to the individual ordering entail that the job
cannot be done. There are some familiar moves to make next, in my
opinion by far the best is to deny to desirability of the Independence
condition Arrow uses, but all I need for now is that it's a familiar
problem.

In recent philosophy there has been some interest in a variant on
Arrow's original problem. The variant folks are interested in differs
from Arrow's in three dimensions.

First, we assume that the citizens are self-interested, or at least
differentially interested enough that we can which of them have their
preferences satisfied can vary freely. This violates Arrow's condition
of Unrestricted Domain. In several of the cases we'll look at, the
violation of this condition is so extensive that all the other
conditions he described can be satisifed.

Second, we drop the assumption that ≿\textsubscript{∀} is complete.
Maybe for some social outcomes, neither is at least as good as the
other. Amartya Sen (\citeproc{ref-Sen1970sec}{{[}1970{]} 2017}) had
originally noted that was enough to avoid the impossibility theorem that
Arrow developed. Unfortunately, as Allan Gibbard
(\citeproc{ref-Gibbard2014}{2014}) quickly showed, there are other
plausible principles that you still have to violate even without this
assumption. Still, it's going to be important here.

Third, we make life much harder for ourselves by dropping the assumption
that there are finitely many citizens. We will assume that there are
countably many citizens, but there may not be finitely many. It turns
out that once we allow this all sorts of new impossibility results come
into play.

This note will be about the distinctive kind of impossibility results we
get when we assume that ≿\textsubscript{\emph{i}} and ≿\textsubscript{∀}
have in their domain not just social outcomes, but lotteries over social
outcomes. Just like with Arrow, there are some very plausible looking
principles that can't all be true in this setting. One of those
principles says that for any two lotteries \emph{X} and \emph{Y}, if
\emph{X}~≿\textsubscript{\emph{i}}~\emph{Y} for all \emph{i}, then
\emph{X}~≿\textsubscript{∀}~\emph{Y}. That looks pretty plausible! How
could \emph{X} not be better than \emph{Y} if everyone is just as happy
with it? And I agree, it is pretty plausible. But, I'll argue, the
things it clashes with are more plausible still.

I do not expect anything like agreement with that comparative
plausibility claim; there is a lot of room for reasonable disagreement
about which plausible principles should be given up when they are shown
to clash. But I hope the clash is interesting.

\section{Why Are We Doing This}\label{sec-why}

There are actually two questions here. Why should we care about
relations like ≿\textsubscript{∀}, and why should we care about how they
behave in worlds where there are infinite populations. Let's take those
in order.

Jake Nebel (\citeproc{ref-Nebel2025}{2025}), who developed one of the
impossibility results I'll be discussing here, starts by inquiring into
what preferences a \emph{benevolent} agent would have. Since some of us
are benevolent, and we may for all we know be living in an infinite
world, that's a reason to take the problems seriously.

While that's a perfectly good reason to be interested in these problems,
it's not the only reason, and it's not really my reason. I'm going to be
less interested in benevolence than in governance. Assume we have a
basically pleasant bureaucrat who has to make a decision that will
affect many people in various, not entirely predicatable, ways. To make
it concrete, assume there is a moderately busy intersection that isn't
working as it stands, and the bureaucrat's job is to redesign it. In
theory we might say that in a democracy what matters should be what
people want the intersection to look like. In practice, we've settled on
representative democracy as our form of government, and approximately no
one changes their vote on the design of a moderately busy intersection,
so really the bureaucrat is unconstrained by the democratic system. (At
least, they are unconstrained as long as they don't do anything
particularly egregious.) In practice, the good bureaucrat is, with
respect to this problem, a well meaning dictator. That doesn't mean they
do what they want. Most traffic engineers care about safety, and traffic
flow, and noise, and cost, and everything else that might be relevant,
and try really hard to balance the competing interests when designing
intersections. That is, they try do turn the various preferences users
have into a social ordering and pick the best option. But how do they
manage this alchemy, of turning the many orderings into one? That's the
kind of project we're interested in here. It's got the same formal
structure as Nebel's project of creating a benevolent preference
ranking, but the differences might matter once we have to make hard
choices about which plausible principle to give up.

But bureaucrats don't have infinite populations to deal with. Well,
maybe they don't. We actually don't know how big the universe is, and
how large the population of human, or otherwise morally salient, people
is or will be. And the `will be' part matters. Our imagined traffic
engineer has to think about people who don't yet exist. If in ten years
a tree near the intersection will have grown in such a way that if they
use design \emph{d} then the intersection will be unsafe for young
children, that's a reason to not implement \emph{d}, even if none of the
people put at risk are yet born. Once we think about possible future
people, the task of our bureaucrats gets much trickier.

Still, it is pretty uncommon for any bureaucrat to make decisions that
affect infinitely many people. Those are the decisions we'd expect to be
made through the democratic process. But, and this is a point Frank Hong
and Jeffrey Sanford Russell (\citeproc{ref-HongRussell2025}{2025}) are
careful to stress, some of these principles we're interested in have
strange consequences about how we compare infinite worlds even if the
worlds being compared only differ in finitely many ways. (Hong and
Russell developed the other big impossibility result we'll be talking
about here.) It turns out that looking hard at the infinite cases has
lessons for how we build up ≿\textsubscript{∀}, even if we don't think
anything that might be done has infinitely large consequences. Those
lessons will be the focus here.

\section{Formalism}\label{sec-formalism}

For the most part we'll be interested in these infinite worlds, though
in \textbf{?@sec-buchak} I'll return to what happens in finite worlds.
We have our infinitely many citizens, the \emph{c\textsubscript{i}}. To
keep things simple, we'll assume that the citizens only have one of two
welfare levels. Nebel denotes these with sad and happy faces, but we'll
just use 0 and 1. That's not as charming, but we don't want our traffic
engineers to be too whimsical. Besides, it simplifies the presentation
in a few ways. To get some lotteries going, we'll also assume that there
are infinitely many fair coins.

A world specifies the welfare level of each of the citizens, and the
results of each of the coin tosses. We'll denote these with two
sequences. I'll use \emph{x}, or when expanded out
\(\langle\)\emph{x}\textsubscript{1},~\(\ldots\)~\emph{x\textsubscript{i}}~\(\ldots \rangle\)
for the values of the citizens. And I'll use \emph{n}, or when expanded
out
\(\langle\)\emph{n}\textsubscript{1},~\(\ldots\)~\emph{n\textsubscript{i}}
\(\ldots \rangle\) for the values of the coins. So a \textbf{world} is a
pair \(\langle\)\emph{x},~\emph{n}\(\rangle\), and a
\textbf{proposition} is a set of worlds. There are three kinds of
propositions that will be especially important.

\begin{itemize}
\tightlist
\item
  A \textbf{distribution} is a proposition that settles everyone's
  welfare levels and nothing else. So if \emph{p} is a proposition, and
  \(\langle\)\emph{x}\textsubscript{1},~\emph{n}\textsubscript{1}\(\rangle\)
  and
  \(\langle\)\emph{x}\textsubscript{2},~\emph{n}\textsubscript{2}\(\rangle\)
  are both in \emph{p}, then
  \emph{x}\textsubscript{1}~=~\emph{x}\textsubscript{2}, and if
  \(\langle\)\emph{x}\textsubscript{1},~\emph{n}\textsubscript{1}\(\rangle\)
  is in \emph{p}, so is
  \(\langle\)\emph{x}\textsubscript{1},~\emph{n}\textsubscript{2}\(\rangle\).
\item
  A \textbf{state} is a proposition that settles how some of the coins
  land and nothing else. So if \emph{p} is a proposition, so is
  \(\langle\)\emph{x}\textsubscript{2},~\emph{n}\textsubscript{1}\(\rangle\).
\item
  A \textbf{lottery} is a proposition that says what distribution is
  actual given some coin flips. So if \emph{p} is a lottery, then there
  is some set \emph{N} of outcomes of the coin flips, and for each
  \emph{n}~∈~\emph{N}, there is a unique
  \(\langle\)\emph{x},~\emph{n}\(\rangle\) in \emph{p}. One consequence
  of this definition is that distributions are degenerate lotteries.
\end{itemize}

I'll use capital letters \emph{X},~\emph{Y}, etc., for lotteries, and
lower case letters \emph{x},~\emph{y}, etc. for distributions. I'll
assume that the comparisons we're interested in,
≿\textsubscript{\emph{i}} and ≿\textsubscript{∀} are defined over pairs
of lotteries. But sometimes I'll write
\emph{x}~≿\textsubscript{∀}~\emph{y}; that means that the degenerate
lottery which returns \emph{x} for sure is at least as good (socially)
as the degenerate lottery that returns \emph{y} for sure.

I'll say that a lottery is \textbf{unconditional} if it includes a pair
\(\langle\)\emph{x},~\emph{n}\(\rangle\), and \textbf{conditional} if it
does not. Each lottery determines a probability for each state
consistent with it, and hence for each distribution. If the lottery is
unconditional, these probabilities are fairly obvious. The probability
of \emph{n\textsubscript{i}}~=~1~is 0.5 for any \emph{i}, for example.
But for conditional lotteries, the probability might be less obvious. If
\emph{N} is the set of coin landings such that
\emph{n}\textsubscript{1}~=~1 or \emph{n}\textsubscript{1}~=~2, and
\emph{X} is a lottery defined in \emph{N}, then
Pr(\emph{n}\textsubscript{1}~=~1~\textbar~\emph{X})~=~2/3, while
Pr(\emph{n}\textsubscript{2}~=~1~\textbar~\emph{X})~=~1/3.

I'll say that a lottery is \textbf{finite} if the set of distributions
consistent with it is finite, and infinite otherwise.

\section{Dramatis Personae}\label{sec-principles}

A lot of the philosophical moves start by showing that some combination
of constraints on ≿\textsubscript{∀} cannot be jointly satisfied. Before
the proofs and arguments start, it helps to have these constraints on
the table.

\subsection{Preference Constraints}\label{preference-constraints}

As noted earlier, we're going to mostly assume that people are
self-interested. Alternatively, we're going to assume that
≿\textsubscript{∀} doesn't aggregate the citizens' preferences, but
their interests. Formally, these two assumptions come to the same thing.

\begin{description}
\tightlist
\item[Self-Interestedness]
\emph{x}~≿\textsubscript{\emph{i}} \emph{y} iff
\emph{x\textsubscript{i}}~≥~\emph{y\textsubscript{i}}
\item[Stochastic Self-Interestedness]
\emph{X}~≿\textsubscript{\emph{i}} \emph{Y} iff
Pr(\emph{x\textsubscript{i}}~=~1~\textbar~\emph{X}) ≥
Pr(\emph{x\textsubscript{i}}~=~1~\textbar~\emph{Y})
\end{description}

The second of these is similar to what Nebel calls ``Simple
Stochasticism for Individuals'' and what Hong and Russell call
``Stochastic Compensation''. It's a little stronger than either of their
principles, but hardly less plausible.

There are weaker versions of \textbf{Stochastic Self-Interestedness}
that only quantify over unconditional and/or finite lotteries (as
Nebel's principle does), but I won't focus on either distinction in what
follows.

\subsection{Dominance Principles}\label{dominance-principles}

These principles constrain how ≿\textsubscript{∀} treats related
conditional and unconditional lotteries. In the principles, a
\textbf{partition} is a set of sets of worlds, such that every world is
in exactly one of the sets. I'll write \emph{E} for a generic partition,
and \emph{E}\textsubscript{1},~\(\ldots\)~for the members of it. I'll
say that a partition is consistent with a lottery iff for each cell of
the partition, the intersection of it with the lottery is non-empty.

\begin{description}
\tightlist
\item[Partitionwise Dominance]
For any unconditional lotteries \emph{X} and \emph{Y} and partition
\emph{E}, if for each \emph{E\textsubscript{i}},
\emph{X}~∧~\emph{E\textsubscript{i}}~≿\textsubscript{∀}~\emph{Y}~∧~\emph{E\textsubscript{i}},
then \emph{X}~≿\textsubscript{∀}~\emph{Y}.
\item[Statewise Dominance]
The principle \textbf{Partitionwise Dominance} holds for each partition
where for each \emph{i}, only one distribution is consistent with
\emph{X}~∧~\emph{E\textsubscript{i}}, and only one distribution is
consistent with \emph{Y}~∧~\emph{E\textsubscript{i}}.
\end{description}

Given the way I've defined lotteries in general, and in particular
conditional lotteries,
\emph{X}~∧~\emph{E\textsubscript{i}}~≿\textsubscript{∀}~\emph{Y}~∧~\emph{E\textsubscript{i}}
just means that \emph{X} is better than \emph{Y} conditional on
\emph{E\textsubscript{i}}. That is, it means that between the lotteries
we get by just looking at the worlds in \emph{E\textsubscript{i}}, by
taking \emph{E\textsubscript{i}} as given, \emph{X} still does better
than \emph{Y}.

Both Nebel, and Hong and Russell, use \textbf{Statewise Dominance}. I'm
going to use the somewhat stronger \textbf{Partitionwise Dominance},
which means the results I get are going to be weaker. That said, I'll
use slightly weaker principles elsewhere, and the way I'll get the core
contradiction will be simpler, so that hopefully makes up for using a
somewhat stronger principle. It's also a little hard to see what could
motivate \textbf{Statewise Dominance} but not \textbf{Partitionwise
Dominance}; the arguments they each give for these principles don't seem
to rely in any particular way on the members of the partition being
states.

In real life cases, we don't want something as weak as \textbf{Statewise
Dominance} anyway. Go back to our traffic engineer. They think it's
50/50 whether a new housing estate will be built just east of the
intersection in the next ten years. But after running the model, they
decide that design \emph{d}\textsubscript{1} will be better than
\emph{d}\textsubscript{2} whether or not the estate is built. That seems
to settle that they should do \emph{d}\textsubscript{1} rather than
\emph{d}\textsubscript{2}. But the little partition we've used, \{Estate
built, Estate not built\}, does not settle all issues about the welfare
of road users. Still, it seems like a good bit of reasoning on the
engineer's part.

But now we should worry that \textbf{Partitionwise Dominance} hasn't
quite got what our engineer needed either. All it says is that
\emph{d}\textsubscript{1} is at least as good as
\emph{d}\textsubscript{2}. We need a stricter version of it. For this
principle we'll use the standard notation
\emph{X}~≻\textsubscript{∀}~\emph{Y} means
\emph{X}~≿\textsubscript{∀}~\emph{Y} and not
\emph{Y}~≿\textsubscript{∀}~\emph{X}.

\begin{description}
\tightlist
\item[Strict Partitionwise Dominance]
For any unconditional lotteries \emph{X} and \emph{Y} and partition
\emph{E}, if for each \emph{E\textsubscript{i}},
\emph{X}~∧~\emph{E\textsubscript{i}}~≻\textsubscript{∀}~\emph{Y}~∧~\emph{E\textsubscript{i}},
then \emph{X}~≻\textsubscript{∀}~\emph{Y}.
\item[Strict Statewise Dominance]
The principle \textbf{Strict Partitionwise Dominance} holds for each
partition where for each \emph{i}, only one distribution is consistent
with \emph{X}~∧~\emph{E\textsubscript{i}}, and only one distribution is
consistent with \emph{Y}~∧~\emph{E\textsubscript{i}}.
\end{description}

These are the dominance principles we'll mainly be using.

\subsection{Pareto Principles}\label{pareto-principles}

We now turn to principles relating individual good and social good. The
first three of these seem fairly obvious given the assumption that there
are only two welfare levels, but actually they seem reasonable more
broadly. We'll write \emph{X}~\textasciitilde~\emph{Y}, with any
subscript, to mean \emph{X}~≿~\emph{Y}~∧~\emph{Y}~≿~\emph{X}.

\begin{description}
\tightlist
\item[Pareto Indifference]
If for all \emph{i},
\emph{x}~\textasciitilde{}\textsubscript{i}~\emph{y}, then
\emph{x}~\textasciitilde{}\textsubscript{∀}~\emph{y}.
\item[Pareto Superiority]
If for all \emph{i}, \emph{x}~≽\textsubscript{i}~\emph{y}, then
\emph{x}~≽\textsubscript{∀}~\emph{y}.
\item[Pareto Strict Superiority]
If for all \emph{i}, \emph{x}~≻\textsubscript{i}~\emph{y}, then
\emph{x}~≻\textsubscript{∀}~\emph{y}.
\end{description}

That is, for any two distributions, for any of the three key relations,
if the relation holds with respect to all individuals, it holds
socially. We're going to primarily use something slightly stronger than
\textbf{Pareto Strict Superiority}, which we'll follow convention in
calling \textbf{Strong Pareto}.

\begin{description}
\tightlist
\item[Pareto Superiority]
If for all \emph{i}, \emph{x}~≽\textsubscript{i}~\emph{y}, and for some
\emph{i}, \emph{x}~≻\textsubscript{i}~\emph{y}, then
\emph{x}~≻\textsubscript{∀}~\emph{y}.
\end{description}

This makes sense again given our traffic engineer. There are these two
designs, \emph{d}\textsubscript{1} and \emph{d}\textsubscript{2}, and
\emph{d}\textsubscript{1} will make some people better off and no one
worse off. That seems to settle that \emph{d}\textsubscript{1} is
strictly better. It doesn't matter whether the city is embedded in a
finite or infinite universe, that still looks like a good inference. And
that's what \textbf{Pareto Superiority} says.

We get \textbf{Ex Ante} versions of all four principles by quantifying
not just over distributions, but over arbitrary lotteries. I won't write
all four of them out, because there is really only one that we'll focus
on. If it fails, the other four look pretty implausible.

\begin{description}
\tightlist
\item[Ex Ante Pareto Indifference]
If for all \emph{i},
\emph{X}~\textasciitilde{}\textsubscript{i}~\emph{Y}, then
\emph{X}~\textasciitilde{}\textsubscript{∀}~\emph{Y}.
\end{description}

The big aim of this paper is to argue that this principle, weak as it
is, is false.

\subsection{Equality Principles}\label{equality-principles}

Next we turn to principles which play the same role as Arrow's
non-dictatorship condition. They are all principles that say that in
principle, the welfare of some group can be outweighed by the welfare of
others.

The first three principles involve \emph{permutations}. A distribution
\emph{y} is a permutation of \emph{x} if there is a function
π:~\(\mathbb{N} \rightarrow \mathbb{N}\) and
\emph{y\textsubscript{i}}~=~\emph{x}\textsubscript{π(\emph{i})}. Abusing
notation a bit, we'll write that as \emph{y}~=~π(\emph{x}). The
\emph{support} of π is the set \{\emph{n}: π(\emph{x})~≠~\emph{x}\}.

\begin{description}
\tightlist
\item[Anonymity]
\emph{x}~\textasciitilde~π(\emph{x})
\item[Finite Anonymity]
If π has finite support, then \emph{x}~\textasciitilde~π(\emph{x})
\end{description}

\textbf{Anonymity} is usually rejected because it conflicts with
\textbf{Strong Pareto}. So \textbf{Finite Anonymity} is a natural
fallback. And it's what we're mostly going to be interested in here. But
you might worry, as Goodman and Lederman
(\citeproc{ref-GoodmanLederemanArXiV}{2024}) do, that we really should
have infinitary equality principles for an infinite world. They take as
their primary equality principle \textbf{Permutation Invariance}.

\begin{description}
\tightlist
\item[Permutation Invariance]
\emph{x}~≽\textsubscript{∀}~\emph{y}~↔︎
π(\emph{x})~≽\textsubscript{∀}~π(\emph{y})
\end{description}

These are all principles that require treating everyone alike. But we
can get some interesting results, as Arrow did, but simply requiring
that we not treat people too differently. Hong and Russell use a
principle they call \textbf{Interpersonal Allocation}.

\begin{description}
\tightlist
\item[Interpersonal Allocation]
If \emph{x} has finite support, then there is some \emph{y} with finite
support, whose support is entirely distinct from \emph{x}'s, such that
\emph{y}~≻\textsubscript{∀}~\emph{x}.
\end{description}

What this says is that for any group of citizens who could get welfare
1, it could be better for some other (presumably larger) group of
citizens to get welfare 1.

I'm going to use a principle that's a little harder to state, but
hopefully just as plausible. To state it, I need some terminology. Let
\emph{x} be a distribution with support \emph{S}, \emph{G} a subset of
\emph{S}, and \emph{G}\textsuperscript{′} set of citizens such that
\emph{G}\textsuperscript{′} ⊃ \emph{G} and
\emph{G}\textsuperscript{′}~∩~\emph{S}~=~\emph{G}. (That is, all the
others in \emph{G}\textsuperscript{′} get 0 in \emph{x}.) Let
\emph{x}{[}\emph{G},~\emph{G}\textsuperscript{′}, \emph{p}{]} be a
lottery such that:

\begin{itemize}
\tightlist
\item
  Everyone in \emph{G}\textsuperscript{′} has chance \emph{p} of getting
  1;
\item
  Everyone else gets, with chance 1, the same welfare they get in
  \emph{x}.
\end{itemize}

Then the equality principle I'll use is:

\begin{description}
\tightlist
\item[Stochastic Interpersonal Allocation]
Let \emph{G} be any subset of the support of \emph{x}. There is some
\emph{G}\textsuperscript{′} such that
\emph{x}{[}\emph{G},~\emph{G}\textsuperscript{′},
1/3{]}~≻\textsubscript{∀}~\emph{x}.
\end{description}

Very roughly, we could argue from \textbf{Interpersonal Allocation} to
\textbf{Stochastic Interpersonal Allocation} as follows. There is some
\emph{G}\textsubscript{1} which is more valuable than \emph{G}, i.e., it
is more valuable that everyone in \emph{G}\textsubscript{1} get a good
outcome than that everyone in \emph{G} get a good outcome. Similarly,
there is some \emph{G}\textsubscript{2} which is more valuable than
\emph{G}\textsubscript{1}. So if
\emph{G}\textsuperscript{′}~=~\emph{G}~∪~\emph{G}\textsubscript{1}~∪~\emph{G}\textsubscript{2},
then \emph{G}\textsuperscript{′} is more than three times as valuable as
\emph{G}. So replacing \emph{G} for sure with a 1/3 chance of
\emph{G}\textsuperscript{′} should be a positive trade-off.

That last paragraph is rough because really none of the steps follow
given just these equality principles. What is true is that given some
other principles listed here, especially \textbf{Ex Ante Pareto
Indifference}, we can tidy that argument up.

\subsection{Structural Principles}\label{structural-principles}

I mentioned at the start that I wasn't going to assume
\textbf{Completeness}.

\begin{description}
\tightlist
\item[Completeness]
\emph{X}~≿\textsubscript{∀}~\emph{Y}~∨~\emph{Y}~≿\textsubscript{∀}~\emph{X}
\end{description}

Indeed, as Goodman and Lederman
(\citeproc{ref-GoodmanLederemanArXiV}{2024}) notes, following Askell
(\citeproc{ref-Askell2018}{2018}), \textbf{Completeness} conflicts with
\textbf{Strong Pareto} and \textbf{Permutation Invariance}. What's more
plausible, as they also note, is \textbf{Finite Completeness}.

\begin{description}
\tightlist
\item[Finite Completeness]
If \emph{x} and \emph{y} are distributions with finite support, then
\emph{x}~≿\textsubscript{∀}~\emph{y}~∨~\emph{x}~≿\textsubscript{∀}~\emph{y}.
\end{description}

As they note, \textbf{Finite Completeness} plus \textbf{Permutation
Invariance} entails \textbf{Finite Anonymity}. I'll explore later two
stronger principles.

\begin{description}
\tightlist
\item[Finite Evaluability]
There is some real-valued function \emph{v} such that (a) if \emph{x}
has finite support, then \emph{v}(\emph{x}) is defined, (b) if \emph{X}
is a lottery that only has finitely many distributions as possible
outcomes, and every one of those distribution has finite support, then
\emph{v}(\emph{X}) is the expected value of the outcome of \emph{X}, and
(c) if \emph{X} and \emph{Y} are two lotteries thereby in the domain of
\emph{v}, \emph{X}~≿\textsubscript{∀}~\emph{Y} iff
\emph{v}(\emph{X})~≥~\emph{v}(\emph{Y}).
\end{description}

That's a mouthful, but the short version is that expected value
maximisation is the right theory for comparing lotteries that have
finitely many outcomes, each of which has finitely many happy people.
Again, this is the kind of thing that we'd need to assume if we wanted
to use expected utility maximisation as our choice rule around here,
while being unsure whether we are part of a finite or infinite universe.

\section{A Contradiction}\label{a-contradiction}

C'est la vie

\subsection*{References}\label{references}
\addcontentsline{toc}{subsection}{References}

\phantomsection\label{refs}
\begin{CSLReferences}{1}{0}
\bibitem[\citeproctext]{ref-Arrow1950}
Arrow, Kenneth J. 1950. {``A Difficulty in the Concept of Social
Welfare.''} \emph{Journal of Political Economy} 58 (4): 328--46. doi:
\href{https://doi.org/10.1086/256963}{10.1086/256963}.

\bibitem[\citeproctext]{ref-Askell2018}
Askell, Amanda. 2018. {``Pareto Principles in Infinite Ethics.''} PhD
thesis, New York University.

\bibitem[\citeproctext]{ref-Gibbard2014}
Gibbard, Allan F. 2014. {``Social Choice and the Arrow Conditions.''}
\emph{Economics and Philosophy} 30 (3): 269--84. doi:
\href{https://doi.org/10.1017/S026626711400025X}{10.1017/S026626711400025X}.

\bibitem[\citeproctext]{ref-GoodmanLederemanArXiV}
Goodman, Jeremy, and Harvey Lederman. 2024. {``Maximal Social Welfare
Relations on Infinite Populations Satisfying Permutation Invariance.''}
\url{https://arxiv.org/abs/arXiv:2408.05851}. arXiv preprint.

\bibitem[\citeproctext]{ref-HongRussell2025}
Hong, Frank, and Jeffrey Sanford Russell. 2025. {``Paradoxes of Infinite
Aggregation.''} \emph{Noûs} 59 (3): 809--27. doi:
\href{https://doi.org/10.1111/nous.12535}{10.1111/nous.12535}.

\bibitem[\citeproctext]{ref-Nebel2025}
Nebel, Jacob M. 2025. {``Infinite Ethics and the Limits of
Impartiality.''} No{û}s. 2025. doi:
\href{https://doi.org/10.1111/nous.70010}{10.1111/nous.70010}.

\bibitem[\citeproctext]{ref-Sen1970sec}
Sen, Amartya. (1970) 2017. \emph{Collective Choice and Social Welfare:}
An expanded edition. Cambridge, MA: Harvard University Press. doi:
\href{https://doi.org/10.4159/9780674974616}{10.4159/9780674974616}.

\end{CSLReferences}



Draft of October 2025.


\end{document}
