% Options for packages loaded elsewhere
\PassOptionsToPackage{unicode}{hyperref}
\PassOptionsToPackage{hyphens}{url}
%
\documentclass[
  10pt,
  letterpaper,
  DIV=11,
  numbers=noendperiod,
  twoside]{scrartcl}

\usepackage{amsmath,amssymb}
\usepackage{setspace}
\usepackage{iftex}
\ifPDFTeX
  \usepackage[T1]{fontenc}
  \usepackage[utf8]{inputenc}
  \usepackage{textcomp} % provide euro and other symbols
\else % if luatex or xetex
  \usepackage{unicode-math}
  \defaultfontfeatures{Scale=MatchLowercase}
  \defaultfontfeatures[\rmfamily]{Ligatures=TeX,Scale=1}
\fi
\usepackage{lmodern}
\ifPDFTeX\else  
    % xetex/luatex font selection
  \setmainfont[ItalicFont=EB Garamond Italic,BoldFont=EB Garamond
Bold]{EB Garamond Math}
  \setsansfont[]{Europa-Bold}
  \setmathfont[]{Garamond-Math}
\fi
% Use upquote if available, for straight quotes in verbatim environments
\IfFileExists{upquote.sty}{\usepackage{upquote}}{}
\IfFileExists{microtype.sty}{% use microtype if available
  \usepackage[]{microtype}
  \UseMicrotypeSet[protrusion]{basicmath} % disable protrusion for tt fonts
}{}
\usepackage{xcolor}
\usepackage[left=1in, right=1in, top=0.8in, bottom=0.8in,
paperheight=9.5in, paperwidth=6.5in, includemp=TRUE, marginparwidth=0in,
marginparsep=0in]{geometry}
\setlength{\emergencystretch}{3em} % prevent overfull lines
\setcounter{secnumdepth}{3}
% Make \paragraph and \subparagraph free-standing
\ifx\paragraph\undefined\else
  \let\oldparagraph\paragraph
  \renewcommand{\paragraph}[1]{\oldparagraph{#1}\mbox{}}
\fi
\ifx\subparagraph\undefined\else
  \let\oldsubparagraph\subparagraph
  \renewcommand{\subparagraph}[1]{\oldsubparagraph{#1}\mbox{}}
\fi


\providecommand{\tightlist}{%
  \setlength{\itemsep}{0pt}\setlength{\parskip}{0pt}}\usepackage{longtable,booktabs,array}
\usepackage{calc} % for calculating minipage widths
% Correct order of tables after \paragraph or \subparagraph
\usepackage{etoolbox}
\makeatletter
\patchcmd\longtable{\par}{\if@noskipsec\mbox{}\fi\par}{}{}
\makeatother
% Allow footnotes in longtable head/foot
\IfFileExists{footnotehyper.sty}{\usepackage{footnotehyper}}{\usepackage{footnote}}
\makesavenoteenv{longtable}
\usepackage{graphicx}
\makeatletter
\def\maxwidth{\ifdim\Gin@nat@width>\linewidth\linewidth\else\Gin@nat@width\fi}
\def\maxheight{\ifdim\Gin@nat@height>\textheight\textheight\else\Gin@nat@height\fi}
\makeatother
% Scale images if necessary, so that they will not overflow the page
% margins by default, and it is still possible to overwrite the defaults
% using explicit options in \includegraphics[width, height, ...]{}
\setkeys{Gin}{width=\maxwidth,height=\maxheight,keepaspectratio}
% Set default figure placement to htbp
\makeatletter
\def\fps@figure{htbp}
\makeatother
% definitions for citeproc citations
\NewDocumentCommand\citeproctext{}{}
\NewDocumentCommand\citeproc{mm}{%
  \begingroup\def\citeproctext{#2}\cite{#1}\endgroup}
\makeatletter
 % allow citations to break across lines
 \let\@cite@ofmt\@firstofone
 % avoid brackets around text for \cite:
 \def\@biblabel#1{}
 \def\@cite#1#2{{#1\if@tempswa , #2\fi}}
\makeatother
\newlength{\cslhangindent}
\setlength{\cslhangindent}{1.5em}
\newlength{\csllabelwidth}
\setlength{\csllabelwidth}{3em}
\newenvironment{CSLReferences}[2] % #1 hanging-indent, #2 entry-spacing
 {\begin{list}{}{%
  \setlength{\itemindent}{0pt}
  \setlength{\leftmargin}{0pt}
  \setlength{\parsep}{0pt}
  % turn on hanging indent if param 1 is 1
  \ifodd #1
   \setlength{\leftmargin}{\cslhangindent}
   \setlength{\itemindent}{-1\cslhangindent}
  \fi
  % set entry spacing
  \setlength{\itemsep}{#2\baselineskip}}}
 {\end{list}}
\usepackage{calc}
\newcommand{\CSLBlock}[1]{\hfill\break\parbox[t]{\linewidth}{\strut\ignorespaces#1\strut}}
\newcommand{\CSLLeftMargin}[1]{\parbox[t]{\csllabelwidth}{\strut#1\strut}}
\newcommand{\CSLRightInline}[1]{\parbox[t]{\linewidth - \csllabelwidth}{\strut#1\strut}}
\newcommand{\CSLIndent}[1]{\hspace{\cslhangindent}#1}

\setlength\heavyrulewidth{0ex}
\setlength\lightrulewidth{0ex}
\usepackage[automark]{scrlayer-scrpage}
\clearpairofpagestyles
\cehead{
  Brian Weatherson
  }
\cohead{
  Misleading Indexicals
  }
\ohead{\bfseries \pagemark}
\cfoot{}
\makeatletter
\newcommand*\NoIndentAfterEnv[1]{%
  \AfterEndEnvironment{#1}{\par\@afterindentfalse\@afterheading}}
\makeatother
\NoIndentAfterEnv{itemize}
\NoIndentAfterEnv{enumerate}
\NoIndentAfterEnv{description}
\NoIndentAfterEnv{quote}
\NoIndentAfterEnv{equation}
\NoIndentAfterEnv{longtable}
\NoIndentAfterEnv{abstract}
\renewenvironment{abstract}
 {\vspace{-1.25cm}
 \quotation\small\noindent\rule{\linewidth}{.5pt}\par\smallskip
 \noindent }
 {\par\noindent\rule{\linewidth}{.5pt}\endquotation}
\KOMAoption{captions}{tableheading}
\makeatletter
\@ifpackageloaded{caption}{}{\usepackage{caption}}
\AtBeginDocument{%
\ifdefined\contentsname
  \renewcommand*\contentsname{Table of contents}
\else
  \newcommand\contentsname{Table of contents}
\fi
\ifdefined\listfigurename
  \renewcommand*\listfigurename{List of Figures}
\else
  \newcommand\listfigurename{List of Figures}
\fi
\ifdefined\listtablename
  \renewcommand*\listtablename{List of Tables}
\else
  \newcommand\listtablename{List of Tables}
\fi
\ifdefined\figurename
  \renewcommand*\figurename{Figure}
\else
  \newcommand\figurename{Figure}
\fi
\ifdefined\tablename
  \renewcommand*\tablename{Table}
\else
  \newcommand\tablename{Table}
\fi
}
\@ifpackageloaded{float}{}{\usepackage{float}}
\floatstyle{ruled}
\@ifundefined{c@chapter}{\newfloat{codelisting}{h}{lop}}{\newfloat{codelisting}{h}{lop}[chapter]}
\floatname{codelisting}{Listing}
\newcommand*\listoflistings{\listof{codelisting}{List of Listings}}
\makeatother
\makeatletter
\makeatother
\makeatletter
\@ifpackageloaded{caption}{}{\usepackage{caption}}
\@ifpackageloaded{subcaption}{}{\usepackage{subcaption}}
\makeatother
\ifLuaTeX
  \usepackage{selnolig}  % disable illegal ligatures
\fi
\usepackage{bookmark}

\IfFileExists{xurl.sty}{\usepackage{xurl}}{} % add URL line breaks if available
\urlstyle{same} % disable monospaced font for URLs
\hypersetup{
  pdftitle={Misleading Indexicals},
  pdfauthor={Brian Weatherson},
  hidelinks,
  pdfcreator={LaTeX via pandoc}}

\title{Misleading Indexicals\thanks{Thanks to Europa Malynicz, Adam
Sennet and Ted Sider for helpful comments.}}
\author{Brian Weatherson}
\date{2002}

\begin{document}
\maketitle
\begin{abstract}
I argue against well informed observer theories about the referent of
indexicals.
\end{abstract}

\setstretch{1.1}
In `Now the French are invading England' Komarine Romdenh-Romluc
(\citeproc{ref-KRR2002}{2002}) offers a new theory of the relationship
between recorded indexicals and their content. Romdenh-Romluc's proposes
that Kaplan's basic idea, that reference is determined by applying a
rule to a context, is correct, but we have to be careful about what the
context is, since it is not always the context of utterance. A few well
known examples illustrate this. The `here' and `now' in `I am not here
now' on an answering machine do not refer to the time and place of the
original utterance, but to the time the message is played back, and the
place its attached telephone is located. Any occurrence of `today' in a
newspaper or magazine refers not to the day the story in which it
appears was written, nor to the day the newspaper or magazine was
printed, but to the cover date of that publication.

Still, it is plausible that for each (token of an) indexical there is a
salient context, and that `today' refers to the day of its context,
`here' to the place of its context, and soon. Romdenh-Romluc takes this
to be true, and then makes a proposal about what the salient context is.
It is `the context that \emph{Ac} would identify on the basis of cues
that she would reasonably take \emph{U} to be exploiting'.
(\citeproc{ref-KRR2002}{2002, 39}) \emph{Ac} is the relevant audience,
`the individual who it is reasonable to take the speaker to be
addressing', and who is assumed to be linguistically competent and
attentive. (So \emph{Ac} might not be the person \emph{U} intends to
address. This will not matter for what follows.) The proposal seems to
suggest that it is impossible to trick a reasonably attentive hearer
about what the referent of a particular indexical is. Since such
trickery does seem possible, Romdenh-Romluc's theory needs (at least)
supplementation. Here are two examples of such tricks.

\begin{quote}
\emph{Example One}\\
Imagine that at my university, the email servers are down, so all
communication from the office staff is by written notes left in our
mailboxes. I notice that one of my colleagues, Bruce, has a rather full
mailbox, and hence must not have been checking his messages for the last
day or two. I also know that Bruce is a forgetful type, and if someone
told him that he'd forgotten about a faculty meeting yesterday, he'd
probably believe them. In fact he hasn't forgotten; the meeting is for
later today. So I decide to play a little trick on him. I write an
official looking note saying `There is a faculty meeting today', leave
it undated, and put it in Bruce's mailbox underneath several other
messages, so it looks like it has been there for a day or two. When
Bruce sees it he is appropriately tricked, and for an instant panics
about the meeting that he has missed.
\end{quote}

It seems to me that what I wrote on the note was \emph{true}. It was
horribly misleading, to be sure, but still \emph{true}. And as a few
people have pointed out over the years, most prominently Bill Clinton I
guess, it is possible to mislead people with the truth. But on
Romdemh-Romluc's proposal, what I said was false, since my audience
(Bruce) reasonably took the context to be a day earlier in the week.

\begin{quote}
\emph{Example Two}\\
This example is closely based on a recent TV commercial. Jack leaves the
following message on Jill's answering machine late one Saturday night.
`Hi Jill, it's Jack. I'm at Rick's. This place is wild. There's lots of
cute girls here, but I'm just thinking about you.' In the background
loud music is playing, as if Jack were at a nightclub, indeed as if Jack
were at Rick's, so Jill reasonably concludes that Jack was at Rick's
when he sent the message, and hence that `here' refers to Rick's. In
fact Jack was home alone, but wanted to hide this fact, so he turned the
stereo up to full volume while leaving the message. Despite the fact
that a reasonable and attentive member of the target audience inferred
on the basis of contextual clues left by Jack that the context was
Rick's, it was not. The context was Jack's house, and `here' in Jack's
message referred to his house. Jack's trick may be less morally
reprehensible than mine, but at least I managed to avoid lying,
something Jack failed to do.
\end{quote}

In Example One I said something true even though what the hearer took me
to say was false. In Example Two Jack says something false, though what
the hearer takes him to say may well be true, assuming that there are a
lot of cute girls at Rick's. Romdenh-Romluc's theory predicts that
neither of these things is possible, so it does not work as it stands.
This, of course, is not to say that anyone else (myself included) has a
\emph{better} theory readily available, so it is unclear whether the
right lesson to draw from these examples is that Romdenh-Romluc's theory
needs to have some epicycles added, or that we need to try a rather
different approach. One simple epicycle makes the theory extensionally
adequate, but philosophically uninteresting. Consider modifying the
theory to require \emph{Ac} to be not just reasonable and attentive, but
informed of \emph{U}'s circumstances. Then the context identified by
\emph{Ac} will be the salient context for determining the referent of
\emph{U}'s indexicals. But saying this is not to offer a theory of
content for recorded indexicals, it is merely to say that ideally placed
observers have access to all the relevant semantic facts. Even this
might be wrong if epistemicism about vagueness is correct, but if that
is true then Romdenh-Romluc's theory is probably radically mistaken, for
then there are facts about content that cannot be reasonably believed,
even by an attentive and informed observer. We still seem to be a fair
distance from having an acceptable theory.

\section*{References}\label{references}
\addcontentsline{toc}{section}{References}

\phantomsection\label{refs}
\begin{CSLReferences}{1}{0}
\bibitem[\citeproctext]{ref-KRR2002}
Romdenh-Romluc, Komarine. 2002. {``Now the French Are Invading
England.''} \emph{Analysis} 62 (1): 34--41. doi:
\href{https://doi.org/10.1093/analys/62.1.34}{10.1093/analys/62.1.34}.

\end{CSLReferences}



\noindent Published in\emph{
Analysis}, 2002, pp. 308-310.

\end{document}
