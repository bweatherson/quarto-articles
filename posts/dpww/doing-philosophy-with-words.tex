% Options for packages loaded elsewhere
% Options for packages loaded elsewhere
\PassOptionsToPackage{unicode}{hyperref}
\PassOptionsToPackage{hyphens}{url}
%
\documentclass[
  11pt,
  letterpaper,
  DIV=11,
  numbers=noendperiod,
  twoside]{scrartcl}
\usepackage{xcolor}
\usepackage[left=1.1in, right=1in, top=0.8in, bottom=0.8in,
paperheight=9.5in, paperwidth=7in, includemp=TRUE, marginparwidth=0in,
marginparsep=0in]{geometry}
\usepackage{amsmath,amssymb}
\setcounter{secnumdepth}{3}
\usepackage{iftex}
\ifPDFTeX
  \usepackage[T1]{fontenc}
  \usepackage[utf8]{inputenc}
  \usepackage{textcomp} % provide euro and other symbols
\else % if luatex or xetex
  \usepackage{unicode-math} % this also loads fontspec
  \defaultfontfeatures{Scale=MatchLowercase}
  \defaultfontfeatures[\rmfamily]{Ligatures=TeX,Scale=1}
\fi
\usepackage{lmodern}
\ifPDFTeX\else
  % xetex/luatex font selection
  \setmainfont[ItalicFont=EB Garamond Italic,BoldFont=EB Garamond
Bold]{EB Garamond Math}
  \setsansfont[]{EB Garamond}
  \setmathfont[]{Garamond-Math}
\fi
% Use upquote if available, for straight quotes in verbatim environments
\IfFileExists{upquote.sty}{\usepackage{upquote}}{}
\IfFileExists{microtype.sty}{% use microtype if available
  \usepackage[]{microtype}
  \UseMicrotypeSet[protrusion]{basicmath} % disable protrusion for tt fonts
}{}
\usepackage{setspace}
% Make \paragraph and \subparagraph free-standing
\makeatletter
\ifx\paragraph\undefined\else
  \let\oldparagraph\paragraph
  \renewcommand{\paragraph}{
    \@ifstar
      \xxxParagraphStar
      \xxxParagraphNoStar
  }
  \newcommand{\xxxParagraphStar}[1]{\oldparagraph*{#1}\mbox{}}
  \newcommand{\xxxParagraphNoStar}[1]{\oldparagraph{#1}\mbox{}}
\fi
\ifx\subparagraph\undefined\else
  \let\oldsubparagraph\subparagraph
  \renewcommand{\subparagraph}{
    \@ifstar
      \xxxSubParagraphStar
      \xxxSubParagraphNoStar
  }
  \newcommand{\xxxSubParagraphStar}[1]{\oldsubparagraph*{#1}\mbox{}}
  \newcommand{\xxxSubParagraphNoStar}[1]{\oldsubparagraph{#1}\mbox{}}
\fi
\makeatother


\usepackage{longtable,booktabs,array}
\usepackage{calc} % for calculating minipage widths
% Correct order of tables after \paragraph or \subparagraph
\usepackage{etoolbox}
\makeatletter
\patchcmd\longtable{\par}{\if@noskipsec\mbox{}\fi\par}{}{}
\makeatother
% Allow footnotes in longtable head/foot
\IfFileExists{footnotehyper.sty}{\usepackage{footnotehyper}}{\usepackage{footnote}}
\makesavenoteenv{longtable}
\usepackage{graphicx}
\makeatletter
\newsavebox\pandoc@box
\newcommand*\pandocbounded[1]{% scales image to fit in text height/width
  \sbox\pandoc@box{#1}%
  \Gscale@div\@tempa{\textheight}{\dimexpr\ht\pandoc@box+\dp\pandoc@box\relax}%
  \Gscale@div\@tempb{\linewidth}{\wd\pandoc@box}%
  \ifdim\@tempb\p@<\@tempa\p@\let\@tempa\@tempb\fi% select the smaller of both
  \ifdim\@tempa\p@<\p@\scalebox{\@tempa}{\usebox\pandoc@box}%
  \else\usebox{\pandoc@box}%
  \fi%
}
% Set default figure placement to htbp
\def\fps@figure{htbp}
\makeatother


% definitions for citeproc citations
\NewDocumentCommand\citeproctext{}{}
\NewDocumentCommand\citeproc{mm}{%
  \begingroup\def\citeproctext{#2}\cite{#1}\endgroup}
\makeatletter
 % allow citations to break across lines
 \let\@cite@ofmt\@firstofone
 % avoid brackets around text for \cite:
 \def\@biblabel#1{}
 \def\@cite#1#2{{#1\if@tempswa , #2\fi}}
\makeatother
\newlength{\cslhangindent}
\setlength{\cslhangindent}{1.5em}
\newlength{\csllabelwidth}
\setlength{\csllabelwidth}{3em}
\newenvironment{CSLReferences}[2] % #1 hanging-indent, #2 entry-spacing
 {\begin{list}{}{%
  \setlength{\itemindent}{0pt}
  \setlength{\leftmargin}{0pt}
  \setlength{\parsep}{0pt}
  % turn on hanging indent if param 1 is 1
  \ifodd #1
   \setlength{\leftmargin}{\cslhangindent}
   \setlength{\itemindent}{-1\cslhangindent}
  \fi
  % set entry spacing
  \setlength{\itemsep}{#2\baselineskip}}}
 {\end{list}}
\usepackage{calc}
\newcommand{\CSLBlock}[1]{\hfill\break\parbox[t]{\linewidth}{\strut\ignorespaces#1\strut}}
\newcommand{\CSLLeftMargin}[1]{\parbox[t]{\csllabelwidth}{\strut#1\strut}}
\newcommand{\CSLRightInline}[1]{\parbox[t]{\linewidth - \csllabelwidth}{\strut#1\strut}}
\newcommand{\CSLIndent}[1]{\hspace{\cslhangindent}#1}



\setlength{\emergencystretch}{3em} % prevent overfull lines

\providecommand{\tightlist}{%
  \setlength{\itemsep}{0pt}\setlength{\parskip}{0pt}}



 


\setlength\heavyrulewidth{0ex}
\setlength\lightrulewidth{0ex}
\usepackage[automark]{scrlayer-scrpage}
\clearpairofpagestyles
\cehead{
  Brian Weatherson
  }
\cohead{
  Doing Philosophy With Words
  }
\ohead{\bfseries \pagemark}
\cfoot{}
\makeatletter
\newcommand*\NoIndentAfterEnv[1]{%
  \AfterEndEnvironment{#1}{\par\@afterindentfalse\@afterheading}}
\makeatother
\NoIndentAfterEnv{itemize}
\NoIndentAfterEnv{enumerate}
\NoIndentAfterEnv{description}
\NoIndentAfterEnv{quote}
\NoIndentAfterEnv{equation}
\NoIndentAfterEnv{longtable}
\NoIndentAfterEnv{abstract}
\renewenvironment{abstract}
 {\vspace{-1.25cm}
 \quotation\small\noindent\emph{Abstract}:}
 {\endquotation}
\newfontfamily\tfont{EB Garamond}
\addtokomafont{disposition}{\rmfamily}
\addtokomafont{title}{\normalfont\itshape}
\let\footnoterule\relax
\KOMAoption{captions}{tableheading}
\makeatletter
\@ifpackageloaded{caption}{}{\usepackage{caption}}
\AtBeginDocument{%
\ifdefined\contentsname
  \renewcommand*\contentsname{Table of contents}
\else
  \newcommand\contentsname{Table of contents}
\fi
\ifdefined\listfigurename
  \renewcommand*\listfigurename{List of Figures}
\else
  \newcommand\listfigurename{List of Figures}
\fi
\ifdefined\listtablename
  \renewcommand*\listtablename{List of Tables}
\else
  \newcommand\listtablename{List of Tables}
\fi
\ifdefined\figurename
  \renewcommand*\figurename{Figure}
\else
  \newcommand\figurename{Figure}
\fi
\ifdefined\tablename
  \renewcommand*\tablename{Table}
\else
  \newcommand\tablename{Table}
\fi
}
\@ifpackageloaded{float}{}{\usepackage{float}}
\floatstyle{ruled}
\@ifundefined{c@chapter}{\newfloat{codelisting}{h}{lop}}{\newfloat{codelisting}{h}{lop}[chapter]}
\floatname{codelisting}{Listing}
\newcommand*\listoflistings{\listof{codelisting}{List of Listings}}
\makeatother
\makeatletter
\makeatother
\makeatletter
\@ifpackageloaded{caption}{}{\usepackage{caption}}
\@ifpackageloaded{subcaption}{}{\usepackage{subcaption}}
\makeatother
\usepackage{bookmark}
\IfFileExists{xurl.sty}{\usepackage{xurl}}{} % add URL line breaks if available
\urlstyle{same}
\hypersetup{
  pdftitle={Doing Philosophy With Words},
  pdfauthor={Brian Weatherson},
  hidelinks,
  pdfcreator={LaTeX via pandoc}}


\title{Doing Philosophy With Words}
\author{Brian Weatherson}
\date{2006}
\begin{document}
\maketitle
\begin{abstract}
This paper discusses the coverage of ordinary language philosophy in
Scott Soames' ``Philosophical Analysis in the Twentieth Century''. After
praising the book's virtues, I raise three points where I dissent from
Soames' take on the history. First, I suggest that there is more to
ordinary language philosophy than the rather implausible version of it
that Soames sees to have been destroyed by Grice. Second, I argue that
confusions between analyticity, necessity and priority are less
important to the ordinary language period than Soames takes them to be.
Finally, I claim that Soames' criticisms of Ryle turn in part on
attributing reductionist positions to Ryle that Ryle did not hold.
\end{abstract}


\setstretch{1.1}
Scott Soames (\citeproc{ref-Soames2003}{2003}) has written two
wonderfully useful books that will be valuable introductions to
twentieth century philosophy. The books arose out of his well-received
classes on the history of twentieth century history at Princeton, and
will be valuable to anyone teaching similar courses. I shall be relying
on them as I teach such a course at Cornell.

The books consist of detailed case studies of important
twentieth-century works. They are best read alongside those original
texts. Anyone who works through the canon in this way will have an
excellent introduction to what twentieth century philosophers were
trying to do. The selections are judicious, and while some are obvious
classics some are rather clever choices of papers that are
representative of the type of work being done at the time. And Soames
doesn't just point to the most important works to study, but the most
important sections of those works.

Soames's discussion of these pieces is always built around an analysis
of their strengths and weaknesses. He praises the praiseworthy, but the
focus, at least in the sections I'm discussing (ordinary language
philosophy from Wittgenstein to Grice), is on where these philosophers
go wrong. This is particularly so when the mistakes are representative
of a theme. There are three main mistakes Soames finds in philosophers
of this period. First, they rely logical positivism long after it had
been shown to be unviable. Second, they disregard the principle that
semantics should be systematic. Third, they ignore the distinction
between necessity and a priority. All three constitute major themes of
Soames's book, and indeed of twentieth century philosophy as Soames sees
it.

These books concentrate, almost to a fault, on discussion of
philosophers' published works, as opposed to the context in which they
are written. Apart from occasionally noting that some books were
released posthumously, we aren't told whether the philosophers who wrote
them are alive, and only in one case are we told when a philosopher was
born. This kind of external information does not seem important to
Soames. He is the kind of historian who would prefer a fourth reading of
Austin's published works to a first reading of his wartime diaries. And
he'd prefer to spend the evening working on refutations, or charitable
reformulations, of Austin's arguments to either. I'm mostly sympathetic
to this approach; this is history of \emph{philosophy} after all. We can
leave discussions of the sociology of 1950s Oxford to those better
qualified. But this choice about what to write about has consequences.

Most of Soames's chapters focus almost exclusively on a particular book
or paper. The exceptions are like the chapter on \emph{Sense and
Sensibilia}, where Soames contrasts Austin's discussion with Ayer's
response. We learn a lot about the most important works that way, but
less about their intellectual environment. So the book doesn't have much
by way of broad discussion about overall trends or movements. There's
very little, for example, about who were the influencers and who the
influenced. There's nothing about how anyone not called `Wittgenstein'
changed their positions in response to criticism. One assumes from the
chronology that Ryle's influence on Austin was greater than Austin's
influence on Ryle, for example, but Soames is silent on whether this is
true.

Soames says at one point that, ``{[}Ryle{]} was, along with
Wittgenstein, J. L. Austin, and Paul Grice, one of the prime movers in
postwar philosophy in England.'' (68). But we aren't really told why
this is so, apart from the discussion of some prominent works of these
four philosophers. (Perhaps Soames has taken the maxim \emph{Show it,
don't say it} rather completely to heart.) Nor are why told why the list
includes those four, and not, say, Strawson or Geach or Anscombe.
Actually Anscombe's absence reminds us that there is almost no
discussion of women in philosophy in the book. That's not Soames fault,
it's a reflection of a long-running systematic problem in philosophy
that the discipline has a hard time recruiting and retaining women.
Could some of that be traced back to what was going on in the ordinary
language period? That kind of questions \emph{can't} be addressed by the
kind of history book that Soames has written, where the focus is on the
best philosophical writing, and not on the broader philosophical
community.

One of the other consequences of the format is that, by necessity, many
important figures are left out, on pain of writing a fifteen-volume
book. In the period under discussion here there was historically
important work by (among many others) Nelson Goodman, Wilfrid Sellars
and Roderick Chisholm, some of which connects up closely to the themes
and interests of the ordinary language philosophers, but none of which
is as much as mentioned. (Goodman is mentioned in the epilogue as
someone Soames regrets not covering.)

Now this can't be a complaint about the book Soames has written, because
it would have been impossible to cover any more figures than he did in
the style and depth that he did. And it would have been impossible to
tell in detail the story of how Ryle's impact on the philosophical world
differed from Austin's, or of the painfully slow integration of women
into the top echelons of philosophy, without making the book be even
more monumental than it is. All we're left with is a half-hearted
expression of regret that he didn't write a different \emph{kind} of
book, one that told us more about the forest, even as we value what he
says about the tallest of the trees.

\section{Grice and The End of Ordinary
Language}\label{grice-and-the-end-of-ordinary-language}

There is one place where Soames stops to survey the field, namely his
discussion of the impact of Grice's work on the ordinary language
tradition. Soames argues that with Grice's William James lectures, the
idea of ordinary language philosophy had ``run their course''. The
position seems to be that Grice overthrew a paradigm that had been
vibrant for two decades, but was running out of steam by the time of
Grice's James lectures. How plausible is this?

The first step is to work out just what it was that Grice
(\citeproc{ref-Grice1989}{1989}) refuted. When summarising the ordinary
language paradigm that he takes Grice to have overthrown, Soames is
uncharacteristically harsh. In Soames's summary one of the
characteristic activities of an ordinary language philosopher is
``opportunistically assembling reminders about how philosophically
significant words are used in ordinary settings'' (216). That \emph{may}
be a fair enough description of \emph{some} mid-century work, but it
isn't a fair summary of the best of the work that Soames has spent the
previous two hundred odd pages discussing. It all suggests that Grice
didn't so much overthrow ordinary language philosophy as much as badly
done ordinary language philosophy, and this category might not include
Strawson, Ryle, Austin and so on.

More importantly, it isn't entirely clear just what it was Grice did
that caused this paradigm shift. In Soames's telling it seems the
development of the speaker meaning/semantic meaning distinction was
crucial, but Austin (\citeproc{ref-Austin1962}{1962}) at least already
recognised this distinction, indeed appealed to it twice in \emph{Sense
and Sensibilia}. Soames mentions the discussion on pages 89 to 91 of
\emph{Sense and Sensibilia} of phrases like ``I see two pieces of
paper'', and there is also the intriguing discussion on pages 128-9 of
the relation between \emph{accurate} and \emph{true} where Austin goes
close to stating Grice's submaxim of concision.

The other suggestion is that Grice restored the legitimacy and
centrality of systematic semantic theorising. It's true Grice did that,
but this doesn't show we have to give up ordinary language philosophy
unless it was impossible to be an ordinary language philosopher and a
systematic semanticist. And it isn't clear that this really is
impossible. It hardly seems \emph{inconsistent} with the kind of
philosophy Austin did (especially in his theory of perception) that one
endorse a systematic semantic theory. (Though Austin \emph{himself}
rarely put forward systematic analyses.) Notably, there are plenty of
very systematic formal semanticists who take Strawson's work on
descriptions seriously, and try and integrate it into formal models. So
we might wonder why Grice's work shouldn't have led to a kind of
ordinary language philosophy where we paid more careful attention to
system-building.

More broadly, we might wonder whether the ordinary language period
really did end. The analysis of knowledge industry (strangely
undiscussed in a work on \emph{analysis} in the twentieth century)
seemed to putter along much the same before and after the official
demise of ordinary language philosophy. And there are affinities between
the ordinary language philosophers and important contemporary research
programs, e.g.~the `Canberra Plan' as described by Frank Jackson
(\citeproc{ref-Jackson1998}{1998}). So perhaps before we asked who
killed ordinary language philosophy (It was Professor Grice! In Emerson
Hall!! With the semantics/pragmatics distinction!!!) we should have made
sure there was a corpse. More on this point presently.

\section{A Whig History?}\label{a-whig-history}

One of the major themes of Soames's discussion is that there are some
systematic problems in twentieth century philosophy that are righted by
the heroes at the end of the story. I already mentioned the heroic role
assigned to Grice. But the real star of the show is Kripke
(\citeproc{ref-Kripke1980}{{[}1972{]} 1980}), who comes in as a deus ex
machina at the end showing how different necessity and a priority are,
and thereby righting all manner of grievous wrongs. That Kripke is an
important figure in twentieth century philosophy is hardly a matter of
dispute, but Soames does stretch a little to find errors for our hero to
correct.

Some of the complaints about philosophers collapsing the necessary/a
priori distinction do hit the target, but don't leave deep wounds in
their victims. For instance, Soames quotes Ryle
(\citeproc{ref-Ryle1954}{1954}) arguing (in \emph{Dilemmas}) that
perception cannot be a physiological process because if it were we
couldn't \emph{know} whether we saw a tree until we found out the result
of complicated brain scans. Soames points out, perfectly correctly, that
the seeing might be necessarily identical to the brain process even if
we don't know, and even can't know without complicated measurements,
whether they are identical. Soames is right that Ryle has made an
epistemological argument here when a metaphysical argument was needed.
But rewriting Ryle so he makes that metaphysical argument isn't hard. If
my seeing the tree is necessarily identical to the brain process, and
the brain process is (as Ryle and Soames seem to agree it is)
individuated by the brain components that implement it, then I couldn't
have seen the tree had one of the salient neurons in my brain been
silently replaced with a functionally equivalent silicon chip. Since it
\emph{is} possible that I could have seen a tree even if a salient
neuron was replaced with a functionally equivalent silicon chip, the
seeing and the brain process are not necessarily identical. So while
Ryle might have slipped here, and Kripke's work does help us correct the
slip, the consequences of this are basically verbal.

A more important charge of ignoring the necessary/a priori distinction
comes in Soames's discussion of Wittgenstein's deflationism about
philosophy. Here is the salient passage.

\begin{quote}
His deflationary conception of philosophy is also consistent with, and
even derivative from, his new ideas about meaning plus a set of
unquestioned philosophical presuppositions he brings to the enterprise.
The philosophical presuppositions include the then current and
widespread assumptions that (i) that philosophical theses are not
empirical, and hence must be necessary and a priori, and (ii) that the
necessary, the a priori and the analytic are one and the same. Because
he takes these assumptions for granted, he takes it for granted that if
there are any philosophical truths, they must be analytic (29).
\end{quote}

This seems to me to be mistaken twice over.

First, it isn't clear to me that there is \emph{any} appeal to concepts
of necessity in the passages in Wittgenstein Soames is summarising here,
and metaphysical necessity simply doesn't seem to have been a major
interest of Wittgenstein's. Wittgenstein does appear to reason that if a
proposition is not empirical it is a priori, but that inference doesn't
go via claims about necessity, and isn't shown to be fallacious by any
of Kripke's examples.

Second, it simply isn't true that philosophers in Wittgenstein's time
took for granted that the analytic and the a priori were one and the
same. To be sure, many philosophers in the early twentieth century
(including many argue the younger Wittgenstein) argued against Kant's
claim that they are distinct, but this isn't quite the same as taking
for granted they are identical. And there are a few places where
Wittgenstein appears to accept that some propositions are synthetic a
priori. For example in \emph{Remarks on the Foundations of Mathematics}
he says it is synthetic a priori that there is no reddish green, (Part
III, para 39) and goes on to say this about primes.

\begin{quote}
The distribution of primes would be an ideal example of what could be
called synthetic a priori, for one can say that it is at any rate not
discoverable by an analysis of the concept of a prime number.
(\citeproc{ref-Wittgenstein1956}{Wittgenstein 1956, pt. III}, para 42)
\end{quote}

Now it is far from obvious what the connection is between remarks such
as these and the remarks about the impossibility of philosophical theses
in the \emph{Investigations}. Indeed it is not obvious whether
Wittgenstein really believed in the synthetic a priori at any stage of
his career. But given his lack of interest in metaphysical necessity,
and openness to the possibility of synthetic a priori claims, it seems
unlikely that he was, tacitly or otherwise, using the argument Soames
gives him to get the deflationary conclusions.\footnote{I'm grateful to
  many correspondants for discussions about Wittgenstein. They convinced
  me, inter alia, that it would be foolish of me to commit to strong
  views of any kind about the role of the synthetic a priori in
  Wittgenstein's later thought, and that the evidence is particularly
  messy because Wittgenstein wasn't as centrally concerned with these
  concepts as we are.}

\section{Getting the Question Right}\label{getting-the-question-right}

As I mentioned above, Soames's is the kind of history that focuses on
the works of prominent philosophers, rather than their historical
context. There's much to be gained from this approach, in particular
about what the greats can tell us about pressing philosophical
questions. But one of the costs is that in focussing on what they say
about \emph{our} questions, we might overlook \emph{their} questions. In
most cases this is a trap Soames avoids, but in the cases of Austin and
Ryle the trap may have been sprung.

Soames sees Austin in \emph{Sense and Sensibilia} as trying to offer us
a new argument against radical scepticism.

\begin{quote}
Austin's ultimate goal is to undermine the coherence of skepticism. His
aim is not just to show that skepticism is unjustified, or implausible,
or that it is a position no one has reason to accept. Rather, his goal
is to prevent skepticism from getting off the ground by denying skeptics
their starting point. (173-4)
\end{quote}

But we don't get much of an interpretative argument that this is really
Austin's goal. Indeed, Soames concedes that Austin ``doesn't always
approach these questions directly'' (172). I'd say he does very little
to approach them at all. To be sure, many contemporary defenders of
direct realism are interested in its anti-sceptical powers, but there's
little to show \emph{Austin} was so moved. Scepticism is not a topic
that even arises in \emph{Sense and Sensibilia} until the chapter on
Warnock, after Austin has finished with the criticism of Ayer that takes
up a large part of the book. And Soames doesn't address the question of
how to square the somewhat dismissive tone Austin takes towards
scepticism in ``Other Minds'' with the view here propounded that Austin
put forward a fairly radical theory of perception as a way of providing
a new answer to the sceptic.

If Austin wasn't trying to refute the sceptic, what was he trying to do?
The simplest explanation is that he thought direct realism was true,
sense-data theories were false, and that ``there is noting so plain
boring a the constant repetition of assertions that are not true, and
sometimes no even faintly sensible; if we can reduce this a bit, it will
all be to the good.'' (\citeproc{ref-Austin1962}{Austin 1962, 5}) I'm
inclined to think that in this case the simplest explanation is the
best, that Austin wrote a series of lectures on perception because he
was interested in the philosophy of perception. Warnock says that
``Austin was genuinely shocked by what appeared to his eye to be
recklessness, hurry, unrealism, and inadequate attention to truth''
(\citeproc{ref-Warnock1989}{Warnock 1989, 154}) and suggests this
explained not only why Austin wrote the lectures but their harsh edge.

There is one larger point one might have wanted to make out of a
discussion of direct realism, or that one might have learned from a
discussion of direct realism, that seems relevant to what comes later in
Soames's book. If we really see objects, not sense-data, then objects
are constituents of intentional states. That suggests that public
objects might be constituents of other states, such as beliefs, and
hence constituents of assertions. Soames doesn't give us a discussion of
these possible historical links between direct realism and direct
reference, and that's too bad because there could be some fertile ground
to work over here. (I'm no expert on the history of the 1960s, so I'm
simply guessing as to whether there is a historical link between direct
realism and direct reference to go along with the strong philosophical
link between the two. But it would be nice if Soames has provided an
indication as to whether those guesses were likely to be productive or
futile.)

Soames gives us no inkling of where theories of direct reference came
from, save from the brilliant mind of Kripke. Apart from the absence of
discussion of any connection between direct realism and direct
reference, there's no discussion of the possible connections between
Wittgenstein's later theories and direct reference, as Howard Wettstein
(\citeproc{ref-Wettstein2004}{2004}) has claimed exist. And there's no
discussion of the (possibly related) fact that Kripke was developing the
work that went into \emph{Naming and Necessity} at the same time as he
was lecturing and writing on Wittgenstein, producing the material that
eventually became \emph{Wittgenstein on Rules and Private Language}.
Kripke is presented here as the first of the moderns\footnote{The first
  of what David Armstrong (\citeproc{ref-Armstrong2000}{2000}) has aptly
  called ``The Age of Conferences''.}, and in many ways he is, but the
ways in which he is the last (or the latest) of the ordinary language
philosophers could be a very valuable part of a history of
philosophy.\footnote{Just in case this gets misinterpreted, what I'm
  suggesting here is that Kripke (and his audiences) might have been
  influenced in interesting ways by philosophy of the 1950s and 1960s,
  \emph{not} that Kripke took his ideas from those philosophers. The
  latter claim has been occasionally made, but on that `debate'
  (\citeproc{ref-Soames1998a}{Soames 1998b},
  \citeproc{ref-Soames1998b}{1998a}) I'm 100\% on Soames's side.}

Matters are somewhat more difficult when it comes to Ryle's \emph{The
Concept of Mind}. Ryle predicted that he would ``be stigmatised as
`behaviourist'\,'' (\citeproc{ref-Ryle1949}{Ryle 1949, 327}) and Soames
obliges, and calls him a verificationist to boot.

\begin{quote}
If beliefs and desires were private mental states {[}says Ryle{]}, then
we could never observe the beliefs and desires of others. But if we
couldn't observe them, then we couldn't know that they exist, {[}which
we can.{]} \ldots{} This argument is no stronger than verificationism in
general, which by 1949 when \emph{The Concept of Mind} was published,
had been abandoned by its main proponents, the logical positivists, for
the simple reason that every precise formulation of it had been
decisively refuted (97-8).
\end{quote}

But Ryle's position here isn't verificationism at all, it's
abductophobia, or fear of inference to underlying causes. Ryle doesn't
think the claim of ghosts in the machine is \emph{meaningless}, he
thinks it is false. The kind of inference to underlying causes he
disparages here is \emph{exactly} the kind of inference to unobservables
that paradigm verificationists, especially Ayer, go out of their way to
\emph{allow}, and in doing so buy all end of trouble.\footnote{It would
  be particularly poor form of me to use a paradigm case argument
  without discussing Soames's very good dissection of Malcolm's paradigm
  case argument in chapter 7 of his book. So let me note my gratitude as
  a Cornellian for all the interesting lines of inquiry Soames finds
  suggested in Malcolm's paper -- his is a paradigm of charitable
  interpretation, a masterful discovery of wheat where I'd only ever
  seen chaff.} And abductophobia is prevalent among many contemporary
\emph{anti}-verificationists, particularly direct realists such as
McDowell (\citeproc{ref-McDowell1996}{1996}), Brewer
(\citeproc{ref-Brewer1999}{1999}) and Smith
(\citeproc{ref-Smith2003}{2003}) who think that if we don't directly
observe beer mugs we can never be sure that beer mugs exist. I basically
agree with Soames that Ryle's argument here (and the same style of
argument recurs repeatedly in \emph{The Concept of Mind}) is very weak,
but it's wrong to call it verificationist.

The issue of behaviourism is trickier. At one level Ryle surely is a
behaviourist, because whatever \emph{behaviourism} means in philosophy,
it includes what Ryle says in \emph{The Concept of Mind}. Ryle is the
reference-fixer for at least one disambiguation of \emph{behaviourist}.
However we label Ryle's views though, it's hard to square what he says
his aims are with the aims Soames attributes to him. In particular,
consider Soames's criticism of Ryle's attempt to show that we don't need
to posit a ghost in the machine to account for talk of intelligence.
(Soames is discussing a long quote from page 47 of \emph{The Concept of
Mind}.)

\begin{quote}
The description Ryle gives here is judicious, and more or less accurate.
But it is filled with words and phrases that seem to refer to causally
efficacious internal mental states---\emph{inferring}, \emph{thinking},
\emph{interpreting}, \emph{responding to objections}, \emph{being on the
lookout for this}, \emph{making sure not to rely on that}, and so on.
Unless all of these can be shown to be nothing more than behavioral
dispositions, Ryle will not have succeeded in establishing that to argue
intelligently is simply to manifest a variety of purely behavioral
dispositions. (106)
\end{quote}

And Soames immediately asks

\begin{quote}
So what are the prospects of reducing all this talk simply to talk about
what behavior would take place in various conditions? (106)
\end{quote}

The answer, unsurprisingly, is that the prospects aren't good. But why
this should bother \emph{Ryle} is never made clear. For Ryle only says
that when we talk of mental properties we talk about people's
dispositions, not that we talk about their \emph{purely behavioural}
dispositions. The latter is Soames's addition. It is rejected more or
less explicitly by Ryle in his discussion of knowing how. ``Knowing
\emph{how}, then, is a disposition, but not a single-track disposition
like a reflex or a habit \ldots{} its exercises can be overt or covert,
deeds performed or deeds imagined, words spoken aloud or words heard in
one's head, pictures painted on canvas or pictures in the mind's eye.''
(\citeproc{ref-Ryle1949}{1949, 46--47}). Nor should Ryle feel compelled
to say that these dispositions are behavioural, given his other
theoretical commitments.

Ryle is opposed in general to talk of `reduction' as the discussion of
mechanism on pages 76ff shows. To be sure there he is talking about
reduction of laws, but he repeatedly makes clear that he regards laws
and dispositions as tightly connected (\citeproc{ref-Ryle1949}{1949, 43,
123ff}) and suggests that we use mental concepts to signal that
psychological rather than physical laws are applicable to the scenario
we're discussing (167). Moreover, he repeatedly talks about mental
events for which it is unclear there is any kind of correlated
\emph{behavioural} disposition, e.g.~the discussion of Johnson's stream
of consciousness on page 58 and the extended discussion of imagination
in chapter 8. Ryle's claim that ``Silent soliloquy is a form of pregnant
non-sayings'' (269) hardly looks like the claim of someone who wanted to
reduce all mental talk to behavioural dispositions, unless one leans
rather hard on `pregnant'. But we aren't told whether Soames leans hard
on this word, for he never quite tells us why he thinks all the
dispositions that Ryle considers must be behavioural dispositions,
rather than (for example) dispositions to produce other dispositions.

To be sure, from a modern perspective it is hard to see where the space
is that Ryle aims to occupy. He wants to eliminate the ghosts, so what
is left for mind to be but physical stuff, and what does physical stuff
do but behave? He's not an eliminativist, so he's ontologically
committed to minds, and he hasn't left anything for them to be but
behavioural dispositions. So we might see it (not unfairly) but that's
not how Ryle sees it.\footnote{Of course he \emph{couldn't} have seen it
  that way since in 1949 he wouldn't have had the concept of ontological
  commitment.} Soames sees Ryle as an ancestor of a reductive
materialist like David Lewis, and a not very successful one at that. But
the Ryle of \emph{The Concept of Mind} has as much in common with
non-reductive materialists, especially when he says that ``not all
questions are physical questions'' (\citeproc{ref-Ryle1949}{1949, 77}),
insists that ``men are not machines, not even ghost-ridden machines''
(\citeproc{ref-Ryle1949}{1949, 81}) and describes Cartesians rather than
mechanists as ``the better soldiers'' (\citeproc{ref-Ryle1949}{1949,
330}) in the war against ignorance. Perhaps a modern anti-dualist should
aim for a reduction of the mental to the physical, but Ryle thought no
such reduction was needed to give up the ghost, and the historian should
record this.

\section{Conclusion}\label{conclusion}

As I said at the top, Soames has written two really valuable books. For
anyone who wants to really understand the most important philosophical
work written between 1900 and 1970, reading through the classics while
constantly referring back to Soames's books to have the complexities of
the philosophy explained will be immensely rewarding. Those who do that
might feel that the people who skip reading the classics and just read
Soames's books get an unreasonably large percentage of the benefits
they've accrued. As noted once or twice above I have some quibbles with
some points in Soames's story, but that shouldn't let us ignore what a
great service Soames has provided by providing these surveys of great
philosophical work.\footnote{Thanks to David Chalmers, Michael Fara,
  John Fischer, Tamar Szabó Gendler, James Klagge, Michael Kremer,
  Ishani Maitra, Aidan McGlynn, Alva Noë, Jonathan Weinberg and Larry
  Wright.}

\subsection*{References}\label{references}
\addcontentsline{toc}{subsection}{References}

\phantomsection\label{refs}
\begin{CSLReferences}{1}{0}
\bibitem[\citeproctext]{ref-Armstrong2000}
Armstrong, D. M. 2000. {``Black Swans: The Formative Influences in
Australian Philosophy.''} In \emph{Rationality and Irrationality},
edited by Berit Brogaard and Barry Smith, 11--17. Kirchberg: Austrian
Ludwig Wittgenstein Society.

\bibitem[\citeproctext]{ref-Austin1962}
Austin, J. L. 1962. \emph{Sense and Sensibilia}. Oxford: Oxford
University Press.

\bibitem[\citeproctext]{ref-Brewer1999}
Brewer, Bill. 1999. \emph{Perception and Reason}. Oxford: Oxford
University Press.

\bibitem[\citeproctext]{ref-Grice1989}
Grice, H. Paul. 1989. \emph{Studies in the Way of Words}. Cambridge,
MA.: Harvard University Press.

\bibitem[\citeproctext]{ref-Jackson1998}
Jackson, Frank. 1998. \emph{From Metaphysics to Ethics: A Defence of
Conceptual Analysis}. Clarendon Press: Oxford.

\bibitem[\citeproctext]{ref-Kripke1980}
Kripke, Saul. (1972) 1980. \emph{Naming and Necessity}. Cambridge:
Harvard University Press.

\bibitem[\citeproctext]{ref-McDowell1996}
McDowell, John. 1996. \emph{Mind and World}. Cambridge, MA: Harvard
University Press.

\bibitem[\citeproctext]{ref-Ryle1949}
Ryle, Gilbert. 1949. \emph{The Concept of Mind}. New York: Barnes;
Noble.

\bibitem[\citeproctext]{ref-Ryle1954}
---------. 1954. \emph{Dilemmas}. Cambridge: Cambridge University Press.

\bibitem[\citeproctext]{ref-Smith2003}
Smith, Michael. 2003. {``Rational Capacities.''} In \emph{Weakness of
Will and Varities of Practical Irrationality}, edited by Sarah Stroud
and Christine Tappolet, 17--38. Oxford: Oxford University Press.

\bibitem[\citeproctext]{ref-Soames1998b}
Soames, Scott. 1998a. {``More Revisionism about Reference.''} In
\emph{The New Theory of Reference}, edited by Paul Humphreys and James
Fetzer, 65--87. Dordrecht: Kluwer.

\bibitem[\citeproctext]{ref-Soames1998a}
---------. 1998b. {``Revisionism about Reference: A Reply to Smith.''}
In \emph{The New Theory of Reference}, edited by Paul Humphreys and
James Fetzer, 13--35. Dordrecht: Kluwer.

\bibitem[\citeproctext]{ref-Soames2003}
---------. 2003. \emph{Philosophical Analysis in the Twentieth Century}.
Princeton: Princeton University Press.

\bibitem[\citeproctext]{ref-Warnock1989}
Warnock, G. J. 1989. \emph{J. L. Austin}. London: Routledge.

\bibitem[\citeproctext]{ref-Wettstein2004}
Wettstein, Howard. 2004. \emph{The Magic Prism}. Oxford: Oxford
University Press.

\bibitem[\citeproctext]{ref-Wittgenstein1956}
Wittgenstein, Ludwig. 1956. \emph{Remarks on the Foundations of
Mathematics}. New York: Macmillan.

\end{CSLReferences}



\noindent Published in\emph{
Philosophical Studies}, 2006, pp. 429-437.


\end{document}
