% Options for packages loaded elsewhere
\PassOptionsToPackage{unicode}{hyperref}
\PassOptionsToPackage{hyphens}{url}
%
\documentclass[
  11pt,
  letterpaper,
  DIV=11,
  numbers=noendperiod,
  twoside]{scrartcl}

\usepackage{amsmath,amssymb}
\usepackage{setspace}
\usepackage{iftex}
\ifPDFTeX
  \usepackage[T1]{fontenc}
  \usepackage[utf8]{inputenc}
  \usepackage{textcomp} % provide euro and other symbols
\else % if luatex or xetex
  \usepackage{unicode-math}
  \defaultfontfeatures{Scale=MatchLowercase}
  \defaultfontfeatures[\rmfamily]{Ligatures=TeX,Scale=1}
\fi
\usepackage{lmodern}
\ifPDFTeX\else  
    % xetex/luatex font selection
    \setmainfont[ItalicFont=EB Garamond Italic,BoldFont=EB Garamond
Bold]{EB Garamond Math}
    \setsansfont[]{EB Garamond}
  \setmathfont[]{Garamond-Math}
\fi
% Use upquote if available, for straight quotes in verbatim environments
\IfFileExists{upquote.sty}{\usepackage{upquote}}{}
\IfFileExists{microtype.sty}{% use microtype if available
  \usepackage[]{microtype}
  \UseMicrotypeSet[protrusion]{basicmath} % disable protrusion for tt fonts
}{}
\usepackage{xcolor}
\usepackage[left=1.1in, right=1in, top=0.8in, bottom=0.8in,
paperheight=9.5in, paperwidth=7in, includemp=TRUE, marginparwidth=0in,
marginparsep=0in]{geometry}
\setlength{\emergencystretch}{3em} % prevent overfull lines
\setcounter{secnumdepth}{3}
% Make \paragraph and \subparagraph free-standing
\makeatletter
\ifx\paragraph\undefined\else
  \let\oldparagraph\paragraph
  \renewcommand{\paragraph}{
    \@ifstar
      \xxxParagraphStar
      \xxxParagraphNoStar
  }
  \newcommand{\xxxParagraphStar}[1]{\oldparagraph*{#1}\mbox{}}
  \newcommand{\xxxParagraphNoStar}[1]{\oldparagraph{#1}\mbox{}}
\fi
\ifx\subparagraph\undefined\else
  \let\oldsubparagraph\subparagraph
  \renewcommand{\subparagraph}{
    \@ifstar
      \xxxSubParagraphStar
      \xxxSubParagraphNoStar
  }
  \newcommand{\xxxSubParagraphStar}[1]{\oldsubparagraph*{#1}\mbox{}}
  \newcommand{\xxxSubParagraphNoStar}[1]{\oldsubparagraph{#1}\mbox{}}
\fi
\makeatother


\providecommand{\tightlist}{%
  \setlength{\itemsep}{0pt}\setlength{\parskip}{0pt}}\usepackage{longtable,booktabs,array}
\usepackage{calc} % for calculating minipage widths
% Correct order of tables after \paragraph or \subparagraph
\usepackage{etoolbox}
\makeatletter
\patchcmd\longtable{\par}{\if@noskipsec\mbox{}\fi\par}{}{}
\makeatother
% Allow footnotes in longtable head/foot
\IfFileExists{footnotehyper.sty}{\usepackage{footnotehyper}}{\usepackage{footnote}}
\makesavenoteenv{longtable}
\usepackage{graphicx}
\makeatletter
\newsavebox\pandoc@box
\newcommand*\pandocbounded[1]{% scales image to fit in text height/width
  \sbox\pandoc@box{#1}%
  \Gscale@div\@tempa{\textheight}{\dimexpr\ht\pandoc@box+\dp\pandoc@box\relax}%
  \Gscale@div\@tempb{\linewidth}{\wd\pandoc@box}%
  \ifdim\@tempb\p@<\@tempa\p@\let\@tempa\@tempb\fi% select the smaller of both
  \ifdim\@tempa\p@<\p@\scalebox{\@tempa}{\usebox\pandoc@box}%
  \else\usebox{\pandoc@box}%
  \fi%
}
% Set default figure placement to htbp
\def\fps@figure{htbp}
\makeatother
% definitions for citeproc citations
\NewDocumentCommand\citeproctext{}{}
\NewDocumentCommand\citeproc{mm}{%
  \begingroup\def\citeproctext{#2}\cite{#1}\endgroup}
\makeatletter
 % allow citations to break across lines
 \let\@cite@ofmt\@firstofone
 % avoid brackets around text for \cite:
 \def\@biblabel#1{}
 \def\@cite#1#2{{#1\if@tempswa , #2\fi}}
\makeatother
\newlength{\cslhangindent}
\setlength{\cslhangindent}{1.5em}
\newlength{\csllabelwidth}
\setlength{\csllabelwidth}{3em}
\newenvironment{CSLReferences}[2] % #1 hanging-indent, #2 entry-spacing
 {\begin{list}{}{%
  \setlength{\itemindent}{0pt}
  \setlength{\leftmargin}{0pt}
  \setlength{\parsep}{0pt}
  % turn on hanging indent if param 1 is 1
  \ifodd #1
   \setlength{\leftmargin}{\cslhangindent}
   \setlength{\itemindent}{-1\cslhangindent}
  \fi
  % set entry spacing
  \setlength{\itemsep}{#2\baselineskip}}}
 {\end{list}}
\usepackage{calc}
\newcommand{\CSLBlock}[1]{\hfill\break\parbox[t]{\linewidth}{\strut\ignorespaces#1\strut}}
\newcommand{\CSLLeftMargin}[1]{\parbox[t]{\csllabelwidth}{\strut#1\strut}}
\newcommand{\CSLRightInline}[1]{\parbox[t]{\linewidth - \csllabelwidth}{\strut#1\strut}}
\newcommand{\CSLIndent}[1]{\hspace{\cslhangindent}#1}

\setlength\heavyrulewidth{0ex}
\setlength\lightrulewidth{0ex}
\usepackage[automark]{scrlayer-scrpage}
\clearpairofpagestyles
\cehead{
  Brian Weatherson
  }
\cohead{
  Humean Supervenience
  }
\ohead{\bfseries \pagemark}
\cfoot{}
\makeatletter
\newcommand*\NoIndentAfterEnv[1]{%
  \AfterEndEnvironment{#1}{\par\@afterindentfalse\@afterheading}}
\makeatother
\NoIndentAfterEnv{itemize}
\NoIndentAfterEnv{enumerate}
\NoIndentAfterEnv{description}
\NoIndentAfterEnv{quote}
\NoIndentAfterEnv{equation}
\NoIndentAfterEnv{longtable}
\NoIndentAfterEnv{abstract}
\renewenvironment{abstract}
 {\vspace{-1.25cm}
 \quotation\small\noindent\emph{Abstract}:}
 {\endquotation}
\newfontfamily\tfont{EB Garamond}
\addtokomafont{disposition}{\rmfamily}
\addtokomafont{title}{\normalfont\itshape}
\let\footnoterule\relax
\KOMAoption{captions}{tableheading}
\makeatletter
\@ifpackageloaded{caption}{}{\usepackage{caption}}
\AtBeginDocument{%
\ifdefined\contentsname
  \renewcommand*\contentsname{Table of contents}
\else
  \newcommand\contentsname{Table of contents}
\fi
\ifdefined\listfigurename
  \renewcommand*\listfigurename{List of Figures}
\else
  \newcommand\listfigurename{List of Figures}
\fi
\ifdefined\listtablename
  \renewcommand*\listtablename{List of Tables}
\else
  \newcommand\listtablename{List of Tables}
\fi
\ifdefined\figurename
  \renewcommand*\figurename{Figure}
\else
  \newcommand\figurename{Figure}
\fi
\ifdefined\tablename
  \renewcommand*\tablename{Table}
\else
  \newcommand\tablename{Table}
\fi
}
\@ifpackageloaded{float}{}{\usepackage{float}}
\floatstyle{ruled}
\@ifundefined{c@chapter}{\newfloat{codelisting}{h}{lop}}{\newfloat{codelisting}{h}{lop}[chapter]}
\floatname{codelisting}{Listing}
\newcommand*\listoflistings{\listof{codelisting}{List of Listings}}
\makeatother
\makeatletter
\makeatother
\makeatletter
\@ifpackageloaded{caption}{}{\usepackage{caption}}
\@ifpackageloaded{subcaption}{}{\usepackage{subcaption}}
\makeatother

\usepackage{bookmark}

\IfFileExists{xurl.sty}{\usepackage{xurl}}{} % add URL line breaks if available
\urlstyle{same} % disable monospaced font for URLs
\hypersetup{
  pdftitle={Humean Supervenience},
  pdfauthor={Brian Weatherson},
  hidelinks,
  pdfcreator={LaTeX via pandoc}}


\title{Humean Supervenience}
\author{Brian Weatherson}
\date{2015}

\begin{document}
\maketitle
\begin{abstract}
Humean supervenience is the conjunction of three theses: Truth
supervenes on being, Anti‐haecceitism, and Spatiotemporalism. The first
clause is a core part of Lewis's metaphysics. The second clause is
related to Lewis's counterpart theory. The third clause says there are
no fundamental relations beyond the spatiotemporal, or fundamental
properties of extended objects. This paper sets out why Humean
Supervenience was so central to Lewis's metaphysics, and why we should
care about it even if there are empirical arguments against
Spatiotemporalism. The project of defending Humean Supervenience was
part of a larger project of philosophical compatibilism, of showing how
the folk picture of the world and the scientific picture could be made
to cohere with relatively little damage to the former and none to the
latter. And Lewis's contributions to that project are independent of
whether the scientific picture of the world ultimately includes
Spatiotemporalism.
\end{abstract}


\setstretch{1.1}
\section{What is Humean
Supervenience?}\label{what-is-humean-supervenience}

As with many aspects of David Lewis's work, it is hard to provide a
better summary of his views than he provided himself. So the following
introduction to what the Humean Supervenience view is will follow the
opening pages of Lewis (\citeproc{ref-Lewis1994a}{1994a}) extremely
closely. But for those readers who haven't read that paper, here's the
nickel version.

Humean Supervenience~is the conjunction of three theses.

\begin{enumerate}
\def\labelenumi{\arabic{enumi}.}
\tightlist
\item
  \textbf{Truth supervenes on being} (\citeproc{ref-Bigelow1988}{Bigelow
  1988}). That is, all the facts about a world supervene on facts about
  which individuals instantiate which fundamental properties and
  relations.
\item
  \textbf{Anti-haeccaetism}. All the facts about a world supervene on
  the distribution of qualitative properties and relations; rearranging
  which properties hang on which `hooks' doesn't change any facts.
\item
  \textbf{Spatio-temporalism}. The only fundamental relations that are
  actually instantiated are spatio-temporal, and all fundamental
  properties are properties of points or point-sized occupants of
  points.
\end{enumerate}

The first clause is a core part of Lewis's metaphysics. It is part of
what it is for some properties and relations to be fundamental that they
characterize the world. Indeed, Lewis thinks something stronger, namely
that the fundamental properties and relations characterize the world
\emph{without redundancy} (\citeproc{ref-Lewis1986a}{Lewis 1986a, 60}).
This probably isn't true, for a reason noted in Sider
(\citeproc{ref-SiderDiss}{1993}). Consider the relations \emph{earlier
than} and \emph{later than}. If these are both fundamental, then there
is some redundancy in the characterisation of the world in terms of
fundamental properties and relations. But there is no reason to believe
that one is fundamental and the other isn't. And it is hard to see how
we could give a complete characterisation of the world without either of
these relations. So we'll drop the claim that the fundamental properties
relations characterise the world without redundancy, and stick to the
weaker claim, namely that the fundamental properties and relations
characterize the world completely.

The second clause is related to Lewis's counterpart theory. Consider
what it would be like for anti-haeccaetism to fail. There would have to
be two worlds, with the same distribution of qualitative properties, but
with different facts obtaining in each. These facts would have to be
non-qualitative facts, presumably facts about which individual plays
which role. So perhaps, to use a well-known example, there could be a
world in which everything is qualitatively as it is in this world, but
in which Barack Obama plays the Julius Caeser role, and vice versa. So
Obama conquers Gaul and crosses the Rubicon, Caeser is born in Hawai'i
and becomes President of the United States. But what could make it the
case that the Gaul-conqueror in that world is really \emph{Obama's}
counterpart, and not Caeser's? Nothing qualitative, and nothing else it
seems is available. So this pseudo-possibility is not really a
possibility. And so on for all other counterexamples to
anti-haeccaetism.

The third clause is the most striking. It says there are no fundamental
relations beyond the spatio-temporal, or fundamental properties of
extended objects. If we assume that `properties' of objects with parts
are really relations between the parts, and anything extended has proper
parts, then the second clause reduces to the first. I think it isn't
unfair to read Lewis as holding both those theses.

Since for Lewis the fundamental qualities are all intrinsic, the upshot
is that the world is characterized by a spatio-temporal distribution of
intrinsic qualities. As Lewis acknowledged, this was considerably more
plausible given older views about the nature of physics than it is now.
We'll return to this point at great length below. But for now the key
point to see the kind of picture Humean Supervenience~offers. The world
is like a giant video monitor. The facts about a monitor's appearance
supervene, plausibly, on intrinsic qualities of the pixels, plus facts
about the spatial arrangement of the pixels. The world is 4-dimensional,
not 2-dimensional like the monitor, but the underlying picture is the
same.

\section{Supervenience}\label{supervenience}

Given the name Humean Supervenience  you might expect it to be possible
to state Humean Supervenience~as a supervenience thesis. But this turns
out to be hard to do. Here is one attempt at stating Humean
Supervenience~as a supervenience thesis that is happily clear, and
unhappily false.

\begin{description}
\tightlist
\item[Strong Modal Humean Supervenience]
For any two worlds where the spatio-temporal distribution of fundamental
qualities is the same, the contingent facts are the same.
\end{description}

But Humean Supervenience~does not make a claim this strong. It is
consistent with Humean Supervenience~that there could be fundamental
non-spatio-temporal relations. The only thing Humean
Supervenience~claims is that no such relations are instantiated. In a
pair of possible worlds where there are such relations, and the
relations vary but the arrangement of qualities is the same,
\textbf{Strong Modal Humean Supervenience} will fail. In the
Introduction to Lewis (\citeproc{ref-Lewis1986b}{1986b}), he suggested
the following weaker version.

\begin{description}
\tightlist
\item[Local Modal Humean Supervenience]
For any two worlds at which no alien properties or relations are
instantiated, if the spatio-temporal distribution of fundamental
qualities is the same at each world, the contingent facts are also the
same.
\end{description}

An alien property(/relation) is a fundamental property(/relation) that
is not actually instantiated. So this version of Humean
Supervenience~says that to get a difference between two worlds, you have
to either have a change in the spatio-temporal arrangement of qualities,
or the instantiation of actually uninstantiated fundamental properties
or relations.

But Lewis eventually decided that wouldn't do either. In response to
Haslanger (\citeproc{ref-Haslanger1994}{1994}), he conceded that
enduring objects would generate counterexamples to \textbf{Local Modal
Humean Supervenience} even if there were no alien properties or
relations. So he fell back to the following, somewhat vaguely stated,
thesis. (See Lewis (\citeproc{ref-Lewis1994a}{1994a}) for the
concession, and Hall (\citeproc{ref-Hall2010}{2010}) for an argument
that he should not have conceded this to Haslanger.)

\begin{description}
\tightlist
\item[Familiar Modal Humean Supervenience]
In any two ``worlds like ours'', if the spatio-temporal distribution of
fundamental qualities is the same at each world, the contingent facts
are also the same (\citeproc{ref-Lewis1994a}{Lewis 1994a, 475}).
\end{description}

What's a ``world like ours'? It isn't, I fear, entirely clear. But this
doesn't matter for the precise statement of Humean Supervenience. The
three theses in section 1 are clear enough, and state what Humean
Supervenience~is. The only difficulty is in stating it as a
\emph{supervenience} thesis.

\section{What is Perfect
Naturalness?}\label{what-is-perfect-naturalness}

That definintion does, however, require that we understand what it is
for some properties and relations to be \emph{fundamental}, or, as Lewis
put it following his discussion in Lewis
(\citeproc{ref-Lewis1983e}{1983}), \emph{perfectly natural}. The
perfectly natural properties and relations play a number of
interconnected roles in Lewis's metaphysics and his broader philosophy.

Most generally, they characterise the difference between real change and
`Cambridge change', and the related difference between real similarity,
and mere sharing of grue-like attributes. This somewhat loose idea is
turned, in \emph{Plurality}, into a definition of duplication.

\begin{quote}
\ldots two things are duplicates iff (1) they have exactly the same
perfectly natural properties, and (2) their parts can be put into
correspondence in such a way that corresponding parts have exactly the
same perfectly natural properties, and stand in the same perfectly
natural relations. (\citeproc{ref-Lewis1986a}{Lewis 1986a, 61})
\end{quote}

The intrinsic properties are then defined as those that are shared
between any two (possible) duplicates. So, as noted above, Humean
Supervenience~says that the spatio-temporal distribution of intrinsic
features of points characterises worlds like ours.

I've gone back and forth between describing these properties as
fundamental and describing them as perfectly natural. And that's because
for Lewis, the perfectly natural properties are in a key sense
fundamental. For reasons to do with the nature of vectorial properties,
I think this is probably wrong
(\citeproc{ref-Weatherson2006-WEATAM}{Weatherson 2006}). That is, we
need to hold that some derivative properties are perfectly natural in
order to get the definition of intrinsicness terms of perfect
naturalness to work. But for Lewis, the perfectly natural properties and
relations are all fundamental.

Part of what Lewis means by saying that some properties are fundamental
is that all the facts about the world supervene on the distribution.
(This is Bigelow's thesis that truth supervenes on being.) But I think
he also means something stronger. The non-fundamental facts don't merely
supervene on the fundamental facts; those non-fundamental facts are true
\emph{because} the fundamental facts are true, and \emph{in virtue of}
the truth of the fundamental facts.

The perfectly natural properties play many other roles in Lewis's
philosophy besides these two. They play a key role in the theory of
laws, for instance. They are a key part of Lewis's solution to the New
Riddle of Induction (\citeproc{ref-Goodman1955}{Goodman 1955}). And they
play an important role in Lewis's theory of content, though just exactly
what that role is is a matter of some dispute. (See Sider
(\citeproc{ref-Sider2001-SIDCOP}{2001}) and Weatherson
(\citeproc{ref-Weatherson2003-WEAWGA}{2003}) for one interpretation, and
Schwarz (\citeproc{ref-Schwarz2009}{2009}) for a conflicting
interpretation.)

Now it is a pretty open question whether any one division of properties
can do all these roles. One way to solve the New Riddle (arguably
Lewis's way, though this is a delicate question of interpretation) is to
be a dogmatist (in the sense of Pryor (\citeproc{ref-Pryor2000}{2000}))
about inductive projections involving a privileged class of properties.
Lewis's discussion of the New Riddle at the end of Lewis
(\citeproc{ref-Lewis1983e}{1983}) sounds like he endorses this view,
with the privileged class being the very same class as fundamentally
determines the structure of the world, and makes for objective
similarity and difference. But why should these classes be the same? It
might make more sense to, for instance, endorse dogmatism about
inductive projections of \emph{observational} properties, rather than
about microphysical properties.

Lewis doesn't attempt to give a theoretically neutral definition of the
perfectly natural properties. Rather, the notion of a perfectly natural
property is introduced by the theoretical role it serves. But that
theoretical role is very ambitious, covering many areas in metaphysics,
epistemology and the theory of content. We might wonder whether claims
like Humean Supervenience~have any content if it turns out nothing quite
plays that theoretical role. I think there is still a clear thesis we
can extract, relying on the connection between intrinsicness and
naturalness. It consists of the following claims:

\begin{itemize}
\tightlist
\item
  There is a small class of properties and relations such that the
  contingent facts at any world supervene on the distribution of these
  properties and relations.
\item
  Each of these properties is an intrinsic property.
\item
  At the actual world, the only relations among these which are
  instantiated are spatio-temporal, and all the contingent facts
  supervene not merely on the distribution of fundamental qualities and
  relations, but also on the distribution of fundamental qualities and
  relations \emph{over points and point-sized occupants of points}.
\end{itemize}

Those theses are distinctively Lewisian, they are clearly entailed by
Humean Supervenience~as Lewis's conceives of them, they are opposed in
one way or another by those who take themselves to reject Humean
Supervenience, but they are free of any commitment to there being a
single class of properties and relations that plays all the roles Lewis
wants the perfectly natural properties and relations to play. So from
now on, when I discuss the viability of Humean Supervenience, I'll be
discussing the viability of this package of views.

\section{Humean Supervenience and other Humean
Theses}\label{humean-supervenience-and-other-humean-theses}

Lewis endorsed many views that we might broadly describe as `Humean'. Of
particular interest here are the following three.

\begin{itemize}
\tightlist
\item
  \textbf{Humean Supervenience}.
\item
  \textbf{Nomological Reductionism}. Nomological properties and
  relations (including lawhood, chance and causation) are not among the
  fundamental properties and relations.
\item
  \textbf{Modal Combinatorialism}. Roughly, anything can co-exist with
  anything else.
\end{itemize}

We've stated Modal Combinatorialism~extremely roughly, and will persist
with using a fairly informal version of it throughout. For an excellent
study of more careful versions of it, see Nolan
(\citeproc{ref-Nolan1996-NOLRU}{1996}). But those details aren't as
important to this debate. What is important for now is that all three of
these theses are associated with what are known as Humean approaches to
metaphysics in the contemporary literature. But how closely connected
are they to each other, or for that matter to Hume?

One question about Humean Supervenience is just how it connects to the
work of the historical Hume. This would be a little easier to answer if
there was a broad scholarly consensus that Hume actually believed the
kind of simple regularity thesis of causation that Lewis attributes to
him at the start of Lewis (\citeproc{ref-Lewis1973b}{1973}). But it
isn't clear that this is Hume's view
(\citeproc{ref-Strawson2000}{Strawson 2000}). What is true is that Hume
was sceptical that we could know more about causation than that it was
manifested in certain distinctive kinds of correlations. But it is a
further step to say that Hume inferred that causation just consists of
these distinctive kinds of correlations.

A second question is how Humean Supervenience, which perhaps should be
referred to as so-called ``Humean Supervenience'', or perhaps even
better as ``Lewisian Supervenience'', relates to the kind of regularity
theory that Lewis attributes to Hume, or to the prohibition on necessary
connections between distinct existences that underlies Modal
Combinatorialism. Lewis seemed to see the three theses as related. Here
he is explaining how he chose to name Humean Supervenience~(and recall
that this \emph{isn't} backed up by any detailed exegesis of Hume).

\begin{quote}
Humean Supervenience~is named in honour of the great denier of necessary
connections. It is the doctrine that all there is to the world is a vast
mosaic of local matters of particular fact just one little thing and
then another. (\citeproc{ref-Lewis1986b}{Lewis 1986b} ix)
\end{quote}

This is a slightly confusing passage, since it isn't clear why a
violation of Humean Supervenience~would constitute a necessary
connection of any kind. We will return to this point below. But it does
seem to make clear that Lewis thought that Humean Supervenience~and
Modal Combinatorialism~were connected, since Modal Combinatorialism~is
much more closely connected to the denial that they can be necessary
connections between distinct existences.

Compare how Lewis introduces Humean Supervenience~when discussing the
role of possible worlds in formulating trans-world supervenience theses
in \emph{Plurality}.

\begin{quote}
Are the laws, chances, and causal relationships nothing but patterns
which supervene on this point-by-point distribution of properties? Could
two worlds differ in their wars without differing, somehow, somewhere,
in local qualitative character? (I discuss this question of `Humean
Supervenience', inconclusively, in the Introduction to my
\emph{Philosophical Papers}, volume II.)
(\citeproc{ref-Lewis1986a}{Lewis 1986a, 14})
\end{quote}

This seems to connect Humean Supervenience~closely to Nomological
Reductionism, since it makes the reducibility of the nomological
properties and relations central to the question of whether Humean
Supervenience~is true. We can also, I think, see Lewis connecting Modal
Combinatorialism~and Nomological Reductionism~in a later passage in
\emph{Plurality} where he discusses why he doesn't believe that laws are
necessary truths.

\begin{quote}
Another use of Modal Combinatorialism is to settle -- or as opponents
might say, to beg -- the question whether the laws of nature are
strictly necessary. They are not \ldots~Episodes of bread-eating are
possible because actual; as are episodes of starvation. Juxtaposed
duplicates of the two, on the grounds that anything can follow anything;
here is a possible world to violate the law bread nourishes. \ldots~It
is no surprise that Modal Combinatorialism prohibited strictly necessary
connections between distinct existences. What I have done is to take a
Humean view about laws and causation, and use it instead as a thesis
about possibility. Same thesis, different emphasis.
(\citeproc{ref-Lewis1986a}{Lewis 1986a, 91})
\end{quote}

So for Lewis, these three theses are meant to be closely connected. And
it is true that in the contemporary literature all three of them are
frequently described as `Humean' theses. (Or at least they are so
described in metaphysics and philosophy of science; again, we're
bracketing questions of historical interpretation here.) But on second
glance, it isn't as clear what the connection between the three theses
could amount to. One immediate puzzle is that Humean Supervenience~is
for Lewis a \emph{contingent} thesis, while the other two theses are
necessary truths. The accounts of causation, lawhood and chance that he
gives in defending Nomological Reductionism~are clearly meant to hold in
all kinds of worlds, not just worlds like ours. (Consider the amount of
effort that is spent in Lewis (\citeproc{ref-Lewis2004a}{2004a}) at
defending the theory of causation from examples involving wizards,
action at a distance and so on.) And the formulation of Modal
Combinatorialism~in \emph{Plurality} leaves little doubt that it is
meant to be necessarily true.

This difference in modal status means that the theses can't be in any
way equivalent. But you might think that they are in some way
reinforcing. Even that isn't so clear. Consider the most dedicated kind
of denier of Modal Combinatorialism, namely the fatalist who thinks that
every truth is a necessary truth. She will \emph{endorse} Humean
Supervenience. After all, she thinks that all the truths about the world
supervene on any category of truths whatsoever, so they'll supervene on
intrinsic properties of point-sized objects.

In the other direction, failures of Humean Supervenience~don't motivate
compromising Modal Combinatorialism. Imagine a world where occasionally
there are pairs of people who can know what each other is thinking, even
though there is no independent informational chain between the two of
them. It is just that a telepathic connection exists. Moreover, there is
no rhyme or reason to when a pair of people will be telepathic; it is
simply the case that some pairs of people are. In such a world, it is
plausible that \emph{being a telepathic pair} will be a fundamental
relation. That's not a problem for Humean Supervenience, since there
aren't any such pairs in this world. But it does mean Humean
Supervenience~is false in that world.

Assume that Daniels and O'Leary are a telepathic pair. Any duplication
of the pair of them will also be telepathic, since by Lewis's preferred
definition of duplication, duplication preserves all fundamental
properties and relations. Does that mean there's a necessary connection
between Daniels and O'Leary? Not really. The spirit of Modal
Combinatorialism~is that you can duplicate any parts of any worlds, and
combine them. One part of our world is Daniels. A duplicate of him need
not include any telepathic connection to O'Leary; indeed, he has
duplicates in worlds in which O'Leary is absent. Another part of the
world is O'Leary; duplicates of him need not include a connection to
Daniels. Putting the two together, there is a world where there are
duplicates of Daniels and O'Leary, but no telepathic connection between
the two. So Modal Combinatorialism~suggests that even when Humean
Supervenience~fails, there won't be a necessary connection between
distinct objects. So Humean Supervenience~really isn't that important to
the idea that there are no necessary connection between distinct
existences.

What's closer to the truth, I think, is that Humean Supervenience~is
\emph{interesting} because of Modal Combinatorialism. If Modal
Combinatorialism~fails, then Humean Supervenience~doesn't capture
anything important. In particular, it doesn't capture the idea that the
nomic is somehow less fundamental than (some features of) the non-nomic.
It is only given Modal Combinatorialism~that we can make these kinds of
priority claims in modal terms. Think about the philosopher who denies
Modal Combinatorialism~on the grounds that laws of nature are
necessarily true. That philosopher will say that the laws supervene on
the distribution of intrinsic properties of points, because the laws
supervene on any set of facts that you like. But they will deny that
this makes the distribution of intrinsic properties of points more
fundamental than the laws. It is only given Modal Combinatorialism~that
we can claim that supervenience theses are any guide whatsoever to
fundamentality.

What about the connection between Nomological Reductionism~and Humean
Supervenience? It can't be equivalence, since Lewis agrees that Humean
Supervenience~fails in worlds in which Nomological Reductionism~is true.
For the same reason, it can't be that failures of Humean
Supervenience~entail failures of Nomological Reductionism. What about
the other direction? Could we imagine Nomological Reductionism~failing
while Humean Supervenience~holds? I think this is a coherent
possibility, but not at all an attractive one. (Compare, in this
respect, the discussion of theories that ``qualify technically as
Humean'' at (\citeproc{ref-Lewis1994a}{Lewis 1994a, 485}).) It requires
that some of the irreducible, nomological properties be intrinsic
properties of point-sized objects. Well, we could imagine two worlds
where \emph{F} and \emph{G} are co-extensive, intrinsic properties of
points, and in one of them it is a law that all \emph{F}s are \emph{G}s,
and in the other it is a law that all \emph{G}s are \emph{F}s, and there
are further intrinsic properties of all the points which are \emph{F}
and \emph{G} which underlie these laws without making a difference to
any of the other facts. So we imagine that the property \emph{being F in
virtue of being G} is held by all these things in one world but not in
the other, and this is a fundamental perfectly natural property. I don't
think any of this is literally inconsistent, and I think filling out the
details could give us a way for Nomological Reductionism~to fail while
the letter of Humean Supervenience~holds. But it would clearly violate
the spirit of Humean Supervenience  and it isn't clear why we should
believe in such `possibilities' anyway.

So in practice, I think that any philosopher who rejects Nomological
Reductionism~is probably going to want to reject Humean Supervenience.
And I think that Lewis saw some of the deepest challenges to Humean
Supervenience~as coming from threats to Nomological Reductionism. In
particular, Lewis thought that the biggest challenges to Humean
Supervenience~came from the difficulties in providing a reductive
account of chance, and the appeal of non-reductive series of causation.

The difficulties in providing a reductive account of chance are
discussed at length in the introduction to Lewis
(\citeproc{ref-Lewis1986b}{1986b}), and in the only paper that has
`Humean Supervenience' in its title, i.e., Lewis
(\citeproc{ref-Lewis1994a}{1994a}). Here is a quick version of the
problem. Chances are not fundamental, so they must supervene on the
distribution of qualities. At least in the very early stages of the
universe, there aren't enough facts about the distribution of qualities
in the past and present to form a suitable subvenient base for the
chances. So whether the chance of \emph{p} is \emph{x} or \emph{y} will,
at least some of the time, depend on how the future of the world turns
out. Now let \emph{p} the proposition that tells the full story about
the future of the world. And assume that \emph{p} is a proposition such
that what its chance is depends on how that future goes. If it goes the
way \emph{p} says it will go, the chance of \emph{p} is \emph{x}; if it
goes some other way, the chance of \emph{p} is \emph{y}. Given a Humean
theory of chance, Lewis says that this is going to be possible.

But now there's a problem. What Lewis calls the Principal Principle~says
that if we know the chance of \emph{p} is \emph{y}, and have no further
information, then our credence in \emph{p} should be \emph{y}. But in
this case, if we knew the chance of \emph{p} was \emph{y}, we could be
sure that \emph{p} would not obtain. So our credence in \emph{p} should
be 0. Here we seem to have reached a contradiction, and it is a
contradiction to Lewis for a long time feared undermined the prospect of
giving a reductive account of chance. The solution he eventually settled
on in Lewis (\citeproc{ref-Lewis1994a}{1994a}) was to slightly modify
the Principal Principle, with the modification being designed to make
very little difference in regular cases, but avoid this contradiction.

Lewis discusses the appeal of non-reductive theories of causation in
several places, most notably for our purposes Lewis
(\citeproc{ref-Lewis2004a}{2004a}) and Lewis
(\citeproc{ref-Lewis2004d}{2004c}). Much of his attention is focused on
the theory developed by Peter Menzies
(\citeproc{ref-Menzies1996}{1996}). Menzies suggests that causation is
the intrinsic relation that does the best job of satisfying folk
platitudes about causation. A consequence of Menzies's view is that
there is something that makes a difference to the intrinsic properties
of pairs of causes and effects which doesn't supervene on either the
intrinsic properties of the two ends of the causal chain, or on the
spatio-temporal relations that hold between them. This something will
either be causation or will be something on which causation depends.
Either way there is a problem for Humean Supervenience, since there will
have to be a perfectly natural relation that is not spatio-temporal.

Lewis's response is to raise problems for the idea that causation could
be an intrinsic relation. One class of worries concerns the very idea
that causation could be a relation. Lewis says that absences can be
causes and effects, but absences can't stand in any relations, so
causation must not be a relation. Another class of worries concerns the
idea that causation could be intrinsic. Causation by double prevention,
says Lewis, doesn't look like it could be intrinsic. But intuitively
there could be causation by double prevention. Yet another class of
worries concerns the idea that causation could be a natural relation, or
that there could be any one thing that satisfies all the platitudes
about causation. The vast array of different ways in which causes can
bring about their effects in the actual world, he says, undermines this
possibility.

Note that in both cases Lewis defends Humean Supervenience~simply by
defending Nomological Reductionism. So I think it is fair to say that
there's a close connection between the two in Lewis's overall theory.

\section{Why Care about Humean
Supervenience}\label{why-care-about-humean-supervenience}

As is well-known, some surprising results in quantum mechanics suggest
that entanglement relations are somehow fundamental
(\citeproc{ref-Maudlin1994}{Maudlin 1994}). This suggests that Humean
Supervenience~is actually false. If that's right, why should we care
about philosophical arguments for Humean Supervenience? Lewis's response
to this challenge is somewhat disconcerting.

\begin{quote}
Really, what I uphold is not so much the truth of Humean
Supervenience~as the \emph{tenability} of it. If physics itself were to
teach me that it is false, I wouldn't grieve.

That might happen: maybe the lesson of Bell's Theorem is exactly that
\ldots~But I am not ready to take lessons in ontology from quantum
physics as it now is. \ldots~If, after quantum theory has been cleaned
up, it still teaches non-locality, I shall submit willingly to the best
of authority.

What I want to fight are \emph{philosophical} arguments against Humean
Supervenience. When philosophers claim that one or another commonplace
feature of the world cannot supervene on the arrangement of qualities, I
make it my business to resist. Being a commonsensical fellow (except
where unactualised possible worlds are concerned) I will seldom deny
that the features in question exist. I grant their existence, and do my
best to show how they can, after all, supervene on the arrangement of
qualities. (\citeproc{ref-Lewis1986b}{Lewis 1986b} xi)
\end{quote}

We can, I think, dismiss the point about quantum physics as it was in
1986. The theory has been cleaned up in just the way Lewis wanted, and
the claims about non-locality remain. Indeed, by the end of his life
Lewis was willing to take lessons in ontology from quantum physics. See,
for example, Lewis (\citeproc{ref-Lewis2004b}{2004b}). So what is at
issue here is whether or not there are philosophical arguments against
Humean Supervenience.

But at this point we might wonder why we should care. If a theory is
false, what does it matter whether its falsehood is shown by philosophy
or by physics? We might compare the dismissive attitude Lewis takes
towards Plantinga's attempts to show that reconstructions of the problem
of evil as an argument do not rely solely on things provable in
first-order logic (\citeproc{ref-Lewis1993b}{Lewis 1993}).

The answer I offered in Weatherson
(\citeproc{ref-Weatherson2009-WEADL}{2009}) was that the philosophical
defence of Humean Supervenience was connected to the point of the last
paragraph quoted above. Lewis wanted to save various features of our
commonsensical picture of the world. And he wanted to do this without
saying that philosophical reflection showed us that the picture of the
world given to us by signs of somehow incomplete. He wanted to defend
what I called `compatibilism', something that I contrasted with
eliminativism and expansionism. The eliminativists want to say that
science shows us that some commonsensical feature of reality doesn't
really exist. (See, for example, Churchland
(\citeproc{ref-Churchland1981}{1981}) for eliminativism about folk
psychological states.) The expansionists want to say that since science
(or at least physics) doesn't recognise certain features of reality, but
they obviously exist, we need to posit that science (or at least
physics) is incomplete. There are many stripes of philosophical
expansionists, from theists to dualists to believers in agent causation.

Lewis wasn't averse in principle to either eliminativism or
expansionism. One could, depending on exactly how one interpreted folk
theory and science, classify him as an eliminativist about gods, and an
expansionist about unactualised possible worlds. But his first tendency
was always to support compatibilism. Compatibilists face what Frank
Jackson (\citeproc{ref-Jackson1998}{1998}) called the `location
problem'. They have to show where the commonsensical features are
located in the scientific picture. That is, they have to show how to
reduce (in at least some sense of `reduce') our commonsensical concepts
to scientific concepts. (Many compatibilists may bristle at the idea
that they have to be reductionists; in recent decades the world has
abounded with `non-reductive physicalists', who are precisely
compatibilists in my sense, but who reject what they call
`reductionism'. But as Lewis (\citeproc{ref-Lewis1994b}{1994b}) argued,
these rejections often turn on reading too much into the notion of
reduction. For that reason, Lewis would not have objected to being
described as a reductionist about many everyday concepts.)

One way to perform such a reduction would be to wait until the best
scientific theory is developed, and show where within it we find minds,
meanings, morals and all the other exciting features of our ordinary
worldview. But that could take a while, and philosophers could use
something to do while waiting. In the meantime we could look for a
recipe that should work no matter what physical theory the scientists
settle on, or at least should work in a very wide range of cases. I
think we can see Lewis's defence of Humean Supervenience~as providing
such a recipe.

It is important to note here that Lewis's defence of Humean
Supervenience~was largely \emph{constructive}. He didn't try to give a
proof that there couldn't be more to the world than the arrangement of
local qualities. At least, he didn't rest a huge amount of weight on
such arguments. The arguments we will look at below for a functional
construal of the nomological are, perhaps, hints at arguments of this
type. But, in general, Lewis defended Humean Supervenience~by explicitly
showing where the ordinary concepts fitted in to a sparse physical
picture of reality, under the assumption that physics tells us that the
world consists of nothing but a spatio-temporal arrangement of intrinsic
qualities.

Now physics tells us no such thing. But it shouldn't matter. If the
recipe Lewis provides works in the case of the `Humean' world, it should
also work in the world physics tells us we actually live in. The
reduction of laws to facts about the distribution of fundamental
qualities, and the reduction of chances and counterfactual dependencies
to facts about laws, and the reduction of causation to facts about
chances and counterfactual dependencies, and the reduction of mind to
facts about causation and the distribution of qualities, and the
reduction of value to facts about minds, and so on are all independent
of whether physics tells us that we have to recognise relationships like
entanglement as fundamental. In other words, if we can solve the
location problem for the Humean world, we can solve it for the actual
world. And solving the location problem is crucial to defending
compatibilism. And whether it is possible to defend compatibilism is a
central concern of metaphysics.

I quoted above a passage from 1986 in which Lewis links Humean
Supervenience~to compatibilism. It's worth noting that he returns to the
point in 1994.

\begin{quote}
The point of defending Humean Supervenience~is not to support
reactionary physics, but rather to resist philosophical arguments that
there are more things in heaven and earth in physics has dreamt of.
Therefore if I defend the \emph{philosophical} tenability of Humean
Supervenience, that defence can doubtless be adapted to whatever better
supervenience thesis may emerge from better physics.
(\citeproc{ref-Lewis1994a}{Lewis 1994a, 474})
\end{quote}

That is, the defence of Humean Supervenience~just is part of the
argument against expansionism, and hence for compatibilism. That was the
defence I offered in Weatherson
(\citeproc{ref-Weatherson2009-WEADL}{2009}) for the interest of Lewis's
defence of Humean Supervenience, even if it were to turn out that Humean
Supervenience~was refuted by physics. I still think much of it is
correct. In particular, I still think that Lewis wanted to defend
compatibilism, and that the defence of Humean Supervenience~is key to
the defence of Humean Supervenience. Indeed, I think there is pretty
strong textual evidence that it was a major part of Lewis's motivation
for defending Humean Supervenience. But this explanation of why the
defence of Humean Supervenience~is significant can't explain why Lewis
was so worried about the failures of Humean theories of chance. After
all, if all we are trying to do is show that science and commonsense are
compatible, we could just take chances to be one of the fundamental
features of reality given to us by science. There isn't any need, from
the perspective of trying to reconcile science and common sense, to give
a reductive account of chance. Yet Lewis clearly thought that giving a
reductive account of chance was crucial to the defence of Humean
Supervenience. As he said,

\begin{quote}
There is one big bad bug: chance. It is here, and here alone, that I
fear defeat. But if I'm beaten here, then the entire campaign goes
kaput. (\citeproc{ref-Lewis1986b}{Lewis 1986b} xiv)
\end{quote}

I now think that attitude is very hard to explain if my earlier views
about the significance of Humean Supervenience~are entirely correct. The
natural conclusion is that there is something \emph{more} that the
defence of Humean Supervenience~is supposed to accomplish. One plausible
interpretation is that what it is supposed to accomplish is a
vindication of the idea that the key nomological concepts are, in a
sense, descriptive. It's easiest to say what this sense is by
contrasting it with the kind of view that Lewis rejected.

We're all familiar with the standard story about `water'. Our ordinary
usage of the term latches onto some stuff in the physical world. That
stuff is H\textsubscript{2}O. Some people think that's because our
ordinary usage determines a property which
H\textsubscript{2}O~satisfies, others because we demonstratively pick
out H\textsubscript{2}O~in ordinary demonstrations of what it is we're
talking about when we use the term `water'. Either way, we get to be
talking about H\textsubscript{2}O~when we use the word `water', even if
we are so ignorant of chemistry that we can't tell hydrogen and oxygen
apart. Moreover, our term continues to pick out H\textsubscript{2}O even
in worlds that are completely free of hydrogen and oxygen, and even if
such worlds have other stuff that plays a very similar functional role
to the role water plays in the actual world.

Lewis was somewhat sceptical of this standard story about `water'
(\citeproc{ref-Lewis2002b}{Lewis 2002}). He thought that the ordinary
term was ambiguous between our usage on which it picked out
H\textsubscript{2}O, and usage on which it picked out a role, a role
that happens to be played by H\textsubscript{2}O~in the actual world but
which could be played by other substances in other worlds. But if he
thought the standard story about `water' was at best, part right, he
thought applying a similar story to `law', `cause' and `chance' was
wildly implausible.

If such a story were right, then we would expect to find worlds where
there was some relation other than causation which played the causal
role. Since the actual world is physical, any world in which nonphysical
things stand in the kind of relations that causes and effects typically
stand in should do. So, for instance, if we have a world where the
castings of spells are frequently followed by transformations from human
to toad form, we should have a world where spells don't cause such
transformations but rather the spellcasting and the transformation stand
in a kind of fool's cause relationship. But we see no such thing. In
such magical worlds, spells cause transformations.

So whatever causation is, it doesn't look to be the kind of thing whose
essence can be discovered by physics. Physics couldn't tell us anything
about the essence of the relationship between the spell and the
transformation into a toad. But, we think, physics can tell us a lot
about the fundamental properties and relations are instantiated in the
actual world. So causation must not be one of them.

Lewis has a number of other arguments against anti-descriptivist views
about individual nomological concepts. These arguments strike me as
rather strong in the case of lawhood and causation, and less strong in
the case of chance.

If \emph{being F} and \emph{being F in virtue of a law} are both
fundamental properties, then a plausible principle of modal
recombination would suggest they could come apart. But they cannot; or
at least they cannot in one direction. We want \emph{being F in virtue
of a law} to entail \emph{being F}. That's easy if lawhood is defined in
terms of fundamental properties of things; but it's hard to see how it
could be if lawhood itself is fundamental
(\citeproc{ref-Lewis1986b}{Lewis 1986b} xii).

A similar argument goes for causation. Assume that causation is a
fundamental intrinsic relation that holds between things at different
times. Consider, for instance, the causal relationship which holds
between a throw of a rock (call it \emph{t}) and the shattering of the
window (call it \emph{s}). As we noted above in the case of Daniels and
O'Leary, several applications of Modal Combinatorialism~suggest that
there will be a world just like this one in which \emph{t} is followed
by \emph{s}, but in which \emph{t} does not cause \emph{s}. But such a
world seems to be impossible. As we also noted above, such a view runs
into trouble with causation by double prevention, which does not look to
be intrinsic.

The last two paragraphs have been extremely quick arguments, but in both
cases it seems to me that they can be tightened up so as to provide good
arguments for some kind of descriptivist stance towards laws and
causation. Chance is another matter.

The first problem is that recombination arguments if anything point
\emph{away} from descriptivism about chance. Any such account will imply
that chances can't, in general, point too far away from frequencies. But
recombination arguments suggest that chances and frequencies can come
arbitrarily far apart. Consider some particular event type \emph{e} that
has a one-half chance of occurring in circumstances \emph{c}. Start with
a world where \emph{c} occurs frequently, and about half the time it is
followed by \emph{e}. Now use recombination to generate a world where
all the \emph{c} ∧ ¬\emph{e} events are deleted, so \emph{c} is always
followed by \emph{e}. Unless we add a lot of bells and whistles to our
theory of chance, it will no longer be the case that the chance of
\emph{e} given \emph{c} is one-half. That is odd; we can't simply take
the first circumstance where \emph{c} occurred and at that moment there
was a one-half chance of it being followed by \emph{e}, and patch it
into an arbitrary world. Bigelow, Collins, and Pargetter
(\citeproc{ref-BigelowCollinsPargetter}{1993}) turn this idea into a
more careful argument against descriptivism about chance. They say that
chances should satisfy the following principle. (In this principle,
\emph{Ch} is the chance function, and various subscripts relativise it
to times and worlds.)

\begin{quote}
Suppose \emph{x} \textgreater{} 0 and
\emph{Ch\textsubscript{tw}}(\emph{A}) = \emph{x}. Then \emph{A} is true
in at least one of those worlds \emph{w}′ that matches \emph{w} up to
time \emph{t} and for which \emph{Ch\textsubscript{t}}(\emph{A}) =
\emph{x}. (\citeproc{ref-BigelowCollinsPargetter}{Bigelow, Collins, and
Pargetter 1993, 459})
\end{quote}

That is, if the chance of \emph{A} at \emph{t} is \emph{x}, and \emph{x}
\textgreater{} 0\emph{, then }A* could occur without changing the
history prior to \emph{t}, and without changing the chance of \emph{A}
at \emph{t}. This seems like a plausible principle of chance, but it
entails the not-so-Humean view that chances at \emph{t} supervene on
history to \emph{t}, not on the full state of the world.

Now as it turns out Lewis \emph{doesn't} rest on recombination arguments
against rival views of chance, and in my view he is wise to do so.
Instead he rests on epistemological arguments. He takes the following
two things to be data points.

\begin{enumerate}
\def\labelenumi{\arabic{enumi}.}
\tightlist
\item
  Something like the Principal Principle is true. The original Principal
  Principle said that if you knew the chance of \emph{p} at \emph{t} was
  \emph{x}, and didn't have any `inadmissible' information (roughly,
  information about how the world developed after \emph{t}), then your
  credence in \emph{p} should be \emph{x}. Lewis tinkered with this
  slightly, as we noted above, but he took it to be a requirement on a
  theory of chance that the Principal Principle turn out at least
  roughly right.
\item
  The correct theory of chance will \emph{explain} the Principal
  Principle.
\end{enumerate}

Lewis frequently wielded this second requirement against rival theories
of chance. Here's one example.

\begin{quote}
I can see, dimly, how it might be rational to conform my credences about
outcomes to my credencs about history, symmetries and frequencies. I
haven't the faintest notion how it might be rational to conform my
credences about outcomes to my credences about some mysterious unHumean
magnitude. Don't try to take away the mystery my saying that this
unHumean magnitude is none other than \emph{chance}!
(\citeproc{ref-Lewis1986b}{Lewis 1986b} xv)
\end{quote}

But this also seems like a weak argument. For one thing, chances are
actually correlated very well with frequencies, and this correlation
does not look at all accidental. It seems very plausible to me that we
should line up our credences with things that are actually correlated
well with frequencies. But, you might protest, shouldn't we have an
explanation of why the Principal Principle is an \emph{a priori}
principle of rationality? I think that before we ask for such an
explanation, we should check how confident we are that the Principal
Principle, or anything else, is part of an \emph{a priori} theory of
rationality. I'm not so confident that we'll be able to do this
(\citeproc{ref-WeathersonSRE}{Weatherson 2005},
\citeproc{ref-Weatherson2007}{2007}).

There are other replies too that we might make. It seems plausible that
we should minimise the expected inaccuracy of our credences
(\citeproc{ref-Joyce1998}{Joyce 1998}). This is true when we consider
not just the \emph{subjective} expected inaccuracy of our credences, but
the \emph{objective} expected inaccuracy of our credences. That is, when
we calculate the expected inaccuracy of someone's credences, using
chances as the probabilities for generating the expectations, it is good
if this expected inaccuracy is as low as possible. But, assuming that we
are using a proper scoring rule for measuring the accuracy of credences,
this means that we must have credences match chances.

More generally, I'm very sceptical of theories that insist our
metaphysics be designed to have complicated epistemological theses fall
out as immediate consequences. Rationality requires that we be
inductivists. Why is that? Here's a bad way to go about answering it:
find a theory of persistence that makes induction obviously rational,
and then require our metaphysics to conform to that theory. I don't
think you'll get a very good theory of persistence that way, and,
relatedly, you won't get a very Lewisian theory of persistence that way.
The demand that the theory of chance play a central role in an
explanation of the Principal Principle strikes me as equally mistaken.

If what I've been saying so far is correct, then chance interacts with
the motivation for Humean Supervenience~in very different ways to how
laws and causation interact. Neither of the two kinds of motivations for
defending Humean Supervenience~against philosophical attacks provides us
with good reason to leave chances out of the subvenient base on which we
say all contingent facts supervene. This is not to yet offer anything
like a positive argument for chances to be part of the fundamental
furniture of reality. Rather, what I've argued here is that a
metaphysics that takes chances as primitives would not be as far removed
from a recognisably Lewisian metaphysics as a metaphysics that takes
laws or causes as primitive, let alone one that takes mind, meanings or
morals as primitive.

\section{Points, Vectors and Lewis}\label{points-vectors-and-lewis}

The other main point from the discussion of the previous section is that
the fact that quantum mechanics raises problems for Humean
Supervenience~does not undercut the philosophical significance of
Lewis's defence of Humean Supervenience. But is Humean
Supervenience~even compatible with classical physics? Perhaps not.

\begin{quote}
Even classical electromagnetism raises a question for Humean
Supervenience~as I stated it. Denis Robinson
(\citeproc{ref-Robinson1989}{1989}) has asked: is a vector field an
arrangement of local qualities? I said qualities were intrinsic; that
means they can never differ between duplicates; and I would have said
offhand that two things can be duplicates even if they point in
different directions. May be this last opinion should be reconsidered,
so that vector-valued magnitudes may count as intrinsic properties. What
else could they be? Any attempt to reconstruct with them as relational
properties seems seriously artificial. (\citeproc{ref-Lewis1994a}{Lewis
1994a, 474})
\end{quote}

The opinion that the Lewis proposes to discard here seems more than an
offhand judgement. It seems to follow from the very way that we
introduce the notion of duplication. Here is Lewis's own attempt to
introduce the notion.

\begin{quote}
We are familiar with cases of approximate duplication, e.g., when we use
copying machines. And we understand that if these machines were more
perfect than they are, the copies they made would be perfect duplicates
of the original. Copy and original would be alike in size and shape and
chemical composition of the ink marks and the paper, alike in
temperature and magnetic alignment and electrostatic charge, alike even
in the exact arrangement of their electrons and quarks. Such duplicates
would be exactly alike we say. They would match perfectly, they would be
qualitatively identical, they would be indiscernible.
(\citeproc{ref-Lewis1983e}{Lewis 1983, 355})
\end{quote}

If Lewis is right that vector-valued magnitudes may count as intrinsic
properties, then there is yet another condition that the perfect copying
machine must satisfy. The original and the duplicate must be parallel.
This isn't the case in most actual copying machines. Usually, the
original is laid flat, while the duplicate is a small angle to make it
easier to collect. This is a feature, not a bug. It is not a way in
which the machine falls short of perfect copying. But if vector-valued
magnitudes are intrinsic qualities, and duplicates share their intrinsic
qualities, it would be. So Lewis is wrong to think that these
vector-valued magnitudes may be intrinsic.

Moreover, the little argument that Lewis gives seems to rest on a
category mistake. What matters here is the division of properties into
intrinsic and extrinsic. But the properties on the kind of things that
can be relational or non-relational. As Humberstone
(\citeproc{ref-Humberstone1996}{1996}) shows, concepts and not
properties of the things that can be relational and non-relational. For
instance the concept \emph{being the same shape as David Lewis actually
was at noon on January 1, 1970}, is a relational concept that presumably
picks out an intrinsic property, namely a shape property. Whether they
are valued magnitudes are intrinsic or extrinsic properties, is somewhat
orthogonal question of whether it is best to pick them out by means of
relational or non-relational concepts.

There is a further issue about the compatibility of Humean
Supervenience~with classical physics. This is a point that has been made
well by Jeremy Butterfield (\citeproc{ref-Butterfield2006}{2006}), and
we can see the problem by looking at the different ways in which Lewis
introduces Humean Supervenience.

\begin{quote}
Humean Supervenience~says that in a world like ours, the fundamental
properties are local qualities: perfectly natural intrinsic properties
of points, or of point-sized occupants of points.
(\citeproc{ref-Lewis1994a}{Lewis 1994a, 474})
\end{quote}

Lewis goes back and forth between local properties and intrinsic
properties of points here. These aren't the same thing. As Butterfield
notes, `local' is used in a few different ways throughout physics. One
simple usage identifies local properties of a point with properties that
supervene on intrinsic features of arbitrarily small regions around the
point. To take an important example, the slope of a curve at a point may
be a local property of the curve at that point without being intrinsic
property of the point.

This raises a question: can we do classical physics with only intrinsic
properties of points, and not even these further local properties?
Butterfield argues, persuasively, that the answer is no. He notes,
however, that there are some very mild weakenings of Humean
Supervenience~that avoid this difficulty. Here is a very simple one.

Call \textbf{Local Supervenience} the following thesis. For any length ε
greater than 0, there is a length \emph{d} less than ε with the
following feature. All the facts about the world supervene on intrinsic
features of objects and regions with diameter at most \emph{d}, plus
facts about the spatio-temporal arrangement of these objects and
regions. This will mean that we can include all local qualities in the
subvenient base, without assuming that these are intrinsic qualities of
points. If the theory of intrinsicness in Weatherson
(\citeproc{ref-Weatherson2006-WEATAM}{2006}) is correct, we'll also be
able to include vector-valued magnitudes in the subvenient base without
assuming that these are intrinsic properties of points. (On my view,
they will end up being intrinsic properties of asymmetrically shaped
regions.) We still won't be able to accommodate entanglement
relationships, but we will be able to capture classical physics. And,
for the reasons discussed in the previous section, it would still be
worthwhile to ask whether there are philosophical objections to Local
Supervenience. A negative answer would greatly assist the arguments for
compatibilism, and for nomological descriptivism.

Butterfield offers from theses like Local Supervenience to Lewis as
friendly suggestions. But he thinks Lewis's focus on points and their
properties would have led him to reject it. I don't want to get into the
business of making counterfactual speculation about what Lewis would or
would not have accepted. But I think he should have been happy to weaken
Humean Supervenience~to something like Local Supervenience. If the point
of defending Humean Supervenience~is not to defend its truth, but rather
to assist in larger arguments for compatibilism, and for nomological
descriptivism, then the big question to ask is whether a defence of
Local Supervenience (against distinctively philosophical objections)
would have served those causes just as well. And I think it's pretty
clear that it would have. Showing that we have no philosophical reason
to posit fundamental non-local features of reality would be enough to
let us ``resist philosophical arguments that there are more things in
heaven and earth in physics has dreamt of''
(\citeproc{ref-Lewis1994a}{Lewis 1994a, 474}). Lewis's work in defending
Humean Supervenience~has been invaluable to those of us who want to join
this resistance. It wouldn't have been undermined if he'd allowed some
local properties into the mix.

\subsection*{References}\label{references}
\addcontentsline{toc}{subsection}{References}

\phantomsection\label{refs}
\begin{CSLReferences}{1}{0}
\bibitem[\citeproctext]{ref-Bigelow1988}
Bigelow, John. 1988. \emph{The Reality of Numbers: A Physicalist's
Philosophy of Mathematics}. Oxford: Oxford.

\bibitem[\citeproctext]{ref-BigelowCollinsPargetter}
Bigelow, John, John Collins, and Robert Pargetter. 1993. {``The Big Bad
Bug: What Are the Humean's Chances?''} \emph{The British Journal for the
Philosophy of Science} 44 (3): 443--62. doi:
\href{https://doi.org/10.1093/bjps/44.3.443}{10.1093/bjps/44.3.443}.

\bibitem[\citeproctext]{ref-Butterfield2006}
Butterfield, Jeremy. 2006. {``Against Pointillisme about Mechanics.''}
\emph{British Journal for the Philosophy of Science} 57 (4): 709--53.
doi: \href{https://doi.org/10.1093/bjps/axl026}{10.1093/bjps/axl026}.

\bibitem[\citeproctext]{ref-Churchland1981}
Churchland, Paul. 1981. {``Eliminative Materialism and the Propositional
Attitudes.''} \emph{Journal of Philosophy} 78 (2): 67--90. doi:
\href{https://doi.org/10.2307/2025900}{10.2307/2025900}.

\bibitem[\citeproctext]{ref-Goodman1955}
Goodman, Nelson. 1955. \emph{Fact, Fiction and Forecast}. Cambridge:
Harvard University Press.

\bibitem[\citeproctext]{ref-Hall2010}
Hall, Ned. 2010. {``David Lewisś Metaphysics.''} In \emph{The Stanford
Encyclopedia of Philosophy}, edited by Edward N. Zalta, Fall 2010.
\url{http://plato.stanford.edu/archives/fall2010/entries/lewis-metaphysics/};
Metaphysics Research Lab, Stanford University.

\bibitem[\citeproctext]{ref-Haslanger1994}
Haslanger, Sally. 1994. {``Humean Supervenience and Enduring Things.''}
\emph{Australasian Journal of Philosophy} 72 (3): 339--59. doi:
\href{https://doi.org/10.1080/00048409412346141}{10.1080/00048409412346141}.

\bibitem[\citeproctext]{ref-Humberstone1996}
Humberstone, I. L. 1996. {``Intrinsic/Extrinsic.''} \emph{Synthese} 108
(2): 205--67. doi:
\href{https://doi.org/10.1007/bf00413498}{10.1007/bf00413498}.

\bibitem[\citeproctext]{ref-Jackson1998}
Jackson, Frank. 1998. \emph{From Metaphysics to Ethics: A Defence of
Conceptual Analysis}. Clarendon Press: Oxford.

\bibitem[\citeproctext]{ref-Joyce1998}
Joyce, James M. 1998. {``A Non-Pragmatic Vindication of Probabilism.''}
\emph{Philosophy of Science} 65 (4): 575--603. doi:
\href{https://doi.org/10.1086/392661}{10.1086/392661}.

\bibitem[\citeproctext]{ref-Lewis1973b}
Lewis, David. 1973. {``Causation.''} \emph{Journal of Philosophy} 70
(17): 556--67. doi:
\href{https://doi.org/10.2307/2025310}{10.2307/2025310}. Reprinted in
his \emph{Philosophical Papers}, Volume 2, Oxford: Oxford University
Press, 1986, 159-172. References to reprint.

\bibitem[\citeproctext]{ref-Lewis1983e}
---------. 1983. {``New Work for a Theory of Universals.''}
\emph{Australasian Journal of Philosophy} 61 (4): 343--77. doi:
\href{https://doi.org/10.1080/00048408312341131}{10.1080/00048408312341131}.
Reprinted in his \emph{Papers in Metaphysics and Epistemology},
Cambridge: Cambridge University Press, 1999, 8-55. References to
reprint.

\bibitem[\citeproctext]{ref-Lewis1986a}
---------. 1986a. \emph{On the Plurality of Worlds}. Oxford: Blackwell
Publishers.

\bibitem[\citeproctext]{ref-Lewis1986b}
---------. 1986b. \emph{Philosophical Papers}. Vol. II. Oxford: Oxford
University Press.

\bibitem[\citeproctext]{ref-Lewis1993b}
---------. 1993. {``Evil for Freedom's Sake?''} \emph{Philosophical
Papers} 22 (3): 149--72. doi:
\href{https://doi.org/10.1080/05568649309506401}{10.1080/05568649309506401}.
Reprinted in his \emph{Papers in Ethics and Social Philosophy},
Cambridge: Cambridge University Press, 2000, 101-127. References to
reprint.

\bibitem[\citeproctext]{ref-Lewis1994a}
---------. 1994a. {``Humean Supervenience Debugged.''} \emph{Mind} 103
(412): 473--90. doi:
\href{https://doi.org/10.1093/mind/103.412.473}{10.1093/mind/103.412.473}.
Reprinted in his \emph{Papers in Metaphysics and Epistemology},
Cambridge: Cambridge University Press, 1999, 224-247. References to
reprint.

\bibitem[\citeproctext]{ref-Lewis1994b}
---------. 1994b. {``Reduction of Mind.''} In \emph{A Companion to the
Philosophy of Mind}, edited by Samuel Guttenplan, 412--31. Oxford:
Blackwell. doi:
\href{https://doi.org/10.1017/CBO9780511625343.019}{10.1017/CBO9780511625343.019}.
Reprinted in his \emph{Papers in Metaphysics and Epistemology}, 1999,
291-324, Cambridge: Cambridge University Press. References to reprint.

\bibitem[\citeproctext]{ref-Lewis2002b}
---------. 2002. {``Tharp's Third Theorem.''} \emph{Analysis} 62 (2):
95--97. doi:
\href{https://doi.org/10.1093/analys/62.2.95}{10.1093/analys/62.2.95}.

\bibitem[\citeproctext]{ref-Lewis2004a}
---------. 2004a. {``Causation as Influence.''} In \emph{Causation and
Counterfactuals}, edited by John Collins, Ned Hall, and L. A. Paul,
75--106. Cambridge: {MIT} Press.

\bibitem[\citeproctext]{ref-Lewis2004b}
---------. 2004b. {``How Many Lives Has {S}chrödinger's Cat?''}
\emph{Australasian Journal of Philosophy} 82 (1): 3--22. doi:
\href{https://doi.org/10.1080/713659799}{10.1080/713659799}.

\bibitem[\citeproctext]{ref-Lewis2004d}
---------. 2004c. {``Void and Object.''} In \emph{Causation and
Counterfactuals}, edited by John Collins, Ned Hall, and L. A. Paul,
277--90. Cambridge: {MIT} Press.

\bibitem[\citeproctext]{ref-Maudlin1994}
Maudlin, Tim. 1994. \emph{Quantum Non-Locality and Relativity:
Metaphysical Intimations of Modern Physics}. Oxford: Blackwell.

\bibitem[\citeproctext]{ref-Menzies1996}
Menzies, Peter. 1996. {``Probabilistic Causation and the Pre-Emption
Problem.''} \emph{Mind} 105 (417): 85--117. doi:
\href{https://doi.org/10.1093/mind/105.417.85}{10.1093/mind/105.417.85}.

\bibitem[\citeproctext]{ref-Nolan1996-NOLRU}
Nolan, Daniel. 1996. {``{Recombination Unbound}.''} \emph{Philosophical
Studies} 84 (2-3): 239--62. doi:
\href{https://doi.org/10.1007/BF00354489}{10.1007/BF00354489}.

\bibitem[\citeproctext]{ref-Pryor2000}
Pryor, James. 2000. {``The Sceptic and the Dogmatist.''} \emph{No{û}s}
34 (4): 517--49. doi:
\href{https://doi.org/10.1111/0029-4624.00277}{10.1111/0029-4624.00277}.

\bibitem[\citeproctext]{ref-Robinson1989}
Robinson, Denis. 1989. {``Matter, Motion and Humean Supervenience.''}
\emph{Australasian Journal of Philosophy} 67 (4): 394--409. doi:
\href{https://doi.org/10.1080/00048408912343921}{10.1080/00048408912343921}.

\bibitem[\citeproctext]{ref-Schwarz2009}
Schwarz, Wolfgang. 2009. \emph{David Lewis: Metaphysik Und Analyse}.
Paderborn: Mentis-Verlag.

\bibitem[\citeproctext]{ref-SiderDiss}
Sider, Theodore. 1993. {``Naturalness, Intrinsicality and
Duplication.''} PhD thesis, University of Massachusetts - Amherst.

\bibitem[\citeproctext]{ref-Sider2001-SIDCOP}
---------. 2001. {``Criteria of Personal Identity and the Limits of
Conceptual Analysis.''} \emph{Philosophical Perspectives} 15: 189--209.
doi:
\href{https://doi.org/10.1111/0029-4624.35.s15.10}{10.1111/0029-4624.35.s15.10}.

\bibitem[\citeproctext]{ref-Strawson2000}
Strawson, Galen. 2000. {``David Hume: Objects and Power.''} In \emph{The
New Hume Debate}, edited by Rupert Read and Kenneth A. Richman, 31--51.
London: Routledge.

\bibitem[\citeproctext]{ref-Weatherson2003-WEAWGA}
Weatherson, Brian. 2003. {``{What Good Are Counterexamples?}''}
\emph{Philosophical Studies} 115 (1): 1--31. doi:
\href{https://doi.org/10.1023/A:1024961917413}{10.1023/A:1024961917413}.

\bibitem[\citeproctext]{ref-WeathersonSRE}
---------. 2005. {``Scepticism, Rationalism and Externalism.''}
\emph{Oxford Studies in Epistemology} 1: 311--31.

\bibitem[\citeproctext]{ref-Weatherson2006-WEATAM}
---------. 2006. {``{The Asymmetric Magnets Problem}.''}
\emph{Philosophical Perspectives} 20: 479--92. doi:
\href{https://doi.org/10.1111/j.1520-8583.2006.00116.x}{10.1111/j.1520-8583.2006.00116.x}.

\bibitem[\citeproctext]{ref-Weatherson2007}
---------. 2007. {``The Bayesian and the Dogmatist.''} \emph{Proceedings
of the Aristotelian Society} 107: 169--85. doi:
\href{https://doi.org/10.1111/j.1467-9264.2007.00217.x}{10.1111/j.1467-9264.2007.00217.x}.

\bibitem[\citeproctext]{ref-Weatherson2009-WEADL}
---------. 2009. {``{David Lewis}.''} In \emph{Stanford Encyclopedia of
Philosophy}, edited by Edward N. Zalta. Metaphysics Research Lab,
Stanford University.

\end{CSLReferences}



\noindent Published in\emph{
A Companion to David Lewis}, 2015, pp. 99-105.


\end{document}
