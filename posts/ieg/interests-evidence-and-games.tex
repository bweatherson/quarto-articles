% Options for packages loaded elsewhere
% Options for packages loaded elsewhere
\PassOptionsToPackage{unicode}{hyperref}
\PassOptionsToPackage{hyphens}{url}
%
\documentclass[
  11pt,
  letterpaper,
  DIV=11,
  numbers=noendperiod,
  twoside]{scrartcl}
\usepackage{xcolor}
\usepackage[left=1.1in, right=1in, top=0.8in, bottom=0.8in,
paperheight=9.5in, paperwidth=7in, includemp=TRUE, marginparwidth=0in,
marginparsep=0in]{geometry}
\usepackage{amsmath,amssymb}
\setcounter{secnumdepth}{3}
\usepackage{iftex}
\ifPDFTeX
  \usepackage[T1]{fontenc}
  \usepackage[utf8]{inputenc}
  \usepackage{textcomp} % provide euro and other symbols
\else % if luatex or xetex
  \usepackage{unicode-math} % this also loads fontspec
  \defaultfontfeatures{Scale=MatchLowercase}
  \defaultfontfeatures[\rmfamily]{Ligatures=TeX,Scale=1}
\fi
\usepackage{lmodern}
\ifPDFTeX\else
  % xetex/luatex font selection
  \setmainfont[ItalicFont=EB Garamond Italic,BoldFont=EB Garamond
Bold]{EB Garamond Math}
  \setsansfont[]{EB Garamond}
  \setmathfont[]{Garamond-Math}
\fi
% Use upquote if available, for straight quotes in verbatim environments
\IfFileExists{upquote.sty}{\usepackage{upquote}}{}
\IfFileExists{microtype.sty}{% use microtype if available
  \usepackage[]{microtype}
  \UseMicrotypeSet[protrusion]{basicmath} % disable protrusion for tt fonts
}{}
\usepackage{setspace}
% Make \paragraph and \subparagraph free-standing
\makeatletter
\ifx\paragraph\undefined\else
  \let\oldparagraph\paragraph
  \renewcommand{\paragraph}{
    \@ifstar
      \xxxParagraphStar
      \xxxParagraphNoStar
  }
  \newcommand{\xxxParagraphStar}[1]{\oldparagraph*{#1}\mbox{}}
  \newcommand{\xxxParagraphNoStar}[1]{\oldparagraph{#1}\mbox{}}
\fi
\ifx\subparagraph\undefined\else
  \let\oldsubparagraph\subparagraph
  \renewcommand{\subparagraph}{
    \@ifstar
      \xxxSubParagraphStar
      \xxxSubParagraphNoStar
  }
  \newcommand{\xxxSubParagraphStar}[1]{\oldsubparagraph*{#1}\mbox{}}
  \newcommand{\xxxSubParagraphNoStar}[1]{\oldsubparagraph{#1}\mbox{}}
\fi
\makeatother


\usepackage{longtable,booktabs,array}
\usepackage{calc} % for calculating minipage widths
% Correct order of tables after \paragraph or \subparagraph
\usepackage{etoolbox}
\makeatletter
\patchcmd\longtable{\par}{\if@noskipsec\mbox{}\fi\par}{}{}
\makeatother
% Allow footnotes in longtable head/foot
\IfFileExists{footnotehyper.sty}{\usepackage{footnotehyper}}{\usepackage{footnote}}
\makesavenoteenv{longtable}
\usepackage{graphicx}
\makeatletter
\newsavebox\pandoc@box
\newcommand*\pandocbounded[1]{% scales image to fit in text height/width
  \sbox\pandoc@box{#1}%
  \Gscale@div\@tempa{\textheight}{\dimexpr\ht\pandoc@box+\dp\pandoc@box\relax}%
  \Gscale@div\@tempb{\linewidth}{\wd\pandoc@box}%
  \ifdim\@tempb\p@<\@tempa\p@\let\@tempa\@tempb\fi% select the smaller of both
  \ifdim\@tempa\p@<\p@\scalebox{\@tempa}{\usebox\pandoc@box}%
  \else\usebox{\pandoc@box}%
  \fi%
}
% Set default figure placement to htbp
\def\fps@figure{htbp}
\makeatother


% definitions for citeproc citations
\NewDocumentCommand\citeproctext{}{}
\NewDocumentCommand\citeproc{mm}{%
  \begingroup\def\citeproctext{#2}\cite{#1}\endgroup}
\makeatletter
 % allow citations to break across lines
 \let\@cite@ofmt\@firstofone
 % avoid brackets around text for \cite:
 \def\@biblabel#1{}
 \def\@cite#1#2{{#1\if@tempswa , #2\fi}}
\makeatother
\newlength{\cslhangindent}
\setlength{\cslhangindent}{1.5em}
\newlength{\csllabelwidth}
\setlength{\csllabelwidth}{3em}
\newenvironment{CSLReferences}[2] % #1 hanging-indent, #2 entry-spacing
 {\begin{list}{}{%
  \setlength{\itemindent}{0pt}
  \setlength{\leftmargin}{0pt}
  \setlength{\parsep}{0pt}
  % turn on hanging indent if param 1 is 1
  \ifodd #1
   \setlength{\leftmargin}{\cslhangindent}
   \setlength{\itemindent}{-1\cslhangindent}
  \fi
  % set entry spacing
  \setlength{\itemsep}{#2\baselineskip}}}
 {\end{list}}
\usepackage{calc}
\newcommand{\CSLBlock}[1]{\hfill\break\parbox[t]{\linewidth}{\strut\ignorespaces#1\strut}}
\newcommand{\CSLLeftMargin}[1]{\parbox[t]{\csllabelwidth}{\strut#1\strut}}
\newcommand{\CSLRightInline}[1]{\parbox[t]{\linewidth - \csllabelwidth}{\strut#1\strut}}
\newcommand{\CSLIndent}[1]{\hspace{\cslhangindent}#1}



\setlength{\emergencystretch}{3em} % prevent overfull lines

\providecommand{\tightlist}{%
  \setlength{\itemsep}{0pt}\setlength{\parskip}{0pt}}



 


\setlength\heavyrulewidth{0ex}
\setlength\lightrulewidth{0ex}
\usepackage[automark]{scrlayer-scrpage}
\clearpairofpagestyles
\cehead{
  Brian Weatherson
  }
\cohead{
  Interests, Evidence and Games
  }
\ohead{\bfseries \pagemark}
\cfoot{}
\makeatletter
\newcommand*\NoIndentAfterEnv[1]{%
  \AfterEndEnvironment{#1}{\par\@afterindentfalse\@afterheading}}
\makeatother
\NoIndentAfterEnv{itemize}
\NoIndentAfterEnv{enumerate}
\NoIndentAfterEnv{description}
\NoIndentAfterEnv{quote}
\NoIndentAfterEnv{equation}
\NoIndentAfterEnv{longtable}
\NoIndentAfterEnv{abstract}
\renewenvironment{abstract}
 {\vspace{-1.25cm}
 \quotation\small\noindent\emph{Abstract}:}
 {\endquotation}
\newfontfamily\tfont{EB Garamond}
\addtokomafont{disposition}{\rmfamily}
\addtokomafont{title}{\normalfont\itshape}
\let\footnoterule\relax
\KOMAoption{captions}{tableheading}
\makeatletter
\@ifpackageloaded{caption}{}{\usepackage{caption}}
\AtBeginDocument{%
\ifdefined\contentsname
  \renewcommand*\contentsname{Table of contents}
\else
  \newcommand\contentsname{Table of contents}
\fi
\ifdefined\listfigurename
  \renewcommand*\listfigurename{List of Figures}
\else
  \newcommand\listfigurename{List of Figures}
\fi
\ifdefined\listtablename
  \renewcommand*\listtablename{List of Tables}
\else
  \newcommand\listtablename{List of Tables}
\fi
\ifdefined\figurename
  \renewcommand*\figurename{Figure}
\else
  \newcommand\figurename{Figure}
\fi
\ifdefined\tablename
  \renewcommand*\tablename{Table}
\else
  \newcommand\tablename{Table}
\fi
}
\@ifpackageloaded{float}{}{\usepackage{float}}
\floatstyle{ruled}
\@ifundefined{c@chapter}{\newfloat{codelisting}{h}{lop}}{\newfloat{codelisting}{h}{lop}[chapter]}
\floatname{codelisting}{Listing}
\newcommand*\listoflistings{\listof{codelisting}{List of Listings}}
\makeatother
\makeatletter
\makeatother
\makeatletter
\@ifpackageloaded{caption}{}{\usepackage{caption}}
\@ifpackageloaded{subcaption}{}{\usepackage{subcaption}}
\makeatother
\usepackage{bookmark}
\IfFileExists{xurl.sty}{\usepackage{xurl}}{} % add URL line breaks if available
\urlstyle{same}
\hypersetup{
  pdftitle={Interests, Evidence and Games},
  pdfauthor={Brian Weatherson},
  hidelinks,
  pdfcreator={LaTeX via pandoc}}


\title{Interests, Evidence and Games}
\author{Brian Weatherson}
\date{2018}
\begin{document}
\maketitle
\begin{abstract}
Pragmatic encroachment theories have a problem with evidence. On the one
hand, the arguments that knowledge is interest-relative look like they
will generalise to show that evidence too is interest-relative. On the
other hand, our best story of how interests affect knowledge presupposes
an interest-invariant notion of evidence. The aim of this paper is to
sketch a theory of evidence that is interest-relative, but which allows
that `best story' to go through with minimal changes. The core idea is
that the evidence someone has is just what evidence a radical
interpreter says they have. And a radical interpreter is playing a kind
of game with the person they are interpreting. The cases that pose
problems for pragmatic encroachment theorists generate fascinating games
between the interpreter and the interpretee. They are games with
multiple equilibria. To resolve them we need to detour into the theory
of equilibrium selection. I'll argue that the theory we need is the
theory of \textbf{risk-dominant equilibria}. That theory will tell us
how the interpreter will play the game, which in turn will tell us what
evidence the person has. The evidence will be interest-relative, because
what the equilibrium of the game is will be interest-relative. But it
will not undermine the story we tell about how interests usually affect
knowledge.
\end{abstract}


\setstretch{1.1}
Pragmatic encroachment theories have a problem with evidence. On the one
hand, the arguments that knowledge is interest-relative look like they
will generalise to show that evidence too is interest-relative. On the
other hand, our best story of how interests affect knowledge presupposes
an interest-invariant notion of evidence.

The aim of this paper is to sketch a theory of evidence that is
interest-relative, but which allows that `best story' to go through with
minimal changes. The core idea is that the evidence someone has is just
what evidence a radical interpreter says they have. And a radical
interpreter is playing a kind of game with the person they are
interpreting. The cases that pose problems for pragmatic encroachment
theorists generate fascinating games between the interpreter and the
interpretee. They are games with multiple equilibria. To resolve them we
need to detour into the theory of equilibrium selection. I'll argue that
the theory we need is the theory of \textbf{risk-dominant equilibria}.
That theory will tell us how the interpreter will play the game, which
in turn will tell us what evidence the person has. The evidence will be
interest-relative, because what the equilibrium of the game is will be
interest-relative. But it will not undermine the story we tell about how
interests usually affect knowledge.

\section{Encroachment, Reduction and
Explanation}\label{encroachmentreductionandexplanation}

I will start with an argument for a familiar disjunctive conclusion:
either knowledge is interest-relative, or scepticism is true. The
argument will resemble arguments to the same disjunctive conclusion in
Hawthorne (\citeproc{ref-Hawthorne2004}{2004}) and Fantl and McGrath
(\citeproc{ref-FantlMcGrath2009}{2009}). Indeed, it is inspired by those
discussions. But it uses less controversial premises than previous
versions.

The argument starts by considering a game, one I'll call the red-blue
game. Here are the rules of the game.

\begin{enumerate}
\def\labelenumi{\arabic{enumi}.}
\tightlist
\item
  Two sentences will be written on the board, one in red, one in blue.
\item
  The player will make two choices.
\item
  First, they will pick a colour, red or blue.
\item
  Second, they say whether the sentence in that colour is true or false.
\item
  If they are right, they win. If not, they lose.
\item
  If they win, they get \$50, and if they lose, they get nothing.
\end{enumerate}

Our player is Parveen. She is an epistemologist who works on pragmatic
encroachment, and (as will become important in a minute), she has
frequently cited both \emph{Knowledge and Lotteries}
(\citeproc{ref-Hawthorne2004}{Hawthorne 2004}), and \emph{Knowledge and
Practical Interests} (\citeproc{ref-Stanley2005}{Stanley 2005}). She
knows the rules of the game, and no other relevant facts about the game.
When the game starts, the following two sentences are written on the
board, the first in red, the second in blue.

\begin{enumerate}
\def\labelenumi{\arabic{enumi}.}
\tightlist
\item
  Two plus two equals four.
\item
  \emph{Knowledge and Lotteries} was published before \emph{Knowledge
  and Practical Interests}.
\end{enumerate}

Intuitively, there is a unique rational play in this game: Red-True.
That is, Parveen announces that she will evaluate the truth value of the
red sentence, and then announce that it's true. That's a sure \$50.

On the other hand, in normal circumstances, we would say that Parveen
does know that \emph{Knowledge and Lotteries} was published before
\emph{Knowledge and Practical Interests}. After all, she has looked up
their publication dates many times in checking over her papers.

There is a puzzle in reconciling these intuitions. The pragmatic
encroachment theorist has a solution to these puzzles. In normal
circumstances, Parveen does know that \emph{Knowledge and Lotteries} was
published before \emph{Knowledge and Practical Interests}. But these are
not normal circumstances. Right now, it matters whether her reason to
believe that \emph{Knowledge and Lotteries} was published before
\emph{Knowledge and Practical Interests} is as strong as her reason to
believe that two plus two equals four. And (unless something very weird
is happening), that isn't true for Parveen. So she knows that red-true
will win, she doesn't know any other play will win, so she should play
Red-True.

If we reject pragmatic encroachment, and we are not sceptics, we should
say that Parveen does know that \emph{Knowledge and Lotteries} was
published before \emph{Knowledge and Practical Interests}. And then it
is a mystery why playing Red-True is more rational than playing
Blue-True. After all, Parveen knows the rules of the game, and she knows
(by hypothesis) the blue sentence is true, so if she can do even basic
logical reasoning in a knowledge preserving way, she knows she will get
as good a result as possible by playing Blue-True. So it is a bit of a
mystery why it would be anything other than maximally rational to play
Blue-True.

One way we might try to resolve this mystery is by saying that although
Parveen knows that Blue-True will win \$50, she super-knows that
Red-True will win \$50. What do we mean here by \emph{super knowledge}?
Think of this as a placeholder for certainty, or knowledge that one
knows, or anything other epistemic state that you think might be
relevant to her practical decision making. Perhaps the fact that she
super-knows what two plus two is, but doesn't super-know when the
epistemology books were published, could be the explanation for why
Red-True is the unique rational play.\footnote{That we need some kind of
  super-knowledge for action, and not mere knowledge, is a popular, and
  natural, explanation of the case. For versions of this explanation,
  obviously with more details than I've given here, see for example
  Jessica Brown (\citeproc{ref-Brown2008}{2008}) and Jennifer Lackey
  (\citeproc{ref-Lackey2010}{2010}).}

But no such explanation can work, because Parveen doesn't super-know
that playing Red-True will win \$50. She super-knows that two plus two
is four. But we have not assumed that she super-knows the rules of the
game. So she doesn't super-know that Red-True will win, she just knows
it. And she also, by hypothesis, knows that Blue-True will win. So
looking at any kind of super-knowledge can't break the intuitive
asymmetry between Red-True and Blue-True.

Put another way, if Parveen knows that \emph{Knowledge and Lotteries}
was published before \emph{Knowledge and Practical Interests}, then she
knows that she is playing the following game.

\begin{enumerate}
\def\labelenumi{\arabic{enumi}.}
\tightlist
\item
  Two sentences will be written on the board, one in red, one in blue.
\item
  The player chooses to play either Blue-True, Blue-False, Red-True, or
  Red-False.
\item
  If they play Blue-True, they win \$50.
\item
  If they play Blue-False, they win nothing.
\item
  If they play Red-True, they win \$50 if the red sentence is true, and
  nothing otherwise.
\item
  If they play Red-False, they win \$50 if the red sentence is false,
  and nothing otherwise.
\end{enumerate}

And is is rational to play Blue-True in that game. (It might also be
rational to pay Red-True depending on what the red sentence is, but it
is always rational to play Blue-True.) Yet it is not rational to play
Blue-True in the original game. So Parveen does not know, when she plays
the original game, that \emph{Knowledge and Lotteries} was published
before \emph{Knowledge and Practical Interests}.

So to avoid pragmatic encroachment here we must deny that Parveen ever
knew that \emph{Knowledge and Lotteries} was published before
\emph{Knowledge and Practical Interests}. On its own, that's not a
sceptical conclusion: lots of people don't know that. But once we go
down that path, it looks like not much knowledge will be left. After
all, we can repeat the game with any number of different things in the
place of the blue sentence. If we adopt the constraint that Parveen only
knows \emph{p}, right now, if it is rationally permissible for her to
play Blue-True when \emph{p} is the blue sentence, no matter what the
red sentence is, then either we have to say very unintuitive things
about rational plays of the game, or we have to say she knows very
little.

So we've got the conclusion that either pragmatic encroachment is true,
or scepticism is true. Since I'm not a sceptic, I'm happy to conclude
that pragmatic encroachment is true. But note that we've done this
without any reference to high stakes situations. The stakes in Parveen's
game are just \$50. That's not nothing, but it's not `high stakes' in
the way that phrase is normally used.

The version of pragmatic encroachment we get is that what matters for
knowledge are not the stakes involved in any bet on \emph{p}, but the
odds.\footnote{Jessica Brown (\citeproc{ref-Brown2008}{2008, 176}) shows
  that pragmatic encroachment theories that rely just on the stakes
  involved are subject to serious counterexample. Katherine Rubin
  (\citeproc{ref-Rubin2015}{2015}) argues that if we have a `global'
  version of pragmatic encroachment, where all our epistemic notions are
  interest-relative, then it is implausible that it is the stakes the
  subject faces that matter for knowledge. Since I'm defending such a
  global version of pragmatic encroachment, Rubin's arguments show that
  it is important that I'm relying on odds, not stakes. Baron Reed
  (\citeproc{ref-Reed2014}{2014}) argues that if it is stakes alone that
  matter to pragmatic encroachment, then agents who the pragmatic
  encroachment theorist takes to be perfectly rational would be subject
  to a Dutch Book.} Parveen loses knowledge because she is being asked,
in effect, to make a super long odds bet on a fact about publication
schedules. She is in no position to rationally make a bet at those odds.
So she doesn't know the fact about publication schedules.

And that's the general principle: agents only know a proposition if they
are in a position to rationally bet on that proposition at the odds
currently being offered to them. In practice, high stakes situations
tend to feature bets at long odds, so in practice much knowledge
dissipates in high stakes cases. But the explanation of the dissipation
is the odds the agent faces, not the stakes.

More precisely, I endorse these principles as constraints on knowledge:

\begin{itemize}
\tightlist
\item
  If the agent knows that \emph{p}, then for any question they have an
  interest in, the answer to that question is identical to the answer to
  that question conditional on \emph{p}.
\item
  When an agent is considering the choice between two options, the
  question of which option has a higher expected utility given their
  evidence is a question they have an interest in.
\end{itemize}

Those principles are meant to not merely be extensionally adequate. They
are meant to explain why agents lose knowledge when considering some
sets of options, like in the Red-Blue game. In some sense, they are
meant to be part of reductive explanations. These reductive explanations
take as primitive inputs facts about the agent's evidence, and facts
about evidential probability. I'm going to set aside worries about the
metaphysics of evidential probability, and just focus on evidence.
Because it turns out that there is a real problem in getting a plausible
theory of evidence that can function as an input to that reductive
explanation.

\section{The Problems with Evidence}\label{theproblemswithevidence}

Go back to the red-blue game. Consider a version of the game where:

\begin{itemize}
\tightlist
\item
  The red sentence is that two plus two equals four.
\item
  The blue sentence is something that, if known, would be part of the
  agent's evidence.
\end{itemize}

I'm going to argue that there are cases where the only rational play is
Red-True, but the blue sentence is something we want to say that,
ordinarily, the subject knows. And I'll argue that this is a problem for
the kind of reductive explanation I just sketched. If pragmatic effects
matter to what the evidence is, we can't take the evidence as a fixed
input into an explanation of how and when pragmatic effects matter.

Let's have Parveen play the game again. She's going to be playing the
game in a restaurant, one in Ann Arbor where she lives. Just before the
game starts, she notices an old friend, Rahul, across the room. Rahul is
someone she knows well, and can ordinarily recognise, but she had no
idea he was in town. She thought Rahul was living in Italy. Still, we
would ordinarily say that she now knows Rahul is in the restaurant;
indeed that he is in the restaurant. It would be perfectly acceptable
for her to say to someone else, ``I saw Rahul here'', for example. Now
the game starts.

\begin{itemize}
\tightlist
\item
  The red sentence is \emph{Two plus two equals four}.
\item
  The blue sentence is \emph{Rahul is in this restaurant}.
\end{itemize}

Now we have a problem. On the one hand, there is only one rational play
here: Red-True. If you haven't seen someone for a long time, then you
can't be completely certain it's them when you spot them across a
restaurant. It would be foolish to be as confident that it's Rahul as
that two and two make four. It looks like this is a case where pragmatic
effects defeat knowledge.

On the other hand, our story for why Parveen loses knowledge here has
run into problems. I wanted to tell a story roughly like the following.
She can't play Blue-True when the probability of the blue sentence,
given her evidence, is less than the probability of the red sentence,
given her evidence. That explanation can only go through if the blue
sentence is itself not part of her evidence, since the probability of
anything given itself is one. So we need a story about how it is that it
is not part of Parveen's evidence that Rahul is not in the restaurant.

That story can't be the one that presupposes facts about what is in
Parveen's evidence. So it can't use facts about the probability of some
proposition given her evidence; at least not in any simple way. If we
can independently identify Parveen's evidence, then we can go back to
using evidential probability. But until we've done that, we're stuck.

There are two options here that seem possible for the pragmatic
encroachment theorist, but not particularly attractive.

One is to say that propositions like \emph{Rahul is in this restaurant}
are never part of Parveen's evidence. Perhaps her evidence just consists
of things like \emph{I am being appeared to Rahul-like}. Such an
approach is problematic for two reasons. The first is that it is subject
to all the usual objections to psychological theories of evidence
(\citeproc{ref-Williamson2007-WILTPO-17}{Williamson 2007}). The second
is that we can re-run the argument with the blue sentence being some
claim about Parveen's psychological state, and still get the result that
the only rational play is Red-True. A retreat to a psychological
conception of evidence will only help with this problem if agents are
infallible judges of their own psychological states, and that is not in
general true (\citeproc{ref-Schwitzgebel2008}{Schwitzgebel 2008}).

Another option is to deny that a reductive explanation is needed here.
Perhaps pragmatic effects, like the particular sentences that are chosen
for this instance of the Red-Blue game, mean that Parveen's evidence no
longer includes facts about Rahul, but this isn't something we can give
a reductive account of. We shouldn't assume that everything will have a
simple reductive explanation, so this isn't so bad in theory. The
problem in practice is that without a reductive explanation, we don't
have a predictive theory of when pragmatic effects matter. And that
seems to be a bad thing. For instance, the following theory is
completely consistent with Parveen's case as described.

\begin{enumerate}
\def\labelenumi{\arabic{enumi}.}
\tightlist
\item
  E=K; i.e., one's evidence is all and only what one knows.
\item
  Someone does not know \emph{p} if the evidential probability of
  \emph{p} is not close enough to one for current purposes.
\item
  Since it is part of Parveen's evidence that Rahul is in the
  restaurant, the probability that he is there is one, so it is close
  enough to one for current purposes.
\item
  So this is not a case where pragmatic effects change what she knows.
\end{enumerate}

That theory seems to me to be badly mistaken, since it goes on to
predict that it is rationally permissible to play Blue-True. But we need
a pragmatic account that says that it is mistaken, and says something
about which alternative situations would not threaten Parveen's
knowledge. We don't yet, as far as I can see, have such an account. The
aim of the rest of this paper is to provide one.\footnote{You can read
  this paper as a reply to the challenge posed by Ichikawa, Jarvis, and
  Rubin (\citeproc{ref-IchikawaEtAl2012}{2012}). They note that there
  are challenges facing the pragmatic encroachment theorist whether they
  make evidence interest-relative, or interest-invariant. I'm going to
  show how to have an interest-relative theory of evidence, and keep
  what was desirable about pragmatic encroachment theories.}

\section{A Simple, but Unsatisfying,
Solution}\label{asimplebutunsatisfyingsolution}

Let's take a step back and look at the puzzle more abstractly. We have
an agent \emph{S}, who has some option \emph{O}, and it really matters
whether or not the value of \emph{O}, i.e., \emph{V}(\emph{O}) is at
least \emph{x}. It is uncontroversial that the agent's evidence includes
some background \emph{K}, and controversial whether it includes some
contested proposition \emph{p}. It is also uncontroversial that
\emph{V}(\emph{O} \textbar{} \emph{p})~⩾~\emph{x}, and we're assuming
that for any proposition \emph{q} that is in the agent's evidence,
\emph{V}(\emph{O}~\textbar~\emph{q})~=~\emph{V}(\emph{O}). That is,
we're assuming the relevant values are conditional on evidence. We can
capture that last assumption with one big assumption that probably isn't
true, but is a harmless idealisation for these purposes. Say there is a
prior value function \emph{V}\textsuperscript{-}, with a similar
metaphysical status to the mythical, mystical prior probability
function. Then for any choice \emph{C},
\emph{V}(\emph{C})~=~\emph{V}\textsuperscript{-}(\emph{C} \textbar{}
\emph{E}), where \emph{E} is the evidence the agent has.

Now we're in a position to state a simple, but unsatisfying, solution.
Let \emph{p} be the proposition that the agent might or might not know,
and the question of whether \emph{V}(\emph{O})~⩾~\emph{x} be the only
salient one that \emph{p} is relevant to. Then the agent knows \emph{p}
only if the following is true:

\[
\frac{V^-(O | K) + V^-(O | K \wedge p)}{2} ⩾ x
\]

That is, we work out the value of \emph{O} with and without the evidence
\emph{p}, and if the average is greater than \emph{x}, good enough!

That solves the problem of Parveen and Rahul. Parveen's evidence may or
may not include that Rahul is in the restaurant. If it does, then
Blue-True has a value of \$50. If it does not, then Blue-True's value is
somewhat lower. Even if the evidence includes that someone who looks a
lot like Rahul is in the restaurant, the value of Blue-True might only
be \$45. Averaging them out, the value is less than \$50. But you'd only
play Blue-True if it was worthwhile it play it instead of Red-True,
which is worth \$50. So you shouldn't play Blue-True.

Great! Well, great except for two monumental problems. The first problem
is that what we've said here really only helps with very simple cases,
where there is a single decision problem that a single contested
proposition is relevant to. We need some way to generalise the case to
less constrained situations. The second (and bigger) problem is that the
solution is completely ad hoc. Why should we use the arithmetic mean of
these two things rather than any other formula that would have implied
the intuitively correct result in the Parveen-Rahul case? Pragmatic
encroachment starts with a very elegant, very intuitive, principle: you
only know the things you can reasonable take to be settled for the
purposes of current deliberation. And that deliberation should be driven
by the aim of maximising expected utility. It does not look like any
such elegant, intuitive, principles will lead to some theorem about
averaging out the value of an option with and without new evidence.

Happily, the two problems have a common solution. But the solution
requires a detour into some technical work. It's time for some game
theory.

\section{Gamifying the Problem}\label{gamifyingtheproblem}

We can usefully think of some philosophical problems as games, and hence
subjects for study using game theoretic techniques. This is especially
when the problems involve interactions of rational agents. Here, for
example, is the game table for Newcomb's problem, with the human who is
usually the focus of the problem as Row, and the demon as
Column.\footnote{In these games, Row chooses a row, and Column chooses a
  column, and that determines the cell that is the outcome of the game.
  The cells include two numbers. The first is Row's payout, and the
  second is Column's. The games are non-competitive; the players are
  simply trying to maximise their own returns, not maximise the
  difference between their return and the other player's return.}

\begin{longtable}[]{@{}lcc@{}}
\toprule\noalign{}
& Predict 1 Box & Predict 2 Boxes \\
\midrule\noalign{}
\endhead
\bottomrule\noalign{}
\endlastfoot
Choose 1 Box & 1000, 1 & 0,0 \\
Choose 2 Boxes & 1001, 0 & 1, 1 \\
\end{longtable}

This game has a unique Nash equilbrium; the bottom right
corner.\footnote{A Nash equilibrium is an outcome of the game where
  every player does as well as they can given the moves of the other
  players. Equivalently, it is an outcome where no player can improve
  their payout by unilaterally defecting from the equilibrium.} And
that's one way of motivating the view that (a) the game is possible, and
(b) the rational move for the human is to choose two boxes.

Let's look at a more complicated game. I'll call it The Interpretation
Game. The game has two players. Just like in Newcomb's problem, one of
them is a human, the other is a philosophical invention. But in this
case the invention is not a demon, but The Radical
Interpreter.\footnote{The Radical Interpreter feels like they should be
  a humanesque character in \emph{Alice in Wonderland} or \emph{The
  Phantom Tollbooth}, but for now they are resolutely abstract.} To know
the payouts for the players, we need to know their value function. More
colloquially, we need to know their goals.

\begin{itemize}
\tightlist
\item
  The Radical Interpreter assigns mental states to Human in such a way
  as to predict Human's actions given Human rationality. We'll assume
  here that evidence is a mental state, so saying what evidence Human
  has is among Radical Interpreter's tasks. (Indeed, in the game play to
  come, it will be their primary task.)
\item
  Human acts so as to maximise the expected utility of their action,
  conditional on the evidence that they have. Human doesn't always know
  what evidence they have; it depends on what The Radical Interpreter
  says.
\end{itemize}

The result is that the game is a coordination game. The Radical
Interpreter wants to assign evidence in a way that predicts rational
Human action, and Human wants to do what's rational given that
assignment of evidence. Coordination games typically have multiple
equilibria, and this one is no exception.

Let's make all that (marginally) more concrete. Human is offered a bet
on \emph{p}. If the bet wins, it wins 1 util; if the bet loses, it loses
100 utils. Human's only choice is to Take or Decline the bet. The
proposition \emph{p}, the subject of the bet, is like the claim that
Rahul is in the restaurant. It is something that is arguably part of
Human's evidence. Unfortunately, it is also arguable that it is not part
of Human's evidence. We will let \emph{K} be the rest of Human's
evidence (apart from \emph{p}, and things entailed by \emph{K} ∪
\{\emph{p}\}), and stipulate that \emph{Pr(}p\emph{~\textbar{}
}K*)~=~0.9. Each party now faces a choice.

\begin{itemize}
\tightlist
\item
  The Radical Interpreter has to choose whether \emph{p} is part of
  Human's evidence or not.
\item
  Human has to decide whether to Take or Decline the bet.
\end{itemize}

The Radical Interpreter achieves their goal if human takes the bet iff
\emph{p} is part of their evidence. If \emph{p} is part of the evidence,
then The Radical Interpreter thinks that the bet has positive expected
utility, so Human will take it. And if \emph{p} is not part of the
evidence, then The Radical Interpreter thinks that the bet has negative
expected utility, so Human will decline it. Either way, The Radical
Interpreter wants Human's action to coordinate with theirs. And Human,
of course, wants to maximise expected utility. So we get the following
table for the game.

\begin{longtable}[]{@{}lcc@{}}
\toprule\noalign{}
& \emph{p}~∈ \emph{E} & \emph{p} ∉ \emph{E} \\
\midrule\noalign{}
\endhead
\bottomrule\noalign{}
\endlastfoot
Take the bet & 1, 1 & -9.1, 0 \\
Decline the bet & 0, 0 & 0, 1 \\
\end{longtable}

We have, in effect, already covered The Radical Interpreter's payouts.
They win in the top-left and lower-right quadrants, and lose otherwise.
Human's payouts are only a little trickier. In the bottom row, they are
guaranteed 0, since the bet is declined. In the top-left, the bet is a
sure winner; their evidence entails it wins. So they get a payout of 1.
In the top-right, the bet wins with probability 0.9, so the expected
return\footnote{I am making a large, if orthodox, assumption here: that
  the payouts that we use for equilibrium analysis should be expected
  returns, not actual returns. I think that's the right thing to do,
  since it is usually impossible to say what the actual return of a game
  is. Even when we say that the payout is a certain number of dollars,
  we are really saying that the return is a certain kind of gamble.
  Maybe the value of the currency will deprecate quickly, and the
  dollars are not that valuable. Maybe the revolution will come and
  wealth will be a liability. Almost all games have probabilistic
  payouts, and this game is no different.} of taking it is 1 × 0.9~-~100
× 0.1~=~-9.1.

There are two Nash equilibria for the game~-~I've bolded them below.

\begin{longtable}[]{@{}lcc@{}}
\toprule\noalign{}
& \emph{p}~∈ \emph{E} & \emph{p} ∉ \emph{E} \\
\midrule\noalign{}
\endhead
\bottomrule\noalign{}
\endlastfoot
Take the bet & \textbf{1, 1} & -9.1, 0 \\
Decline the bet & 0, 0 & \textbf{0, 1} \\
\end{longtable}

The mathematical result that there are two equilibria to this game
should not come as a surprise. In discussing games like this earlier, we
said that general principles connecting evidence, knowledge and action
are not predictive; they are consistent both with \emph{p} being part of
the evidence, and with it not being part of the evidence. The general
principles we had stated rule out, in effect, non-equilibrium solutions
to games like this one. But they are not predictive in cases where there
are multiple equilibria.

To make more progress, we need to turn to more contested areas of game
theory. In particular, we need to look at some work on equilibrium
choice. We'll introduce this material via a game that is inspired by an
example of Rousseau's.

\section{Equilibrium Selection
Principles}\label{equilibriumselectionprinciples}

At an almost maximal level of abstraction, a two player, two option each
game looks like this.

\begin{longtable}[]{@{}lcc@{}}
\toprule\noalign{}
& \emph{a} & \emph{b} \\
\midrule\noalign{}
\endhead
\bottomrule\noalign{}
\endlastfoot
\emph{A} & \emph{r}\textsubscript{11}, \emph{c}\textsubscript{11} &
\emph{r}\textsubscript{12}, \emph{c}\textsubscript{12} \\
\emph{B} & \emph{r}\textsubscript{21}, \emph{c}\textsubscript{21} &
\emph{r}\textsubscript{22}, \emph{c}\textsubscript{22} \\
\end{longtable}

We're going to focus on games that have the following eight properties:

\begin{itemize}
\tightlist
\item
  \emph{r}\textsubscript{11}~\textgreater~\emph{r}\textsubscript{21}
\item
  \emph{r}\textsubscript{22}~\textgreater~\emph{r}\textsubscript{12}
\item
  \emph{c}\textsubscript{11}~\textgreater~\emph{c}\textsubscript{12}
\item
  \emph{c}\textsubscript{22}~\textgreater~\emph{c}\textsubscript{21}
\item
  \emph{r}\textsubscript{11}~\textgreater~\emph{r}\textsubscript{22}
\item
  \emph{c}\textsubscript{11}~⩾~\emph{c}\textsubscript{22}
\item
  (\emph{r}\textsubscript{21}+\emph{r}\textsubscript{22})/2~\textgreater~(\emph{r}\textsubscript{11}+\emph{r}\textsubscript{12})/2
\item
  (\emph{c}\textsubscript{12}+\emph{c}\textsubscript{22})/2~\textgreater~(\emph{c}\textsubscript{11}+\emph{r}\textsubscript{21})/2
\end{itemize}

The first four clauses say that the game has two (strict) Nash
equilibria: \emph{Aa} and \emph{Bb}. The fifth and sixth clauses say
that the \emph{Aa} equilibria is \textbf{Pareto-optimal}: no one prefers
the other equilibria to it. In fact it says something a bit stronger:
one of the players strictly prefers the \emph{Aa} equilibria, and the
other player does not prefer \emph{Bb}. The seventh and eighth clauses
say that the \emph{Bb} equilibria is \textbf{risk-optimal}.
Risk-optimality is a somewhat complicated notion in general; see
Harsanyi and Selten (\citeproc{ref-HarsanyiSelten1988}{1988}) for more
details. But for our purposes, we can focus on a simple characterisation
of it. Neither player would prefer playing \emph{A}/\emph{a} to playing
\emph{B}/\emph{b} if they thought it was a coin flip which equilibrium
the other player was aiming for.

I'm going to offer an argument from Hans Carlsson and Eric van Damme
(\citeproc{ref-CarlssonVanDamme1993}{1993}) for the idea that in these
games, rational players will end up at \emph{Bb}. The game that Human
and The Radical Interpreter are playing fits these eight conditions, and
The Radical Interpreter is perfectly rational, so this will imply that
in that game, The Radical Interpreter will say that \emph{p}~∉~\emph{E},
which is what we aimed to show.

Games satisfying these eight inequalities are sometimes called
\emph{Stag Hunt} games. There is some flexibility, and some vagueness,
in which of the eight inequalities need to be strict, but that level of
detail isn't important here. The name comes from a thought experiment in
Rousseau's \emph{Discourse on Inequality}.

\begin{quote}
{[}T{]}hey were perfect strangers to foresight, and were so far from
troubling themselves about the distant future, that they hardly thought
of the morrow. If a deer was to be taken, every one saw that, in order
to succeed, he must abide faithfully by his post: but if a hare happened
to come within the reach of any one of them, it is not to be doubted
that he pursued it without scruple, and, having seized his prey, cared
very little, if by so doing he caused his companions to miss theirs.
(\citeproc{ref-Rousseau1913}{Rousseau 1913, 209--10})
\end{quote}

It is rather interesting to think through which real-life situations are
best modeled as Stag Hunts, especially in situations where people have
thought that the right model was a version of Prisoners' Dilemma. This
kind of thought is one way in to appreciating the virtues of Rousseau's
political outlook, and especially the idea that social coordination
might not require anything like the heavy regulatory presence that, say,
Hobbes thought was needed. But that's a story for another day. What
we're going to be interested in is why Rousseau was right to think that
a `stranger to foresight', who is just focussing on this game, should
take the rabbit.

To make matters a little easier, we'll focus on a very particular
instance of Stag Hunt, as shown here. (From here I'm following Carlsson
and van Damme very closely; this is their example, with just the
labelling slightly altered.)

\begin{longtable}[]{@{}lcc@{}}
\toprule\noalign{}
& \emph{a} & \emph{b} \\
\midrule\noalign{}
\endhead
\bottomrule\noalign{}
\endlastfoot
\emph{A} & 4, 4 & 0, 3 \\
\emph{B} & 3, 0 & 3, 3 \\
\end{longtable}

At first glance it might seem like \emph{Aa} is the right choice; it
produces the best outcome. This isn't like Prisoners Dilemma, where the
best collective outcome is dominated. In fact \emph{Aa} is the best
outcome for each individual. But it is risky, and Carlsson and van Damme
show how to turn that risk into an argument for choosing \emph{Bb}.

Embed this game in what they call a \emph{global game}. We'll start the
game with each player knowing just that they will play a game with the
following payout table, with \emph{x} to be selected at random from a
flat distribution over {[}-1,~5{]}.

\begin{longtable}[]{@{}lcc@{}}
\toprule\noalign{}
& \emph{a} & \emph{b} \\
\midrule\noalign{}
\endhead
\bottomrule\noalign{}
\endlastfoot
\emph{A} & 4, 4 & 0, x \\
\emph{B} & x, 0 & x, x \\
\end{longtable}

Before they play the game, each player will get a noisy signal about the
value of \emph{x}. There will be signals \emph{s\textsubscript{R}} and
\emph{s\textsubscript{C}} chosen (independently) from a flat
distribution over {[}\emph{x}~-~0.25,~\emph{x}~+~0.25{]}, and shown to
Row and Column respectively. So each player will know the value of
\emph{x} to within ¼, and know that the other player knows it to within
¼ as well. But this is a margin of error model, and in those models
there is very little that is common knowledge. That, they argue, makes a
huge difference.

In particular, they prove that iterated deletion of strictly dominated
strategies (almost) removes all but one strategy pair.\footnote{A sketch
  of the proof is in Appendix One.} Each player will play
\emph{A}/\emph{a} if the signal is greater than 2, and \emph{B}/\emph{b}
otherwise.\footnote{Strictly speaking, we can't rule out various mixed
  strategies when the signal is precisely 2, but this makes little
  difference, since that occurs with probability 0.} Surprisingly, this
shows that players should play the risk-optimal strategy even when they
know the other strategy is Pareto-optimal. When a player gets a signal
in (2,~3.75), then they know that \emph{x}~\textless~\emph{4}, so
\emph{Bb} is the Pareto-optimal equilibrium. But the logic of the global
game suggests the risk-dominant equilibrium is what to play.

Carlsson and van Damme go on to show that many of the details of this
case don't matter. As long as (a) there is a margin of error in each
side's estimation of the payoffs, and (b) every choice is a dominant
option in some version of the global game, then iterated deletion of
strongly dominant strategies will lead to each player making the
risk-dominant choice.

I conclude from that that risk-dominant choices are rational in these
games. There is a limit assumption involved here; what's true for games
with arbitrarily small margins of error is true for games with no margin
of error. (We'll come back to that assumption below.) And since The
Radical Interpreter is rational, they will play the strategy that is not
eliminated by deleting dominant strategies. That is, they will play the
risk-dominant strategy.

In game with Human, the rational (i.e., risk-dominant) strategy for The
Radical Interpreter is to say that \emph{p}~∉~\emph{E}. And in the case
of Parveen and Rahul, rational (i.e., risk-dominant) strategy for The
Radical Interpreter is to say that it is not part of Parveen's evidence
that Rahul is in the restaurant. And this is an interest-relative theory
of evidence; had Parveen been playing a different game, The Radical
Interpreter would have said that it is part of Parveen's evidence that
Rahul was in the restaurant.

And from this point we can say all the things we wanted to say about the
case. If it is part of Parveen's evidence that Rahul is in the
restaurant, then she knows this. Conversely, if she knows it, then The
Radical Interpreter would have said it is part of her evidence, so it is
part of her evidence. Parveen will perform the action that maximises
expected utility given her evidence. And she will lose knowledge when
that disposition makes her do things that would be known to be
sub-optimal if she didn't lose knowledge.

In short, this model gives us a way to keep what was good about the
pragmatic encroachment theory, while also allowing that evidence can be
interest-relative. It does require a slightly more complex theory of
rationality than we had previously used. Rather than just say that
agents maximise evidential expected utility, we have to say that they
play risk-dominant strategies in coordination games. But it turns out
that this is little more than saying that they maximise evidential
expected utility, and they expect others (at least perfectly rational
abstract others) to do the same, and they expect those others to expect
they will maximise expected utility, and so on.

\section{Objections and Replies}\label{objectionsandreplies}

We'll end the body of the paper with some objections that might be
raised to this model. And then the appendix will contain proofs of a
couple of the formal claims.

\emph{Objection}: The formal result of the previous section only goes
through if we assume that the agents do not know precisely what the
payoffs are in the game. We shouldn't assume that what holds for
arbitrarily small margins of error will hold in the limit, i.e., when
they do know the payoffs.

\emph{Reply}: If pushed, I would defend the use limit assumptions like
this to resolve hard cases like Stag Hunt. But I don't need that
assumption here, What we really need is that Parveen doesn't know
precisely the probability of Rahul being in the restaurant given the
rest of her evidence. Given that evidence is not luminous, as Williamson
(\citeproc{ref-Williamson2000}{2000}) shows, this is a reasonable
assumption. So the margin of error assumption that Carlsson and van
Damme make is not, in our case, an assumption that merely makes the math
easier; it is built into the case.

\emph{Objection}: Even if Parveen doesn't know the payoffs precisely,
The Radical Interpreter does. They are an idealisation, so they can be
taken to be ideal.

\emph{Reply}: It turns out that Carlsson and van Damme's result doesn't
require that both parties are ignorant of the precise values of the
payoffs. As long as one party doesn't know the exact value of the
payoff, the argument goes through. I prove this in Appendix Two.

\emph{Objection}: The formal argument requires that in the `global game'
there are values for \emph{x} that make \emph{a} the dominant choice.
These cases serve as a base step for an inductive argument that follows.
But in Parveen's case, there is no such setting for \emph{x}, so the
inductive argument can't get going.

\emph{Reply}: What matters is that there are values of \emph{x} such
that \emph{a} is the strictly dominant choice, and Human (or Parveen)
doesn't know that they know that they know, etc., that those values are
not actual. And that's true in our case. For all Human (or Parveen)
knows that they know that they know that they know\ldots, the
proposition in question is not part of their evidence under a maximally
expansive verdict on The Radical Interpreter's part. So the relevant
cases are there in the model, even if for some high value of \emph{n}
they are known\textsuperscript{\emph{n}} not to obtain.

\emph{Objection}: This model is much more complex than the simple
motivation for pragmatic encroachment.

\emph{Reply}: Sadly, this is true. I would like to have a simpler model,
but I don't know how to create one. The argument I gave earlier that our
simple principles underdetermine what to say in cases like Parveen and
Rahul's seems fairly compelling. So more complexity will be needed, one
way or another. I think paying this price in complexity is worth it
overall, but I can see how some people might think otherwise.

\emph{Objection}: Change the case involving Human so that the bet loses
15 utils if \emph{p} is false, rather than 100. Now the risk-dominant
equilibrium is that Human takes the bet, and The Radical Interpreter
says that \emph{p} is part of Human's evidence. But note that if it was
clearly true that \emph{p} was not part of Human's evidence, then this
would still be too risky a situation for them to know \emph{p}. So
whether it is possible that \emph{p} is part of Human's evidence
matters.

\emph{Reply}: This is all true, and it shows that the view I'm putting
forward is incompatible with some programs in epistemology. In
particular, it is incompatible with E=K, since the what it takes to be
evidence on this story is slightly different from what it takes to be
knowledge. I don't think E=K is so intuitively obvious that this refutes
the theory, but it is potentially a cost that I have to give it up.

\emph{Objection}: Carlsson and van Damme discuss one kind of global
game. But there are other global games that have different equilibria.
For instance, changing the method by which the noisy signal is selected
would change the equilibrium of the global game. So this kind of
argument can't show that the risk-dominant equilibrium is the one true
solution.

\emph{Reply}: This is somewhat true. There are other ways of embedding
the game involving Human and The Radical Interpreter in global games
that lead to different outcomes. They are usually somewhat artificial;
e.g., by having the signal be systematically biased in one way. But what
really matters is the game where the error in Human's knowledge of the
payoffs is determined by their actual epistemic limitations. I think
that will lead to something like the model we have here. But it is
possible that the final result will differ a bit from what I have here,
or (more likely) have some indeterminacy about just how interests
interact with evidence and knowledge. The precise details are ultimately
less important to me than whether we can provide a motivated story of
how interests affect knowledge and evidence that does not presuppose we
know what the agent's evidence is. And the method I've outlined here
shows that we can do that, even if we end up tinkering a bit with the
details.

\section*{Appendix One: Carlsson and van Damme's
Game}\label{appendixone:carlssonandvandammesgame}
\addcontentsline{toc}{section}{Appendix One: Carlsson and van Damme's
Game}

Two players, Row (or R) and Column (or C) will a version of the
following game.

\begin{longtable}[]{@{}lcc@{}}
\toprule\noalign{}
& \emph{a} & \emph{b} \\
\midrule\noalign{}
\endhead
\bottomrule\noalign{}
\endlastfoot
\emph{a} & 4, 4 & 0, \emph{x} \\
\emph{b} & \emph{x}, 0 & \emph{x}, \emph{x} \\
\end{longtable}

They won't be told what \emph{x} is, but they will get a noisy signal of
\emph{x}, drawn from an even distribution over
{[}\emph{x}~-~0.25,~\emph{x}~+~0.25{]}. Call these signals
\emph{s\textsubscript{R}} and \emph{s\textsubscript{C}}. Each player
must then choose \emph{a}, getting either 4 or 0 depending on the other
player's choice, or choose \emph{b}, getting \emph{x} for sure.

Before getting the signal, the players must choose a strategy. A
strategy is a function from signals to choices. Since the higher the
signal is, the better it is to play \emph{b}, we can equate strategies
with `tipping points', where the player plays \emph{b} if the signal is
above the tipping point, and \emph{a} below the tipping point. Strictly
speaking, a tipping point will pick out not a strategy but an
equivalence class of strategies, which differ in how they act if the
signal is the tipping point. But since that happens with probability 0,
the strategies in the equivalence class have the same expected return,
and so we won't aim to distinguish them.

Also, strictly speaking, there are strategies that are not tipping
points, because they map signals onto probabilities of playing \emph{a},
where the probability decreases as \emph{a} rises. I won't discuss these
directly, but it isn't too hard to see how these are shown to be
suboptimal using the argument that is about to come. It eases exposition
to focus on the pure strategies, and to equate these with tipping
points. And since my primary aim here is to explain why the result
holds, not to simply repeat an already existing proof, I'll mostly
ignore these mixed strategies.

Call the tipping points for Row and Column respectively
\emph{T\textsubscript{R}} and \emph{T\textsubscript{C}}. Since the game
is symmetric, we'll just have to show that in conditions of common
knowledge of rationality, \emph{T\textsubscript{R}}~=~2. It follows by
symmetry that \emph{T\textsubscript{C}}~=~2 as well. And the only rule
we'll use is iterated deletion of strictly dominated strategies. That
is, we'll assume players won't play strategies where another strategy
does better no matter what the opponent chooses, and they won't play
strategies where another strategy does better provided the other player
does not play a dominated strategy, and they won't play strategies where
another strategy does better provided the other player does not play a
strategy ruled out by these first two conditions, and so on.

The return to a strategy is uncertain, even given the other player's
strategy. But given the strategies of each player, we can work out an
expected return for each player. And that's what we'll assume is the
return to a strategy pair.

Note first that \emph{T\textsubscript{R}}~=~4.25 strictly dominates any
strategy where \emph{T\textsubscript{R}}~=~\emph{y}~\textgreater~4.25.
If \emph{s\textsubscript{R}}~∈~(4.25,~\emph{y}), then
\emph{T\textsubscript{R}} is guaranteed to return above 4, and the
alternative strategy is guaranteed to return 4. In all other cases, the
strategies have the same return. And there is some chance that
\emph{s\textsubscript{R}}~∈~(4.25,~\emph{y}). So we can delete all
strategies \emph{T\textsubscript{R}}~=~\emph{y}~\textgreater~4.25, and
similarly all strategies
\emph{T\textsubscript{C}}~=~\emph{y}~\textgreater~4.25. By similar
reasoning, we can rule out \emph{T\textsubscript{R}}~\textless~-0.25 and
\emph{T\textsubscript{C}}~\textless~-0.25.

If \emph{s\textsubscript{R}}~∈~{[}-0.75,~4.75{]}, then it is equally
likely that \emph{x} is above \emph{s\textsubscript{R}} as it is below
it. Indeed, the posterior distribution of \emph{x} is flat over
{[}\emph{s\textsubscript{R}}~-~0.25,~\emph{s\textsubscript{R}}~+~0.25{]}.
From this it follows that the expected return of playing \emph{b} after
seeing signal \emph{s\textsubscript{R}} is just
\emph{s\textsubscript{R}}.

Now comes the important step. Assume that we know that
\emph{T\textsubscript{C}}~⩽~\emph{y}~\textgreater~2. Now consider the
expected return of playing \emph{a} given various values for
\emph{s\textsubscript{R}}~\textgreater~2. Given that the lower
\emph{T\textsubscript{C}} is, the higher the expected return is of
playing \emph{a}, we'll just work on the simple case where
\emph{T\textsubscript{C}}~=~\emph{y}, realizing that this is an upper
bound on the expected return of \emph{a} given
\emph{T\textsubscript{C}}~⩽~\emph{y}. The expected return of \emph{a} is
4 times the probability that Column will play \emph{a}, i.e., 4 times
the probability that
\emph{s\textsubscript{C}}~\textless~\emph{T\textsubscript{C}}. Given all
the symmetries that have been built into the puzzle, we know that the
probability that
\emph{s\textsubscript{C}}~\textless~\emph{s\textsubscript{R}} is 0.5. So
the expected return of playing \emph{a} is at most 2 if
\emph{s\textsubscript{R}}~⩾~\emph{y}. But the expected return of playing
\emph{b} is, as we showed in the last paragraph,
\emph{s\textsubscript{R}}, which is greater than 2. So it is better to
play \emph{b} than \emph{a} if \emph{s\textsubscript{R}}~⩾~\emph{y}. And
the difference is substantial, so even if \emph{s\textsubscript{R}} is
epsilon less than that \emph{y}, it will still be better to play
\emph{b}. (This is hand-wavy of course, but we'll make it rigorous in
just a second.)

So if \emph{T\textsubscript{C}}~⩽~\emph{y}~\textgreater~2 we can prove
that \emph{T\textsubscript{R}} should be lower still, because given that
assumption it is better to play \emph{b} even if the signal is just less
than \emph{y}. Repeating this reasoning over and over again pushes us to
it being better to play \emph{b} than \emph{a} as long as
\emph{s\textsubscript{R}}~\textgreater~2. And the same kind of reasoning
from the opposite end pushes us to it being better to play \emph{a} than
\emph{b} as long as \emph{s\textsubscript{R}}~\textless~2. So we get
\emph{s\textsubscript{R}}~=~2 as the uniquely rational solution to the
game.

Let's make that a touch more rigorous. Assume that
\emph{T\textsubscript{C}}~=~\emph{y}, and \emph{s\textsubscript{R}} is
slightly less than \emph{y}. In particular, we'll assume that
\emph{z}~=~\emph{y}~-~\emph{s\textsubscript{R}} is in (0,~0.5). Then the
probability that \emph{s\textsubscript{C}}~\textless~\emph{y} is
0.5~+~2\emph{z}~-~2\emph{z}\textsuperscript{2}. So the expected return
of playing \emph{a} is 2~+~8\emph{z}~-~8z\textsuperscript{2}. And the
expected return of playing \emph{b} is, again,
\emph{s\textsubscript{R}}. These will be equal when the following is
true. (The working out is a tedious but trivial application of the
quadratic formula, plus some rearranging.)

\[ 
s_R = y + \frac{\sqrt{145-32y} - 9}{16}
\]

So if we know that \emph{T\textsubscript{C}}~⩾~\emph{y}, we know that

\[
T_R \geqslant y + \frac{\sqrt{145-32y} - 9}{16}
\]

which will be less than \emph{y} if \emph{y}~\textgreater~2. And then by
symmetry, we know that \emph{T\textsubscript{C}} must be at most as
large as that as well. And then we can use that fact to derive a further
upper bound on \emph{T\textsubscript{R}} and hence on
\emph{T\textsubscript{C}}, and so on. And this will continue until we
push both down to 2. It does require quite a number of steps of iterated
deletion. Here is the upper bound on the threshold after \emph{n} rounds
of deletion of dominated strategies. (These numbers are precise for the
first two rounds, then just to three significant figures after that.)

\begin{longtable}[]{@{}cc@{}}
\toprule\noalign{}
Round & Upper Bound on Threshold \\
\midrule\noalign{}
\endhead
\bottomrule\noalign{}
\endlastfoot
1 & 4.250 \\
2 & 3.875 \\
3 & 3.599 \\
4 & 3.378 \\
5 & 3.195 \\
6 & 3.041 \\
7 & 2.910 \\
8 & 2.798 \\
9 & 2.701 \\
10 & 2.617 \\
\end{longtable}

That is, \emph{T\textsubscript{R}}~=~4.25 dominates any strategy with a
tipping point above 4.25. And \emph{T\textsubscript{R}}~=~3.875
dominates any strategy with a higher tipping point than that, assuming
\emph{T\textsubscript{C}}~⩽~4.25. And \emph{T\textsubscript{R}}~≅~3.599
dominates any strategy with a higher tipping point than that, assuming
\emph{T\textsubscript{C}}~⩽~3.875. And so on.

And similar reasoning shows that at each stage not only are all
strategies with higher tipping points dominated, but so are strategies
that assign positive probability (whether it is 1 or less than 1), to
playing \emph{a} when the signal is above the `tipping point'. So this
kind of reasoning rules out all mixed strategies (except those that
respond probabilistically to \emph{s\textsubscript{R}}~=~2).

So we've shown what was intended, namely that iterated deletion of
dominated strategies will rule out all strategies except the
risk-optimal equilibrium. We needed the possibility that \emph{x} is
greater than the maximal return for \emph{a} to get the iterated
dominance going. And we needed the signal to have an error bar to it, so
that each round of iteration removes more strategies. But that's all we
needed; the particular values we chose are irrelevant to the proof.

\section*{Appendix Two: The Modified
Game}\label{appendixtwo:themodifiedgame}
\addcontentsline{toc}{section}{Appendix Two: The Modified Game}

The aim of this section is to prove something that Carllson and van
Damme did not prove, namely that the analysis of the previous appendix
goes through with very little change if one party gets a perfect signal,
while the other gets a noisy signal. That is, we're going to consider
the game that is just like the game of appendix one, but it is common
knowledge that the signal Column gets, \emph{s\textsubscript{C}}, equals
\emph{x}.

Since the game is no longer symmetric, we can't appeal to the symmetry
of the game as we frequently did in the previous appendix. But this only
slows the proof down, it doesn't stop it.

We can actually rule out slightly more at the first step in this game
than in the previous game. Since Column could not be wrong about
\emph{x}, Column knows that if \emph{s\textsubscript{C}}~\textgreater~4
then playing \emph{b} dominates playing \emph{a}. So one round of
deleting dominated strategies rules out
\emph{T\textsubscript{C}}~\textgreater~4, as well as ruling out
\emph{T\textsubscript{R}}~\textgreater~4.25.

At any stage, if we know
\emph{T\textsubscript{C}}~⩽~\emph{y}~\textgreater~2, then
\emph{T\textsubscript{R}}~=~\emph{y} dominates
\emph{T\textsubscript{R}}~\textgreater~\emph{y}. That's because if
\emph{s\textsubscript{R}} ⩾~\emph{y}, and
\emph{T\textsubscript{C}}~⩽~\emph{y}, then the probability that Column
will play \emph{a} (given Row's signal) is less than 0.5. After all, the
signal is just as likely to be above \emph{x} as below it (as long as
the signal isn't too close to the extremes). So if
\emph{s\textsubscript{R}} is at or above \emph{T\textsubscript{C}}, then
it is at least 0.5 likely that \emph{s\textsubscript{C}}~=~\emph{x} is
at or above \emph{T\textsubscript{C}}. So the expected return of playing
\emph{a} is at most 2. But the expected return of playing \emph{b}
equals the signal, which is greater than 2. So if Row knows
\emph{T\textsubscript{C}}~⩽~\emph{y}~\textgreater~2, Row also knows it
is better to play \emph{b} if \emph{s\textsubscript{R}} ⩾~\emph{y}. And
that just means that \emph{T\textsubscript{R}}~⩽~\emph{y}.

Assume now that it is common knowledge that
\emph{T\textsubscript{R}}~⩽~\emph{y}, for some \emph{y}~\textgreater~2.
And assume that \emph{x}~=~\emph{s\textsubscript{C}} is just a little
less than \emph{y}. In particular, define
\emph{z}~=~\emph{y}~-~\emph{x}, and assume \emph{z}~∈~(0, 0.25). We want
to work out the upper bound on the expected return to Column of playing
\emph{a}. (The return of playing \emph{b} is known, it is \emph{x}.) The
will be highest when \emph{T\textsubscript{R}} is lowest, so assume
\emph{T\textsubscript{R}}~⩽~\emph{y}. Then the probability that Row
plays \emph{a} is (1~+~2\emph{z})/2. So the expected return of playing
\emph{a} is 2~+~4\emph{z}, i.e., 2~+~4(\emph{y}~-~\emph{x}). That will
be greater than \emph{x} only when \emph{x}~\textless~(2~+~4\emph{y})/5.

And so if it is common knowledge that
\emph{T\textsubscript{R}}~⩽~\emph{y}, then it is best for Column to play
\emph{b} unless \emph{x}~\textless~(2~+~4\emph{y})/5. That is, if it is
common knowledge that \emph{T\textsubscript{R}}~⩽~\emph{y}, then
\emph{T\textsubscript{C}} must be at most
\emph{x}~\textless~(2~+~4\emph{y})/5.

So now we proceed in a zig-zag fashion. At one stage, we show that
\emph{T\textsubscript{R}} must be as low as \emph{T\textsubscript{C}}.
At the next, we show that if it has been proven that
\emph{T\textsubscript{R}} takes a particular value greater than 2, then
\emph{T\textsubscript{C}} must be lower still. And this process will
eventually rule out all values for \emph{T\textsubscript{R}} and
\emph{T\textsubscript{C}} greater than 2.

This case is crucial to the story of the paper because The Radical
Interpreter probably does not have an error bar in their estimation of
the game they are playing. But it turns out the argument for
risk-dominant equilibria being the unique solution to interpretation
games is consistent with that. As long as one player has a margin of
error, each player should play the risk-dominant equilibria.

\subsection*{References}\label{references}
\addcontentsline{toc}{subsection}{References}

\phantomsection\label{refs}
\begin{CSLReferences}{1}{0}
\bibitem[\citeproctext]{ref-Brown2008}
Brown, Jessica. 2008. {``Subject-Sensitive Invariantism and the
Knowledge Norm for Practical Reasoning.''} \emph{No{û}s} 42 (2):
167--89. doi:
\href{https://doi.org/10.1111/j.1468-0068.2008.00677.x}{10.1111/j.1468-0068.2008.00677.x}.

\bibitem[\citeproctext]{ref-CarlssonVanDamme1993}
Carlsson, Hans, and Eric van Damme. 1993. {``Global Games and
Equilibrium Selection.''} \emph{Econometrica} 61 (5): 989--1018. doi:
\href{https://doi.org/10.2307/2951491}{10.2307/2951491}.

\bibitem[\citeproctext]{ref-FantlMcGrath2009}
Fantl, Jeremy, and Matthew McGrath. 2009. \emph{Knowledge in an
Uncertain World}. Oxford: Oxford University Press.

\bibitem[\citeproctext]{ref-HarsanyiSelten1988}
Harsanyi, John C., and Reinhard Selten. 1988. \emph{A General Theory of
Equilibrium Selection in Games}. Cambridge, MA: {MIT} Press.

\bibitem[\citeproctext]{ref-Hawthorne2004}
Hawthorne, John. 2004. \emph{Knowledge and Lotteries}. Oxford: Oxford
University Press.

\bibitem[\citeproctext]{ref-IchikawaEtAl2012}
Ichikawa, Jonathan Jenkins, Benjamin Jarvis, and Katherine Rubin. 2012.
{``Pragmatic Encroachment and Belief-Desire Psychology.''}
\emph{Analytic Philosophy} 53 (4): 327--43. doi:
\href{https://doi.org/10.1111/j.2153-960X.2012.00564.x}{10.1111/j.2153-960X.2012.00564.x}.

\bibitem[\citeproctext]{ref-Lackey2010}
Lackey, Jennifer. 2010. {``Acting on Knowledge.''} \emph{Philosophical
Perspectives} 24: 361--82. doi:
\href{https://doi.org/10.1111/j.1520-8583.2010.00196.x}{10.1111/j.1520-8583.2010.00196.x}.

\bibitem[\citeproctext]{ref-Reed2014}
Reed, Baron. 2014. {``Practical Matters Do Not Affect Whether You
Know.''} In \emph{Contemporary Debates in Epistemology}, edited by
Matthias Steup, John Turri, and Ernest Sosa, 2nd ed., 95--106.
Chicester: Wiley-Blackwell.

\bibitem[\citeproctext]{ref-Rousseau1913}
Rousseau, Jean-Jacques. 1913. \emph{Social Contract \& Discourses}.
Translated by G. D. H. Cole. New York: J. M. Dent \& Sons. Translated by
G. D. H. Cole.

\bibitem[\citeproctext]{ref-Rubin2015}
Rubin, Katherine. 2015. {``Total Pragmatic Encroachment and Epistemic
Permissiveness.''} \emph{Pacific Philosophical Quarterly} 96: 12--38.
doi: \href{https://doi.org/10.1111/papq.12060}{10.1111/papq.12060}.

\bibitem[\citeproctext]{ref-Schwitzgebel2008}
Schwitzgebel, Eric. 2008. {``The Unreliability of Naive
Introspection.''} \emph{Philosophical Review} 117 (2): 245--73. doi:
\href{https://doi.org/10.1215/00318108-2007-037}{10.1215/00318108-2007-037}.

\bibitem[\citeproctext]{ref-Stanley2005}
Stanley, Jason. 2005. \emph{{Knowledge and Practical Interests}}. Oxford
University Press.

\bibitem[\citeproctext]{ref-Williamson2000}
Williamson, Timothy. 2000. \emph{{Knowledge and its Limits}}. Oxford
University Press.

\bibitem[\citeproctext]{ref-Williamson2007-WILTPO-17}
---------. 2007. \emph{{The Philosophy of Philosophy}}. Blackwell.

\end{CSLReferences}



\noindent Published in\emph{
Episteme}, 2018, pp. 329-344.


\end{document}
