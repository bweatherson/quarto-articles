% Options for packages loaded elsewhere
\PassOptionsToPackage{unicode}{hyperref}
\PassOptionsToPackage{hyphens}{url}
%
\documentclass[
  10pt,
  letterpaper,
  DIV=11,
  numbers=noendperiod,
  twoside]{scrartcl}

\usepackage{amsmath,amssymb}
\usepackage{setspace}
\usepackage{iftex}
\ifPDFTeX
  \usepackage[T1]{fontenc}
  \usepackage[utf8]{inputenc}
  \usepackage{textcomp} % provide euro and other symbols
\else % if luatex or xetex
  \usepackage{unicode-math}
  \defaultfontfeatures{Scale=MatchLowercase}
  \defaultfontfeatures[\rmfamily]{Ligatures=TeX,Scale=1}
\fi
\usepackage{lmodern}
\ifPDFTeX\else  
    % xetex/luatex font selection
    \setmainfont[ItalicFont=EB Garamond Italic,BoldFont=EB Garamond
Bold]{EB Garamond Math}
    \setsansfont[]{Europa-Bold}
  \setmathfont[]{Garamond-Math}
\fi
% Use upquote if available, for straight quotes in verbatim environments
\IfFileExists{upquote.sty}{\usepackage{upquote}}{}
\IfFileExists{microtype.sty}{% use microtype if available
  \usepackage[]{microtype}
  \UseMicrotypeSet[protrusion]{basicmath} % disable protrusion for tt fonts
}{}
\usepackage{xcolor}
\usepackage[left=1in, right=1in, top=0.8in, bottom=0.8in,
paperheight=9.5in, paperwidth=6.5in, includemp=TRUE, marginparwidth=0in,
marginparsep=0in]{geometry}
\setlength{\emergencystretch}{3em} % prevent overfull lines
\setcounter{secnumdepth}{3}
% Make \paragraph and \subparagraph free-standing
\makeatletter
\ifx\paragraph\undefined\else
  \let\oldparagraph\paragraph
  \renewcommand{\paragraph}{
    \@ifstar
      \xxxParagraphStar
      \xxxParagraphNoStar
  }
  \newcommand{\xxxParagraphStar}[1]{\oldparagraph*{#1}\mbox{}}
  \newcommand{\xxxParagraphNoStar}[1]{\oldparagraph{#1}\mbox{}}
\fi
\ifx\subparagraph\undefined\else
  \let\oldsubparagraph\subparagraph
  \renewcommand{\subparagraph}{
    \@ifstar
      \xxxSubParagraphStar
      \xxxSubParagraphNoStar
  }
  \newcommand{\xxxSubParagraphStar}[1]{\oldsubparagraph*{#1}\mbox{}}
  \newcommand{\xxxSubParagraphNoStar}[1]{\oldsubparagraph{#1}\mbox{}}
\fi
\makeatother


\providecommand{\tightlist}{%
  \setlength{\itemsep}{0pt}\setlength{\parskip}{0pt}}\usepackage{longtable,booktabs,array}
\usepackage{calc} % for calculating minipage widths
% Correct order of tables after \paragraph or \subparagraph
\usepackage{etoolbox}
\makeatletter
\patchcmd\longtable{\par}{\if@noskipsec\mbox{}\fi\par}{}{}
\makeatother
% Allow footnotes in longtable head/foot
\IfFileExists{footnotehyper.sty}{\usepackage{footnotehyper}}{\usepackage{footnote}}
\makesavenoteenv{longtable}
\usepackage{graphicx}
\makeatletter
\def\maxwidth{\ifdim\Gin@nat@width>\linewidth\linewidth\else\Gin@nat@width\fi}
\def\maxheight{\ifdim\Gin@nat@height>\textheight\textheight\else\Gin@nat@height\fi}
\makeatother
% Scale images if necessary, so that they will not overflow the page
% margins by default, and it is still possible to overwrite the defaults
% using explicit options in \includegraphics[width, height, ...]{}
\setkeys{Gin}{width=\maxwidth,height=\maxheight,keepaspectratio}
% Set default figure placement to htbp
\makeatletter
\def\fps@figure{htbp}
\makeatother
% definitions for citeproc citations
\NewDocumentCommand\citeproctext{}{}
\NewDocumentCommand\citeproc{mm}{%
  \begingroup\def\citeproctext{#2}\cite{#1}\endgroup}
\makeatletter
 % allow citations to break across lines
 \let\@cite@ofmt\@firstofone
 % avoid brackets around text for \cite:
 \def\@biblabel#1{}
 \def\@cite#1#2{{#1\if@tempswa , #2\fi}}
\makeatother
\newlength{\cslhangindent}
\setlength{\cslhangindent}{1.5em}
\newlength{\csllabelwidth}
\setlength{\csllabelwidth}{3em}
\newenvironment{CSLReferences}[2] % #1 hanging-indent, #2 entry-spacing
 {\begin{list}{}{%
  \setlength{\itemindent}{0pt}
  \setlength{\leftmargin}{0pt}
  \setlength{\parsep}{0pt}
  % turn on hanging indent if param 1 is 1
  \ifodd #1
   \setlength{\leftmargin}{\cslhangindent}
   \setlength{\itemindent}{-1\cslhangindent}
  \fi
  % set entry spacing
  \setlength{\itemsep}{#2\baselineskip}}}
 {\end{list}}
\usepackage{calc}
\newcommand{\CSLBlock}[1]{\hfill\break\parbox[t]{\linewidth}{\strut\ignorespaces#1\strut}}
\newcommand{\CSLLeftMargin}[1]{\parbox[t]{\csllabelwidth}{\strut#1\strut}}
\newcommand{\CSLRightInline}[1]{\parbox[t]{\linewidth - \csllabelwidth}{\strut#1\strut}}
\newcommand{\CSLIndent}[1]{\hspace{\cslhangindent}#1}

\setlength\heavyrulewidth{0ex}
\setlength\lightrulewidth{0ex}
\usepackage[automark]{scrlayer-scrpage}
\clearpairofpagestyles
\cehead{
  Brian Weatherson
  }
\cohead{
  Keynes, Uncertainty and Interest Rates
  }
\ohead{\bfseries \pagemark}
\cfoot{}
\makeatletter
\newcommand*\NoIndentAfterEnv[1]{%
  \AfterEndEnvironment{#1}{\par\@afterindentfalse\@afterheading}}
\makeatother
\NoIndentAfterEnv{itemize}
\NoIndentAfterEnv{enumerate}
\NoIndentAfterEnv{description}
\NoIndentAfterEnv{quote}
\NoIndentAfterEnv{equation}
\NoIndentAfterEnv{longtable}
\NoIndentAfterEnv{abstract}
\renewenvironment{abstract}
 {\vspace{-1.25cm}
 \quotation\small\noindent\rule{\linewidth}{.5pt}\par\smallskip
 \noindent }
 {\par\noindent\rule{\linewidth}{.5pt}\endquotation}
\KOMAoption{captions}{tableheading}
\makeatletter
\@ifpackageloaded{caption}{}{\usepackage{caption}}
\AtBeginDocument{%
\ifdefined\contentsname
  \renewcommand*\contentsname{Table of contents}
\else
  \newcommand\contentsname{Table of contents}
\fi
\ifdefined\listfigurename
  \renewcommand*\listfigurename{List of Figures}
\else
  \newcommand\listfigurename{List of Figures}
\fi
\ifdefined\listtablename
  \renewcommand*\listtablename{List of Tables}
\else
  \newcommand\listtablename{List of Tables}
\fi
\ifdefined\figurename
  \renewcommand*\figurename{Figure}
\else
  \newcommand\figurename{Figure}
\fi
\ifdefined\tablename
  \renewcommand*\tablename{Table}
\else
  \newcommand\tablename{Table}
\fi
}
\@ifpackageloaded{float}{}{\usepackage{float}}
\floatstyle{ruled}
\@ifundefined{c@chapter}{\newfloat{codelisting}{h}{lop}}{\newfloat{codelisting}{h}{lop}[chapter]}
\floatname{codelisting}{Listing}
\newcommand*\listoflistings{\listof{codelisting}{List of Listings}}
\makeatother
\makeatletter
\makeatother
\makeatletter
\@ifpackageloaded{caption}{}{\usepackage{caption}}
\@ifpackageloaded{subcaption}{}{\usepackage{subcaption}}
\makeatother

\ifLuaTeX
  \usepackage{selnolig}  % disable illegal ligatures
\fi
\usepackage{bookmark}

\IfFileExists{xurl.sty}{\usepackage{xurl}}{} % add URL line breaks if available
\urlstyle{same} % disable monospaced font for URLs
\hypersetup{
  pdftitle={Keynes, Uncertainty and Interest Rates},
  pdfauthor={Brian Weatherson},
  hidelinks,
  pdfcreator={LaTeX via pandoc}}


\title{Keynes, Uncertainty and Interest Rates}
\author{Brian Weatherson}
\date{2001}

\begin{document}
\maketitle
\begin{abstract}
Uncertainty plays an important role in \emph{The General Theory},
particularly in the theory of interest rates. Keynes did not provide a
theory of uncertainty, but he did make some enlightening remarks about
the direction he thought such a theory should take. I argue that some
modern innovations in the theory of probability allow us to build a
theory which captures these Keynesian insights. If this is the right
theory, however, uncertainty cannot carry its weight in Keynes's
arguments. This does not mean that the conclusions of these arguments
are necessarily mistaken; in their best formulation they may succeed
with merely an appeal to risk.
\end{abstract}


\setstretch{1.1}
Keynes (\citeproc{ref-Keynes1936}{1936}) clearly saw an important role
for uncertainty in his \emph{General Theory}. However, few
contemporaries agreed with him, and subsequent `Keynesians' generally
obliterated the distinction between risk and uncertainty. In part this
was caused by Keynes's informal presentation of his views on uncertainty
in \emph{The General Theory}. This paper has two aims. The first is to
sketch a formal theory of uncertainty which captures Keynes's insights
about the risk/uncertainty distinction. I argue that the theory of
imprecise probabilities developed in recent years best captures Keynes's
intuitions about uncertainty. In particular this theory provides a
formal distinction between risk and uncertainty, and allows for an
analysis of Keynes's `weight' of arguments. However, the second aim is
to show that if this is right then Keynes was wrong to draw the economic
consequences of uncertainty that he did. In broad terms, I argue that
uncertainty is economically impotent. It only has effects in conjunction
with some other feature of models or the world, such as missing markets
or agent irrationality. But these features plus the existence of risk
are sufficient to get the conclusions Keynes wants. These conclusions of
Keynes might be right, but if so they can be justified without reference
to Keynesian uncertainty. At the end of the day, uncertainty is not as
economically interesting as it appears.

\section{Imprecise Probabilities}\label{imprecise-probabilities}

In the classical, or Bayesian\footnote{For this paper I follow Walley
  (\citeproc{ref-Walley1991}{1991}) in describing those theorists who
  require that all agents have precise degrees of belief and these
  degrees form a probability function as Bayesians. There is some
  dispute as to the accuracy of this labelling, particularly as some
  paradigm case Bayesians, such as Jeffrey
  (\citeproc{ref-Jeffrey1983}{1983}) and Fraassen
  (\citeproc{ref-vanFraassen1990}{1990}), accept that degrees of belief
  can be vague. However, there is probably no other name as convenient
  or as recognisable.}, model of rationality all rational agents have
precise degrees of belief, or credences, in each proposition. There is a
probability function \emph{Bel} such that for any proposition \emph{A},
there is a number \emph{Bel}(\emph{A}). So if an agent believes \emph{p}
to degree \emph{x} she believes \emph{p} to degree 1-\emph{x}. This is
appropriate for some propositions. For example, if \emph{p} is a
proposition about the decay of an atom with known half-life, or about
any event with a known objective chance and hence subject to risk and
not uncertainty, the agent's credences should reflect the chances. Since
chances are precise and form a probability function, the credences will
also have these properties. The Bayesian theory assumes that all
situations can be treated by analogy with these.

As Keynes pointed out in the famous \emph{QJE} article
(\citeproc{ref-Keynes1937}{Keynes 1937b}), this analogy is clearly
mistaken. When \emph{p} is about the price of copper in thirty years, we
do not know the chance that \emph{p} will be true. And we do not have
enough information to form a precise credence. As Keynes had argued in
his \emph{Treatise on Probability} sixteen years earlier, attempts to
avoid this problem by appeal to a Principle of Indifference lead to
contradiction. In \emph{The General Theory} he noted that he still
approved of this little argument (\citeproc{ref-Keynes1936}{Keynes 1936,
152}). Hence Bayesians have no way of representing our ignorance in
uncertain situations. They say that all rational agents have a precise
epistemic attitude towards each proposition, believing it to some
precise degree, whereas ignorance consists in not having such a precise
attitude.

The theory of imprecise probabilities avoids all of these difficulties.
The theory is quite old, dating back to work by Gerhard Tintner
(\citeproc{ref-Tintner1941}{1941}) and A.~G. Hart
(\citeproc{ref-Hart1942}{1942}), but has only received extensive
consideration recently. The best modern summaries are by the philosopher
Isaac Levi (\citeproc{ref-Levi1980}{1980}) and the statistician Peter
Walley (\citeproc{ref-Walley1991}{1991}). There are minor differences,
but the theory I shall give captures all the common ingredients.
According to Bayesians, states of rational agents are represented by a
single probability function Pr; in the imprecise theory they are
represented by a set of probability functions \emph{S}. The agent's
credence in \emph{p} is vague over the set of values that Pr(\emph{p})
takes for Pr~∈~\emph{S}. In the extreme case, for every
\emph{x}~∈~{[}0,~1{]} there will be a Pr~∈~\emph{S} such that
Pr(\emph{p})~=~\emph{x}. This represents almost total ignorance about
\emph{p}. The set \emph{S} is called the `representor' of the agent
whose credences it represents.

It is important to stress what \emph{S} represents, because there has
been some confusion over this\footnote{See, for example, Gärdenfors and
  Sahlin (\citeproc{ref-GardenforsSahlin1982}{1982}), Levi
  (\citeproc{ref-Levi1982}{1982}).}. The Pr do not represent the agent's
hypotheses about the correct distribution of objective chances. I use
the phrase `objective chance', or just `chance', to refer to a property
that plays a certain role in fundamental physics, the property which
makes it the case that the whirrings of atoms in the void is
indeterminate. Modern physics, or at least the most popular versions of
it, teaches that chance infects all fundamental physical events. These
chances fulfill all the properties that anyone has ever wanted in
probabilities. They reflect long-run frequencies of repeated events,
they put restrictions on reasonable degrees of belief, they can be
properly applied to single cases, and so on. If all fundamental physical
events are chance events, then arbitrary Boolean combinations of
fundamental physical events should also, presumably, be chance events.
But any event whatsoever is some combination of fundamental physical
events, though for many it may not be clear which combination. So
baseball games, romantic affairs and stock market movements are all
chance events in this sense, even though they are not, for instance,
repeatable events. Of course, trying to predict these using the laws of
physics will be even less productive than trying to predict them using
the methods we currently have available. Saying where all the atoms, or
quarks, currently are is humanly impossible, and perhaps theoretically
impossible as well. Even allowing for this, computing where they will
move before they get there is beyond the capacity of any possible
machine.

I distinguish between a situation where the agent does not know the
objective chance of some proposition, and a situation where the agent
has no precise credence in that proposition. An agent can have a precise
credence in \emph{p} without knowing its objective chance. If the agent
believes that a certain number of chance distributions are possible, and
gives each of them a precise credence, this entails she has a precise
credence in each event. (Imagine we see a fair coin be tossed, and land,
but do not see how it falls. The objective chance that it shows heads is
either one, if it does, or zero, otherwise. But the appropriate credence
in the proposition, the coin has landed heads, is one half.) Rather the
Pr represent the precise credence distributions that are consistent with
real imprecise distribution. For example, for some rational agent, and
some proposition \emph{p}, the agent's epistemic state will determine
that she believes \emph{p} to a greater degree than 0.2, and a lesser
degree than 0.4, but there will be no more facts about the matter. (In
this case \emph{S} will include a function Pr such that
Pr(\emph{p})~=~\emph{x} for each \emph{x} ∈
{[}0.2,\textasciitilde0.4{]}.) If we ask her whether she thinks \emph{p}
is more likely than some proposition, call it \emph{q}, which she
believes to degree 0.3, she will not be able to say one way or the
other. And this is not just because she lacks rationality or powers of
introspective observation. It is no requirement of rationality that she
believe \emph{p} is more likely, less likely or equally likely than
\emph{q} As Levi and Walley have pointed out, the Bayesian arguments
purporting to show this is a constraint on rationality have been
hopelessly circular.

The reasons for wanting to be able to represent uncertainty were
stressed by Keynes, and are generally well known. Before showing why
this theory captures Keynes's intuitions about uncertainty, I will
briefly mention two nice formal features of the theory of imprecise
probabilities. On many theories of uncertainty, particularly those that
represent uncertain agents as having interval valued degrees of belief,
it is hard to explain comparative statements, like ``\emph{p} seems more
likely to me than \emph{q}''. These comparatives are crucial to our
everyday practices of probabilistic reasoning. We say \emph{p} is more
probable than \emph{q} according to \emph{S} iff for all Pr~∈~\emph{S},
Pr(\emph{p})~\textgreater~Pr(\emph{q}). This lets us say, as seems
right, that \emph{A} is more probable than \emph{A}~∧~\emph{B} for
almost all propositions \emph{A}, \emph{B}.

The second formal advantage is that we now have a simple way to update
epistemic states on receiving new evidence. Let \emph{S} be the agent's
current representor, and the new evidence be \emph{e}. Then the updated
representor, \emph{S\textsubscript{e}} is given as follows:
\emph{S\textsubscript{e}} = \{Pr(·~\textbar~\emph{e}): Pr~∈~\emph{S}\}

That is, we just conditionalise every probability function in \emph{S}.
Again, updating has proven problematic for some approaches to
uncertainty. The theory of evidence functions, developed by Dempster
(\citeproc{ref-Dempster1967}{1967}) and Shafer
(\citeproc{ref-Shafer1976}{1976}) allows that an agent can know that if
either \emph{e} or ¬\emph{e} comes in as evidence, their credence in
\emph{p} will rise. This seems absurd; we can know before an experiment
that whatever happens we'll be more confident in \emph{p} than we are
now.

To take a famous example, three prisoners \emph{X}, \emph{Y} and
\emph{Z} are about to be exiled to Elba. The governor decides on a whim
that he will pardon one, and casts a fair die to choose which. He tells
the guards who is pardoned, but instructs them not to tell the prisoners
yet. \emph{X} pleads futilely with his guard, and finally asks, ``Can
you tell me the name of one of the others who won't be pardoned.'' The
guard, realising this will not reveal \emph{x}'s fate, agrees to answer.
\emph{X} thinks that if \emph{Y} is pardoned, the guard will say
\emph{Z}, so there is at least a one-third probability of that. And if
\emph{Z} is pardoned, the guard will say \emph{Y}, so there is also at
least a one-third probability of that. But if he is pardoned, what the
guard will have to decide what to say, and we can't make probability
judgements about free human decisions. On the Dempster-Shafer theory,
the probability of \emph{X} being freed is one-third, but the
probability of \emph{X} being freed and the guard saying \emph{Y} goes
to Elba is zero, and the probability of \emph{X} being freed and the
guard saying \emph{Z} goes to Elba is zero. This is just a standard
failure of additivity, and not at all objectionable. The problem is that
when the guard says that \emph{Y} will go to Elba, or that \emph{Z} will
go to Elba, the probability of \emph{x} being freed rises to one-half.
(I will not go through the mathematics here, because it can be found in
any book on the Dempster-Shafer theory. See, for example, Walley
(\citeproc{ref-Walley1991}{1991}) or Yager, Fedrizzi, and Kacprzyk
(\citeproc{ref-Yager1994}{1994}).) Since \emph{X} did not learn about
his chances of freedom, this seems like a rather odd result. The theory
of imprecise probabilities avoids this problem. It can be easily shown
that on this theory for any evidence \emph{e} if the probability of
\emph{p} given \emph{e} is greater than the probability of \emph{p},
then the probability of \emph{p} given ¬\emph{e} is less than the
probability of \emph{p}. (Again Walley (\citeproc{ref-Walley1991}{1991})
contains the proof.)

\section{Keynes and Imprecise
Probabilities}\label{keynes-and-imprecise-probabilities}

Obviously enough, this is not the theory that Keynes formally endorses,
either in his \emph{Treatise on Probability}
(\citeproc{ref-Keynes1921}{Keynes 1921}) or his economic writings.
Nevertheless, I think it is an important theory for understanding
Keynes's use of uncertainty. This is because it, and it alone, captures
all of the underlying motivations of Keynes's theory of uncertainty.
Hence any economic consequences of uncertainty Keynes wants to draw will
have to be derivable from this theory.

I have so far spoken blithely of `Keynes's theory of uncertainty',
implicitly assuming there is such a unique theory. In recent years a
number of authors (e.g. (\citeproc{ref-Runde1994a}{Runde 1994a};
\citeproc{ref-Davis1994}{Davis 1994}; \citeproc{ref-Coates1996}{Coates
1996}; \citeproc{ref-Bateman1996}{Bateman 1996})) have questioned this
assumption, saying that Keynes changed his theory between the writing of
the \emph{Treatise on Probability} and \emph{The General Theory}. I will
not deal directly with such criticisms here for a number of reasons.
First, the main dispute is over whether probabilities are given by logic
or are `merely subjective', and that debate is independent of the debate
about the effects of allowing imprecise probabilities. Secondly, there
are obvious space constraints. Many of these alternative interpretations
were put forward in book length arguments, and a fair response to them
would not be short. Thirdly, and perhaps most importantly, I take it
that the methodological game here is inference to the best explanation.
Whatever criticisms I make of others' interpretations would be rather
weak unless I showed that some other overall story was more persuasive.
And if I come up with a more persuasive story here criticisms of their
accounts will be slightly redundant. So I hope the reader at least
permits the indulgence of setting out a theory of Keynes's ideas
predicated on this rather controversial assumption.

In the \emph{Treatise on Probability} (Keynes
(\citeproc{ref-Keynes1921}{1921}), hereafter \emph{TP}) Keynes says that
probability is essentially a property of ordered pairs of propositions,
or what he calls arguments. He writes \emph{p}/\emph{q}~=~α, for the
probability of hypothesis \emph{p} on evidence \emph{q} is α. Now this
value α is rather unusual. It sometimes is a number, but sometimes not;
it sometimes can be compared to all numbers, but sometimes not; it
sometimes can be compared to other probability values such as β, but
sometimes not and it can enter into arithmetic operations. As a
consequence probabilities are subject to all the usual rules of the
classical probability calculus. For example, whenever \emph{p} and
\emph{r} are inconsistent, then
(\emph{p}~∨~\emph{r})/\emph{q}~=~\emph{p}/\emph{q}~+~\emph{r}/\emph{q}
always holds, even when none of these values is numerical.

These five properties are rather perplexing. Indeed, Keynes's failure to
explain or justify them fully is one of the main criticisms that Ramsey
(\citeproc{ref-RamseyTruthProb}{Ramsey 1926, 161--66}) launches at
Keynes's theory. But on this theory they all fall out as consequences of
our definitions. If \emph{p}/\emph{q}~=~α then α will be numerical iff
there is some \emph{x} such that for all Pr~∈~\emph{S},
Pr(\emph{p}~\textbar~\emph{q})~=~\emph{x}. Similarly
α~\textgreater~\emph{y}, for real valued \emph{y}, iff
Pr(\emph{p}~\textbar~\emph{q})~\textgreater~\emph{y} for all
Pr~∈~\emph{S}. A similar definition holds for α~\textless~\emph{y} and
α~=~\emph{y}, from which it can be seen that it is possible that α is
neither greater than, less than, nor equal to \emph{y}. If none of these
hold we say that α and \emph{y} are incomparable. If
\emph{p}/\emph{q}~=~α and \emph{r}/\emph{s}~=~β then α \textgreater{} β
iff for all Pr~∈~\emph{S},
Pr(\emph{p}~\textbar~\emph{q})~\textgreater~Pr(\emph{r}~\textbar~\emph{s}).
Again similar definitions of less than and equal to apply, and the
consequence of all these is that sometimes α and β will be comparable,
sometimes not.

Ramsey is right to question the intelligibility of Keynes's use of
addition and multiplication. We know what it means to add and multiply
numbers, but we have no idea what it is to add or multiply non-numerical
entities. However, on this theory addition and multiplication are
perfectly natural. Since we represent α and β by sets, generally
intervals, then α~+~β and α~·~β will be sets. They are defined as
follows. Again let \emph{p}/\emph{q}~=~α and \emph{r}/\emph{s}~=~β.

\begin{longtable}[]{@{}rl@{}}
\toprule\noalign{}
\endhead
\bottomrule\noalign{}
\endlastfoot
α~+~β = & \{x: ∃ Pr ∈ S
(Pr(\emph{p}~\textbar~\emph{q})~+~Pr(\emph{r}~\textbar~*s) = x)\} \\
α · β = & \{x: ∃ Pr ∈ S (Pr(\emph{p}~\textbar~\emph{q}) ·
Pr(\emph{r}~\textbar~*s) = x)\} \\
\end{longtable}

These definitions are natural in the sense that we are entitled to say
that the `+' in α~+~β means the same as the `+' in 2~+~3. And the
definitions show why Keynes's α's and β's will obey the axioms of the
probability calculus. Even if \emph{p}/\emph{q} and ¬\emph{p}/\emph{q}
are non-numerical, \emph{p}/\emph{q}~+~¬\emph{p}/\emph{q} will equal
\{1\}, or effectively 1. So we have something like the additivity axiom,
without its normal counterintuitive baggage. The main problem with
additivity is that sometimes we may have very little confidence in
either \emph{p} or ¬\emph{p}, but we are certain that
\emph{p}~∨~¬\emph{p}. If we measure confidence by the lower bound on
these probability intervals, this is all possible on our theory. Our
technical apparatus removes much of the mystery behind Keynes's theory,
and fends off an important objection of Ramsey's.

The most famous of Keynes's conceptual innovations in the \emph{TP} is
his introduction of `weight'. He does this in the following, relatively
opaque, paragraph.

\begin{quote}
As the relevant evidence at our disposal increases, the magnitude of the
probability of the argument may either decrease or increase, according
as the new knowledge strengthens the unfavourable or the favourable
evidence; but \emph{something} seems to have increased in either case,
-- we have a more substantial basis upon which to rest our conclusion. I
express this by saying that an accession of new evidence increases the
\emph{weight} of an argument. New evidence will sometimes decrease the
probability of an argument, but it will always increase its `weight'
(\citeproc{ref-Keynes1921}{Keynes 1921, 77}, italics in original).
\end{quote}

The idea is that \emph{p}/\emph{q} measures how the evidence in \emph{q}
is balanced between supporting \emph{p} and supporting ¬\emph{p}. The
concept of weight is needed if we want to also know how much evidence
there is. Note that weight only increases when relevant evidence comes
in, not when any evidence comes in. The weight of the argument from my
evidence to ``Oswald killed JFK'' is not increased when I discover the
Red Sox won last night.

The simplest definition of relevance is that new evidence \emph{e} is
irrelevant to \emph{p} given old evidence \emph{q} iff
\emph{p}/(\emph{q}∧\emph{e})~=~\emph{p}/\emph{q}, and relevant
otherwise. Now there is a problem. Two pieces of evidence
\emph{e}\textsubscript{1} and \emph{e}\textsubscript{2} can be
irrelevant taken together, but relevant taken separately. For a general
example, let \emph{e}\textsubscript{1} be \emph{p} ∨ \emph{r} and
\emph{e}\textsubscript{2} be ¬\emph{p} ∨ \emph{r}, for almost any
proposition \emph{r}. If I receive \emph{e}\textsubscript{1} and
\emph{e}\textsubscript{2} sequentially, the weight of the argument from
my evidence to \emph{p} will have increased twice as I receive these new
pieces of evidence. So it must be higher than it was when I started. But
if I just received the two pieces of evidence at once, as one piece of
evidence, I would have properly regarded it as irrelevant. Hence the
weight in question would be unchanged. So it looks as if weight depends
implausibly not on what the evidence is, but on the order in which it
was obtained.

Keynes avoids this implausibility by tightening up the definition of
irrelevance. He says that \emph{e} is irrelevant to \emph{p}/\emph{q}
iff there are no propositions \emph{e}\textsubscript{1} and
\emph{e}\textsubscript{2} such that \emph{e} is logically equivalent to
\emph{e}\textsubscript{1}∧\emph{e}\textsubscript{2} and either
\emph{e}\textsubscript{1} or \emph{e}\textsubscript{2} is relevant to
\emph{p}/\emph{q}. Unfortunately, as I noted in the previous paragraph
for virtually any such evidence proposition there will be such
propositions \emph{e}\textsubscript{1} and \emph{e}\textsubscript{2}.
This was first noticed by Carnap (\citeproc{ref-Carnap1950}{1950}).
Keynes, had he noticed this, would have had three options. He could
conceded that everything is relevant to everything, including last
night's baseball results to the identity of Kennedy's assassin; he could
have conceded that the order in which evidence appears does matter, or
he could have given up the claim that new relevant evidence
\emph{always} increases the weight of arguments.

The last option is plausible. Runde (\citeproc{ref-Runde1990}{1990})
defends it, but for quite different reasons. He thinks weight measures
the ratio of evidence we have to total evidence we believe is available.
Since new evidence might lead us to believe there is much more evidence
available than we had previously suspected, the weight might go down. I
believe it holds for a quite different reason, one borne out by Keynes's
use of uncertainty in his economics. In \emph{The General Theory}
(Keynes (\citeproc{ref-Keynes1936}{1936}), hereafter \emph{GT}) Keynes
stresses the connection between uncertainty and `low weight' (\emph{GT}:
148n). If we regard \emph{p} as merely risky the weight of the argument
from our evidence to \emph{p} is high, if we regard \emph{p} as
uncertain the weight is low. In the \emph{Quarterly Journal of
Economics} article he argues that gambling devices are, or can be
thought to be, free of uncertainty, whereas human actions are subject to
uncertainty. So the intervention of humans can take a situation from
being risky to being uncertain, and hence decrease the weight in
question.

For example, imagine we are playing a rather simple form of poker, where
each player is dealt five cards and then bets on who has the best hand.
Before the bets start, I can work out the chance that some other player,
say Monica, has a straight. So my credence in the proposition
\emph{Monica has a straight} will be precise. But as soon as the betting
starts, my credence in this will vary, and will probably become
imprecise. Do those facial ticks mean that she is happy with the cards
or disappointed? Is she betting high because she has a strong hand or
because she is bluffing? Before the betting starts we have risk, but no
uncertainty, because the relevant probabilities are all known. After
betting starts, uncertainty is rife.

The poker example supports my analysis of weight. If weight of argument
rises with reduction of uncertainty, then in some rare circumstances
weight of arguments \emph{decreases} with new evidence. Let
{[}\emph{x}\textsubscript{1}, \emph{x}\textsubscript{2}{]} be the set
given by \{\emph{x}: Pr(p\emph{~\textbar~}q) = \emph{x}\} for some
Pr~∈~\emph{S}, where \emph{S} is the agent's representor. Then the
weight of the argument from \emph{p} to \emph{q}, for this agent, is
1-(\emph{x}\textsubscript{2} - \emph{x}\textsubscript{1}). That is, the
weight is one when the agent has a precise degree of belief in \emph{p},
zero when she is totally uncertain, and increasing the narrower the
interval {[}\emph{x}\textsubscript{1}, \emph{x}\textsubscript{2}{]}
gets. Now in most cases new relevant evidence will increase the weight,
but in some cases, like when we are watching Monica, this will not
happen. I follow Lawson (\citeproc{ref-Lawson1985}{1985}) in saying that
\emph{p} is uncertain for an agent with evidence \emph{q} iff
\emph{p}/\emph{q} is non-numerical, i.e.~iff the weight of the argument
from \emph{q} to \emph{p} is less than one. Hence we get the connection
between uncertainty and weight Keynes wanted. I also claim that the
bigger \emph{x}\textsubscript{2} - \emph{x}\textsubscript{1} is, the
more \emph{p}/\emph{q} is unlike a real number, the more uncertain
\emph{p} is. Keynes clearly intended uncertainty to admit of degrees
Keynes (\citeproc{ref-Keynes1937}{1937b}), so this move is faithful to
his intent.

Keynes's theory of probability is based around some non-numerical values
whose nature and behaviour is left largely unexplained, and a concept of
weight which is subject to a telling and simple objection. Nevertheless,
his core ideas, that probabilities can but need not be precise, and that
we need a concept like weight as well as just probability, both seem
right for more general reasons. Hence the theory here, which captures
the Keynesian intuitions while explaining away his mysterious
non-numerical values and making the concept of weight more rigorous,
looks to be as good as it gets for a Keynesian theory of uncertainty.

One particularly attractive feature of the account is how conservative
it is at the technical level. We do not need to change our \emph{logic},
change which things we think are logical truths, or which things follow
from which other things, in order to support our account of uncertainty.
This is in marked contrast to accounts based on fuzzy logic or on logics
of vagueness. Not only are such changes in the logic unmotivated, they
appear to lead to mistakes. No matter how uncertain we are about how the
stock will move over the day, we know it will either close higher or not
close higher; and we know it will not both close higher and not close
higher. The classical laws of excluded middle and non-contradiction seem
to hold even in cases of massive uncertainty. This seems to pose a
serious problem for theories of uncertainty based on alternative logics.
The best approach is one, like the theory here, which is innovative in
how it accounts for uncertainty, and conservative in the logic it
presupposes.

So as a theory of uncertainty I think this account has a lot to be said
for it. However, it cannot support the economic arguments Keynes rests
on it.

\section{The Economic Consequences of
Uncertainty}\label{the-economic-consequences-of-uncertainty}

Uncertainty can impact on the demand for an investment in two related
ways. First, it can affect the value of that particular investment;
secondly, it can affect the value of other things which compete with
that investment for capital. The same story is true for investment as a
whole. First, uncertainty may reduce demand for investment
\emph{directly} by making a person who would otherwise be tempted to
invest more cautious and hence reluctant to invest. Secondly, if this
direct impact is widespread enough, it will increase the demand for
money, and hence its price. But the price of money is just the market
rate of interest. And the return that an investment must be expected to
make before anyone, even an investor not encumbered by uncertainty, will
make it is the rate of interest.

When uncertainty reduces investment by increasing interest rates, I will
say it has an \emph{indirect} impact on investment. Keynes has an
argument for the existence of this indirect impact. First, he takes the
amount of consumption as a given (\emph{GT}:~245). Or more precisely,
for any period he takes the amount of available resources that will
\emph{not} be allocated to consumption as a given. There are three
possible uses for these resources: they can be invested, they can be
saved as bonds or loans, or they can be hoarded as money. There are many
different types of investment, but Keynes assumes that any agent will
already have made their judgement as to which is the best of these, so
we need only consider that one. There will also be many different length
bonds which the agent can hold. So as to simplify the discussion, Keynes
proposes just treating these two at a time, with the shorter length bond
called `money' and the longer length loan called `debts' (\emph{GT}:
167n). Hence the rate of interest is the difference between the expected
return of the shorter bond over the life of the longer bond and the
return of the longer bond. So the rate of interest that we are
interested in need not be positive, and when the two bond lengths are
short will usually be zero. However, it is generally presumed in
discussions that the rate is positive. Now, Keynes assumes that an agent
will only allocate resources to investment if investment looks to be at
least as worthwhile as holding money, and at least as worthwhile as
holding debts. In other words, he makes the standard reduction of
\emph{n}-way choice to a set of 2-way choices\footnote{Standard, but I
  bring it up because the modern theorist whose decision theory is
  closest to the one Keynes seems to adopt, Levi, explicitly rejects it.}.
Usually if someone is of a mind to invest they will not favour holding
money over holding debts. The only motivation for holding money, given
positive interest rates, could be a desire to have accessible command
over purchasing power, and investment foregoes that command. So in
practice we only need look at two of the three possible pairwise choices
here. Hence I will ignore the choice between investing and holding
money, and only look at the money-debt choice and the debt-investment
trade-off.

Holding a debt provides a relatively secure return in terms of money.
Relatively secure because there is the possibility of default. In
practice, this means that there is not a sharp distinction between debts
and investments, rather a continuum with say government bonds at one
extreme and long-term derivatives at the other. Some activities that
have the formal structure of `debts', like say provision of venture
capital, will be closer to the investment end of the continuum. Unlike
debts then, investments as a rule do not have a secure return in terms
of money. In most cases they do not even have a precise expected return
(\emph{GT}: 149; Keynes (\citeproc{ref-Keynes1937}{1937b, 113})). Keynes
does not presume that this means that people never invest unless the
expected return on the investment is greater than the expected (indeed,
known) return on debts. He says explicitly that were this true then
`there might not be much investment'. Instead, he says that investment
under uncertainty depends on `confidence' (\emph{GT}: 150). Therefore,
the following looks compatible with his position.

Bayesians say that each gamble has a precise expected value. The
expected return on a bet that pays \$1 if some fair coin lands heads is
50 cents. On this theory, expected values are imprecise, because
probabilities are imprecise. Formally, say E\textsubscript{Pr}(G) = α
means that the expected return on \emph{G} according to probability
function Pr is α. Roughly, the expected value for an agent of a gamble
\emph{G} will be \{x: ∃Pr ∈ S: (E\textsubscript{Pr}(G) = x)\}, the set
of expected values of the bet according to each probability function in
the agent's representor. Note that these are different from the possible
outcomes of the bet. As we saw in the case of the coin, expected value
can differ from any possible value of the bet. So let the expected value
of investing a certain sum be {[}α, β{]}, and the expected value of
buying a debt with that money be χ. Then the agent will invest iff (1 -
ρ)α~+~ρβ ⩾ χ, where ρ ∈ {[}0, 1{]} measures the `state of
confidence'.\footnote{In case the reader fears I am being absurdly
  formal with an essentially informal idea, Keynes had such a variable,
  there described as measuring the `state of the news', in early drafts,
  but it did not survive to the final stage. So my proposal is not a
  million miles from what Keynes intended merely by virtue of being
  algebraic.} Now when a crisis erupts, ρ will go to 0, and investment
will dry up. In such cases the decision theory is similar to the one
advanced by Levi (\citeproc{ref-Levi1980}{1980}), Strat
(\citeproc{ref-Strat1990}{1990}) and Jaffray
(\citeproc{ref-Jaffray1994}{1994}). Since we are interested in a theory
of unemployment, we are primarily interested in the cases where ρ is
quite low, in which cases we can say uncertainty is reducing investment.

That last statement might seem dubious at face value. In part, what I
mean by it is this. When ρ is low the value of a set of bets will in
general be \emph{more} than the sum of the value of the bets taken
separately. Because individual investors are fearful of exposure to
uncertainty, which is presumably what ρ being low means, sets of
investments which if undertaken collectively would be profitable (and
everyone agrees that they would) will not be undertaken individually.
This suggests a reason that theorists have thought government
intervention might be appropriate in times of crisis. Alternatively, if
ρ is low then the value of an investment, how much we will be prepared
to pay for it, will probably be lower than our best estimate of its
expected return, assuming the latter to be near (α~+~β) /2.

I shall focus more closely on the indirect effects of uncertainty in
section 5. The central idea is that the rate of interest, being the
price of money, is completely determined in the market for money.
However, this market has some rather strange properties. After all,
money is barren, and it can generally be traded for something that is
not barren. So, as Keynes puts it, why would anyone `outside a lunatic
asylum', want it? Why would the demand for money not drop to zero as
soon as the rate of interest is positive?

\begin{quote}
Because, partly on reasonable and partly on instinctive grounds, our
desire to hold money as a store of wealth is a barometer of the degree
of our distrust of our own calculations and conventions concerning the
future \ldots{} The possession of actual money lulls our disquietude;
and the premium which we require to make us part with money is the
measure of the degree of our disquietude
(\citeproc{ref-Keynes1937}{Keynes 1937b, 116}).
\end{quote}

Therefore, more uncertainty means more demand for money means higher
interest rates. The rest of the story is standard. Even the confident
agent will be disinclined to invest once the rate of interest rises.
Using the little decision theory outlined above, more uncertainty means
the gap between α and β grows, which if ρ is low will tend to reduce
(1-ρ)α~+~ρβ, the `certainty equivalent' of the expectation of the
investment's worth. On the other hand, uncertainty on the part of the
community will tend, for similar reasons, to increase χ. Either way,
investment suffers, and hence so does employment.

\section{Uncertainty and Money}\label{uncertainty-and-money}

There is something very odd about all that we have done so far. Agents
react to uncertainty by making their returns measured in dollars more
stable. However, in doing so they make their returns measured in any
other good less stable. If you have no idea what the price of widgets
will be in twelve months time, then holding only widgets increases the
uncertainty about how many dollars you will be worth then. However, it
makes you more certain about how many widgets you will be worth. Why
this preference for money? We deserve an explanation as to why one kind
of uncertainty is given such a central place and other kinds are
completely ignored.

Keynes has one explanation. He argues, or perhaps assumes, essentialism
about money. Indeed the title of chapter 17 of \emph{The General Theory}
is `The Essential Properties of Interest and Money'. These essential
properties are entirely functional. As Hicks puts it, ``Money is defined
by its functions \ldots{} money is what money does''
(\citeproc{ref-Hicks1967}{Hicks 1967, 1}). The explanation is that
agents try to minimise uncertainty relative to whatever plays the
functional role of money. Therefore, the explanation does not rely on
any mystical powers of dollar bills. Rather, the work is done by the
functional analysis of money.

As a first approximation, the functional role money plays is that it is
a medium of exchange. Keynes does not think this is quite the essential
property; rather he says that money is essentially `liquid', and
perceived to be liquid. This means that if we hold money we are in a
position to discharge obligations and make new purchases as they seem
appropriate with greatest convenience and least cost. Even this is not
what is given as the official essential property of money. To make the
proof that demand for money is not demand for labour easier Keynes takes
the essential properties of money to be its negligible elasticities of
production and substitution. However, he makes clear that these are
important because of their close connection to liquidity (\emph{GT}:
241). Indeed, when he comes to define a non-monetary economy, he simply
defines it as one where there is no good such that the benefits it
confers via its liquidity, its `liquidity premium' exceeds the carrying
costs of the good. So the properties of having a negligible elasticity
of production and substitution seem necessary but insufficient for
something to be money.

The reason that money uncertainty is more problematic than widget
uncertainty is just that money is liquid. At the end of the day, the
point of holding investments, bonds or money is not to maximise the
return in terms of such units; it is to be used somehow for consumption.
Hence, we prefer, \emph{ceteris paribus}, to store wealth in ways that
can be easily exchanged for consumption goods as and when required.
Further, we may be about to come across more information about
productive uses for our wealth, and if we do, we would prefer to have
the least inconvenience about changing how we use wealth. Money is going
to be the best store of wealth for each of these purposes. The strength
of these preferences determines the liquidity premium that attaches to
money.

So Keynes's story here is essentially a `missing markets' story. If
there were markets for every kind of transaction there would be no
liquidity premium attaching to money, and hence no reason to be averse
to uncertainty in terms of money returns as opposed to uncertainty in
terms of X's shares returns. There is a methodological difference here
between decision theorists and economists. In decision theory it is
common to specify what choices an agent does have. These will usually be
finite, or at least simply specified. In economics it is more common to
specify what choices an agent does not have, which markets are
`missing'. In a sense the difference is purely cosmetic, but it can
change the way problems are looked at. Since Keynes requires here some
markets to be missing, it might be worth investigating what happens here
from the more restrictive framework ordinarily applied in decision
theory.

In some decision-theoretic contexts, we can prefer liquidity even when
we are completely certain about what our choices are and what their
outcomes will be. Say we are in a game where the object is to maximise
our money over 2 days. We start with \$100. On day 1, we have a choice
of buying for \$100 a ticket that will pay \$200 at the end of day 2,
and is non-transferable, or doing nothing. On day 2, if we still have
our \$100, we can buy with it a voucher which pays \$300 at the end of
day 2, or do nothing. Obviously, the best strategy is to do nothing on
day 1, and buy the voucher on day 2. The point is just that money here
has enough of a liquidity premium on day 1 that we are prepared to hold
it and earn no interest for that day rather than buy the ticket (or two
day bond) which will earn interest. So uncertainty is not a necessary
condition for liquidity premia to exist. On the other hand, perhaps it
is necessary for liquidity premia to exist in a world something like
ours, where agents neither have all the choices they would have in a
perfect market, nor as few as in this simple game. If we added a market
for tickets and vouchers to our simple game the prices would be fixed so
that money would lose its liquidity premium. Keynes suggests something
like this is true for the worlds he is considering: ``\emph{uncertainty}
as to the future course of the rate of interest is the sole intelligible
explanation of the type of liquidity preference {[}under
consideration{]}'' (\emph{GT}:~201). However here he merely means lack
of certainty; there is no proof that if every agent had precise
credences liquidity preference ought to disappear. So it looks like
uncertainty in the sense discussed here, vague reasonable beliefs, does
no theoretical work. Perhaps this is a bit quick, as the little game I
considered is so far from a real-life situation. So I will look more
closely at the effects uncertainty is supposed to have. Since it has
received the bulk of the theoretical attention, I start with the
indirect effects of uncertainty.

\section{Uncertainty and Liquidity
Preference}\label{uncertainty-and-liquidity-preference}

Keynes thinks the question of why money is demanded at all, why we do
not all move from holding money into holding debts as soon as the rate
of interest goes positive, needs answering. And he thinks the answer
here will be particularly relevant to theories about the rate of
interest. If the market in general is at equilibrium then the market in
trades between any two goods must also be in equilibrium; in particular
it cannot be that there are people holding money who would be prepared
to buy debts at the current interest rate. So if the equilibrium
interest rate is positive, there must be some people who would prefer to
hold money than hold debts. This fact Keynes takes to be central to the
correct theory of the rate of interest. Hence, to determine what the
rate of interest will be, and what will cause it to change, I need to
determine what causes a demand for money.

Keynes distinguishes four motives for holding money (\emph{GT}: Ch. 13;
(\citeproc{ref-Keynes1937b}{Keynes 1937a, 215--23})). Two of these, the
transactions motive and the finance motive, need not detain us. They
just relate to the need to make payments in money and on time. The
third, the speculative motive, is often linked to uncertainty, and
indeed Keynes does so (\emph{GT}: 201). But `uncertainty' here is just
used to mean absence of certainty, that is the existence of risk, which
as noted above is not how I am using `uncertainty'. As Runde
(\citeproc{ref-Runde1994b}{1994b}) points out, an agent who is certain
as to future movements in interest rates may still hold money for
speculative reasons, as long as other agents who are not so certain have
made mistaken judgements. The fourth motive will hold most of my
attention. Keynes argues that we may hold money for purely precautionary
reasons.

\begin{quote}
To provide for contingencies requiring sudden expenditure and for
unforeseen opportunities of advantageous purchases, and also to hold an
asset of which the value is fixed in terms of money to met a subsequent
liability fixed in terms of money, are further motives for holding cash
(\emph{GT}: 196).
\end{quote}

Davidson (\citeproc{ref-Davidson1988}{1988},
\citeproc{ref-Davidson1991}{1991}) justifies this as follows.
Uncertainty arises whenever agents do not have sufficient knowledge to
calculate the numerical probability of an event. This is given a rather
frequentist gloss in Davidson, but that is not necessary. His idea is
that we know what the probability of \emph{p} is when we know the
frequency of \emph{p}-type events in the past and we know the future
will resemble the past in this respect. The latter is cashed out as
saying \emph{p} is governed by an `ergodic process'. We can replace all
this by saying that \emph{p} is subject to uncertainty whenever we do
not know its objective chance, whether or not objective chance ought to
be analysed by frequentist approaches. Davidson then argues that since
for most \emph{p} we do not have this knowledge, we have to adopt
`sensible' approaches like holding money.

Runde (\citeproc{ref-Runde1994b}{1994b}) objects that Davidson's story
is incoherent. On Davidson's theoretical story there are only two
epistemic states relative to \emph{p} that are possible. An agent can
know the chance of \emph{p}, in which case their credence is set equal
to it, or they are completely uncertain about it. In the latter case
there can be no reason for taking some action rather than another. Now
the reason that it is `sensible' to hold money is that we expect money
to be liquid. However, we do not know the chance of money remaining
liquid; whether or not money remains liquid is not determined by an
ergodic process. Hence, we have no reason for letting that partial
belief be a guide to action.

This is a fair criticism, but it can be met by amending the theory
rather than by giving it up. On my theory, if an agent knows the chance
of \emph{p} they will have a precise degree of belief in \emph{p}. When
they do not their degree of belief will, in general, be vague but not
totally vague. As with Keynes, I have uncertainty come in degrees. This
amendment is enough to rescue Davidson's theory. An agent might not know
the chance that money will become illiquid in the next short period of
time, but they might know enough for it to be reasonable to have a
credence in that proposition which is vague over some small interval
close to zero. It may still be sensible to hold some money even when the
expected return on other investments really is vague. But is it sensible
to prefer fixed to uncertain returns? In other words, is there a direct
effect of uncertainty that makes people prefer bonds to investments?

\section{Uncertainty and Indecision}\label{uncertainty-and-indecision}

As Keynes repeatedly stressed, investment is not like a game of chance
where the expected results are known in advance. And this is part of the
explanation for the extreme instability in investment levels compared to
other economic variables.

\begin{quote}
The state of long-term expectation \ldots{} does not solely depend on
the most probable forecast we can make. It also depends on the
\emph{confidence} with which we make this forecast (\emph{GT}: 148).

Human decisions affecting the future, whether personal or political or
economic, cannot depend on strict mathematical expectation, since the
basis for making such calculations does not exist \ldots{} it is our
innate urge to activity which makes the wheels go round, our rational
selves choosing between the alternatives as best we are able,
calculating where we can, but often falling back for our motive on whim
or sentiment or chance (\emph{GT}: 162-3).
\end{quote}

The most charitable reading of Keynes here is to say he agreed, in
principle, with what is sometimes referred to as a Horwitz-style
decision rule. If the expected return of an investment is vague over
{[}α, β{]} then its `value' is given by (1-ρ)α~+~ρβ, where ρ
∈~{[}0,~1{]} is a measure of confidence. By the 1937 article, he has
become more interested in the special case where confidence has
collapsed and ρ is approaching 0. This interpretation would explain all
his references to decision-making under uncertainty in \emph{The General
Theory} and subsequent discussion, provided we make the safe assumption
that `cold calculation' would only have us spend \emph{x} on an
investment with expected return {[}α, β{]} when α ⩾~\emph{x}. In
particular, any interpretation of the underlying decision theory here
will have to give some role to `whim or sentiment or chance', and I give
it a variable, `ρ'. With this theory, I have the extensions needed to
avoid Runde's objection to Davidson. I have a continuum of degrees of
uncertainty, rather than a raw dichotomy, and I have an explanation of
why it is `sensible' to prefer gambles with known expected returns, at
least when ρ is relatively low.

This theory is meant to serve two related purposes. It is meant to show
why we might prefer money to debts, even though our best estimate of the
expected return of the debts is positive, and again it is meant to show
why we might prefer debts to investments even when our best estimate of
the expected return of the investment is higher. And I think if the
decision rule stipulated were plausible, it would show that uncertainty
did have an economic effect. In particular, I think it would show both
that in times of crises when ρ heads down, the level of investment will
decrease even with other things being equal, and that collective action
can be justified even when individual action is not. That is, the
government can make sets of investments that are expected to be
profitable although none of the individual investments is expected to be
profitable.

The decision theory does not, however, seem plausible. First, there are
some technical problems for this theory. The problem is that if ρ
\textless{} ½, then in cases where uncertainty is guaranteed to increase
in the near future agents following this rule will make decisions they
are sure to regret. For example, assume an agent with ρ = ⅓ now has
credence ½ in \emph{p}, but knows that some evidence will come in such
that her credence in \emph{p} will become vague over {[}0.3,~0.7{]}
whatever the result of the experiment. As we saw in the case of poker
players, this is plausible in some situations. The agent will now pay 50
cents for a bet which pays \$1 if \emph{p} and nothing otherwise, but
after the evidence comes in she'll sell that bet for about 44 cents,
incurring a sure loss. I leave it to the reader to judge the importance
of these technical problems, given the rarity of cases where uncertainty
is guaranteed to rise.

There is also a philosophical problem. What precisely is ρ supposed to
represent? If it is some kind of belief, its effects should have been
incorporated into the credences. If it is some kind of desire its
effects should have been incorporated into the evaluation of each of the
states. This objection could be avoided, perhaps, if Keynes was trying
to argue against the theory that investors just maximise dollar expected
returns. It is not entirely clear whom Keynes thinks he is arguing
against at some points. If this is his enemy, he is fighting a straw
man, one who is vulnerable to much simpler objections. Whoever thought
that all investment is profit driven, that no one ever went into
business because they thought it would be fun to run a newspaper?
Keynes's only viable opponents here are saying that investors calculate
the expected return, in utils, of each possible investment and choose
the one whose returns are highest. Now perhaps for many dollar returns
are the most important factor in determining util returns, but this is
certainly not the only cause.

If ρ represents something which is neither a belief nor a desire, then
it is hard to see what effect it could have on action. Perhaps there are
some exceptions to the rule that actions are caused only by beliefs and
desires combining in the right way, such as actions caused by values,
but these appear irrelevant to Keynes's considerations, and he does not
appeal to such exemptions. After all, he describes investment decisions
made where the `cold calculations' do not determine what should be done
as being made by `whim or sentiment or chance'. Now whims and sentiments
are surely desires, although chance is in a different boat. If he had
just said `chance' here he may have committed himself to a different
decision theory, one where the agent can under uncertainty make any
decision which is not known to be sub-optimal. But this does not justify
the conclusion that uncertainty decreases investment; under that theory
it is random whether uncertainty increases or decreases investment.
Hence Keynes appears to be implausibly committed to a mental state which
is neither a belief nor a desire but affects action.

It might be objected here that I am relying on an overly individualistic
theory of motivation; that what Keynes is committed to is nothing more
than what anyone who has learned the difference between Robinson Crusoe
economics and real-world economics would believe. There is an important
truth behind this objection: the social causes of action cannot be
overlooked. But this is not what I have done. The core assumption I made
is that the only mental states relevant to action are beliefs and
desires. Now the beliefs and desires that are relevant may not be
(directly) concerned with the action at hand; they may be beliefs and
desires about how society will view this action, or about similar
actions which may or may not be performed by other members in society.
And the beliefs and desires may not have as their immediate cause
careful inference by the agent in question; they may be caused by the
wave of panic or optimism in which the agent is caught up. In the real
world, agents do not always change their beliefs and desires by
reflection on new evidence, often emotion plays a larger role. So
society has both evidential and non-evidential effects on action. But
every time, the causal chain goes via the beliefs and desires of the
agent. Society causes actions by causing changes in the beliefs and
desires of individuals. It is wrong to think that action is never caused
by beliefs and desires about society, it is wrong to think that society
never directly causes beliefs and desires which lead to action, but none
of this implies that there can be mental states other than belief and
desire relevant to action.

\section{Disquietude}\label{disquietude}

There are some comments from Keynes that suggest this reading is a
little unfair. Rather than having a distinctive decision theory, he
perhaps has a distinctive theory about what ought enter into the
decision-theoretic calculations. The standard theory for why there is a
demand for insurance is the falling marginal utility of money. Agents
purchase insurance, and accept a lower expected dollar return because
with insurance their expected util return, at the end of the duration of
the insurance, is higher than if they had not purchased. This is the
story given in, for example, Freidman and Savage
(\citeproc{ref-FreidmanSavage1952}{1952}) where the existence of demand
for insurance is taken as evidence for the declining marginal utility of
money. But there is another reason agents might buy insurance. They
might simply feel happier, over the duration of the insured period,
knowing that they have insurance and are hence exposed to fewer risks or
uncertainties than otherwise. If this is true then their expected
`wealth' in both dollars and utils at the end of a period might be lower
if they insure than if otherwise, but it will be worthwhile because of
the benefits during the period. Keynes suggests that this same desire
for quietude can cause a demand for money. I presume, though it is not
entirely clear, that this desire should be included within the
precautionary motives for holding money.

\begin{quote}
There are not two separate factors affecting the rate of investment,
namely, the schedule of the marginal efficiency of capital {[}the
expected return of investments{]} and the state of confidence. The state
of confidence is relevant because it is one of the major factors
determining the former (\emph{GT}: 149).

For the fact that each individual investor flatters himself that his
commitment is ``liquid'' \ldots{} calms his nerves and makes him much
more willing to run a risk (\emph{GT}: 160).

The possession of actual money lulls our disquietude; and the premium
which we require to make us part with money is the measure of the degree
of our disquietude (\citeproc{ref-Keynes1937}{Keynes 1937b, 116}).

A liquidity premium \ldots{} is not even expected to be rewarded. It is
a payment, not for the expectation of increased tangible income at the
end of the period, but for an increase sense of comfort and confidence
during the period (\citeproc{ref-Keynes1938}{Keynes 1938, 293--94}).
\end{quote}

This explanation of the demand for certain returns is in some ways
conservative and some ways radical. It is conservative because it does
not immediately change the technical properties of preference. Many
heterodox theories of preference drop such theoretical restrictions as
transitivity of preferences. By contrast the theory Keynes appears to be
advocating is it least in principle conservative on this front. Agents
are still going round maximising expected utility, just now it is
expected utility over a period, not at the end of the period.

But it is not all conservative. If we explain economic decisions in
terms of the disquietude of the investor we discard the distinction
between investment and consumption. It was always known that there were
some goods that were not comfortably categorised, particularly cars, but
this move makes every good in part a consumption good. If all this meant
was that some helpful classifications have to be questioned, it would
not be important. Rather, its importance flows from its implications for
the norms for investment. It is always irrational to make an investment
which will incur a sure loss. This principle is used to derive
wide-ranging implications for decision-theory. But it is not irrational
to make a consumption decision which will result in sure loss at the end
of a period in exchange for goods during that period. It is not always
irrational to pay ten dollars for a movie ticket, even though this will
incur a sure loss in the sense the buyer will surely have less wealth at
the end of the movie than if they had not bought the ticket.

Given this, the technical complaint I raised against the Horvitz-style
decision rule misses the target. And the philosophical concern about
what ρ represents is irrelevant. If the expected returns only measure
how much various gambles will be worth at the end of the period, then
some desires have not yet been included in our calculations. That is, ρ
represents some desires but the theory is not guilty of double-counting.
So far this all seems to work, and explain the role of uncertainty.
Indeed, I think this is the best extension of Keynes's views in this
area.

While there seem to be few theoretical objections which can be raised at
this point, there is a rather telling empirical objection. The only role
given to disquietude in this theory is in deciding between alternatives
where the returns on at least one are uncertain. But it seems
implausible that disquietude could have this effect, but have no effect
when choices are being made between alternatives where at least one is
risky. I doubt the feelings of disquiet would be any different were I to
have a large fortune riding on a roulette wheel or a baseball game.
Disquietude arises because we do not know what will happen; maybe for
some people it is greater when we do not know the expected returns, but
I doubt it. Again, perhaps there is an explanation for demand for money
in the real world to be found here, but uncertainty plays no role in the
story, or at best a small cameo.

\section{Summary}\label{summary}

Keynes argued that uncertainty has a major economic impact. By driving
people to store their wealth in ways with more stable returns, it
increases the demand for cash and decreases the demand for investments.
Not only does it drive down investments in this direct way, the
increased demand for cash leads to higher interest rates and hence
people are driven out of investment into bonds. However, there are a few
problems with the story. First, the motivation for demanding returns
fixed with respect to a certain good can only be that the markets
between that good and other goods are more complete. But if that is the
case there is a reason to demand that good even when the world is
completely certain. Secondly, the only decision-theoretic justification
for this demand for fixed returns could be the disquiet generated by not
knowing the return. This follows from the formalisation of uncertainty
advocated in sections 1 and 2. But this disquiet could just as easily be
generated by risk as by uncertainty. So Keynes has not shown that
uncertainty has any particular economic impact. That's the bad news. The
good news is that many of the arguments seem to work without the
reliance on uncertainty.

\subsection*{References}\label{references}
\addcontentsline{toc}{subsection}{References}

\phantomsection\label{refs}
\begin{CSLReferences}{1}{0}
\bibitem[\citeproctext]{ref-Bateman1996}
Bateman, Bradley. 1996. \emph{Keynes's Uncertain Revolution.} Ann Arbor:
University of Michigan Press.

\bibitem[\citeproctext]{ref-Carnap1950}
Carnap, Rudolf. 1950. \emph{Logical Foundations of Probability}.
Chicago: University of Chicago Press.

\bibitem[\citeproctext]{ref-Coates1996}
Coates, John. 1996. \emph{The Claims of Common Sense}. Cambridge:
Cambridge University Press.

\bibitem[\citeproctext]{ref-Davidson1988}
Davidson, Paul. 1988. {``A Technical Definition of Uncertainty and the
Long-Run Non-Neutrality of Money.''} \emph{Cambridge Journal of
Economics} 12: 329--38. doi:
\href{https://doi.org/10.1093/oxfordjournals.cje.a035063}{10.1093/oxfordjournals.cje.a035063}.

\bibitem[\citeproctext]{ref-Davidson1991}
---------. 1991. {``Is Probability Theory Relevant for Uncertainty? A
Post Keynesian Perspective.''} \emph{Journal of Economic Perspectives} 5
(1): 129--44. doi:
\href{https://doi.org/10.1257/jep.5.1.129}{10.1257/jep.5.1.129}.

\bibitem[\citeproctext]{ref-Davis1994}
Davis, John. 1994. \emph{Keynes's Philosophical Development}. Cambridge:
Cambridge University Press.

\bibitem[\citeproctext]{ref-Dempster1967}
Dempster, Arthur. 1967. {``Upper and Lower Probabilities Induced by a
Multi-Valued Mapping.''} \emph{Annals of Mathematical Statistics} 38:
325--39. doi:
\href{https://doi.org/10.1214/aoms/1177698950}{10.1214/aoms/1177698950}.

\bibitem[\citeproctext]{ref-vanFraassen1990}
Fraassen, Bas van. 1990. {``Figures in a Probability Landscape.''} In
\emph{Truth or Consequences}, edited by J. M. Dunn and A. Gupta,
345--56. Amsterdam: Kluwer.

\bibitem[\citeproctext]{ref-FreidmanSavage1952}
Freidman, M., and L. Savage. 1952. {``The Expected Utility Hypothesis
and the Measurability of Utility.''} \emph{Journal of Political Economy}
60 (6): 463--74. doi:
\href{https://doi.org/10.1086/257308}{10.1086/257308}.

\bibitem[\citeproctext]{ref-GardenforsSahlin1982}
Gärdenfors, Peter, and Nils-Eric Sahlin. 1982. {``Unreliable
Probabilities, Risk Taking and Decision Making.''} \emph{Synthese} 53
(3): 361--86. doi:
\href{https://doi.org/10.1007/bf00486156}{10.1007/bf00486156}.

\bibitem[\citeproctext]{ref-Hart1942}
Hart, A. G. 1942. {``Risk, Uncertainty and the Unprofitability of
Compounding Probabilities.''} In \emph{Studies in Mathematical Economics
and Econometrics}, edited by F. McIntyre O. Lange and T. O. Yntema.,
110--18. Chicago: University of Chicago Press.

\bibitem[\citeproctext]{ref-Hicks1967}
Hicks, John. 1967. \emph{Critical Essays in Monetary Theory}. Oxford:
Clarendon Press.

\bibitem[\citeproctext]{ref-Jaffray1994}
Jaffray, J. Y. 1994. {``Decision Making with Belief Functions.''} In
\emph{Advances in the Dempster- Shafer Theory of Evidence}, edited by R.
Yager, M. Fedrizzi, and J. Kacprzyk, 331--52. New York: John Wiley.

\bibitem[\citeproctext]{ref-Jeffrey1983}
Jeffrey, Richard. 1983. {``Bayesianism with a Human Face.''} In
\emph{Testing Scientific Theories}, edited by J. Earman (ed.).
Minneapolis: University of Minnesota Press.

\bibitem[\citeproctext]{ref-Keynes1921}
Keynes, John Maynard. 1921. \emph{Treatise on Probability}. London:
Macmillan.

\bibitem[\citeproctext]{ref-Keynes1936}
---------. 1936. \emph{The General Theory of Employment, Interest and
Money}. London: Macmillan.

\bibitem[\citeproctext]{ref-Keynes1937b}
---------. 1937a. {``The Ex Ante Theory of the Rate of Interest.''}
\emph{Economic Journal} 47 (188): 663--68. doi:
\href{https://doi.org/10.2307/2225323}{10.2307/2225323}. Reprinted in
\cite[XIV 215-223]{KeynesCW}, references to reprint.

\bibitem[\citeproctext]{ref-Keynes1937}
---------. 1937b. {``The General Theory of Employment.''}
\emph{Quarterly Journal of Economics} 51 (2): 209--23. doi:
\href{https://doi.org/10.2307/1882087}{10.2307/1882087}. Reprinted in
\cite[XIV 109-123]{KeynesCW}, references to reprint.

\bibitem[\citeproctext]{ref-Keynes1938}
---------. 1938. {``Letter to Hugh Townshend Dated 7 December.''} In
\emph{The Collected Writings of John Maynard Keynes}, by John Maynard
Keynes, 14:293--94. London: Macmillan. Edited by D. E. Moggridge.

\bibitem[\citeproctext]{ref-Lawson1985}
Lawson, Tony. 1985. {``Uncertainty and Economic Analysis.''}
\emph{Economic Journal} 95 (380): 909--27. doi:
\href{https://doi.org/10.2307/2233256}{10.2307/2233256}.

\bibitem[\citeproctext]{ref-Levi1980}
Levi, Isaac. 1980. \emph{The Enterprise of Knowledge}. Cambridge, MA.:
MIT Press.

\bibitem[\citeproctext]{ref-Levi1982}
---------. 1982. {``Ignorance, Probability and Rational Choice.''}
\emph{Synthese} 53 (3): 387--417. doi:
\href{https://doi.org/10.1007/bf00486157}{10.1007/bf00486157}.

\bibitem[\citeproctext]{ref-RamseyTruthProb}
Ramsey, Frank. 1926. {``Truth and Probability.''} In \emph{Philosophical
Papers}, edited by D. H. Mellor, 52--94. Cambridge: Cambridge University
Press.

\bibitem[\citeproctext]{ref-Runde1990}
Runde, Jochen. 1990. {``Keynesian Uncertainty and the Weight of
Arguments.''} \emph{Economics and Philosophy} 6 (2): 275--93. doi:
\href{https://doi.org/10.1017/s0266267100001255}{10.1017/s0266267100001255}.

\bibitem[\citeproctext]{ref-Runde1994a}
---------. 1994a. {``Keynes After Ramsey: In Defence of {`a Treatise on
Probability'}.''} \emph{Studies in the History and Philosophy of
Science} 25 (1): 97--124. doi:
\href{https://doi.org/10.1016/0039-3681(94)90022-1}{10.1016/0039-3681(94)90022-1}.

\bibitem[\citeproctext]{ref-Runde1994b}
---------. 1994b. {``Keynesian Uncertainty and Liquidity Preference.''}
\emph{Cambridge Journal of Economics} 18: 129--44. doi:
\href{https://doi.org/10.1093/oxfordjournals.cje.a035266}{10.1093/oxfordjournals.cje.a035266}.

\bibitem[\citeproctext]{ref-Shafer1976}
Shafer, Glenn. 1976. \emph{A Mathematical Theory of Evidence}.
Princeton: Princeton University Press.

\bibitem[\citeproctext]{ref-Strat1990}
Strat, Thomas. 1990. {``Decision Analysis Using Belief Functions.''}
\emph{International Journal of Approximative Reasoning} 4 (5-6):
391--417. doi:
\href{https://doi.org/10.1016/0888-613x(90)90014-s}{10.1016/0888-613x(90)90014-s}.

\bibitem[\citeproctext]{ref-Tintner1941}
Tintner, Gerhard. 1941. {``The Theory of Choice Under Subjective Risk
and Uncertainty.''} \emph{Econometrica} 9 (3/4): 298--304. doi:
\href{https://doi.org/10.2307/1907198}{10.2307/1907198}.

\bibitem[\citeproctext]{ref-Walley1991}
Walley, Peter. 1991. \emph{Statisical Reasoning with Imprecise
Probabilities}. London: Chapman \& Hall.

\bibitem[\citeproctext]{ref-Yager1994}
Yager, R., M. Fedrizzi, and J. Kacprzyk, eds. 1994. \emph{Advances in
the Dempster- Shafer Theory of Evidence}. New York: John Wiley.

\end{CSLReferences}



\noindent Published in\emph{
Cambridge Journal of Economics}, 2001, pp. 47-62.


\end{document}
