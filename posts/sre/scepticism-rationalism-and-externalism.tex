% Options for packages loaded elsewhere
% Options for packages loaded elsewhere
\PassOptionsToPackage{unicode}{hyperref}
\PassOptionsToPackage{hyphens}{url}
%
\documentclass[
  11pt,
  letterpaper,
  DIV=11,
  numbers=noendperiod,
  twoside]{scrartcl}
\usepackage{xcolor}
\usepackage[left=1.1in, right=1in, top=0.8in, bottom=0.8in,
paperheight=9.5in, paperwidth=7in, includemp=TRUE, marginparwidth=0in,
marginparsep=0in]{geometry}
\usepackage{amsmath,amssymb}
\setcounter{secnumdepth}{3}
\usepackage{iftex}
\ifPDFTeX
  \usepackage[T1]{fontenc}
  \usepackage[utf8]{inputenc}
  \usepackage{textcomp} % provide euro and other symbols
\else % if luatex or xetex
  \usepackage{unicode-math} % this also loads fontspec
  \defaultfontfeatures{Scale=MatchLowercase}
  \defaultfontfeatures[\rmfamily]{Ligatures=TeX,Scale=1}
\fi
\usepackage{lmodern}
\ifPDFTeX\else
  % xetex/luatex font selection
  \setmathfont[]{Garamond-Math}
\fi
% Use upquote if available, for straight quotes in verbatim environments
\IfFileExists{upquote.sty}{\usepackage{upquote}}{}
\IfFileExists{microtype.sty}{% use microtype if available
  \usepackage[]{microtype}
  \UseMicrotypeSet[protrusion]{basicmath} % disable protrusion for tt fonts
}{}
\usepackage{setspace}
% Make \paragraph and \subparagraph free-standing
\makeatletter
\ifx\paragraph\undefined\else
  \let\oldparagraph\paragraph
  \renewcommand{\paragraph}{
    \@ifstar
      \xxxParagraphStar
      \xxxParagraphNoStar
  }
  \newcommand{\xxxParagraphStar}[1]{\oldparagraph*{#1}\mbox{}}
  \newcommand{\xxxParagraphNoStar}[1]{\oldparagraph{#1}\mbox{}}
\fi
\ifx\subparagraph\undefined\else
  \let\oldsubparagraph\subparagraph
  \renewcommand{\subparagraph}{
    \@ifstar
      \xxxSubParagraphStar
      \xxxSubParagraphNoStar
  }
  \newcommand{\xxxSubParagraphStar}[1]{\oldsubparagraph*{#1}\mbox{}}
  \newcommand{\xxxSubParagraphNoStar}[1]{\oldsubparagraph{#1}\mbox{}}
\fi
\makeatother


\usepackage{longtable,booktabs,array}
\usepackage{calc} % for calculating minipage widths
% Correct order of tables after \paragraph or \subparagraph
\usepackage{etoolbox}
\makeatletter
\patchcmd\longtable{\par}{\if@noskipsec\mbox{}\fi\par}{}{}
\makeatother
% Allow footnotes in longtable head/foot
\IfFileExists{footnotehyper.sty}{\usepackage{footnotehyper}}{\usepackage{footnote}}
\makesavenoteenv{longtable}
\usepackage{graphicx}
\makeatletter
\newsavebox\pandoc@box
\newcommand*\pandocbounded[1]{% scales image to fit in text height/width
  \sbox\pandoc@box{#1}%
  \Gscale@div\@tempa{\textheight}{\dimexpr\ht\pandoc@box+\dp\pandoc@box\relax}%
  \Gscale@div\@tempb{\linewidth}{\wd\pandoc@box}%
  \ifdim\@tempb\p@<\@tempa\p@\let\@tempa\@tempb\fi% select the smaller of both
  \ifdim\@tempa\p@<\p@\scalebox{\@tempa}{\usebox\pandoc@box}%
  \else\usebox{\pandoc@box}%
  \fi%
}
% Set default figure placement to htbp
\def\fps@figure{htbp}
\makeatother


% definitions for citeproc citations
\NewDocumentCommand\citeproctext{}{}
\NewDocumentCommand\citeproc{mm}{%
  \begingroup\def\citeproctext{#2}\cite{#1}\endgroup}
\makeatletter
 % allow citations to break across lines
 \let\@cite@ofmt\@firstofone
 % avoid brackets around text for \cite:
 \def\@biblabel#1{}
 \def\@cite#1#2{{#1\if@tempswa , #2\fi}}
\makeatother
\newlength{\cslhangindent}
\setlength{\cslhangindent}{1.5em}
\newlength{\csllabelwidth}
\setlength{\csllabelwidth}{3em}
\newenvironment{CSLReferences}[2] % #1 hanging-indent, #2 entry-spacing
 {\begin{list}{}{%
  \setlength{\itemindent}{0pt}
  \setlength{\leftmargin}{0pt}
  \setlength{\parsep}{0pt}
  % turn on hanging indent if param 1 is 1
  \ifodd #1
   \setlength{\leftmargin}{\cslhangindent}
   \setlength{\itemindent}{-1\cslhangindent}
  \fi
  % set entry spacing
  \setlength{\itemsep}{#2\baselineskip}}}
 {\end{list}}
\usepackage{calc}
\newcommand{\CSLBlock}[1]{\hfill\break\parbox[t]{\linewidth}{\strut\ignorespaces#1\strut}}
\newcommand{\CSLLeftMargin}[1]{\parbox[t]{\csllabelwidth}{\strut#1\strut}}
\newcommand{\CSLRightInline}[1]{\parbox[t]{\linewidth - \csllabelwidth}{\strut#1\strut}}
\newcommand{\CSLIndent}[1]{\hspace{\cslhangindent}#1}



\setlength{\emergencystretch}{3em} % prevent overfull lines

\providecommand{\tightlist}{%
  \setlength{\itemsep}{0pt}\setlength{\parskip}{0pt}}



 


\setlength\heavyrulewidth{0ex}
\setlength\lightrulewidth{0ex}
\usepackage[automark]{scrlayer-scrpage}
\clearpairofpagestyles
\cehead{
  Brian Weatherson
  }
\cohead{
  Scepticism, Rationalism, and Externalism
  }
\ohead{\bfseries \pagemark}
\cfoot{}
\makeatletter
\newcommand*\NoIndentAfterEnv[1]{%
  \AfterEndEnvironment{#1}{\par\@afterindentfalse\@afterheading}}
\makeatother
\NoIndentAfterEnv{itemize}
\NoIndentAfterEnv{enumerate}
\NoIndentAfterEnv{description}
\NoIndentAfterEnv{quote}
\NoIndentAfterEnv{equation}
\NoIndentAfterEnv{longtable}
\NoIndentAfterEnv{abstract}
\renewenvironment{abstract}
 {\vspace{-1.25cm}
 \quotation\small\noindent\emph{Abstract}:}
 {\endquotation}
\newfontfamily\tfont{EB Garamond}
\addtokomafont{disposition}{\rmfamily}
\addtokomafont{title}{\normalfont\itshape}
\let\footnoterule\relax

\makeatletter
\renewcommand{\@maketitle}{%
  \newpage
  \null
  \vskip 2em%
  \begin{center}%
  \let \footnote \thanks
    {\itshape\huge\@title \par}%
    \vskip 0.5em%  % Reduced from default
    {\large
      \lineskip 0.3em%  % Reduced from default 0.5em
      \begin{tabular}[t]{c}%
        \@author
      \end{tabular}\par}%
    \vskip 0.5em%  % Reduced from default
    {\large \@date}%
  \end{center}%
  \par
  }
\makeatother
\RequirePackage{lettrine}

\renewenvironment{abstract}
 {\quotation\small\noindent\emph{Abstract}:}
 {\endquotation\vspace{-0.02cm}}

\setmainfont{EB Garamond Math}[
  BoldFont = {EB Garamond SemiBold},
  ItalicFont = {EB Garamond Italic},
  RawFeature = {+smcp},
]

\newfontfamily\scfont{EB Garamond Regular}[RawFeature=+smcp]
\renewcommand{\textsc}[1]{{\scfont #1}}

\renewcommand{\LettrineTextFont}{\scfont}
\KOMAoption{captions}{tableheading}
\makeatletter
\@ifpackageloaded{caption}{}{\usepackage{caption}}
\AtBeginDocument{%
\ifdefined\contentsname
  \renewcommand*\contentsname{Table of contents}
\else
  \newcommand\contentsname{Table of contents}
\fi
\ifdefined\listfigurename
  \renewcommand*\listfigurename{List of Figures}
\else
  \newcommand\listfigurename{List of Figures}
\fi
\ifdefined\listtablename
  \renewcommand*\listtablename{List of Tables}
\else
  \newcommand\listtablename{List of Tables}
\fi
\ifdefined\figurename
  \renewcommand*\figurename{Figure}
\else
  \newcommand\figurename{Figure}
\fi
\ifdefined\tablename
  \renewcommand*\tablename{Table}
\else
  \newcommand\tablename{Table}
\fi
}
\@ifpackageloaded{float}{}{\usepackage{float}}
\floatstyle{ruled}
\@ifundefined{c@chapter}{\newfloat{codelisting}{h}{lop}}{\newfloat{codelisting}{h}{lop}[chapter]}
\floatname{codelisting}{Listing}
\newcommand*\listoflistings{\listof{codelisting}{List of Listings}}
\makeatother
\makeatletter
\makeatother
\makeatletter
\@ifpackageloaded{caption}{}{\usepackage{caption}}
\@ifpackageloaded{subcaption}{}{\usepackage{subcaption}}
\makeatother
\usepackage{bookmark}
\IfFileExists{xurl.sty}{\usepackage{xurl}}{} % add URL line breaks if available
\urlstyle{same}
\hypersetup{
  pdftitle={Scepticism, Rationalism, and Externalism},
  pdfauthor={Brian Weatherson},
  hidelinks,
  pdfcreator={LaTeX via pandoc}}


\title{Scepticism, Rationalism, and Externalism}
\author{Brian Weatherson}
\date{2006}
\begin{document}
\maketitle
\begin{abstract}
I argue that we have to accept one of the three isms in the title.
Either inductive scepticism is true, or we have substantial contingent a
priori knowledge, or a strongly externalist theory of knowledge is
correct.
\end{abstract}


\setstretch{1.1}
\lettrine{T}{his paper} is about three of the most prominent debates in
modern epistemology. The conclusion is that three \emph{prima facie}
appealing positions in these debates cannot be held simultaneously.

The first debate is \textbf{scepticism vs anti-scepticism}. My
conclusions apply to \emph{most} kinds of debates between sceptics and
their opponents, but I will focus on the inductive sceptic, who claims
we cannot come to know what will happen in the future by induction. This
is a fairly weak kind of scepticism, and I suspect many philosophers who
are generally anti-sceptical are attracted by this kind of scepticism.
Still, even this kind of scepticism is quite unintuitive. I'm pretty
sure I know (1) on the basis of induction.

\begin{description}
\tightlist
\item[(1)]
It will snow in Ithaca next winter.
\end{description}

Although I am taking a very strong version of anti-scepticism to be
intuitively true here, the points I make will generalise to most other
versions of scepticism. (Focussing on the inductive sceptic avoids some
potential complications that I will note as they arise.)

The second debate is a version of \textbf{rationalism vs empiricism}.
The kind of rationalist I have in mind accepts that some deeply
contingent propositions can be known a priori, and the empiricist I have
in mind denies this. Kripke showed that there are \emph{contingent}
propositions that can be known a priori. One example is \emph{Water is
the watery stuff of our acquaintance}. (`Watery' is David Chalmers's
nice term for the properties of water by which folk identify it.) All
the examples Kripke gave are of propositions that are, to use Gareth
Evans's term, deeply necessary (\citeproc{ref-Evans1979}{Evans 1979}).
It is a matter of controversy presently just how to analyse Evans's
concepts of deep necessity and contingency, but most of the
controversies are over details that are not important right here. I'll
simply adopt Stephen Yablo's recent suggestion: a proposition is deeply
contingent if it could have \emph{turned out} to be true, and could have
\emph{turned out} to be false (\citeproc{ref-Yablo2002}{Yablo
2002})\footnote{If you prefer the `two-dimensional' way of talking, a
  deeply contingent proposition is one that is true in some possible
  world `considered as actual'. See Chalmers
  (\citeproc{ref-Chalmers2006}{2006}) for a thorough discussion of ways
  to interpret this phrase, and the broader notion of so-called `deep'
  contingency. Nothing that goes on here will turn on any of the fine
  distinctions made in that debate - the relevant propositions will be
  deeply contingent in every plausible sense.}. Kripke did not provide
examples of any \emph{deeply} contingent propositions knowable a priori,
though nothing he showed rules out their existence.

The final debate is a version of \textbf{internalism vs externalism}
about epistemic justification. The internalist I have in mind endorses a
very weak kind of access internalism. Say that a class of properties
(intuitively, a determinable) is \emph{introspective} iff any beliefs an
agent has about which property in the class (which determinate) she
instantiates are guaranteed to not be too badly mistaken.\footnote{That
  a property is introspective does not mean that whenever a subject
  instantiates it she is in a position to form a not too badly mistaken
  belief about it. Even if the subject instantiates the property she may
  not possess sufficient concepts in order to have beliefs about it. And
  even if she has the concept she may simply have more pressing
  cognitive needs than forming certain kinds of belief. Many agents have
  no beliefs about the smell in their ordinary environment much of the
  time, for example, and this does not show that phenomenal smell
  properties are not introspective. All that is required is that if she
  has any beliefs at all about which determinate she instantiates, the
  beliefs are immune to massive error.} (Since `too badly' is vague,
`introspective' will be vague too, but as we'll see this won't matter to
the main argument.) My internalist believes the following two claims:

\begin{itemize}
\tightlist
\item
  Which propositions an agent can justifiably believe supervenes in
  which introspective properties she instantiates, and this is knowable
  a priori.\footnote{There is a delicate ambiguity in this expression to
    which a referee drew my attention. The intended meaning is that for
    any two agents who instantiate the same introspective properties,
    belief in the same propositions is justified. What's not intended is
    that if there's an agent who justifiably believes \emph{p}, and the
    introspective properties they instantiate are
    \emph{F}\textsubscript{1}, \ldots, \emph{F\textsubscript{n}}, then
    any agent who instantiates \emph{F}\textsubscript{1}, \ldots,
    \emph{F\textsubscript{n}} is justified in believing \emph{p}. For
    there might be some other introspective property
    \emph{F\textsubscript{n}}\textsubscript{+1} they instantiate that
    justifies belief in \emph{q}, and \emph{q} might be a defeater for
    \emph{p}. The `unintended' claim would be a very strong, and very
    implausible, claim about the subvenient basis for justification.}
\item
  There exist some introspective properties and some deeply contingent
  propositions about the future such that it's a priori that whoever
  instantiates those properties can justifiably believe those
  propositions.
\end{itemize}

My externalist denies one or other of these claims. Typically, she holds
that no matter what introspective properties you have, unless some
external condition is satisfied (such as the reliability of the
connection between instantiating those properties and the world being
the way you believe it is) you lack justification. Alternatively, she
holds that the connection between introspective properties and
justification is always a posteriori. (Or, of course, she might deny
both.)

My argument will be that the combination of anti-scepticism, empiricism
and internalism is untenable. Since there's quite a bit to be said for
each of these claims individually, that their combination is untenable
means we are stuck with a fairly hard choice: accept scepticism, or
rationalism, or externalism. Of the three, it \emph{may} seem that
externalism is the best, but given how weak the version of internalism
is that I'm using, I think we should take the rationalist option
seriously.\footnote{Rationalism is supported by BonJour
  (\citeproc{ref-BonJour1997}{1997}) and Hawthorne
  (\citeproc{ref-Hawthorne2002}{2002}), and my argument owes a lot to
  each of their discussions.} In this paper I'll just argue against the
combination of anti-scepticism, empiricism and internalism, and leave it
to the reader to judge which of the three to reject.

Very roughly, the argument for the trilemma will be as follows. There
are some propositions \emph{q} such that these three claims are
true.\footnote{Aesthetically it would be preferable to have the
  antecedent of (3) be just that empiricism is true, but unfortunately
  this does not seem to be possible.}

\begin{description}
\tightlist
\item[(2)]
If anti-scepticism is true, then I either know \emph{q} a priori or a
posteriori.
\item[(3)]
If internalism and empiricism are true, I do not know \emph{q} a priori.
\item[(4)]
If internalism is true, I do not know \emph{q} a posteriori.
\end{description}

Much of the paper will be spent giving us the resources to find, and
state, such a \emph{q}, but to a first approximation, think of \emph{q}
as being a proposition like \emph{I am not a brain-in-a-vat whose
experiences are as if they were a normal person}.\footnote{I.e. I am not
  a brain-in-a-vat* in the sense of Cohen
  (\citeproc{ref-Cohen1998}{1998}).} The important features of \emph{q}
are that (a) it is entailed by propositions we take ourselves to know,
(b) it is possibly false and (c) if something is evidence for it, then
any evidence is evidence for it. I will claim that by looking at
propositions like this, propositions that say in effect that I am not
being misled in a certain way, it is possible to find a value for
\emph{q} such that (2), (3) and (4) are all true. From that it follows
that

For most of the paper I will assume that internalism and anti-scepticism
are true, and use those hypotheses to derive rationalism. The paper will
conclude with a detailed look at the role internalism plays in the
argument, and this will give us some sense of what an anti-sceptical
empiricist externalism may look like.

\section{A Sceptical Argument}\label{a-sceptical-argument}

Among the many things I know about future, one of the firmest is (1).

\begin{description}
\tightlist
\item[(2)]
It will snow in Ithaca next winter.
\end{description}

I know this on the basis of inductive evidence about the length of
meteorological cycles and the recent history of Ithaca in winter. The
inductive sceptic now raises the spectre of Winter Wonderland, a kind of
world that usually has the same meteorological cycles as ours, and has
the same history, but in which it is sunny every day in Ithaca next
winter.\footnote{If she is convinced that there is no possible world
  with the \emph{same} history as ours and no snow in Ithaca next
  winter, the sceptic will change her story so Winter Wonderland's past
  differs imperceptibly from the past in our world. She doesn't think
  this issue is particularly relevant to the \emph{epistemological}
  debate, no matter how interesting the scientific and metaphysical
  issues may be, and I agree with her.} She says that to know (1) we
must know that (5) is false, and we do not.

\begin{description}
\tightlist
\item[(5)]
I am living in Winter Wonderland.
\end{description}

Just how does reflection (5) affect my confidence that I know (1)? The
sceptic might just appeal to the intuition that I don't know that (5) is
false. But I don't think I have that intuition, and if I do it is much
weaker than my intuition that I know (1) and that I can infer (5) from
(1). James Pryor (\citeproc{ref-Pryor2000}{2000, 527--29}) has suggested
the sceptic is better off using (5) in the following interesting
argument.\footnote{Pryor is discussing the external world sceptic, not
  the inductive sceptic, so the premises here are a little different to
  those he provides.}

\begin{description}
\tightlist
\item[(6)]
Either you don't know you're not living in Winter Wonderland; or, if you
do know that, it's because that knowledge rests in part on your
inductive knowledge that it will snow in Ithaca next winter.
\item[(7)]
If you're to know (1) on the basis of certain experiences or grounds
\emph{e}, then for every \emph{q} which is ``bad'' relative to \emph{e}
and (1), you have to be in a position to know \emph{q} to be false in a
non-question-begging way---i.e., you have to be in a position to know
\emph{q} to be false antecedently to knowing that it will snow next
winter on the basis of \emph{e}.
\item[(8)]
(5) is ``bad'' relative to any course of experience \emph{e} and (1).
\item[C]
You can't know (1), that it will snow next winter on the basis of your
current experiences.
\end{description}

An alternative hypothesis \emph{q} is ``bad'' in the sense used here iff
(to quote Pryor) ``it has the special features that characterise the
sceptic's scenarios---whatever those features turn out to be.'' (527) To
a first approximation, \emph{q} is bad relative to \emph{p} and \emph{e}
iff you're meant to be able to know \emph{p} on the basis of \emph{e},
but \emph{q} is apparently compatible with \emph{e}, even though it is
not compatible with \emph{p}.

Pryor argues that the best response to the external world sceptic is
\textbf{dogmatism}. On this theory you can know \emph{p} on the basis of
\emph{e} even though you have no prior reason to rule out alternatives
to \emph{p} compatible with \emph{e}. Pryor only defends the dogmatic
response to the external world sceptic, but it's worth considering the
dogmatist response to inductive scepticism. According to this response,
I \emph{can} come to know I'm not in Winter Wonderland on the basis of
my experiences to date, even though I didn't know this a priori. So
dogmatism is a version of empiricism, and it endorses (6).\footnote{It
  is a version of the kind of internalism discussed in footnote 2, since
  according to the dogmatist seeming to see that \emph{p} can be
  sufficient justification for belief in \emph{p}. Pryor's preferred
  version of dogmatism is also internalist in the slightly stronger
  sense described in the text, but it seems possible that one could be a
  dogmatist without accepting that internalist thesis. One could accept,
  for instance, that seeming to see that \emph{p} justifies a belief
  that \emph{p}, but also think that seeming to see that \emph{q}
  justifies a belief that \emph{p} iff there is a known reliable
  connection between \emph{q} and \emph{p}. As I said, even the weaker
  version of internalism is sufficient to generate a conflict with
  anti-scepticism and empiricism, provided we just focus on the
  propositions that can be justifiably believed on the basis of
  introspective properties.} The false premise in this argument,
according to the dogmatist, is (7). We can know it will snow even though
the Winter Wonderland hypothesis is bad relative to this conclusion and
our actual evidence, and we have no prior way to exclude it.

Pryor notes that the sceptic could offer a similar argument concerning
justification, and the dogmatist offers a similar response.

\begin{description}
\tightlist
\item[(9)]
Either you're not justified in believing that you're not in Winter
Wonderland; or, if you are justified in believing this, it's because
that justification rests in part on your justified belief that it will
snow in Ithaca next winter.
\item[(10)]
If you're to have justification for believing (1) on the basis of
certain experiences or grounds \emph{e}, then for every \emph{q} which
is ``bad'' relative to \emph{e} and (1), you have to have antecedent
justification for believing \emph{q} to be false---justification which
doesn't rest on or presuppose any \emph{e}-based justification you may
have for believing (1).
\item[(11)]
(5) is ``bad'' relative to any course of experience \emph{e} you could
have and (1).
\item[C]
You can't justifiably believe it will snow in Ithaca next winter on the
basis of past experiences.
\end{description}

The dogmatist rejects (10), just as she rejects (7). I shall spend most
of my time in the next two sections arguing for (10), returning to (7)
only at the end. For it seems there are compelling reasons to accept
(10), and hold that the problem with this argument is either with (9) or
(11).\footnote{Just which is wrong then? That depends on how ``bad'' is
  defined. On our final definition (8) will fail, but there are other
  sceptical arguments, using other sceptical hypotheses, on which (6)
  fails.}

\section{Dominance Arguments}\label{dominance-arguments}

The primary argument for (10) will turn on a dominance principle: if you
will be in a position to justifiably believe \emph{p} whatever evidence
you get, and you know this, then you are now justified in believing
\emph{p}. This kind of reasoning is perfectly familiar in decision
theory: if you know that one of \emph{n} states obtains, and you know
that in each of those states you should do X rather than Y, then you
know now (or at least you should know) that you should do X rather than
Y. This is a very plausible principle, and equivalent epistemic
principles are just as viable. Dominance reasoning can directly support
(10) and hence indirectly support (7). (As Vann McGee
(\citeproc{ref-McGee1999}{1999}) showed, the dominance principle in
decision theory has to be qualified for certain kinds of agents with
unbounded utility functions who are faced with a decision tree with
infinitely many branches. Such qualifications do not seem at all
relevant here.)

It will be useful to start with an unsound argument for (10), because
although this argument is unsound, it fails in an instructive way.
Before I can present the argument I need to make an attempt at
formalising Pryor's concept of badness.

\begin{quote}
\emph{q} is \textbf{bad} relative to \emph{e} and \emph{p}
=\textsubscript{df} \emph{q} is deeply contingent, you know \emph{p}
entails ¬\emph{q}, and for any possible evidence \emph{e}′ (that you
could have had at the time your total evidence is actually \emph{e})
there exists a \emph{p}′ such that you know \emph{p}′ entails ¬\emph{q}
and you are justified in believing \emph{p}′ on the basis of \emph{e}′
if \emph{e}′ is your total evidence.
\end{quote}

Roughly, the idea is that a bad proposition is one that would be
justifiably ruled out by any evidence, despite the fact that it could
turn out to be true.\footnote{Note that there's a subtle shift here in
  our conception of badness. Previously we said that bad propositions
  are those you allegedly know on the basis of your actual evidence (if
  you know \emph{p}) even though they are logically consistent with that
  evidence. Now we say that they are propositions you could rule out on
  \emph{any} evidence, even though they are consistent with your actual
  total evidence. This is a somewhat narrower class of proposition, but
  focussing on it strengthens the sceptic's case
  appreciably.{[}\^{}12{]}} Using this definition we can present an
argument for rationalism. The argument will use some fairly general
premises connecting justification, evidence and badness. If we were just
interested in this case we could replace \emph{q} with (5), \emph{r}
with the proposition that (5) is false, \emph{e} with my current
evidence, and \emph{e}′ with some evidence that would undermine my
belief that (5) is false, if such evidence could exist. The intuitions
behind the argument may be clearer if you make those substitutions when
reading through the argument. But because the premises are interesting
beyond their application to this case, I will present the argument in
its more general form.

\begin{description}
\tightlist
\item[(12)]
If you are justified in believing (1) on the basis of \emph{e}, and you
know (1) entails ¬(5), then you are justified in believing ¬(5) when
your evidence is \emph{e}.
\item[(13)]
If you are justified in believing \emph{r} (at time \emph{t}) on the
basis of \emph{e}, then there is some other possible evidence \emph{e}′
(that you could have at \emph{t}) such that you would not be justified
in believing \emph{r} were your total evidence \emph{e}′.
\item[(14)]
If you are justified in believing \emph{r}, and there is no evidence
\emph{e} such that \emph{e} is part of your evidence and you are
justified in believing \emph{r} on the basis of \emph{e}, then you are
justified in believing \emph{r} a priori.
\item[(15)]
By definition, \emph{q} is \textbf{bad} relative to \emph{e} and
\emph{p} iff \emph{q} is deeply contingent, you know \emph{p} entails
¬\emph{q}, and for any possible evidence \emph{e}′ (that you could have
when your evidence is \emph{e}) there exists a \emph{p}′ such that you
know \emph{p}′ entails ¬\emph{q} and you are justified in believing
\emph{p}′ on the basis of \emph{e}′ if \emph{e}′ is your total evidence.
\item[(16)]
So, if \emph{q} is bad relative to \emph{e} and (1), and you are
justified in believing (1) on the basis of \emph{e}, then you are
justified in believing ¬\emph{q} a priori.
\end{description}

(The references to times in (13) and (15) is just to emphasise that we
are talking about your current evidence, and ways it could be. That you
could observe Winter Wonderland next winter doesn't count as a relevant
alternative kind of evidence \emph{now}.)

Our conclusion (16) entails (10), since (10) merely required that for
every bad proposition relative to \emph{e} and (1), you have
`antecedent' justification for believing that proposition to be false,
while (16) says this justification is a priori. (`Antecedent'
justification need not be a priori as long as it arrives before the
particular evidence you have for (1). This is why (16) is strictly
stronger than (10).) So if (10) is false then one of these premises must
be false. I take (15) to define ``bad'', so it cannot be false. Note
that given this definition we cannot be certain that (5) is bad. We will
return to this point a few times.

Which premise should the dogmatist reject? (12) states a fairly mundane
closure principle for justified belief. And (13) follows almost
automatically from the notion of `basing'. A belief can hardly be based
in some particular evidence if any other evidence would support it just
as well. This does not mean that such a belief cannot be rationally
\emph{caused} by the particular evidence that you have, just that the
evidence cannot be the rational \emph{basis} for that belief. The
dogmatist objects to (14). There is a prima facie argument for (14), but
as soon as we set it out we see why the dogmatist is correct to stop us
here.

Consider the following argument for (14), which does little more than
lay out the intuition (14) is trying to express. Assume \emph{r} is such
that for any possible evidence \emph{e}, one would be justified in
believing \emph{r} with that evidence. Here's a way to reason a priori
to \emph{r}. Whatever evidence I get, I will be justified in believing
that \emph{q}. So I'm now justified in believing that \emph{r}, before I
get the evidence. Compare a simple decision problem where there is one
unknown variable, and it can one of two values, but whichever value it
takes it is better for one to choose X rather than Y. That is sufficient
to make it true now that one should choose X rather than Y. Put this
way, the argument for (14) is just a familiar dominance argument.

Two flaws with this argument for (14) stand out, each of them arising
because of disanalogies with the decision theoretic case.

First, when we apply dominance reasoning in decision theory, we look at
cases where it would be better to take X rather than Y in every possible
case, \emph{and this is known}. This point is usually not stressed,
because it's usually just assumed in decision theory problems that the
players know the consequences of their actions given the value of
certain unknown variables. It's not obviously a good idea to assume this
without comment in applications of decision theory, and it's clearly a
bad idea to make the same kind of assumption in epistemology. Nothing in
the antecedent of (14) specifies that we can know, let alone know a
priori, that if our evidence is \emph{e} then we are justified in
believing \emph{r}. Even if this is true, even if it is necessarily
true, it may not be knowable.

Second, in the decision theory case we presupposed it is known that the
variable can take only one of two values. Again, there in nothing in the
antecedent of (14) to guarantee the parallel. Even if an agent knows of
every possible piece of evidence that if she gets that evidence she will
be justified in believing \emph{r}, she may not be in a position to
justifiably conclude \emph{r} now because she may not know that these
are all the possible pieces of evidence. In other words, she can only
use dominance reasoning to conclude \emph{r} if she knows \emph{de
dicto}, and not merely \emph{de re}, of every possible body of evidence
that it justifies \emph{r}.

So the quick argument for (14) fails. Still, it only failed because (14)
left out two qualifications. If we include those qualifications, and
adjust the other premises to preserve validity, the argument will work.
To make this adjustment, we need a new definition of badness.

\begin{quote}
\emph{q} is \textbf{bad} relative to \emph{e} and \emph{p}
=\textsubscript{df}

\begin{enumerate}
\def\labelenumi{\arabic{enumi}.}
\tightlist
\item
  \emph{q} is deeply contingent;
\item
  \emph{p} is known to entail ¬\emph{q}; and
\item
  it is knowable a priori that for any possible evidence \emph{e}′ there
  exists a \emph{p}′ such that \emph{p}′ is known to entail ¬\emph{q},
  and one is justified in believing \emph{p}′ on the basis of \emph{e}′.
\end{enumerate}
\end{quote}

The aim still is to find an argument for some claim stronger than (10)
in sceptical argument 2. If we can do that, and if as the sceptic
suggests (5) really is bad, then the only anti-sceptical response to
sceptical argument 2 will be rationalism. So the fact that this looks
like a sound argument for a slightly stronger conclusion than (10) is a
large step in our argument that anti-scepticism plus internalism entails
rationalism. (I omit the references to times from here on.)\footnote{Again,
  if you don't think \emph{I exist} should be a priori, the conclusion
  of (17) should be that \emph{I exist}~⊃~\emph{r} is a priori.}

\begin{description}
\tightlist
\item[(12)]
If you are justified in believing (1) on the basis of \emph{e}, and you
know (1) entails ¬(5), then you are justified in believing ¬(5) when
your evidence is \emph{e}.
\item[(13)]
If you are justified in believing \emph{r} on the basis of \emph{e},
then there is some other possible evidence \emph{e}′ such that you would
not be justified in believing \emph{r} were your total evidence
\emph{e}′.
\item[(17)]
If you know you are justified in believing \emph{r}, and you know a
priori that there is no evidence \emph{e} you have such that you are
justified in believing \emph{r} on the basis of \emph{e}, then you are
justified in believing \emph{r} a priori.
\item[(18)]
By definition, \emph{q} is \textbf{bad} relative to \emph{e} and
\emph{p} iff \emph{q} is deeply contingent, \emph{p} is known to entail
¬\emph{q}, and it is knowable a priori that for any possible evidence
\emph{e}′ there exists a \emph{p}′ such that \emph{p}′ is known to
entail ¬\emph{q}, and one is justified in believing \emph{p}′ on the
basis of \emph{e}′.
\item[(19)]
So, if \emph{q} is bad relative to \emph{e} and (1), and you are
justified in believing (1) on the basis of \emph{e}, then you are
justified in believing ¬\emph{q} a priori.
\end{description}

This is a sound argument for (19), and hence for (10), but as noted on
this definition of ``bad'' (11) may be false. If the Winter Wonderland
hypothesis is to be bad it must be a priori knowable that on any
evidence whatsoever, you'd be justified in believing it to be false. But
as we will now see, although no evidence could justify you in believing
the Winter Wonderland hypothesis to be true, it is not at all obvious
that you are always justified in believing it is false.

\section{Hunting the Bad Proposition}\label{hunting-the-bad-proposition}

A proposition is bad if it is deeply contingent but if you could
justifiably believe it to be false on the basis of your current
evidence, you could justifiably believe it to be false a priori. If a
bad proposition exists, then we are forced to choose between rationalism
and scepticism. To the extent that rationalism is unattractive,
scepticism starts to look attractive. I think Pryor is right that this
kind of argument tacitly underlies many sceptical arguments. The
importance of propositions like (5) is not that it's too hard to know
them to be false. The arguments of those who deny closure principles for
knowledge notwithstanding, it's very intuitive that it's \emph{easier}
to know (5) is false than to know (1) is true. So why does reflection on
(5) provide more comfort to the inductive sceptic than reflection on
(1)? The contextualist has one answer, that thinking about (5) moves the
context to one where sceptical doubts are salient. Pryor's work suggests
a more subtle answer. Reflecting on (5) causes us to think about
\emph{how} we could come to know it is false, and prima facie it might
seem we could not know that a priori or a posteriori. It's that dilemma,
and not the mere salience of the Winter Wonderland possibility, that
drives the best sceptical argument. But this argument assumes that (5)
could not be known to be false on the basis of empirical evidence,
i.e.~that it is bad. If it is not bad, and nor is any similar
proposition, then we can easily deflect the sceptical argument. However,
if we assume internalism, we can \emph{construct} a bad proposition.

The prima facie case that (5) is bad (relative to (1) and our current
evidence \emph{e} -- I omit these relativisations from now on) looks
strong. The negation of (5) is (20), where \emph{H} is a proposition
that summarises the relevant parts of the history of the
world.\footnote{I assume \emph{H} includes a `that's all that's relevant
  clause' to rule out defeaters. That is, it summaries the relevant
  history of the world \emph{as such}.}

\begin{description}
\tightlist
\item[(20)]
Either ¬\emph{H} or it will snow in Ithaca next winter.
\end{description}

Now one may argue that (5) is bad as follows. Either our evidence
justifies believing ¬\emph{H} or it doesn't. If it does, then it clearly
justifies believing (20), for ¬\emph{H} trivially entails it. If it does
not, then we are justified in believing \emph{H}, and whenever we are
justified believing the world's history is \emph{H}, we can inductively
infer that it will snow in Ithaca next winter. The problem with this
argument, however, is fairly clear: the step from the assumption that we
are not justified in believing ¬\emph{H} to the conclusion we are
justified in believing \emph{H} is a modal fallacy. We might be
justified in believing neither \emph{H} nor its negation. In such a
situation, it's not obvious we could justifiably infer (20). So (5) may
not be bad.

A suggestion John Hawthorne (\citeproc{ref-Hawthorne2002}{2002}) makes
seems to point to a proposition that is more plausibly bad. Hawthorne
argues that disjunctions like (21) are knowable a priori, and this
suggests that (22), its negation, is bad.

\begin{description}
\tightlist
\item[(21)]
Either my evidence is not \emph{e} or it will snow in Ithaca next
winter.
\item[(22)]
My evidence is \emph{e} and it will not snow in Ithaca next winter.
\end{description}

Hawthorne does not provide a dominance argument that (21) is knowable a
priori. Instead he makes a direct appeal to the idea that whatever kinds
of inference we can draw now the basis of our evidence we could have
drawn prior to getting \emph{e} as conditional conclusions, conditional
on getting \emph{e}. So if I can now know it will snow in Ithaca next
winter, prior to getting \emph{e} I cold have known the material
conditional \emph{If my evidence is e, it will snow in Ithaca}, which is
equivalent to (21). It's not clear this analogy works, since when we do
such hypothetical reasoning we take someone to \emph{know} that our
evidence is \emph{e}, and this may cause some complications. Could we
find a dominance argument to use instead? One might be tempted by the
following argument.

\begin{description}
\tightlist
\item[(23)]
I know a priori that if my evidence is \emph{e}, then I am justified in
believing the second disjunct of (21).
\item[(24)]
I know a priori that if my evidence is not \emph{e}, then I am justified
in believing the first disjunct of (21)
\item[(25)]
I know a priori that if I am justified in believing a disjunct of (21) I
am justified in believing the disjunction (21).
\item[(26)]
I know a priori that my evidence is either \emph{e} or not \emph{e}.
\item[(27)]
So, I'm justified a priori in believing (21).
\end{description}

The problem here is the second premise, (24). It's true that if my
evidence is not \emph{e} then the first disjunct of (21) is true. But
there's no reason to suppose I am justified in believing any true
proposition about my evidence. Timothy Williamson
(\citeproc{ref-Williamson2000-WILKAI}{2000} ch.~8) has argued that the
problem with many sceptical arguments is that they assume agents know
what their evidence is. I doubt that's really the flaw in sceptical
arguments, but it certainly is the flaw in the argument that (22) is
bad.

The problem with using (22) is that the argument for its badness relied
on quite a strong privileged access thesis: whenever my evidence is not
\emph{e} I am justified in believing it is not. If we can find a weaker
privileged access thesis that is true, we will be able to find a
proposition similar to (22) that is bad. And the very argument
Williamson gives against the thesis that we always know what our
evidence is will show us how to find such a thesis.

Williamson proposes a margin-of-error model for certain kinds of
knowledge. On this model, X knows that \emph{p} iff (roughly) \emph{p}
is true in all situations within X's margin-of-error.\footnote{There's a
  considerable amount of idealisation here. What's really true is that X
  is in a position to know anything true in all situations within her
  margin-of-error. Since we're working out what is a priori knowable,
  I'll assume agents are idealised so they know what they are in a
  position to know. This avoids needless complications we get from
  multiplying the modalities that are in play.} The intuitive idea is
that all of the possibilities are arranged in some metric space, with
the distance between any two worlds being the measure of their
similarity with respect to X. Then X knows all the things that are true
in all worlds within some sphere centred on the actual world, where the
radius of that sphere is given by how accurate she is at forming
beliefs.

One might think this would lead to the principle B:
\emph{p}~→~K¬K¬\emph{p}, that is, if \emph{p} is true then X knows that
she does not know ¬\emph{p}. Or, slightly more colloquially, if \emph{p}
is true then X knows that for all she knows \emph{p} is true. (I use K
here as a modal operator. K\emph{A} means that X, the salient subject,
knows that \emph{A}.) On a margin-of-error model \emph{p}~→~K¬K¬\emph{p}
is false only if \emph{p} is actually true and there is a nearby
(i.e.~within the margin-of-error) situation where the agent knows
¬\emph{p}. But if \emph{nearby} is symmetric this is impossible, because
the truth of \emph{p} in this situation will rule out the knowability of
¬\emph{p} in that situation.

As Williamson points out, that quick argument is fallacious, since it
relies on a too simplistic margin-of-error model. He proposes a more
complicated account: \emph{p} is known at \emph{s} iff there is a
distance \emph{d} greater than the margin-of-error and for any situation
\emph{s}′ such that the distance between \emph{s} and \emph{s}′ is less
than \emph{d}, \emph{p} is true at \emph{s}′. Given this model, we
cannot infer \emph{p}~→~K¬K¬\emph{p}. Indeed, the only distinctive modal
principle we can conclude is K\emph{p}~→~\emph{p}. However, as Delia
Graff Fara (\citeproc{ref-Fara2002}{2002}) has shown, if we make certain
density assumptions on the space of available situations, we can recover
the principle (27) within this account.\footnote{If we translate K as □
  and ¬K¬ as ◇, (24) can be expressed as the modal formula
  \emph{p}~→~□◇◇\emph{p}.}

\begin{description}
\tightlist
\item[(27)]
\emph{p}~→~K¬KK¬\emph{p}
\end{description}

To express the density assumption, let
\emph{d}(\emph{s}\textsubscript{1}, \emph{s}\textsubscript{2}) be the
`distance' between \emph{s}\textsubscript{1} and
\emph{s}\textsubscript{2}, and \emph{m} the margin-of-error. The
assumption then is that there is a \emph{k} \textgreater{} 1 such that
for any \emph{s}\textsubscript{1}, \emph{s}\textsubscript{2} such that
\emph{d}(\emph{s}\textsubscript{1},~\emph{s}\textsubscript{2})~\textless~\emph{km},
there is an \emph{s}\textsubscript{3} such that
\emph{d}(\emph{s}\textsubscript{1},
\emph{s}\textsubscript{3})~\textless~\emph{m} and
\emph{d}(\emph{s}\textsubscript{3},
\emph{s}\textsubscript{2})~\textless~\emph{m}. And this will be made
true if there is some epistemic situation roughly `half-way' between
\emph{s}\textsubscript{1} and \emph{s}\textsubscript{2}.\footnote{Fara
  actually gives a slightly stronger principle than this, but this
  principle is sufficient for her purposes, and since it is weaker than
  Fara's, it is a little more plausible. But the underlying idea here,
  that we can get strong modal principles out of margin-of-error models
  by making plausible assumptions about density, is taken without
  amendment from her paper.} That is, all we have to assume to recover
(27) within the margin-of-error model is that the space of possible
epistemic situations is suitably dense. Since the margin-of-error model,
and Fara's density assumption, are both appropriate for introspective
knowledge, (27) is true when \emph{p} is a proposition about the agent's
own knowledge.

To build the bad proposition now, let \emph{G} be a quite general
property of evidence, one that is satisfied by everyone with a
reasonable acquaintance with Ithaca's weather patterns, but still
precise enough that it is a priori that everyone whose evidence is
\emph{G} is justified in believing it will snow in Ithaca next winter.
The internalist, remember, is committed to such a \emph{G} existing and
it being an introspective property. Now consider the following
proposition, which I shall argue is bad.\footnote{If you preferred the
  amended version of (11) discussed in footnote 12, the bad proposition
  is \emph{I don't exist or} (28) \emph{is true}.}

\begin{description}
\tightlist
\item[(28)]
I know that I know my evidence is \emph{G}, and it will not snow in
Ithaca next winter.
\end{description}

The negation of (28) is (29).

\begin{description}
\tightlist
\item[(29)]
It will snow in Ithaca next winter, or I don't know that I know my
evidence is \emph{G}.
\end{description}

It might be more intuitive to read (29) as the material conditional
(29a), though since English conditionals aren't material conditionals
this seems potentially misleading.

\begin{description}
\tightlist
\item[(29a)]
If I know that I know that my evidence is \emph{G}, then it will snow in
Ithaca next winter.
\end{description}

To avoid confusions due to the behaviour of conditionals, I'll focus on
the disjunction (29). Assume for now that the margin-of-error model is
appropriate for propositions about my own evidence. I will return below
to the plausibility of this assumption. This assumption implies that
principle (27) is always correct when \emph{p} is a proposition about my
evidence. Given this, we can prove (28) is bad. Note that all my
possible evidential states either are, or are not, \emph{G}. If they are
\emph{G} then by hypothesis I am justified in believing that it will
snow in Ithaca next winter and hence I am justified in believing (29).
If they are not, then by the principle (27) I know that I don't know
that I know my evidence is \emph{G}, so I can come to know (29), so I am
justified in believing (29). So either way I am justified in believing
(29). It's worth noting that at no point here did I assume that I knew
whether my evidence was \emph{G}, though I do assume that I know that
having evidence that is \emph{G} justifies belief in snow next winter.

All of this assumes the margin-of-error model looks appropriate for
introspective properties. If it isn't, then we can't assume that (27) is
true when \emph{p} is a proposition about the introspective properties I
satisfy, and hence the argument that (29) is knowable a priori fails.
There's one striking problem with assuming a priori that we can use the
margin-of-error model in all situations. It is assumed (roughly) that
anything that is true in all possibilities within a certain sphere with
the subject's beliefs at the centre is known. This sphere must include
the actual situation, or some propositions that are actually false may
be true throughout the sphere. Since for propositions concerning
non-introspective properties there is no limit to how badly wrong the
subject can be, we cannot set any limits a priori to the size of the
sphere. So a priori the only margin-of-error model we can safely use is
the sceptical model that says the subject knows that \emph{p} iff
\emph{p} is true in all situations. For introspective properties the
margin-of-error can be limited, because it is constitutive of
introspective properties that the speakers beliefs about whether they
possess these properties are not too far from actuality. So there seems
to be no problem with using Williamson's nice model as long as we
restrict our attention to introspective properties.

If belief in (29) can be justified a priori, and it is true, does that
mean it is knowable a priori? If we want to respect Gettier intuitions,
then we must not argue directly that since our belief in (29) is
justified, and it is true, then we know it. Still, being justified and
true is not irrelevant to being known. I assume here, far from
originally, that it is a reasonable \emph{presumption} that any
justified true belief is an item of knowledge. This presumption can be
defeated, if the belief is inferred from a false premise, or if the
justification would vanish should the subject acquire some evidence she
should have acquired, or if there is a very similar situation in which
the belief is false, but it is a reasonable presumption. Unless we
really are in some sceptical scenario, there is no ``defeater'' that
prevents our belief in (29) being an item of knowledge. We certainly did
not infer it from a false premise, there is no evidence we \emph{could}
get that would undermine it, and situations in which it is false are
very far from actuality.

Since there are no such defeaters, it is reasonable to infer we can
\emph{know} (29) a priori. The important premises grounding this
inference are an anti-sceptical premise, that we can know (1) on the
basis of our current evidence, and the internalist premise that we used
several times in the above argument. This completes the argument that
the combination of empiricism, internalism and anti-scepticism is
untenable.

\section{How Externalism Helps}\label{how-externalism-helps}

It should be obvious how the rationalist can respond to the above
argument - by simply accepting the conclusion. Ultimately I think that's
the best response to this argument. As Hawthorne notes, rationalism is
the natural position for fallibilists about knowledge to take, for it is
just the view that we can know something a priori even though we could
turn out to be wrong. In other words, it's just fallibilism about a
priori knowledge. Since fallibilism about a posteriori knowledge seems
true, and there's little reason to think fallibilism about the a priori
would be false if fallibilism about the a posteriori is true, the
rationalist's position is much stronger than many have
assumed.\footnote{As BonJour points out, rationalism has fallen into
  such disrepute that many authors leave it out even of surveys of the
  options. This seems unwarranted given the close connection between
  rationalism and the very plausible thesis of fallibilism.} The
inductive sceptic also has an easy response - reject the initial premise
that in my current situation I know that it will snow in Ithaca next
winter. There are other responses that deserve closer attention: first,
the inductive sceptic who is not a universal sceptic, and in particular
is not a sceptic about perception, and second the externalist.

I said at the start that the argument generalises to most kinds of
scepticism. One kind of theorist, the inductive sceptic who thinks we
can nonetheless acquire knowledge through perception, may think that the
argument does not touch the kind of anti-sceptical, internalist,
empiricist position she adopts. The kind of theorist I have in mind says
that the objects and facts we perceive are constitutive of the evidence
we receive. So given we are getting the evidence we are actually
getting, these objects must exist and those facts must be true. She says
that if I'd started with (30), instead of (1), my argument would have
ended up claiming that (31) is bad for some \emph{G}.

\begin{description}
\tightlist
\item[(30)]
A hand exists.
\item[(31)]
A hand exists, or I don't know that I know that I'm perceiving a hand.
\end{description}

She then says that (31) is not deeply contingent, since in any situation
where the first disjunct is false the second is true, so it cannot be
bad. This response is correct as far as it goes, but it does not go far
enough to deserve the name anti-sceptical. For it did not matter to the
above argument, or to this response that (1) is about the future. All
that mattered was that (1) was not \emph{entailed} by our evidence. So
had (1) been a proposition about the present that we cannot directly
perceive, such as that it is not snowing in Sydney \emph{right now}, the
rest of the argument would have been unaffected. The summary here is
that if one is suitably externalist about perception, so one thinks the
existence of perceptual states entail the existence of the things being
perceived, one can accept this argument, accept internalism, accept
empiricism, and not be an \emph{external world} sceptic. For it is
consistent with such a position that one know the existence of the
things one perceives. But on this picture one can know very little
beyond that, so for most practical purposes, the position is still a
sceptical one.

The externalist response is more interesting. Or, to be more precise,
the externalist reponse\emph{s} are more interesting. Although I have
appealed to internalism a couple of times in the above argument, it
might not be so clear how the externalist can respond. Indeed, it may be
worried that by exercising a little more care in various places I could
have shown that everyone must accept either rationalism or scepticism.
That is the conclusion Hawthorne derives in his paper on deeply
contingent a priori knowledge, though as noted above he uses somewhat
more contentious reasoning than I do in order to get there. To conclude,
I will argue that the internalism is crucial to the argument I have
presented, and I will spell out how the externalist can get out of the
trap I've set above.

One easy move that's available to an externalist is to deny that any
facts about justification are a priori. That blocks the move that says
we can find a \emph{G} such that it's a priori that anyone whose
evidence is \emph{G} can know that it will snow in Ithaca next year.
This is not an essential feature of externalism. One can be an
externalist about justification and still think it is a priori that if
one's evidence has the property \emph{is reliably correlated with snow
in the near future} then it justifies belief that it will shortly snow.
But the position that all facts about justification are a posteriori
fits well with a certain kind of naturalist attitude, and people with
that attitude will find it easy to block the sceptical argument I've
presented.

Can, however, we use an argument like mine to argue against an
anti-sceptic empiricist externalist who thinks some of the facts about
justification \emph{can} be discovered a priori? The strategy I've used
to build the argument is fairly transparent: find a disjunctive a priori
knowable proposition by partitioning the possible evidence states into a
small class, and adding a disjunct for every cell of the partition. In
every case, the disjunct that is added is one that is known to be known
given that evidence. If one of the items of knowledge is ampliative, if
it goes beyond the evidence, then it is possible the disjunction will be
deeply contingent. But the disjunction is known no matter what.

If internalism is true, then the partition can divide up evidential
states according to the introspective properties of the subject. If
externalism is true, then such a partition may not be that
\emph{useful}, because we cannot infer much about what the subject is
justified in believing from the introspective properties she
instantiates. Consider, for example, the above partition of subjects
into the \emph{G} and the not-\emph{G}, where \emph{G} is some
introspective property, intuitively one somewhat connected with it
snowing in Ithaca next year. The subjects that are not-\emph{G} know
that they don't know they know they are \emph{G}, because they aren't.
Externalists need not object to this stage of the argument. They can,
and should, accept that a margin-of-error model is appropriate for
introspective properties. Since it's part of the nature of introspective
properties that we can't be \emph{too} badly wrong about which ones we
instantiate, we're guaranteed to satisfy some reliability clause, so
there's no ground there to deny the privileged access principle I
defended above.

The problem is what to say about the cases where the subject is
\emph{G}. Externalists should say that some such subjects are justified
in believing it will snow in Ithaca next winter, and some are not. For
simplicity, I'll call the first group the reliable ones and the others
the unreliable ones. If I'm \emph{G} and reliable, then I'm justified in
believing it will snow, and hence in believing (29). But if I'm \emph{G}
and unreliable, then I'm not justified in believing this. Indeed, if I'm
\emph{G} and unreliable, there is no obvious argument that I'm justified
in believing \emph{either} of the disjuncts of (29). Since this is a
possible evidential state, externalists should think there is no
dominance argument that (29) is a priori knowable.

Could we solve this by adding another disjunct, one that is guaranteed
to be known if I'm \emph{G} and unreliable? There is no reason to
believe we could. If we're unreliable, there is no guarantee that we
will \emph{know} we are unreliable. Indeed, we may well believe we are
reliable. So there's no proposition we can add to our long disjunction
while saying to ourselves, ``In the case where the subject is \emph{G}
and unreliable, she can justifiably believe \emph{this} disjunct.'' If
the subject is unreliable, she may not have \emph{any} justified beliefs
about the external world. But this is just to say the above recipe for
constructing bad propositions breaks down. Externalists should have no
fear that anything like this approach could be used to construct a
proposition they should find bad. This is obviously not a positive
argument that anti-sceptical empiricist externalism is tenable, but it
does suggest that such a position is immune to the kind of argument I
have presented here.\footnote{This paper has been presented at Cornell
  University and the Inland Northwest Philosophy Conference, and each
  time I received valuable feedback. Thanks also to David Chalmers,
  Harold Hodes, Nicholas Sturgeon and, especially, Tamar Szabó Gendler
  for very helpful comments on various drafts of the paper.}

\subsection*{References}\label{references}
\addcontentsline{toc}{subsection}{References}

\phantomsection\label{refs}
\begin{CSLReferences}{1}{0}
\bibitem[\citeproctext]{ref-BonJour1997}
BonJour, Laurence. 1997. \emph{In Defense of Pure Reason}. Cambridge:
Cambridge University Press.

\bibitem[\citeproctext]{ref-Chalmers2006}
Chalmers, David. 2006. {``Foundations of Two-Dimensional Semantics.''}
In \emph{Two-Dimensional Semantics}, edited by Manuel Garcia-Carpintero
and Josep Macià, 55--140. Oxford: Oxford University Press.

\bibitem[\citeproctext]{ref-Cohen1998}
Cohen, Stewart. 1998. {``Two Kinds of Skeptical Argument.''}
\emph{Philosophy and Phenomenological Research} 58 (1): 143--59. doi:
\href{https://doi.org/10.2307/2653634}{10.2307/2653634}.

\bibitem[\citeproctext]{ref-Evans1979}
Evans, Gareth. 1979. {``Reference and Contingency.''} \emph{Monist} 62:
161--89.

\bibitem[\citeproctext]{ref-Fara2002}
Fara, Delia Graff. 2002. {``An Anti-Epistemicist Consequence of Margin
for Error Semantics for Knowledge.''} \emph{Philosophy and
Phenomenological Research} 64 (1): 127--42. doi:
\href{https://doi.org/10.1111/j.1933-1592.2002.tb00146.x}{10.1111/j.1933-1592.2002.tb00146.x}.
This paper was first published under the name {``Delia Graff.''}

\bibitem[\citeproctext]{ref-Hawthorne2002}
Hawthorne, John. 2002. {``Deeply Contingent a Priori Knowledge.''}
\emph{Philosophy and Phenomenological Research} 65 (2): 247--69. doi:
\href{https://doi.org/10.1111/j.1933-1592.2002.tb00201.x}{10.1111/j.1933-1592.2002.tb00201.x}.

\bibitem[\citeproctext]{ref-McGee1999}
McGee, Vann. 1999. {``An Airtight Dutch Book.''} \emph{Analysis} 59 (4):
257--65. doi:
\href{https://doi.org/10.1093/analys/59.4.257}{10.1093/analys/59.4.257}.

\bibitem[\citeproctext]{ref-Pryor2000}
Pryor, James. 2000. {``The Sceptic and the Dogmatist.''} \emph{No{û}s}
34 (4): 517--49. doi:
\href{https://doi.org/10.1111/0029-4624.00277}{10.1111/0029-4624.00277}.

\bibitem[\citeproctext]{ref-Williamson2000-WILKAI}
Williamson, Timothy. 2000. \emph{{Knowledge and its Limits}}. Oxford
University Press.

\bibitem[\citeproctext]{ref-Yablo2002}
Yablo, Stephen. 2002. {``Coulda, Woulda, Shoulda.''} In
\emph{Conceivability and Possibility}, edited by Tamar Szabó Gendler and
John Hawthorne, 441--92. Oxford: Oxford University Press.

\end{CSLReferences}



\noindent Published in\emph{
Oxford Studies in Epistemology}, 2006, pp. 311-331.


\end{document}
