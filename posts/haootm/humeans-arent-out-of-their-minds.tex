% Options for packages loaded elsewhere
\PassOptionsToPackage{unicode}{hyperref}
\PassOptionsToPackage{hyphens}{url}
%
\documentclass[
  11pt,
  letterpaper,
  DIV=11,
  numbers=noendperiod,
  twoside]{scrartcl}

\usepackage{amsmath,amssymb}
\usepackage{setspace}
\usepackage{iftex}
\ifPDFTeX
  \usepackage[T1]{fontenc}
  \usepackage[utf8]{inputenc}
  \usepackage{textcomp} % provide euro and other symbols
\else % if luatex or xetex
  \usepackage{unicode-math}
  \defaultfontfeatures{Scale=MatchLowercase}
  \defaultfontfeatures[\rmfamily]{Ligatures=TeX,Scale=1}
\fi
\usepackage{lmodern}
\ifPDFTeX\else  
    % xetex/luatex font selection
    \setmainfont[ItalicFont=EB Garamond Italic,BoldFont=EB Garamond
Bold]{EB Garamond Math}
    \setsansfont[]{EB Garamond}
  \setmathfont[]{Garamond-Math}
\fi
% Use upquote if available, for straight quotes in verbatim environments
\IfFileExists{upquote.sty}{\usepackage{upquote}}{}
\IfFileExists{microtype.sty}{% use microtype if available
  \usepackage[]{microtype}
  \UseMicrotypeSet[protrusion]{basicmath} % disable protrusion for tt fonts
}{}
\usepackage{xcolor}
\usepackage[left=1.1in, right=1in, top=0.8in, bottom=0.8in,
paperheight=9.5in, paperwidth=7in, includemp=TRUE, marginparwidth=0in,
marginparsep=0in]{geometry}
\setlength{\emergencystretch}{3em} % prevent overfull lines
\setcounter{secnumdepth}{3}
% Make \paragraph and \subparagraph free-standing
\makeatletter
\ifx\paragraph\undefined\else
  \let\oldparagraph\paragraph
  \renewcommand{\paragraph}{
    \@ifstar
      \xxxParagraphStar
      \xxxParagraphNoStar
  }
  \newcommand{\xxxParagraphStar}[1]{\oldparagraph*{#1}\mbox{}}
  \newcommand{\xxxParagraphNoStar}[1]{\oldparagraph{#1}\mbox{}}
\fi
\ifx\subparagraph\undefined\else
  \let\oldsubparagraph\subparagraph
  \renewcommand{\subparagraph}{
    \@ifstar
      \xxxSubParagraphStar
      \xxxSubParagraphNoStar
  }
  \newcommand{\xxxSubParagraphStar}[1]{\oldsubparagraph*{#1}\mbox{}}
  \newcommand{\xxxSubParagraphNoStar}[1]{\oldsubparagraph{#1}\mbox{}}
\fi
\makeatother


\providecommand{\tightlist}{%
  \setlength{\itemsep}{0pt}\setlength{\parskip}{0pt}}\usepackage{longtable,booktabs,array}
\usepackage{calc} % for calculating minipage widths
% Correct order of tables after \paragraph or \subparagraph
\usepackage{etoolbox}
\makeatletter
\patchcmd\longtable{\par}{\if@noskipsec\mbox{}\fi\par}{}{}
\makeatother
% Allow footnotes in longtable head/foot
\IfFileExists{footnotehyper.sty}{\usepackage{footnotehyper}}{\usepackage{footnote}}
\makesavenoteenv{longtable}
\usepackage{graphicx}
\makeatletter
\newsavebox\pandoc@box
\newcommand*\pandocbounded[1]{% scales image to fit in text height/width
  \sbox\pandoc@box{#1}%
  \Gscale@div\@tempa{\textheight}{\dimexpr\ht\pandoc@box+\dp\pandoc@box\relax}%
  \Gscale@div\@tempb{\linewidth}{\wd\pandoc@box}%
  \ifdim\@tempb\p@<\@tempa\p@\let\@tempa\@tempb\fi% select the smaller of both
  \ifdim\@tempa\p@<\p@\scalebox{\@tempa}{\usebox\pandoc@box}%
  \else\usebox{\pandoc@box}%
  \fi%
}
% Set default figure placement to htbp
\def\fps@figure{htbp}
\makeatother
% definitions for citeproc citations
\NewDocumentCommand\citeproctext{}{}
\NewDocumentCommand\citeproc{mm}{%
  \begingroup\def\citeproctext{#2}\cite{#1}\endgroup}
\makeatletter
 % allow citations to break across lines
 \let\@cite@ofmt\@firstofone
 % avoid brackets around text for \cite:
 \def\@biblabel#1{}
 \def\@cite#1#2{{#1\if@tempswa , #2\fi}}
\makeatother
\newlength{\cslhangindent}
\setlength{\cslhangindent}{1.5em}
\newlength{\csllabelwidth}
\setlength{\csllabelwidth}{3em}
\newenvironment{CSLReferences}[2] % #1 hanging-indent, #2 entry-spacing
 {\begin{list}{}{%
  \setlength{\itemindent}{0pt}
  \setlength{\leftmargin}{0pt}
  \setlength{\parsep}{0pt}
  % turn on hanging indent if param 1 is 1
  \ifodd #1
   \setlength{\leftmargin}{\cslhangindent}
   \setlength{\itemindent}{-1\cslhangindent}
  \fi
  % set entry spacing
  \setlength{\itemsep}{#2\baselineskip}}}
 {\end{list}}
\usepackage{calc}
\newcommand{\CSLBlock}[1]{\hfill\break\parbox[t]{\linewidth}{\strut\ignorespaces#1\strut}}
\newcommand{\CSLLeftMargin}[1]{\parbox[t]{\csllabelwidth}{\strut#1\strut}}
\newcommand{\CSLRightInline}[1]{\parbox[t]{\linewidth - \csllabelwidth}{\strut#1\strut}}
\newcommand{\CSLIndent}[1]{\hspace{\cslhangindent}#1}

\setlength\heavyrulewidth{0ex}
\setlength\lightrulewidth{0ex}
\usepackage[automark]{scrlayer-scrpage}
\clearpairofpagestyles
\cehead{
  Brian Weatherson
  }
\cohead{
  Humeans Aren’t Out of Their Minds
  }
\ohead{\bfseries \pagemark}
\cfoot{}
\makeatletter
\newcommand*\NoIndentAfterEnv[1]{%
  \AfterEndEnvironment{#1}{\par\@afterindentfalse\@afterheading}}
\makeatother
\NoIndentAfterEnv{itemize}
\NoIndentAfterEnv{enumerate}
\NoIndentAfterEnv{description}
\NoIndentAfterEnv{quote}
\NoIndentAfterEnv{equation}
\NoIndentAfterEnv{longtable}
\NoIndentAfterEnv{abstract}
\renewenvironment{abstract}
 {\vspace{-1.25cm}
 \quotation\small\noindent\emph{Abstract}:}
 {\endquotation}
\newfontfamily\tfont{EB Garamond}
\addtokomafont{disposition}{\rmfamily}
\addtokomafont{title}{\normalfont\itshape}
\let\footnoterule\relax
\KOMAoption{captions}{tableheading}
\makeatletter
\@ifpackageloaded{caption}{}{\usepackage{caption}}
\AtBeginDocument{%
\ifdefined\contentsname
  \renewcommand*\contentsname{Table of contents}
\else
  \newcommand\contentsname{Table of contents}
\fi
\ifdefined\listfigurename
  \renewcommand*\listfigurename{List of Figures}
\else
  \newcommand\listfigurename{List of Figures}
\fi
\ifdefined\listtablename
  \renewcommand*\listtablename{List of Tables}
\else
  \newcommand\listtablename{List of Tables}
\fi
\ifdefined\figurename
  \renewcommand*\figurename{Figure}
\else
  \newcommand\figurename{Figure}
\fi
\ifdefined\tablename
  \renewcommand*\tablename{Table}
\else
  \newcommand\tablename{Table}
\fi
}
\@ifpackageloaded{float}{}{\usepackage{float}}
\floatstyle{ruled}
\@ifundefined{c@chapter}{\newfloat{codelisting}{h}{lop}}{\newfloat{codelisting}{h}{lop}[chapter]}
\floatname{codelisting}{Listing}
\newcommand*\listoflistings{\listof{codelisting}{List of Listings}}
\makeatother
\makeatletter
\makeatother
\makeatletter
\@ifpackageloaded{caption}{}{\usepackage{caption}}
\@ifpackageloaded{subcaption}{}{\usepackage{subcaption}}
\makeatother

\usepackage{bookmark}

\IfFileExists{xurl.sty}{\usepackage{xurl}}{} % add URL line breaks if available
\urlstyle{same} % disable monospaced font for URLs
\hypersetup{
  pdftitle={Humeans Aren't Out of Their Minds},
  pdfauthor={Brian Weatherson},
  hidelinks,
  pdfcreator={LaTeX via pandoc}}


\title{Humeans Aren't Out of Their Minds}
\author{Brian Weatherson}
\date{2006}

\begin{document}
\maketitle
\begin{abstract}
Humeans about causation say that in some situations, whether C causes E
depends on events far away from C and E. John Hawthorne has objected to
this feature of the view. Whether one has a mind depends on what causal
relations obtain between the parts of one's brain. But whether one has a
mind does not depend on what happens far far away. I reply on behalf of
the Humean. In the cases Hawthorne is worried about, the Humean can and
should deny the problematic long-range dependence. One advantage of
Humeanism is that it lets us make sense of the idea that the laws could
be different in different parts of the universe. In the cases Hawthorne
is worried about, that's exactly what is happening - the laws are
different here to how they are over there. And what causal relations
obtain here is only dependent on what happens in places the laws are the
same.
\end{abstract}


\setstretch{1.1}
Humeanism is ``the thesis that the whole truth about a world like ours
supervenes on the spatiotemporal distribution of local qualities.''
(\citeproc{ref-Lewis1994a}{Lewis 1994, 473}) Since the whole truth about
our world contains truths about causation, causation must be located in
the mosaic of local qualities that the Humean says constitute the whole
truth about the world. The most natural ways to do this involve
causation being in some sense extrinsic. To take the simplest possible
Humean analysis, we might say that \emph{c} causes \emph{e} iff
throughout the mosaic events of the same type as \emph{c} are usually
followed by events of type \emph{e}. For short, the causal relation is
the constant conjunction relation. Whether this obtains is determined by
the mosaic, so this is a Humean theory, but it isn't determined just by
\emph{c} and \emph{e} themselves, so whether \emph{c} causes \emph{e} is
extrinsic to the pair. Now this is obviously a bad theory of causation,
but the fact that causation is extrinsic is retained even by good Humean
theories of causation. John Hawthorne
(\citeproc{ref-Hawthorne2004-Humeans}{2004}) objects to this feature of
Humeanism. I'm going to argue that his arguments don't work, but first
we need to clear up three preliminaries about causation and
intrinsicness.

First, my wording so far has been cagey because I haven't wanted to say
that Humeans typically take causation to be an extrinsic
\emph{relation}. That's because the greatest Humean of them all, David
Lewis, denies that causation is a relation at all, and hence that it is
an extrinsic relation (\citeproc{ref-Lewis2004d}{Lewis 2004b}). We can
go some way to avoiding this complication by talking, as Hawthorne does,
about properties of regions, and asking the property of containing a
duplicate of \emph{c} that causes a duplicate of \emph{e} is intrinsic
or extrinsic.\footnote{This move requires the not wholly uncontroversial
  assumption that regions are the kinds of things that can have
  properties. But I'll happily make that assumption here. Note that the
  formulation here allows that the property denoted might be intrinsic
  for some \emph{c} and \emph{e} and extrinsic for others. I'll say
  causation is extrinsic if the property denoted is extrinsic for some
  choice of \emph{c} and \emph{e}, even if it is intrinsic for others,
  as it might be if, for example, no region could possess the property
  because \emph{c} is a part of \emph{e}.} Humeans typically take
causation to be extrinsic in this sense.

Second, nothing in Humeanism \emph{requires} that causation is extrinsic
in that sense. If one analysed causation as \emph{that intrinsic
relation that actually most tightly correlates with the constant
conjunction relation}, then one would have guaranteed that causation was
an intrinsic relation. Moreover, one would have a perfectly Humean
theory of causation. (A perfectly awful theory, to be sure, but still a
Humean one.) Peter Menzies (\citeproc{ref-Menzies1996}{1996},
\citeproc{ref-Menzies1999}{1999}) develops a more sophisticated version
of such a theory, and though Menzies describes his view as anti-Humean,
one can locate the relation we've defined here in the Humean mosaic, so
such an approach might be consistent with Humeanism in the intended
sense.

Third, there is good reason, independent of Humeanism, to accept that
causation is extrinsic. As Ned Hall (\citeproc{ref-Hall2004}{2004})
argues, it is very hard to square the intrinsicness of causation with
the possibility of causation by omission. Given the choice between these
two, I'm going to accept causation by omission without much hesitation.
There is one powerful objection to the possibility of causation by
omission, namely that if there is any causation by omission then there
is a lot more than is intuitively plausible. But since Sarah McGrath
(\citeproc{ref-McGrath2005}{2005}) has a good response to that
objection, I feel happy accepting there is causation by omission. So I
accept that causation is extrinsic, for reasons totally independent of
Humeanism. Since Hawthorne appeals to no feature of Humeanism beyond the
Humean's acceptance of the extrinsicness of causation, we can take his
argument to be an argument against the causal extrinsicalist, i.e.~the
theorist who accepts causation is extrinsic in the above sense. To see
that the argument doesn't go through, we need to consider what exactly
the causal extrinsicalist is committed to. I'll explore this by looking
at some other examples of properties of regions.

Some regions contain uncles and some do not. This seems to be an
extrinsic property of regions. My house does not contain any uncles
right now, but there are duplicates of it, in worlds where my brothers
have children, where it does contain an uncle, namely my counterpart.
Consider the smallest region containing the earth from the stratosphere
in from the earth's formation to its destruction. Call this region
\emph{e}. Any duplicate of \emph{e} also contains uncles, including
several uncles of mine. You can't duplicate the earth without producing
a duplicate of me who is, in the duplicate world, the nephew of the
duplicates of my uncles. So it is intrinsic to \emph{e} that it contain
an uncle, even though this is an extrinsic property of regions. (There
is much more discussion of extrinsic properties that are possessed
intrinsically in Humberstone (\citeproc{ref-Humberstone1996}{1996}).)

This possibility, that a region might intrinsically possess an extrinsic
property, poses a problem for Hawthorne's argument. Here is his
presentation of it.

\begin{quote}
\begin{enumerate}
\def\labelenumi{\arabic{enumi}.}
\tightlist
\item
  An intrinsic duplicate of any region wholly containing me will contain
  a being with my conscious life.
\item
  There are causal requirements on my conscious life.
\end{enumerate}

Therefore, Humeanism is false.
(\citeproc{ref-Hawthorne2004-Humeans}{Hawthorne 2004, 351--52})
\end{quote}

The problem is that this argument isn't valid. What follows from (1) and
(2) is that any region containing Hawthorne must possess some causal
properties intrinsically. (As Hawthorne argues on page 356.) And what
Humeanism entails is that causal properties are extrinsic properties of
regions. But there is no incompatibility here, for it is possible that
extrinsic properties are possessed intrinsically, as we saw in the
discussion of uncles.

Hawthorne's argument would go through if Humeans, and causal
extrinsicalists more generally, were committed to the stronger claim
that regions never possess causal properties intrinsically. But it
doesn't seem that Humeans \emph{should} be committed to this claim.
Consider again \emph{e} and all its duplicates. Any such duplicate will
contain a duplication of the event of Booth's shooting Lincoln, and
Lincoln dying.\footnote{There is a potential complication here in that
  arguably in some such worlds, e.g.~worlds where there is another
  planet on the opposite side of the sun to duplicate-earth where people
  are immediately `resurrected' when the `die' on duplicate-earth. In
  such a world you might say that duplicate-Lincoln doesn't really die
  on duplicate-Earth, but merely has the duplicate-earth part of his
  life ended. We'll understand `dying' in such a way that this counts as
  dying.} Will it also be the case that duplicate-Booth's shooting in
this world \emph{causes} duplicate-Lincoln's dying? If so, and this
seems true, then it is intrinsic to \emph{e} that it contains an event
of a shooting causing a dying, even though the property of containing a
shooting causing a dying is extrinsic.

It would be a bad mistake to offer the following epistemological
argument that in all duplicates of \emph{e}, duplicate-Booth's shooting
causes duplicate-Lincoln's dying.

\begin{enumerate}
\def\labelenumi{\arabic{enumi}.}
\tightlist
\item
  If there was a duplicate of \emph{e} where duplicate-Booth's shooting
  does not cause duplicate-Lincoln's dying, then we would not know
  whether Booth's shooting causes Lincoln's dying without investigating
  what happens outside \emph{e}.
\item
  We can know that Booth's shooting caused Lincoln's dying without
  investigating outside \emph{e}.
\item
  So, there is no duplicate of \emph{e} where duplicate-Booth's shooting
  does not cause duplicate-Lincoln's dying.
\end{enumerate}

The problem with this argument is that even there are worlds containing
such duplicates, we might know a priori that we do not live in such a
world, just as we know a priori that we do not live in a world where
certain kinds of sceptical scenarios unfold
(\citeproc{ref-Hawthorne2002}{Hawthorne 2002};
\citeproc{ref-WeathersonSRE}{Weatherson 2005}).

A better argument against the existence of such a world is that if it is
possible, it should be conceivable. But it is basically impossible to
conceive such a world. Even if throughout the universe shootings like
Booth's are usually followed by something other than dying, say that
shooting in most parts of the universe causes diseases to be cured, the
large-scale regularity within \emph{e} (or its duplicate) of shootings
being followed by dying suffices to ground the claim that shootings
cause dyings in a good Humean theory. The crucial assumption here is
that \emph{local} regularities count for more than \emph{global}
regularities. If the local regularities deviate too far from the global
regularities, then Humeans can and should say that different nomic
claims (and hence causal claims) are true in this part of the world to
the rest of the universe. If they say this, they can say that regions
can have causal features (such as containing a shooting causing a dying)
intrinsically even though causal features are extrinsic properties.

To illustrate the kind of Humean theory that would have such a
consequence, consider the following variant on the constant conjunction
theory of causation. The theory I'm imagining says that \emph{c} causes
\emph{e} iff whenever an event of the same type as \emph{c} occurs
within a 50 mile radius of where \emph{c} occurred, it was followed by
an event of type \emph{e}. Call this the 50 mile constant conjunction
theory of causation.\footnote{I assume here that events can be properly
  said to have locations. Spelling out this assumption in detail will
  require some serious metaphysics, particularly when it comes to
  omissions. Where exactly does my omission to stop the Iraq War take
  place? Here at my kitchen table where I am? In Iraq, where the war is?
  In Washington, if that's where I'd be were I doing something to stop
  the war? These questions are hard, though not so hard that we should
  give up on the very natural idea that events have locations.} On the
50 mile constant conjunction theory of causation, it won't be intrinsic
to Ford's Theatre that it contained a causal event involving Booth
shooting Lincoln, but it will be intrinsic to any sphere of radius 50
miles or more centred on the theatre that it contains such a causal
event. So on this theory causal properties can be intrinsic to a region,
though they are still extrinsic properties of such a region.

That's a very implausible Humean theory, but when we look at the details
of David Lewis's Humean picture, we can see the outlines of a more
plausible theory with the same consequences. Lewis of course doesn't
offer a simple regularity theory of causation. Rather, he first argues
that laws are the extremely simple, extremely informative true
propositions (\citeproc{ref-Lewis1973a}{Lewis 1973, 73}). That is, he
offers a sophisticated regularity theory of laws. Then he analyses
counterfactual dependence in terms of lawhood
(\citeproc{ref-Lewis1973a}{Lewis 1973},
\citeproc{ref-Lewis1979c}{1979}). Finally he analyses causation in terms
of counterfactual dependence (\citeproc{ref-Lewis2004a}{Lewis 2004a}).
The philosophical theory meets the Humean mosaic most closely on the
issue of what a law is. If we can offer a theory of laws that allows
extra sensitivity to local facts, while remaining Humean, we can plug
this back into Lewis's theories concerning counterfactual dependence and
hence causation without upsetting its Humean credentials.

Now there is a good reason to think that a Humean theory of laws should
be locally sensitive. (I'm indebted here to long ago conversations with
James Chase.) Humeans typically believe in fairly unrestricted
principles of recombination. And they believe that laws are not
necessarily true. So there could be worlds with very different laws. So
there is a world which `patches' together part of the world with laws
\emph{L}\textsubscript{1} with a world with laws
\emph{L}\textsubscript{2}. If the parts are large and isolated enough,
it would be foolish to say that within those parts nothing is
law-governed, or that within those regions there is no counterfactual
dependence, or no causation. Much better to say that regularities
obtaining within such a region are sufficiently simple and informative
to count as laws. In our patchwork world, the laws might simply say
\emph{In r}\textsubscript{1}, \emph{L}\textsubscript{1} \emph{and in
r}\textsubscript{2}, \emph{L}\textsubscript{2}. Provided the terms
denoting the regions are not too gruesome, these will plausibly be
Humean, even Lewisian, laws.

Let's bring all this back to Hawthorne's example. Hawthorne argues that
certain causal facts are intrinsic to the region containing his body.
The challenge for the Humean is to say how this could be so when
Hawthorne could be embedded in a world where very different regularities
obtain. The simple answer is to say that in such worlds, laws like
\emph{In r, L}, where \emph{r} picks out the region Hawthorne's body
occupies, and \emph{L} picks out a real-world law, will be true, simple
and informative. It is informative because any duplicate of Hawthorne's
body is a very complicated entity, containing billions of billions of
particles interacting in systematic ways, ways that are nicely
summarised by real-world laws. Simplicity is a little harder to make
out, but note that there is a reasonably sharp boundary between
Hawthorne's body and the rest of the world
(\citeproc{ref-Lewis1993c}{Lewis 1993}), so there should be a natural
enough way to pick it out. In other words, even if we embed a Hawthorne
duplicate in a world with very different regularities, Humeans will
still have good reason to say that the laws, and hence the facts about
counterfactual dependence and causation, inside that duplicate are not
changed. So not only is it logically possible that Hawthorne's premises
are true and his conclusion false, we can motivate a Humean position
that endorses the truth of Hawthorne's premises and the falsity of the
conclusion.

Since Hawthorne's argument is invalid then, we can accept the premises
without giving up Humeanism. But I think it is worthwhile to note that
his (1) also can be questioned. Hawthorne notes that it is rejected by
those such as Dretske and Lewis who say that phenomenal character is
determined in part by kind membership. (See Lycan
(\citeproc{ref-Lycan2001}{2001}) for a longer defence of this kind of
rejection of (1).) Hawthorne thinks that the intuitive plausibility of
(1) constitutes a serious objection to those views. But by reflecting a
little on the phenomenology of what I'll call \emph{totality qualia}, we
can undermine the intuitive case for (1).

Tweedledee is facing a perfectly symmetrical scene. His visual field is
symmetric, with two gentle mountains rising to his left and his right
and a symmetric plain in between them. All he can hear are two birds
singing in perfect harmony, one behind his left ear and one behind his
right ear. The smells of the field seem to envelope him rather than
coming from any particular direction. There is a cool breeze blowing
directly on his face. It's a rather pleasant scene, and the overwhelming
feeling is one of \emph{symmetry}.

Tweedledum is very much like Tweedledee. Indeed, Tweedledum contains a
duplicate of Tweedledee as a proper part. But Tweedledum also has some
sensors in his skin, and brain cells in what corresponds to a
suspiciously empty part of Tweedledee's brain, that allow him to detect,
and \emph{feel}, where the magnetic fields are in the vicinity. And
sadly, though Tweedledum is facing a duplicate of the scene facing
Tweedledee, there is a major disturbance in the magnetic field just to
Tweedledum's left. This produces a jarring sensation in Tweedledum's
left side. As a consequence, Tweedledum does not share Tweedledee's
feeling of symmetry.

Whether a picture is symmetric is a property of its internal features,
but it is also a feature that can be destroyed without changing the
internal features by just adding more material to one side. It is a
\emph{totality} property of pictures, a property the picture has because
it stops just where it does.\footnote{Ted Sider
  (\citeproc{ref-Sider2001}{2001}, \citeproc{ref-Sider2003}{2003})
  stresses the importance to a theory of intrinsicness of properties
  that are instantiated in virtue of the object \emph{not} bearing
  relations to other objects. My example here is closely modeled on
  examples from his papers.} Similarly, totality qualia are qualia that
we have in part because we don't have any more feelings than we actually
do. Feelings of symmetry are totality qualia in this sense, as are many
of the feelings of calm and peacefulness associated with Tweedledee's
state. It is \emph{not} intuitive that totality qualia should be
intrinsic to a region. Indeed, it seems intuitive that a duplicate of me
that was extended to produce \emph{more} sensory features would lack
these feelings. Hence a duplicate of me would not share my conscious
life in \emph{all} respects, so Hawthorne's premise (1) is also false.
To be sure, these totality qualia are a somewhat speculative suggestion,
but the Humean does not \emph{need} them since Hawthorne's anti-Humean
argument is invalid.

\section*{References}\label{references}
\addcontentsline{toc}{section}{References}

\phantomsection\label{refs}
\begin{CSLReferences}{1}{0}
\bibitem[\citeproctext]{ref-Hall2004}
Hall, Ned. 2004. {``Causation and the Price of Transitivity.''} In
\emph{Causation and Counterfactuals}, edited by John Collins, Ned Hall,
and L. A. Paul, 181--203. Cambridge: MIT Press.

\bibitem[\citeproctext]{ref-Hawthorne2002}
Hawthorne, John. 2002. {``Deeply Contingent a Priori Knowledge.''}
\emph{Philosophy and Phenomenological Research} 65 (2): 247--69. doi:
\href{https://doi.org/10.1111/j.1933-1592.2002.tb00201.x}{10.1111/j.1933-1592.2002.tb00201.x}.

\bibitem[\citeproctext]{ref-Hawthorne2004-Humeans}
---------. 2004. {``Humeans Are Out of Their Minds.''} \emph{No{û}s} 38
(2): 351--58. doi:
\href{https://doi.org/10.1111/j.1468-0068.2004.00473.x}{10.1111/j.1468-0068.2004.00473.x}.

\bibitem[\citeproctext]{ref-Humberstone1996}
Humberstone, I. L. 1996. {``Intrinsic/Extrinsic.''} \emph{Synthese} 108
(2): 205--67. doi:
\href{https://doi.org/10.1007/bf00413498}{10.1007/bf00413498}.

\bibitem[\citeproctext]{ref-Lewis1973a}
Lewis, David. 1973. \emph{Counterfactuals}. Oxford: Blackwell
Publishers.

\bibitem[\citeproctext]{ref-Lewis1979c}
---------. 1979. {``Counterfactual Dependence and Time's Arrow.''}
\emph{No{û}s} 13 (4): 455--76. doi:
\href{https://doi.org/10.2307/2215339}{10.2307/2215339}. Reprinted in
his \emph{Philosophical Papers}, Volume 2, Oxford: Oxford University
Press, 1986, 32-52. References to reprint.

\bibitem[\citeproctext]{ref-Lewis1993c}
---------. 1993. {``Many, but Almost One.''} In \emph{Ontology,
Causality, and Mind: Essays on the Philosophy of {D. M. Armstrong}},
edited by Keith Campbell, John Bacon, and Lloyd Reinhardt, 23--38.
Cambridge: Cambridge University Press. doi:
\href{https://doi.org/10.1017/CBO9780511625343.010}{10.1017/CBO9780511625343.010}.
Reprinted in his \emph{Papers in Metaphysics and Epistemology},
Cambridge: Cambridge University Press, 1999, 164-182. References to
reprint.

\bibitem[\citeproctext]{ref-Lewis1994a}
---------. 1994. {``Humean Supervenience Debugged.''} \emph{Mind} 103
(412): 473--90. doi:
\href{https://doi.org/10.1093/mind/103.412.473}{10.1093/mind/103.412.473}.
Reprinted in his \emph{Papers in Metaphysics and Epistemology},
Cambridge: Cambridge University Press, 1999, 224-247. References to
reprint.

\bibitem[\citeproctext]{ref-Lewis2004a}
---------. 2004a. {``Causation as Influence.''} In \emph{Causation and
Counterfactuals}, edited by John Collins, Ned Hall, and L. A. Paul,
75--106. Cambridge: {MIT} Press.

\bibitem[\citeproctext]{ref-Lewis2004d}
---------. 2004b. {``Void and Object.''} In \emph{Causation and
Counterfactuals}, edited by John Collins, Ned Hall, and L. A. Paul,
277--90. Cambridge: {MIT} Press.

\bibitem[\citeproctext]{ref-Lycan2001}
Lycan, William. 2001. {``The Case for Phenomenal Externalism.''}
\emph{Philosophical Perspectives} 15: 17--35. doi:
\href{https://doi.org/10.1111/0029-4624.35.s15.2}{10.1111/0029-4624.35.s15.2}.

\bibitem[\citeproctext]{ref-McGrath2005}
McGrath, Sarah. 2005. {``Causation by Omission: A Dilemma.''}
\emph{Philosophical Studies} 123 (1-2): 125--48. doi:
\href{https://doi.org/10.1007/s11098-004-5216-z}{10.1007/s11098-004-5216-z}.

\bibitem[\citeproctext]{ref-Menzies1996}
Menzies, Peter. 1996. {``Probabilistic Causation and the Pre-Emption
Problem.''} \emph{Mind} 105 (417): 85--117. doi:
\href{https://doi.org/10.1093/mind/105.417.85}{10.1093/mind/105.417.85}.

\bibitem[\citeproctext]{ref-Menzies1999}
---------. 1999. {``Intrinsic Versus Extrinsic Conceptions of
Causation.''} In \emph{Causation and Laws of Nature}, edited by Howard
Sankey, 313--29. Dordrecht: Kluwer.

\bibitem[\citeproctext]{ref-Sider2001}
Sider, Theodore. 2001. {``Maximality and Intrinsic Properties.''}
\emph{Philosophy and Phenomenological Research} 63 (2): 357--64. doi:
\href{https://doi.org/10.1111/j.1933-1592.2001.tb00109.x}{10.1111/j.1933-1592.2001.tb00109.x}.

\bibitem[\citeproctext]{ref-Sider2003}
---------. 2003. {``Maximality and Microphysical Supervenience.''}
\emph{Philosophy and Phenomenological Research} 66 (1): 139--49. doi:
\href{https://doi.org/10.1111/j.1933-1592.2003.tb00247.x}{10.1111/j.1933-1592.2003.tb00247.x}.

\bibitem[\citeproctext]{ref-WeathersonSRE}
Weatherson, Brian. 2005. {``Scepticism, Rationalism and Externalism.''}
\emph{Oxford Studies in Epistemology} 1: 311--31.

\end{CSLReferences}



\noindent Published in\emph{
Noûs}, 2006, pp. 529-535.


\end{document}
