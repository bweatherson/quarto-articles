% Options for packages loaded elsewhere
\PassOptionsToPackage{unicode}{hyperref}
\PassOptionsToPackage{hyphens}{url}
%
\documentclass[
  10pt,
  letterpaper,
  DIV=11,
  numbers=noendperiod,
  twoside]{scrartcl}

\usepackage{amsmath,amssymb}
\usepackage{setspace}
\usepackage{iftex}
\ifPDFTeX
  \usepackage[T1]{fontenc}
  \usepackage[utf8]{inputenc}
  \usepackage{textcomp} % provide euro and other symbols
\else % if luatex or xetex
  \usepackage{unicode-math}
  \defaultfontfeatures{Scale=MatchLowercase}
  \defaultfontfeatures[\rmfamily]{Ligatures=TeX,Scale=1}
\fi
\usepackage{lmodern}
\ifPDFTeX\else  
    % xetex/luatex font selection
    \setmainfont[ItalicFont=EB Garamond Italic,BoldFont=EB Garamond
Bold]{EB Garamond Math}
    \setsansfont[]{Europa-Bold}
  \setmathfont[]{Garamond-Math}
\fi
% Use upquote if available, for straight quotes in verbatim environments
\IfFileExists{upquote.sty}{\usepackage{upquote}}{}
\IfFileExists{microtype.sty}{% use microtype if available
  \usepackage[]{microtype}
  \UseMicrotypeSet[protrusion]{basicmath} % disable protrusion for tt fonts
}{}
\usepackage{xcolor}
\usepackage[left=1in, right=1in, top=0.8in, bottom=0.8in,
paperheight=9.5in, paperwidth=6.5in, includemp=TRUE, marginparwidth=0in,
marginparsep=0in]{geometry}
\setlength{\emergencystretch}{3em} % prevent overfull lines
\setcounter{secnumdepth}{3}
% Make \paragraph and \subparagraph free-standing
\makeatletter
\ifx\paragraph\undefined\else
  \let\oldparagraph\paragraph
  \renewcommand{\paragraph}{
    \@ifstar
      \xxxParagraphStar
      \xxxParagraphNoStar
  }
  \newcommand{\xxxParagraphStar}[1]{\oldparagraph*{#1}\mbox{}}
  \newcommand{\xxxParagraphNoStar}[1]{\oldparagraph{#1}\mbox{}}
\fi
\ifx\subparagraph\undefined\else
  \let\oldsubparagraph\subparagraph
  \renewcommand{\subparagraph}{
    \@ifstar
      \xxxSubParagraphStar
      \xxxSubParagraphNoStar
  }
  \newcommand{\xxxSubParagraphStar}[1]{\oldsubparagraph*{#1}\mbox{}}
  \newcommand{\xxxSubParagraphNoStar}[1]{\oldsubparagraph{#1}\mbox{}}
\fi
\makeatother


\providecommand{\tightlist}{%
  \setlength{\itemsep}{0pt}\setlength{\parskip}{0pt}}\usepackage{longtable,booktabs,array}
\usepackage{calc} % for calculating minipage widths
% Correct order of tables after \paragraph or \subparagraph
\usepackage{etoolbox}
\makeatletter
\patchcmd\longtable{\par}{\if@noskipsec\mbox{}\fi\par}{}{}
\makeatother
% Allow footnotes in longtable head/foot
\IfFileExists{footnotehyper.sty}{\usepackage{footnotehyper}}{\usepackage{footnote}}
\makesavenoteenv{longtable}
\usepackage{graphicx}
\makeatletter
\newsavebox\pandoc@box
\newcommand*\pandocbounded[1]{% scales image to fit in text height/width
  \sbox\pandoc@box{#1}%
  \Gscale@div\@tempa{\textheight}{\dimexpr\ht\pandoc@box+\dp\pandoc@box\relax}%
  \Gscale@div\@tempb{\linewidth}{\wd\pandoc@box}%
  \ifdim\@tempb\p@<\@tempa\p@\let\@tempa\@tempb\fi% select the smaller of both
  \ifdim\@tempa\p@<\p@\scalebox{\@tempa}{\usebox\pandoc@box}%
  \else\usebox{\pandoc@box}%
  \fi%
}
% Set default figure placement to htbp
\def\fps@figure{htbp}
\makeatother
% definitions for citeproc citations
\NewDocumentCommand\citeproctext{}{}
\NewDocumentCommand\citeproc{mm}{%
  \begingroup\def\citeproctext{#2}\cite{#1}\endgroup}
\makeatletter
 % allow citations to break across lines
 \let\@cite@ofmt\@firstofone
 % avoid brackets around text for \cite:
 \def\@biblabel#1{}
 \def\@cite#1#2{{#1\if@tempswa , #2\fi}}
\makeatother
\newlength{\cslhangindent}
\setlength{\cslhangindent}{1.5em}
\newlength{\csllabelwidth}
\setlength{\csllabelwidth}{3em}
\newenvironment{CSLReferences}[2] % #1 hanging-indent, #2 entry-spacing
 {\begin{list}{}{%
  \setlength{\itemindent}{0pt}
  \setlength{\leftmargin}{0pt}
  \setlength{\parsep}{0pt}
  % turn on hanging indent if param 1 is 1
  \ifodd #1
   \setlength{\leftmargin}{\cslhangindent}
   \setlength{\itemindent}{-1\cslhangindent}
  \fi
  % set entry spacing
  \setlength{\itemsep}{#2\baselineskip}}}
 {\end{list}}
\usepackage{calc}
\newcommand{\CSLBlock}[1]{\hfill\break\parbox[t]{\linewidth}{\strut\ignorespaces#1\strut}}
\newcommand{\CSLLeftMargin}[1]{\parbox[t]{\csllabelwidth}{\strut#1\strut}}
\newcommand{\CSLRightInline}[1]{\parbox[t]{\linewidth - \csllabelwidth}{\strut#1\strut}}
\newcommand{\CSLIndent}[1]{\hspace{\cslhangindent}#1}

\setlength\heavyrulewidth{0ex}
\setlength\lightrulewidth{0ex}
\usepackage[automark]{scrlayer-scrpage}
\clearpairofpagestyles
\cehead{
  Brian Weatherson
  }
\cohead{
  Disagreements, Philosophical and Otherwise
  }
\ohead{\bfseries \pagemark}
\cfoot{}
\makeatletter
\newcommand*\NoIndentAfterEnv[1]{%
  \AfterEndEnvironment{#1}{\par\@afterindentfalse\@afterheading}}
\makeatother
\NoIndentAfterEnv{itemize}
\NoIndentAfterEnv{enumerate}
\NoIndentAfterEnv{description}
\NoIndentAfterEnv{quote}
\NoIndentAfterEnv{equation}
\NoIndentAfterEnv{longtable}
\NoIndentAfterEnv{abstract}
\renewenvironment{abstract}
 {\vspace{-1.25cm}
 \quotation\small\noindent\rule{\linewidth}{.5pt}\par\smallskip
 \noindent }
 {\par\noindent\rule{\linewidth}{.5pt}\endquotation}
\KOMAoption{captions}{tableheading}
\makeatletter
\@ifpackageloaded{caption}{}{\usepackage{caption}}
\AtBeginDocument{%
\ifdefined\contentsname
  \renewcommand*\contentsname{Table of contents}
\else
  \newcommand\contentsname{Table of contents}
\fi
\ifdefined\listfigurename
  \renewcommand*\listfigurename{List of Figures}
\else
  \newcommand\listfigurename{List of Figures}
\fi
\ifdefined\listtablename
  \renewcommand*\listtablename{List of Tables}
\else
  \newcommand\listtablename{List of Tables}
\fi
\ifdefined\figurename
  \renewcommand*\figurename{Figure}
\else
  \newcommand\figurename{Figure}
\fi
\ifdefined\tablename
  \renewcommand*\tablename{Table}
\else
  \newcommand\tablename{Table}
\fi
}
\@ifpackageloaded{float}{}{\usepackage{float}}
\floatstyle{ruled}
\@ifundefined{c@chapter}{\newfloat{codelisting}{h}{lop}}{\newfloat{codelisting}{h}{lop}[chapter]}
\floatname{codelisting}{Listing}
\newcommand*\listoflistings{\listof{codelisting}{List of Listings}}
\makeatother
\makeatletter
\makeatother
\makeatletter
\@ifpackageloaded{caption}{}{\usepackage{caption}}
\@ifpackageloaded{subcaption}{}{\usepackage{subcaption}}
\makeatother

\usepackage{bookmark}

\IfFileExists{xurl.sty}{\usepackage{xurl}}{} % add URL line breaks if available
\urlstyle{same} % disable monospaced font for URLs
\hypersetup{
  pdftitle={Disagreements, Philosophical and Otherwise},
  pdfauthor={Brian Weatherson},
  hidelinks,
  pdfcreator={LaTeX via pandoc}}


\title{Disagreements, Philosophical and Otherwise}
\author{Brian Weatherson}
\date{2013}

\begin{document}
\maketitle
\begin{abstract}
The Equal Weight View of disagreement says that if an agent sees that an
epistemic peer disagrees with her about p, the agent should change her
credence in p to half way between her initial credence, and the peer's
credence. But it is hard to believe the Equal Weight View for a
surprising reason; not everyone believes it. And that means that if one
did believe it, one would be required to lower one's belief in it in
light of this peer disagreement. Brian Weatherson explores the options
for how a proponent of the Equal Weight View might respond to this
difficulty, and how this challenge fits into broader arguments against
the Equal Weight View.
\end{abstract}


\setstretch{1.1}
This paper started life as a short note I wrote around New Year 2007
while in Minneapolis. It was originally intended as a blog post. That
might explain, if not altogether excuse, the flippant tone in places.
But it got a little long for a post, so I made it into the format of a
paper and posted it to my website. The paper has received a lot of
attention, so it seems like it will be helpful to see it in print. Since
a number of people have responded to the argument as stated, I've
decided to just reprint the article warts and all, and make a few
comments at the end about how I see its argument in the context of the
subsequent debate.

\section*{Disagreeing about Disagreement
(2007)}\label{disagreeing-about-disagreement-2007}
\addcontentsline{toc}{section}{Disagreeing about Disagreement (2007)}

I argue with my friends a lot. That is, I offer them reasons to believe
all sorts of philosophical conclusions. Sadly, despite the quality of my
arguments, and despite their apparent intelligence, they don't always
agree. They keep insisting on principles in the face of my wittier and
wittier counterexamples, and they keep offering their own dull alleged
counterexamples to my clever principles. What is a philosopher to do in
these circumstances? (And I don't mean get better friends.)

One popular answer these days is that I should, to some extent, defer to
my friends. If I look at a batch of reasons and conclude \emph{p}, and
my equally talented friend reaches an incompatible conclusion \emph{q},
I should revise my opinion so I'm now undecided between \emph{p} and
\emph{q}. I should, in the preferred lingo, assign equal weight to my
view as to theirs. This is despite the fact that I've looked at their
reasons for concluding \emph{q} and found them wanting. If I hadn't, I
would have already concluded \emph{q}. The mere fact that a friend (from
now on I'll leave off the qualifier `equally talented and informed',
since all my friends satisfy that) reaches a contrary opinion should be
reason to move me. Such a position is defended by Richard Feldman
(\citeproc{ref-Feldman2005-FELRTE}{2005},
\citeproc{ref-Feldman2006-FELEPA}{2006}), David Christensen
(\citeproc{ref-Christensen2007-CHREOD}{2007}) and Adam Elga
(\citeproc{ref-Elga2007-ELGRAD}{2007}).

This equal weight view, hereafter EW, is itself a philosophical
position. And while some of my friends believe it, some of my friends do
not. (Nor, I should add for your benefit, do I.) This raises an odd
little dilemma. If EW is correct, then the fact that my friends disagree
about it means that I shouldn't be particularly confident that it is
true, since EW says that I shouldn't be too confident about any position
on which my friends disagree. But, as I'll argue below, to consistently
implement EW, I have to be maximally confident that it is true. So to
accept EW, I have to inconsistently both be very confident that it is
true and not very confident that it is true. This seems like a problem,
and a reason to not accept EW. We can state this argument formally as
follows, using the notion of a peer and an expert. Some people are peers
if they are equally philosophically talented and informed as each other,
and one is more expert than another if they are more informed and
talented than the other.

\begin{enumerate}
\def\labelenumi{\arabic{enumi}.}
\tightlist
\item
  There are peers who disagree about EW, and there is no one who is an
  expert relative to them who endorses EW.
\item
  If 1 is true, then according to EW, my credence in EW should be less
  than 1.
\item
  If my credence in EW is less than 1, then the advice that EW offers in
  a wide range of cases is incoherent.
\item
  So, the advice EW offers in a wide range of cases is incoherent.
\end{enumerate}

The first three sections of this paper will be used to defend the first
three premises. The final section will look at the philosophical
consequences of the conclusion.

\section{Peers and EW}\label{peers-and-ew}

Thomas Kelly (\citeproc{ref-Kelly2005-KELTES}{2005}) has argued against
EW and in favour of the view that a peer with the irrational view should
defer to a peer with the rational view. Elga helpfully dubs this the
`right reasons' view. Ralph Wedgwood
(\citeproc{ref-Wedgwood2007-WEDNON}{2007} Ch. 11) has argued against EW
and in favour of the view that one should have a modest `egocentric
bias', i.e.~a bias towards one's own beliefs. On the other hand, as
mentioned above, Elga, Christensen and Feldman endorse versions of EW.
So it certainly looks like there are very talented and informed
philosophers on either side of this debate.

Now I suppose that if we were taking EW completely seriously, we would
at this stage of the investigation look very closely at whether these
five really are epistemic peers. We could pull out their grad school
transcripts, look at the citation rates for their papers, get reference
letters from expert colleagues, maybe bring one or two of them in for
job-style interviews, and so on. But this all seems somewhat
inappropriate for a scholarly journal. Not to mention a little
tactless.\footnote{Though if EW is correct, shouldn't the scholarly
  journals be full of just this information?} So I'll just stipulate
that they seem to be peers in the sense relevant for EW, and address one
worry a reader may have about my argument.

An objector might say, ``Sure it seems antecedently that Kelly and
Wedgwood are the peers of the folks who endorse EW. But take a look at
the arguments for EW that have been offered. They look like good
arguments, don't they? Doesn't the fact that Kelly and Wedgwood don't
accept these arguments mean that, however talented they might be in
general, they obviously have a blind spot when it comes to the
epistemology of disagreement? If so, we shouldn't treat them as experts
on this question.'' There is something right about this. People can be
experts in one area, or even many areas, while their opinions are
systematically wrong in another. But the objector's line is unavailable
to defenders of EW.

Indeed, these defenders have been quick to distance themselves from the
objector. Here, for instance, is Elga's formulation of the EW view, a
formulation we'll return to below.

\begin{quote}
Your probability in a given disputed claim should equal your prior
conditional probability in that claim. Prior to what? Prior to your
thinking through the claim, and finding out what your advisor thinks of
it. Conditional on what? On whatever you have learned about the
circumstances of how you and your advisor have evaluated the claim.
(\citeproc{ref-Elga2007-ELGRAD}{Elga 2007, 490})
\end{quote}

The fact that Kelly and Wedgwood come to different conclusions can't be
enough reason to declare that they are not peers. As Elga stresses, what
matters is the prior judgment of their acuity. And Elga is right to
stress this. If we declared anyone who doesn't accept reasoning that we
find compelling not a peer, then the EW view will be trivial. After all,
the EW view only gets its force from cases as described in the
introduction, where our friends reject reasoning we accept, and accept
reasons we reject. If that makes them not a peer, the EW view never
applies. So we can't argue that anyone who rejects EW is thereby less of
an expert in the relevant sense than someone who accepts it, merely in
virtue of their rejection of EW. So it seems we should accept premise 1.

\section{Circumstances of Evaluation}\label{circumstances-of-evaluation}

Elga worries about the following kind of case. Let \emph{p} be that the
sum of a certain series of numbers, all of them integers, is 50. Let
\emph{q} be that the sum of those numbers is \(400e\). My friend and I
both add the numbers, and I conclude \emph{p} while he concludes
\emph{q}. It seems that there is no reason to defer to my friend. I
know, after all, that he has made some kind of mistake. The response,
say defenders of EW, is that deference is context-sensitive. If I know,
for example, that my friend is drunk, then I shouldn't defer to him.
More generally, as Elga puts it, how much I should defer should depend
on what I know about the circumstances.

Now this is relevant because one of the relevant circumstances might be
that my friend has come to a view that I regard as insane. That's what
happens in the case of the sums. Since my prior probability that my
friend is right given that he has an insane seeming view is very low, my
posterior probability that my friend is right should also, according to
Elga, be low. Could we say that, although antecedently we regard
Wedgwood and Kelly as peers of those they disagree with, that the
circumstance of their disagreement is such that we should disregard
their views?

It is hard to see how this would be defensible. It is true that a
proponent of EW will regard Kelly and Wedgwood as wrong. But we can't
say that we should disregard the views of all those we regard as
mistaken. That leads to trivialising EW, for reasons given above. The
claim has to be that their views are so outrageous, that we wouldn't
defer to anyone with views that outrageous. And this seems highly
implausible. But that's the only reason that premise 2 could be false.
So we should accept premise 2.

\section{A Story about Disagreement}\label{a-story-about-disagreement}

The tricky part of the argument is proving premise 3. To do this, I'll
use a story involving four friends, Apollo, Telemachus, Adam and Tom.
The day before our story takes place, Adam has convinced Apollo that he
should believe EW, and organise his life around it. Now Apollo and
Telemachus are on their way to Fenway Park to watch the Red Sox play the
Indians. There have been rumours flying around all day about whether the
Red Sox injured star player, David Ortiz, will be healthy enough to
play. Apollo and Telemachus have heard all the competing reports, and
are comparing their credences that Ortiz will play. (Call the
proposition that he will play \emph{p}.) Apollo's credence in \emph{p}
is 0.7, and Telemachus's is 0.3. In fact, 0.7 is the rational credence
in p given their shared evidence, and Apollo truly believes that it
is.\footnote{This is obviously somewhat of an idealisation, since there
  won't usually be a unique precise rational response to the evidence.
  But I don't think this idealisation hurts the argument to follow. I
  should note that the evidence here \emph{excludes} their statements of
  their credences, so I really mean the evidence that they brought to
  bear on the debate over whether \emph{p}.} And, as it turns out, the
Red Sox have decided but not announced that Ortiz will play, so \emph{p}
is true.

Despite these facts, Apollo lowers his credence in \emph{p}. In accord
with his newfound belief in EW, he changes his credence in \emph{p} to
0.5. Apollo is sure, after all, that when it comes to baseball
Telemachus is an epistemic peer. At this point Tom arrives, and with a
slight disregard for the important baseball game at hand, starts trying
to convince them of the right reasons view on disagreement. Apollo is
not convinced, but Telemachus thinks it sounds right. As he puts it, the
view merely says that the rational person believes what the rational
person believes. And who could disagree with that?

Apollo is not convinced, and starts telling them the virtues of EW. But
a little way in, Tom cuts him off with a question. ``How probable,'' he
asks Apollo, ``does something have to be before you'll assert it?''

Apollo says that it has to be fairly probable, though just what the
threshold is depends on just what issues are at stake. But he agrees
that it has to be fairly high, well above 0.5 at least.

``Well,'' says Tom, ``in that case you shouldn't be defending EW in
public. Because you think that Telemachus and I are the epistemic peers
of you and Adam. And we think EW is false. So even by EW's own lights,
the probability you assign to EW should be 0.5. And that's not a high
enough probability to assert it.'' Tom's speech requires that Apollo
regard he and Telemachus as Apollo's epistemic peers with regard to this
question. By premises 1 and 2, Apollo should do this, and we'll assume
that he does.

So Apollo agrees with all this, and agrees that he shouldn't assert EW
any more. But he still plans to use it, i.e.~to have a credence in
\emph{p} of 0.5 rather than 0.7. But now Telemachus and Tom press on him
the following analogy.

Imagine that there were two competing experts, each of whom gave
differing views about the probability of \emph{q}. One of the experts,
call her Emma, said that the probability of \emph{q}, given the
evidence, is 0.5. The other expert, call her Rae, said that the
probability of \emph{q}, given the evidence, is 0.7. Assuming that
Apollo has the same evidence as the experts, but he regards the experts
as experts at evaluating evidence, what should his credence in \emph{q}
be? It seems plausible that it should be a weighted average of what Emma
says and what Rae says. In particular, it should be 0.5 only if Apollo
is maximally confident that Emma is the expert to trust, and not at all
confident that Rae is the expert to trust.

The situation is parallel to the one Apollo actually faces. EW says that
his credence in \emph{p} should be 0.5. The right reason view says that
his credence in \emph{p} should be 0.7. Apollo is aware of both of these
facts. So his credence in \emph{p} should be 0.5 iff he is certain that
EW is the theory to trust, just as his credence in \emph{q} should be
0.5 iff he is certain that Emma is the expert to trust. Indeed, a
credence of 0.5 in \emph{p} is incoherent unless Apollo is certain EW is
the theory to trust. But Apollo is not at all certain of this. His
credence in EW, as is required by EW itself, is 0.5. So as long as
Apollo keeps his credence in p at 0.5, he is being incoherent. But EW
says to keep his credence in p at 0.5. So EW advises him to be
incoherent. That is, EW offers incoherent advice. We can state this more
carefully in an argument.

\begin{enumerate}
\def\labelenumi{\arabic{enumi}.}
\setcounter{enumi}{4}
\tightlist
\item
  EW says that Apollo's credence in p should be 0.5.
\item
  If 5, then EW offers incoherent advice unless it also says that
  Apollo's credence in EW should be 1.
\item
  EW says that Apollo's credence in EW should be 0.5.
\item
  So, EW offers incoherent advice.
\end{enumerate}

Since Apollo's case is easily generalisable, we can infer that in a
large number of cases, EW offers advice that is incoherent. Line 7 in
this argument is hard to assail given premises 1 and 2 of the master
argument. But I can imagine objections to each of the other lines.

\emph{Objection}: Line 6 is false. Apollo can coherently have one
credence in p while being unsure about whether it is the rational
credence to have. In particular, he can coherently have his credence in
p be 0.5, while he is unsure whether his credence in p should be 0.5 or
0.7. In general there is no requirement for agents who are not
omniscient to have their credences match their judgments of what their
credences should be.

\emph{Replies}: I have two replies to this, the first dialectical and
the second substantive.

The dialectical reply is that if the objector's position on coherence is
accepted, then a lot of the motivation for EW fades away. A core idea
behind EW is that Apollo was unsure before the conversation started
whether he or Telemachus would have the most rational reaction to the
evidence, and hearing what each of them say does not provide him with
more evidence. (See the `bootstrapping' argument in Elga
(\citeproc{ref-Elga2007-ELGRAD}{2007}) for a more formal statement of
this idea.) So Apollo should have equal credence in the rationality of
his judgment and of Telemachus's judgment.

But if the objector is correct, Apollo can do that without changing his
view on EW one bit. He can, indeed should, have his credence in \emph{p}
be 0.7, while being uncertain whether his credence in p should be 0.7
(as he thinks) or 0.3 (as Telemachus thinks). Without some principle
connecting what Apollo should think about what he should think to what
Apollo should think, it is hard to see why this is not the uniquely
rational reaction to Apollo's circumstances. In other words, if this is
an objection to my argument against EW, it is just as good an objection
to a core argument for EW.

The substantive argument is that the objector's position requires
violating some very weak principles concerning rationality and
higher-order beliefs. The objector is right that, for instance, in order
to justifiably believe that \emph{p} (to degree \(d\)), one need not
know, or even believe, that one is justified in believing \emph{p} (to
that degree). If nothing else, the anti-luminosity arguments in
Williamson (\citeproc{ref-Williamson2000-WILKAI}{2000}) show that to be
the case. But there are weaker principles that are more plausible, and
which the objector's position has us violate. In particular, there is
the view that we can't both be justified in believing that \emph{p} (to
degree \(d\)), while we know we are not justified in believing that we
are justified in believing \emph{p} (to that degree). In symbols, if we
let \(Jp\) mean that the agent is justified in believing \emph{p}, and
box and diamond to be epistemic modals, we have the principle
\textbf{MJ} (for Might be Justified).

\begin{description}
\tightlist
\item[MJ]
\emph{Jp} → ◇\emph{JJp}
\end{description}

This seems like a much more plausible principle, since if we know we
aren't justified in believing we're justified in believing \emph{p}, it
seems like we should at least suspend judgment in \emph{p}. That is, we
shouldn't believe \emph{p}. That is, we aren't justified in believing
\emph{p}. But the objector's position violates principle \textbf{MJ}, or
at least a probabilistic version of it, as we'll now show.

We aim to prove that the objector is committed to Apollo being justified
in believing \emph{p} to degree 0.5, while he knows he is not justified
in believing he is justified in believing \emph{p} to degree 0.5. The
first part is trivial, it's just a restatement of the objector's view,
so it is the second part that we must be concerned with.

Now, either EW is true, or it isn't true. If it is true, then Apollo is
not justified in having a greater credence in it than 0.5. But his only
justification for believing p to degree 0.5 is EW. He's only justified
in believing he's justified in believing \emph{p} if he can justify his
use of EW in it. But you can't justify a premise in which your rational
credence is 0.5. So Apollo isn't justified in believing he is justified
in believing \emph{p}. If EW isn't true, then Apollo isn't even
justified in believing \emph{p} to degree 0.5. And he knows this, since
he knows EW is his only justification for lowering his credence in
\emph{p} that far. So he certainly isn't justified in believing he is
justified in believing \emph{p} to degree 0.5 Moreover, every premise in
this argument has been a premise that Apollo knows to obtain, and he is
capable of following all the reasoning. So he knows that he isn't
justified in believing he is justified in believing \emph{p} to degree
0.5, as required.

The two replies I've offered to the objector complement one another. If
someone accepts \textbf{MJ}, then they'll regard the objector's position
as incoherent, since we've just shown that \textbf{MJ} is inconsistent
with that position. If, on the other hand, someone rejects \textbf{MJ}
and everything like it, then they have little reason to accept EW in the
first place. They should just accept that Apollo's credence in p should
be, as per hypothesis the evidence suggests, 0.7. The fact that an
epistemic peer disagrees, in the face of the same evidence, might give
Apollo reason to doubt that this is in fact that uniquely rational
response to the evidence. But, unless we accept a principle like
\textbf{MJ}, that's consistent with Apollo retaining the rational
response to the evidence, namely a credence of 0.7 in p.~So it is hard
to see how someone could accept the objector's argument, while also
being motivated to accept EW. In any case, I think \textbf{MJ} is
plausible enough on its own to undermine the objector's
position.\footnote{Added in 2010: I still think there's a dilemma here
  for EW, but I'm less convinced than I used to be that \textbf{MJ} is
  correct.}

\emph{Objection}: Line 5 is false. Once we've seen that the credence of
EW is 0.5, then Apollo's credence in first-order claims such as p
should, as the analogy with q suggests, be a weighted average of what EW
says it should be, and what the right reason view says it should be. So,
even by EW's own lights, Apollo's credence in p should be 0.6.

\emph{Replies}: Again I have a dialectical reply, and a substantive
reply.

The dialectical reply is that once we make this move, we really have
very little motivation to accept EW. There is, I'll grant, some
intuitive plausibility to the view that when faced with a disagreeing
peer, we should think the right response is half way between our
competing views. But there is no intuitive plausibility whatsoever to
the view that in such a situation, we should naturally move to a
position three-quarters of the way between the two competing views, as
this objector suggests. Much of the argument for EW, especially in
Christensen, turns on intuitions about cases, and the objector would
have us give all of that up. Without those intuitions, however, EW falls
in a heap.

The substantive reply is that the idea behind the objection can't be
coherently sustained. The idea is that we should first apply EW to
philosophical questions to work out the probability of different
theories of disagreement, and then apply those probabilities to
first-order disagreements. The hope is that in doing so we'll reach a
stable point at which EW can be coherently applied. But there is no such
stable point. Consider the following series of questions.

\begin{description}
\tightlist
\item[Q1]
Is EW true?
\end{description}

Two participants say yes, two say no. We have a dispute, leading to our
next question.

\begin{description}
\tightlist
\item[Q2]
What is the right reaction to the disagreement over Q1?
\end{description}

EW answers this by saying our credence in EW should be 0.5. But that's
not what the right reason proponents say. They don't believe EW, so they
have no reason to move their credence in EW away from 0. So we have
another dispute, and we can ask

\begin{description}
\tightlist
\item[Q3]
What is the right reaction to the disagreement over Q2?
\end{description}

EW presumably says that we should again split the difference. Our
credence in EW might now be 0.25, half-way between the 0.5 it was after
considering Q2, and what the right reasons folks say. But, again, those
who don't buy EW will disagree, and won't be moved to adjust their
credence in EW. So again there's a dispute, and again we can ask

\begin{description}
\tightlist
\item[Q4]
What is the right reaction to the disagreement over Q3?
\end{description}

This could go on for a while. The only `stable point' in the sequence is
when we assign a credence of 0 to EW. That's to say, the only way to
coherently defend the idea behind the objection is to assign credence 0
to EW. But that's to give up on EW. As with the previous objection, we
can't hold on to EW and object to the argument.

\section{Summing Up}\label{summing-up}

The story I've told here is a little idealised, but otherwise common
enough. We often have disagreements both about first-order questions,
and about how to resolve this disagreement. In these cases, there is no
coherent way to assign equal weight to all prima facie rational views
both about the first order question and the second order,
epistemological, question. The only way to coherently apply EW to all
first order questions is to put our foot down, and say that despite the
apparent intelligence of our philosophical interlocutors, we're not
letting them dim our credence in EW. But if we are prepared to put our
foot down here, why not about some first-order question or other? It
certainly isn't because we have more reason to believe an
epistemological theory like EW than we have to believe first order
theories about which there is substantive disagreement. So perhaps we
should hold on to those theories, and let go of EW.

\section*{Afterthoughts}\label{afterthoughts}
\addcontentsline{toc}{section}{Afterthoughts}

I now think that the kind of argument I presented in the 2007 paper is
not really an argument against EW as such, but an argument against one
possible motivation for EW. I also think that alternate motivations for
EW are no good, so I still think it is an important argument. But I
think it's role in the dialectic is a little more complicated than I
appreciated back then.

Much of my thinking about disagreement problems revolves around the
following table. The idea behind the table, and much of the related
argument, is due to Thomas Kelly
(\citeproc{ref-Kelly2010-KELPDA}{2010}). In the table, \emph{S} and
\emph{T} antecedently had good reasons to take themselves to be
epistemic peers, and they know that their judgments about \emph{p} are
both based on \emph{E}. In fact, \emph{E} is excellent evidence for
\emph{p}, but only \emph{S} judges that \emph{p}; \emph{T} judges that
¬\emph{p}. Now let's look at what seems to be the available evidence for
and against \emph{p}.

\begin{longtable}[]{@{}cc@{}}
\toprule\noalign{}
\endhead
\bottomrule\noalign{}
\endlastfoot
\textbf{Evidence for} \emph{p} & \textbf{Evidence against} \emph{p} \\
\emph{S}'s judgment that \emph{p} & \emph{T}'s judgment that
¬\emph{p} \\
\emph{E} & \\
\end{longtable}

Now that doesn't look to me like a table where the evidence is equally
balanced for and against \emph{p}. Even granting that the judgments are
evidence over and above \emph{E}, and granting that how much weight we
should give to judgments should track our \emph{ex ante} judgments of
their reliability rather than our \emph{ex post} judgments of their
reliability, both of which strike me as false but necessary premises for
EW, it \emph{still} looks like there is more evidence for \emph{p} than
against \emph{p}.\footnote{By \emph{ex ante} and \emph{ex post} I mean
  before and after we learn about \emph{S} and \emph{T}'s use of
  \emph{E} to make a judgment about \emph{p}. I think that should change
  how reliable we take \emph{S} and \emph{T} to be, and that this should
  matter to what use, if any, we put their judgments, but it is crucial
  to EW that we ignore this evidence. Or, at least, it is crucial to EW
  that \emph{S} and \emph{T} ignore this evidence.} There is strictly
more evidence for \emph{p} than against it, since \emph{E} exists. If we
want to conclude that \emph{S} should regard \emph{p} and ¬\emph{p} as
equally well supported for someone in her circumstance, we have to show
that the table is somehow wrong. I know of three possible moves the EW
defender could make here.

David Christensen (\citeproc{ref-Christensen2010-CHRDQB}{2011}), as I
read him, says that the table is wrong because when we are representing
the evidence \emph{S} has, we should not include her own judgment.
There's something plausible to this. Pretend for a second that \emph{T}
doesn't exist, so it's clearly rational for \emph{S} to judge that
\emph{p}. It would still be wrong of \emph{S} to say, ``Since \emph{E}
is true, \emph{p}. And I judged that \emph{p}, so that's another reason
to believe that \emph{p}, because I'm smart.'' By hypothesis, \emph{S}
is smart, and that smart people judge things is reason to believe those
things are true. But this doesn't work when the judgment is one's own.
This is something that needs explaining in a full theory of the
epistemic significance of judgment, but let's just take it as a given
for now.\footnote{My explanation is that evidence screens any judgments
  made on the basis of that evidence, in the sense of screening to be
  described below.} Now the table, or at least the table as is relevant
to \emph{S}, looks as follows.

\begin{longtable}[]{@{}cc@{}}
\toprule\noalign{}
\endhead
\bottomrule\noalign{}
\endlastfoot
\textbf{Evidence for} \emph{p} & \textbf{Evidence against} \emph{p} \\
\emph{E} & \emph{T}'s judgment that ¬\emph{p} \\
\end{longtable}

But I don't think this does enough to support EW, or really anything
like it. First, it won't be true in general that the two sides of this
table balance. In many cases, \emph{E} is strong evidence for \emph{p},
and \emph{T}'s judgment won't be particularly strong evidence against
\emph{p}. In fact, I'd say the kind of case where \emph{E} is much
better evidence for \emph{p} than \emph{T}'s judgment is against
\emph{p} is the statistically normal kind. Or, at least, it is the
normal kind of case modulo the assumption that \emph{S} and \emph{T}
have the same evidence. In cases where that isn't true, learning that
\emph{T} thinks ¬\emph{p} is good evidence that \emph{T} has evidence
against \emph{p} that you don't have, and you should adjust accordingly.
But by hypothesis, \emph{S} knows that isn't the case here. So I don't
see why this should push us even close to taking \emph{p} and ¬\emph{p}
to be equally well supported.

The other difficulty for defending EW by this approach is that it seems
to undermine the original motivations for the view. As Christensen
notes, the table above is specifically for \emph{S}. Here's what the
table looks like for \emph{T}.

\begin{longtable}[]{@{}cc@{}}
\toprule\noalign{}
\endhead
\bottomrule\noalign{}
\endlastfoot
\textbf{Evidence for} \emph{p} & \textbf{Evidence against} \emph{p} \\
\emph{S}'s judgment that \emph{p} & \\
\emph{E} & \\
\end{longtable}

It's no contest! So \emph{T} should firmly believe \emph{p}. But that
isn't the intuition anyone gets, as far as I can tell, in any of the
cases motivating EW. And the big motivation for EW comes from intuitions
about cases. Once we acknowledge that these intuitions are unreliable,
as we'd have to do if we were defending EW this way, we seem to lack any
reason to accept EW.

The second approach to blocking the table is to say that \emph{T}'s
judgment is an undercutting defeater for the support \emph{E} provides
for \emph{p}. This looks superficially promising. Having a smart person
say that your evidence supports something other than you thought it did
seems like it could be an undercutting defeater, since it is a reason to
think the evidence supports something else, and hence doesn't support
what you thought it does. And, of course, if \emph{E} is undercut, then
the table just has one line on it, and the two sides look equal.

But it doesn't seem like it can work in general, for a reason that Kelly
(\citeproc{ref-Kelly2010-KELPDA}{2010}) makes clear. We haven't said
what \emph{E} is so far. Let's start with a case where \emph{E} consists
of the judgments of a million other very smart people that \emph{p}.
Then no one, not even the EW theorist, will think that \emph{T}'s
judgment undercuts the support \emph{E} provides to \emph{p}. Indeed,
even if \emph{E} just consists of one other person's judgment, it won't
be undercut by \emph{T}'s judgment. The natural thought for an
EW-friendly person to have in that case is that since there are two
people who think \emph{p}, and one who thinks ¬\emph{p}, then \emph{S}'s
credence in \emph{p} should be ⅔. But that's impossible if \emph{E},
i.e., the third person's judgment, is undercut by \emph{T}'s judgment.
It's true that \emph{T}'s judgment will partially \emph{rebut} the
judgments that \emph{S}, and the third party, make. It will move the
probability of \emph{p}, at least according to EW, from 1 to ⅔. But that
evidence won't be in any way \emph{undercut}.

And as Kelly points out, evidence is pretty fungible. Whatever support
\emph{p} gets from other people's judgments, it could get very similar
support from something other than a judgment. We get roughly the same
evidence for \emph{p} by learning that a smart person predicts \emph{p}
as learning that a successful computer model predicts \emph{p}. So the
following argument looks sound to me.

\begin{enumerate}
\def\labelenumi{\arabic{enumi}.}
\tightlist
\item
  When \emph{E} consists of other people's judgments, the support it
  provides to \emph{p} is not undercut by \emph{T}'s judgment.
\item
  If the evidence provided by other people's judgments is not undercut
  by \emph{T}'s judgment, then some non-judgmental evidence is not
  undercut by \emph{T}'s judgment.
\item
  So, not all non-judgmental evidence is not undercut by \emph{T}'s
  judgment.
\end{enumerate}

So it isn't true in general that the table is wrong because \emph{E} has
been defeated by an undercutting defeater.

There's another problem with the defeat model in cases where the initial
judgments are not full beliefs. Change the case so \emph{E} provides
basically no support to either \emph{p} or ¬\emph{p}. In fact, \emph{E}
is just irrelevant to \emph{p}, and the agent's have nothing to base
either a firm or a probabilistic judgment about \emph{p} on. For this
reason, \emph{S} declines to form a judgment, but \emph{T} forms a firm
judgment that \emph{p}. Moreover, although both \emph{S} and \emph{T}
are peers, that's because they are both equally poor at making judgments
about cases like \emph{p}. Here's the table then:

\begin{longtable}[]{@{}cc@{}}
\toprule\noalign{}
\endhead
\bottomrule\noalign{}
\endlastfoot
\textbf{Evidence for} \emph{p} & \textbf{Evidence against} \emph{p} \\
\emph{T}'s judgment that \emph{p} & \\
\end{longtable}

Since \emph{E} is irrelevant, it doesn't appear, either before or after
we think about defeaters. And since \emph{T} is not very competent,
that's not great evidence for \emph{p}. But EW says that \emph{S} should
`split the difference' between her initial agnositicism, and \emph{T}'s
firm belief in \emph{p}. I don't see how that could be justified by
\emph{S}'s evidence.

So that move doesn't work either, and we're left with the third option
for upsetting the table. This move is, I think, the most promising of
the lot. It is to say that \emph{S}'s own judgment \emph{screens off}
the evidence that \emph{E} provides. So the table is misleading, because
it `double counts' evidence.

The idea of screening I'm using here, at least on behalf of EW, comes
from Reichenbach's \emph{The Direction of Time}, and in particular from
his work on deriving a principle that lets us infer events have a common
cause. The notion was originally introduced in probabilistic terms. We
say that \emph{C} screens off the positive correlation between \emph{B}
and \emph{A} if the following two conditions are met.

\begin{enumerate}
\def\labelenumi{\arabic{enumi}.}
\tightlist
\item
  \emph{A} and \emph{B} are positively correlated probabilistically,
  i.e.~Pr(\emph{A} \textbar{} \emph{B}) \textgreater{} Pr(\emph{A}).
\item
  Given \emph{C}, \emph{A} and \emph{B} are probabilistically
  independent,\\
  i.e.~Pr(\emph{A} \textbar{} \emph{B} ∧ \emph{C}) = Pr(\emph{A}
  \textbar{} \emph{C}).
\end{enumerate}

I'm interested in an evidential version of screening. If we have a
probabilistic analysis of evidential support, the version of screening
I'm going to offer here is identical to the Reichenbachian version just
provided. But I want to stay neutral on whether we should think of
evidence probabilistically.\footnote{In general I'm sceptical of always
  treating evidence probabilistically. Some of my reasons for scepticism
  are in Weatherson (\citeproc{ref-Weatherson2007}{2007}).} When I say
that \emph{C} screens off the evidential support that \emph{B} provides
to \emph{A}, I mean the following. (Both these clauses, as well as the
statement that \emph{C} screens off \emph{B} from \emph{A}, are made
relative to an evidential background. I'll leave that as tacit in what
follows.)

\begin{enumerate}
\def\labelenumi{\arabic{enumi}.}
\tightlist
\item
  \emph{B} is evidence that \emph{A}.
\item
  \emph{B} ∧ \emph{C} is no better evidence that \emph{A} than \emph{C}
  is.\footnote{Branden Fitelson pointed out to me that the probabilistic
    version entails one extra condition, namely that ¬\emph{B} ∧
    \emph{C} is no worse evidence for \emph{A} than \emph{C} is. But I
    think that extra condition is irrelevant to disagreement debates, so
    I'm leaving it out.}
\end{enumerate}

Here is one stylised example of where screening helps conceptualise
things. Detective Det is trying to figure out whether suspect Sus
committed a certain crime. Let \emph{A} be that Sus is guilty, \emph{B}
be that Sus's was seen near the crime scene near the time the crime was
committed, and \emph{C} be that Sus was at the crime scene when the
crime was committed. Then both clauses are satisfied. \emph{B} is
evidence for \emph{A}; that's why we look for witnesses who place the
suspect near the crime scene. But given the further evidence \emph{C},
then \emph{B} is neither here nor there with respect to \emph{A}. We're
only interested in finding if Sus was near the crime scene because we
want to know whether he was at the crime scene. If we know that he was
there, then learning he was seen near there doesn't move the
investigation along. So both clauses of the definition of screening are
satisfied.

When there is screened evidence, there is the potential for double
counting. It would be wrong to say that if we know \emph{B} ∧ \emph{C}
we have two pieces of evidence against Sus. Similarly, if a judgment
screens off the evidence it is based on, then the table `double counts'
the evidence for \emph{p}. Removing the double counting, by removing
\emph{E}, makes the table symmetrical. And that's just what EW needs.

So the hypothesis that judgments screen the evidence they are based on,
or JSE for short, can help EW respond to the argument from this table.
But JSE is vulnerable to regress arguments. I now think that the
argument in `Disagreeing about Disagreement' is a version of the regress
argument against JSE. So really it's an argument against the most
promising response to a particularly threatening argument against EW.

Unfortunately for EW, those regress arguments are actually quite good.
To see ths, let's say an agent makes a judgment on the basis of
\emph{E}, and let \emph{J} be the proposition that that judgment was
made. JSE says that \emph{E} is now screened off, and the agent's
evidence is just \emph{J}. But with that evidence, the agent presumably
makes a new judgment. Let \emph{J}′ be the proposition that that
judgment was made. We might ask now, does \emph{J}′ sit alongside
\emph{J} as extra evidence, is it screened off by \emph{J}, or does it
screen off \emph{J}? The picture behind JSE, the picture that says that
judgments on the basis of some evidence screen that evidence, suggest
that \emph{J}′ should in turn screen \emph{J}. But now it seems we have
a regress on our hands. By the same token, \emph{J}\(\prime \prime\),
the proposition concerning the new judgment made on the basis of
\emph{J}′, should screen off \emph{J}′, and the proposition
\emph{J}\(\prime \prime \prime\) about the fourth judgment made, should
screen off J\(\prime \prime\), and so on. The poor agent has no
unscreened evidence left! Something has gone horribly wrong.

I think this regress is ultimately fatal for JSE. But to see this, we
need to work through the possible responses that a defender of JSE could
make. There are really just two moves that seem viable. One is to say
that the regress does not get going, because \emph{J} is better evidence
than \emph{J}′, and perhaps screens it. The other is to say that the
regress is not vicious, because all these judgments should agree in
their content. I'll end the paper by addressing these two responses.

The first way to avoid the regress is to say that there is something
special about the first level. So although \emph{J} screens \emph{E}, it
isn't the case that \emph{J}′ screens \emph{J}. That way, the regress
doesn't start. This kind of move is structurally like the move Adam Elga
(\citeproc{ref-Elga2010-ELGHTD}{2010}) has recently suggested. He argues
that we should adjust our views about first-order matters in (partial)
deference to our peers, but we shouldn't adjust our views about the
right response to disagreement in this way.

It's hard to see what could motivate such a position, either about
disagreement or about screening. It's true that we need some kind of
stopping point to avoid these regresses. But the most natural stopping
point is the very first level. Consider a toy example. It's common
knowledge that there are two apples and two oranges in the basket, and
no other fruit. (And that no apple is an orange.) Two people disagree
about how many pieces of fruit there are in the basket. \emph{A} thinks
there are four, \emph{B} thinks there are five, and both of them are
equally confident. Two other people, \emph{C} and \(D\), disagree about
what \emph{A} and \emph{B} should do in the face of this disagreement.
All four people regard each other as peers. Let's say \emph{C}'s
position is the correct one (whatever that is) and \(D\)'s position is
incorrect. Elga's position is that \emph{A} should partially defer to
\emph{B}, but \emph{C} should not defer to \(D\). This is, intuitively,
just back to front. \emph{A} has evidence that immediately and obviously
entails the correctness of her position. \emph{C} is making a
complicated judgment about a philosophical question where there are
plausible and intricate arguments on each side. The position \emph{C} is
in is much more like the kind of case where experience suggests a
measure of modesty and deference can lead us away from foolish errors.
If anyone should be sticking to their guns here, it is \emph{A}, not
\emph{C}.

The same thing happens when it comes to screening. Let's say that
\emph{A} has some evidence that (a) she has made some mistakes on simple
sums in the past, but (b) tends to massively over-estimate the
likelihood that she's made a mistake on any given puzzle. What should
she do? One option, in my view the correct one, is that she should
believe that there are four pieces of fruit in the basket, because
that's what the evidence obviously entails. Another option is that she
should be not very confident there are four pieces of fruit in the
basket, because she makes mistakes on these kinds of sums. Yet another
option is that she should be pretty confident (if not completely
certain) that there are four pieces of fruit in the basket, because if
she were not very confident about this, this would just be a
manifestation of her over-estimation of her tendency to err. The
`solution' to the regress we're considering here says that the second of
these three reactions is the uniquely rational reaction. The idea behind
the solution is that we should respond to the evidence provided by
first-order judgments, and correct that judgment for our known biases,
but that we shouldn't in turn correct for the flaws in our
self-correcting routine. I don't see what could motivate such a
position. Either we just rationally respond to the evidence, and in this
case just believe there are four pieces of fruit in the basket, or we
keep correcting for errors we make in any judgment. It's true that the
latter plan leads either to regress or to the kind of ratificationism
we're about to critically examine. But that's not because the
disjunction is false, it's because the first disjunct is true.

A more promising way to avoid the regress is suggested by some other
work of Elga's, in this case a paper he co-wrote with Andy Egan
(\citeproc{ref-Egan2005-EGAICB}{Egan and Elga 2005}). Their idea, as I
understand them, is that for any rational agent, any judgment they make
must be such that when they add the fact that they made that judgment to
their evidence (or, perhaps better given JSE, replace their evidence
with the fact that they made that judgment), the rational judgment to
make given the new evidence has the same content as the original
judgment. So if you're rational, and you come to believe that \emph{p}
is likely true, then the rational thing to believe given you've made
that judgment is that \emph{p} is likely true.

Note that this isn't as strong a requirement as it may first seem. The
requirement is not that any time an agent makes a judgment, rationality
requires that they say on reflection that it is the correct judgments.
Rather, the requirement is that the only judgments rational agents make
are those judgments that, on reflection, she would reflectively endorse.
We can think of this as a kind of ratifiability constraint on judgment,
like the ratifiability constraint on decision making that Richard
Jeffrey uses to handle Newcomb cases Jeffrey
(\citeproc{ref-JeffreyLogicOfDecision}{1983}).

To be a little more precise, a judgment is ratifiable for agent \emph{S}
just in case the rational judgment for \emph{S} to make conditional on
her having made that judgment has the same content as the original
judgment. The thought then is that we avoid the regress by saying
rational agents always make ratifiable judgments. If the agent does do
that, there isn't much of a problem with the regress; once she gets to
the first level, she has a stable view, even once she reflects on it.

It seems to me that this assumption, that only ratifiable judgments are
rational, is what drives most of the arguments in Egan and Elga's paper
on self-confidence, so I don't think this is a straw-man move. Indeed,
as the comparison to Jeffrey suggests, it has some motivation behind it.
Nevertheless it is false. I'll first note one puzzling feature of the
view, then one clearly false implication of the view.

The puzzling feature is that in some cases there may be nothing we can
rationally do which is ratifiable. One way this can happen involves a
slight modification of Egan and Elga's example of the
directionaly-challenged driver. Imagine that when I'm trying to decide
whether \emph{p}, for any \emph{p} in a certain field, I know (a) that
whatever judgment I make will usually be wrong, and (b) if I conclude my
deliberations without making a judgment, then \emph{p} is usually true.
If we also assume JSE, then it follows there is no way for me to end
deliberation. If I make a judgment, I will have to retract it because of
(a). But if I think of ending deliberation, then because of (b) I'll
have excellent evidence that \emph{p}, and it would be irrational to
ignore this evidence. (Nicholas Silins (\citeproc{ref-Silins2005}{2005})
has used the idea that failing to make a judgment can be irrational in a
number of places, and those arguments motivated this example.)

This is puzzling, but not obviously false. It is plausible that there
are some epistemic dilemmas, where any position an agent takes is going
to be irrational. (By that, I mean it is at least as plausible that
there are epistemic dilemmas as that there are moral dilemmas, and I
think the plausibility of moral dilemmas is reasonably high.) That a
case like the one I've described in the previous paragraph is a dilemma
is perhaps odd, but no reason to reject the theory.

The real problem, I think, for the ratifiability proposal is that there
are cases where unratifiable judgments are clearly preferable to
ratifiable judgments. Assume that I'm a reasonably good judge of what's
likely to happen in baseball games, but I'm a little over-confident. And
I know I'm over-confident. So the rational credence, given some
evidence, is usually a little closer to ½ than I admit. At risk of being
arbitrarily precise, let's say that if \emph{p} concerns a baseball
game, and my credence in \emph{p} is \emph{x}, the rational credence in
\emph{p}, call it \emph{y}, for someone with no other information than
this is given by:

\[
y = x + \frac{sin(2\pi x)}{50}
\]

To give you a graphical sense of how that looks, the dark line in this
graph is \emph{y}, and the lighter diagonal line is \emph{y} = \emph{x}.

\begin{figure}[H]

{\centering \pandocbounded{\includegraphics[keepaspectratio]{sinewave.JPG}}

}

\caption{A sine wave around \emph{x}=\emph{y}}

\end{figure}%

Note that the two lines intersect at three points: (0, 0), (½, ½) and
(1, 1). So if my credence in \emph{p} is either 0, ½ or 1, then my
judgment is ratifiable. Otherwise, it is not. So the ratifiability
constraint says that for any \emph{p} about a baseball game, my credence
in \emph{p} should be either 0, ½ or 1. But that's crazy. It's easy to
imagine that I know (a) that in a particular game, the home team is much
stronger than the away team, (b) that the stronger team usually, but far
from always, wins baseball games, and (c) I'm systematically a little
over-confident about my judgments about baseball games, in the way just
described. In such a case, my credence that the home team will win
should be high, but less than 1. That's just what the ratificationist
denies is possible.

This kind of case proves that it isn't always rational to have
ratifiable credences. It would take us too far afield to discuss this in
detail, but it is interesting to think about the comparison between the
kind of case I just discussed, and the objections to backwards induction
reasoning in decision problems that have been made by Pettit and Sugden
(\citeproc{ref-Pettit1989-PETTBI}{1989}), and by Stalnaker
(\citeproc{ref-Stalnaker1996}{1996}, \citeproc{ref-Stalnaker1998}{1998},
\citeproc{ref-Stalnaker1999}{1999}). The backwards induction reasoning
they criticise is, I think, a development of the idea that decisions
should be ratifiable. And the clearest examples of when that reasoning
fails concern cases where there is a unique ratifiable decision, and it
is guaranteed to be one of the worst possible outcomes. The example I
described in the last few paragraphs has, quite intentionally, a similar
structure.

The upshot of all this is that I think these regress arguments work.
They aren't, I think, directly an argument against EW. What they are is
an argument against the most promising way the EW theorist has for
arguing that the table I started with misstates \emph{S}'s epistemic
situation. Given that the regress argument against JSE works though, I
don't see any way of rescuing EW from this argument.

\subsection*{References}\label{references}
\addcontentsline{toc}{subsection}{References}

\phantomsection\label{refs}
\begin{CSLReferences}{1}{0}
\bibitem[\citeproctext]{ref-Christensen2007-CHREOD}
Christensen, David. 2007. {``Epistemology of Disagreement: The Good
News.''} \emph{Philosophical Review} 116 (2): 187--217. doi:
\href{https://doi.org/10.1215/00318108-2006-035}{10.1215/00318108-2006-035}.

\bibitem[\citeproctext]{ref-Christensen2010-CHRDQB}
---------. 2011. {``Disagreement, Question-Begging and Epistemic
Self-Criticism.''} \emph{Philosophers' Imprint} 11 (6): 1--22.
\url{http://hdl.handle.net/2027/spo.3521354.0011.006}.

\bibitem[\citeproctext]{ref-Egan2005-EGAICB}
Egan, Andy, and Adam Elga. 2005. {``{I Can't Believe I'm Stupid}.''}
\emph{Philosophical Perspectives} 19 (1): 77--93. doi:
\href{https://doi.org/10.1111/j.1520-8583.2005.00054.x}{10.1111/j.1520-8583.2005.00054.x}.

\bibitem[\citeproctext]{ref-Elga2007-ELGRAD}
Elga, Adam. 2007. {``Reflection and Disagreement.''} \emph{No{û}s} 41
(3): 478--502. doi:
\href{https://doi.org/10.1111/j.1468-0068.2007.00656.x}{10.1111/j.1468-0068.2007.00656.x}.

\bibitem[\citeproctext]{ref-Elga2010-ELGHTD}
---------. 2010. {``How to Disagree about How to Disagree.''} In
\emph{Disagreement}, edited by Ted Warfield and Richard Feldman,
175--87. Oxford: Oxford University Press.

\bibitem[\citeproctext]{ref-Feldman2005-FELRTE}
Feldman, Richard. 2005. {``Respecting the Evidence.''}
\emph{Philosophical Perspectives} 19 (1): 95--119. doi:
\href{https://doi.org/10.1111/j.1520-8583.2005.00055.x}{10.1111/j.1520-8583.2005.00055.x}.

\bibitem[\citeproctext]{ref-Feldman2006-FELEPA}
---------. 2006. {``Epistemological Puzzles about Disagreement.''} In
\emph{Epistemology Futures}, edited by Stephen Cade Hetherington,
216--26. Oxford: Oxford University Press.

\bibitem[\citeproctext]{ref-JeffreyLogicOfDecision}
Jeffrey, Richard C. 1983. \emph{The Logic of Decision}. 2nd ed. Chicago:
University of Chicago Press.

\bibitem[\citeproctext]{ref-Kelly2005-KELTES}
Kelly, Thomas. 2005. {``The Epistemic Significance of Disagreement.''}
\emph{Oxford Studies in Epistemology} 1: 167--96.

\bibitem[\citeproctext]{ref-Kelly2010-KELPDA}
---------. 2010. {``Peer Disagreement and Higher Order Evidence.''} In
\emph{Disagreement}, edited by Ted Warfield and Richard Feldman,
111--74. Oxford: Oxford University Press.

\bibitem[\citeproctext]{ref-Pettit1989-PETTBI}
Pettit, Philip, and Robert Sugden. 1989. {``The Backward Induction
Paradox.''} \emph{Journal of Philosophy} 86 (4): 169--82. doi:
\href{https://doi.org/10.2307/2026960}{10.2307/2026960}.

\bibitem[\citeproctext]{ref-Silins2005}
Silins, Nicholas. 2005. {``Deception and Evidence.''}
\emph{Philosophical Perspectives} 19: 375--404. doi:
\href{https://doi.org/10.1111/j.1520-8583.2005.00066.x}{10.1111/j.1520-8583.2005.00066.x}.

\bibitem[\citeproctext]{ref-Stalnaker1996}
Stalnaker, Robert. 1996. {``Knowledge, Belief and Counterfactual
Reasoning in Games.''} \emph{Economics and Philosophy} 12: 133--63. doi:
\href{https://doi.org/10.1017/S0266267100004132}{10.1017/S0266267100004132}.

\bibitem[\citeproctext]{ref-Stalnaker1998}
---------. 1998. {``Belief Revision in Games: Forward and Backward
Induction.''} \emph{Mathematical Social Sciences} 36 (1): 31--56. doi:
\href{https://doi.org/10.1016/S0165-4896(98)00007-9}{10.1016/S0165-4896(98)00007-9}.

\bibitem[\citeproctext]{ref-Stalnaker1999}
---------. 1999. {``Extensive and Strategic Forms: Games and Models for
Games.''} \emph{Research in Economics} 53 (3): 293--319. doi:
\href{https://doi.org/10.1006/reec.1999.0200}{10.1006/reec.1999.0200}.

\bibitem[\citeproctext]{ref-Weatherson2007}
Weatherson, Brian. 2007. {``The Bayesian and the Dogmatist.''}
\emph{Proceedings of the Aristotelian Society} 107: 169--85. doi:
\href{https://doi.org/10.1111/j.1467-9264.2007.00217.x}{10.1111/j.1467-9264.2007.00217.x}.

\bibitem[\citeproctext]{ref-Wedgwood2007-WEDNON}
Wedgwood, Ralph. 2007. \emph{The Nature of Normativity}. Oxford: Oxford
University Press.

\bibitem[\citeproctext]{ref-Williamson2000-WILKAI}
Williamson, Timothy. 2000. \emph{{Knowledge and its Limits}}. Oxford
University Press.

\end{CSLReferences}



\noindent Published in\emph{
The Epistemology of Disagreement: New Essays}, 2013, pp. 54-77.


\end{document}
