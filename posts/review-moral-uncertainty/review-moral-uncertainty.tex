% Options for packages loaded elsewhere
\PassOptionsToPackage{unicode}{hyperref}
\PassOptionsToPackage{hyphens}{url}
%
\documentclass[
  11pt,
  letterpaper,
  DIV=11,
  numbers=noendperiod,
  twoside]{scrartcl}

\usepackage{amsmath,amssymb}
\usepackage{setspace}
\usepackage{iftex}
\ifPDFTeX
  \usepackage[T1]{fontenc}
  \usepackage[utf8]{inputenc}
  \usepackage{textcomp} % provide euro and other symbols
\else % if luatex or xetex
  \usepackage{unicode-math}
  \defaultfontfeatures{Scale=MatchLowercase}
  \defaultfontfeatures[\rmfamily]{Ligatures=TeX,Scale=1}
\fi
\usepackage{lmodern}
\ifPDFTeX\else  
    % xetex/luatex font selection
    \setmainfont[ItalicFont=EB Garamond Italic,BoldFont=EB Garamond
Bold]{EB Garamond Math}
    \setsansfont[]{EB Garamond}
  \setmathfont[]{Garamond-Math}
\fi
% Use upquote if available, for straight quotes in verbatim environments
\IfFileExists{upquote.sty}{\usepackage{upquote}}{}
\IfFileExists{microtype.sty}{% use microtype if available
  \usepackage[]{microtype}
  \UseMicrotypeSet[protrusion]{basicmath} % disable protrusion for tt fonts
}{}
\usepackage{xcolor}
\usepackage[left=1.1in, right=1in, top=0.8in, bottom=0.8in,
paperheight=9.5in, paperwidth=7in, includemp=TRUE, marginparwidth=0in,
marginparsep=0in]{geometry}
\setlength{\emergencystretch}{3em} % prevent overfull lines
\setcounter{secnumdepth}{3}
% Make \paragraph and \subparagraph free-standing
\makeatletter
\ifx\paragraph\undefined\else
  \let\oldparagraph\paragraph
  \renewcommand{\paragraph}{
    \@ifstar
      \xxxParagraphStar
      \xxxParagraphNoStar
  }
  \newcommand{\xxxParagraphStar}[1]{\oldparagraph*{#1}\mbox{}}
  \newcommand{\xxxParagraphNoStar}[1]{\oldparagraph{#1}\mbox{}}
\fi
\ifx\subparagraph\undefined\else
  \let\oldsubparagraph\subparagraph
  \renewcommand{\subparagraph}{
    \@ifstar
      \xxxSubParagraphStar
      \xxxSubParagraphNoStar
  }
  \newcommand{\xxxSubParagraphStar}[1]{\oldsubparagraph*{#1}\mbox{}}
  \newcommand{\xxxSubParagraphNoStar}[1]{\oldsubparagraph{#1}\mbox{}}
\fi
\makeatother


\providecommand{\tightlist}{%
  \setlength{\itemsep}{0pt}\setlength{\parskip}{0pt}}\usepackage{longtable,booktabs,array}
\usepackage{calc} % for calculating minipage widths
% Correct order of tables after \paragraph or \subparagraph
\usepackage{etoolbox}
\makeatletter
\patchcmd\longtable{\par}{\if@noskipsec\mbox{}\fi\par}{}{}
\makeatother
% Allow footnotes in longtable head/foot
\IfFileExists{footnotehyper.sty}{\usepackage{footnotehyper}}{\usepackage{footnote}}
\makesavenoteenv{longtable}
\usepackage{graphicx}
\makeatletter
\newsavebox\pandoc@box
\newcommand*\pandocbounded[1]{% scales image to fit in text height/width
  \sbox\pandoc@box{#1}%
  \Gscale@div\@tempa{\textheight}{\dimexpr\ht\pandoc@box+\dp\pandoc@box\relax}%
  \Gscale@div\@tempb{\linewidth}{\wd\pandoc@box}%
  \ifdim\@tempb\p@<\@tempa\p@\let\@tempa\@tempb\fi% select the smaller of both
  \ifdim\@tempa\p@<\p@\scalebox{\@tempa}{\usebox\pandoc@box}%
  \else\usebox{\pandoc@box}%
  \fi%
}
% Set default figure placement to htbp
\def\fps@figure{htbp}
\makeatother

\setlength\heavyrulewidth{0ex}
\setlength\lightrulewidth{0ex}
\usepackage[automark]{scrlayer-scrpage}
\clearpairofpagestyles
\cehead{
  Brian Weatherson
  }
\cohead{
  Review of “Moral Uncertainty and Its Consequences”
  }
\ohead{\bfseries \pagemark}
\cfoot{}
\makeatletter
\newcommand*\NoIndentAfterEnv[1]{%
  \AfterEndEnvironment{#1}{\par\@afterindentfalse\@afterheading}}
\makeatother
\NoIndentAfterEnv{itemize}
\NoIndentAfterEnv{enumerate}
\NoIndentAfterEnv{description}
\NoIndentAfterEnv{quote}
\NoIndentAfterEnv{equation}
\NoIndentAfterEnv{longtable}
\NoIndentAfterEnv{abstract}
\renewenvironment{abstract}
 {\vspace{-1.25cm}
 \quotation\small\noindent\emph{Abstract}:}
 {\endquotation}
\newfontfamily\tfont{EB Garamond}
\addtokomafont{disposition}{\rmfamily}
\addtokomafont{title}{\normalfont\itshape}
\let\footnoterule\relax
\KOMAoption{captions}{tableheading}
\makeatletter
\@ifpackageloaded{caption}{}{\usepackage{caption}}
\AtBeginDocument{%
\ifdefined\contentsname
  \renewcommand*\contentsname{Table of contents}
\else
  \newcommand\contentsname{Table of contents}
\fi
\ifdefined\listfigurename
  \renewcommand*\listfigurename{List of Figures}
\else
  \newcommand\listfigurename{List of Figures}
\fi
\ifdefined\listtablename
  \renewcommand*\listtablename{List of Tables}
\else
  \newcommand\listtablename{List of Tables}
\fi
\ifdefined\figurename
  \renewcommand*\figurename{Figure}
\else
  \newcommand\figurename{Figure}
\fi
\ifdefined\tablename
  \renewcommand*\tablename{Table}
\else
  \newcommand\tablename{Table}
\fi
}
\@ifpackageloaded{float}{}{\usepackage{float}}
\floatstyle{ruled}
\@ifundefined{c@chapter}{\newfloat{codelisting}{h}{lop}}{\newfloat{codelisting}{h}{lop}[chapter]}
\floatname{codelisting}{Listing}
\newcommand*\listoflistings{\listof{codelisting}{List of Listings}}
\makeatother
\makeatletter
\makeatother
\makeatletter
\@ifpackageloaded{caption}{}{\usepackage{caption}}
\@ifpackageloaded{subcaption}{}{\usepackage{subcaption}}
\makeatother

\usepackage{bookmark}

\IfFileExists{xurl.sty}{\usepackage{xurl}}{} % add URL line breaks if available
\urlstyle{same} % disable monospaced font for URLs
\hypersetup{
  pdftitle={Review of ``Moral Uncertainty and Its Consequences''},
  pdfauthor={Brian Weatherson},
  hidelinks,
  pdfcreator={LaTeX via pandoc}}


\title{Review of ``Moral Uncertainty and Its Consequences''}
\author{Brian Weatherson}
\date{2002}

\begin{document}
\maketitle
\begin{abstract}
Review of Ted Lockhart, ``Moral Uncertainty and Its Consequences''.
Oxford: Oxford University Press, 2000.
\end{abstract}


\setstretch{1.1}
For many years now, Peter Singer has been arguing that we should not eat
meat, and that we should give more money to famine relief. Many have
been convinced, but many more remain sceptical. However, on one point
most of us would agree: the actions that Singer recommends here are
certainly morally permissible. One rarely feels a twang of moral doubt
when eating tofu curry or writing cheques to Oxfam. Even if we do not
find Singer totally convincing, we may still feel this moral doubt when
eating sirloin, or spending frivolously rather than charitably. If we
accept the main principle in Ted Lockhart's book \emph{Moral Uncertainty
and Its Consequences}, these twangs of moral doubt should be sufficient
to make us amend our behaviour.

The main principle Lockhart endorses is that we should perform actions
that we are maximally confident are morally permissible. We might be
quite confident that having the sirloin is morally permissible, but if
we are not certain, and we are certain the tofu is permissible, we
should stick to tofu. Similarly, if we are certain that large donations
to famine relief are permissible, and not certain that not making these
donations is permissible, the chequebook should come out. The principle
is not just for left-wingers. As Lockhart notes, approvingly, it can
also be used in anti-abortion arguments. In most cases, not having an
abortion is almost certainly permissible. Perhaps there is an exception
for cases of extreme fetal deformity, but not in everyday cases. So if
the woman considering an abortion wants to do the action that is most
probably morally permissible, and has any doubts about the
permissibility of the procedure, she should decline the abortion.

The bulk of Lockhart's book is devoted to case studies where this
principle is deployed, and amendments to the principle generated by
considerations of these cases are adopted. The cases include abortion,
patient confidentiality, Roe v Wade and, briefly, charitable giving. The
theme behind the studies is that even if people cannot come to agreement
on what is morally right, they can come to agreement on what should be
done according to the principle, at least as variously amended, and this
should be sufficient to provide recommendations for action. Lockhart
stresses that if this line of reasoning is correct, then applied
ethicists can provide good advice on practical action without
conclusively resolving apparently intractable ethical problems.

There are three main amendments Lockhart suggests to the principle.
First, he suggests that if moral rightness comes in degrees, we should
maximise the expected moral rightness of our actions, rather than the
probability that we are doing the right thing. Secondly, in situations
where we cannot work out which action maximises expected rightness,
because perhaps we do not have perfect access to the relevant subjective
probabilities, we should choose the action which most probably maximises
expected rightness, or more generally has the highest expected expected
degree of moral rightness. And thirdly, he says that we should maximise
the expected rightness of courses of action, rather than of individual
actions. One might quibble with these amendments, particularly I think
with the second, but they do not seem to affect the core philosophical
issues.

The principle has some rather striking consequences, so striking we
might fear for its refutation by a quick modus tollens. Lockhart, of
course, does not think this is so. He does not discuss the vegetarianism
issue, and endorses the anti-abortion implications, but argues that the
principle need not have such striking implications concerning charitable
giving. He notes that for some people, those who think it probable
enough that substantial charitable giving is a very bad thing to do,
because we have such strong obligations to ourselves and those nearest
and dearest, his principle does not recommend such giving (109).

There is a more direct reason for thinking the principle stands in need
of some further clarification and defence. It is rather unclear what
kind of norm the principle is stating, and hence what force the should
in it is has. Lockhart says it is a norm of rational action, but it
seems in practice to be neither that, nor a moral norm. To see this,
consider the following case where someone clearly does not follow the
principle. While on her way to visit a sick friend in hospital, Jane is
convinced by a fellow subway rider that morality requires an impersonal
concern for the whole world. She is convinced that morality requires
that she not visit her friend, but instead find the patient most in need
of a visitor, and see them. But when she gets to the hospital, her new
moral belief is not strong enough to overcome her desire to visit her
friend in need, which, feeling a little guilty, she does.

Assuming that Jane's newfound moral beliefs are wrong, and that in fact
she did the right thing, what criticisms can we make of her action? Not
that it was immoral, because she did the right thing, visiting her sick
friend, and she acted for the right reason, acting out of care for her
friend. Nor, it seems, that it was prudentially irrational, for she did
what she believed would best satisfy her desires. Perhaps the fact that
her new moral beliefs were not sufficiently motivating indicates a lack
of resolve, or even a weakness of will, but alternatively one might
think that Jane displayed commendable, and virtuous, common sense in not
abandoning her friend precipitously. In any case, I doubt Jane's action
cannot be criticised, even if her resolve can be. Since Jane clearly
violated Lockhart's principle, she did not act in the way she thought
most likely to be morally permissible, but her action seems immune from
criticism, that suggests the principle should not be an action guiding
norm.

One might argue that Jane has a moral responsibility to desire to do the
right thing, and if she had this desire, she would have been rationally
required to not visit her friend. If one believes in such a
responsibility, then one will think that Jane acted against a desire she
should have, that she was, at best, lucky that she did the right thing,
and hence she was irrational. Lockhart compares such agents, who do the
right thing against their better judgement, to gamblers who bet their
life savings on unlikely, but ultimately successful, outcomes. (34)

This line of reasoning, however, ultimately does not provide grounds for
criticising Jane. A moral agent may well have a moral responsibility to
desire to do the things that happen to be the right things to do. For
example, she may well have a responsibility to want to visit her sick
friends, and to help those in need, and not cause harm to others. But
she does not have a responsibility to want to do the right thing,
whatever it turns out to be. Indeed, she would be a worse moral agent if
many of her actions were motivated by such a desire. She should want to
visit her friend because she cares about her friend, not because it is,
in the abstract, the right thing to do. Michael Smith has described the
desire to do the right thing, whatever it turns out to be, as a moral
fetish, and this often seems appropriate. (\emph{The Moral Problem},
Oxford: Blackwell, 1994, p.~76)

It is no discredit to Jane that she lacks this general desire, and in
some cases it may be a virtue. If Jane has the general desire, if in
Smith's terminology she is a moral fetishist, then she may be
prudentially required to follow Lockhart's principle, but not otherwise,
and she is not required, by any normative standard, to be a moral
fetishist. If Jane (virtuously) does not have that general desire to do
the right thing, whatever it turns out to be, then she is importantly
dissimilar to the gambler, who does (and should) desire to bet on the
successful outcome, whatever it turns out to be.

Whatever the merits of Lockhart's main principle, his approach raises
several fascinating theoretical questions. For example, there is a
substantial literature on what the motivational effects of coming to
hold a new moral view are, and what they should be. But what is, and
what should be, the motivational effects of coming to hold, say, that it
is more probable than not that meat eating is permissible? From a
different angle, if moral attitudes are more like desires than like
beliefs, as some expressivists suggest, then can we even have the
attitude that it is more probable than not that meat eating is
permissible? Although in general Lockhart says little directly on these
theoretical questions, it is a great service to show how they arise.

If Lockhart's main principle is correct, it has rather radical
implications for how applied ethics is practised. Even if it is not,
consideration of the issues Lockhart raises may provide a novel and
valuable outlook on some familiar theoretical questions.

\vspace{1cm}



\noindent Published in\emph{
Mind}, 2002, pp. 693-696.


\end{document}
