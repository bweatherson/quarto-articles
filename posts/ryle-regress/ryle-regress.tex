% Options for packages loaded elsewhere
\PassOptionsToPackage{unicode}{hyperref}
\PassOptionsToPackage{hyphens}{url}
%
\documentclass[
  10pt,
  letterpaper,
  DIV=11,
  numbers=noendperiod,
  twoside]{scrartcl}

\usepackage{amsmath,amssymb}
\usepackage{setspace}
\usepackage{iftex}
\ifPDFTeX
  \usepackage[T1]{fontenc}
  \usepackage[utf8]{inputenc}
  \usepackage{textcomp} % provide euro and other symbols
\else % if luatex or xetex
  \usepackage{unicode-math}
  \defaultfontfeatures{Scale=MatchLowercase}
  \defaultfontfeatures[\rmfamily]{Ligatures=TeX,Scale=1}
\fi
\usepackage{lmodern}
\ifPDFTeX\else  
    % xetex/luatex font selection
    \setmainfont[ItalicFont=EB Garamond Italic,BoldFont=EB Garamond
Bold]{EB Garamond Math}
    \setsansfont[]{Europa-Bold}
  \setmathfont[]{Garamond-Math}
\fi
% Use upquote if available, for straight quotes in verbatim environments
\IfFileExists{upquote.sty}{\usepackage{upquote}}{}
\IfFileExists{microtype.sty}{% use microtype if available
  \usepackage[]{microtype}
  \UseMicrotypeSet[protrusion]{basicmath} % disable protrusion for tt fonts
}{}
\usepackage{xcolor}
\usepackage[left=1in, right=1in, top=0.8in, bottom=0.8in,
paperheight=9.5in, paperwidth=6.5in, includemp=TRUE, marginparwidth=0in,
marginparsep=0in]{geometry}
\setlength{\emergencystretch}{3em} % prevent overfull lines
\setcounter{secnumdepth}{3}
% Make \paragraph and \subparagraph free-standing
\makeatletter
\ifx\paragraph\undefined\else
  \let\oldparagraph\paragraph
  \renewcommand{\paragraph}{
    \@ifstar
      \xxxParagraphStar
      \xxxParagraphNoStar
  }
  \newcommand{\xxxParagraphStar}[1]{\oldparagraph*{#1}\mbox{}}
  \newcommand{\xxxParagraphNoStar}[1]{\oldparagraph{#1}\mbox{}}
\fi
\ifx\subparagraph\undefined\else
  \let\oldsubparagraph\subparagraph
  \renewcommand{\subparagraph}{
    \@ifstar
      \xxxSubParagraphStar
      \xxxSubParagraphNoStar
  }
  \newcommand{\xxxSubParagraphStar}[1]{\oldsubparagraph*{#1}\mbox{}}
  \newcommand{\xxxSubParagraphNoStar}[1]{\oldsubparagraph{#1}\mbox{}}
\fi
\makeatother


\providecommand{\tightlist}{%
  \setlength{\itemsep}{0pt}\setlength{\parskip}{0pt}}\usepackage{longtable,booktabs,array}
\usepackage{calc} % for calculating minipage widths
% Correct order of tables after \paragraph or \subparagraph
\usepackage{etoolbox}
\makeatletter
\patchcmd\longtable{\par}{\if@noskipsec\mbox{}\fi\par}{}{}
\makeatother
% Allow footnotes in longtable head/foot
\IfFileExists{footnotehyper.sty}{\usepackage{footnotehyper}}{\usepackage{footnote}}
\makesavenoteenv{longtable}
\usepackage{graphicx}
\makeatletter
\newsavebox\pandoc@box
\newcommand*\pandocbounded[1]{% scales image to fit in text height/width
  \sbox\pandoc@box{#1}%
  \Gscale@div\@tempa{\textheight}{\dimexpr\ht\pandoc@box+\dp\pandoc@box\relax}%
  \Gscale@div\@tempb{\linewidth}{\wd\pandoc@box}%
  \ifdim\@tempb\p@<\@tempa\p@\let\@tempa\@tempb\fi% select the smaller of both
  \ifdim\@tempa\p@<\p@\scalebox{\@tempa}{\usebox\pandoc@box}%
  \else\usebox{\pandoc@box}%
  \fi%
}
% Set default figure placement to htbp
\def\fps@figure{htbp}
\makeatother
% definitions for citeproc citations
\NewDocumentCommand\citeproctext{}{}
\NewDocumentCommand\citeproc{mm}{%
  \begingroup\def\citeproctext{#2}\cite{#1}\endgroup}
\makeatletter
 % allow citations to break across lines
 \let\@cite@ofmt\@firstofone
 % avoid brackets around text for \cite:
 \def\@biblabel#1{}
 \def\@cite#1#2{{#1\if@tempswa , #2\fi}}
\makeatother
\newlength{\cslhangindent}
\setlength{\cslhangindent}{1.5em}
\newlength{\csllabelwidth}
\setlength{\csllabelwidth}{3em}
\newenvironment{CSLReferences}[2] % #1 hanging-indent, #2 entry-spacing
 {\begin{list}{}{%
  \setlength{\itemindent}{0pt}
  \setlength{\leftmargin}{0pt}
  \setlength{\parsep}{0pt}
  % turn on hanging indent if param 1 is 1
  \ifodd #1
   \setlength{\leftmargin}{\cslhangindent}
   \setlength{\itemindent}{-1\cslhangindent}
  \fi
  % set entry spacing
  \setlength{\itemsep}{#2\baselineskip}}}
 {\end{list}}
\usepackage{calc}
\newcommand{\CSLBlock}[1]{\hfill\break\parbox[t]{\linewidth}{\strut\ignorespaces#1\strut}}
\newcommand{\CSLLeftMargin}[1]{\parbox[t]{\csllabelwidth}{\strut#1\strut}}
\newcommand{\CSLRightInline}[1]{\parbox[t]{\linewidth - \csllabelwidth}{\strut#1\strut}}
\newcommand{\CSLIndent}[1]{\hspace{\cslhangindent}#1}

\setlength\heavyrulewidth{0ex}
\setlength\lightrulewidth{0ex}
\usepackage[automark]{scrlayer-scrpage}
\clearpairofpagestyles
\cehead{
  Brian Weatherson
  }
\cohead{
  Intellectual Skill and the Rylean Regress
  }
\ohead{\bfseries \pagemark}
\cfoot{}
\makeatletter
\newcommand*\NoIndentAfterEnv[1]{%
  \AfterEndEnvironment{#1}{\par\@afterindentfalse\@afterheading}}
\makeatother
\NoIndentAfterEnv{itemize}
\NoIndentAfterEnv{enumerate}
\NoIndentAfterEnv{description}
\NoIndentAfterEnv{quote}
\NoIndentAfterEnv{equation}
\NoIndentAfterEnv{longtable}
\NoIndentAfterEnv{abstract}
\renewenvironment{abstract}
 {\vspace{-1.25cm}
 \quotation\small\noindent\rule{\linewidth}{.5pt}\par\smallskip
 \noindent }
 {\par\noindent\rule{\linewidth}{.5pt}\endquotation}
\KOMAoption{captions}{tableheading}
\makeatletter
\@ifpackageloaded{caption}{}{\usepackage{caption}}
\AtBeginDocument{%
\ifdefined\contentsname
  \renewcommand*\contentsname{Table of contents}
\else
  \newcommand\contentsname{Table of contents}
\fi
\ifdefined\listfigurename
  \renewcommand*\listfigurename{List of Figures}
\else
  \newcommand\listfigurename{List of Figures}
\fi
\ifdefined\listtablename
  \renewcommand*\listtablename{List of Tables}
\else
  \newcommand\listtablename{List of Tables}
\fi
\ifdefined\figurename
  \renewcommand*\figurename{Figure}
\else
  \newcommand\figurename{Figure}
\fi
\ifdefined\tablename
  \renewcommand*\tablename{Table}
\else
  \newcommand\tablename{Table}
\fi
}
\@ifpackageloaded{float}{}{\usepackage{float}}
\floatstyle{ruled}
\@ifundefined{c@chapter}{\newfloat{codelisting}{h}{lop}}{\newfloat{codelisting}{h}{lop}[chapter]}
\floatname{codelisting}{Listing}
\newcommand*\listoflistings{\listof{codelisting}{List of Listings}}
\makeatother
\makeatletter
\makeatother
\makeatletter
\@ifpackageloaded{caption}{}{\usepackage{caption}}
\@ifpackageloaded{subcaption}{}{\usepackage{subcaption}}
\makeatother

\usepackage{bookmark}

\IfFileExists{xurl.sty}{\usepackage{xurl}}{} % add URL line breaks if available
\urlstyle{same} % disable monospaced font for URLs
\hypersetup{
  pdftitle={Intellectual Skill and the Rylean Regress},
  pdfauthor={Brian Weatherson},
  hidelinks,
  pdfcreator={LaTeX via pandoc}}


\title{Intellectual Skill and the Rylean Regress}
\author{Brian Weatherson}
\date{2017}

\begin{document}
\maketitle
\begin{abstract}
Intelligent activity requires the use of various intellectual skills.
While these skills are connected to knowledge, they should not be
identified with knowledge. There are realistic examples where the skills
in question come apart from knowledge. That is, there are realistic
cases of knowledge without skill, and of skill without knowledge.
Whether a person is intelligent depends, in part, on whether they have
these skills. Whether a particular action is intelligent depends, in
part, on whether it was produced by an exercise of skill. These claims
promote a picture of intelligence that is in tension with a strongly
intellectualist picture, though they are not in tension with a number of
prominent claims recently made by intellectualists. (The picture is Rex
Whistler's portrait of Ryle, from
\href{https://en.wikipedia.org/wiki/Gilbert_Ryle\#/media/File:Rex_Whistler_-_Gilbert_Ryle,_Fellow.jpg}{Wikipedia}.)
\end{abstract}


\setstretch{1.1}
In recent work about know how, Rylean regress arguments have largely
dropped out of focus. They play little role in the anti-intellectualist
arguments of various kinds in the papers collected in Bengson and
Moffett (\citeproc{ref-BengsonMoffett}{2011}). They are used as
something like target practice by intellectualists like Jason Stanley
(\citeproc{ref-Stanley2011}{2011}), who uses the first chapter of his
book to dispose of them before getting onto the real business. And even
Yuri Cath, who in other work has launched sharp critiques of
intellectualism, has argued that the regress arguments for
anti-intellectualism don't work (\citeproc{ref-Cath2011}{Cath 2011},
\citeproc{ref-Cath2013}{2013}). The majority view seems to be that Carl
Ginet (\citeproc{ref-Ginet1975}{1975}) basically showed these arguments
didn't work, and it's time to move onto other considerations for or
against intellectualism.

I think this isn't exactly right. In particular, I think regress
arguments can be used to show a few different things. For one, they can
be used to refute a precisification of this thesis, which plays a key
role in some intellectualist arguments.

\begin{itemize}
\tightlist
\item
  Only volitional actions are normatively assessable.
\end{itemize}

Once we have seen that thesis is false, we need a new picture of how
action can be at once intelligent and non-volitional. Some
considerations similar to those adduced by Ryle
(\citeproc{ref-Ryle1949}{1949}) concerning agents who either concentrate
on irrelevant considerations, or ignore relevant ones, show there is a
role for intellectual skill that cannot be identified with any piece of
knowledge that. And some further considerations, similar to those
adduced by Cath (\citeproc{ref-Cath2011}{2011}), suggest that this
intellectual skill can't even be constituted by a piece of knowledge
that. So regress arguments, I'll argue, can do quite a lot to motivate
the thought that there was a lot wrong with the intellectual picture
Ryle tried to attack.

The position I'm going to be defending is a long way from the strongest
kinds of Rylean position that contemporary intellectualists such as
Stanley are focussed on arguing against. My focus is primarily on
intellectual skill. This has some relevance for debates about know how,
though less relevance for debates about the semantics of know how
ascriptions. This focus on skill rather than know how ascriptions is
hardly novel; it is continuing a trend that we see exemplified in recent
work by, inter alia, Carlotta Pavese (\citeproc{ref-PavesePhD}{2013}),
Ellen Fridland (\citeproc{ref-Fridland2014}{2014}) and Cheng-Hung Tsai
(\citeproc{ref-Tsai2014}{2014}). And in fact that conclusions I'll draw
here are, I think, similar to the ones that Fridland draws.

Once we move towards thinking about skill, we can get varieties of
anti-intellectualism that are very different from those that were the
focus of most philosophical discussion until very recently. For example,
the anti-intellectualist view I'm defending is consistent with the
following four theses.

\begin{itemize}
\tightlist
\item
  Instances of intellectual skill are usually, and perhaps always, not
  happily reported using know how ascriptions.
\item
  Know how ascriptions are rarely, if ever, reports of intellectual
  skill, and are frequently reports of propositional knowledge.
\item
  Intellectual skill is guided by, and dependent on, propositional
  knowledge.
\item
  Propositional knowledge is not behaviourally inert.
\end{itemize}

Not just is the view consistent with these four, I'm fairly confident
that the last three at least are true. But that's all consistent with
the view that intellectual skill is not itself propositional knowledge.
And it's all consistent with the view that we can learn philosophically
significant conclusions from Ryle's regress arguments.

One disclaimer before I start. Although this paper is heavily influenced
by (\citeproc{ref-Ryle1945}{Ryle 1945}, \citeproc{ref-Ryle1949}{1949}),
and sympathetic interpreters of Ryle such as Jennifer Hornsby
(\citeproc{ref-Hornsby2011}{2011}), I make no attempt at Ryle exegesis
here. I think there's a decent case to be made that Ryle was sympathetic
to the position defended here, but I'm going to leave that debate for
another day.

\section{The Volitional Regress}\label{the-volitional-regress}

Define a \textbf{volitional} action to be one that is preceeded by a
volition to perform that very action. And say an action is
\textbf{normatively assessable} if it can properly be assessed using
terms like \emph{praiseworthy, blameworthy, intelligent} or
\emph{stupid}. Note that I'm ruling out assessments as \emph{good} or
\emph{bad} as versions of normative assessment, in the relevant sense.
Someone who has a good digestive system is not, thereby, normatively
assessable in the stipulative sense I'm using. Both of these definitions
are to an extent stipulative; the terms `volitional' and `normative' can
sensibly receive many other definitions. Still, I will stick to these
definitions here. In light of those stipulations, consider the following
set of propositions.

\begin{enumerate}
\def\labelenumi{\arabic{enumi}.}
\tightlist
\item
  Only volitional actions are normatively assessable.
\item
  The action of forming a volition is normatively assessable.
\item
  Some public actions, such as making a move in a chess game, are
  normatively assessable.
\end{enumerate}

It should be obvious that this leads to a regress. Whether the kind of
regress in question is impossible, or even impractical, is a tricky
question. (See Robert K. Meyer (\citeproc{ref-Meyer1987}{1987}) for some
of the complications that arise when trying to reason about regresses.)
But it is commonly assumed in this literature that the kind of regress
that these three premises lead to is problematic.

Since the third premise is obviously true, the issue is whether the
first or second is false. But it seems that second is true as well. Just
as we can assess a person's actions as praise or blameworthy,
intelligent or stupid, we can assess the process by which she decided to
perform those actions in the same way. Consider two people who make the
same, as it turns out great, chess move in the same situation. The first
notices an initially appealing counter to her move, and sees after
careful thought that it won't work. The second simply doesn't notice the
counter, and is stumped when her opponent makes it. It seems the first
has engaged in a more intelligent practice of volition formation than
the second. Or imagine a third player, whose initial analysis of the
move starts by considering a recipe for arroz con leche. Unless there
turns out to be an unnoticed connection here, this looks even less
intelligent than the second player.

On the other hand, the first premise is rather unintuitive. To borrow an
example from Angela Smith (\citeproc{ref-AngelaSmith2005}{2005}), it is
blameworthy to forget a friend's birthday, although forgettings are
rarely volitional. So we must reject 1 or 2, and while 1 is subject to
independent counterexample, 2 seems independently plausible. So 1 must
be false.\footnote{This argument is obviously rather quick, and I doubt
  will persuade someone already convinced of 1. For much more extensive
  arguments against 1, see the Smith, Ryan and Steup articles cited in
  the text, plus Adams (\citeproc{ref-Adams1985}{1985}).}

That's already a substantial conclusion. Something like 1 is behind
William Alston's famous, and influential, arguments against
deontological approaches to epistemology
(\citeproc{ref-Alston1988}{Alston 1988}). But the negation of 1 is not a
novel claim; I'm saying nothing here that Smith didn't say in her
rejection of the ``volitional view of responsibility''
(\citeproc{ref-AngelaSmith2005}{Smith 2005, 238}). And similar views
have been put forward by other critics of Alston such as Sharon Ryan
(\citeproc{ref-Ryan2003}{2003}) and Matthias Steup
(\citeproc{ref-Steup2008}{2008}).

But still, the fact that 1 is false seems not to have been sufficiently
appreciated in the recent literature on intellectualism. To see one
place where it is relevant, consider this set of propositions, which
also seem to trigger a regress.

\begin{enumerate}
\def\labelenumi{\arabic{enumi}.}
\setcounter{enumi}{3}
\tightlist
\item
  Intelligent action requires the triggering of a prior representation
  of knowledge relevant to the action.
\item
  The triggering of a representation, when done well, is an intelligent
  action.
\item
  Some public actions, such as making a move in a chess game, are
  intelligent.
\end{enumerate}

Again, these propositions obviously trigger a regress, and that seems
like good evidence to take one of them to be false. This is very similar
to one of the regresses Jason Stanley considers in chapter 2 of his
(\citeproc{ref-Stanley2011}{2011}). And Stanley thinks the false
proposition is 5. He writes ``Triggering a representation can be done
\emph{poorly} or \emph{well}. But this does not show it can be done
\emph{intelligently} or \emph{stupidly}.''
(\citeproc{ref-Stanley2011}{Stanley 2011, 16}) Indeed, he writes that
since ``triggering representations is something we do automatically''
(\citeproc{ref-Stanley2011}{Stanley 2011, 16}) a statement like 5 is a
``manifest implausibility'' (\citeproc{ref-Stanley2011}{Stanley 2011,
16}). But the argument here relies on 1. If you think things done
non-volitionally can be intelligent or stupid, it isn't too much of a
stretch to think that things done automatically can be intelligent or
stupid. Indeed, Smith's birthday example is already enough to undermine
Stanley's point; forgetting a friend's birthday seems automatic in the
sense he has in mind, but is also stupid.

More generally, it seems very intuitive to describe everyday cases in
such a way that 5 must be true. For example, Billy asks Suzy whether she
thinks Jill's party will be a success. There are a lot of things that
are common knowledge between the two of them. One is that Jill is a
proficient party host. Another is that Jill has invited all of their
colleagues, including Jack. Another is that parties which Jack attends
are rarely successes. But Suzy thinks for a minute, remembers that Jack
is away in Ohio, and says that it will be a success.

It was smart of Suzy to think about Jack's whereabouts. It wasn't,
perhaps, necessary. If she'd just reasoned from Jill's general
proficiency to the success of the party, she would have got to the right
conclusion. But it was better to note a possible complication, and check
that it wouldn't actually get in the way.

It would have been stupid to perform the same activity for many other
kinds of possible complications. If Suzy had thought to herself, ``The
party will be a disaster if there's an alien invasion in the middle of
it, but there's no reason to think the aliens will invade just now, so
I'll keep on thinking it will be a success,'' that would have been
stupid. Other possible complications are not stupid to consider, but
they are intellectual mistakes. The party won't be a success if there's
a police raid in the middle, based on a mistaken view the police have
about where a particular drug dealer lives. Police do make mistakes, so
even if Jill isn't a drug dealer, this could be a genuine concern,
depending on how nearby the mistakes are. But if the nearest mistake was
a botched raid in a neighbouring state in the previous year, it's wrong
for Suzy to worry about this before answering Billy's question.

Stanley's view has to be that I've been misusing adjectives
systematically through the last two paragraphs. I shouldn't have said
that it was smart of Suzy to consider Jack's whereabouts, or that it
would have been stupid to consider the alien invasion. Rather, it was
just her cognitive system working well when she considered Jack, and
would have been working poorly had she considered the aliens, and
sub-optimally had she considered the police. This doesn't seem at all
the natural way to describe the case to me, in part because I'm not sure
I see the difference Stanley is hinting at. Intelligence just is the
good operation of the cognitive system, and stupidity its poor
operation.

So these two regresses lead to two interesting conclusions. First, some
non-volitional actions are normatively assessable. Second, intelligent
action does not always require the prior triggering of a representation
of relevant knowledge. Both of these are interesting. Both of these are
negations of part of what you might consider ``the intellectualist
picture''. (Cath (\citeproc{ref-Cath2013}{2013}) notes that Ryle often
refers to the regresses as arguments against this picture, not against
any particular thesis.) But neither of them get us very close to a
distinction between know how and know that, or between intellectual
skill and know that. The next section addresses some ways we might move
closer to arguments against more central intellectualist claims.

\section{Picturing Intelligent
Action}\label{picturing-intelligent-action}

As noted in the introduction, my plan is not to offer an argument with
regress like premises, and the conclusion that intellectual skill is
distinct from propositional knowledge, or that know how is distinct from
propositional knowledge. What I do want to do is sketch a picture of
human intelligence (at a very high level of generality) that presupposes
that intellectual skill is not identical to propositional knowledge, and
suggest some considerations to the effect that no similarly plausible
picture exists in which intellectual skill and propositional knowledge
are identified. The thought here is not that the only way out of the
regress involves distinguishing skill and knowledge -- and perhaps
distinguishing know how from know that -- but rather that the best way
out does.

Start with a well known, if not obviously authentic,
exchange.\footnote{I thought this example was purely fictional, coming
  from the Monty Python sketch reproduced in Dempsey
  (\citeproc{ref-MontyPython}{2012, 741}). But Ben Wolfson pointed out
  to me that it's recorded as a true story in Hadley
  (\citeproc{ref-Hadley1903}{1903, 255}).

  ~~~~~It's actually striking how few really good off-the-cuff quips
  there are in recorded history. The famous one attributed to Wilde, ``I
  have nothing to declare but my genius'', is probably apocryphal, and
  in any case sounds prepared. Lists of famous come-backs and ripostes
  are usually crowded with written responses. Word play is hard.}

\begin{quote}
Oscar Wilde: I wish I'd said that.

James McNeill Whistler: You will Oscar; you will.
\end{quote}

Assuming this really happened, that's a clever response. It's an
occupational hazard of philosophers to think that the ability to come up
with quick, clever responses is somehow central to intelligence. But we
can reject that wildly implausible view without thinking that it's wrong
to think of these quips as a manifestation of a kind of intelligence.

Now let's think of how someone could have come up with this response.
Even before we start researching the neural patterns behind quips like
this, we can be pretty sure the following is not what happened in
Whistler's brain. He first made an exhaustive list of all possible
responses, from ``Green ideas sleep furiously'' to what he actually
said, then figured out which would be best, then produced the best one.
On this wildly implausible model, the reply would be intelligent because
it would reflect the speaker's ability to properly evaluate this list of
responses. That's implausible because the list is simply too big.
Indeed, it is in principle infinitely large. The list is too big to
survey not just consciously, but subconsciously.

Coming up with a response like this requires first coming up with a
narrower list of possible responses, and then evaluating which is best
from that list.\footnote{Or, perhaps even more plausibly, coming up with
  a short list of possible openings, choosing the best, and doing what
  one can to figure out how to complete the response while uttering the
  start of it. Thanks here to Ben Wolfson.} There's a romantic model of
intellect where the list in question consists of just the reply actually
issued. On this model the perfect reply appears fully and perfectly
formed in the mind of the intelligent person. Now such a model may often
fit the phenomenology, but I don't think we should give that much
credit. It's an empirical question how many possible replies are
represented in the mind in a situation like this, before the chosen
reply is issued. What's not an empirical issue is whether the list of
possible replies that is represented in the mind is finite or infinite.
It simply must be finite, which means that there must be better and
worse lists to consider. And that suggests that there is some skill
involved in coming up with the list.

One could reject this last conclusion. One could try saying that the
coming up with a list of possible replies is no manifestation of skill,
but the skill is only involved in the evaluation and selection of
replies. But this seems to generate a bizarre explanation about why the
less skilled interlocutor comes up with worse replies. The model,
presumably, is that the lack of skill does not explain having the wrong
list of replies to choose between. Rather, what explains their less
skilled reply must simply be that they misevaluated the possible
replies. But that doesn't fit the observed data. It's much easier to see
of someone else's reply that it was clever than it is to come up with a
clever reply.

It could also be objected that the model I've suggested is much too
simple. It isn't just that the mind issues a list of options, then
evaluates them, and then selects the best. A more plausible model
involves more recursive steps. The mind first generates a list of
options, selects the best, then generates a list of refinements of that
best option, selects the best of those, and so on. Perhaps when we
consider superficial forms of intelligence, such as quips, it makes
sense to consider a `one-step' model, where a list is generated and
evaluated, followed by a speech. But when one is choosing one's words
carefully, as in say Wilde's writing, the simple model I've described
feels much too simple.

But although the simple model is too simple for considered writing, the
general structure must be right. Even a writer working at a leisurely
pace, such as Joyce taking decades to write \emph{Finnegans Wake},
doesn't have time to consider, even subconsciously, all possible
constructions. There are still too many. And nor is it true that the
difference between Joyce's skill and ours is that he realises the value
of the sentences we all represent. The rest of us didn't simply misjudge
the value of ``Nobirdy aviar soar anywing to eagle it''
(\citeproc{ref-Joyce1939}{Joyce 1939/2012, 505}); we simply didn't token
it. The ability to token mental representations like that is part of
what Joyce's genius consists in.

I've focussed so far on cases where it is a priori implausible that
human thinkers start by surveying the range of possible things they
could do. It is also interesting to look at cases where this is in
principle possible, but doesn't seem to happen in practice. There have
been, traditionally, major differences in the style of play between
human and computer chess players. (Since so many young players learn
from machines these days, Kasparov (\citeproc{ref-Kasparov2010}{2010})
suggests these differences are diminishing.\footnote{Thanks to Bernard
  Kobes and John Collins for helpful discussions about the chess
  examples.}) This isn't necessarily because humans can't consider all
options on the chess board. Usually there will be fewer than a hundred
available moves, and a human could consider each. But that isn't, it
seems, how humans think. They don't allocate equal resources to working
through each of the possible options. As a result, computers often come
up with surprising kinds of moves. Now computers are actually very good
at chess, so these pre-deliberative allocations of cognitive resources
may not have been optimal. Perhaps it would have been better for
traditional chess players to spend more time thinking through unlikely
progressions of the game. But it is evidence that even when we could use
an unintelligent method for beginning inquiry, namely recursively
generating the possible options, we prefer to use intelligent methods.

So intelligent action, at least in humans in the kinds of situations
humans normally find themselves in, consists in part of making
intelligent choices about where to start inquiry. Given that intelligent
action need not be volitional, as we established above, it isn't
surprising that being intelligent consists in part in starting in the
right places. But perhaps this intelligence is just itself a kind of
knowledge. It is, we might suspect, just the knowledge of what a good
starting point will be. Or, since we will want to start with all and
only the considerations relevant to a given inquiry, it is just
knowledge of what is relevant.

The resulting picture is both perfectly intellectualist, and immune to
the regresses considered above. The intelligent person knows what is
relevant to what inquiry. Her choice of starting points is guided by
this knowledge. (The `guidance' metaphor recurs frequently in Stanley's
work.) This isn't because it leads to a volition to start just here.
Such a volition would be self-defeating, since in the relevant sense of
`start', by the time this volition is formed, one has already started,
and indeed started elsewhere. Nor can she be guided by even a triggered
representation of this knowledge of relevance. Again, if that happens,
she is in the relevant sense starting elsewhere. But perhaps
propositional knowledge can guide directly; not by generating volitions,
and without even being represented anew.

Now I don't think this picture is right. But it isn't incoherent either,
and it takes work to see why it isn't right.

One bad argument against this picture starts with the idea that skills
are active, while knowledge is passive. The thought is that the person
who knows a lot is like the Tortoise in Louis Carroll's dialogue
(\citeproc{ref-Carroll1895}{Carroll 1895}), only able to add more
premises but never to reach a conclusion. It is only with skill that we
can get to the conclusion. Stanley rightly objects to this argument on
the grounds that it just isn't true that knowledge is passive in the
relevant sense. We should not, as Stanley puts it,
``over-intellectualize knowing that''.
(\citeproc{ref-Stanley2012-Replies}{Stanley 2012, 773}). (A similar
point is made in Stalnaker (\citeproc{ref-Stalnaker2012}{2012}).)
Knowing that \emph{p} is not just a matter of having \emph{p} written in
a knowledge box somewhere in the brain; it can in part be constituted by
active dispositions.

A better argument looks at the very different modal profiles of
intellectual skill and knowledge of relevance. Someone can know that
something is irrelevant and yet lack the skill to ignore it; or they can
know that something is relevant and yet lack the skill to consider it in
a timely manner. Examples from the other direction, where there is skill
without knowledge, are a little more contentious, but we'll look at some
possible cases of those too. But first we'll run through two examples to
show how easy it is to have knowledge without skill.

Alice has spent a lot of money on video-conferencing equipment. But it
isn't working at all well, and she now has to decide whether to try and
patch it into something better, or buy a whole new system. She knows the
sunk cost fallacy is a fallacy; that buying a new system would make the
previous purchases a waste is no reason to not buy a new system,
especially if doing so is good value compared to the cost of buying a
`patch'. But she can't bring herself to ignore this fact when
deliberating. Even though she eventually makes the right decision and
buys new equipment, she takes much longer about this than she would have
if, say, the existing equipment was old enough that she could easily
conceptualise it as obsolete.

Bob is trying to solve a puzzle about the properties of functions from
rationals to rationals. He knows that it is often helpful, when solving
such puzzles, to transform the puzzle into one about functions from
ordered pairs of integers to ordered pairs of integers. He knows that in
the sense that if you asked him whether it could be useful to consider
that transformation of the puzzle, he would immediately say yes, and
this answer would come with the phenemenology of recollection, not of
new insight. But no one does ask him that question, and the
transformation in question simply never occurs to Bob. Since the
untransformed puzzle is very hard, while the transformed puzzle is
manageable, Bob never solves the problem.

It seems to me that what's happened in both cases is that the agent has
some knowledge, but is incapable of using it. What they lack is a skill.
In particular, they lack what Fridland
(\citeproc{ref-Fridland2014}{2014, 2746}) calls `selective, top-down,
automatic attention'. Alice keeps attending to something she should not,
even though she knows she should not. Bob fails to attend to something
he should, although in some sense he knows that is what he should attend
to. Bob's case is one of the reasons I find the picture of skill
presented by Stanley and Williamson
(\citeproc{ref-StanleyWilliamson2016}{2017}) unhelpful. They say skill
is a disposition to form knowledge. But Bob has the important knowledge.
The disposition he lacks is the disposition to activate that knowledge,
and let it guide deliberation. That's what constitutes his lack of
skill.

It's true that knowledge isn't completely passive. If Alice never
appealed to the fact that the sunk cost fallacy is a fallacy in her
reasoning, we wouldn't say that she knows it. If none of Bob's answers
were guided by the existence of natural and useful transformations
between rational numbers and ordered pairs of integers, we wouldn't say
he knows such transformations are natural and useful. I'm here agreeing
with Stanley and Stalnaker that knowledge is itself a kind of
disposition. And intellectual skill is a kind of disposition too. But
they are very different dispositions. In particular, they have very
different triggering conditions. Bob lacks some skill because he does
not call to mind this fact about rational numbers right now. He has the
salient knowledge about rational numbers because he is disposed to use
the facts in question often enough.

So intellectual skill and knowledge of relevance have different
manifestation conditions, and so they are not identical. But we can say
something stronger than that. The cases of Alice and Bob are not in any
way unusual. Examples where we forget the salience of some
consideration, or can't get an irrelevant point out of our heads, are
frequent. In principle, one could respond to the arguments I've made so
far by saying that while knowledge of relevance is not identical to
skill, nevertheless the two are as closely linked as, say, a material
object and the matter that constitutes it. And if I had to resort to
bizarre cases of the kind we torture introductory students with to make
my point, I'd say that would be the right response. But given how normal
Alice and Bob's cases are, this seems like the wrong move. Skill and
knowledge don't just come apart in theory, they come apart in practice,
frequently.

\section{Four Objections}\label{fourobjections}

So far I've defended three theses that are in tension with some forms of
intellectualism. They are:

\begin{itemize}
\tightlist
\item
  Some non-volitional actions are normatively assessable.
\item
  Not all intelligent action is preceded by the triggering of
  representations of relevant knowledge.
\item
  Intellectual skill, in particular the intellectual skill associated
  with starting inquiry in the right place, is not identical to any
  piece of propositional knowledge.
\end{itemize}

While this doesn't show that, for instance, know how and know that are
distinct, and is completely silent on what we should say about know how
ascriptions, it does undermine some intellectualist programs. I'll
conclude with some objections either to the arguments I've put forward,
or to their significance.

\emph{Objection}: Even if all of this is true, there may still be a
sense in which intellectualism is true. After all, it could still be
that knowledge guides action in a suitable way. (Compare
(\citeproc{ref-Stanley2011}{Stanley 2011, 2}).)

\emph{Reply}: This could be true. Whether it is a win for
intellectualism depends a bit on the boring question of how we settle
the term `intellectualism', and a bit on more interesting questions
about priority. Let's start by distinguishing five theories we might
call intellectualist.

\begin{description}
\tightlist
\item[Identity Intellectualism]
The possession of an intellectual skill just is the possession of a
piece of knowledge.
\item[Constitution Intellectualism]
The possession of an intellectual skill is, always, constituted by a
piece of knowledge.
\item[Weak Constitution Intellectualism]
The possession of an intellectual skill is, often, constituted by a
piece of knowledge.
\item[Causal Intellectualism]
The possession of an intellectual skill is, always, caused by the
possession of a piece of knowledge.
\item[Weak Causal Intellectualism]
The possession of an intellectual skill is, often, caused by the
possession of a piece of knowledge.
\end{description}

This paper has been arguing against Identity Intellectualism. I think
the falsity of this is as much as we could reasonably hope to prove
using regress arguments. (I think I'm here agreeing with Wiggins
(\citeproc{ref-Wiggins2009}{2009}) and Hornsby
(\citeproc{ref-Hornsby2011}{2011}).) And the considerations behind the
regress argument do, I think, show it to be false. If someone wants to
insist that by intellectualism, they mean something weaker than this,
I'm not going to quarrel over terminology. I'll just note that Identity
Intellectualism is an interesting, and false, thesis.

The arguments here are clearly not arguments against either form of Weak
Intellectualism. Indeed, they are naturally understood as the kind of
cases that confirm Weak Intellectualism. Mathematics students, like Bob,
train by learning a lot of mathematical facts. And it's hard to see how
they could develop the relevant skills without knowing some important
facts. This is, I suspect, the general case. Skillfully bringing the
right considerations to bear on a problem requires, and is probably the
causal consequence of, knowing a lot of relevant facts. (Tsai
(\citeproc{ref-Tsai2014}{2014}) makes clear how one can simultaneously
hold that skills are in part constituted by knowledge of facts without
having an intellectualist picture of skill.)

But what of the other two intellectualist theories? Do we have reason to
think that there are some skills that are not constituted by, or not
caused by, the possession of factual knowledge? One way to quickly show
that would be to show that there can be skills without the related
knowledge. Perhaps that's not just sufficient for rejecting
Constitutive/Causal Intellectualism, but necessary. If knowledge without
skills is possible, as in Alice and Bob's cases, and skills without
knowledge were impossible, that asymmetry would call out for
explanation. And something in the vicinity of Constitutive or Causal
Intellectualism would be a very good candidate explanation.

There are (at least) two promising routes to showing that there can be
skills without knowledge. One is due to Imogen Dickie
(\citeproc{ref-Dickie2012}{2012}). She argues that since there are so
many different routes to skill than there are to knowledge, we should
expect that there will be cases of skill that are causally prior to
knowledge. Jason Stanley (\citeproc{ref-Stanley2012-Replies}{2012})
replies that Dickie's argument assumes an overly narrow conception of
propositional knowledge. This is a fascinating debate, but I don't have
anything useful to add to it, so I'll just note the existence of this
route, and move on.

The other route is due to Yuri Cath (\citeproc{ref-Cath2011}{2011}). He
suggests that facts in virtue of which a person might lose propositional
knowledge do not always bring about a loss of knowledge that. I'm going
to sketch a Cath-style argument that we can have intellectual skills
without knowledge. I think the argument has some force, though there are
more ways to resist it than there are to resist the argument against
Identity Intellectualism.

Ross and Rachel are economics students taking an exam. They are given a
hard question asking about the likely effects of an exogenous shock, say
an earthquake affecting an area the supplies crucial raw materials, on
some related markets. The question is hard, with the relevant causal
pathways being interconnected and often opposing. The only plausible way
forward is to use a model and search for equilibrium points in the
model. That's what Ross and Rachel have both been taught to do. And in
fact both of them quickly select the right kind of model, with just the
right amount of complexity in it to answer the question without being
overburdened, and set out on the difficult algebra involved in solving
the question.

So far it looks like both Ross and Rachel have shown intellectual skill.
Now it turns out Ross and Rachel have very different views about the
role of models in economic thinking. (My own thinking about models has
been heavily influenced by Strevens (\citeproc{ref-Strevens2008}{2008}
ch.~8) and Davey (\citeproc{ref-Davey2011}{2011}), and I rely on their
insights in what follows.) These models involve, as all models do, some
serious idealisation. Most notably, they assume that all the relevant
actors are perfectly rational utility maximisers. Rachel hasn't given
much thought to this assumption, though she knows it to be literally
false. But if pressed, she would say some reasonably sensible things
about why she was using the model. For one thing, the familiar failures
of human rationality aren't obviously relevant to the puzzle being
presented. For another, they've been taught that using these models is a
good way to solve problems, and that testimonial evidence carries some
weight. And for another, it's an exam, and it is likely that questions
have been selected to test how well students can use the models they
have been taught. If those are her background, implicit, views, I think
it is plausible to say that Rachel knows that the model is relevant to
the exam question, even if she couldn't produce a theory of
idealisations in economics of the standards of the best philosophers.

Ross's views about models are rather different. He thinks the familiar
models in economics work, when they do, because the background
assumptions are strictly and literally true. He thinks economic agents
are utility maximisers, and the apparent evidence to the contrary is due
to sloppy experimental design. He thinks markets are always in general
equilibrium. And so he thinks that the only sources of error in
predictions we can make about markets are from errors about things like
the costs of extracting raw materials after the earthquake. This
perspective is, of course, grossly mistaken. Moreover, Ross thinks that
if the assumptions were not correct, there would be no point in using
the models. This too is a mistake, though perhaps not as dramatic as his
other mistakes.

Now even if Ross and Rachel aren't thinking about these philosophical
views about the nature of models, I think they are relevant to whether
each of them know that the models are relevant to the puzzle. In
particular, I think Rachel does know that the models are relevant, while
Ross's belief that they are relevant is more like a lucky guess than a
piece of knowledge. Still, I think we should say that Ross showed skill
in using this model rather than a more or less complex model, or a
different kind of model, or no model at all. So he is a case of
intellectual skill without knowledge of relevance.

I don't think this case is conclusive. I can think of at least four ways
someone might reasonably object to the case.

\begin{enumerate}
\def\labelenumi{\arabic{enumi}.}
\tightlist
\item
  It might be argued that despite his false views about why the models
  are relevant, he really does know that they are relevant. In other
  words, we would have another counterexample, to be added to those
  discussed by Warfield (\citeproc{ref-Warfield2005}{2005}) and Luzzi
  (\citeproc{ref-Luzzi2010}{2010}), to the theory that false beliefs
  cannot generate knowledge.
\item
  It might be argued that Ross is not really skilled, since it is a
  matter of luck that the falsity of his beliefs does not lead him to
  false conclusions here.
\item
  It might be argued that although Ross doesn't know that this model is
  relevant, his skill is constituted by, or caused by, some other
  knowledge he has.
\item
  It might be argued that the broad picture of the role of idealisations
  in scientific reasoning that I'm adopting from Strevens and Davey is
  mistaken, and this fatally undermines my use of the case to argue
  against intellectualism.
\end{enumerate}

I don't think these arguments are going to ultimately work. But it's
clear we are a long way from Rylean regress arguments here. And that's
where I think the debate about regress arguments should end. We have a
good argument against Identity Intellectualism. And we have some
suggestive considerations that seem to tell against Constitutive and
Causal Intellectualism, but whether these arguments ultimately work will
depend on considerations independent of the regress.

\emph{Objection}: Stanley and Williamson
(\citeproc{ref-StanleyWilliamson2016}{2017}) have recently defended the
idea that skill is a disposition to form knowledge. And they back this
up with empirical analysis of intelligent motor skills, especially
drawing on the survey by Yarrow, Brown, and Krakauer
(\citeproc{ref-Yarrow2009}{2009}). Is this kind of intellectualism
subject to the regress worries?

\emph{Reply}: Once we are taking the dispositions themselves to be the
skills, not the underlying knowledge, it feels that we are a long way
from traditional intellectualism. But the view is independently
interesting, and it is a useful segue to thinking about the relationship
between intellectual skills, as conceived of in this paper, and motor
skills.

I've already mentioned that the Bob example does not seem to fit well
with Stanley and Williamson's paradigm. And there is something
suspicious about a theory of physical skill that divorces it so strongly
from the physical. To be a skilled batsman requires more than
dispositions to get knowledge, one might suspect. Stanley and Williamson
have a reply to this suspicion. They write,

\begin{quote}
Consider the difference between someone who can bench-press a maximum of
100 pounds and someone who can bench press 150 pounds. We may suppose
that both employ the same technique; only brute strength makes the
difference between them. Both are equally skilled ... Any view of skill
must account for such cases. In particular, it must explain why
strength, speed, and stamina are not themselves skills.
(\citeproc{ref-StanleyWilliamson2016}{Stanley and Williamson 2017, 721})
\end{quote}

But even if strength is not a skill, it might be a prerequisite for a
skill. A batsman whose degenerative back condition means he lacks the
flexibility to deploy his trademark pull shot has lost a skill, even if
he hasn't lost any dispositions to form knowledge. There is a puzzle as
to why qualitative physical differences matter so much to skill
attributions why quantitative ones do not. If you can't turn to pull the
ball, you've lost a skill, but if a muscle strength decline reduces the
power of your pull shot, your skills haven't declined. But that
difference doesn't justify making skills entirely cognitive.

Still, there is a cognitive angle. One central point of this paper is
completely consistent with Stanley and Williamson's picture; motor
skills often require forming the right knowledge. The skilled batsman
doesn't just pick up many characteristics of the bowler's delivery, they
pick up the ones that are most relevant to the trajectory of the ball.
As the Bob example shows, they also have to activate that knowledge for
it really to be a skill, but that's not a new objection.

There is one other cognitive aspect of motor skill that Yarrow, Brown,
and Krakauer (\citeproc{ref-Yarrow2009}{2009}) draw attention to, and
which fits very nicely with the theme of this paper. It's a specific
instance of a much more wide-ranging skill. Sometimes an agent knows
that in some time some evidence, drawn from a large space, will come in.
She will shortly thereafter have to act in response to the evidence. She
has some time to plan now. What should she do? In many such cases,
backwards induction is impossible; there are too many possible pieces of
evidence that could come in, and planning for each of them is a waste of
resources. On the other hand, not planning at all is also a waste of the
time she now has, and will lack once the evidence comes in. The solution
is to do some planning. And there is a real skill involved in getting
the resource allocation right, and neither wasting effort planning for
unlikely scenarios, nor wasting the ability to be prepared before one
needs to act.

Yarrow, Brown, and Krakauer (\citeproc{ref-Yarrow2009}{2009, 590--91})
suggest the same thing happens at a very low level. Highly skilled
athletes are making many places in advance of knowing exactly how they
will act. Part of the skill involved is allocating the right resources
to each of these planning activities. Many of them will ultimately be
wasteful, since they are plans for eventualities that do not arise. And
one failure condition is that a single plan is not selected, and the
agent performs some combination of multiple plans that are worse than
either one plan. That failure state is part of the evidence that there
is this low-level planning going on before actions. But it is a real
skill, and part of the skill is focussing on just the right things.

So motor skills often have as a constituent part intellectual skills.
Some of those skills are closely tied to knowledge; for instance, having
priors that track frequencies. Sometimes the skill involved is in
focussing on the evidence that the posterior probability is maximally
sensitive to, and reacting to that evidence. Sometimes the skill is not
attending to evidence that is just going to be unhelpful noise in the
activity in question (\citeproc{ref-Yarrow2009}{Yarrow, Brown, and
Krakauer 2009, 589}). And sometimes it is in allocating just the right
resources to forward planning. All of these seem like intellectual
skills, and parts of motor skills. We could try to squeeze all of them
into a framework of being dispositions to form knowledge, but it seems
more perspicuous to just present the plurality of ways in which the
intellect and the body interact, rather than trying to find a single
framework.

\emph{Objection}: Appeal to skill does not stop the regress. If we need
to posit something, say a skill, that comes between the possession of
knowledge and the use of knowledge in reasoning or action, then we also
to posit something that comes between the possession of a skill, and the
use of that skill in reasoning or action. (Compare Stanley
(\citeproc{ref-Stanley2011}{2011, 26})).

\emph{Reply}: What I'm going to say here is similar to what Jeremy Fantl
(\citeproc{ref-Fantl2011}{2011}) said in a response to an earlier
version of Stanley's argument, so I'll be brief. Skills are
dispositions. We don't need to posit anything that comes between the
disposition and its triggering. If a string is disposed to produce a
middle C when struck, and it is struck, we don't need to posit an extra
intermediary between the striking and the note. Dispositions stop
regresses.

But, you might insist, couldn't the same be true of knowledge? After
all, on a broadly functionalist construal of the mental, knowledge is a
kind of disposition. My reply is in theory knowledge could stop such a
regress, but in practice it is unlikely. An agent could be facing a
problem where the possible considerations and options can be enumerated
without using any particular skill, and the options are few enough that
they can be each considered in turn. That is the situation an agent
playing a relatively simple game might face. But it isn't the general
human condition. In practice, we face problems every moment where it
requires skill to bring the right considerations to bear, at least given
the processing capacities we have available.

\emph{Objection}: There are semantic arguments that attributions of know
how are attributions of propositional knowledge. This shows that Ryle
was wrong to draw a broad distinction between know how and know that.

\emph{Reply}: I'm not making any claims about either know how or about
`know how'. I am making some claims about skill, and those imply some
claims about `skill'. But I'm sympathetic to the idea that reports of
know how are often reports of some kind of practical propositional
knowledge. I certainly haven't offered any arguments, nor I think any
considerations in the direction of an argument, against this view.

Indeed, there are a lot of intellectualst positions that I'm not arguing
against here. Anti-intellectualism is often tied up with the view that
there is an important distinction between theoretical and practical
fields. The arguments I've developed here suggest that if there is such
a distinction, then proving mathematical theorems is on the `practical'
side. I think that's a strange enough conclusion that it is time to
change our terminology. That's why I've talked about the distinction
between intellectual skills and knowledge, not the distinction (if such
there is) between know how and know that, or between praxis and theory.

\subsection*{References}\label{references}
\addcontentsline{toc}{subsection}{References}

\phantomsection\label{refs}
\begin{CSLReferences}{1}{0}
\bibitem[\citeproctext]{ref-Adams1985}
Adams, Robert Merrihew. 1985. {``Involuntary Sins.''}
\emph{Philosophical Review} 94 (1): 3--31. doi:
\href{https://doi.org/10.2307/2184713}{10.2307/2184713}.

\bibitem[\citeproctext]{ref-Alston1988}
Alston, William. 1988. {``The Deontological Conception of Epistemic
Justification.''} \emph{Philosophical Perspectives} 2: 257--99. doi:
\href{https://doi.org/10.2307/2214077}{10.2307/2214077}.

\bibitem[\citeproctext]{ref-BengsonMoffett}
Bengson, John, and Marc Moffett, eds. 2011. \emph{Knowing How}. Oxford:
Oxford University Press.

\bibitem[\citeproctext]{ref-Carroll1895}
Carroll, Lewis. 1895. {``What the Tortoise Said to Achilles.''}
\emph{Mind} 4 (14): 278--80. doi:
\href{https://doi.org/10.1093/mind/iv.14.278}{10.1093/mind/iv.14.278}.

\bibitem[\citeproctext]{ref-Cath2011}
Cath, Yuri. 2011. {``Knowing How Without Knowing That.''} In
\emph{Knowing How}, edited by John Bengson and Marc Moffett, 113--35.
Oxford: Oxford University Press.

\bibitem[\citeproctext]{ref-Cath2013}
---------. 2013. {``Regarding a Regress.''} \emph{Pacific Philosophical
Quarterly} 94 (3): 358--88. doi:
\href{https://doi.org/10.1111/papq.12004}{10.1111/papq.12004}.

\bibitem[\citeproctext]{ref-Davey2011}
Davey, Kevin. 2011. {``Idealizations and Contextualism in Physics.''}
\emph{Philosophy of Science} 78 (1): 16--38. doi:
\href{https://doi.org/10.1086/658093}{10.1086/658093}.

\bibitem[\citeproctext]{ref-MontyPython}
Dempsey, Luke, ed. 2012. \emph{Monty Python's Flying Circus: Complete
and Annotated...all the Bits}. New York: Black Dog \& Leventhal
Publishers.

\bibitem[\citeproctext]{ref-Dickie2012}
Dickie, Imogen. 2012. {``Skill Before Knowledge.''} \emph{Philosophy and
Phenomenological Research} 85 (3): 737--45. doi:
\href{https://doi.org/10.1111/j.1933-1592.2012.00638.x}{10.1111/j.1933-1592.2012.00638.x}.

\bibitem[\citeproctext]{ref-Fantl2011}
Fantl, Jeremy. 2011. {``Ryle's Regress Defended.''} \emph{Philosophical
Studies} 156 (1): 121--30. doi:
\href{https://doi.org/10.1007/s11098-011-9800-8}{10.1007/s11098-011-9800-8}.

\bibitem[\citeproctext]{ref-Fridland2014}
Fridland, Ellen. 2014. {``They've Lost Control: Reflections on Skill.''}
\emph{Synthese} 191 (12): 2729--50. doi:
\href{https://doi.org/10.1007/s11229-014-0411-8}{10.1007/s11229-014-0411-8}.

\bibitem[\citeproctext]{ref-Ginet1975}
Ginet, Carl. 1975. \emph{Knowledge, Perception and Memory}. Dordrecht:
Riedel.

\bibitem[\citeproctext]{ref-Hadley1903}
Hadley, Frank A. 1903. {``Whistler, the Man, as Told in Anecdote.''}
\emph{Brush \& Pencil} 12 (5): 334--59. doi:
\href{https://doi.org/10.2307/25505918}{10.2307/25505918}.

\bibitem[\citeproctext]{ref-Hornsby2011}
Hornsby, Jennifer. 2011. {``Ryle's \emph{Knowing How}, and Knowing How
to Act.''} In \emph{Knowing How}, edited by John Bengson and Marc
Moffett, 80--98. Oxford: Oxford University Press.

\bibitem[\citeproctext]{ref-Joyce1939}
Joyce, James. 1939/2012. \emph{Finnegans Wake}. Oxford World's Classics.
Oxford: Oxford University Press.

\bibitem[\citeproctext]{ref-Kasparov2010}
Kasparov, Garry. 2010. {``The Chess Master and the Computer.''}
\emph{The New York Review of Books}. The New York Review of Books.
\url{http://www.nybooks.com/articles/archives/2010/feb/11/the-chess-master-and-the-computer/}.

\bibitem[\citeproctext]{ref-Luzzi2010}
Luzzi, Federico. 2010. {``Counter-Closure.''} \emph{Australasian Journal
of Philosophy} 88 (4): 673--83. doi:
\href{https://doi.org/10.1080/00048400903341770}{10.1080/00048400903341770}.

\bibitem[\citeproctext]{ref-Meyer1987}
Meyer, Robert K. 1987. {``God Exists!''} \emph{No{û}s} 21 (3): 345--61.
doi: \href{https://doi.org/10.2307/2215186}{10.2307/2215186}.

\bibitem[\citeproctext]{ref-PavesePhD}
Pavese, Carlotta. 2013. {``The Unity and Scope of Knowledge.''} PhD
thesis, Rutgers University, New Brunswick.

\bibitem[\citeproctext]{ref-Ryan2003}
Ryan, Sharon. 2003. {``Doxastic Compatibilism and the Ethics of
Belief.''} \emph{Philosophical Studies} 114 (1-2): 47--79. doi:
\href{https://doi.org/10.1023/A:1024409201289}{10.1023/A:1024409201289}.

\bibitem[\citeproctext]{ref-Ryle1945}
Ryle, Gilbert. 1945. {``Knowing How and Knowing That.''}
\emph{Proceedings of the Aristotelian Society} 46 (1): 1--16. doi:
\href{https://doi.org/10.1093/aristotelian/46.1.1}{10.1093/aristotelian/46.1.1}.

\bibitem[\citeproctext]{ref-Ryle1949}
---------. 1949. \emph{The Concept of Mind}. New York: Barnes; Noble.

\bibitem[\citeproctext]{ref-AngelaSmith2005}
Smith, Angela M. 2005. {``Responsibility for Attitudes: Activity and
Passivity in Mental Life.''} \emph{Ethics} 115 (2): 236--71. doi:
\href{https://doi.org/10.1086/426957}{10.1086/426957}.

\bibitem[\citeproctext]{ref-Stalnaker2012}
Stalnaker, Robert. 2012. {``Intellectualism and the Objects of
Knowledge.''} \emph{Philosophy and Phenomenological Research} 85 (3):
754--61. doi:
\href{https://doi.org/10.1111/j.1933-1592.2012.00640.x}{10.1111/j.1933-1592.2012.00640.x}.

\bibitem[\citeproctext]{ref-Stanley2011}
Stanley, Jason. 2011. \emph{Know How}. Oxford: Oxford University Press.

\bibitem[\citeproctext]{ref-Stanley2012-Replies}
---------. 2012. {``Replies to Dickie, Schroeder and Stalnaker.''}
\emph{Philosophy and Phenomenological Research} 85 (3): 762--78. doi:
\href{https://doi.org/10.1111/j.1933-1592.2012.00641.x}{10.1111/j.1933-1592.2012.00641.x}.

\bibitem[\citeproctext]{ref-StanleyWilliamson2016}
Stanley, Jason, and Timothy Williamson. 2017. {``Skill.''} \emph{No{û}s}
51 (4): 713--26. doi:
\href{https://doi.org/10.1111/nous.12144}{10.1111/nous.12144}.

\bibitem[\citeproctext]{ref-Steup2008}
Steup, Matthias. 2008. {``Doxastic Freedom.''} \emph{Synthese} 161 (3):
375--92. doi:
\href{https://doi.org/10.1007/s11229-006-9090-4}{10.1007/s11229-006-9090-4}.

\bibitem[\citeproctext]{ref-Strevens2008}
Strevens, Michael. 2008. \emph{Depth: An Account of Scientific
Explanations}. Cambridge, MA: Harvard University Press.

\bibitem[\citeproctext]{ref-Tsai2014}
Tsai, Cheng-hung. 2014. {``The Structure of Practical Expertise.''}
\emph{Philosophia} 42 (2): 539--54. doi:
\href{https://doi.org/10.1007/s11406-013-9513-7}{10.1007/s11406-013-9513-7}.

\bibitem[\citeproctext]{ref-Warfield2005}
Warfield, Ted A. 2005. {``Knowledge from Falsehood.''}
\emph{Philosophical Perspectives} 19: 405--16. doi:
\href{https://doi.org/10.1111/j.1520-8583.2005.00067.x}{10.1111/j.1520-8583.2005.00067.x}.

\bibitem[\citeproctext]{ref-Wiggins2009}
Wiggins, David. 2009. {``Knowing How to and Knowing That.''} In
\emph{Wittgenstein and Analytic Philosophy: Essays for p. M. S. Hacker},
edited by Hans-Johann Glock and John Hyman, 263--77. Oxford: Oxford
University Press.

\bibitem[\citeproctext]{ref-Yarrow2009}
Yarrow, Kielan, Peter Brown, and John W. Krakauer. 2009. {``Inside the
Brain of an Elite Athlete: The Neural Processes That Support High
Achievement in Sports.''} \emph{Nature Reviews Neuroscience} 10 (8):
585--96. doi: \href{https://doi.org/10.1038/nrn2672}{10.1038/nrn2672}.

\end{CSLReferences}



\noindent Published in\emph{
Philosophical Quarterly}, 2017, pp. 370-386.


\end{document}
