% Options for packages loaded elsewhere
\PassOptionsToPackage{unicode}{hyperref}
\PassOptionsToPackage{hyphens}{url}
%
\documentclass[
  10pt,
  letterpaper,
  DIV=11,
  numbers=noendperiod,
  twoside]{scrartcl}

\usepackage{amsmath,amssymb}
\usepackage{setspace}
\usepackage{iftex}
\ifPDFTeX
  \usepackage[T1]{fontenc}
  \usepackage[utf8]{inputenc}
  \usepackage{textcomp} % provide euro and other symbols
\else % if luatex or xetex
  \usepackage{unicode-math}
  \defaultfontfeatures{Scale=MatchLowercase}
  \defaultfontfeatures[\rmfamily]{Ligatures=TeX,Scale=1}
\fi
\usepackage{lmodern}
\ifPDFTeX\else  
    % xetex/luatex font selection
  \setmainfont[ItalicFont=EB Garamond Italic,BoldFont=EB Garamond
Bold]{EB Garamond Math}
  \setsansfont[]{Europa-Bold}
  \setmathfont[]{Garamond-Math}
\fi
% Use upquote if available, for straight quotes in verbatim environments
\IfFileExists{upquote.sty}{\usepackage{upquote}}{}
\IfFileExists{microtype.sty}{% use microtype if available
  \usepackage[]{microtype}
  \UseMicrotypeSet[protrusion]{basicmath} % disable protrusion for tt fonts
}{}
\usepackage{xcolor}
\usepackage[left=1in, right=1in, top=0.8in, bottom=0.8in,
paperheight=9.5in, paperwidth=6.5in, includemp=TRUE, marginparwidth=0in,
marginparsep=0in]{geometry}
\setlength{\emergencystretch}{3em} % prevent overfull lines
\setcounter{secnumdepth}{3}
% Make \paragraph and \subparagraph free-standing
\ifx\paragraph\undefined\else
  \let\oldparagraph\paragraph
  \renewcommand{\paragraph}[1]{\oldparagraph{#1}\mbox{}}
\fi
\ifx\subparagraph\undefined\else
  \let\oldsubparagraph\subparagraph
  \renewcommand{\subparagraph}[1]{\oldsubparagraph{#1}\mbox{}}
\fi


\providecommand{\tightlist}{%
  \setlength{\itemsep}{0pt}\setlength{\parskip}{0pt}}\usepackage{longtable,booktabs,array}
\usepackage{calc} % for calculating minipage widths
% Correct order of tables after \paragraph or \subparagraph
\usepackage{etoolbox}
\makeatletter
\patchcmd\longtable{\par}{\if@noskipsec\mbox{}\fi\par}{}{}
\makeatother
% Allow footnotes in longtable head/foot
\IfFileExists{footnotehyper.sty}{\usepackage{footnotehyper}}{\usepackage{footnote}}
\makesavenoteenv{longtable}
\usepackage{graphicx}
\makeatletter
\def\maxwidth{\ifdim\Gin@nat@width>\linewidth\linewidth\else\Gin@nat@width\fi}
\def\maxheight{\ifdim\Gin@nat@height>\textheight\textheight\else\Gin@nat@height\fi}
\makeatother
% Scale images if necessary, so that they will not overflow the page
% margins by default, and it is still possible to overwrite the defaults
% using explicit options in \includegraphics[width, height, ...]{}
\setkeys{Gin}{width=\maxwidth,height=\maxheight,keepaspectratio}
% Set default figure placement to htbp
\makeatletter
\def\fps@figure{htbp}
\makeatother
% definitions for citeproc citations
\NewDocumentCommand\citeproctext{}{}
\NewDocumentCommand\citeproc{mm}{%
  \begingroup\def\citeproctext{#2}\cite{#1}\endgroup}
\makeatletter
 % allow citations to break across lines
 \let\@cite@ofmt\@firstofone
 % avoid brackets around text for \cite:
 \def\@biblabel#1{}
 \def\@cite#1#2{{#1\if@tempswa , #2\fi}}
\makeatother
\newlength{\cslhangindent}
\setlength{\cslhangindent}{1.5em}
\newlength{\csllabelwidth}
\setlength{\csllabelwidth}{3em}
\newenvironment{CSLReferences}[2] % #1 hanging-indent, #2 entry-spacing
 {\begin{list}{}{%
  \setlength{\itemindent}{0pt}
  \setlength{\leftmargin}{0pt}
  \setlength{\parsep}{0pt}
  % turn on hanging indent if param 1 is 1
  \ifodd #1
   \setlength{\leftmargin}{\cslhangindent}
   \setlength{\itemindent}{-1\cslhangindent}
  \fi
  % set entry spacing
  \setlength{\itemsep}{#2\baselineskip}}}
 {\end{list}}
\usepackage{calc}
\newcommand{\CSLBlock}[1]{\hfill\break\parbox[t]{\linewidth}{\strut\ignorespaces#1\strut}}
\newcommand{\CSLLeftMargin}[1]{\parbox[t]{\csllabelwidth}{\strut#1\strut}}
\newcommand{\CSLRightInline}[1]{\parbox[t]{\linewidth - \csllabelwidth}{\strut#1\strut}}
\newcommand{\CSLIndent}[1]{\hspace{\cslhangindent}#1}

\setlength\heavyrulewidth{0ex}
\setlength\lightrulewidth{0ex}
\usepackage[automark]{scrlayer-scrpage}
\clearpairofpagestyles
\cehead{
  Brian Weatherson
  }
\cohead{
  Four Problems in Decision Theory
  }
\ohead{\bfseries \pagemark}
\cfoot{}
\makeatletter
\newcommand*\NoIndentAfterEnv[1]{%
  \AfterEndEnvironment{#1}{\par\@afterindentfalse\@afterheading}}
\makeatother
\NoIndentAfterEnv{itemize}
\NoIndentAfterEnv{enumerate}
\NoIndentAfterEnv{description}
\NoIndentAfterEnv{quote}
\NoIndentAfterEnv{equation}
\NoIndentAfterEnv{longtable}
\NoIndentAfterEnv{abstract}
\renewenvironment{abstract}
 {\vspace{-1.25cm}
 \quotation\small\noindent\rule{\linewidth}{.5pt}\par\smallskip
 \noindent }
 {\par\noindent\rule{\linewidth}{.5pt}\endquotation}
\KOMAoption{captions}{tableheading}
\makeatletter
\@ifpackageloaded{caption}{}{\usepackage{caption}}
\AtBeginDocument{%
\ifdefined\contentsname
  \renewcommand*\contentsname{Table of contents}
\else
  \newcommand\contentsname{Table of contents}
\fi
\ifdefined\listfigurename
  \renewcommand*\listfigurename{List of Figures}
\else
  \newcommand\listfigurename{List of Figures}
\fi
\ifdefined\listtablename
  \renewcommand*\listtablename{List of Tables}
\else
  \newcommand\listtablename{List of Tables}
\fi
\ifdefined\figurename
  \renewcommand*\figurename{Figure}
\else
  \newcommand\figurename{Figure}
\fi
\ifdefined\tablename
  \renewcommand*\tablename{Table}
\else
  \newcommand\tablename{Table}
\fi
}
\@ifpackageloaded{float}{}{\usepackage{float}}
\floatstyle{ruled}
\@ifundefined{c@chapter}{\newfloat{codelisting}{h}{lop}}{\newfloat{codelisting}{h}{lop}[chapter]}
\floatname{codelisting}{Listing}
\newcommand*\listoflistings{\listof{codelisting}{List of Listings}}
\makeatother
\makeatletter
\makeatother
\makeatletter
\@ifpackageloaded{caption}{}{\usepackage{caption}}
\@ifpackageloaded{subcaption}{}{\usepackage{subcaption}}
\makeatother
\ifLuaTeX
  \usepackage{selnolig}  % disable illegal ligatures
\fi
\usepackage{bookmark}

\IfFileExists{xurl.sty}{\usepackage{xurl}}{} % add URL line breaks if available
\urlstyle{same} % disable monospaced font for URLs
\hypersetup{
  pdftitle={Four Problems in Decision Theory},
  pdfauthor={Brian Weatherson},
  hidelinks,
  pdfcreator={LaTeX via pandoc}}

\title{Four Problems in Decision Theory}
\author{Brian Weatherson}
\date{2024}

\begin{document}
\maketitle
\begin{abstract}
In recent years the literature on decision theory has become disjointed.
There isn't as much discussion as there should be on how different
problems impact one another. This paper aims to bring together work on
problems involving demons, problems about attitudes to risk, problems
about incomplete preferences, and problems about dynamic choice. In the
first three of these cases, I end up defending a pre-existing view. I
defend a ratificationist approach to problems with demons, the orthodox
expected utility approach to risk, and the permissibility of incomplete
preferences. These views are familiar, but seeing how they are related
to a common strengthens the case for each of them. The most novel part
of the view is the theory of dynamic choice that I offer: a sequence of
choices is rational only if both the so-called `resolute' and
`sophisticated' theories of dynamic choice would permit it. This theory
would be implausible if paired with many rival solutions to the first
three problems, but fits nicely with the view I'll develop through the
paper.
\end{abstract}

\setstretch{1.1}
Contemporary decision theory has become disjointed. There is less
overlap than there should be in work on adjacent problems. This paper
aims to undo some of that, by showing that four problems that have
largely been worked on in isolation cast useful light on each other. In
particular, I'll argue that we can go a long way towards solving all
four problems by working through the consequences of a plausible
principle that I'll call the Single Choice Principle.

The Single Choice Principle (hereafter, SCP) relates theories of static
choice and dynamic choice. In particular, it says that for a narrow
class of games, it doesn't matter whether you think of the game as
involving a static, strategic choice, or a dynamic choice that is made
during a game. One way into the principle is to think about an oddity in
the way Newcomb's Problem is normally introduced.

\section{Newcomb's Problem}\label{sec-newcomb}

\subsection{Standard Version}\label{sec-newcomb-standard}

In the standard vignette that goes with Newcomb's Problem
(\citeproc{ref-Nozick1969}{Nozick 1969}), it is a dynamic game. The
demon makes a \emph{prediction}, and then the human (hereafter, Chooser)
makes a choice. Chooser doesn't know what Demon did, but they do know
that Demon has acted. So the natural presentation of Newcomb's Problem
is in a tree like Figure~\ref{fig-standard-newcomb}.\footnote{I'll
  assume \$1,000 is worth 1 util. I think this assumption of constant
  marginal utility is close to incoherent, and it will get relaxed
  later, but it's harmless for now.}

\subsection{Tree}

\begin{figure}

\centering{

\includegraphics{fourprob_files/figure-pdf/fig-standard-newcomb-1.png}

}

\caption{\label{fig-standard-newcomb}Newcomb's Problem.}

\end{figure}%

\subsection{Table}

\begin{longtable}[]{@{}ccc@{}}
\caption{Newcomb's Problem}\label{tbl-standard-newcomb}\tabularnewline
\toprule\noalign{}
& P1 & P2 \\
\midrule\noalign{}
\endfirsthead
\toprule\noalign{}
& P1 & P2 \\
\midrule\noalign{}
\endhead
\bottomrule\noalign{}
\endlastfoot
1 & 1000 & 0 \\
2 & 1001 & 1 \\
\end{longtable}

I'll go over the details of how to read diagrams like
Figure~\ref{fig-standard-newcomb} in Section~\ref{sec-decision-tree}.
All you need to know for now is that the game starts at the open node,
here at the top, and it moves along by the agent (Demon or Chooser)
making choices. The dotted lines around the two nodes where Chooser acts
mean that those two nodes are in the same \textbf{information set}. That
is, when Chooser is at either one of those nodes, the strongest thing
Chooser knows is that they are somewhere or other in the set.\footnote{This
  formalism only really makes sense if we presuppose the right epistemic
  logic is S5, and there are good reasons to not make that assumption in
  general (\citeproc{ref-Humberstone2016}{Humberstone 2016, 380--402}).
  For this paper we'll treat it as a simplifying assumption that really
  should be relaxed in subsequent work.} So this tree represents the
standard vignette for Newcomb's Problem. Demon makes a prediction - I'm
in general using PX for Demon predicting X - and Chooser knows that the
prediction has been made, and that either P1 or P2 happened, but chooses
without knowing which it is. Then the game is resolved.

What Table~\ref{tbl-standard-newcomb} shows is a subtly different story.
In Table~\ref{tbl-standard-newcomb}, each player chooses a
\emph{strategy}. A strategy for a player in a tree like
Figure~\ref{fig-standard-newcomb} is a decision about what to do at each
node in the tree where that player has to move.\footnote{In game theory,
  it is usually specified that strategies include decisions about what
  to do at nodes that are ruled out by earlier moves in that very
  strategy. In theory I'm assuming this whenever I talk about
  strategies; in practice it doesn't matter for any application in this
  paper.} So what Table~\ref{tbl-standard-newcomb} represents is a
situation where each player chooses a strategy simultaneously, and that
determines a result for the game. It differs from
Figure~\ref{fig-standard-newcomb} in part in that it's symmetric; there
is no hint that Demon moves first.

There is a lot of disagreement about Newcomb's Problem, but here is one
point of universal agreement: Figure~\ref{fig-standard-newcomb} and
Table~\ref{tbl-standard-newcomb} have the same solutions. It would be
incoherent to prefer taking 1 box in one of these puzzles and 2 boxes in
the other, or to say that both options were choice-worthy in one puzzle
but not the other. They may not represent exactly the same problem, they
don't pose exactly the same question to Chooser, but they should get the
same answer (or answers).

I'm going to agree with the unanimous verdict on this point, but I'll
start dissenting from orthodox opinion very soon. And one way into my
dissent is to ask, why should Figure~\ref{fig-standard-newcomb} and
Table~\ref{tbl-standard-newcomb} get the same answer? What principle is
someone who gives different answers to the two questions violating? I
have a suggestion for what principle that might be, the SCP, but to make
that suggestion plausible we need a couple more examples.

\subsection{Variant 1: Coin-then-Demon}\label{variant-1-coin-then-demon}

Consider a variant on Newcomb's Problem I'll call Coin-Then-Demon. In
this game a fair coin will be flipped and shown to Demon and Chooser. If
it lands Heads, Chooser will get \$5,000 and the game ends. Otherwise,
they play standard Newcomb Problem. Figure~\ref{fig-coin-then-demon}
shows the game tree for this game, with Nature moving first, and the
probabilities of Nature's moves shown. And
Table~\ref{tbl-coin-then-demon} shows the strategy table for it, with
the payouts shown in expected value.\footnote{I will drop the assumption
  that Chooser maximises expected value in Section~\ref{sec-buchak}, but
  it's a harmless assumption for now.}

\subsection{Tree}

\begin{figure}

\centering{

\includegraphics{fourprob_files/figure-pdf/fig-coin-then-demon-1.png}

}

\caption{\label{fig-coin-then-demon}Coin-then-Demon}

\end{figure}%

\subsection{Table}

\begin{longtable}[]{@{}ccc@{}}
\caption{Coin-then-Demon}\label{tbl-coin-then-demon}\tabularnewline
\toprule\noalign{}
& P1 & P2 \\
\midrule\noalign{}
\endfirsthead
\toprule\noalign{}
& P1 & P2 \\
\midrule\noalign{}
\endhead
\bottomrule\noalign{}
\endlastfoot
1 & 502.5 & 2.5 \\
2 & 503 & 3 \\
\end{longtable}

I have two hypotheses about
Figure~\ref{fig-coin-then-demon}/Table~\ref{tbl-coin-then-demon}; one of
which I think everyone will agree with, and one that might be more
controversial. The less controversial hypothesis is that in this game,
as in standard Newcomb's Problem, it doesn't matter whether Chooser is
playing the dynamic game (i.e., Figure~\ref{fig-coin-then-demon}) or the
strategic game (i.e., Table~\ref{tbl-coin-then-demon}). Whichever
options are choice-worthy in one are choice-worthy in the other. The
more controversial hypothesis is that the reason these two games are
rationally equivalent is exactly the same as the reason that the two
forms of Newcomb Problem I presented should get the same answer.

\subsection{Variant 2: Demon-then-Coin}\label{variant-2-demon-then-coin}

One more example and we're basically done. In the game I'll call
Demon-Then-Coin, the coin is only flipped if Demon predicts Chooser
takes one box. If the coin lands heads, Chooser gets \$5,000, and the
game ends. If either Demon predicts 2 boxes, or the coin lands tails,
Chooser makes a selection, knowing that one or other of these disjuncts
obtained. Then the game ends. The tree for this game is
Figure~\ref{fig-demon-then-coin}, and the strategy table is
Table~\ref{tbl-demon-then-coin}.

\subsection{Tree}

\begin{figure}

\centering{

\includegraphics{fourprob_files/figure-pdf/fig-demon-then-coin-1.png}

}

\caption{\label{fig-demon-then-coin}Demon-then-Coin}

\end{figure}%

\subsection{Table}

\begin{longtable}[]{@{}ccc@{}}
\caption{Demon-then-Coin}\label{tbl-demon-then-coin}\tabularnewline
\toprule\noalign{}
& P1 & P2 \\
\midrule\noalign{}
\endfirsthead
\toprule\noalign{}
& P1 & P2 \\
\midrule\noalign{}
\endhead
\bottomrule\noalign{}
\endlastfoot
1 & 502.5 & 0 \\
2 & 503 & 1 \\
\end{longtable}

If Chooser was planning on picking 1 box, they have a little evidence
against the accuracy of Demon's predictions. If in the other games they
thought the probability that Demon mispredicted was \emph{e}, in this
case they should (if they plan to choose 1 box) have a probability of
error of roughly 2\emph{e}. But if \emph{e} was small enough to start
with, and I'll assume throughout that Demon's error likelihood is
arbitrarily small, this shouldn't make a difference.

Again, I'm going to argue that the dynamic game,
Figure~\ref{fig-demon-then-coin}, and the strategic game,
Table~\ref{tbl-demon-then-coin}, should get the same solutions. Indeed,
they should get the same solutions for the same reason the previous two
pairs of decisions should get the same solutions. That reason, I'll
argue, is the Single Choice Principle.

\subsection{Single Choice Principle}\label{sec-scp-definition}

Here is what the Single Choice Principle (hereafter, SCP) says:

\begin{quote}
\textbf{Single Choice Principle (SCP)}\\
In any decision tree in which all the nodes where Chooser acts are in a
single information set, an option is choice-worthy in the dynamic form
of the game iff it is choice-worthy in the strategic form of the game.
\end{quote}

The SCP is a highly restricted version of a claim that dynamic and
static games are in some sense equivalent. The strong version of the
view says that there is some mapping from the set of rational choices in
a tree to the set of possible choices in the strategic version of that
tree. Exactly how that mapping should be understood is tricky in the
general case, but since (a) the general principle is extremely
controversial, and (b) I'm not endorsing the general principle, I won't
fuss over the details. What I will fuss over is getting clearer about
what the SCP does and doesn't say.

The SCP doesn't just say that on any run through the game, Chooser only
makes one choice. Rather, it says that Chooser only has one possible
choice to make in the game. This point might be clearer with an example.
Imagine Chooser and Demon are playing a simple kind of ultimatum game.
Demon has to propose a split of a \$3 pot; they can either propose \$2
for Demon and \$1 for Chooser, or vice versa. Chooser then has a take it
or leave it choice. If they take, each party gets the money Demon
proposes; if they leave, each party gets \$0. Assume Demon is
arbitrarily good at predicting Chooser's strategy, and that Demon
prefers more money to less\footnote{Also assume Demon will flip a coin
  if they expect each option to have equal return}. The game tree is in
Figure~\ref{fig-ultimatum}, and the strategy table is in
Table~\ref{tbl-ultimatum}.

\subsection{Tree}

\begin{figure}

\centering{

\includegraphics{fourprob_files/figure-pdf/fig-ultimatum-1.png}

}

\caption{\label{fig-ultimatum}Ultimatum Game}

\end{figure}%

\subsection{Table}

\begin{table}

\caption{\label{tbl-ultimatum}Two representations of the strategic form
of ultimatum game}

\begin{minipage}{0.50\linewidth}

\subcaption{\label{tbl-ultimatum-game}Demon's Decisions}

\centering{

\begin{tabular}{ccc}
\toprule
 & D2C1 & D1C2\\
\midrule
\textbf{TT} & 1 & 2\\
\textbf{TL} & 1 & 0\\
\textbf{LT} & 0 & 2\\
\textbf{LL} & 0 & 0\\
\bottomrule
\end{tabular}

}

\end{minipage}%
%
\begin{minipage}{0.50\linewidth}

\subcaption{\label{tbl-ultimatum-demon}Demon's Predictions}

\centering{

\begin{tabular}{ccccc}
\toprule
 & PTT & PTL & PLT & PLL\\
\midrule
\textbf{TT} & 1 & 1 & 2 & 1.5\\
\textbf{TL} & 1 & 1 & 0 & 0.5\\
\textbf{LT} & 0 & 0 & 2 & 1\\
\textbf{LL} & 0 & 0 & 0 & 0\\
\bottomrule
\end{tabular}

}

\end{minipage}%

\end{table}%

Most philosophers would say that in the dynamic form of the game,
Figure~\ref{fig-ultimatum}, the only sensible thing to do is TT;
whatever the demon does, it's better to take more money than less. But
many would also say that in the strategic form,
Table~\ref{tbl-ultimatum}, some other strategy might be appropriate. For
instance, Evidential Decision Theory says that in
Table~\ref{tbl-ultimatum}, the right strategy is LT.\footnote{This is
  easier to see in Table~\ref{tbl-ultimatum-demon}; EDT says to just
  look at the numbers in the main diagonal and choose the strategy with
  the highest one.} The SCP does not rule out this combination. It will
ultimately have something to say about EDT, but it doesn't object to
this pair of views. That's because in Figure~\ref{fig-ultimatum} there
are two possible choices for Chooser to make, even if they will
ultimately only make one of them, and the SCP only applies to games with
just one possible choice. That makes it a more plausible principle, but
surprisingly does little to reduce its philosophical significance.

\section{Defending the SCP}\label{sec-scp-defence}

\subsection{A Sample Violation}\label{sec-sample-violation}

The argument for the SCP is that violations of it are in various ways
incoherent. It helps to have a sample violation on the table. Imagine
Chooser is going to play the following game.

\begin{figure}

\centering{

\includegraphics{fourprob_files/figure-pdf/fig-sample-violation-1.png}

}

\caption{\label{fig-sample-violation}Switching Example}

\end{figure}%

Figure~\ref{fig-sample-violation} resembles
Figure~\ref{fig-demon-then-coin}, with two notable differences. First,
the coin is now weighted, and has an 8/9 chance of landing Heads.
Second, if Chooser must choose, either option is an equilibrium.

Table~\ref{tbl-sample-violation-late} shows the decision table Chooser
faces if they must make a choice in Figure~\ref{fig-sample-violation},
and Table~\ref{tbl-sample-violation-early} shows the expected payouts of
the two strategies Chooser could select.

\begin{table}

\caption{\label{tbl-sample-violation}Payout tables for
Figure~\ref{fig-sample-violation}.}

\begin{minipage}{0.50\linewidth}

\subcaption{\label{tbl-sample-violation-late}Dynamic version.}

\centering{

\begin{tabular}{ccc}
\toprule
 & PU & PD\\
\midrule
\textbf{Up} & 72 & 36\\
\textbf{Down} & 18 & 54\\
\bottomrule
\end{tabular}

}

\end{minipage}%
%
\begin{minipage}{0.50\linewidth}

\subcaption{\label{tbl-sample-violation-early}Strategic version.}

\centering{

\begin{tabular}{ccc}
\toprule
 & PU & PD\\
\midrule
\textbf{Up} & 48 & 36\\
\textbf{Down} & 42 & 54\\
\bottomrule
\end{tabular}

}

\end{minipage}%

\end{table}%

I'll take as my sample violator of the SCP a Chooser who prefers Up in
the dynamic version, and Down in the strategic version. As we'll see in
Section~\ref{sec-multiple}, many decision theorists agree with this
Chooser. But everything I say should generalise to any violation.

\subsection{Ramsey Test}\label{sec-ramsey}

To choose a strategy is to make true a bunch of conditionals. Adopting
the strategy Down in Figure~\ref{fig-sample-violation} just is saying
``If I have to choose, I'll choose Down''. As Ramsey
(\citeproc{ref-RamseyGeneralProp}{{[}1929{]} 1990}) said, the way to
tell which such conditional to make true is to hypothetically add the
antecedent to one's stock of belief, and then decide which unconditional
claim you'd like to make true. But Chooser does not do that. If they
believed that they had to choose, they would choose Up. So their
strategic preference implies that they believe that if they had to
choose, they would choose Down, but adding the supposition that they
have to choose, they choose Up. This combination is incoherent, and so
violations of the SCP are incoherent.

I think this argument is decisive; it's incoherent to adopt a strategy
in games like Figure~\ref{fig-demon-then-coin} or
Figure~\ref{fig-sample-violation} that is different from what one knows
one would do if one had to carry the strategy out. That's just not how
conditionals work. But in case not everyone is convinced, I'll run
through some other arguments. The SCP will do a lot of work, and it is
worth getting the foundations as secure as possible.

\subsection{Intuitions about Change}\label{sec-intuitions-change}

This argument starts with a story. Imagine the game master (GM) is
chatting to Chooser (C), before Chooser plays
Figure~\ref{fig-sample-violation}.

\begin{quote}
GM: What are you thinking of playing?\\
C: I might not have a choice.\\
GM: True, but assume you have to choose.\\
C: Then Up, I guess.\\
GM: You know, Demon can be really slow in making a prediction. Do you
want to write your choice down in an envelope, and we'll open it if it's
needed? C: Oh sure. I'm writing Down.\\
GM: Why did you change your mind?
\end{quote}

We could continue the conversation, but I want to focus on the
presupposition of GM's last question. It seems appropriate to presuppose
here that Chooser has changed their mind. This presupposition requires
the SCP. If the SCP is false, Chooser has simply given different answers
to different questions. First they were asked what to do in the dynamic
game, and they said Up. Then they were asked what to do in the strategic
game, and they said Down. Giving different answers to different
questions is not changing one's mind.

GM's question seems appropriate. Chooser was first asked what they
planned to do in a particular situation. Then they were asked for a
strategy that would only be activated in that very situation. When they
give different answers to those questions, it sounds like they changed
their mind. That implies the questions are fundamentally the same, which
is what the SCP says.

\subsection{Unifying the Examples}\label{sec-unity}

Philosophers do not agree about Newcomb's Problem. But they do agree
that in each of the examples in Section~\ref{sec-newcomb}, the same
choice is rational in the strategic and dynamic form of the game. This
isn't because they think that in general strategic and dynamic forms are
equivalent. Indeed, for many theorists there is no unifying story about
why each of these pairs of problems gets the same answer. It is just a
fact that the theory treats the dynamic and strategic problems the same
way.

I think it is better to have an explanation for why each of the pairs of
problems gets the same answers, and for that explanation to be the same
across the three pairs. The SCP provides such an explanation, and that's
a point in its favour.

\subsection{No Reward}\label{sec-no-reward}

One reason that I introduced
Figure~\ref{fig-ultimatum}/Table~\ref{tbl-ultimatum} earlier is that
it's the kind of case where it's most plausible that the strategic and
dynamic choices might be distinct. Think about the pair of choices that
EDT recommends: in the dynamic game, play TT; in the strategic game,
play LT. While I ultimately disagree with this, I do think this is a
plausible thing for EDT to say. The strategy LT has two big advantages
that some will think make up for the fact that it is not what one would
do dynamically. First, it differs from the dynamically rational play
only in a situation which is, conditional on being played, highly
unlikely to come about. Second, there is a reward, at least in
expectation, for playing this dynamically irrational strategy; the
strategy has a payout of 2 while the dynamically rational strategy only
has a payout of 1. Either one of these facts will, at least to some
people, make it rational to treat dynamic and strategic games
differently; what's distinctive about
Figure~\ref{fig-ultimatum}/Table~\ref{tbl-ultimatum} is that both
reasons are there.

In dynamic problems where the SCP applies, neither of these reasons can
apply. There is no possible strategic advantage to playing the
dynamically irrational strategy. There is no parallel to saying, ``I'm
playing LT because, even though leaving money would be irrational, I
almost certainly won't have to carry that part of the strategy out, and
in exchange for this tiny risk, I'm getting rewarded.'' Doing something
dynamically irrational at the only point one can possibly move can't
have advantages elsewhere; you're going to get the same payout elsewhere
no matter what. So whatever reason one could have in other cases for
treating dynamic and strategic problems separately can't apply here;
there isn't enough of an `elsewhere' for one's bad decision at the one
and only place one moves to be compensated.

\subsection{Game Theory}\label{sec-game-theory}

Assume, following Harper (\citeproc{ref-Harper1986}{1986}), that
decision problems involving Chooser and Demon are \emph{games}. For
Demon, they are coordination games; Demon acts as if they get rewarded
for making the `same' play as Chooser. Chooser's payouts can be more
complicated, but the examples decision theorists use usually track some
familiar game.\footnote{Even before Harper, Lewis
  (\citeproc{ref-Lewis1979en}{1979}) had pointed out the connection
  between Newcomb's Problem and Prisoners' Dilemma.} Further assume,
following all the standard textbooks, that the theory of rational choice
for game players is based around some equilibrium concept.\footnote{I'm
  including rationalizability, as defined by Bernheim
  (\citeproc{ref-Bernheim1984}{1984}) and Pearce
  (\citeproc{ref-Pearce1984}{1984}), as an equilibrium concept.} Then
rational players will follow the SCP. For all the standard equilibrium
concepts, the equilibria of the dynamic and strategic games are the same
if each player only has one possible choice.

\subsection{Sure Thing}\label{sec-surething}

Finally, there is one very natural argument for the SCP that for various
reasons I don't want to lean on too heavily. If Chooser violates the
SCP, then they violate the Sure Thing Principle. They think Up is at
least as good as Down both conditional on the game ending without them
making a choice, and on that not happening. But they think Down is
better overall. If Sure Thing can be taken as a basic assumption, the
SCP immediately follows.

There are three problems with this line of reasoning. The first is
pragmatic. It's well known that various theories I'm arguing against
here, like EDT, and Buchak's non-standard treatment of risk, violate
Sure Thing. It's not a new argument against them to say that they
violate the SCP, if the only reason to believe the SCP is Sure Thing.
The second is that it isn't obvious that the theories that are best
supported by the SCP are not consistent with Sure Thing. Dmitri Gallow
(\citeproc{ref-Gallownd}{n.d.}) argues that what he calls `stable'
decision theories are bound to violate Sure Thing. It's arguable that
his arguments can be generalised to provide a reason to think theories
supported by the SCP (which will typically not be stable in his sense)
also violate Sure Thing. And the third is that even if this argument
fails, there is a bad company objection to Sure Thing that I'll get to
in Section~\ref{sec-negdom}. So it's useful that we have the other four
arguments to fall back on.

\section{Flagging Assumptions}\label{sec-flagging}

I'm going to make three assumptions in what follows, and it helps to
have them on the table.

One is that we can sensibly talk about demons who are arbitrarily
accurate at predicting Chooser's strategy. One reason for making this
assumption is that all the problems we discuss could be rephrased if we
just assumed Demon was at least epsilon better than chance at predicting
Chooser's strategy, and this is a realistic assumption. It would
complicate the algebra considerably in what follows to do this, without
making the examples clearer. A second reason for making the assumption
is that standard approaches to game theory assume each player is a demon
who can predict the other players' strategies with arbitrary precision,
so we're just deferring to orthodox opinion in a notable research topic
by making this assumption.

A second is that decision problems are fully specified by setting out
the states (assumed to be causally independent of actions), the
available actions, the payoffs for each state-action pair, and the
conditional probability of each state given each action. Here I'm
following Gallow (\citeproc{ref-Gallownd}{n.d.}), whose formalism for
decision problems only includes places for these variables. The primary
motivation for this assumption is that it is common in the literature to
describe a decision problem in a way that only specifies these factors -
the state-action payoffs and the state probabilities conditional on
actions - and presupposes without comment that enough has been said to
specify which actions ae rational. So I think this assumption is rather
widespread, if often implicit. Still, it is a substantial assumption.
Some decision theorists think that which choices are rational turns on
other factors, such as Chooser's prior unconditional probabilities for
various states, or Chooser's attitude to risk. I'll note below where
this assumption matters, but in general I'll be following Gallow in
making this assumption.

And a third is that in any dynamic choice, a choice at a point is
permissible only if it would be permissible were that starting point of
a decision problem. This rules out so-called \emph{sophisticated}
approaches to dynamic choice. And I'll come back in
Section~\ref{sec-dual-mandate} to what happens if this assumption is
relaxed.

\section{Problem 1: Demons and Multiple Equilibria}\label{sec-multiple}

Table~\ref{tbl-generic-demon} is a completely generic form of a 2*2
decision problem involving demons. Without loss of generality, I'll
assume all payouts are positive. I'll also assume all payouts are
distinct; dropping this requires fussing about edge cases that are not
relevant here.\footnote{These edge cases are important for thinking
  about the significance of weak dominance, but that's not relevant to
  this paper.}

\begin{longtable}[]{@{}ccc@{}}
\caption{A generic demon
problem}\label{tbl-generic-demon}\tabularnewline
\toprule\noalign{}
& PU & PD \\
\midrule\noalign{}
\endfirsthead
\toprule\noalign{}
& PU & PD \\
\midrule\noalign{}
\endhead
\bottomrule\noalign{}
\endlastfoot
\textbf{Up} & \emph{a} & \emph{b} \\
\textbf{Down} & \emph{c} & \emph{d} \\
\end{longtable}

Say that an option is a (strict) equilibrium if, assuming Demon predicts
correctly, Chooser's payout for choosing it is (strictly) greater than
their payout for choosing any other option. The focus of this section is
only problems where both Up and Down are strict equilibria. In
particular, focus on problems that satisfy these three constraints.

\begin{enumerate}
\def\labelenumi{\arabic{enumi}.}
\tightlist
\item
  \emph{a} \textgreater{} \emph{c}, and \emph{d} \textgreater{}
  \emph{b}.
\item
  \emph{a} \textgreater{} \emph{d}.
\item
  \emph{b} \textgreater{} \emph{c}.
\end{enumerate}

The first says that Up and Down are both strict equilibria. The second
says that Up has the highest payout among the equilibria. The third says
that Up has a higher off-equilibrium payout than Down does. Many
theorists who disagree about other questions in decision theory say that
these three facts suffice to make Up uniquely choice-worthy.

Evidential Decision Theorists say that 2 alone suffices for choosing Up
over Down.

Consider next the theory that says only equilibria are choice-worthy,
and among equilibria, one should choose the equilibrium with the highest
expected payout. Versions of this theory are endorsed by Jeffrey
(\citeproc{ref-Jeffrey1983b}{1983}), Arntzenius
(\citeproc{ref-Arntzenius2008}{2008}), and Gustafsson
(\citeproc{ref-Gustafsson2011}{2011}). Given 1 and 2, this theory says
to choose up.

Finally, consider the theory that says one should (in two options games)
minimise possible regret. That is, one should choose Up if the possible
Regret from choosing Up, \emph{d}~-~\emph{b}, is less than the possible
regret from choosing Down, \emph{a}~-~\emph{c}. Wedgwood
(\citeproc{ref-Wedgwood2013a}{2013}), Gallow
(\citeproc{ref-Gallow2020}{2020}), Podgorski
(\citeproc{ref-Podgorski2022}{2022}), and Barnett
(\citeproc{ref-Barnett2022}{2022}) endorse this claim, though they go on
to say very different things about cases with three or more options.
Given constraints 2 and 3, these theories also say to choose Up.

I'll argue that in any such problem, both Up and Down are choice-worthy.
I'm not the first to say this. Jack Spencer
(\citeproc{ref-Spencer2021b}{2021}) and Melissa Fusco
(\citeproc{ref-Fuscond}{n.d.}) also say that both equilibria are
choice-worthy.\footnote{Fusco and Spencer disagree on a lot of other
  questions, and I think the SCP tends to favour Fusco's side of their
  disagreement, but it would be a huge digression to sort out those
  details.} This implies that in any problem where Up and Down are both
strict equilibria, they are both choice-worthy, since there is nothing
more that could make Up uniquely choice-worthy. Just what is the
relationship between being an equilibrium and being choice-worthy is a
question for another day, but in 2*2 games, they pick out the same
options.

Consider the dynamic game shown in Figure~\ref{fig-edt}.

\begin{figure}

\centering{

\includegraphics{fourprob_files/figure-pdf/fig-edt-1.png}

}

\caption{\label{fig-edt}A tree that shows SCP is inconsistent with
several theories.}

\end{figure}%

The payouts on the right are taken from Table~\ref{tbl-generic-demon},
and the payout if the game exits without a choice is half way between
the two payouts if Demon predicts Down. I've left the probability of
exit if Demon predicts Up as a variable \emph{p}. The decision tables
for the one choice in the dynamic game, and the strategic choice for the
whole game, are in Table~\ref{tbl-edt}.

\begin{table}

\caption{\label{tbl-edt}Payout tables for Figure~\ref{fig-edt}.}

\begin{minipage}{0.50\linewidth}

\subcaption{\label{tbl-edt-late}Dynamic version.}

\centering{

\begin{tabular}{ccc}
\toprule
 & PU & PD\\
\midrule
\textbf{Up} & \emph{a} & \emph{b}\\
\textbf{Down} & \emph{c} & \emph{d}\\
\bottomrule
\end{tabular}

}

\end{minipage}%
%
\begin{minipage}{0.50\linewidth}

\subcaption{\label{tbl-edt-early}Strategic version.}

\centering{

\begin{tabular}{ccc}
\toprule
 & PU & PD\\
\midrule
\textbf{Up} & \emph{p}(\emph{b}+\emph{d})/2 +
(1-\emph{p})\emph{a} & \emph{b}\\
\textbf{Down} & \emph{p}(\emph{b}+\emph{d})/2 +
(1-\emph{p})\emph{c} & \emph{d}\\
\bottomrule
\end{tabular}

}

\end{minipage}%

\end{table}%

If \emph{p} is large enough, then in Table~\ref{tbl-edt-early}, the
bottom-right will be larger than the top-left, and the bottom-left will
be larger than the top-right. So if constraints 1-3 are sufficient
conditions for one option to be uniquely choice-worthy, then if \emph{p}
is large enough, Down will be uniquely choice-worthy in
Table~\ref{tbl-edt-early}. But Up is uniquely choice-worthy in
Table~\ref{tbl-edt-late}, so this will violate the SCP.

How large does \emph{p} have to be? As long as
\emph{p}~\textgreater~(2\emph{a}~‑~2\emph{d})/(2\emph{a}~‑~\emph{b}~‑~\emph{d}),
then the bottom-left will be greater than the top-right. This is
consistent with \emph{p}~\textless~1, since
\emph{d}~\textgreater~\emph{b}. As long as
\emph{p}~\textgreater~(2\emph{b}~‑~2\emph{c})/(\emph{b}~+~\emph{d}~‑2\emph{c}),
then the bottom-right will be greater than the top-left. Again, since
\emph{d}~\textgreater~\emph{b}, this is consistent with
\emph{p}~\textless~1. We can guarantee both conditions are met if:

\[
p = 1 - \frac{(d-b)^2}{(2a-b-d)(b+d-2c)}
\]

Assume we use the values from Table~\ref{tbl-sample-violation-late}, so
\emph{a}~=~72, \emph{b}~=~36, \emph{c}~=~18, and \emph{d}~=~54. Then
this formula says that \emph{p}~=~1/9. If we set \emph{p} to 1/9, we get
the tree shown in Figure~\ref{fig-sample-violation}. What I've argued
here is that that example is perfectly general. If constraints 1-3
suffice for Up being uniquely choice-worthy, then any game where there
are two equilibria but Up is uniquely choice-worthy can be embedded in a
dynamic game where it is the game Chooser will face if they ever have to
choose, but in the strategic form of the game, Down is uniquely
choice-worthy. So saying 1-3 suffices for Up being uniquely
choice-worthy leads to systematic violations of the SCP. Since the SCP
is true, these theories must be false.

If constraints 2 and 3 don't suffice to say that a particular strict
equilibrium in a 2\emph{2 game is not choice-worthy, it's hard to see
what further constraints could make a difference. So I conclude that, if
the SCP is true, then in 2}2 games with two strict equilibria, both
options are choice-worthy. This shows that Spencer
(\citeproc{ref-Spencer2021b}{2021}) and Fusco
(\citeproc{ref-Fuscond}{n.d.}) are right about these problems.

That's a case where a theorist who thinks constraints 1-3 suffice for Up
being uniquely choice-worthy will violate the SCP. As we noted at the
start of this section, many theories do say those constraints suffice
for unique choice-worthiness, so they are all wrong.

In a 2*2 game with multiple strict equilibria, the only theory that's
compatible with the SCP, and hence the only plausible theory, is that
both equilibria are choice-worthy.

\section{Problem 2: Ordering}\label{sec-ordering}

\subsection{Introducing the Problem}\label{sec-ordering-intro}

Standard approaches to decision theory assign to Chooser a probability
function and a utility function, both defined over (some) propositions.
The domain of each function is some subset of the reals; the interval
{[}0,1{]} for the probability, and some bounded interval for the
utilities. The real numbers have a distinctive topology. Among other
things, they are totally ordered: for any two numbers, either one is
greater, or they are equal. So assuming that probabilities and utilities
are numerical involves assuming that for any two propositions, the
probability(/utility) of the first is either greater than, less than, or
equal to, that of the other. Call this assumption Ordering.

Ordering is controversial, both for probabilities and utilities. For
probabilities, it has been criticised since Keynes's \emph{Treatise on
Probability} (\citeproc{ref-Keynes1921}{1921}), and in recent times has
been criticised by, among others, Peter Walley
(\citeproc{ref-Walley1991}{1991}) and James Joyce
(\citeproc{ref-Joyce2010}{2010}). For utilities, the most prominent
contemporary critic is Ruth Chang (\citeproc{ref-Chang2002}{2002},
\citeproc{ref-Chang2015}{2015}).

Just like there are many critics of Ordering, there are many defenders.
Dorr, Nebel, and Zuehl (\citeproc{ref-DorrEtAl2023}{2023}) defend it on
semantic grounds. Adam Elga (\citeproc{ref-Elga2010}{2010}) argues that
violations of Ordering for probabilities leads to susceptibility to a
money pump. John (\citeproc{ref-Broome2001}{\textbf{Broome2001?}}) and
Johan Gustafsson (\citeproc{ref-Gustafsson2022}{2022}) make similar
arguments in favour of Ordering for utilities.

Even critics of Ordering have noted its unintuitive characteristics.
Bradley and Steele (\citeproc{ref-BradleySteele2016}{2016}) argue that
violations of Ordering for probabilities imply it can be rational to pay
to avoid information. Harvey Lederman (\citeproc{ref-Ledermannd}{2024})
argues that violations of Ordering for utilities leads to violations of
a principle he calls Negative Dominance, which I'll return to below.

Both Bradley and Steele, and Lederman, think that ultimately Ordering
should be rejected, and we should live with these unintuitive results.
They are both pointing out troubling features of their own view.
(Something philosophers should do more often.) In each case it isn't
hard to convert the argument they give to a problem for the other kind
of Ordering violation. If Ordering fails for utilities, a Bradley and
Steele-style argument shows that it is worth paying to avoid
information, and if it fails for probabilities, a Lederman style
argument shows that Negative Dominance fails.

I'm going to offer a new defence of Ordering violations. The defence has
two parts. In this section, I'll argue that even if Ordering holds for
probabilities and for values of states, it does not hold for values of
actions. A bit loosely, even if Ordering is true for preferences over
ends, it isn't true for preferences over means. This shows we have
independent reason to reject any principle that entails Ordering is true
in general. That includes many of the premises in arguments that have
been offered against Ordering. Then in Section~\ref{sec-dual-mandate},
I'm going to argue that many of the criticisms of views that permit
Ordering violations presuppose a false view about how rational dynamic
choice works.

\subsection{Argument from Sweeteners}\label{sec-sweeteners}

In Section~\ref{sec-multiple} I argued that in
Table~\ref{tbl-sample-violation-late}, both Up and Down are
choice-worthy. This is consistent with Ordering only if Chooser is
indifferent between Up and Down. If Chooser is indifferent, then
`sweetening' one of the options, by increasing its payout in all states,
should break the tie between them. But it doesn't. If we start with
Table~\ref{tbl-edt-late} and add 1 to the payout for Up, we get
Table~\ref{tbl-sweetened}. And the argument in
Section~\ref{sec-multiple} shows that both Up and Down are choice-worthy
in Table~\ref{tbl-sweetened}.

\begin{longtable}[]{@{}ccc@{}}
\caption{Table~\ref{tbl-sample-violation-late} with \textbf{Up}
sweetened by 1.}\label{tbl-sweetened}\tabularnewline
\toprule\noalign{}
& PU & PD \\
\midrule\noalign{}
\endfirsthead
\toprule\noalign{}
& PU & PD \\
\midrule\noalign{}
\endhead
\bottomrule\noalign{}
\endlastfoot
\textbf{Up} & 73 & 37 \\
\textbf{Down} & 18 & 54 \\
\end{longtable}

So the SCP implies that Ordering fails in cases like these. Up and Down
are both choice-worthy, so neither is better than the other, and they
aren't equally good.

\subsection{Argument from β}\label{sec-ux3b2}

There is something odd about using preference orderings in an empirical
theory of choice. We never observe preference orderings; we only ever
observe choices. A tradition tracing back to Samuelson
(\citeproc{ref-Samuelson1938}{1938}) says decision theory should start
with choice dispositions, not preferences.\footnote{D. Wade Hands
  (\citeproc{ref-Hands2014}{2014}) has much more on this tradition,
  focusing on Samuelson's complicated relationship to it. The SCP fits
  more neatly with a view that takes choice as analytically prior to
  utility and even to preference.} For any set of options S, let C(S) be
the set of options Chooser regards as choice-worthy.\footnote{Empirically,
  Chooser may select different members of S on different occasions.}
This need not be a radical break with the idea that decision theory is
based around preferences. Given some intuitive constraints on C, we can
generate a preference relation out of it, and that preference relation
will satisfy Ordering. It turns out, however, that the SCP is
inconsistent with some of those constraints.

Amartya Sen (\citeproc{ref-Sen1971}{1971}) noted that (given some other
intuitive but not uncontroversial constraints), Ordering is equivalent
to the following condition on C. He gave it the unmemorable name
Principle β.

\begin{description}
\tightlist
\item[β]
If S ⊆ S', and \{A, B\} ⊆ C(S), then A ∈ C(S') iff B ∈ C(S').
\end{description}

That is, if A and B are both initially choice-worthy, then after some
options are added, either they are both choice-worthy, or neither
is.\footnote{To see why this might be plausible, imagine each option has
  a numerical value, and the choice-worthy ones are those with maximal
  value.} If β fails, then either Ordering fails, or something much less
intuitive than Ordering or β fails. And the SCP is inconsistent with β.

Extend Table~\ref{tbl-sample-violation-late} so there is a third option:
eXit. This option returns 60 whatever Demon does. And if Demon predicts
eXit, they flip a fair coin to decide whether to play PU or PD. So the
table looks like Table~\ref{tbl-beta-violation}.

\begin{longtable}[]{@{}ccc@{}}
\caption{A violation of Principle
β}\label{tbl-beta-violation}\tabularnewline
\toprule\noalign{}
& PU & PD \\
\midrule\noalign{}
\endfirsthead
\toprule\noalign{}
& PU & PD \\
\midrule\noalign{}
\endhead
\bottomrule\noalign{}
\endlastfoot
\textbf{Up} & 73 & 37 \\
\textbf{Down} & 18 & 54 \\
\textbf{eXit} & 60 & 60 \\
\end{longtable}

In this table, Down is not choice-worthy, since it is dominated. So
although the SCP says that C(\{U,~D\})~=~(\{U,~D\}),
C(\{U,~D,~X\})~=~(\{U,~X\}), violating β. So the SCP is inconsistent
with Ordering.

\subsection{Argument from Dominance}\label{sec-negdom}

Harvey Lederman (\citeproc{ref-Ledermannd}{2024}) notes that given some
other plausible assumptions, Ordering is inconsistent with a principle
he calls Negative Dominance.\footnote{Though note that, like Sen,
  Lederman raises doubts about these `other plausible assumptions'. Note
  also that I've rephrased his principle a little to match the notation
  of this paper.}

\begin{quote}
\textbf{Negative Dominance}\\
If \emph{p} is a lottery proposition, and A~\textgreater~B, then either
A~\textgreater{}\textsubscript{\emph{p}}~B, or
A~\textgreater{}\textsubscript{¬\emph{p}}~B.
\end{quote}

In \textbf{Negative Dominance}, \textgreater{} is a strict preference
relation, and \textgreater{}\textsubscript{\emph{p}} is the preference
relation conditional on \emph{p}. The principle says that if A is
strictly preferred to B, it also must be strictly preferred conditional
on at least one outcome of a lottery. Most theories that violate
Ordering violate Negative Dominance, and conversely most theories that
violate Negative Dominance violate Ordering. Given the SCP, it seems
plausible Negative Dominance fails. Let H be that a particular fair coin
lands Heads, T that it lands Tails, H8 a bet that pays 8 if H and 0
otherwise, and T8 a bet that pays 8 if T and 0 otherwise. In
Table~\ref{tbl-negdom}, Chooser can play Up, Down or eXit, and Demon has
made an arbitrarily accurate prediction. PU means they predicted Up; PDX
means they predicted Down or eXit.

\begin{longtable}[]{@{}ccc@{}}
\caption{A violation of Negative
Dominance}\label{tbl-negdom}\tabularnewline
\toprule\noalign{}
& PU & PDX \\
\midrule\noalign{}
\endfirsthead
\toprule\noalign{}
& PU & PDX \\
\midrule\noalign{}
\endhead
\bottomrule\noalign{}
\endlastfoot
\textbf{Up} & 1+H8 & 0 \\
\textbf{Down} & 0 & 1+T8 \\
\textbf{eXit} & 2 & 2 \\
\end{longtable}

This is a violation of Negative Dominance given these four assumptions.

\begin{enumerate}
\def\labelenumi{\arabic{enumi}.}
\tightlist
\item
  A strategy is not choice-worthy if it is dominated, including if it is
  dominated by a mixed strategy.
\item
  If A is choice-worthy and B is not, then A~\textgreater~B.
\item
  If B is choice-worthy and A is available, then ¬(A~\textgreater~B).
\item
  If B strictly dominates A, then ¬(A~\textgreater~B).
\end{enumerate}

In Table~\ref{tbl-negdom}, eXit is dominated by a 50/50 mix of Up and
Down, so it is not choice-worthy (by 1). Since Up is choice-worthy,
U~\textgreater~X (by 2). Conditional on Heads, Table~\ref{tbl-negdom}
becomes Table~\ref{tbl-negdom-heads}.

\begin{longtable}[]{@{}ccc@{}}
\caption{Table~\ref{tbl-negdom} conditional on
Heads}\label{tbl-negdom-heads}\tabularnewline
\toprule\noalign{}
& PU & PDX \\
\midrule\noalign{}
\endfirsthead
\toprule\noalign{}
& PU & PDX \\
\midrule\noalign{}
\endhead
\bottomrule\noalign{}
\endlastfoot
\textbf{Up} & 9 & 0 \\
\textbf{Down} & 0 & 1 \\
\textbf{eXit} & 2 & 2 \\
\end{longtable}

In Table~\ref{tbl-negdom-heads}, Down is dominated and so eliminated.
And post-deletion, the arguments in Section~\ref{sec-multiple} show both
options are choice-worthy. So ¬(U~\textgreater~X) (by 3). And if Tails,
eXit strictly dominates Up, so ¬(U~\textgreater~X) (by 4). Hence given
the SCP, and these four assumptions, Negative Dominance fails.

\subsection{Conclusion}\label{sec-conclusion}

In general, it isn't easy to convert a theory of choice-worthiness into
a theory of preference. There are some special cases when there is a
natural way to do this, but they mostly rely on principles like β
holding. Still, the arguments in this section suggest that however we
convert a theory of choice consistent with the SCP into a theory of
preference, Ordering will fail. And it will fail even if Ordering holds
for credences (contra Keynes), and holds for values (contra Chang).

The philosophical significance of this is that it shows several
objections to views like Keynes's and Chang's over-generate, and hence
are mistaken. Any argument that shows all Ordering violations are
problematic entails that the SCP is false. But the arguments for the
RCP, especially the Ramsey Test argument from Section~\ref{sec-ramsey},
are more compelling than almost all the alleged problems for Ordering
violations. So even though I've assumed in this section that Ordering
holds for credences and values, the arguments here show that this
assumption is not in general warranted. As long as there are other
agents around (either Demons or rational human game players), Ordering
will fail, so general objections to Ordering must be wrong.

I said `almost all' in the previous paragraph because there is one
objection to Ordering violations that the arguments of this section
don't touch: the objection from Elga (\citeproc{ref-Elga2010}{2010}),
(\citeproc{ref-Broome2001}{\textbf{Broome2001?}}) and Gustafsson
(\citeproc{ref-Gustafsson2022}{2022}) that Ordering violations lead to
choosing dominated strategies. Those arguments turn on an assumption
about dynamic choice that we have independent reason to reject, and I'll
come back to them in Section~\ref{sec-dual-mandate}. But first I'll make
a small note about how philosophically significant this failure of
Ordering is.

\section{Interlude on Suppositional Decision
Theories}\label{sec-suppositional}

When a student starts decision theory, they are introduced to a view
that is simple, elegant, and wrong. The view starts by assuming that
Chooser, has some actions \emph{A} available, with \emph{a} an arbitrary
action from \emph{A}. There are some possible states \emph{S}, with
\emph{s} an arbitrary such state. Two numerical functions are given: a
probability function Pr over states, and a value function \emph{V} over
pairs of actions and states.

The simple, elegant, and wrong theory is that Chooser should value each
act \emph{a} by its expected value, and choose the one with the highest
value. That is, Chooser should select \emph{a} to maximise
Σ\textsubscript{\emph{s}~∈~\emph{S}}~Pr(\emph{s})\emph{V}(\emph{as}).

If Chooser has any causal influence over the states, this theory gives
bad advice. Assume \emph{A} is \{\emph{a}, \emph{b}\}, \emph{S} is
\{\emph{s},~\emph{t}\}, and \emph{a} will cause \emph{s} to be actual,
while \emph{b} will cause \emph{t}. And assume \emph{V} is described in
Table~\ref{tbl-joycewindow}.

\begin{longtable}[]{@{}ccc@{}}
\caption{A counterexample to the simple
theory.}\label{tbl-joycewindow}\tabularnewline
\toprule\noalign{}
& \emph{s} & \emph{t} \\
\midrule\noalign{}
\endfirsthead
\toprule\noalign{}
& \emph{s} & \emph{t} \\
\midrule\noalign{}
\endhead
\bottomrule\noalign{}
\endlastfoot
\emph{a} & 1 & 1001 \\
\emph{b} & 0 & 1000 \\
\end{longtable}

In Table~\ref{tbl-joycewindow} Chooser should do \emph{b} and bring
about the best state, but whatever Pr says, the simple theory says to do
\emph{a}. So far every decision theorist would agree. But here agreement
ends. There is no agreement on either why the simple theory fails in
this case, or what should go in its place. The most famous disagreement
is about whether it is significant that the states and actions are
causally connected, or merely evidentially connected, but there are many
other disputes beyond that.

Still, in all of the disagreement, there is one common thread. Most
decision theorists endorse a \textbf{suppositional} account of problems
like Table~\ref{tbl-joycewindow}. This terminology comes from James
Joyce(\citeproc{ref-Joyce1999}{1999} Ch. 6), and the point that most
modern theories are suppositional is made by Adam Elga
(\citeproc{ref-Elga2022}{2022}) and Michael Nielsen
(\citeproc{ref-Nielsen2023}{2023}).\footnote{My terminology largely
  follows Elga's.} A suppositional theory starts with a function from
probability functions and actions to probability functions, such that
Pr\textsuperscript{A} is the result of modifying the prior probability
Pr by `supposing' that A. At this stage, the only assumption about this
function is that Pr\textsuperscript{A}(A)~=~1. The theory then says to
choose the A such that the expected value of V with respect to
Pr\textsuperscript{A} us maximised.

The point that Joyce, Elga, and Nielsen make is that many mainstream
approaches to decision theory fit this description, and they differ
solely with respect to what they think the `suppositional' function is.
If Pr\textsuperscript{A}(·)~=~Pr(·~\textbar~A), then the suppositional
theory is evidential decision theory. If
Pr\textsuperscript{A}(·)~=~Pr(A~□→~·), the suppositional theory is the
causal theory defended by Gibbard and Harper
(\citeproc{ref-GibbardHarper1978}{1978}). And so on for several other
theories.

Not all theories are suppositional. The regret-minimising theories I
discussed in Section~\ref{sec-multiple} are not suppositional, and nor
is the risk-sensitive theory I'll discuss in Section~\ref{sec-buchak}.
One consequence of the arguments in Section~\ref{sec-ordering} is that
the SCP rules out any suppositional theory. All suppositional theories
endorse Ordering, and the SCP entails that Ordering is false, so the SCP
rules out suppositional theories.

The non-suppositional theory I prefer is a version of causal
ratificationism. I call it Gamified Decision Theory (GDT). It uses the
following formula for valuing options:

\begin{description}
\tightlist
\item[GDT]
V(a) =
Σ\textsubscript{\emph{s}~∈~\emph{S}}~Pr′(\emph{s})\emph{V}(\emph{as})
\end{description}

In this formula, Pr\textsuperscript{′} is the probability distribution
over states after Chooser has made their decision. GDT says that only
options that have maximal value using this formula are
choice-worthy.\footnote{My preferred version of GDT adds several more
  constraints to this - it has a separate constraint for ruling out
  weakly dominated options, and a constraint for solving beer-quiche
  games, and maybe a constraint for ruling out mixed strategies in
  coordination games. But this is a necessary condition for
  choice-worthiness.} This allows that different options, with different
values, could be choice-worthy. All that matters is that given the
probability distribution over states that Chooser has after making the
choice, the chosen option has maximal expected value. The SCP doesn't
entail GDT, but GDT is a theory to adopt given the SCP.

When I say the SCP rules out all suppositional theories, it's important
to remember the assumptions I flagged back in
Section~\ref{sec-flagging}. I'm assuming that decision problems are
fully specified by describing the payouts for each act-state pair, and
the conditional probability of each state conditional on each act. Some
suppositional theories, including the one Joyce himself defends, think
that the unconditional probability of each state is also relevant to the
rationality of each choice. This argument does not rule those theories
out. To be sure, I don't have a proof that any such suppositional theory
is consistent with the SCP, and the relevant assumption about the
irrelevance of unconditional probabilities is widespread, but the
argument of this section does make essential use of the assumption, and
it is worth noting that here.

\section{Problem 3: Dynamic Choice}\label{sec-dual-mandate}

This section is about dynamic decision problems. I'll start by saying
more carefully what they are, rather than the informal presentation
we've used so far.

\subsection{Decision Trees}\label{sec-decision-tree}

A \textbf{decision tree} is a sextuple ⟨\emph{W}, \emph{R}, \emph{V},
\emph{a}, \emph{I}, Pr⟩ such that:

\begin{itemize}
\tightlist
\item
  \emph{W} is a finite set of nodes. One of these nodes, call it
  \emph{o} for origin, is designated as the initial node. (This is the
  open circle in the diagrams.)
\item
  \emph{R} is a relation on \emph{W} such that for any
  \emph{x}~∈~\emph{W}, ¬\emph{xRo}, and if \emph{y}~≠~\emph{o}, there is
  a unique \emph{x} such that \emph{xRy}. Intuitively, the decision
  problem starts at \emph{o}, and continues by moving from a node
  \emph{x} to another node \emph{y} such that \emph{xRy} until there is
  nowhere further to go. Say that \emph{x} is a predecessor of \emph{y}
  if \emph{xR+y}, where \emph{R+} is the ancestral of \emph{R}.
\item
  \emph{V} is a value function. It maps each terminal node of \emph{W}
  to a real number. A node \emph{x} is a terminal node iff there is no
  \emph{y} such that \emph{xRy}.
\item
  \emph{a} is a function from non-terminal nodes in \emph{W} to the set
  \{C,~D,~N\} that says who the agent is for each node. Intuitively, C
  is for Chooser, D is for Demon, and N is for Nature. That agent
  `chooses' where the game goes next.
\item
  \emph{I} is a partition of the nodes the non-terminal nodes
  \emph{x:~a(x)~=}C. The elements of this partition are called
  information sets. Intuitively, when Chooser reaches a node where they
  must choose, they know that they are in one member of this partition,
  i.e., one information set, and nothing more. Any two nodes in the same
  information set have the same number of outbound links.
\item
  Pr is a conditional probability function. It says that given a
  \emph{strategy} for Chooser, and that a particular non-terminal node
  \emph{x} which is assigned to Demon or Nature has reached, what the
  probability is that we'll move to some further node \emph{y} such that
  \emph{xRy}. If \emph{x} is assigned to Nature, this probability is
  independent of Chooser's strategy.
\end{itemize}

A \textbf{strategy} for one of the three players, Demon, Chooser or
Nature, is a function from all the nodes in the tree which are assigned
to them, to the move they will make if that node is reached. Given any
decision tree, one can generate a \textbf{strategic decision problem}
where the possible actions are strategies for Chooser, and the states
are pairs of strategies for Demon and strategies for Nature. The SCP
says that if there is only one cell in \emph{I}, this strategic problem
has the same solutions as the dynamic problem represented by the tree.
But if \emph{I} has more than one cell, in general the problems will be
distinct.

\subsection{Resolute Choice}\label{sec-resolute}

There are two standard positions in philosophy for how to navigate
decision trees: the resolute view and the sophisticated view.\footnote{These
  are ordinarily used in cases where preferences change over time, but
  here we're applying them in cases where preferences are constant. I'm
  not sure the terms have completely standard denotations across
  philosophy; my use is somewhat stipulative. For more background, see
  (\citeproc{ref-Steele2018}{\textbf{Steele2018?}}).} We'll start with
the resolute view. It says that Chooser should solve the strategic
problem, and then having chosen a strategy, stick to it whatever
happens. For example, Resolute Evidential Decision Theory (REDT) says
Chooser should adopt the strategy with the highest conditional expected
utility, and stay with it whatever happens.\footnote{REDT is similar to
  the Functional Decision Theory of Levinstein and Soares
  (\citeproc{ref-LevinsteinSoares2020}{2020}); the example below is also
  intended as a counterexample to their view.} This can lead to some odd
results. Figure~\ref{fig-redt}/Table~\ref{tbl-redt} is a variant on
Newcomb's Problem.\footnote{It is similar to what's sometimes called an
  open box Newcomb's Problem.}

\subsection{Tree}

\begin{figure}

\centering{

\includegraphics{fourprob_files/figure-pdf/fig-redt-1.png}

}

\caption{\label{fig-redt}An open box Newcomb's Problem.}

\end{figure}%

\subsection{Table}

\begin{longtable}[]{@{}ccc@{}}
\caption{An open box Newcomb's Problem.}\label{tbl-redt}\tabularnewline
\toprule\noalign{}
& P1 & P2 \\
\midrule\noalign{}
\endfirsthead
\toprule\noalign{}
& P1 & P2 \\
\midrule\noalign{}
\endhead
\bottomrule\noalign{}
\endlastfoot
1 & 6 & 0 \\
2 & 11 & 5 \\
\end{longtable}

Chooser sees Demon's prediction before they act. Roughly speaking,
Demon's predictions are 95\% accurate. More precisely, the probability
of each prediction given each strategy for Chooser is given in
Table~\ref{tbl-redt-strategy}. I've also added the expected value
(according to EDT) of each strategy.

\begin{longtable}[]{@{}cccc@{}}
\caption{Demon's prediction probabilities for
Figure~\ref{fig-redt}/Table~\ref{tbl-redt}.}\label{tbl-redt-strategy}\tabularnewline
\toprule\noalign{}
& P1 & P2 & Exp Util \\
\midrule\noalign{}
\endfirsthead
\toprule\noalign{}
& P1 & P2 & Exp Util \\
\midrule\noalign{}
\endhead
\bottomrule\noalign{}
\endlastfoot
1-if-P1, 1-if-P2 & 0.95 & 0.05 & 5.7 \\
1-if-P1, 2-if-P2 & 0.5 & 0.5 & 5.5 \\
2-if-P1, 1-if-P2 & 0.5 & 0.5 & 5.5 \\
2-if-P1, 2-if-P2 & 0.05 & 0.95 & 5.25 \\
\end{longtable}

Since the strategy of always playing 1 has highest return, REDT says to
play it. And it says to do that even when, as is possible, Demon plays
P2, so it is guaranteed to get the worst return. This isn't very
plausible, so REDT doesn't seem attractive. We'll come back to whether
this is because it is resolute or because it is evidential, but there is
an attractive diagnosis. REDT gives the wrong result because it gives
too much weight to earlier plans; it commits the sunk cost fallacy. This
diagnosis supports the sophisticated theory.

\subsection{Sophisticated Choice}\label{sec-sophisticated}

The \textbf{sophisticated} theory (at least as I'm using the term) says
that Chooser should take each node as a new choice, treat their past
choices as fixed, and treat their future choices as another more-or-less
knowable part of the world, and do whatever is best given those
constraints. The \textbf{pure sophisticated} theory adds two further
constraints to that, which I'll call Openness and Separation\footnote{The
  term Openness is mine, but it's a commonly used principle. The term
  Separation is from (\citeproc{ref-Bader2019}{\textbf{Bader2019?}}).
  Rothfus (\citeproc{ref-Rothfusnd}{forthcoming}) uses `Independence'
  for a similar principle}.

\begin{description}
\tightlist
\item[Openness]
Any credal distribution over Chooser's own future actions is rationally
permissible provided that it gives probability 1 to Chooser selecting
some rational action at each point.
\end{description}

Separation The rational choices at an information set are independent of
payouts in parts of the tree which cannot be reached from that
information set.

I'm going to accept the sophisticated theory, but reject both Openness
and Separation. Any theory consistent with the SCP gets into trouble in
\textbf{?@fig-sophisticated-first} if we accept both of these additional
claims.

{[}Include diagram here{]}

In fig-sophisticated-first, Chooser first decides whether to eXit, or
Risk. If they eXit, they get 2. If they Risk, they play
\textbf{?@tbl-sophisticated-first}.

\begin{longtable}[]{@{}ccc@{}}
\caption{The risky subgame in \textbf{?@fig-sophisticated-first}.
\{\#tbl-sophisticated-first\} :::}\tabularnewline
\toprule\noalign{}
& PU & PD \\
\midrule\noalign{}
\endfirsthead
\toprule\noalign{}
& PU & PD \\
\midrule\noalign{}
\endhead
\bottomrule\noalign{}
\endlastfoot
Up & 5 & 0 \\
Down & 0 & 1 \\
\end{longtable}

In \textbf{?@tbl-sophisticated-first}, both options are possible. So by
Openness, having credence 1 that they'll play Up is rational. If the
eXit payout was -1, the only options to survive deletion of dominated
strategies would be Risk-Up and Risk-Down, and in that situation the SCP
implies that both options are rational. So by Separation, Down is
choice-worthy in \textbf{?@fig-sophisticated-first}. But this
combination is absurd, since the strategy Risk-Down is dominated by both
eXit-Up and eXit-Down. As
(\citeproc{ref-Spencer202x}{\textbf{Spencer202x?}}) and Rothfus
(\citeproc{ref-Rothfusnd}{forthcoming}) put it, the result is
dynamically inconsistent.\footnote{Spencer is interested primarily in an
  even stricter principle than domination that gets violated here.
  Chooser plays Risk-Down, and the best possible payoff from it is worse
  than the worst possible payout from either eXit strategy.}

\subsection{Dual Mandate}\label{sec-dual-mandate-defined}

There is a better option than both of these. Following Reinhard
(\citeproc{ref-Selten1974}{\textbf{Selten1974?}}), the standard approach
to dynamic choice in game theory says that both the resolute and
sophisticated views are half-right. They both give necessary conditions
for a series of moves being choice-worthy, but neither gives a
sufficient condition. Decision theorists should adopt the same approach.
In a decision tree, rational Chooser will make a choice that satisfies
both of the following constraints.

\begin{enumerate}
\def\labelenumi{\arabic{enumi}.}
\tightlist
\item
  They will adopt a strategy that is choice-worthy in the strategic form
  of the tree.
\item
  At every information set, they will make a choice that is
  choice-worthy at that moment, holding fixed their strategy elsewhere.
\end{enumerate}

If we assume Ordering, imposing both of these constraints is
implausible, since it is very unlikely that any strategy will meet both
criteria.\footnote{There are moves available here though. We could just
  accept that in those cases Chooser faces a dilemma, and accept there
  are lots of dilemmas around. Or we could adopt a theory like the one
  (\citeproc{ref-Stalnaker199x}{\textbf{Stalnaker199x?}}) adopts, where
  1 and 2 line up. Though note this would require giving up at least one
  of the assumptions from Section~\ref{sec-flagging}.} But without
Ordering, typically the strategies permitted by 1 and 2 will be
overlapping sets.\footnote{This is especially true if we permit mixed
  strategies. But that's a point for another paper.}

Call the view that in dynamic choices rational Chooser will conform to
both 1 and 2 the \emph{dual-mandate} view of dynamic choice. It says
that both the resolute and sophisticated views provide necessary
conditions for

\section{Problem 4: Risk-Sensitivity}\label{sec-buchak}

Think about what value of \emph{x} would make Chooser indifferent
between these two options, and why that would be the right value:

\begin{enumerate}
\def\labelenumi{\arabic{enumi}.}
\tightlist
\item
  \$1,000,000
\item
  A gamble that returns \$2,000,000 with probability \emph{x}, and \$0
  with probability 1-\emph{x}.
\end{enumerate}

What factors are relevant to solving for \emph{x}? One factor is the
declining marginal utility of money. Money primarily has exchange value,
and if Chooser won \$2,000,000, Chooser would exchange the second
million for things they chose not to exchange the first million for, so
the second million has less value. That's one reason that \emph{x} is
well above ½.

But is it the only reason? The orthodox answer is that it is. Lara
Buchak (\citeproc{ref-BuchakRisk}{2013}) has argued that it is not. We
also need to know how much Chooser values, or more likely disvalues,
risk. That is, we need to know how risk-seeking, or risk-averse, Chooser
is.

The orthodox view is that all we need to know are three numbers. In what
follows, let \emph{b} be Chooser's current wealth in millions, and V the
function from wealth (in millions), to utility. Since V is only
determined up to positive affine transformations, we can stipulate two
of these values for V.

\begin{itemize}
\tightlist
\item
  V(\emph{b}), stipulated to be 0.
\item
  V(\emph{b} + 1), stipulated to be 1.
\item
  V(\emph{b} + 1), which we'll label \emph{c}.
\end{itemize}

On the standard view, the gamble's value is \emph{cx}, so Chooser is
indifferent between it and the money iff \emph{x}~=~1/\emph{c}. On
Buchak's view, rational Chooser has a risk function \emph{f}, that
measures their sensitivity to risk. The function must be monotonic
increasing, with \emph{f}(0)~=~0, and \emph{f}(1)~=~1. If Chooser is
risk-averse, then typically \emph{f}(\emph{x})~\textless~\emph{x}.
Buchak's view reduces to the orthodox view if
\emph{f}(\emph{x})~=~\emph{x}. I'm going to argue that given the SCP,
\emph{f}(\emph{x}) does equal \emph{x}. I'm far from the first to argue
for \emph{f}(\emph{x})~=~\emph{x}.\footnote{See Briggs
  (\citeproc{ref-Briggs2015}{2015}) and Thoma
  (\citeproc{ref-Thoma2019}{2019}) for different arguments to the same
  conclusion.} The value of the argument here is that it uses the same
principle, the SCP, that is relevant to so many other problems.

The core of Buchak's theory is a non-standard way of valuing a gamble.
For simplicity, we'll focus on gambles with finitely many outcomes.
Associate a gamble with a random variable \emph{O}, which takes values
\emph{o}\textsubscript{1}, \ldots, \emph{o\textsubscript{n}}, where
\emph{o\textsubscript{j}}~\textgreater~\emph{o\textsubscript{i}} iff
\emph{j}~\textgreater~\emph{i}. Buchak says that the risk-weighted
expected utility of \emph{O} is given by this formula, where \emph{f} is
the agent's risk-weighting function.

\[
REU(O) = o_1 + \sum_{i = 2}^n f(\Pr(O \geq o_i))(o_i - o_{i-1})
\]

The decision rule then is simple: choose the gamble with the highest
REU.

The key notion here is the risk function \emph{f}, which we introduced
earlier. I'm going to show that if
\emph{f}(\emph{x})~=~\emph{x}\textsuperscript{2}, then we get a
violation of the SCP. This is If \emph{f} is the identity function, then
this definition becomes a slightly non-standard way of defining expected
utility. Buchak allows it to be much more general. The key constraints
are that \emph{f} is monotonically increasing, that \emph{f}(0)~=~0 and
\emph{f}(1)~=~1. In general, if \emph{f}(\emph{x})~\textless~\emph{x},
Chooser is more risk-averse than an expected utility maximiser, while if
\emph{f}(\emph{x})~\textgreater~\emph{x}, Chooser is more risk-seeking.
The former case is more relevant to everyday intuitions, and it's what
I'll focus on. Indeed, I'll focus on the case where
\emph{r}(\emph{x})~=~\emph{x}\textsuperscript{2}, which is also a case
Buchak uses a lot.

There are a number of good reasons to like Buchak's theory, but it is
inconsistent with the Single Choice Principle. I'll show this for the
case \emph{r}(\emph{x})~=~\emph{x}\textsuperscript{2}, but it's not much
harder to produce similar examples for any value of \emph{r} other than
\emph{r}(\emph{x})~=~\emph{x}. In Figure~\ref{fig-buchak} at stage 1 a
fair die will be rolled. If it lands 1 or 2, Nature moves Left; if it
lands 3 or 4, Nature moves Middle; otherwise, Nature moves Right. If
Nature moves Left, the game ends, and Chooser gets 1. Otherwise Chooser
is told that Nature did not move Left, but not whether they moved Middle
or Right. If Chooser selects Down, they get 1. If Chooser selects Up,
they get 5 if Nature moved Middle, and 0 otherwise.

\begin{figure}

\centering{

\includegraphics{fourprob_files/figure-pdf/fig-buchak-1.png}

}

\caption{\label{fig-buchak}Tree Diagram of the counterexample to REU.}

\end{figure}%

Table~\ref{tbl-buchak-early} shows the strategic table of
Figure~\ref{fig-buchak}, and Table~\ref{tbl-buchak-late} shows the
decision table Chooser faces at the time they have to choose.

\begin{table}

\caption{\label{tbl-panel}Two strategy tables for
Figure~\ref{fig-buchak}.}

\begin{minipage}{0.50\linewidth}

\subcaption{\label{tbl-buchak-early}The strategy table at game start.}

\centering{

\begin{tabular}{cccc}
\toprule
 & \textbf{Left} & \textbf{Middle} & \textbf{Right}\\
\midrule
\textbf{Up} & 1 & 5 & 0\\
\textbf{Down} & 1 & 1 & 1\\
\bottomrule
\end{tabular}

}

\end{minipage}%
%
\begin{minipage}{0.50\linewidth}

\subcaption{\label{tbl-buchak-late}The strategy table at choice time.}

\centering{

\begin{tabular}{ccc}
\toprule
 & \textbf{Middle} & \textbf{Right}\\
\midrule
\textbf{Up} & 5 & 0\\
\textbf{Down} & 1 & 1\\
\bottomrule
\end{tabular}

}

\end{minipage}%

\end{table}%

In Table~\ref{tbl-buchak-early}, the REU of Down is 1 (since that's the
only possible outcome), and the REU of Up is 8/9. There is a 2/3 chance
of getting at least 1, so that's worth 4/9, and there's a 1/3 chance of
getting another 4, so that's also worth 4/9, and adding those gives 8/9.
So the optimal strategy, according to REU theory, is Down. That is, REU
says to prefer the strategy \emph{Choose Down if you have to choose} to
the strategy \emph{Choose Up if you have to choose}. But if we get to
the choice point, we're at Table~\ref{tbl-buchak-late}. And in that
table the REU of Up is 5 times 1/4, i.e., 5/4. So at that point, REU
says to choose Up. What REU says to do if you have to choose is
different to which strategy it chooses for the one and only point you
have to choose at. That is, Buchak's theory violates the SCP, and so
should be rejected.

This is far from the first objection to Buchak's view, but something
interesting follows from the connection, via the SCP, to demonic
problems. Many decision theorists reject Buchak's view about non-demonic
problems in favour of orthodox expected utility theory, but also endorse
views inconsistent with the SCP.\footnote{Evidential Decision Theorists
  reject both Buchak's view and the SCP, and in this respect they are
  far from alone.} Those theorists can't coherently use this argument
against Buchak's view. Nor can they use any other argument whose
premises entail the SCP, such as an argument from the Sure Thing
Principle. It would take too long to argue for this here, but I suspect
there is no way to do this.\footnote{Note that if we don't want to argue
  from first principles, but instead try to capture intuitions about
  cases, we should prefer Buchak's theory because of how it handles the
  Allais (\citeproc{ref-Allais1953}{1953}) paradox.} If we reject the
SCP, the weight of reasons favours Buchak's view over orthodoxy. I'll
leave this as a challenge for theorists unconvinced by
Section~\ref{sec-scp-defence}: what is the best argument against
Buchak's view, and in favour of expectationist orthodoxy, whose premises
do not entail the SCP? The separation in the literature between demonic
and non-demonic problems has obscured how hard a challenge this is.

\section{Conclusion}\label{sec-conclusion}

Not yet written

\subsection*{References}\label{references}
\addcontentsline{toc}{subsection}{References}

\phantomsection\label{refs}
\begin{CSLReferences}{1}{0}
\bibitem[\citeproctext]{ref-Allais1953}
Allais, M. 1953. {``Le Comportement de l'homme Rationnel Devant Le
Risque: Critique Des Postulats Et Axiomes de l'ecole Americaine.''}
\emph{Econometrica} 21 (4): 503--46. doi:
\href{https://doi.org/10.2307/1907921}{10.2307/1907921}.

\bibitem[\citeproctext]{ref-Arntzenius2008}
Arntzenius, Frank. 2008. {``No Regrets; or, Edith Piaf Revamps Decision
Theory.''} \emph{Erkenntnis} 68 (2): 277--97. doi:
\href{https://doi.org/10.1007/s10670-007-9084-8}{10.1007/s10670-007-9084-8}.

\bibitem[\citeproctext]{ref-Barnett2022}
Barnett, David James. 2022. {``Graded Ratifiability.''} \emph{Journal of
Philosophy} 119 (2): 57--88. doi:
\href{https://doi.org/10.5840/jphil202211925}{10.5840/jphil202211925}.

\bibitem[\citeproctext]{ref-Bernheim1984}
Bernheim, B. Douglas. 1984. {``Rationalizable Strategic Behavior.''}
\emph{Econometrica} 52 (4): 1007--28. doi:
\href{https://doi.org/10.2307/1911196}{10.2307/1911196}.

\bibitem[\citeproctext]{ref-BradleySteele2016}
Bradley, Seamus, and Katie Steele. 2016. {``Can Free Evidence Be Bad?
Value of Informationfor the Imprecise Probabilist.''} \emph{Philosophy
of Science} 83 (1): 1--28. doi:
\href{https://doi.org/10.1086/684184}{10.1086/684184}.

\bibitem[\citeproctext]{ref-Briggs2015}
Briggs, Ray. 2015. {``Costs of Abandoning the Sure-Thing Principle.''}
\emph{Canadian Journal of Philosophy} 45 (5): 827--40. doi:
\href{https://doi.org/10.1080/00455091.2015.1122387}{10.1080/00455091.2015.1122387}.

\bibitem[\citeproctext]{ref-BuchakRisk}
Buchak, Lara. 2013. \emph{Risk and Rationality}. Oxford: Oxford
University Press.

\bibitem[\citeproctext]{ref-Chang2002}
Chang, Ruth. 2002. {``The Possibility of Parity.''} \emph{Ethics} 112
(4): 659--88. doi:
\href{https://doi.org/10.1086/339673}{10.1086/339673}.

\bibitem[\citeproctext]{ref-Chang2015}
---------. 2015. {``Value Incomparability and Incommensurability.''} In
\emph{The Oxford Handbook of Value Theory}, edited by Iwao Hirose and
Jonas Olson, 205--24. Oxford: Oxford University Press. doi:
\href{https://doi.org/10.1093/oxfordhb/9780199959303.013.0012}{10.1093/oxfordhb/9780199959303.013.0012}.

\bibitem[\citeproctext]{ref-DorrEtAl2023}
Dorr, Cian, Jacob M. Nebel, and Jake Zuehl. 2023. {``The Case for
Comparability.''} \emph{Noûs} 57 (2): 414--53. doi:
\href{https://doi.org/10.1111/nous.12407}{10.1111/nous.12407}.

\bibitem[\citeproctext]{ref-Elga2010}
Elga, Adam. 2010. {``Subjective Probabilities Should Be Sharp.''}
\emph{Philosophers' Imprint} 10: 1--11.

\bibitem[\citeproctext]{ref-Elga2022}
---------. 2022. {``Confession of a Causal Decision Theorist.''}
\emph{Analysis} 82 (2): 203--13. doi:
\href{https://doi.org/10.1093/analys/anab040}{10.1093/analys/anab040}.

\bibitem[\citeproctext]{ref-Fuscond}
Fusco, Melissa. n.d. {``Absolution of a Causal Decision Theorist.''}
\emph{No{û}s}. doi:
\href{https://doi.org/10.1111/nous.12459}{10.1111/nous.12459}. Early
view.

\bibitem[\citeproctext]{ref-Gallow2020}
Gallow, J. Dmitri. 2020. {``The Causal Decision Theorist's Guide to
Managing the News.''} \emph{The Journal of Philosophy} 117 (3): 117--49.
doi:
\href{https://doi.org/10.5840/jphil202011739}{10.5840/jphil202011739}.

\bibitem[\citeproctext]{ref-Gallownd}
---------. n.d. {``The Sure Thing Principle Leads to Instability.''}
Philosophical Quarterly.
\url{https://philpapers.org/archive/GALTST-2.pdf}.

\bibitem[\citeproctext]{ref-GibbardHarper1978}
Gibbard, Allan, and William Harper. 1978. {``Counterfactuals and Two
Kinds of Expected Utility.''} In \emph{Foundations and Applications of
Decision Theory}, edited by C. A. Hooker, J. J. Leach, and E. F.
McClennen, 125--62. Dordrecht: Reidel.

\bibitem[\citeproctext]{ref-Gustafsson2011}
Gustafsson, Johan E. 2011. {``A Note in Defence of Ratificationism.''}
\emph{Erkenntnis} 75 (1): 147--50. doi:
\href{https://doi.org/10.1007/s10670-010-9267-6}{10.1007/s10670-010-9267-6}.

\bibitem[\citeproctext]{ref-Gustafsson2022}
---------. 2022. \emph{Money-Pump Arguments}. Cambridge: Cambridge
University Press.

\bibitem[\citeproctext]{ref-Hands2014}
Hands, D. Wade. 2014. {``{Paul Samuelson and Revealed Preference
Theory}.''} \emph{History of Political Economy} 46 (1): 85--116. doi:
\href{https://doi.org/10.1215/00182702-2398939}{10.1215/00182702-2398939}.

\bibitem[\citeproctext]{ref-Harper1986}
Harper, William. 1986. {``Mixed Strategies and Ratifiability in Causal
Decision Theory.''} \emph{Erkenntnis} 24 (1): 25--36. doi:
\href{https://doi.org/10.1007/BF00183199}{10.1007/BF00183199}.

\bibitem[\citeproctext]{ref-Humberstone2016}
Humberstone, Lloyd. 2016. \emph{Philsophical Applications of Modal
Logic}. Milton Keynes: College Publications.

\bibitem[\citeproctext]{ref-Jeffrey1983b}
Jeffrey, Richard C. 1983. \emph{The Logic of Decision}. 2nd ed. Chicago:
University of Chicago Press.

\bibitem[\citeproctext]{ref-Joyce1999}
Joyce, James M. 1999. \emph{The Foundations of Causal Decision Theory}.
Cambridge: Cambridge University Press.

\bibitem[\citeproctext]{ref-Joyce2010}
---------. 2010. {``A Defence of Imprecise Credences in Inference and
Decision Making.''} \emph{Philosophical Perspectives} 24 (1): 281--323.
doi:
\href{https://doi.org/10.1111/j.1520-8583.2010.00194.x}{10.1111/j.1520-8583.2010.00194.x}.

\bibitem[\citeproctext]{ref-Keynes1921}
Keynes, John Maynard. 1921. \emph{Treatise on Probability}. London:
Macmillan.

\bibitem[\citeproctext]{ref-Ledermannd}
Lederman, Harvey. 2024. {``Of Marbles and Matchsticks.''} \emph{Oxford
Studies in Epistemology}.

\bibitem[\citeproctext]{ref-LevinsteinSoares2020}
Levinstein, Benjamin Anders, and Nate Soares. 2020. {``Cheating Death in
Damascus.''} \emph{Journal of Philosophy} 117 (5): 237--66. doi:
\href{https://doi.org/10.5840/jphil2020117516}{10.5840/jphil2020117516}.

\bibitem[\citeproctext]{ref-Lewis1979en}
Lewis, David. 1979. {``Prisoners' Dilemma Is a {N}ewcomb Problem.''}
\emph{Philosophy and Public Affairs} 8 (3): 235--40.

\bibitem[\citeproctext]{ref-Nielsen2023}
Nielsen, Michael. 2023. {``Only {CDT} Values Knowledge.''}
\emph{Analysis} 84 (1): 67--82. doi:
\href{https://doi.org/10.1093/analys/anad041}{10.1093/analys/anad041}.

\bibitem[\citeproctext]{ref-Nozick1969}
Nozick, Robert. 1969. {``Newcomb's Problem and Two Principles of
Choice.''} In \emph{Essays in Honor of Carl {G}. Hempel: A Tribute on
the Occasion of His Sixty-Fifth Birthday. Hempel: A Tribute on the
Occasion of His Sixty-Fifth Birthday}, edited by Nicholas Rescher,
114--46. Riedel: Springer.

\bibitem[\citeproctext]{ref-Pearce1984}
Pearce, David G. 1984. {``Rationalizable Strategic Behavior and the
Problem of Perfection.''} \emph{Econometrica} 52 (4): 1029--50. doi:
\href{https://doi.org/10.2307/1911197}{10.2307/1911197}.

\bibitem[\citeproctext]{ref-Podgorski2022}
Podgorski, Aberlard. 2022. {``Tournament Decision Theory.''}
\emph{No{û}s} 56 (1): 176--203. doi:
\href{https://doi.org/10.1111/nous.12353}{10.1111/nous.12353}.

\bibitem[\citeproctext]{ref-RamseyGeneralProp}
Ramsey, Frank. (1929) 1990. {``General Propositions and Causality.''} In
\emph{Philosophical Papers}, edited by D. H. Mellor, 145--63. Cambridge:
Cambridge University Press.

\bibitem[\citeproctext]{ref-Rothfusnd}
Rothfus, Gerard J. forthcoming. {``Evidence, Causality, and Sequential
Choice.''} \emph{Theory and Decision}, forthcoming. doi:
\href{https://doi.org/10.1007/s11238-024-09990-y}{10.1007/s11238-024-09990-y}.

\bibitem[\citeproctext]{ref-Samuelson1938}
Samuelson, Paul A. 1938. {``A Note on the Pure Theory of Consumer's
Behaviour.''} \emph{Econometrica} 5 (17): 61--71. doi:
\href{https://doi.org/10.2307/2548836}{10.2307/2548836}.

\bibitem[\citeproctext]{ref-Sen1971}
Sen, Amartya. 1971. {``Choice Functions and Revealed Preference.''}
\emph{Review of Economic Studies} 38 (3): 307--17. doi:
\href{https://doi.org/10.2307/2296384}{10.2307/2296384}.

\bibitem[\citeproctext]{ref-Spencer2021b}
Spencer, Jack. 2021. {``Rational Monism and Rational Pluralism.''}
\emph{Philosophical Studies} 178: 1769--1800. doi:
\href{https://doi.org/10.1007/s11098-020-01509-9}{10.1007/s11098-020-01509-9}.

\bibitem[\citeproctext]{ref-Thoma2019}
Thoma, Johanna. 2019. {``Risk Aversion and the Long Run.''}
\emph{Ethics} 129 (2): 230--53. doi:
\href{https://doi.org/10.1086/699256}{10.1086/699256}.

\bibitem[\citeproctext]{ref-Walley1991}
Walley, Peter. 1991. \emph{Statisical Reasoning with Imprecise
Probabilities}. London: Chapman \& Hall.

\bibitem[\citeproctext]{ref-Wedgwood2013a}
Wedgwood, Ralph. 2013. {``Gandalf's Solution to the Newcomb Problem.''}
\emph{Synthese} 190 (14): 2643--75. doi:
\href{https://doi.org/10.1007/s11229-011-9900-1}{10.1007/s11229-011-9900-1}.

\end{CSLReferences}



\noindent Unpublished. Posted online in 2024.

\end{document}
