% Options for packages loaded elsewhere
% Options for packages loaded elsewhere
\PassOptionsToPackage{unicode}{hyperref}
\PassOptionsToPackage{hyphens}{url}
%
\documentclass[
  11pt,
  letterpaper,
  DIV=11,
  numbers=noendperiod,
  twoside]{scrartcl}
\usepackage{xcolor}
\usepackage[left=1.1in, right=1in, top=0.8in, bottom=0.8in,
paperheight=9.5in, paperwidth=7in, includemp=TRUE, marginparwidth=0in,
marginparsep=0in]{geometry}
\usepackage{amsmath,amssymb}
\setcounter{secnumdepth}{3}
\usepackage{iftex}
\ifPDFTeX
  \usepackage[T1]{fontenc}
  \usepackage[utf8]{inputenc}
  \usepackage{textcomp} % provide euro and other symbols
\else % if luatex or xetex
  \usepackage{unicode-math} % this also loads fontspec
  \defaultfontfeatures{Scale=MatchLowercase}
  \defaultfontfeatures[\rmfamily]{Ligatures=TeX,Scale=1}
\fi
\usepackage{lmodern}
\ifPDFTeX\else
  % xetex/luatex font selection
  \setmainfont[ItalicFont=EB Garamond Italic,BoldFont=EB Garamond
Bold]{EB Garamond Math}
  \setsansfont[]{EB Garamond}
  \setmathfont[]{Garamond-Math}
\fi
% Use upquote if available, for straight quotes in verbatim environments
\IfFileExists{upquote.sty}{\usepackage{upquote}}{}
\IfFileExists{microtype.sty}{% use microtype if available
  \usepackage[]{microtype}
  \UseMicrotypeSet[protrusion]{basicmath} % disable protrusion for tt fonts
}{}
\usepackage{setspace}
% Make \paragraph and \subparagraph free-standing
\makeatletter
\ifx\paragraph\undefined\else
  \let\oldparagraph\paragraph
  \renewcommand{\paragraph}{
    \@ifstar
      \xxxParagraphStar
      \xxxParagraphNoStar
  }
  \newcommand{\xxxParagraphStar}[1]{\oldparagraph*{#1}\mbox{}}
  \newcommand{\xxxParagraphNoStar}[1]{\oldparagraph{#1}\mbox{}}
\fi
\ifx\subparagraph\undefined\else
  \let\oldsubparagraph\subparagraph
  \renewcommand{\subparagraph}{
    \@ifstar
      \xxxSubParagraphStar
      \xxxSubParagraphNoStar
  }
  \newcommand{\xxxSubParagraphStar}[1]{\oldsubparagraph*{#1}\mbox{}}
  \newcommand{\xxxSubParagraphNoStar}[1]{\oldsubparagraph{#1}\mbox{}}
\fi
\makeatother


\usepackage{longtable,booktabs,array}
\usepackage{calc} % for calculating minipage widths
% Correct order of tables after \paragraph or \subparagraph
\usepackage{etoolbox}
\makeatletter
\patchcmd\longtable{\par}{\if@noskipsec\mbox{}\fi\par}{}{}
\makeatother
% Allow footnotes in longtable head/foot
\IfFileExists{footnotehyper.sty}{\usepackage{footnotehyper}}{\usepackage{footnote}}
\makesavenoteenv{longtable}
\usepackage{graphicx}
\makeatletter
\newsavebox\pandoc@box
\newcommand*\pandocbounded[1]{% scales image to fit in text height/width
  \sbox\pandoc@box{#1}%
  \Gscale@div\@tempa{\textheight}{\dimexpr\ht\pandoc@box+\dp\pandoc@box\relax}%
  \Gscale@div\@tempb{\linewidth}{\wd\pandoc@box}%
  \ifdim\@tempb\p@<\@tempa\p@\let\@tempa\@tempb\fi% select the smaller of both
  \ifdim\@tempa\p@<\p@\scalebox{\@tempa}{\usebox\pandoc@box}%
  \else\usebox{\pandoc@box}%
  \fi%
}
% Set default figure placement to htbp
\def\fps@figure{htbp}
\makeatother


% definitions for citeproc citations
\NewDocumentCommand\citeproctext{}{}
\NewDocumentCommand\citeproc{mm}{%
  \begingroup\def\citeproctext{#2}\cite{#1}\endgroup}
\makeatletter
 % allow citations to break across lines
 \let\@cite@ofmt\@firstofone
 % avoid brackets around text for \cite:
 \def\@biblabel#1{}
 \def\@cite#1#2{{#1\if@tempswa , #2\fi}}
\makeatother
\newlength{\cslhangindent}
\setlength{\cslhangindent}{1.5em}
\newlength{\csllabelwidth}
\setlength{\csllabelwidth}{3em}
\newenvironment{CSLReferences}[2] % #1 hanging-indent, #2 entry-spacing
 {\begin{list}{}{%
  \setlength{\itemindent}{0pt}
  \setlength{\leftmargin}{0pt}
  \setlength{\parsep}{0pt}
  % turn on hanging indent if param 1 is 1
  \ifodd #1
   \setlength{\leftmargin}{\cslhangindent}
   \setlength{\itemindent}{-1\cslhangindent}
  \fi
  % set entry spacing
  \setlength{\itemsep}{#2\baselineskip}}}
 {\end{list}}
\usepackage{calc}
\newcommand{\CSLBlock}[1]{\hfill\break\parbox[t]{\linewidth}{\strut\ignorespaces#1\strut}}
\newcommand{\CSLLeftMargin}[1]{\parbox[t]{\csllabelwidth}{\strut#1\strut}}
\newcommand{\CSLRightInline}[1]{\parbox[t]{\linewidth - \csllabelwidth}{\strut#1\strut}}
\newcommand{\CSLIndent}[1]{\hspace{\cslhangindent}#1}



\setlength{\emergencystretch}{3em} % prevent overfull lines

\providecommand{\tightlist}{%
  \setlength{\itemsep}{0pt}\setlength{\parskip}{0pt}}



 


\setlength\heavyrulewidth{0ex}
\setlength\lightrulewidth{0ex}
\usepackage[automark]{scrlayer-scrpage}
\clearpairofpagestyles
\cehead{
  Brian Weatherson
  }
\cohead{
  Against Inconclusive Updating Rules
  }
\ohead{\bfseries \pagemark}
\cfoot{}
\makeatletter
\newcommand*\NoIndentAfterEnv[1]{%
  \AfterEndEnvironment{#1}{\par\@afterindentfalse\@afterheading}}
\makeatother
\NoIndentAfterEnv{itemize}
\NoIndentAfterEnv{enumerate}
\NoIndentAfterEnv{description}
\NoIndentAfterEnv{quote}
\NoIndentAfterEnv{equation}
\NoIndentAfterEnv{longtable}
\NoIndentAfterEnv{abstract}
\renewenvironment{abstract}
 {\vspace{-1.25cm}
 \quotation\small\noindent\emph{Abstract}:}
 {\endquotation}
\newfontfamily\tfont{EB Garamond}
\addtokomafont{disposition}{\rmfamily}
\addtokomafont{title}{\normalfont\itshape}
\let\footnoterule\relax
\KOMAoption{captions}{tableheading}
\makeatletter
\@ifpackageloaded{caption}{}{\usepackage{caption}}
\AtBeginDocument{%
\ifdefined\contentsname
  \renewcommand*\contentsname{Table of contents}
\else
  \newcommand\contentsname{Table of contents}
\fi
\ifdefined\listfigurename
  \renewcommand*\listfigurename{List of Figures}
\else
  \newcommand\listfigurename{List of Figures}
\fi
\ifdefined\listtablename
  \renewcommand*\listtablename{List of Tables}
\else
  \newcommand\listtablename{List of Tables}
\fi
\ifdefined\figurename
  \renewcommand*\figurename{Figure}
\else
  \newcommand\figurename{Figure}
\fi
\ifdefined\tablename
  \renewcommand*\tablename{Table}
\else
  \newcommand\tablename{Table}
\fi
}
\@ifpackageloaded{float}{}{\usepackage{float}}
\floatstyle{ruled}
\@ifundefined{c@chapter}{\newfloat{codelisting}{h}{lop}}{\newfloat{codelisting}{h}{lop}[chapter]}
\floatname{codelisting}{Listing}
\newcommand*\listoflistings{\listof{codelisting}{List of Listings}}
\makeatother
\makeatletter
\makeatother
\makeatletter
\@ifpackageloaded{caption}{}{\usepackage{caption}}
\@ifpackageloaded{subcaption}{}{\usepackage{subcaption}}
\makeatother
\usepackage{bookmark}
\IfFileExists{xurl.sty}{\usepackage{xurl}}{} % add URL line breaks if available
\urlstyle{same}
\hypersetup{
  pdftitle={Against Inconclusive Updating Rules},
  pdfauthor={Brian Weatherson},
  hidelinks,
  pdfcreator={LaTeX via pandoc}}


\title{Against Inconclusive Updating Rules}
\author{Brian Weatherson}
\date{2024}
\begin{document}
\maketitle
\begin{abstract}
Many formal epistemologists think that there should be a rule for how
rational agents update their credences when some part of their credal
state is adjusted in a way that does not involve learning, with
certainty, some proposition. The most popular such rule is Jeffrey
conditionalization. This note argues that no such rule is correct. The
only rational rule of update is that one should conditionalize on one's
evidence.
\end{abstract}


\setstretch{1.1}
\section{Introduction}\label{sec-intro}

The two central commitments of Bayesian epistemology are that credences
should obey the probability calculus, and that they should be updated by
conditionalization on new evidence. But many Bayesian epistemologists
hold that there are other rules too. After all, they say,
conditionalization has its limits. To update on \emph{E} means to move
one's credence in \emph{E} to 1, and hence to be willing to bet on
\emph{E} at any odds. Since there are few if any propositions that we
would bet on at any odds, there isn't much evidence in this strong
sense. So something else is needed. In particular, they say, we need
extra rules that say what to do when some other change is externally
imposed on a credal state. The simplest such imposition is that some
proposition \emph{p} has its credence set to be \emph{x}. In that case
the most popular rule is Jeffrey conditionalisation\footnote{For more
  details on Jeffrey conditionalization, see Lin
  (\citeproc{ref-Lin2022}{\textbf{Lin2022?}} sec 5.3)}:

\[
\Pr_{\text{New}}(q) = x\Pr(q | p) + (1-x)\Pr(q | \neg p)
\]

There are a couple of initial inconclusive reasons to be sceptical of
the possibility of any such rule.

One reason, loosely following Williamson
(\citeproc{ref-Williamson2000}{2000}), says that all updating involves
learning, learning involves acquiring evidence, evidence is
propositional, and an externally imposed move of one's credences is not
the acquisition of a proposition.\footnote{Though note that someone else
  has a particular credence is a proposition. This will become important
  in Section~\ref{sec-defer}.} Now as an argument against the importance
of Jeffrey conditionalisation, this is question-begging at every step.
Still, question-begging arguments can be valuable.\footnote{As Lewis
  (\citeproc{ref-Lewis1982c}{1982}) points out, the most powerful
  argument against dialethism is blatantly question-begging.} They can
help those of us who are antecedently disposed to reject a view see
where our rejection is coming from. But it would be nice to turn that
into an argument.

A second reason, loosely following
(\citeproc{ref-vanFraassen1981}{\textbf{vanFraassen1981?}}), says that
this rule is too much of a special case. Unconditional credences are
special cases of two things. They are expectations of a very special
kind of random variable: one whose value is 1 when a particular
proposition is true, and 0 otherwise. And they are conditional credences
conditional on a trivial proposition. We might hope that any rule around
here would fall out as a special case of a plausible rule that is
general enough to cover these more general cases. The absence of such a
rule from the literature might give one pause here. But again, this is
hardly a conclusive argument. Maybe someone will develop such a rule
tomorrow.

There are a couple of natural ways that one might try to build such a
rule: the way of deference, and the way of accuracy. Unfortunately, they
both don't work. And they don't work in ways that undermine the
normative force of any rule that is restricted just to changes in the
credence of a single, unconditional, proposition. Let's take these in
turn.

\section{Deference}\label{sec-defer}

Start with the following natural thought. What one should do when one
changes one's credence in \emph{p} to \emph{x} is what one would do if
one learned that an expert to whom one properly defers has credence
\emph{x} in \emph{p}. This approach has two attractive features. First,
it is easy to see its justification. Even a sceptic who only believed
that updating goes by learning could agree that sometimes one should act
as if one had learned something about an expert. Second, it promises to
generalise. One could also ask about what credences one would have if
one learned the expert's credence in \emph{p} given \emph{q} was
\emph{x}, or one learned the expert's expected value for a particular
random variable was \emph{x}.

The rule requires that one be able to identify experts to whom one
properly defers. But here again there is a simple model. Let Thinker be
the person doing the deferring. Let Experimenter be someone who was a
cognitive duplicate of Thinker just before conducting an experiment, and
now they have conducted the experiment and learned the
result.\footnote{The underlying picture I'm using here is taken from
  Blackwell (\citeproc{ref-Blackwell1951}{1951}).} Assume the evidence
Experimenter gets is luminous; they know what they learned, and they
know what they didn't learn. So we don't have any concerns about
learning having negative value, as happens when evidence is not luminous
((\citeproc{ref-Dasnd}{\textbf{Dasnd?}})). Then, given the venerable
rule that one should use as much evidence as possible
((\citeproc{ref-Hosiasson1931}{\textbf{Hosiasson1931?}});
(\citeproc{ref-Torsellnd}{\textbf{Torsellnd?}})), Thinker should defer
to Experimenter.

All this seems fine, but the problem is that we haven't uniquely
specified an Experimenter here. Different background assumptions about
what the Experimenter is like might lead to different posterior
credences if following this rule. So any general rule, like Jeffrey
conditionalization, which is silent about the nature of the
experimenter, cannot be justified this way. Let's see this with an
example.

Let X, Y and Z be normal distributions with mean 0 and variance 1. In
symbols, each of them is \(\mathcal{N}\)(0,1). So the sum of any two of
them has distribution \(\mathcal{N}\)(0,2), and the sum of all three has
distribution \(\mathcal{N}\)(0,3). Let \emph{p} be the proposition that
this sum, X~+~Y~+~Z, is positive, and \emph{q} be the proposition that
the sum is greater than 1. Let C be a probability function that
incorporates all these facts, but has no other direct information about
X, Y, and Z. So C(\emph{p})~=~½, since in all respects C's opinions are
symmetric around 0.

C knows some things about A and B. Both of them know everything C knows
about X, Y, Z, and each are logically and mathematically omniscient, and
know precisely what evidence they have. Also, A just conducted an
experiment that revealed the value of X, and nothing else, and B
conducted an experiment that revealed the value of X + Y, and nothing
else. Both A and B are experts for C, in the above sense. What we'll do
now is work out what C's posterior credence in \emph{q} should be after
first learning that A(\emph{p})~=~0.6, and second learning that
B(\emph{p})~=~0.6.

We can work out the value of X, which we'll call \emph{x}, from
A(\emph{p})~=~0.6. In what follows, \(\Phi\)(\emph{x}) is the cumulative
distribution for the standard normal distribution, i.e., for
\(\mathcal{N}\)(0,1), and \(\Phi\)\textsuperscript{-1} is its inverse.
If X~=~\emph{x}, then \emph{p} is true iff Y~+~Z~\textgreater{}
-\emph{x}. Since Y~+~Z is a normal distribution with mean 0 and variance
2, i.e., standard deviation \(\sqrt{2}\), the probability of this is
\(\Phi\)(\(\frac{x}{\sqrt{2}}\)). So
\emph{x}~=~\(\sqrt{2}\Phi\)\textsuperscript{-1}(0.6), which is about
0.358. So A(\emph{q}) is the probability that Y~+~Z is greater than
1~-~\emph{x}, i.e., about 0.642. Since the distribution of Y~+~Z is
\(\mathcal{N}\)(0,2), we can work out this probability: it is very close
to 0.325. So if C learns that A's credence in \emph{p} is 0.6, C's new
credence in \emph{q} should be about 0.325.

We can also do the same calculation for B(\emph{p})~=~0.6. Since the
only thing B doesn't know is Z, and Z has distribution
\(\mathcal{N}\)(0,1), it must be that X~+~Y has value
\(\Phi\)\textsuperscript{-1}(0.6), which is about 0.253. So for B, the
probability of \emph{q} is the probability that Z is greater than
(about) 0.747. That is, B(\emph{q}) is about 0.228. So if C learns that
B's credence in \emph{p} is 0.6, C's new credence in \emph{q} should be
about 0.228.

Now we have a problem. There is no such thing as \emph{the} credence in
\emph{q} that an expert to whom C should defer would have, if their
credence in \emph{p} was 0.6. If that expert knows X, the posterior
credence in \emph{q} is around 0.325; if the expert knows X~+~Y, the
posterior credence in \emph{q} is around 0.228. For what it's worth,
neither of these is the number you'd get by Jeffrey conditionalising C's
prior credence with the condition that \emph{p} goes to 0.6; that would
take \emph{q} to around 0.338.

So the way of deference doesn't work. The idea seems right; see what an
expert who does this would do. But there are too many experts who do
\emph{this}, and they differ in other respects. Let's move on to a
different strategy.

\section{Accuracy}\label{sec-accuracy}

A popular move in recent years, tracing back to Joyce
(\citeproc{ref-Joyce1998}{1998}), has been to justify probabilistic
principles by appeal to accuracy. As
(\citeproc{ref-GreavesWallace200x}{\textbf{GreavesWallace200x?}}) show,
we can justify ordinary conditionalisation this way. And there is a
natural way to try to justify inconclusive updates too. Assume Thinker
starts with some credence function, and consider the class of
probabilistically coherent credence functions that satisfy some
constraint. Ask which of them has highest expected accuracy given
Thinker's prior credences. That should be the one that Thinker adopts
after updating on the new constraint.

Here's a simple example of how this might work. Say that there are three
exclusive and exhaustive possibilities: \emph{p}, \emph{q}, and
\emph{r}. Initially Thinker has credence 1/6 in \emph{p}, 1/6 in
\emph{q}, and 2/3 in \emph{r}, and the new constraint is that \emph{p}
moves to 0.3. Assume for now that we're using the Brier score to measure
inaccuracy. If the credence in \emph{p} is fixed, there is only one
degree of freedom left. Let \emph{c} be Thinker's new credence in
\emph{r}, so their credence in \emph{q} is 0.7-\emph{c}. Then the
expected inaccuracy

\phantomsection\label{refs}
\begin{CSLReferences}{1}{0}
\bibitem[\citeproctext]{ref-Blackwell1951}
Blackwell, David. 1951. {``Comparison of Experiments.''}
\emph{Proceedings of the Berkeley Symposium on Mathematical Statistics
and Probability} 2 (1): 93--102.

\bibitem[\citeproctext]{ref-Joyce1998}
Joyce, James M. 1998. {``A Non-Pragmatic Vindication of Probabilism.''}
\emph{Philosophy of Science} 65 (4): 575--603. doi:
\href{https://doi.org/10.1086/392661}{10.1086/392661}.

\bibitem[\citeproctext]{ref-Lewis1982c}
Lewis, David. 1982. {``Logic for Equivocators.''} \emph{No{û}s} 16 (3):
431--41. doi:
\href{https://doi.org/10.1017/cbo9780511625237.009}{10.1017/cbo9780511625237.009}.
Reprinted in his \emph{Papers in Philosophical Logic}, Cambridge:
Cambridge University Press, 1998, 97-110. References to reprint.

\bibitem[\citeproctext]{ref-Williamson2000}
Williamson, Timothy. 2000. \emph{{Knowledge and its Limits}}. Oxford
University Press.

\end{CSLReferences}



\noindent Unpublished. Posted online in 2024.


\end{document}
