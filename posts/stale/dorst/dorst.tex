% Options for packages loaded elsewhere
\PassOptionsToPackage{unicode}{hyperref}
\PassOptionsToPackage{hyphens}{url}
%
\documentclass[
  10pt,
  letterpaper,
  DIV=11,
  numbers=noendperiod,
  twoside]{scrartcl}

\usepackage{amsmath,amssymb}
\usepackage{setspace}
\usepackage{iftex}
\ifPDFTeX
  \usepackage[T1]{fontenc}
  \usepackage[utf8]{inputenc}
  \usepackage{textcomp} % provide euro and other symbols
\else % if luatex or xetex
  \usepackage{unicode-math}
  \defaultfontfeatures{Scale=MatchLowercase}
  \defaultfontfeatures[\rmfamily]{Ligatures=TeX,Scale=1}
\fi
\usepackage{lmodern}
\ifPDFTeX\else  
    % xetex/luatex font selection
  \setmainfont[ItalicFont=EB Garamond Italic,BoldFont=EB Garamond
Bold]{EB Garamond Math}
  \setsansfont[]{Europa-Bold}
  \setmathfont[]{Garamond-Math}
\fi
% Use upquote if available, for straight quotes in verbatim environments
\IfFileExists{upquote.sty}{\usepackage{upquote}}{}
\IfFileExists{microtype.sty}{% use microtype if available
  \usepackage[]{microtype}
  \UseMicrotypeSet[protrusion]{basicmath} % disable protrusion for tt fonts
}{}
\usepackage{xcolor}
\usepackage[left=1in, right=1in, top=0.8in, bottom=0.8in,
paperheight=9.5in, paperwidth=6.5in, includemp=TRUE, marginparwidth=0in,
marginparsep=0in]{geometry}
\setlength{\emergencystretch}{3em} % prevent overfull lines
\setcounter{secnumdepth}{3}
% Make \paragraph and \subparagraph free-standing
\ifx\paragraph\undefined\else
  \let\oldparagraph\paragraph
  \renewcommand{\paragraph}[1]{\oldparagraph{#1}\mbox{}}
\fi
\ifx\subparagraph\undefined\else
  \let\oldsubparagraph\subparagraph
  \renewcommand{\subparagraph}[1]{\oldsubparagraph{#1}\mbox{}}
\fi


\providecommand{\tightlist}{%
  \setlength{\itemsep}{0pt}\setlength{\parskip}{0pt}}\usepackage{longtable,booktabs,array}
\usepackage{calc} % for calculating minipage widths
% Correct order of tables after \paragraph or \subparagraph
\usepackage{etoolbox}
\makeatletter
\patchcmd\longtable{\par}{\if@noskipsec\mbox{}\fi\par}{}{}
\makeatother
% Allow footnotes in longtable head/foot
\IfFileExists{footnotehyper.sty}{\usepackage{footnotehyper}}{\usepackage{footnote}}
\makesavenoteenv{longtable}
\usepackage{graphicx}
\makeatletter
\def\maxwidth{\ifdim\Gin@nat@width>\linewidth\linewidth\else\Gin@nat@width\fi}
\def\maxheight{\ifdim\Gin@nat@height>\textheight\textheight\else\Gin@nat@height\fi}
\makeatother
% Scale images if necessary, so that they will not overflow the page
% margins by default, and it is still possible to overwrite the defaults
% using explicit options in \includegraphics[width, height, ...]{}
\setkeys{Gin}{width=\maxwidth,height=\maxheight,keepaspectratio}
% Set default figure placement to htbp
\makeatletter
\def\fps@figure{htbp}
\makeatother
% definitions for citeproc citations
\NewDocumentCommand\citeproctext{}{}
\NewDocumentCommand\citeproc{mm}{%
  \begingroup\def\citeproctext{#2}\cite{#1}\endgroup}
\makeatletter
 % allow citations to break across lines
 \let\@cite@ofmt\@firstofone
 % avoid brackets around text for \cite:
 \def\@biblabel#1{}
 \def\@cite#1#2{{#1\if@tempswa , #2\fi}}
\makeatother
\newlength{\cslhangindent}
\setlength{\cslhangindent}{1.5em}
\newlength{\csllabelwidth}
\setlength{\csllabelwidth}{3em}
\newenvironment{CSLReferences}[2] % #1 hanging-indent, #2 entry-spacing
 {\begin{list}{}{%
  \setlength{\itemindent}{0pt}
  \setlength{\leftmargin}{0pt}
  \setlength{\parsep}{0pt}
  % turn on hanging indent if param 1 is 1
  \ifodd #1
   \setlength{\leftmargin}{\cslhangindent}
   \setlength{\itemindent}{-1\cslhangindent}
  \fi
  % set entry spacing
  \setlength{\itemsep}{#2\baselineskip}}}
 {\end{list}}
\usepackage{calc}
\newcommand{\CSLBlock}[1]{\hfill\break\parbox[t]{\linewidth}{\strut\ignorespaces#1\strut}}
\newcommand{\CSLLeftMargin}[1]{\parbox[t]{\csllabelwidth}{\strut#1\strut}}
\newcommand{\CSLRightInline}[1]{\parbox[t]{\linewidth - \csllabelwidth}{\strut#1\strut}}
\newcommand{\CSLIndent}[1]{\hspace{\cslhangindent}#1}

\setlength\heavyrulewidth{0ex}
\setlength\lightrulewidth{0ex}
\usepackage[automark]{scrlayer-scrpage}
\clearpairofpagestyles
\cehead{
  Brian Weatherson
  }
\cohead{
  Trusting and Valuing Infinite Frames
  }
\ohead{\bfseries \pagemark}
\cfoot{}
\makeatletter
\newcommand*\NoIndentAfterEnv[1]{%
  \AfterEndEnvironment{#1}{\par\@afterindentfalse\@afterheading}}
\makeatother
\NoIndentAfterEnv{itemize}
\NoIndentAfterEnv{enumerate}
\NoIndentAfterEnv{description}
\NoIndentAfterEnv{quote}
\NoIndentAfterEnv{equation}
\NoIndentAfterEnv{longtable}
\NoIndentAfterEnv{abstract}
\renewenvironment{abstract}
 {\vspace{-1.25cm}
 \quotation\small\noindent\rule{\linewidth}{.5pt}\par\smallskip
 \noindent }
 {\par\noindent\rule{\linewidth}{.5pt}\endquotation}
\usepackage{amsfonts}
\KOMAoption{captions}{tableheading}
\makeatletter
\@ifpackageloaded{caption}{}{\usepackage{caption}}
\AtBeginDocument{%
\ifdefined\contentsname
  \renewcommand*\contentsname{Table of contents}
\else
  \newcommand\contentsname{Table of contents}
\fi
\ifdefined\listfigurename
  \renewcommand*\listfigurename{List of Figures}
\else
  \newcommand\listfigurename{List of Figures}
\fi
\ifdefined\listtablename
  \renewcommand*\listtablename{List of Tables}
\else
  \newcommand\listtablename{List of Tables}
\fi
\ifdefined\figurename
  \renewcommand*\figurename{Figure}
\else
  \newcommand\figurename{Figure}
\fi
\ifdefined\tablename
  \renewcommand*\tablename{Table}
\else
  \newcommand\tablename{Table}
\fi
}
\@ifpackageloaded{float}{}{\usepackage{float}}
\floatstyle{ruled}
\@ifundefined{c@chapter}{\newfloat{codelisting}{h}{lop}}{\newfloat{codelisting}{h}{lop}[chapter]}
\floatname{codelisting}{Listing}
\newcommand*\listoflistings{\listof{codelisting}{List of Listings}}
\makeatother
\makeatletter
\makeatother
\makeatletter
\@ifpackageloaded{caption}{}{\usepackage{caption}}
\@ifpackageloaded{subcaption}{}{\usepackage{subcaption}}
\makeatother
\ifLuaTeX
  \usepackage{selnolig}  % disable illegal ligatures
\fi
\usepackage{bookmark}

\IfFileExists{xurl.sty}{\usepackage{xurl}}{} % add URL line breaks if available
\urlstyle{same} % disable monospaced font for URLs
\hypersetup{
  pdftitle={Trusting and Valuing Infinite Frames},
  pdfauthor={Brian Weatherson},
  hidelinks,
  pdfcreator={LaTeX via pandoc}}

\title{Trusting and Valuing Infinite Frames}
\author{Brian Weatherson}
\date{2024}

\begin{document}
\maketitle
\begin{abstract}
I show that some recent results concerning finite probability frames do
not generalise to infinite frames.
\end{abstract}

\setstretch{1.1}
Say a probability frame is an ordered pair ⟨W, P⟩ such that W is a set
(intuitively, of worlds), and P is a function from W to probability
functions defined on w. Say that π is a probability function defined
over w. Intuitively, π is Novice's credence function, and P maps members
of W to the credence a special person called The Expert has at each
world. The philosophical question in the background, and it will stay in
the background for this note, is what relationship must hold between π
and ⟨W, P⟩ to say that Novice defers to The Expert.

Kevin Dorst (\citeproc{ref-Dorst2019}{2019}) made some suggestions about
what this relationship must look like, but it turned out these weren't
quite right. In Dorst et al. (\citeproc{ref-DorstEtAl2021}{2021}), it is
shown that if W is finite, then the following two claims are equivalent.
(I'll state the claims formally, then explain my notation.)

\begin{description}
\tightlist
\item[Total Trust]
E(X~\textbar~\{\emph{w}:~E(X,~P(\emph{w}))~⩾~\emph{t}\},~π)~⩾~\emph{t}
\item[Value]
If O is a set of options, \emph{s} is a recommended strategy for O, and
\emph{o} is a member of O, then E(\emph{s},~π)~⩾~E(\emph{o},~π).
\end{description}

There is a bit there to unpack. E is the expectation function, so E(X,
Pr) is the expectation of X according to Pr. We'll generalise this a
little to say that E(X~\textbar~p, Pr) is the expectation of X according
to Pr(•~\textbar~p). So here's what Total Trust says. Take any random
variable X. (In this context, a random variable is just a function from
W to \(\mathbb{R}\).) Update π on the proposition that consists of all
and only worlds where The Expert at that world has an expected value for
X at least equal to \emph{t}. After that update, Novice also expects X's
value to be at least \emph{t}.

A decision problem is just a set of random variables. Intuitively, the
problem is that Novice has to choose which bet to take from the set O,
with the return being the value of the chosen bet at the actual world. A
strategy for Novice is to defer the decision to The Expert. Formally, a
strategy \emph{s} is a function from W to O; it picks a random variable,
i.e., a bet, at each world. For a strategy to be \emph{recommended}, it
has to satisfy two constraints. First, if P(\emph{i})~=~P(\emph{j}),
then \emph{s}(\emph{i})~=~\emph{s}(\emph{j}). A strategy can't be more
discriminating than The Expert's credences. For the second constraint,
associate each strategy \emph{s} with a random variable S such that
S(\emph{w})~=~\emph{s}(\emph{w})(\emph{w}). That is, the value of S at
\emph{w} is the return at \emph{w} of the member of O that \emph{s}
selects at \emph{w}. The constraint then says that for any
\emph{O}\textsubscript{1} ∈ O,
E(\emph{s}(\emph{w}),~P(\emph{w}))~⩾~E(O\textsubscript{1},~P(\emph{w})).
That is, at \emph{w}, The Expert does not think there is a member of O
that in expectation does better than \emph{s}(\emph{w}). Given all that,
Value says that The Novice does not expect any member of O to do better
than any recommended strategy.

One of the things Dorst et al. (\citeproc{ref-DorstEtAl2021}{2021}) show
is that if W and O are finite, then Total Trust and Value are
equivalent. As they put it, π Totally Trusts a frame ⟨W, P⟩ iff it
Values that frame. What I'll show is that this equivalence breaks down
when we drop the finiteness assumptions. Indeed, it breaks down even
when W is countably infinite.

Start with a frame I'll call \textbf{Coin}. A fair coin will be flipped
repeatedly until it lands Tails. Let F be a random variable such that
F~=~\emph{x} iff the coin is flipped \emph{x} times. (If the coin never
lands Tails, we'll stipulate that F~=~1. Since this has probability 0,
it doesn't make a difference to what follows, but this case will matter
below.) Novice knows these facts about F, so
π(F~=~\emph{x})~=~2\textsuperscript{-\emph{x}}. If F~=~\emph{x}, then
The Expert knows F~⩾~\emph{x}, and updates on that. That is,
P(F~=~\emph{x})~=~π(•\textbar F~⩾~x). For any positive integer \emph{i},
let O\textsubscript{i} be the random variable that takes value 0 at
F~=~\emph{j}~when \emph{j}~⩽~\emph{i}, and value
2\textsuperscript{\emph{i}} at F~=~\emph{j} when \emph{j}~\textgreater{}
\emph{i}. Let O be the set of each O\textsubscript{\emph{i}}. The
strategy \emph{s} such that
\emph{s}(F~=~\emph{i})~=~O\textsubscript{\emph{i}} is recommended, as
can be easily checked. But E(\emph{s})~=~0, while for any \emph{o} ∈~O,
E(\emph{o},~π)~= ½. So Value fails on \textbf{Coin}.

On the other hand, π does Totally Trust \textbf{Coin}. {[}Note to self,
I need a proof here.{]}

So π Totally Trusts \textbf{Coin}, but doesn't Value it. So the
equivalence between Total Trust and Value fails here. But you might very
reasonably object on two scores. First, the value function used to
generate the counterexample was unbounded, and we know that unbounded
value functions lead to all sorts of paradoxes. Second, I didn't just
make W infinite, I made O infinite as well, so this isn't a minimal
generalisation of the original claim. It turns out that if we put both
these constraints on, then the equivalence fails in the other direction:
It is possible to get a frame that π Values, but does not Totally Trust.

Call the following frame \textbf{Bentham}. Again, a coin will be flipped
until it lands Tails. If it ever lands

\phantomsection\label{refs}
\begin{CSLReferences}{1}{0}
\bibitem[\citeproctext]{ref-Dorst2019}
Dorst, Kevin. 2019. {``Evidence: A Guide for the Uncertain.''}
\emph{Philosophy and Phenomenological Research} 100 (3): 586--632. doi:
\href{https://doi.org/10.1111/phpr.12561}{10.1111/phpr.12561}.

\bibitem[\citeproctext]{ref-DorstEtAl2021}
Dorst, Kevin, Benjamin A. Levinstein, Bernhard Salow, Brooke E. Husic,
and Branden Fitelson. 2021. {``Deference Done Better.''}
\emph{Philosophical Perspectives} 35 (1): 99--150. doi:
\href{https://doi.org/10.1111/phpe.12156}{10.1111/phpe.12156}.

\end{CSLReferences}



\noindent \vspace{1in} In progress.

\end{document}
