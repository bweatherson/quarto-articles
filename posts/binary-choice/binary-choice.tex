% Options for packages loaded elsewhere
% Options for packages loaded elsewhere
\PassOptionsToPackage{unicode}{hyperref}
\PassOptionsToPackage{hyphens}{url}
%
\documentclass[
  11pt,
  letterpaper,
  DIV=11,
  numbers=noendperiod,
  twoside]{scrartcl}
\usepackage{xcolor}
\usepackage[left=1.1in, right=1in, top=0.8in, bottom=0.8in,
paperheight=9.5in, paperwidth=7in, includemp=TRUE, marginparwidth=0in,
marginparsep=0in]{geometry}
\usepackage{amsmath,amssymb}
\setcounter{secnumdepth}{3}
\usepackage{iftex}
\ifPDFTeX
  \usepackage[T1]{fontenc}
  \usepackage[utf8]{inputenc}
  \usepackage{textcomp} % provide euro and other symbols
\else % if luatex or xetex
  \usepackage{unicode-math} % this also loads fontspec
  \defaultfontfeatures{Scale=MatchLowercase}
  \defaultfontfeatures[\rmfamily]{Ligatures=TeX,Scale=1}
\fi
\usepackage{lmodern}
\ifPDFTeX\else
  % xetex/luatex font selection
  \setmainfont[ItalicFont=EB Garamond Italic,BoldFont=EB Garamond
Bold]{EB Garamond Math}
  \setsansfont[]{EB Garamond}
  \setmathfont[]{Garamond-Math}
\fi
% Use upquote if available, for straight quotes in verbatim environments
\IfFileExists{upquote.sty}{\usepackage{upquote}}{}
\IfFileExists{microtype.sty}{% use microtype if available
  \usepackage[]{microtype}
  \UseMicrotypeSet[protrusion]{basicmath} % disable protrusion for tt fonts
}{}
\usepackage{setspace}
% Make \paragraph and \subparagraph free-standing
\makeatletter
\ifx\paragraph\undefined\else
  \let\oldparagraph\paragraph
  \renewcommand{\paragraph}{
    \@ifstar
      \xxxParagraphStar
      \xxxParagraphNoStar
  }
  \newcommand{\xxxParagraphStar}[1]{\oldparagraph*{#1}\mbox{}}
  \newcommand{\xxxParagraphNoStar}[1]{\oldparagraph{#1}\mbox{}}
\fi
\ifx\subparagraph\undefined\else
  \let\oldsubparagraph\subparagraph
  \renewcommand{\subparagraph}{
    \@ifstar
      \xxxSubParagraphStar
      \xxxSubParagraphNoStar
  }
  \newcommand{\xxxSubParagraphStar}[1]{\oldsubparagraph*{#1}\mbox{}}
  \newcommand{\xxxSubParagraphNoStar}[1]{\oldsubparagraph{#1}\mbox{}}
\fi
\makeatother


\usepackage{longtable,booktabs,array}
\usepackage{calc} % for calculating minipage widths
% Correct order of tables after \paragraph or \subparagraph
\usepackage{etoolbox}
\makeatletter
\patchcmd\longtable{\par}{\if@noskipsec\mbox{}\fi\par}{}{}
\makeatother
% Allow footnotes in longtable head/foot
\IfFileExists{footnotehyper.sty}{\usepackage{footnotehyper}}{\usepackage{footnote}}
\makesavenoteenv{longtable}
\usepackage{graphicx}
\makeatletter
\newsavebox\pandoc@box
\newcommand*\pandocbounded[1]{% scales image to fit in text height/width
  \sbox\pandoc@box{#1}%
  \Gscale@div\@tempa{\textheight}{\dimexpr\ht\pandoc@box+\dp\pandoc@box\relax}%
  \Gscale@div\@tempb{\linewidth}{\wd\pandoc@box}%
  \ifdim\@tempb\p@<\@tempa\p@\let\@tempa\@tempb\fi% select the smaller of both
  \ifdim\@tempa\p@<\p@\scalebox{\@tempa}{\usebox\pandoc@box}%
  \else\usebox{\pandoc@box}%
  \fi%
}
% Set default figure placement to htbp
\def\fps@figure{htbp}
\makeatother


% definitions for citeproc citations
\NewDocumentCommand\citeproctext{}{}
\NewDocumentCommand\citeproc{mm}{%
  \begingroup\def\citeproctext{#2}\cite{#1}\endgroup}
\makeatletter
 % allow citations to break across lines
 \let\@cite@ofmt\@firstofone
 % avoid brackets around text for \cite:
 \def\@biblabel#1{}
 \def\@cite#1#2{{#1\if@tempswa , #2\fi}}
\makeatother
\newlength{\cslhangindent}
\setlength{\cslhangindent}{1.5em}
\newlength{\csllabelwidth}
\setlength{\csllabelwidth}{3em}
\newenvironment{CSLReferences}[2] % #1 hanging-indent, #2 entry-spacing
 {\begin{list}{}{%
  \setlength{\itemindent}{0pt}
  \setlength{\leftmargin}{0pt}
  \setlength{\parsep}{0pt}
  % turn on hanging indent if param 1 is 1
  \ifodd #1
   \setlength{\leftmargin}{\cslhangindent}
   \setlength{\itemindent}{-1\cslhangindent}
  \fi
  % set entry spacing
  \setlength{\itemsep}{#2\baselineskip}}}
 {\end{list}}
\usepackage{calc}
\newcommand{\CSLBlock}[1]{\hfill\break\parbox[t]{\linewidth}{\strut\ignorespaces#1\strut}}
\newcommand{\CSLLeftMargin}[1]{\parbox[t]{\csllabelwidth}{\strut#1\strut}}
\newcommand{\CSLRightInline}[1]{\parbox[t]{\linewidth - \csllabelwidth}{\strut#1\strut}}
\newcommand{\CSLIndent}[1]{\hspace{\cslhangindent}#1}



\setlength{\emergencystretch}{3em} % prevent overfull lines

\providecommand{\tightlist}{%
  \setlength{\itemsep}{0pt}\setlength{\parskip}{0pt}}



 


\setlength\heavyrulewidth{0ex}
\setlength\lightrulewidth{0ex}
\usepackage[automark]{scrlayer-scrpage}
\clearpairofpagestyles
\cehead{
  Brian Weatherson
  }
\cohead{
  Is Choice Binary?
  }
\ohead{\bfseries \pagemark}
\cfoot{}
\makeatletter
\newcommand*\NoIndentAfterEnv[1]{%
  \AfterEndEnvironment{#1}{\par\@afterindentfalse\@afterheading}}
\makeatother
\NoIndentAfterEnv{itemize}
\NoIndentAfterEnv{enumerate}
\NoIndentAfterEnv{description}
\NoIndentAfterEnv{quote}
\NoIndentAfterEnv{equation}
\NoIndentAfterEnv{longtable}
\NoIndentAfterEnv{abstract}
\renewenvironment{abstract}
 {\vspace{-1.25cm}
 \quotation\small\noindent\emph{Abstract}:}
 {\endquotation}
\newfontfamily\tfont{EB Garamond}
\addtokomafont{disposition}{\rmfamily}
\addtokomafont{title}{\normalfont\itshape}
\let\footnoterule\relax
\usepackage{amsfonts}
\KOMAoption{captions}{tableheading}
\makeatletter
\@ifpackageloaded{caption}{}{\usepackage{caption}}
\AtBeginDocument{%
\ifdefined\contentsname
  \renewcommand*\contentsname{Table of contents}
\else
  \newcommand\contentsname{Table of contents}
\fi
\ifdefined\listfigurename
  \renewcommand*\listfigurename{List of Figures}
\else
  \newcommand\listfigurename{List of Figures}
\fi
\ifdefined\listtablename
  \renewcommand*\listtablename{List of Tables}
\else
  \newcommand\listtablename{List of Tables}
\fi
\ifdefined\figurename
  \renewcommand*\figurename{Figure}
\else
  \newcommand\figurename{Figure}
\fi
\ifdefined\tablename
  \renewcommand*\tablename{Table}
\else
  \newcommand\tablename{Table}
\fi
}
\@ifpackageloaded{float}{}{\usepackage{float}}
\floatstyle{ruled}
\@ifundefined{c@chapter}{\newfloat{codelisting}{h}{lop}}{\newfloat{codelisting}{h}{lop}[chapter]}
\floatname{codelisting}{Listing}
\newcommand*\listoflistings{\listof{codelisting}{List of Listings}}
\makeatother
\makeatletter
\makeatother
\makeatletter
\@ifpackageloaded{caption}{}{\usepackage{caption}}
\@ifpackageloaded{subcaption}{}{\usepackage{subcaption}}
\makeatother
\usepackage{bookmark}
\IfFileExists{xurl.sty}{\usepackage{xurl}}{} % add URL line breaks if available
\urlstyle{same}
\hypersetup{
  pdftitle={Is Choice Binary?},
  pdfauthor={Brian Weatherson},
  hidelinks,
  pdfcreator={LaTeX via pandoc}}


\title{Is Choice Binary?}
\author{Brian Weatherson}
\date{2025}
\begin{document}
\maketitle
\begin{abstract}
There is a natural view about how preference relates to choice. On this
view, an option is choiceworthy iff no alternative is strictly preferred
to it. I'm going to argue against this, for two reasons. First, the view
makes a false prediction about which options are choiceworthy in games
and in choices between options that differ along multiple dimensions.
Second, choice is fundamentally about how options are evaluated prior to
a decision, while preference is about how they are (or will be)
evaluated post decision. Several consequences of rejecting this natural
view are explored, including how it simplifies the relationship between
game theory and decision theory, and how it complicates several debates
about so-called `incomplete' preferences.
\end{abstract}


\setstretch{1.1}
Subjective decision theory concerns norms about three kinds of thing:
values, preferences, and choices. A common assumption, at least
implictly, is that norms about preferences are prior to norms on values
and choices. One way to put this assumption, following Amartya Sen
(\citeproc{ref-Sen1970sec}{{[}1970{]} 2017}), is that choice functions
are \emph{binary}; they are grounded in binary relations of preference
and indifference.

I'm going to argue against this for two reasons. First, preferences,
being binary comparisons, don't provide a rich enough base to ground all
the norms. Sometimes we need to take as primitive comparisons the
chooser (hereafter, Chooser) makes between larger sets of options.
Second, preference is an ex post notion, in a sense to be made clearer
starting in Section~\ref{sec-multieq}, while choiceworthiness is an ex
ante notion. And ex ante norms are not grounded in ex post ones.

\section{Choice and Choiceworthiness}\label{sec-choice}

Choice gets less attention in philosophical decision theory than one
might have expected. The focus is usually on either value, e.g., this
has value 17 and that has value 12, or preference, e.g., this is
preferable to that. Norms on choice are almost an afterthought in
standard presentations. After a long discussion of values, preferences,
or both, the typical theorist breezily says that the norm is to choose
the most valuable, or most preferred option.

There is a long tradition in economics, going back to Paul Samuelson
(\citeproc{ref-Samuelson1938}{1938}) and Herman Chernoff
(\citeproc{ref-Chernoff1954}{1954}), of taking choice to be primary.
Some of this literature rested on largely behaviourist or positivist
assumptions. It was better to theorise with and about choice because it
was observable, unlike preferences or values. The picture was not
dissimilar to this recently expressed view,

\begin{quote}
Standard economics does not address mental processes and, as a result,
economic abstractions are typically not appropriate for describing them.
(\citeproc{ref-GulPesendorfer2008}{Gul and Pesendorfer 2008, 24})
\end{quote}

That's not going to be my approach here. I'm going to start not with
observable choice dispositions, like the economists, or with choice
frequencies, as psychologists like Luce (\citeproc{ref-Luce1959}{1959})
do, but with judgments about choiceworthiness. In familiar
terminology\footnote{See, for instance, Hansson and Grüne-Yanoff
  (\citeproc{ref-sep-preferences}{2024}), where I learned about the Gul
  and Pesendorfer quote}, I'm taking a mentalist approach not a
behaviourist approach. Much of the formal work on choice theory has been
done by theorists from the behaviourist side, and I'll be inevitably
drawing from them. But the most important source I'll be using is
someone much more sympathetic to mentalism: Amartya Sen.~In particular,
I'll draw heavily on his ``Collective Choice and Social Welfare''
(\citeproc{ref-Sen1970sec}{Sen {[}1970{]} 2017}), and also on the
literature that grew out of that book.

I'm not going to take a stance on the metaphysics of choicewortiness
judgments. I'll sometimes talk as if they are beliefs, but if someone
wanted to have a sharp belief-desire split, and hold that
choiceworthiness, like preference, involves an interplay of the two, I
wouldn't object. The real assumption is that the mental state being
ascribed in choiceworthiness ascriptions is the same kind of state as
that being ascribed in preference ascriptions. The biggest difference is
that choiceworthiness is a relation to an arbitrarily sized set of
options, while preference is a relation to a pair of options.

\section{Values First?}\label{sec-values}

Both to clarify what kind of question I'm asking, and to set aside one
kind of answer, I'm going to start not with preferences or choices, but
with numerical values. At first glance, it might seem that many decision
theorists take values to ground the other norms. One should prefer the
more valuable. So we get literatures like the Newcomb's Problem
literature which revolve largely around how to state the value of
uncertain outcomes (like taking both boxes) in terms of the values and
probabilities of those outcomes.

On second glance, though, it is implausible that views like the value of
this outcome is 17 are really what ground the other attitudes. This is
so for four main reasons.

First, it's surprising to have a numerical measure like this not have a
unit. We'll sometimes say things like this outcome has value 17
\emph{utils} but this is a placeholder, not a real unit like kilograms
or volts. This is related to the second reason.

The orthodox view is that these values are only defined up to a positive
affine transformation. If it's appropriate to represent Chooser with
value function \emph{v}, it's appropriate to represent them with any
value function \emph{f} where
\emph{f}(\emph{o})~=~\emph{av}(\emph{o})~+~\emph{b}, for positive
\emph{a}. Why is this transformation allowed? Because all the values are
doing is reflecting Chooser's preferences over outcomes and lotteries,
and this transformation doesn't change those preferences. In other
words, the transformation is allowed, and the definedness claim is true,
because values are not a basic ground.

Third, it's not at all obvious why values should be anything like
numbers. Indeed, the thought that they should be numbers starts causing
problems when we get to various puzzles about infinite goods.\footnote{See,
  e.g., Nover and Hàjek (\citeproc{ref-Nover2004}{2004}), or Goodman and
  Lederman (\citeproc{ref-GoodmanLederemanArXiV}{2024}).} Why should
values be numbers rather than quintuples of numbers ordered lexically?
If one takes preferences to be primary, and generates utility functions
via representation theorems, as in Ramsey
(\citeproc{ref-RamseyTruthProb}{1926}) or Neumann and Morgenstern
(\citeproc{ref-vNM1944}{1944}), there is a reason for this. But if
values are primitive, it seems like an unanswered and I'd say
unanswerable question.

Finally, there is something very strange about the idea of values that
are not in any way comparative. How valuable something is just seems
like it should be a notion that reflects the value of alternatives to
it.

None of these are meant to be particularly contentious claims. Even
theorists who primarily talk about values would, I suspect, agree that
claims about values, and norms on values, are ultimately grounded in
something comparative like preferences.

What is going to be contentious is the claim that preferences can't do
the job, and judgments of, and norms about, choiceworthiness ultimately
ground the values and the preferences.

\section{Coherence}\label{sec-coherence}

There is one other striking thing about the picture we get in Neumann
and Morgenstern (\citeproc{ref-vNM1944}{1944}), and which is still I
think broadly endorsed by contemporary decision theorists. The aim is to
put norms on preference, and hence on values and choices. But these
norms are almost always defined in terms of other preferences. For
example, if one strictly prefers \emph{x} to \emph{y}, and \emph{y} to
\emph{z}, one should prefer \emph{x} to \emph{z}. It is strange to talk
about grounding the normative facts about preference when other
preferences play such a crucial role.

If there's a puzzle here, there are two (related) ways out. One is to
take the view, which perhaps Hume held, that the only constraints on
preference are coherence constraints. Another is to say that while there
might be some non-coherence constraints on preference, e.g., it is in
fact wrong to prefer the world's destruction to one's finger getting
scratched, these are part of a separate subject to decision theory. The
result is the same: decision theory largely is about what it takes for
various preferences to cohere with one another.

I don't particularly agree with this picture, but I'm going to accept
half of it for the purposes of this paper. That is, I'll assume the
person who prefers the destruction of the world to the scratching of
their finger is either not violating a norm, or, if they are, they are
violating a different kind of norm to the person who violates the
independence condition in Neumann and Morgenstern
(\citeproc{ref-vNM1944}{1944}). Decision theory, on the latter view,
takes Chooser's preferences over ends as given and judges Chooser on how
well their instrumental preferences serve these preferences over ends.

I'll argue that even if decision theory is about coherence, it should be
about coherence between choiceworthiness judgments. So the big question
is not whether preferences should satisfy transitivity or independence,
but whether their choiceworthiness judgments should satisfy the
conditions described in Section~\ref{sec-properties}. As we'll see, the
questions about which conditions are genuine coherence constraints on
choiceworthiness is tied up with the metaphysical question about the
priority of preferences and choices.

\section{Sen on Preference}\label{sec-sen}

****START EDITING HERE****

Sen starts with a binary relation \emph{R}, defined over options, which
is glossed as \emph{xRy} means that \emph{x} is at least as good as
\emph{y}. This is potentially misleading. It does not mean that \emph{x}
is either better than \emph{y}, or exactly as good as \emph{y}. Rather,
it means that \emph{x} is no worse than \emph{y}. Sen introduces two
more binary relations, \emph{xPy}, meaning \emph{x} is preferred to
\emph{y}, and \emph{xIy}, meaning the chooser is between \emph{x} and
\emph{y}. These can both be defined in terms of \emph{R}, as in (1) and
(2). (Throughout, I'm leaving off wide scope universal quantifiers over
free variables.)

\begin{enumerate}
\def\labelenumi{(\arabic{enumi})}
\tightlist
\item
  \emph{xPy}~↔︎~(\emph{xRy}~∧~¬\emph{yRx})
\item
  \emph{xIy}~↔︎~(\emph{xRy}~∧~\emph{yRx})
\end{enumerate}

A more common way of doing things in contemporary philosophy is to start
with \emph{P} and a fourth relation \emph{E}, where \emph{xEy} means
that \emph{x} and \emph{y} are equally good. On this picture, both (1)
and (2), but the explanatory direction in (1) is right-to-left. So
\emph{xRy} just is ¬\emph{xPy}, and then \emph{xIy} is still defined via
(2). On the version Sen uses, it's a little trickier to define \emph{E},
but (3) looks like a plausible conjecture.

\begin{enumerate}
\def\labelenumi{(\arabic{enumi})}
\setcounter{enumi}{2}
\tightlist
\item
  \emph{xEy}~↔︎~{[}(\emph{xRz}~↔︎~\emph{yRz})~∧~(\emph{zRx}~↔︎~\emph{zRy}){]}
\end{enumerate}

That is, two options are equally good iff they are substitutable in
other preference relations. Given all these results, we can show that
the following claims are all tightly connected.

\begin{enumerate}
\def\labelenumi{(\arabic{enumi})}
\setcounter{enumi}{3}
\tightlist
\item
  \emph{xPy}~∨~\emph{xEy}~∨~\emph{yPx}
\item
  (\emph{xPy}~∧~\emph{yIz})~→ \emph{xPz}
\item
  (\emph{xIy}~∧~\emph{yIz})~→ \emph{xIz}
\end{enumerate}

(4) is what Chang (\citeproc{ref-Chang2017}{2017}) calls the trichotomy
thesis. (5) is what Sen calls PI-transitivity, and (6) is what he calls
II-transitivity.

Sen makes very few assumptions about \emph{R}, but it will simplify our
discussion to start introducing some assumptions here.\footnote{In part
  this is because he was interested in exploring what assumptions about
  preference are crucial to the impossibility theorem that Arrow
  (\citeproc{ref-Arrow1951}{1951}) derives. He initially noticed that
  without (6), the theorem didn't go through. This turned out to be less
  significant than it seemed, because Allan Gibbard
  (\citeproc{ref-Gibbard2014}{2014}) proved that a very similar theorem
  can be proven even without (6). See Sen (\citeproc{ref-Sen1969}{1969})
  for the original optimism that this might lead to an interesting way
  out of the Arrovian results, and Sen
  (\citeproc{ref-Sen1970sec}{{[}1970{]} 2017}) for a more pessimistic
  assessment in light of Gibbard's result. Sen reports that Gibbard
  originally proved his result in a term paper for a seminar at Harvard
  in 1969 that he co-taught with Arrow and Rawls. Much of what I'm
  saying in this paper can be connected in various ways to the
  literature on Arrow's impossibility theorem, but I won't draw out
  those connections here.} We'll assume that \emph{R} is reflexive,
everything is at least as good as itself, and that \emph{P} is
transitive. (Sen calls this quasi-transitivity.) Sen
(\citeproc{ref-Sen1970sec}{{[}1970{]} 2017, 66}) notes that if \emph{P}
is transitive and \emph{R} is `complete' in the sense of the next
paragraph, then (5) and (6) are equivalent. It's also easy to show that
given (3) plus these assumptions, (4) and (6) are equivalent.\footnote{Proof:
  Assume (4) is false. So the right hand side of (3) is false. Without
  much loss of generality, assume that \emph{xRz}~∧~¬\emph{yRz}; the
  other cases all go much the same way. So all the disjuncts are false.
  From ¬\emph{xPy} and ¬\emph{yPx} we get \emph{yRx}~∧~\emph{xRy}, i.e.,
  \emph{xIy}. And \emph{xRz} implies \emph{zIx}. So we have a
  counterexample to II-transitivity, since \emph{zIx} and \emph{xIy},
  but since ¬\emph{yRz}, \emph{yIz} is false. So if (4) is false, (6) is
  false. In the other direction, assume we have a counterexample to (6),
  i.e., \emph{xIy} and \emph{yIz} but not \emph{xIz}. From \emph{xIy} we
  immediately get that the two outer disjuncts of (4) are false. From
  \emph{yIz} we get \emph{yRz} and \emph{zRy}. So if \emph{xEy}, (3)
  implies that \emph{xRz} and \emph{zRx}, i.e., \emph{xIz}. But we
  assumed that ¬\emph{xIz}. So all three disjuntcs of (4) are false.
  That is, if (6) fails, so does (4), completing the proof that they are
  equivalent.}

What should we call the principle (4)? Most philosophers call it
\emph{completeness}, and its denial \emph{incompleteness}. In his
economic work, Sen (\citeproc{ref-Sen1970sec}{{[}1970{]} 2017}) uses the
term `completeness' for a different property of preference relations,
namely \emph{xRy}~∨~\emph{yRx}. This is a useful notion to have. If
Chooser has never thought of \emph{x}, there is a good sense in which
\emph{xRy}~∨~\emph{yRx} is false, even though of course ¬\emph{yPx} is
true. Still, using the term for (4) is more common in philosophy. When
writing primarily for philosophers, Sen (\citeproc{ref-Sen2004}{2004})
uses `completeness' for (4), and I'll do the same. I'll call use
\emph{definedness} for \emph{xRy}~∨~\emph{yRx}, and unless stated
otherwise, will assume it holds.

There isn't as much discussion of (4) as such in the economics
literature, but there is a long tradition of discussing (6), going back
to important works by Wallace E. Armstrong
(\citeproc{ref-Armstrong1939}{1939}, \citeproc{ref-Armstrong1948}{1948},
\citeproc{ref-Armstrong1950}{1950}). In most of those works it is
assumed that \emph{P} is transitive, so (4) and (6) are equivalent, so
this is really discussing the same thing. Still, it makes the
terminology confusing.

When it makes it clearer, I'll use the term Chang
(\citeproc{ref-Chang2017}{2017}) suggests for (4). That is, I'll say
that preference relations for which this holds are \emph{trichotomous}.

\section{Properties of Choice Functions}\label{sec-properties}

In philosophy we're familiar enough with possible properties of
preference relations, e.g., that they are transitive, reflexive,
acyclic, etc, that these terms don't need to be defined. We're mostly
less familiar with properties of choice functions. So in this short
section I'll lay out six properties that will be important in what
follows. The first four are discussed in some detail by Sen
(\citeproc{ref-Sen1970sec}{{[}1970{]} 2017}), and I'll use his
terminology for them. The fifth is due to Aizerman and Malishevski
(\citeproc{ref-Aizerman1981}{1981}), and is usually named after
Aizerman. The sixth is discussed by Blair et al.
(\citeproc{ref-Blair1976}{1976}).

\begin{description}
\tightlist
\item[Property α]
(\emph{x}~∈~\emph{C}(\emph{S})~∧~\emph{x}~∈~\emph{T}~∧~\emph{T}~⊆~\emph{S})~→
\emph{x}~∈~\emph{C}(\emph{T})
\end{description}

That is, if \emph{x} is choiceworthy in a larger set, then it is
choiceworthy in any smaller set it is a member of. This is sometimes
called the \emph{Chernoff condition}, after Herman Chernoff
(\citeproc{ref-Chernoff1954}{1954}), and sometimes called
\emph{contraction consistency}.

\begin{description}
\tightlist
\item[Property β]
(\emph{x}~∈~\emph{C}(\emph{T})~∧~\emph{y}~∈~\emph{C}(\emph{T})~∧~\emph{T}~⊆~\emph{S})~→
(\emph{x}~∈~\emph{C}(\emph{S}) ↔ \emph{y}~∈~\emph{C}(\emph{S}))
\end{description}

That is, if \emph{x} and \emph{y} are both choiceworthy in a smaller
set, then in any larger set they are either both choiceworthy or neither
is. Intuitively, if \emph{x} is not chosen in the larger set because one
of the new members is better than it, then the new member is also better
than \emph{y}.

\begin{description}
\tightlist
\item[Property γ]
(\emph{x}~∈~\emph{C}(\emph{S})~∧~\emph{x}~∈~\emph{C}(\emph{T}))~→
(\emph{x}~∈~\emph{C}(\emph{S}~∪~\emph{T}))
\end{description}

That is, if \emph{x} is choiceworthy in two sets, it is choiceworthy in
their union. This is sometimes called \emph{expansion}, e.g., in Moulin
(\citeproc{ref-Moulin1985}{1985}).

\begin{description}
\tightlist
\item[Property δ]
(\emph{x}~∈~\emph{C}(\emph{T})~∧~\emph{y}~∈~\emph{C}(\emph{T})~∧~\emph{T}~⊆~\emph{S})~→
(\{\emph{y}\} ≠ \emph{C}(\emph{S}))
\end{description}

This is a weakening of β. It says that if \emph{x} and \emph{y} are both
choiceworthy in the smaller set, then after options are added, it can't
be that only one of them is the only choiceworthy option remaining. If
\emph{x} is not choiceworthy in the larger set, that's because some
other option, not \emph{y}, is chosen in place of it.

\begin{description}
\tightlist
\item[Property Aiz]
(\emph{C}(\emph{S})~⊆~\emph{T}~∧~\emph{T}~⊆~\emph{S})~→
\emph{C}(\emph{T})~⊆~\emph{C}(\emph{S})
\end{description}

That is, if the smaller set contains all of the choiceworthy members of
the larger set, then no option is choiceworthy in the smaller set but
not the larger set. If \emph{x} is a unchoiceworthy member of \emph{S},
then the only way to make it choiceworthy is by deleting choiceworthy
members of \emph{S}, not unchoiceworthy ones.

\begin{description}
\tightlist
\item[Path Independence]
\emph{C}(\emph{S}~∪~\emph{T}) =
\emph{C}(\emph{C}(\emph{S})~∪~\emph{C}(\emph{T}))
\end{description}

The same options are choiceworthy from a union of two sets as are
choiceworthy from the union of the choiceworthy members of those sets.
This is a sort of independence of irrelevant alternatives principle; the
availability or otherwise of unchoiceworthy members of \emph{S} and
\emph{T} doesn't affect what should be chosen from \emph{S}~∪~\emph{T}.

I'll describe the effects of these properties in more detail in
subsequent sections.

\section{Property α}\label{sec-alpha}

This is the most commonly used constraint on choice functions, and it
does seem intuitive. If \emph{x} is choiceworthy from a larger set,
deleting unchosen options shouldn't make it choiceworthy. Sen
(\citeproc{ref-Sen1970sec}{{[}1970{]} 2017, 323--26}) discusses two
possible counterexamples.

One is where the presence of options on the menu gives Chooser relevant
information. If the only two options are tea with a particular friend or
staying home, Chooser will take tea. But if the option of cocaine with
that friend is added, Chooser will stay home. The natural thing to say
here is that when one gets new information, \emph{C} changes, so there
isn't really a single \emph{C} here which violates α.\footnote{For a
  quick argument for that, if Chooser learns the only options are tea
  and staying home because the friend has just run out of cocaine, they
  might still stay home.}

The more interesting case is where the value Chooser puts on options is
dependent on what options are available. So imagine Chooser prefers more
cake to less, but does not want to take the last slice. If the available
options are zero slices or one slice of cake, Chooser will take zero.
But if two slices of cake is an option, Chooser will take one, again
violating α.

This is a trickier case, and the natural thing to say is that Chooser
doesn't really have the same options in the two cases. Taking the last
slice of cake isn't the same thing as taking one slice when two are
available. But this move has costs. In particular, it makes it hard to
say that \emph{C} should be defined for any set of options. It doesn't
clearly make sense to ask Chooser to pick between \emph{taking one
slice, which is the last}, and \emph{taking three slices when five are
available}.

Still, I'm going to set those issues aside and assume, like most
theorists do, and mostly assume that α is a constraint on coherent
choice functions, and that choice functions are defined over arbitrary
sets of options. In Section~\ref{sec-mixed} I'll come back to this last
assumption, but otherwise I'll assume it is in place.

\section{Assumptions}\label{sec-assumptions}

I've said a few times I'm assuming this or that, so it's a good time to
put in one place the assumptions I'm making. These aren't intended to
stack the deck in my favour; if any of them are false, I think it makes
something like my view (a) more plausible, but (b) harder to state.
Anyway, here's what has been assumed.

\begin{enumerate}
\def\labelenumi{\arabic{enumi}.}
\tightlist
\item
  \emph{P} is transitive, i.e., \emph{xRy}~∧~\emph{yRz}~→ \emph{xRz}
\item
  \emph{R} is `defined', i.e., \emph{xRy}~∨~\emph{yRx}.
\item
  \emph{R} is reflexive, i.e., \emph{xRx}.
\item
  \emph{C} is non-empty, i.e., \emph{C}(\emph{S}) ≠ ∅.
\item
  \emph{C} is defined everywhere, i.e., there is a universe of options
  \emph{U} all subsets of \emph{U} are in the domain of \emph{C}.
\item
  \emph{C} satisfies α.
\item
  The universe \emph{U} of options, that \emph{S} is a subset of, and
  \emph{x} is drawn from, is finite.
\end{enumerate}

In Section~\ref{sec-alpha} we saw one reason to reject 5, namely that we
might want to individuate options in terms of what else is available.
We'll see another in Section~\ref{sec-mixed} when we get to games with
no pure strategy equilibrium. But unless stated otherwise, I'll be
assuming these six things without comment.

When \emph{R} satisfies 1-3, I'll follow Luce
(\citeproc{ref-Luce1956}{1956}) and call it a \emph{semiorder}. When it
also satisfies trichotomy, i.e., (4), I'll follow standard practice and
call it a \emph{weak order}.

\section{Defining binariness}\label{sec-defining}

With these seven assumptions on board, it's easy to state what it is for
a choice function to be binary. Without them there would be a lot of
choice points in the definition, but now everything is simple.

First, we'll define an inversion function \emph{B} (for binary) that
maps preference relations to choice functions, and vice-versa. Both of
these are sets of ordered pairs, and we'll define the ordered pairs
directly. I'll assume that there is a universe \emph{U} of options, and
every option and set of options is drawn from it.

If the input to \emph{B} is a preference relation \emph{R}:

\begin{enumerate}
\def\labelenumi{(\arabic{enumi})}
\setcounter{enumi}{6}
\tightlist
\item
  \emph{B}(\emph{R}) = \{⟨\emph{S}, \emph{x}⟩:
  ∀\emph{y}(\emph{y}~∈~\emph{S}~→ \emph{xRy})\}
\end{enumerate}

That is, \emph{B}(\emph{R}) is the choice function which for any set
\emph{S} selects what Sen calls `maximal' members, those members to
which nothing is strictly preferred.\footnote{Hansson
  (\citeproc{ref-Hansson2009}{2009}) calls this the `liberal
  maximisation' rule. He contrasts it with five other rules, which are
  distinct in general but equivalent given \emph{R} is a semiorder.}

If the input to \emph{B} is a choice function \emph{C}:

\begin{enumerate}
\def\labelenumi{(\arabic{enumi})}
\setcounter{enumi}{7}
\tightlist
\item
  \emph{B}(\emph{C}) = \{⟨\emph{x}, \emph{y}⟩: x~∈~\emph{C}(\{\emph{x},
  \emph{y}\})\}
\end{enumerate}

That is, \emph{B}(\emph{R}) is the preference relation which says that
in any choices from pair sets, an element is chosen only if it is at
least as good as the other member. Sen
(\citeproc{ref-Sen1970sec}{{[}1970{]} 2017, 319}) calls this the `base
relation' as opposed to a more complicated `revealed preference
relation', and notes that the two are equivalent given α. Since we're
assuming α, we'll use the simpler version.

A choice function \emph{C} is \textbf{binary} if (9) holds:\footnote{Sen
  calls these functions `basic binary', but the distinction he's drawing
  attention to by adding `basic' doesn't make a difference given
  \emph{R} is a semi-order and α.}

\begin{enumerate}
\def\labelenumi{(\arabic{enumi})}
\setcounter{enumi}{8}
\tightlist
\item
  \emph{C} = \emph{B}(\emph{B}(\emph{C}))
\end{enumerate}

That is, if you convert \emph{C} into a preference relation, and back
into a choice function, you get the same thing back.

The core claim of this paper is that there are coherent choice functions
which are not binary. A related claim is that a plausible pair of
coherence constraints that you can state using \emph{B} do not in fact
hold. The constraints are that if \emph{C} and \emph{R} are an agent's
choice function and preference relation, then \emph{C} =
\emph{B}(\emph{R}), and \emph{R} = \emph{B}(\emph{C}).

\section{Property β}\label{property-ux3b2}

If we start with choice functions, the definition of \emph{E} in (3) is
too simple. A better definition is in (10).

\begin{enumerate}
\def\labelenumi{(\arabic{enumi})}
\setcounter{enumi}{9}
\tightlist
\item
  \emph{xEy} ↔ ∀\emph{S}(\{\emph{x}, \emph{y}\}~⊆~\emph{S}~→
  (\emph{x}~∈~\emph{C}(\emph{S}) ↔ \emph{y}~∈~\emph{C}(\emph{S})))
\end{enumerate}

That is, \emph{x} and \emph{y} are equal iff one is never chosen when
the other is not.\footnote{Without α, this is too weak, since it doesn't
  entail that \emph{x} and \emph{y} are intersubstitutable in general.
  But we won't worry about that.} Given this notion of equality, there
is an intuitive gloss on β: Two options are both choiceworthy iff they
are equal.\footnote{This gloss also assumes α.}

To see this, think about choice functions that are defined by starting
with numerical value functions, e.g., expected utility, and saying that
the choiceworthy options are those with maximal value. If \emph{x} and
\emph{y} are both choiceworthy in any set, they must have the same
value. That means in any set where either is choiceworthy, i.e., either
has maximal value, they both have maximal value, so both are
choiceworthy.

More generally, given the assumptions we're making, \emph{C} satisfies β
iff \emph{B}(\emph{C}) is trichotomous, which is equivalent to
\emph{B}(\emph{C}) satisfying II-transitivity. Unsurprisingly, the two
historically significant cases of intuitive counterexamples to
II-transitivity also generate intuitive counterexamples to β.

The first example involves distinct but indistinguishable
options.\footnote{The idea that humans can't distinguish similar options
  is important in Fechner (\citeproc{ref-Fechner1860}{1860}), a work
  which is discussed in Beiser (\citeproc{ref-sep-fechner}{2024}). The
  earliest connection I've found between this and indifference being
  intransitive is in Armstrong (\citeproc{ref-Armstrong1939}{1939}).
  Armstrong's example is rather confusing; the one I'll use here is
  based on Luce (\citeproc{ref-Luce1956}{1956}).} Assume that Chooser
prefers more sugar in their coffee to less, but can only tell two
options apart if they differ by 10 grains of sugar or more. Now consider
these three options:

\begin{quote}
\emph{x} = Coffee with 100 grains of sugar.\\
\emph{y} = Coffee with 106 grains of sugar.\\
\emph{z} = Coffee with 112 grains of sugar.
\end{quote}

This is said to be a counterexample to II-transitivity because Chooser
is indifferent between \emph{x} and \emph{y}, and between \emph{y} and
\emph{z}, but strictly prefers \emph{z} to \emph{x}. It's also a
counterexample to β. Chooser would choose either from \emph{x} and
\emph{y}, but when \emph{z} is added, \emph{y} is still choiceworthy but
\emph{z} is not.

This example was historically important, but it's not discussed that
much in the contemporary philosophical literature. It could be because
philosophers were convinced by the argument in Fara
(\citeproc{ref-Fara2001}{2001}) that phenomenal indistinguishability is
in fact transitive. But it did get widely discussed in economics,
especially once it started being discussed by R. Duncan Luce
(\citeproc{ref-Luce1956}{1956}, \citeproc{ref-Luce1959}{1959}), who used
these examples to argue that preferences form a semi-order.

The cases that were more important in the philosophical literature are
what Chang (\citeproc{ref-Chang1997}{1997}) call `small improvement'
cases. The earliest case I know of fitting this form is from Luce and
Raiffa (\citeproc{ref-LuceRaiffa1957}{1957}).\footnote{I haven't found a
  case like this in Luce's sole-authored works, and indeed Debreu
  (\citeproc{ref-Debreu1960}{1960}) notes that a related case raises
  problems for one of the central assumptions of Luce
  (\citeproc{ref-Luce1959}{1959}).} (In this,
\emph{P}(\emph{x},\emph{y}) is the probability that Chooser will select
\emph{x} when \emph{x} and \emph{y} are both available.)

\begin{quote}
Suppose that \emph{a} and \emph{b} are two alternatives of roughly
comparable value to some person, e.g., trips from New York City to Paris
and to Rome. Let \emph{c} be alternative a plus \$20 and \emph{d} be
alternative \emph{b} plus \$20. Clearly, in general \emph{P}(\emph{a},
\emph{c}) = 0 and \emph{P}(\emph{b}, \emph{d}) = 0. It also seems
perfectly plausible that for some people \emph{P}(\emph{b}, \emph{c})
\textgreater{} 0 and \emph{P}(\emph{a}, \emph{d}) \textgreater{} 0, in
which event \emph{a} and \emph{b} are not comparable, and so axiom 2
{[}i.e., (4){]} is violated. (\citeproc{ref-LuceRaiffa1957}{Luce and
Raiffa 1957, 375})
\end{quote}

An example with the same structure, involving a boy, a bicycle, and a
bell, is discussed in Lehrer and Wagner
(\citeproc{ref-LehrerWagner1985}{1985}), and mistakenly attributed to
Armstrong (\citeproc{ref-Armstrong1939}{1939}).\footnote{Many authors
  subsequently made the same attribution; if you want to see some
  examples, search for the word `pony' among the citations of
  Armstrong's paper on Google Scholar.}

The usual way these cases are discussed, starting with Luce and Raiffa,
is that they violate a certain kind of comparability. For example, Luce
and Raiffa say there is a sense in which the two holidays are `not
comparable'. I want to resist this reading. The core intuition here, I
think, is that β fails. Chooser would choose either option from
\{\emph{a}, \emph{b}\}, but if \emph{c} is added as an option, \emph{a}
becomes unchoiceworthy. If we add the assumption that \emph{R} =
\emph{B}(\emph{C}), then it does follow that trichotomy fails, and there
is a sense in which they are incomparable. But without that assumption,
it's consistent to say that these are counterexamples to β but not to
trichotomy. We'll return to this point in Section~\ref{sec-dorr}.

\section{Properties γ and δ}\label{sec-gamma}

Assume \emph{R} does not satisfy trichotomy, but is a semiorder, and
\emph{C} = \emph{B}(\emph{R}). Then β will fail, but γ and δ will hold.
Conversely, for any \emph{C} where γ and δ hold, there is a semiorder
\emph{R} such that \emph{C} = \emph{B}(\emph{R})
(\citeproc{ref-Sen1970sec}{Sen {[}1970{]} 2017, 320}). We're not going
to be very interested in δ, but we will be very interested in γ.

The reason γ holds when \emph{R} is a semiorder and \emph{C} =
\emph{B}(\emph{R}) is instructive. If \emph{x} is choiceworthy among
\emph{S}, then nothing in \emph{S} is better than \emph{x}. Similarly,
if \emph{x} is choiceworthy among \emph{T}, then nothing in \emph{T} is
better than \emph{x}. So nothing in \emph{S}~∪~\emph{T} is better than
\emph{x}. So \emph{x} is choiceworthy among
\emph{S}~∪~\emph{T}.\footnote{This argument doesn't appeal to \emph{R}
  being a semiorder, but if it is not, most theorists would not define
  \emph{B} this way.}

Conversely, if there are cases where \emph{C} should not satisfy γ, then
we'll have an argument that \emph{C} should not be based in some
semiorder \emph{R}. Showing that there are such cases will be one of the
main tasks of the rest of this paper.

We had two kinds of counterexamples to β, but only one of them will be
relevant here. I don't think there are any intuitive counterexamples to
γ that start with Fechner-style reflections on the intransitivity of
indifference. But there are going to be variations on the bicycle and
pony example that generate intuitive counterexamples to γ. We'll come
back to these in Section~\ref{sec-badcomp}.

It is common to say that when \emph{C} = \emph{B}(\emph{R}) for some
semiorder \emph{R}, that \emph{C} is \textbf{rationalizable}, and when
\emph{R} is a partial order, that \emph{C} is \textbf{rationalizable by
a partial order}. I find this terminology tendentious - why should
semiorders be the only things that can make \emph{C} rational? And as
we'll see in Section~\ref{sec-games}, it conflicts with the notion of a
choice being \emph{rationalizable} in game theory. But it's a common
enough terminology that I wanted to mention it here.

\section{Aizerman's Property}\label{sec-aiz}

The property Aiz is not particularly intuitive. Fortunately, it turns
out to be equivalent, given our assumptions to one that is:
Path-Independence.\footnote{Unless stated otherwise, the results in this
  section, along with proofs, can be found in the helpful survey by
  Hervé Moulin (\citeproc{ref-Moulin1985}{1985}).} That principle says
that to find what's choiceworthy from a union of sets, you only have to
consider which options are choiceworthy in the smaller sets.

Note that this isn't just saying that the options choiceworthy from the
union are choiceworthy from one of the members. That is implied by α.
What it is saying is that whether the unchosen options from S and T are
or aren't on the menu doesn't make a difference to which options are
choiceworthy from the union.

There is a very natural kind of model where α and Path-Independence
holds, but β and γ do not. Let \emph{O} be a set of total orderings of
\emph{U}. (A total ordering is a relation \emph{R} such that
\emph{xRy}~∨~\emph{yRx}~∨~x=y.) Then \emph{C}(\emph{S}) is the set of
Pareto-optimal members of \emph{S} relative to those orderings. That is,
it is the members of \emph{S} such that no other member of \emph{S} is
better according to every member of the set of orderings.

In fact, it turns out the converse of this is also true. If \emph{C}
satisfies α and Aiz, then there is some set of total orderings such that
\emph{C}(\emph{S}) is the set of Pareto optimal members of \emph{S}
according to that set.

It might seem strange after all the talk of weak orderings and
semiorders that we're now using total orders. Given that \emph{U} is
finite, this turns out to be unsurprising. Any semiorder, and hence any
weak order, is such that there is some set of total orders such that
\emph{x} is strictly preferred to \emph{y} in the semiorder iff it is
preferred in all the total orders. (In fact we can put a sharp cap on
how many such orders there must be.) So being Pareto optimal relative to
some set of semiorders (or weak orders) is equivalent to being Pareto
optimal relative to a larger set of total orders.

If \emph{C} is determined by a set of orders in this way, it is said to
be \textbf{pseudorationalizable}. These choice functions are not always
binary. Consider one simple example, where \emph{U} is
\{\emph{x},~\emph{y},~\emph{z}\},
\emph{C}(\emph{U})~=~\{\emph{x},~\emph{z}\}, and for any other \emph{S},
\emph{C}(\emph{S})~=~\emph{S}. That is the choice function determined by
the pair of orderings
\emph{x}~\textgreater~\emph{y}~\textgreater~\emph{z}~and
\emph{z}~\textgreater~\emph{y}~\textgreater~\emph{x}. This satisfies α,
δ and Aiz, but not β or γ. And it isn't binary. \emph{B}(\emph{C}) is
the universal relation \emph{R}, since whenever \emph{S} is a pair set,
\emph{C}(\emph{S})~=~\emph{S}.

I'm going to argue over subsequent sections that this is a coherent
choice function, and hence not all coherent choice functions are binary.

\section{Preference and Trade}\label{sec-trade}

Let's go back to why we might have wanted to focus on binary preference
relations. One natural reason is that our primary aim in looking at
preferene is to explain trade, and it is natural to give preferences a
central role in explaining trade. If Chooser trades a cow for some magic
beans, it's natural to explain that by saying Chooser preferred the
magic beans.

When we are looking to explain trade in a non-monetary economy, that
seems like a good reason to give preferences a central role in the
story. But very little trade these days involves barter like this. Most
trade involves money. Money is primarily valuable instrumentally. If
Chooser buys some shoes for \$100, we could say that Chooser prefers the
shoes to the money. But that doesn't seem like the end of the story,
since there is a reason why Chooser values the money as they do. The
deeper point is that the money Chooser has gives them a budget
constraint, and Chooser really thinks that the shoes are a better use of
the \$100 than anything else available. That is, it seems more
informative to describe Chooser as choosing the shoes from the menu of
things that cost \$100 than to describe them as preferring the shoes to
the money, since it's the fact about choice that explains why they have
that preference.

Not everyone will think these kinds of pragmatic considerations about
what we're aiming to explain should be relevant to what kinds of
attitudes we take to be primary. If so, you should think this point
about monetary economies is irrelevant to whether we should start with
preferences or choices. But if you do think the pragmatic considerations
are relevant, then you should think that choice functions are a better
starting point than preference relations for explaining economic
behaviour in monetary economies.

To be sure, if the original motivations for starting with preferences
are bad, the original motivations for moving to choice functions might
be worse. The focus on choice functions originally came from economists
who thought that preferences were unobservable, and hence that we should
focus on something observable like choice. If that's the best argument
against taking preference to be the basic notion, and hence accepting
\emph{C}~=~\emph{B}(\emph{R}), I'd stick with preferences. But as we'll
see, there are better reasons to be sceptical of this preference-first
approach.

\section{Degenerate Games}\label{sec-games}

Say a two-player game is \textbf{degenerate} iff the payoffs to one of
the players are constant in all outcomes of the game. For convenience,
say Column is the player with a constant outcome. So
Table~\ref{tbl-upmid}, Table~\ref{tbl-middown} and
Table~\ref{tbl-allthree} are degenerate games.

\begin{longtable}[]{@{}rcc@{}}
\caption{A degenerate two-option game}\label{tbl-upmid}\tabularnewline
\toprule\noalign{}
& Left & Right \\
\midrule\noalign{}
\endfirsthead
\toprule\noalign{}
& Left & Right \\
\midrule\noalign{}
\endhead
\bottomrule\noalign{}
\endlastfoot
Up & 10,0 & 0,0 \\
Middle & 1,0 & 1,0 \\
\end{longtable}

\begin{longtable}[]{@{}rcc@{}}
\caption{Another degenerate two-option
game}\label{tbl-middown}\tabularnewline
\toprule\noalign{}
& Left & Right \\
\midrule\noalign{}
\endfirsthead
\toprule\noalign{}
& Left & Right \\
\midrule\noalign{}
\endhead
\bottomrule\noalign{}
\endlastfoot
Middle & 1,0 & 1,0 \\
Down & 0,0 & 10,0 \\
\end{longtable}

\begin{longtable}[]{@{}rcc@{}}
\caption{A degenerate three-option
game}\label{tbl-allthree}\tabularnewline
\toprule\noalign{}
& Left & Right \\
\midrule\noalign{}
\endfirsthead
\toprule\noalign{}
& Left & Right \\
\midrule\noalign{}
\endhead
\bottomrule\noalign{}
\endlastfoot
Up & 10,0 & 0,0 \\
Middle & 1,0 & 1,0 \\
Down & 0,0 & 10,0 \\
\end{longtable}

Start with the following two assumptions, which seem fairly weak
relative to most practice in game theory.

\begin{itemize}
\tightlist
\item
  If a move is part of a Nash equilibrium, it is choiceworthy.
\item
  A move is choiceworthy only if there is some probability distribution
  over the other player's moves such that the move maximises expected
  utility given that distribution.\footnote{The notion of a
    rationalizable choice, in the sense of Bernheim
    (\citeproc{ref-Bernheim1984}{1984}) and Pearce
    (\citeproc{ref-Pearce1984}{1984}), slightly strengthens this. A
    choice is rationalizable iff it maximises expected utility relative
    to a probability assignment that only gives positive probability to
    the other player, or players, making rationalizable choices. That's
    circular as stated, but one can remove the circularity at the cost
    of making the definition somewhat less intuitive.}
\end{itemize}

In degenerate games, these necessary and sufficient conditions for
choiceworthiness coincide, but in general they are rather
different.\footnote{Since rationalizability is between these notions, it
  also coincides with them for degenerate games.}

In Table~\ref{tbl-upmid} and Table~\ref{tbl-middown}, both options are
choiceworthy by this standard. Middle-Right is a Nash equilibrium in
Table~\ref{tbl-upmid}, and Middle-Left is a Nash equilibrium in
Table~\ref{tbl-middown}. But in Table~\ref{tbl-allthree}, the only
choiceworthy options are Up and Down. Whatever probability Row assigns
to Left/Right, Middle will not maximise expected utility. So this is a
counterexample to γ. Middle is choiceworthy from \{Up, Middle\} and from
\{Middle, Down\}, but not from their union.

It follows immediately from Lemma 3 in Pearce
(\citeproc{ref-Pearce1984}{1984}) that in degenerate games, a choice
satisfies those conditions for being choiceworthy iff it is not strictly
dominated, where this includes being dominated by mixed strategies. So
in Table~\ref{tbl-allthree}, Middle is strictly dominated by the 50/50
mixture of Up and Down. This means that the choices will satisfy
Path-Independence. An option is not dominated by the options in
\emph{S}~∪~\emph{T} iff it is not dominated by the undominated options
in \emph{S}~∪~\emph{T}, i.e., by the options in
\emph{C}(\emph{S})~∪~\emph{C}(\emph{T}). So removing options that are
not choiceworthy in \emph{S} and \emph{T} from the union doesn't change
what is undominated, i.e., choiceworthy.

This last paragraph is the start of a pattern in the examples that
follow. Although I'll be arguing against γ, I won't be arguing against
Aiz/Path-Independence. I'm not going to offer anything like a conclusive
argument for Aiz, but the pattern suggests that it is the right
constraint to add to α.

\section{Choice Under Uncertainty}\label{sec-uncertainty}

Luce and Raiffa (\citeproc{ref-LuceRaiffa1957}{1957}) discuss what they
call choices under `uncertainty', by which they mean choices where
Chooser cannot assign probabilities to the states. Peterson
(\citeproc{ref-Peterson2017}{2017}) calls these choices under
`ignorance'. None of the proposed decisive rules for choice under
uncertainty/ignorance are particularly compelling; all lead to very
strange outcomes.

The best approach, in my opinion, is to treat these choices like
degenerate games. Indeed, degenerate games are really a paradigm of
choice under ignornace; Row has no reason to assign any particular
probability to Column's choice. Further, what the game theory textbooks
say about degenerate games seems fairly plausible; any undominated
option is choiceworthy. The same goes for choices under ignorance; any
undominated option is choiceworthy.

If that's right, then the three examples from Section~\ref{sec-games}
can be repurposed as examples of choice under uncertainty, replacing
Left and Right with \emph{p} and ¬\emph{p}, and the same analysis will
hold. Again γ will fail, because Middle is choiceworthy when there is
one other option, but not when they are two. So there's no binary
comparison of Middle with the other two options that explains the facts
about what is choiceworthy in the three cases.

\section{Multiple Equilibria}\label{sec-multieq}

This is a decision theory paper, so we need to introduce a demon who can
reliably predict Chooser's choices. We'll work with a version of what
Skyrms (\citeproc{ref-Skyrms1982}{1982}) calls `Nice Demon'. In
Table~\ref{tbl-nice-demon}, Chooser selects Up or Down, and Demon either
predicts Up (PU), or predicts Down (PD). Whatever Chooser does, Demon is
very likely to have predicted correctly.

\begin{longtable}[]{@{}rcc@{}}
\caption{First version of Nice
Demon}\label{tbl-nice-demon}\tabularnewline
\toprule\noalign{}
& PU & PD \\
\midrule\noalign{}
\endfirsthead
\toprule\noalign{}
& PU & PD \\
\midrule\noalign{}
\endhead
\bottomrule\noalign{}
\endlastfoot
Up & 6 & 0 \\
Down & 0 & 4 \\
\end{longtable}

Jack Spencer (\citeproc{ref-Spencer2023}{2023}) argues against views,
like that defended by Dmitri Gallow (\citeproc{ref-Gallow2020}{2020}),
which say only Up is choiceworthy in Table~\ref{tbl-nice-demon}. His
argument relies on a simple principle. If Chooser plans to play Down,
then Chooser knows Down will have the best return, and it's not
irrational to make the choice one knows will have the best return.
Gallow (\citeproc{ref-Gallow2024}{2024}) argues that knowledge might not
be a strong enough state to make this principle work, because of high
stakes gambles. I think knowledge should play the role in practical
reasoning Spencer assigns it (Weatherson
(\citeproc{ref-WeathersonKAHIS}{2024})), so I'll assume Spencer is right
here, and both Up and Down are choice-worthy in
Table~\ref{tbl-nice-demon}. Indeed generally in any problem with a nice
demon, any option that would be best were it chosen, is choiceworthy for
just the reason Spencer gives.

Now add a third option, Exit, which has a guaranteed return of 1. So the
game table looks like Table~\ref{tbl-nice-demon-exit}.

\begin{longtable}[]{@{}rcc@{}}
\caption{Second version of Nice
Demon}\label{tbl-nice-demon-exit}\tabularnewline
\toprule\noalign{}
& PU & PD \\
\midrule\noalign{}
\endfirsthead
\toprule\noalign{}
& PU & PD \\
\midrule\noalign{}
\endhead
\bottomrule\noalign{}
\endlastfoot
Up & 6 & 0 \\
Exit & 1 & 1 \\
Down & 0 & 4 \\
\end{longtable}

In Table~\ref{tbl-nice-demon-exit}, Exit is not choiceworthy. Whatever
credences Chooser has about what Demon has done, it is better in
expectation to choose one of Up or Exit. But if either Up or Down were
unavailable, Exit would be choiceworthy. So again this case is a
counterexample to γ.

There is an important general lesson from this case. What makes an
option choiceworthy in cases like this is that it is utility maximising
\emph{once it is chosen}. We'll turn next to a more dramatic
illustration of this point.

\section{Mixed Strategies}\label{sec-mixed}

Demon has stopped being nice, and now wants to play Rock-Paper-Scissors
with Chooser. Given Demon's powers, this could go badly for Chooser.
Happily, Chooser can choose to randomise\footnote{I'm not going to argue
  for this here, but I think it is part of being ideally practically
  rational that one is able to randomise, just like it is part of being
  ideally practically rational that one can make calculations
  costlessly.}, and all Demon can predict is the probability that
Chooser's random process will come up with any option. So the best thing
for Chooser to do is to pick one of the three options at random. That
alone will maximise expected utility conditional on being chosen.

Not coincidentally, the only Nash equilibrium of Rock-Paper-Scissors is
for each player to randomise. Nash equilibria are arguably the only
sensible strategies if one assumes that every other player has
Demon-like abilities to detect what one is doing. But it's a long
running puzzle in game theory how it can be uniquely rational to
randomise. Why can choosing randomly be better than choosing one of the
things one is randomising between? To turn this rhetorical question into
an argument, note that the following three principles are inconsistent.

\begin{enumerate}
\def\labelenumi{(\arabic{enumi})}
\setcounter{enumi}{10}
\tightlist
\item
  Randomising is the only choiceworthy strategy in Rock-Paper-Scissors.
\item
  If only one choice is choiceworthy, it is rationally preferred to all
  other choices.
\item
  It is irrational to prefer a random mixture of some choices to every
  one of the choices.
\end{enumerate}

Since (11) is a standard view in game theory, (12) is a standard view in
choice theory, and (13) is a standard view in preference-based decision
theory, it is a little disconcerting to see they are inconsistent.

The example in Section~\ref{sec-multieq} shows how to steer through this
trilemma. Choiceworthiness is fundamentally an ex ante notion, and
preference is fundamentally an ex post notion. The reason Spencer's view
about Table~\ref{tbl-nice-demon} is right is not that the rational
chooser is indifferent between Up and Down. It's that they don't have
preferences between them until they have chosen, and once they choose,
they prefer the choice.

Similarly in Rock-Paper-Scissors (especially against a Demon), what's
true is that prior to deciding, the only rational choice is to
randomise. Once one has chosen, one shouldn't have any preference over
the options; they each have the same expected utility. Indeed, one
should not prefer to re-randomise rather than just put into effect the
result of the random process.

So we should reject (12) in its most natural interpretation. Randomising
in Rock-Paper-Scissors is the only choiceworthy option, but until a
choice is made, Chooser simply shouldn't have preferences over these
options.

Most of the arguments in this paper against the binariness of choice
turn on counterexamples to γ, but this is a distinct argument.
Sometimes, as in Rock-Paper-Scissors, there are grounds for rational
choice, but no grounds for rational preference. The only preferences
that would ground the choice would violate (13). So rational
choiceworthiness cannot be grounded in rational preference.

This is the deepest reason why \emph{C} = \emph{B}(\emph{R}) must be
wrong; \emph{C} and \emph{R} are fundamentally about different kinds of
attitudes. \emph{C} is about what is rational prior to making a choice,
\emph{R} is about what's rational having made a choice. Outside of
Newcomb-like cases, this distinction won't often matter, but it is
another reason that the equation fails.

\section{Multiple Attributes and Decisiveness}\label{sec-sartre}

Sartre (\citeproc{ref-Sartre1946}{1946/2007}) has a famous example of a
young man, we'll call him Pierre, caught between two imperatives. The
actual example is complicated in interesting ways, but we'll work with a
very simple version of this. Pierre lives in occupied France during
WWII, and feels torn between his duty to care for his ailing mother, and
his duty to fight for his occupied country. What should he do?

The case is horribly underdescribed, but the following verdicts have
seemed plausible to many people. First, Pierre can rationally, and
morally, choose what to do here. This are both noble impulses, and it's
fine to follow other. Second, and this follows from the first and the
fact the case is underdescribed, the options are not equally good. After
all, a small improvement to either would not break a tie between them.
Third, there is something wrong about Pierre going back and forth
between the two choices; he should make a choice and stick to it. The
intuition I'm interested in here is that there is something
intrinsically good about settling on a choice and sticking to it. What
makes this intuition less than fully clear is that in any practical
version of the case, it will be very bad for Pierre to oscillate between
the two views. He could spend the whole war travelling between England
and France as he changes his mind on where he should be, and that would
be bad. The intuition I want to press is that there is something good
about taking a stance and committing to it, even outside of the
practical costs of changing one's mind.\footnote{As Moss
  (\citeproc{ref-Moss2015}{2015}) points out, it is less clear this is
  intrinsically bad if there is more time between the reconsiderations;
  it makes more sense to change one's mind than to rapidly flip-flop.
  That is perhaps one way to support the intuition in the text. If
  Pierre resembles the young Thomas Schelling (as discussed by Holton
  (\citeproc{ref-Holton1999}{1999})), firmly committing to one plan and
  then another over the course of successive nights, there's something
  wrong about that even if it isn't practically bad.}

A very simple model, which captures these intuitions, is that Pierre's
situation is surprisingly like the person playing
Table~\ref{tbl-nice-demon}. There are two options here, and either is
acceptable, but once one is chosen, it becomes the preferred one. There
are two good values here, caring for family and caring for country, and
Pierre's fundamental choice is to adopt one of these as his value. As
Chang (\citeproc{ref-Chang2024}{2024}) puts it, he chooses to put `his
whole self' behind one of those values. Once that choice is made, his
preferences and his actions follow naturally.

I do not want to say this is the only way to understand this example.
Perhaps Pierre could, as Chang puts it, \emph{drift} into one choice.
Perhaps he could adopt one path and sometime later rationally regret his
choice, because the other value strikes his later self as more
important. All I mean to say is that it seems possible that Pierre could
be someone for whom both choices are rational, but once he has committed
to one of them, it would be irrational to turn around and do the other.
If that's right, toy models like Table~\ref{tbl-nice-demon} are
surprisingly good models of real examples like this.

\section{β and Incompleteness}\label{sec-dorr}

The Pierre case in Section~\ref{sec-sartre} is, if the intuitions I was
prodding are correct, a straightforward counterexample to β. Just to
spell it out,

\begin{quote}
\emph{x} = Help mother\\
\emph{y} = Fight Nazis\\
\emph{z} = Help mother plus one extra ration book
\end{quote}

If the choices are \emph{x} and \emph{y}, either is acceptable. If the
choices are \emph{x}, \emph{y} and \emph{z}, \emph{x} alone is
unchoiceworthy. So β fails.

This is the Small Improvement argument, and it is often thought to be an
argument against completeness, i.e., (4). The point of this section is
to say why that argument might fail, even if the argument against β
works.

Imagine someone was convinced by the arguments in Dorr, Nebel, and Zuehl
(\citeproc{ref-DorrEtAl2023}{2023}) that (4) must be true. Their
arguments turn on semantic properties of comparatives; since \emph{R} is
a comparative, they say it must be complete since all comparatives are
complete.\footnote{For what it's worth, I think this argument fails
  because of the case of `stronger' in logic. They address this case,
  but I don't think their response works. But it would be a huge
  digression to follow that thread through.} Now it would be very
strange if the semantics of comparatives in natural language entailed
that these intuitions about β had to be false. And in fact these claims
about semantics do not entail that.\footnote{I'm focussing here on the
  argument in Dorr, Nebel, and Zuehl
  (\citeproc{ref-DorrEtAl2023}{2023}), but a similar response works if
  someone is convinced of (4) by the argument in Broome
  (\citeproc{ref-Broome1997}{1997}).}

The following view seems to me to be coherent, and if so it shows why
this view of comparatives does not raise a problem for β.

\begin{itemize}
\tightlist
\item
  β fails in cases like Pierre's.
\item
  An option is choiceworthy for Pierre iff no option is
  \emph{determinately} preferred to it.
\item
  To make a choice, Pierre must determine which value is really his.
\item
  Once he does that, his preferences will satisfy (4).
\item
  Before he does that there are two possibilities. One is that his
  preferences aren't even defined over these options, so asking which is
  preferred is like asking whether the number 7 is taller, shorter, or
  the same height as justice. Another is that it is vague what Pierre's
  preferences are, but any resolution of the vagueness makes (4) true.
  This latter option fits nicely with the idea that \emph{C} should
  satisfy Aiz, since it should be determined by a set of orderings, each
  of them the possible precisifications, or determinations, of his
  current state.\footnote{This view of preference mirrors the view of
    credence defended by Carr (\citeproc{ref-Carr2020}{2020}).}
\end{itemize}

One objection to this view is that it seems to imply that Pierre could
rationally choose one option while he prefers, but does not
determinately prefer, another. But this isn't what the view implies.
Once Pierre makes the choice, he must, if he is rational, determine that
his preferences match it. Preference, on this view, is fundamentally an
ex post notion. Just like the player in Table~\ref{tbl-nice-demon} must,
if they are rational, form beliefs that make their choice optimal,
what's true of Pierre is that he can make a choice while some other
option being preferred is consistent with his \emph{prior} preferences.
But preference, on this view, is fundamentally an ex post notion. If
Pierre is rational, he will choose what he prefers. It's just not true
that he'll always choose what he preferred prior to the choice. Choice,
on this picture, is prior to preference, both analytically and perhaps,
in cases like this, causally.

When I say this is coherent, I don't mean to half-heartedly say that it
is correct. My preferred view is that Pierre could rationally drift (in
Chang's sense) into either option, and if he does, (4) would fail even
ex post. All I mean to argue here is that the case against β doesn't
turn on this view, and is compatible with preferences being complete.

\section{Bad Compromises}\label{sec-badcomp}

My version of the Pierre example was very simple, but it allows for some
interesting complications. As stated, you might think Pierre isn't
thinking through his choices well enough. He should join the local
resistance, so he can stay close enough to his mother to help, while
also fighting the Nazis.

But maybe that's a terrible option. We can easily imagine that the
resistance is either so useless that it does practically nothing, or so
good at recruiting that it has little useful work. It's just as easy to
imagine that it creates busywork that dramatically reduces how much he
can care for his mother, while not doing much to help the war effort. At
risk of trivialising the issue, we can imagine that Pierre's options
look like this, where the two columns represent how much each option
respects/promotes the relevant value.

\begin{longtable}[]{@{}rcc@{}}
\caption{Pierre's options, including
resistance}\label{tbl-pierreresist}\tabularnewline
\toprule\noalign{}
& Care for mother & Fight Nazis \\
\midrule\noalign{}
\endfirsthead
\toprule\noalign{}
& Care for mother & Fight Nazis \\
\midrule\noalign{}
\endhead
\bottomrule\noalign{}
\endlastfoot
Stay Home & 10 & 0 \\
Join resistance & 1 & 1 \\
Join Free French & 0 & 10 \\
\end{longtable}

I'll come back to whether this is plausible in the next two sections,
but for now I'll just report an intuition that Pierre should not join
the resistance; it's almost the worst of both worlds.

But the case might still be a counterexample to γ. Consider these two
variants on the case.

\begin{quote}
\textbf{No exit}: While Pierre is deliberating, he hears that the
options for getting to the Free French have been decisively cut off. (In
the original, he worries this might happen.) Now his only two options
are to stay home, or to join the reistance.
\end{quote}

\begin{quote}
\textbf{Promise}: Pierre's brother Jean is fighting the Nazis. Pierre
has promised Jean that if Jean is killed, Pierre will take up the fight
in some way. Sadly, Jean is killed, and Pierre regards this promise as
binding. Now his only options are to join the resistance, or join the
Free French.
\end{quote}

In either case, the resistance seems choiceworthy. In both cases, it is
the option that does best of the remaining choices on one of the
criteria. Pierre could decide he endorses that criteria as his own, and
acts accordingly. So the resistance is choiceworthy amongst either pair
of options, but not amongst their union.

This case is a counterexample to γ, but it does not seem to generate
counterexamples to Path Independence. I can't see any variant of the
case where the presence or absence of unchoiceworthy options makes a
difference to what is choiceworthy.

\section{Levi and Sen}\label{sec-levisen}

In \emph{Hard Choices}, Isaac Levi (\citeproc{ref-Levi1986}{1986})
defended a view where the choiceworthy options are only those that
maximise value on some resolution of the incompleteness in the agent's
values. Levi also had views about what further constraints there should
be on choice, so he did not defen the view I've been discussing where
any option that is maximal on any resolution is choiceworthy. But his
views are still relevant here, because this requirement that a choice be
maximal on a resolution meant that he was committed to γ failing, and
choices not being binary.

A central example which he uses, and which Sen
(\citeproc{ref-Sen2004}{2004}) picks up on, involves an executive
looking to hire a secretary. I'll follow Sen and call the executive
Ms.~Jones.\footnote{I've also changed the secretaries' names.} She is
looking for a secretary with good typing skills and good stenography
skills. (This is the 1980s.) We'll conceive of these, a bit arbitrarily,
as distinct values. There are three candidates: Jack, Danny, and Luke,
and their value on each measure is given in Table~\ref{tbl-secretaries}.
(I've slightly adjusted the numbers to match the earlier examples.)

\begin{longtable}[]{@{}rcc@{}}
\caption{Three candidates for a secretarial
position}\label{tbl-secretaries}\tabularnewline
\toprule\noalign{}
& Typing Skill & Stenography Skill \\
\midrule\noalign{}
\endfirsthead
\toprule\noalign{}
& Typing Skill & Stenography Skill \\
\midrule\noalign{}
\endhead
\bottomrule\noalign{}
\endlastfoot
Jack & 10 & 0 \\
Danny & 1 & 1 \\
Luke & 0 & 10 \\
\end{longtable}

Levi argues that if the numbers are like this, and Danny is barely
better than the other two on each metric, he is not choiceworthy. That's
true even though he might be choiceworthy if one or the other candidates
was unavailable.

Sen argues that the right choice theory in this case should be
``inarticulate'', and say that any of the three is choiceworthy. He
responds to the intuition Levi presents with a dilemma.

On the first horn, we understand these numbers as representing something
an objective measure of the skills of the candidates at each of the
tasks. As Sen points out, it's easy to imagine situations where someone
who is not abysmal at either half of the job is more useful than someone
who is an expert on one, and abysmal on the other.

On the other horn, we measure the ``importance''
(\citeproc{ref-Sen2004}{Sen 2004, 53}) of each skill for the task at
hand. Sen argues, by analogy with the difficulty in establishing a
social welfare function out of the welfare of each individual, that
there will be no way to do this. Let's turn to how we might go about it.

\section{Lotteries, Choices, and Values}\label{sec-lotteries}

In this section I'll offer a response, on Levi's behalf, to Sen's
dilemma. The example will draw heavily on recent work by Harvey Lederman
(\citeproc{ref-Lederman2025}{forthcoming}).\footnote{See also Lederman
  (\citeproc{ref-Lederman2023}{2023}), and Tarsney, Lederman, and Spears
  (\citeproc{ref-LedermanEtAl2025}{forthcoming}). This is not to say
  Lederman would endorse anything like this response; as we'll see in
  the next section he is sympathetic to Sen's view. But every principle
  I'll use here is discussed, one way or another, in his work, and I've
  drawn heavily on that discussion in what follows.} We'll start by
imagining that Ms.~Jones might have a choice not of secretaries, but of
agencies, and she has a reasonable credal distribution over the skills
of the people a particular agency might send. That is, each choice of
agency will be a choice of a \emph{lottery}, where she doesn't choose a
package of skills, but a probability distribution over some outcomes,
where each outcome is a secretary with a numerical skill on each
attribute.

To set this up, I'll need three new bits of notation. I'll write
\emph{Lxy} for a lottery that has equal chance of returning outcomes
\emph{x} and \emph{y}, where these might be secretaries or further
lotteries. I'll write \emph{x}~=~\emph{y} for what I've previously
written as \emph{xEy}. It means that \emph{x} and \emph{y} are equally
good by Ms.~Jones's lights. It's perhaps suboptimal to introduce new
notation for an old concept, but the mix of L's and E's became hard to
read. Finally, I'll write ⟨\emph{x}, \emph{y}⟩ for a secretary with
skill \emph{x} at typing and skill \emph{y} at stenography.

If the numbers in Table~\ref{tbl-secretaries} measure importance, then
Ms.~Jones's preferences should satisfy (a special case of) what Lederman
calls \textbf{Unidimensional Expectations}.

\begin{description}
\tightlist
\item[Unidimensional Expectations (UE)]
L⟨\emph{x}\textsubscript{1}, \emph{y}⟩⟨\emph{x}\textsubscript{2},
\emph{y}⟩ = ⟨(\emph{x}\textsubscript{1} + x\textsubscript{2})/2,
\emph{y}⟩

L⟨\emph{x}, \emph{y}\textsubscript{1}⟩⟨\emph{x},
\emph{y}\textsubscript{2}⟩ = ⟨\emph{x}, (\emph{y}\textsubscript{1} +
y\textsubscript{2})/2⟩
\end{description}

That is, the value of a lottery where the possible outcomes agree on one
dimension also has the same value on that dimension, and has the
expected value of the other dimension. If the lottery does not involve
value conflict, old fashioned expected value maximisation is the way to
go.

This is enough to rule out the example Sen has in mind on the first
horn, where a secretary with skill 1 on either dimension is more than
half as valuable as a secretary with skill 10. But it's not enough to
say that Ms.~Jones should not hire Danny. Unidimensional Expectations is
consistent with one resolution of his indeterminate value being that the
value of ⟨\emph{x},\emph{y}⟩ is \emph{xy}. If that's a permissible
resolution, then there will be a resolution on which Danny is maximally
valuable. But there are further principles that do rule out Danny. The
most intuitive argument I know uses the following.

\begin{description}
\tightlist
\item[Substitution of Identicals (SI)]
If \emph{x} = \emph{y}, then \emph{Lxz} = \emph{Lyz} and \emph{Lzx} =
\emph{Lzy}.
\item[No Trade-Off (NT)]
L⟨\emph{x}, \emph{x}⟩⟨\emph{y}, \emph{y}⟩ = ⟨(\emph{x} +
\emph{y})/2,(\emph{x} + \emph{y})/2
\item[Rearrangement of Outcomes (RO)]
\emph{L}(\emph{Lxy})(\emph{Lvw}) = \emph{L}(\emph{Lxw})(\emph{Lvy})
\item[Weak Independence (WI)]
If \emph{x} = \emph{Lxy} then \emph{x} = \emph{y}
\end{description}

Substitution of Identicals follows naturally from the idea that the
outcomes are truly equal, so it doesn't matter whether one has a lottery
has an outcome that results in one or the other. No Trade-Off, like
Unidimensional Expectations, says that when we are just considering
strictly better and worse options, so there are no relevant
complications about resolving indeterminacy, we're back in the land of
expected utility maximisation. Rearrangement of Outcomes follows from
the fact that the compound lotteries on either side of the identity sign
each have probability 1/4 of returning one of those four outcomes. And
Weak Independence, which is probably the most contentious of the lot,
says that if \emph{y} is not exactly as good as \emph{x}, then a 1/2
chance of \emph{y} should not be exactly as good as \emph{x}. Given
these principles, we can argue as follows.

\begin{longtable}[]{@{}rll@{}}
\toprule\noalign{}
\endhead
\bottomrule\noalign{}
\endlastfoot
1. & ⟨5, 5⟩ = \emph{L}⟨10, 5⟩⟨0,5⟩ & UD \\
2. & ⟨10, 5⟩ = \emph{L}⟨10, 10⟩⟨10,0⟩ & UD \\
3. & ⟨0, 5⟩ = \emph{L}⟨0, 10⟩⟨0,0⟩ & UD \\
4. & ⟨5, 5⟩ = \emph{L}(\emph{L}⟨10, 10⟩⟨10,0⟩)⟨0,5⟩ & 1,3 SI \\
5. & ⟨5, 5⟩ = \emph{L}(\emph{L}⟨10, 10⟩⟨10,0⟩)(\emph{L}⟨0, 10⟩⟨0,0⟩) &
2, 4 SI \\
6. & ⟨5, 5⟩ = \emph{L}(\emph{L}⟨10, 10⟩⟨0,0⟩)(\emph{L}⟨0, 10⟩⟨10,0⟩) & 5
RO \\
7. & ⟨5, 5⟩ = \emph{L}⟨10, 10⟩⟨0,0⟩ & NT \\
8. & ⟨5, 5⟩ = \emph{L}⟨5, 5⟩(\emph{L}⟨0, 10⟩⟨10,0⟩) & 6, 7 SI \\
9. & ⟨5, 5⟩ = \emph{L}⟨0, 10⟩⟨10,0⟩ & 8 WI \\
\end{longtable}

To complete the argument we just need the plausible principles that (a)
\emph{x} is not choiceworthy from \emph{S} when \emph{x} is not
choiceworthy from the set consisting of \emph{x} and \emph{Lyz}, for
\emph{y}, \emph{z} in \emph{S}, and (b) \emph{x} is not choiceworthy
from \emph{S} when an option that is strictly better on every dimension
is in \emph{S}.

There are several assumptions here, and any one of them could be the
subject of a whole paper. But they each seem very plausible, and they
show how we can ground the intuition that Levi started with in a
sensible theory of preferences over lotteries involving payoffs that
differ in multiple dimensions.

\section{Negative Dominance}\label{sec-negdom}

Harvey Lederman (\citeproc{ref-Lederman2025}{forthcoming}) notes that
the picture I've developed, where Danny is unchoiceworthy, has the
following strange consequence. It violates what he calls
\textbf{Negative Dominance}. This is the version of Negative Dominance
most relevant to the case.\footnote{In all the quotes I'll change the
  names and example to match the one I'm using.}

\begin{description}
\tightlist
\item[Negative Dominance (Goodness)]
If one game of chance is better for {[}Ms.~Jones{]} than another, some
prize in the first game is better for her than some prize in the second.
(\citeproc{ref-Lederman2025}{Lederman forthcoming, 13})
\end{description}

The main application of this is to reject the idea that L(Jake)(Luke) is
better than Danny. We'll come back to the fact that the second option
here is an outcome as well as a gamble.

This seems like a very plausible principle if (and I think only if) one
thinks that the role of decision theory is to come up with coherence
constraints on \emph{preferences}. And that that is the role of decision
theory follows naturally from the idea that all the notions of decision
theory are ultimately grounded in preferences. As Lederman argues,
preferences about lotteries have to be grounded in something about their
prizes, and if preferences are fundamental, presumably they have to be
grounded in preferences over their prizes. But if choiceworthiness is
prior to preference, that last step doesn't follow.

So when Lederman says,

\begin{quote}
a strict preference for one game of chance over another must be
explained by a strict preference for one of the \emph{prizes} of the
first, by comparison to one of the prizes of the second
(\citeproc{ref-Lederman2025}{Lederman forthcoming, 18}, emphasis in
original)
\end{quote}

we should question the uses of `one'. What's true is that attitudes
towards games of chance should be somehow explained by attitudes towards
the prizes, but these attitudes need not be preferences. For instance,
in the secretaries example, the preference for the lottery over Danny is
sufficiently explained by the fact that Danny is not choiceworthy when
all three secretaries are available. More generally, we should accept
this principle:

\begin{description}
\tightlist
\item[Negative Dominance for Choices]
If one game of chance is choiceworthy when an option is available and
not choiceworthy, that option is not choiceworthy when it and all the
prizes of the first lottery are available.
\end{description}

Given α and γ, this will entail Lederman's version of the principle, at
least as restricted to comparisons between lotteries and outcomes. But
without them, it does not.

That said, if we adopt a Levi-style view, we cannot generalise Negative
Dominance for Choices to comparisons between lotteries. Here's an
example, structurally parallel to the main example in Tarsney, Lederman,
and Spears (\citeproc{ref-LedermanEtAl2025}{forthcoming}), which shows
this. Ms.~Jones is now trying to hire a programmer, and she has four
candidates, each of which has the skills in the four languages she cares
about shown in Table~\ref{tbl-programmers}.

\begin{longtable}[]{@{}rcccc@{}}
\caption{Four programmers, and their
skills}\label{tbl-programmers}\tabularnewline
\toprule\noalign{}
& Python & Java & C & Ruby \\
\midrule\noalign{}
\endfirsthead
\toprule\noalign{}
& Python & Java & C & Ruby \\
\midrule\noalign{}
\endhead
\bottomrule\noalign{}
\endlastfoot
Jane & 6 & 6 & 0 & 0 \\
Dolly & 0 & 0 & 6 & 6 \\
Lily & 5 & 0 & 5 & 0 \\
Suzy & 0 & 5 & 0 & 5 \\
\end{longtable}

When the menu consists of any set of the programmers, all the options
are choiceworthy. But Ms.~Jones strictly prefers L(Jane)(Dolly) to
L(Lily)(Suzy), since the former lottery is better in expectation on all
four dimensions. Lederman is right that this needs to be explained, that
it should be explained in terms of evaluative features of the prizes
(i.e., the programmers), and if the explanation uses expected value, we
should explain why expected value matters. No explanation in terms of
the choiceworthiness of some options will work, and hence no explanation
in terms of the choiceworthiness of options from pair sets (i.e.,
preferences) will work.

The fact to be explained is that when L(Jane)(Dolly) and L(Lily)(Suzy),
only the former is choiceworthy. Here's how I explain it:

\begin{enumerate}
\def\labelenumi{\arabic{enumi}.}
\tightlist
\item
  Ms.~Jones has four values, and it is indeterminate how they should be
  balanced. This means both that she hasn't decided how to balance them,
  and maybe it is unnecessary, or even inadvisable, to balance them.
\item
  Given Unidirectional Expectations and No Trade-Offs (as discussed in
  Section~\ref{sec-lotteries}), the permissible only trade-offs are
  linear mixtures of the values.
\item
  Given the result from Pearce (\citeproc{ref-Pearce1984}{1984})
  discussed in Section~\ref{sec-games} (his Lemma 3), a lottery is best
  on no linear resolution of the indecision in point 1 iff some
  available lottery over other choices is better in expectation on every
  value.
\item
  A lottery is choiceworthy from a menu of other lotteries (or options)
  iff it is optimal on some permissible resolution of this indecision.
\end{enumerate}

If L(Lily)(Suzy) were choiceworthy, by 4 it would have to be best on
some resolution of Ms.~Jones's values. By 1 and 2, this means that it is
best on some linear mixture of these values. By 3, that means it is
better in expectation on one of these values. But it is not; on all four
dimensions L(Jane)(Dolly) has expected value 3, and L(Lily)(Suzy) has
expected value 2.5. Even though Ms.~Jones has not resolved the
indeterminacy in her values, the fact that any resolution would mean she
prefers the first lottery is enough reason to prefer the first lottery.

In short, the focus on expected values comes not from any particular
importance on expectations as such, but from the thought that
permissible reactions to indeterminacy in values are constrained by
permissible reactions to resolutions of that indeterminacy, combined
with (a) constraints on resolutions like Unidimensional Expectations and
No Trade-Offs, and (b) Pearce's result linking expected value to linear
mixtures of values.

\section{Conclusion}\label{conclusion}

This paper has been ultimately about the grounding of facts about
rational choice. I've been mostly concerned to argue against a popular,
if largely implicit, view: rational choice is grounded in rational
preference. If Chooser wants a holiday, and is choosing where to go,
which destinations are rationally choiceworthy is grounded in Chooser's
(rational) preferences over pairs of choices. I've rejected this for
three reasons:

\begin{enumerate}
\def\labelenumi{\arabic{enumi}.}
\tightlist
\item
  As argued in Section~\ref{sec-multieq} and Section~\ref{sec-mixed},
  choiceworthiness is an ex ante concept, and preference is an ex post
  concept, and hence choiceworthiness is analytically prior to
  preference, so should not be grounded in preference.
\item
  When there are no Newcomb-like considerations, and so the ex ante/ex
  post distinction doesn't matter, preferences are just choiceworthiness
  judgments with respect to pair sets, and pair sets aren't normatively
  distinctive.
\item
  Any choice function that violates γ cannot be generated from a
  preference relation, and there are many reasons for endorsing choice
  functions which violate γ.
\end{enumerate}

If choiceworthiness is not grounded in preferences, what is it grounded
in? Here there are a few options.

We could say that there are a special class of options, call them
outcomes, and choiceworthiness in general is grounded in
choiceworthiness with respect to sets of these outcomes, or perhaps in
sets of outcomes and simple bets involving outcomes. This would mirror
the approach to subjective decision theory that grounds everything in
special classes of preferences, i.e., preferences over outcomes and
simple bets.

I'd prefer to ground choiceworthiness in values, plurally, and the
subject's identification with some of these values. What makes joining
the resistance unchoiceworthy for Pierre is that he has two values, and
there is no way of balancing these values which makes joining the
resistance optimal. But the other two options are both choiceworthy
because they are both valuable, and Pierre could decide that one of the
values they represent is one of his value, or even that it is something
good to do while working out what his values really are.

That's to say, once we drop the idea that binary comparisons, i.e.,
preferences, have a special role in grounding rational choice, it could
make sense to focus on more general comparisons, e.g., choiceworthiness
from sets of options, or from properties of individual options, i.e.,
values. Whichever way we go, it will be easier to defend the intuitive
idea that hard choices like Pierre's violate β than if we start try to
ground choiceworthiness in preference.

\subsection*{References}\label{references}
\addcontentsline{toc}{subsection}{References}

\phantomsection\label{refs}
\begin{CSLReferences}{1}{0}
\bibitem[\citeproctext]{ref-Aizerman1981}
Aizerman, M., and A. Malishevski. 1981. {``General Theory of Best
Variants Choice: Some Aspects.''} \emph{IEEE Transactions on Automatic
Control} 26 (5): 1030--40. doi:
\href{https://doi.org/10.1109/TAC.1981.1102777}{10.1109/TAC.1981.1102777}.

\bibitem[\citeproctext]{ref-Armstrong1939}
Armstrong, W. E. 1939. {``The Determinateness of the Utility
Function.''} \emph{The Economic Journal} 49 (195): 453--67. doi:
\href{https://doi.org/10.2307/2224802}{10.2307/2224802}.

\bibitem[\citeproctext]{ref-Armstrong1948}
---------. 1948. {``Uncertainty and the Utility Function.''} \emph{The
Economic Journal} 58 (229): 1--10. doi:
\href{https://doi.org/10.2307/2226342}{10.2307/2226342}.

\bibitem[\citeproctext]{ref-Armstrong1950}
---------. 1950. {``A Note on the Theory of Consumer's Behaviour.''}
\emph{Oxford Economic Papers} 2 (1): 119--22. doi:
\href{https://doi.org/10.1093/oxfordjournals.oep.a041384}{10.1093/oxfordjournals.oep.a041384}.

\bibitem[\citeproctext]{ref-Arrow1951}
Arrow, Kenneth J. 1951. \emph{Social Choice and Individual Values}. New
York: John Wiley \& Sons.

\bibitem[\citeproctext]{ref-sep-fechner}
Beiser, Frederick C. 2024. {``{Gustav Theodor Fechner}.''} In \emph{The
{Stanford} Encyclopedia of Philosophy}, edited by Edward N. Zalta and
Uri Nodelman, {S}ummer 2024.
\url{https://plato.stanford.edu/archives/sum2024/entries/fechner/};
Metaphysics Research Lab, Stanford University.

\bibitem[\citeproctext]{ref-Bernheim1984}
Bernheim, B. Douglas. 1984. {``Rationalizable Strategic Behavior.''}
\emph{Econometrica} 52 (4): 1007--28. doi:
\href{https://doi.org/10.2307/1911196}{10.2307/1911196}.

\bibitem[\citeproctext]{ref-Blair1976}
Blair, Douglas H, George Bordes, Jerry S Kelly, and Kotaro Suzumura.
1976. {``Impossibility Theorems Without Collective Rationality.''}
\emph{Journal of Economic Theory} 11 (3): 361--79. doi:
\href{https://doi.org/10.1016/0022-0531(76)90047-8}{10.1016/0022-0531(76)90047-8}.

\bibitem[\citeproctext]{ref-Broome1997}
Broome, John. 1997. {``Is Incommensurability Vagueness?''} In
\emph{Incommensurability, Comparability and Pracatical Reason}, edited
by Ruth Chang, 67--89. Cambridge, MA: Harvard University Press.

\bibitem[\citeproctext]{ref-Carr2020}
Carr, Jennifer Rose. 2020. {``Imprecise Evidence Without Imprecise
Credences.''} \emph{Philosophical Studies} 177 (9): 2735--58. doi:
\href{https://doi.org/10.1007/s11098-019-01336-7}{10.1007/s11098-019-01336-7}.

\bibitem[\citeproctext]{ref-Chang1997}
Chang, Ruth. 1997. {``Introduction.''} In \emph{Incommensurability,
Incomparability and Practical Reason.}, edited by Ruth Chang, 1--34.
Cambridge, MA: Harvard University Press.

\bibitem[\citeproctext]{ref-Chang2017}
---------. 2017. {``Hard Choices.''} \emph{Journal of the American
Philosophical Association} 3 (1): 1--21. doi:
\href{https://doi.org/10.1017/apa.2017.7}{10.1017/apa.2017.7}.

\bibitem[\citeproctext]{ref-Chang2024}
---------. 2024. {``What's so Hard about Hard Choices?''} \emph{Erasmus
Journal for Philosophy and Economics} 17 (1): 272--86. doi:
\href{https://doi.org/10.23941/ejpe.v17i1.872}{10.23941/ejpe.v17i1.872}.

\bibitem[\citeproctext]{ref-Chernoff1954}
Chernoff, Herman. 1954. {``Rational Selection of Decision Functions.''}
\emph{Econometrica} 22 (4): 422--43. doi:
\href{https://doi.org/10.2307/1907435}{10.2307/1907435}.

\bibitem[\citeproctext]{ref-Debreu1960}
Debreu, Gerard. 1960. {``Review of \emph{Individual Choice Behavior: A
Theoretical Analysis}, by {R. Duncan Luce}.''} \emph{American Economic
Review} 50 (1): 186--88.

\bibitem[\citeproctext]{ref-DorrEtAl2023}
Dorr, Cian, Jacob M. Nebel, and Jake Zuehl. 2023. {``The Case for
Comparability.''} \emph{Noûs} 57 (2): 414--53. doi:
\href{https://doi.org/10.1111/nous.12407}{10.1111/nous.12407}.

\bibitem[\citeproctext]{ref-Fara2001}
Fara, Delia Graff. 2001. {``Phenomenal Continua and the Sorites.''}
\emph{Mind} 110 (440): 905--36. doi:
\href{https://doi.org/10.1093/mind/110.440.905}{10.1093/mind/110.440.905}.
This paper was first published under the name {``Delia Graff.''}

\bibitem[\citeproctext]{ref-Fechner1860}
Fechner, Gustav. 1860. \emph{Elemente Der Psychophysik}. Leipzig:
Breitkopf und H{ä}rtel.

\bibitem[\citeproctext]{ref-Gallow2020}
Gallow, J. Dmitri. 2020. {``The Causal Decision Theorist's Guide to
Managing the News.''} \emph{The Journal of Philosophy} 117 (3): 117--49.
doi:
\href{https://doi.org/10.5840/jphil202011739}{10.5840/jphil202011739}.

\bibitem[\citeproctext]{ref-Gallow2024}
---------. 2024. {``It Can Be Irrational to Knowingly Choose the
Best.''} \emph{Australasian Journal of Philosophy} 103 (2): 540--46.
doi:
\href{https://doi.org/10.1080/00048402.2024.2310197}{10.1080/00048402.2024.2310197}.

\bibitem[\citeproctext]{ref-Gibbard2014}
Gibbard, Allan F. 2014. {``Social Choice and the Arrow Conditions.''}
\emph{Economics and Philosophy} 30 (3): 269--84. doi:
\href{https://doi.org/10.1017/S026626711400025X}{10.1017/S026626711400025X}.

\bibitem[\citeproctext]{ref-GoodmanLederemanArXiV}
Goodman, Jeremy, and Harvey Lederman. 2024. {``Maximal Social Welfare
Relations on Infinite Populations Satisfying Permutation Invariance.''}
\url{https://arxiv.org/abs/arXiv:2408.05851}. arXiv preprint.

\bibitem[\citeproctext]{ref-GulPesendorfer2008}
Gul, Faruk, and Wolfgang Pesendorfer. 2008. {``The Case for Mindless
Economics.''} In \emph{Foundations of Positive and Normative Economics},
edited by Andrew Caplin and Andrew Schotter, 2--40. Oxford: Oxford
University Press. doi:
\href{https://doi.org/10.1093/acprof:oso/9780195328318.003.0001}{10.1093/acprof:oso/9780195328318.003.0001}.

\bibitem[\citeproctext]{ref-Hansson2009}
Hansson, Sven Ove. 2009. {``Preference-Based Choice Functions: A
Generalized Approach.''} \emph{Synthese} 171 (2): 257--69. doi:
\href{https://doi.org/10.1007/s11229-009-9650-5}{10.1007/s11229-009-9650-5}.

\bibitem[\citeproctext]{ref-sep-preferences}
Hansson, Sven Ove, and Till Grüne-Yanoff. 2024. {``{Preferences}.''} In
\emph{The {Stanford} Encyclopedia of Philosophy}, edited by Edward N.
Zalta and Uri Nodelman, {W}inter 2024.
\url{https://plato.stanford.edu/archives/win2024/entries/preferences/};
Metaphysics Research Lab, Stanford University.

\bibitem[\citeproctext]{ref-Holton1999}
Holton, Richard. 1999. {``Intention and Weakness of Will.''} \emph{The
Journal of Philosophy} 96 (5): 241--62. doi:
\href{https://doi.org/10.2307/2564667}{10.2307/2564667}.

\bibitem[\citeproctext]{ref-Lederman2025}
Lederman, Harvey. forthcoming. {``Of Marbles and Matchsticks.''}
\emph{Oxford Studies in Epistemology}, forthcoming. Online at
\url{https://philpapers.org/rec/LEDOMA-2}; references to online version.

\bibitem[\citeproctext]{ref-Lederman2023}
---------. 2023. {``Incompleteness, Independence, and Negative
Dominance.''} Online at \url{https://philpapers.org/archive/LEDIIA.pdf}.

\bibitem[\citeproctext]{ref-LehrerWagner1985}
Lehrer, Keith, and Carl Wagner. 1985. {``Intransitive Indifference: The
Semi-Order Problem.''} \emph{Synthese} 65: 249--56. doi:
\href{https://doi.org/10.1007/BF00869302}{10.1007/BF00869302}.

\bibitem[\citeproctext]{ref-Levi1986}
Levi, Isaac. 1986. \emph{Hard Choices}. Cambridge: Cambridge University
Press.

\bibitem[\citeproctext]{ref-Luce1956}
Luce, R. Duncan. 1956. {``Semiorders and a Theory of Utility
Discrimination.''} \emph{Econometrica} 24 (2): 178--91. doi:
\href{https://doi.org/10.2307/1905751}{10.2307/1905751}.

\bibitem[\citeproctext]{ref-Luce1959}
---------. 1959. \emph{Individual Choice Behavior: A Theoretical
Analysis}. New York: Wiley.

\bibitem[\citeproctext]{ref-LuceRaiffa1957}
Luce, R. Duncan, and Howard Raiffa. 1957. \emph{Games and Decisions:
Introduction and Critical Survey}. New York: Wiley.

\bibitem[\citeproctext]{ref-Moss2015}
Moss, Sarah. 2015. {``Time-Slice Epistemology and Action Under
Indeterminacy.''} \emph{Oxford Studies in Epistemology} 5: 172--94. doi:
\href{https://doi.org/10.1093/acprof:oso/9780198722762.003.0006}{10.1093/acprof:oso/9780198722762.003.0006}.

\bibitem[\citeproctext]{ref-Moulin1985}
Moulin, Hervé. 1985. {``Choice Functions over a Finite Set: A
Summary.''} \emph{Social Choice and Welfare} 2 (2): 147--60.

\bibitem[\citeproctext]{ref-vNM1944}
Neumann, John von, and Oskar Morgenstern. 1944. \emph{Theory of Games
and Economic Behavior}. Princeton, NJ: Princeton University Press.

\bibitem[\citeproctext]{ref-Nover2004}
Nover, Harris, and Alan Hàjek. 2004. {``Vexing Expectations.''}
\emph{Mind} 113 (450): 237--49. doi:
\href{https://doi.org/10.1093/mind/113.450.237}{10.1093/mind/113.450.237}.

\bibitem[\citeproctext]{ref-Pearce1984}
Pearce, David G. 1984. {``Rationalizable Strategic Behavior and the
Problem of Perfection.''} \emph{Econometrica} 52 (4): 1029--50. doi:
\href{https://doi.org/10.2307/1911197}{10.2307/1911197}.

\bibitem[\citeproctext]{ref-Peterson2017}
Peterson, Martin. 2017. \emph{An Introduction to Decision Theory}.
Second. Cambridge: Cambridge University Press.

\bibitem[\citeproctext]{ref-RamseyTruthProb}
Ramsey, Frank. 1926. {``Truth and Probability.''} In \emph{Philosophical
Papers}, edited by D. H. Mellor, 52--94. Cambridge: Cambridge University
Press.

\bibitem[\citeproctext]{ref-Samuelson1938}
Samuelson, Paul A. 1938. {``A Note on the Pure Theory of Consumer's
Behaviour.''} \emph{Econometrica} 5 (17): 61--71. doi:
\href{https://doi.org/10.2307/2548836}{10.2307/2548836}.

\bibitem[\citeproctext]{ref-Sartre1946}
Sartre, Jean-Paul. 1946/2007. {``Existentialism Is a Humanism.''} In
\emph{Existentialism Is a Humanism}, translated by Annie Cohen-Solal,
17--72. New Haven: Yale University Press.

\bibitem[\citeproctext]{ref-Sen1969}
Sen, Amartya. 1969. {``Quasi-Transitivity, Rational Choice and
Collective Decisions.''} \emph{The Review of Economic Studies} 36 (3):
381--93. doi: \href{https://doi.org/10.2307/2296434}{10.2307/2296434}.

\bibitem[\citeproctext]{ref-Sen2004}
---------. 2004. {``Incompleteness and Reasoned Choice.''}
\emph{Synthese} 140 (May): 43--59. doi:
\href{https://doi.org/10.1023/B:SYNT.0000029940.51537.b3}{10.1023/B:SYNT.0000029940.51537.b3}.

\bibitem[\citeproctext]{ref-Sen1970sec}
---------. (1970) 2017. \emph{Collective Choice and Social Welfare:} An
expanded edition. Cambridge, MA: Harvard University Press.

\bibitem[\citeproctext]{ref-Skyrms1982}
Skyrms, Brian. 1982. {``Causal Decision Theory.''} \emph{Journal of
Philosophy} 79 (11): 695--711.

\bibitem[\citeproctext]{ref-Spencer2023}
Spencer, Jack. 2023. {``Can It Be Irrational to Knowingly Choose the
Best?''} \emph{Australasian Journal of Philosophy} 101 (1): 128--39.
doi:
\href{https://doi.org/10.1080/00048402.2021.1958880}{10.1080/00048402.2021.1958880}.

\bibitem[\citeproctext]{ref-LedermanEtAl2025}
Tarsney, Christian, Harvey Lederman, and Dean Spears. forthcoming. {``A
Dominance Argument Against Incompleteness.''} \emph{Philosophical
Review}, forthcoming. Online at
\url{https://philpapers.org/archive/TARSTS.pdf}.

\bibitem[\citeproctext]{ref-WeathersonKAHIS}
Weatherson, Brian. 2024. \emph{Knowledge: A Human Interest Story}.
Cambridge: Open Book Publishers.

\end{CSLReferences}



\noindent Draft of June 2025


\end{document}
