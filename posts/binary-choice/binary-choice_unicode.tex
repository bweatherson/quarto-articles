% Options for packages loaded elsewhere
% Options for packages loaded elsewhere
\PassOptionsToPackage{unicode}{hyperref}
\PassOptionsToPackage{hyphens}{url}
%
\documentclass[
  11pt,
  letterpaper,
  DIV=11,
  numbers=noendperiod,
  twoside]{scrartcl}
\usepackage{xcolor}
\usepackage[left=1.1in, right=1in, top=0.8in, bottom=0.8in,
paperheight=9.5in, paperwidth=7in, includemp=TRUE, marginparwidth=0in,
marginparsep=0in]{geometry}
\usepackage{amsmath,amssymb}
\setcounter{secnumdepth}{3}
\usepackage{iftex}
\ifPDFTeX
  \usepackage[T1]{fontenc}
  \usepackage[utf8]{inputenc}
  \usepackage{textcomp} % provide euro and other symbols
\else % if luatex or xetex
  \usepackage{unicode-math} % this also loads fontspec
  \defaultfontfeatures{Scale=MatchLowercase}
  \defaultfontfeatures[\rmfamily]{Ligatures=TeX,Scale=1}
\fi
\usepackage{lmodern}
\ifPDFTeX\else
  % xetex/luatex font selection
  \setmainfont[ItalicFont=EB Garamond Italic,BoldFont=EB Garamond
Bold]{EB Garamond Math}
  \setsansfont[]{EB Garamond}
  \setmathfont[]{Garamond-Math}
\fi
% Use upquote if available, for straight quotes in verbatim environments
\IfFileExists{upquote.sty}{\usepackage{upquote}}{}
\IfFileExists{microtype.sty}{% use microtype if available
  \usepackage[]{microtype}
  \UseMicrotypeSet[protrusion]{basicmath} % disable protrusion for tt fonts
}{}
\usepackage{setspace}
% Make \paragraph and \subparagraph free-standing
\makeatletter
\ifx\paragraph\undefined\else
  \let\oldparagraph\paragraph
  \renewcommand{\paragraph}{
    \@ifstar
      \xxxParagraphStar
      \xxxParagraphNoStar
  }
  \newcommand{\xxxParagraphStar}[1]{\oldparagraph*{#1}\mbox{}}
  \newcommand{\xxxParagraphNoStar}[1]{\oldparagraph{#1}\mbox{}}
\fi
\ifx\subparagraph\undefined\else
  \let\oldsubparagraph\subparagraph
  \renewcommand{\subparagraph}{
    \@ifstar
      \xxxSubParagraphStar
      \xxxSubParagraphNoStar
  }
  \newcommand{\xxxSubParagraphStar}[1]{\oldsubparagraph*{#1}\mbox{}}
  \newcommand{\xxxSubParagraphNoStar}[1]{\oldsubparagraph{#1}\mbox{}}
\fi
\makeatother


\usepackage{longtable,booktabs,array}
\usepackage{calc} % for calculating minipage widths
% Correct order of tables after \paragraph or \subparagraph
\usepackage{etoolbox}
\makeatletter
\patchcmd\longtable{\par}{\if@noskipsec\mbox{}\fi\par}{}{}
\makeatother
% Allow footnotes in longtable head/foot
\IfFileExists{footnotehyper.sty}{\usepackage{footnotehyper}}{\usepackage{footnote}}
\makesavenoteenv{longtable}
\usepackage{graphicx}
\makeatletter
\newsavebox\pandoc@box
\newcommand*\pandocbounded[1]{% scales image to fit in text height/width
  \sbox\pandoc@box{#1}%
  \Gscale@div\@tempa{\textheight}{\dimexpr\ht\pandoc@box+\dp\pandoc@box\relax}%
  \Gscale@div\@tempb{\linewidth}{\wd\pandoc@box}%
  \ifdim\@tempb\p@<\@tempa\p@\let\@tempa\@tempb\fi% select the smaller of both
  \ifdim\@tempa\p@<\p@\scalebox{\@tempa}{\usebox\pandoc@box}%
  \else\usebox{\pandoc@box}%
  \fi%
}
% Set default figure placement to htbp
\def\fps@figure{htbp}
\makeatother


% definitions for citeproc citations
\NewDocumentCommand\citeproctext{}{}
\NewDocumentCommand\citeproc{mm}{%
  \begingroup\def\citeproctext{#2}\cite{#1}\endgroup}
\makeatletter
 % allow citations to break across lines
 \let\@cite@ofmt\@firstofone
 % avoid brackets around text for \cite:
 \def\@biblabel#1{}
 \def\@cite#1#2{{#1\if@tempswa , #2\fi}}
\makeatother
\newlength{\cslhangindent}
\setlength{\cslhangindent}{1.5em}
\newlength{\csllabelwidth}
\setlength{\csllabelwidth}{3em}
\newenvironment{CSLReferences}[2] % #1 hanging-indent, #2 entry-spacing
 {\begin{list}{}{%
  \setlength{\itemindent}{0pt}
  \setlength{\leftmargin}{0pt}
  \setlength{\parsep}{0pt}
  % turn on hanging indent if param 1 is 1
  \ifodd #1
   \setlength{\leftmargin}{\cslhangindent}
   \setlength{\itemindent}{-1\cslhangindent}
  \fi
  % set entry spacing
  \setlength{\itemsep}{#2\baselineskip}}}
 {\end{list}}
\usepackage{calc}
\newcommand{\CSLBlock}[1]{\hfill\break\parbox[t]{\linewidth}{\strut\ignorespaces#1\strut}}
\newcommand{\CSLLeftMargin}[1]{\parbox[t]{\csllabelwidth}{\strut#1\strut}}
\newcommand{\CSLRightInline}[1]{\parbox[t]{\linewidth - \csllabelwidth}{\strut#1\strut}}
\newcommand{\CSLIndent}[1]{\hspace{\cslhangindent}#1}



\setlength{\emergencystretch}{3em} % prevent overfull lines

\providecommand{\tightlist}{%
  \setlength{\itemsep}{0pt}\setlength{\parskip}{0pt}}



 


\setlength\heavyrulewidth{0ex}
\setlength\lightrulewidth{0ex}
\usepackage[automark]{scrlayer-scrpage}
\clearpairofpagestyles
\cehead{
  Brian Weatherson
  }
\cohead{
  Is Choice Binary?
  }
\ohead{\bfseries \pagemark}
\cfoot{}
\makeatletter
\newcommand*\NoIndentAfterEnv[1]{%
  \AfterEndEnvironment{#1}{\par\@afterindentfalse\@afterheading}}
\makeatother
\NoIndentAfterEnv{itemize}
\NoIndentAfterEnv{enumerate}
\NoIndentAfterEnv{description}
\NoIndentAfterEnv{quote}
\NoIndentAfterEnv{equation}
\NoIndentAfterEnv{longtable}
\NoIndentAfterEnv{abstract}
\renewenvironment{abstract}
 {\vspace{-1.25cm}
 \quotation\small\noindent\emph{Abstract}:}
 {\endquotation}
\newfontfamily\tfont{EB Garamond}
\addtokomafont{disposition}{\rmfamily}
\addtokomafont{title}{\normalfont\itshape}
\let\footnoterule\relax
\usepackage{amsfonts}
\KOMAoption{captions}{tableheading}
\makeatletter
\@ifpackageloaded{caption}{}{\usepackage{caption}}
\AtBeginDocument{%
\ifdefined\contentsname
  \renewcommand*\contentsname{Table of contents}
\else
  \newcommand\contentsname{Table of contents}
\fi
\ifdefined\listfigurename
  \renewcommand*\listfigurename{List of Figures}
\else
  \newcommand\listfigurename{List of Figures}
\fi
\ifdefined\listtablename
  \renewcommand*\listtablename{List of Tables}
\else
  \newcommand\listtablename{List of Tables}
\fi
\ifdefined\figurename
  \renewcommand*\figurename{Figure}
\else
  \newcommand\figurename{Figure}
\fi
\ifdefined\tablename
  \renewcommand*\tablename{Table}
\else
  \newcommand\tablename{Table}
\fi
}
\@ifpackageloaded{float}{}{\usepackage{float}}
\floatstyle{ruled}
\@ifundefined{c@chapter}{\newfloat{codelisting}{h}{lop}}{\newfloat{codelisting}{h}{lop}[chapter]}
\floatname{codelisting}{Listing}
\newcommand*\listoflistings{\listof{codelisting}{List of Listings}}
\makeatother
\makeatletter
\makeatother
\makeatletter
\@ifpackageloaded{caption}{}{\usepackage{caption}}
\@ifpackageloaded{subcaption}{}{\usepackage{subcaption}}
\makeatother
\usepackage{bookmark}
\IfFileExists{xurl.sty}{\usepackage{xurl}}{} % add URL line breaks if available
\urlstyle{same}
\hypersetup{
  pdftitle={Is Choice Binary?},
  pdfauthor={Brian Weatherson},
  hidelinks,
  pdfcreator={LaTeX via pandoc}}


\title{Is Choice Binary?}
\author{Brian Weatherson}
\date{2025}
\begin{document}
\maketitle
\begin{abstract}
There is a natural view about how preference relates to choice. On this
view, an option is choiceworthy iff no alternative is strictly preferred
to it. I'm going to argue against this, for two reasons. First, the view
makes a false prediction about which options are choiceworthy in games
and in choices between options that differ along multiple dimensions.
Second, choice is fundamentally about how options are evaluated prior to
a decision, while preference is about how they are (or will be)
evaluated post decision. Several consequences of rejecting this natural
view are explored, including how it simplifies the relationship between
game theory and decision theory, and how it complicates several debates
about so-called `incomplete' preferences.
\end{abstract}


\setstretch{1.1}
Decision theory is, as the name suggests, about how decisions should be
made. In real life situations, a chooser has many options available. But
the usual way decision theory is formulated gives special place to the
chooser's thoughts either about individual options, or about pairs of
options. Typically, it either says here's a value function, which maps
options to values, and chooser should choose the one with highest value,
or it says here's a preference relation, and chosoer should choose one
to which no other option is preferred. In this paper, I'm going to argue
both these approaches are mistaken. Decision theory has to pay attention
to the distinctive features of choices involving more than two options.
This is both interesting in its own right, and has implications for
theories of so-called `incomplete' preferences.

\section{Historical Background}\label{sec-history}

There is a long tradition in both the economic literature (e.g.,
Samuelson (\citeproc{ref-Samuelson1938}{1938})) and the psychological
literature (e.g., Luce (\citeproc{ref-Luce1956}{1956})), of taking
choice dispositions to be the central focus of study, as opposed to
values or preferences. Some of the reasons for this were based on
behaviourist or positivist commitments. The theorists would have
endorsed this recent statement of the irrelevance of mental states to
economics.

\begin{quote}
Standard economics does not address mental processes and, as a result,
economic abstractions are typically not appropriate for describing them.
(\citeproc{ref-GulPesendorfer2008}{Gul and Pesendorfer 2008, 24})
\end{quote}

That's not going to be my approach here. I'm going to start not with
observable choice dispositions, like Samuelson, or choice frequencies,
like Luce, but with chooser's beliefs about which options are
choiceworthy. In familiar terminology\footnote{See, for instance,
  (\citeproc{ref-SEPPreferences}{\textbf{SEPPreferences?}}), where I
  learned about the Gul and Pesendorfer quote}, I'm taking a mentalist
approach not a behaviourist approach. Much of the formal work on choice
theory has been done by theorists from the behaviourist side, and I'll
be inevitably drawing from them. But the most important source I'll be
using is someone much more sympathetic to mentalism: Amartya Sen.~In
particular, I'll draw heavily on his ``Collective Choice and Social
Welfare'' (\citeproc{ref-Sen1970sec}{Sen {[}1970{]} 2017}). So I need to
start with the way he talks about preferences in that book, which is
slightly different to the current philosophical orthodoxy.

Sen starts with a binary relation \emph{R}, defined over options, which
is glossed as \emph{xRy} means that \emph{x} is at least as good as
\emph{y}. This is potentially misleading. It does not mean that \emph{x}
is either better than \emph{y}, or exactly as good as \emph{y}. Rather,
it means that \emph{x} is no worse than \emph{y}. Sen introduces two
more binary relations, \emph{xPy}, meaning \emph{x} is preferred to
\emph{y}, and \emph{xIy}, meaning the chooser is between \emph{x} and
\emph{y}. These can both be defined in terms of \emph{R}, as in (1) and
(2). (Throughout, I'm leaving off wide scope universal quantifiers over
free variables.)

\begin{enumerate}
\def\labelenumi{(\arabic{enumi})}
\tightlist
\item
  \emph{xPy}~↔︎ (\emph{xRy} ∧ ¬\emph{yRx})
\item
  \emph{xIy}~↔︎ (\emph{xRy} ∧ \emph{yRx})
\end{enumerate}

A more common way of doing things in contemporary philosophy is to start
with \emph{P} and a fourth relation \emph{E}, where \emph{xEy} means
that \emph{x} and \emph{y} are equally good. On this picture, both (1)
and (2), but the explanatory direction in (1) is right-to-left. So
\emph{xRy} just is ¬\emph{xPy}, and then \emph{xIy} is still defined via
(2). On the version Sen uses, it's a little trickier to define \emph{E},
but (3) looks like a plausible conjecture.

\begin{enumerate}
\def\labelenumi{(\arabic{enumi})}
\setcounter{enumi}{2}
\tightlist
\item
  \emph{xEy} ↔︎ {[}(\emph{xRz} ↔︎ \emph{yRz}) ∧ (\emph{zRx} ↔︎
  \emph{zRy}){]}
\end{enumerate}

That is, two options are equally good iff they are substitutable in
other preference relations. Given all these results, we can show that
the following claims are all tightly connected.

\begin{enumerate}
\def\labelenumi{(\arabic{enumi})}
\setcounter{enumi}{3}
\item
  \emph{xPy} ∨ \emph{xEy} ∨ \emph{yPx}
\item
  (\emph{xPy} ∧ \emph{yIz}) → \emph{xPz}
\item
  (\emph{xIy} ∧ \emph{yIz}) → \emph{xIz}
\item
  is what Chang (\citeproc{ref-Chang2017}{2017}) calls the trichotomy
  thesis. (5) is what Sen calls PI-transitivity, and (6) is what he
  calls II-transitivity.
\end{enumerate}

Sen makes very few assumptions about \emph{R}, but it will simplify our
discussion to start introducing some assumptions here.\footnote{In part
  this is because he was interested in exploring what assumptions about
  preference are crucial to the impossibility theorem that Arrow
  (\citeproc{ref-Arrow1951}{1951}) derives. He initially noticed that
  without (6), the theorem didn't go through. This turned out to be less
  significant than it seemed, because Allan Gibbard
  (\citeproc{ref-Gibbard2014}{2014}) proved that a very similar theorem
  can be proven even without (6). See Sen (\citeproc{ref-Sen1969}{1969})
  for the original optimism that this might lead to an interesting way
  out of the Arrovian results, and Sen
  (\citeproc{ref-Sen1970sec}{{[}1970{]} 2017}) for a more pessimistic
  assessment in light of Gibbard's result. Sen reports that Gibbard
  originally proved his result in a term paper for a seminar at Harvard
  in 1969 that he co-taught with Arrow and Rawls. Much of what I'm
  saying in this paper can be connected in various ways to the
  literature on Arrow's impossibility theorem, but I won't draw out
  those connections here.} We'll assume that \emph{R} is reflexive,
everything is at least as good as itself, and that \emph{P} is
transitive. (Sen calls this quasi-transitivity.) Sen
(\citeproc{ref-Sen1970sec}{{[}1970{]} 2017, 64}??) proves that if
\emph{P} is transitive, then (5) and (6) are equivalent. It's also easy
to show that given (3) plus these assumptions, (4) and (6) are
equivalent.\footnote{Proof: Assume (4) is false. So the right hand side
  of (3) is false. Without much loss of generality, assume that
  \emph{xRz} ∧ ¬\emph{yRz}; the other cases all go much the same way. So
  all the disjuncts are false. From ¬\emph{xPy} and ¬\emph{yPx} we get
  \emph{yRx} ∧ \emph{xRy}, i.e., \emph{xIy}. And \emph{xRz} implies
  \emph{zIx}. So we have a counterexample to II-transitivity, since
  \emph{zIx} and \emph{xIy}, but since ¬\emph{yRz}, \emph{yIz} is false.
  So if (4) is false, (6) is false. In the other direction, assume we
  have a counterexample to (6), i.e., \emph{xIy} and \emph{yIz} but not
  \emph{xIz}. From \emph{xIy} we immediately get that the two outer
  disjuncts of (4) are false. From \emph{yIz} we get \emph{yRz} and
  \emph{zRy}. So if \emph{xEy}, (3) implies that \emph{xRz} and
  \emph{zRx}, i.e., \emph{xIz}. But we assumed that ¬\emph{xIz}. So all
  three disjuntcs of (4) are false. That is, if (6) fails, so does (4),
  completing the proof that they are equivalent.}

What should we call the principle (4)? Most philosophers call it
\emph{completeness}, and its denial \emph{incompleteness}. In his
economic work, Sen (\citeproc{ref-Sen1970sec}{{[}1970{]} 2017}) uses the
term `completeness' for a different property of preference relations,
namely \emph{xRy} ∨ \emph{yRx}. This is a useful notion to have. If
Chooser has never thought of \emph{x}, there is a good sense in which
\emph{xRy} ∨ \emph{yRx} is false, even though of course ¬\emph{yPx} is
true. Still, using the term for (4) is more common in philosophy. When
writing primarily for philosophers,
(\citeproc{ref-Sen2004}{\textbf{Sen2004?}}) uses `completeness' for (4),
and I'll do the same. I'll call use \emph{definedness} for \emph{xRy} ∨
\emph{yRx}, and unless stated otherwise, will assume it holds.

There isn't as much discussion of (4) as such in the economics
literature, but there is a long tradition of discussing (6), going back
to important works by Henri
(\citeproc{ref-Poincare1903}{\textbf{Poincare1903?}}) and Wallace E.
Armstrong (\citeproc{ref-Armstrong1939}{1939}). In most of those works
it is assumed that \emph{P} is transitive, so (4) and (6) are
equivalent, so this is really discussing the same thing. Still, it makes
the terminology confusing.

When it makes it clearer, I'll use the term Chang
(\citeproc{ref-Chang2017}{2017}) suggests for (4). That is, I'll say
that preference relations for which this holds are \emph{trichotomous}.

\section{Properties of Choice Functions}\label{sec-properties}

In philosophy we're familiar enough with possible properties of
preference relations, e.g., that they are transitive, reflexive,
acyclic, etc, that these terms don't need to be defined. We're mostly
less familiar with properties of choice functions. So in this short
section I'll lay out six properties that will be important in what
follows. The first four are discussed in some detail by Sen
(\citeproc{ref-Sen1970sec}{{[}1970{]} 2017}), and I'll use his
terminology for them. The fifth is due to Aizerman and Malishevski
(\citeproc{ref-Aizerman1981}{1981}), and is usually named after
Aizerman. The sixth is discussed by Blair et al.
(\citeproc{ref-Blair1976}{1976}).

\begin{description}
\tightlist
\item[Property α]
(\emph{x} ∈ \emph{C}(\emph{S}) ∧ \emph{x} ∈ \emph{T} ∧ \emph{T} ⊆
\emph{S}) → \emph{x} ∈ \emph{C}(\emph{T})
\end{description}

That is, if \emph{x} is choiceworthy in a larger set, then it is
choiceworthy in any smaller set it is a member of. This is sometimes
called the \emph{Chernoff condition}, after Herman Chernoff
(\citeproc{ref-Chernoff1954}{1954}), and sometimes called
\emph{contraction consistency}.

\begin{description}
\tightlist
\item[Property β]
(\emph{x} ∈ \emph{C}(\emph{T}) ∧ \emph{y} ∈ \emph{C}(\emph{T}) ∧
\emph{T} ⊆ \emph{S}) → (\emph{x} ∈ \emph{C}(\emph{S}) ↔ \emph{y} ∈
\emph{C}(\emph{S}))
\end{description}

That is, if \emph{x} and \emph{y} are both choiceworthy in a smaller
set, then in any larger set they are either both choiceworthy or neither
is. Intuitively, if \emph{x} is not chosen in the larger set because one
of the new members is better than it, then the new member is also better
than \emph{y}.

\begin{description}
\tightlist
\item[Property γ]
(\emph{x} ∈ \emph{C}(\emph{S}) ∧ \emph{x} ∈ \emph{C}(\emph{T})) →
(\emph{x} ∈ \emph{C}(\emph{S} ∪ \emph{T}))
\end{description}

That is, if \emph{x} is choiceworthy in two sets, it is choiceworthy in
their union. This is sometimes called \emph{expansion}, e.g., in Moulin
(\citeproc{ref-Moulin1985}{1985}).

\begin{description}
\tightlist
\item[Property δ]
(\emph{x} ∈ \emph{C}(\emph{T}) ∧ \emph{y} ∈ \emph{C}(\emph{T}) ∧
\emph{T} ⊆ \emph{S}) → (\{\emph{y}\} ≠ \emph{C}(\emph{S}))
\end{description}

This is a weakening of β. It says that if \emph{x} and \emph{y} are both
choiceworthy in the smaller set, then after options are added, it can't
be that only one of them is the only choiceworthy option remaining. If
\emph{x} is not choiceworthy in the larger set, that's because some
other option, not \emph{y}, is chosen in place of it.

\begin{description}
\tightlist
\item[Property Aiz]
(\emph{C}(\emph{S}) ⊂eq \emph{T} ∧ \emph{T} ⊂eq \emph{S}) →
\emph{C}(\emph{T}) ⊂eq \emph{C}(\emph{S})
\end{description}

That is, if the smaller set contains all of the choiceworthy members of
the larger set, then no option is choiceworthy in the smaller set but
not the larger set. If \emph{x} is a nonchoiceworthy member of \emph{S},
then the only way to make it choiceworthy is by deleting choiceworthy
members of \emph{S}, not nonchoiceworthy ones.

\begin{description}
\tightlist
\item[Path Independence]
\emph{C}(\emph{S} ∪ \emph{T}) = \emph{C}(\emph{C}(\emph{S}) ∪
\emph{C}(\emph{T}))
\end{description}

The same options are choiceworthy from a union of two sets as are
choiceworthy from the union of the choiceworthy members of those sets.
This is a sort of independence of irrelevant alternatives principle; the
availability or otherwise of nonchoiceworthy members of \emph{S} and
\emph{T} doesn't affect what should be chosen from \emph{S} ∪ \emph{T}.

I'll describe the effects of these properties in more detail in
subsequent sections.

\section{Property α}\label{sec-alpha}

This is the most commonly used constraint on choice functions, and it
does seem intuitive. If \emph{x} is choiceworthy from a larger set,
deleting unchosen options shouldn't make it choiceworthy. Sen
(\citeproc{ref-Sen1970sec}{{[}1970{]} 2017}, xx) discusses two possible
counterexamples.

One is where the presence of options on the menu gives Chooser relevant
information. If the only two options are tea with a particular friend or
staying home, Chooser will take tea. But if the option of cocaine with
that friend is added, Chooser will stay home. The natural thing to say
here is that when one gets new information, \emph{C} changes, so there
isn't really a single \emph{C} here which violates α.\footnote{For a
  quick argument for that, if Chooser learns the only options are tea
  and staying home because the friend has just run out of cocaine, they
  might still stay home.}

The more interesting case is where the value Chooser puts on options is
dependent on what options are available. So imagine Chooser prefers more
cake to less, but does not want to take the last slice. If the available
options are zero slices or one slice of cake, Chooser will take zero.
But if two slices of cake is an option, Chooser will take one, again
violating α.

This is a trickier case, and the natural thing to say is that Chooser
doesn't really have the same options in the two cases. Taking the last
slice of cake isn't the same thing as taking one slice when two are
available. But this move has costs. In particular, it makes it hard to
say that \emph{C} should be defined for any set of options. It doesn't
clearly make sense to ask Chooser to pick between \emph{taking one
slice, which is the last}, and \emph{taking three slices when five are
available}.

Still, I'm going to set those issues aside and assume, like most
theorists do, and mostly assume that α is a constraint on coherent
choice functions, and that choice functions are defined over arbitrary
sets of options. In Section~\ref{sec-mixed} I'll come back to this last
assumption, but otherwise I'll assume it is in place.

\section{Assumptions}\label{sec-assumptions}

I've said a few times I'm assuming this or that, so it's a good time to
put in one place the assumptions I'm making. These aren't intended to
stack the deck in my favour; if any of them are false, I think it makes
something like my view (a) more plausible, but (b) harder to state.
Anyway, here's what has been assumed.

\begin{enumerate}
\def\labelenumi{\arabic{enumi}.}
\tightlist
\item
  \emph{P} is transitive, i.e., \emph{xRy} ∧ \emph{yRz} → \emph{xRz}
\item
  \emph{R} is `defined', i.e., \emph{xRy} ∨ \emph{yRx}.
\item
  \emph{R} is reflexive, i.e., \emph{xRx}.
\item
  \emph{C} is non-empty, i.e., \emph{C}(\emph{S}) ≠ ∅.
\item
  \emph{C} is defined everywhere, i.e., there is a universe of options
  \emph{U} all subsets of \emph{U} are in the domain of \emph{C}.
\item
  \emph{C} satisfies α.
\item
  The universe \emph{U} of options, that \emph{S} is a subset of, and
  \emph{x} is drawn from, is finite.
\end{enumerate}

In Section~\ref{sec-alpha} we saw one reason to reject 5, namely that we
might want to individuate options in terms of what else is available.
We'll see another in Section~\ref{sec-mixed} when we get to games with
no pure strategy equilibrium. But unless stated otherwise, I'll be
assuming these six things without comment.

When \emph{R} satisfies 1-3, I'll follow Luce
(\citeproc{ref-Luce1956}{1956}) and call it a \emph{semiorder}. When it
also satisfies trichotomy, i.e., (4), I'll follow standard practice and
call it a \emph{weak order}.

\section{Defining Binaryness}\label{sec-defining}

With these six assumptions on board, it's easy to state what it is for a
choice function to be binary. Without them there would be a lot of
choice points in the definition, but now everything is simple.

First, we'll define an inversion function \emph{B} (for binary) that
maps preference relations to choice functions, and vice-versa. Both of
these are sets of ordered pairs, and we'll define the ordered pairs
directly. I'll assume that there is a universe \emph{U} of options, and
every option and set of options is drawn from it.

If the input to \emph{B} is a preference relation \emph{R}:

\begin{enumerate}
\def\labelenumi{(\arabic{enumi})}
\setcounter{enumi}{6}
\tightlist
\item
  \emph{B}(\emph{R}) = \{⟨\emph{S}, \emph{x}⟩: ∀\emph{y}(\emph{y} ∈
  \emph{S} → \emph{xRy})\}
\end{enumerate}

That is, \emph{B}(\emph{R}) is the choice function which for any set
\emph{S} selects what Sen calls `maximal' members, those members to
which nothing is strictly preferred.\footnote{Hansson
  (\citeproc{ref-Hansson2009}{2009}) calls this the `liberal
  maximisation' rule. He contrasts it with five other rules, which are
  distinct in general but equivalent given \emph{R} is a semiorder.}

If the input to \emph{B} is a choice function \emph{C}:

\begin{enumerate}
\def\labelenumi{(\arabic{enumi})}
\setcounter{enumi}{7}
\tightlist
\item
  \emph{B}(\emph{C}) = \{⟨\emph{x}, \emph{y}⟩: x ∈ \emph{C}(\{\emph{x},
  \emph{y}\})\}
\end{enumerate}

That is, \emph{B}(\emph{R}) is the preference relation which says that
in any choices from pair sets, an element is chosen only if it is at
least as good as the other member. Sen
(\citeproc{ref-Sen1970sec}{{[}1970{]} 2017, 319}??) calls this the `base
relation' as opposed to a more complicated `revealed preference
relation', and notes that the two are equivalent given α. Since we're
assuming α, we'll use the simpler version.

A choice function \emph{C} is \textbf{binary} if (9) holds\footnote{Sen
  calls these functions `basic binary', but the distinction he's drawing
  doesn't make a difference given \emph{R} is a semi-order and α.}

\begin{enumerate}
\def\labelenumi{(\arabic{enumi})}
\setcounter{enumi}{8}
\tightlist
\item
  \emph{C} = \emph{B}(\emph{B}(\emph{C}))
\end{enumerate}

That is, if you convert \emph{C} into a preference relation, and back
into a choice function, you get the same thing back.

The core claim of this paper is that there are coherent choice functions
which are not binary. A related claim is that a plausible pair of
coherence constraints that you can state using \emph{B} do not in fact
hold. The constraints are that if \emph{C} and \emph{R} are an agent's
choice function and preference relation, then \emph{C} =
\emph{B}(\emph{R}), and \emph{R} = \emph{B}(\emph{C}). We'll start by
looking at a debate which has presupposed these coherence constraints
should be applied.

\section{Property β}\label{property-ux3b2}

If we start with choice functions, the definition of \emph{E} in (3) is
too simple. A better definition is in (10).

\begin{enumerate}
\def\labelenumi{(\arabic{enumi})}
\setcounter{enumi}{9}
\tightlist
\item
  \emph{xEy} ↔ ∀\emph{S}(\{\emph{x}, \emph{y}\} ⊂eq \emph{S} → (\emph{x}
  ∈ \emph{C}(\emph{S}) ↔ \emph{y} ∈ \emph{C}(\emph{S}))
\end{enumerate}

That is, \emph{x} and \emph{y} are equal iff one is never chosen when
the other is not.\footnote{Without α, this is too weak, since it doesn't
  entail that \emph{x} and \emph{y} are intersubstitutable in general.
  But we won't worry about that.} Given this notion of equality, there
is an intuitive gloss on β: Two options are both choiceworthy iff they
are equal.\footnote{This gloss also assumes α}

To see this, think about choice functions that are defined by starting
with numerical value functions, e.g., expected utility, and saying that
the choiceworthy options are those with maximal value. If \emph{x} and
\emph{y} are both choiceworthy in any set, they must have the same
value. That means in any set where either is choiceworthy, i.e., either
has maximal value, they both have maximal value, so both are
choiceworthy.

More generally, given the assumptions we're making, \emph{C} satisfies β
iff \emph{B}(\emph{C}) is trichotomous, which is equivalent to
\emph{B}(\emph{C}) satisfying II-transitivity. Unsurprisingly, the two
historically significant cases of intuitive counterexamples to
II-transitivity also generate intuitive counterexamples to β.

The first example, tracing back at least to
(\citeproc{ref-Poincare1903}{\textbf{Poincare1903?}}), involves distinct
but indistinguishable options. Assume that Chooser prefers more sugar in
their coffee to less, but can only tell two options apart if they differ
by 10 grains of sugar or more. Now consider these three options:

\begin{quote}
\emph{x} = Coffee with 100 grains of sugar.\\
\emph{y} = Coffee with 106 grains of sugar.\\
\emph{z} = Coffee with 112 grains of sugar.
\end{quote}

This is said to be a counterexample to II-transitivity because Chooser
is indifferent between \emph{x} and \emph{y}, and between \emph{y} and
\emph{z}, but strictly prefers \emph{z} to \emph{x}. It's also a
counterexample to β. Chooser would choose either from \emph{x} and
\emph{y}, but when \emph{z} is added, \emph{y} is still choiceworthy but
\emph{z} is not.

This example was historically important, but it's not discussed that
much in the contemporary philosophical literature. It could be because
philosophers were convinced by the argument in Fara
(\citeproc{ref-Fara2001}{2001}, check this) that phenomenal
indistinguishability is in fact transitive. But it did get widely
discussed in economics. Poincaré's thought gets picked up in a series of
papers by Wallace Armstrong (\citeproc{ref-Armstrong1939}{1939},
\citeproc{ref-Armstrong1948}{1948}, \citeproc{ref-Armstrong1950}{1950}),
who used it as the basis of an argument against the orthodox choice
theory of the day. Armstrong in turn influenced R. Duncan Luce
(\citeproc{ref-Luce1956}{1956}, \citeproc{ref-Luce1959}{1959}), who used
these examples to argue that preferences form a semi-order. In a review
of Luce's book, Gerard Debreu (\citeproc{ref-Debreu1960}{1960}) argued
that the model Luce developed for these cases wouldn't work for cases
where the options differed along different dimensions.

The class of cases Debreu discussed became central to the philosophical
literature, though by this time to connection to choice functions had
been dropped. Such cases are central to what Ruth Chang
(\citeproc{ref-Chang1997}{1997}) called the `small improvement' argument
against trichotomous preferences.\footnote{See Chang
  (\citeproc{ref-Chang2017}{2017}, xxx) for a history of this.} Here's
one example of the small improvement argument.\footnote{This example is
  in Lehrer and Wagner (\citeproc{ref-LehrerWagner1985}{1985}), where it
  is mistakenly attributed to Armstrong
  (\citeproc{ref-Armstrong1939}{1939}). (Many authors subsequently made
  the same attribution; if you want to see some examples, search for the
  word `pony' among the citations of Armstrong's paper on Google
  Scholar.) Debreu has an example where two alternatives are basically
  equal, not marginally different. To the best of my knowledge, the
  earliest example with the small improvement structure is (maybe
  Tversky, maybe Luce and Suppes - gotta finish this).} A boy is
indifferent between receiving a bicycle and a pony for Christmas. He
would prefer a bicycle with a bell to one without a bell, but he would
be indifferent between either kind of bicycle and the pony. The usual
way this case is analysed is as follows.

\begin{quote}
\emph{x} = Bicycle without bell\\
\emph{y} = Pony\\
\emph{z} = Bicycle with bell
\end{quote}

The boy is indifferent between \emph{x} and \emph{y}, and between
\emph{y} and \emph{z}, but strictly prefers \emph{z} to \emph{x}. I want
to resist this reading. The core intuition here, I think, is that β
fails. The boy would choose either option from \{\emph{x}, \emph{y}\},
but if \emph{z} is added as an option, \emph{x} becomes unchoiceworthy.
If we add the assumption that \emph{R} = \emph{B}(\emph{C}), then the
usual analysis returns, and both trichotomy and II-intransitivity fail.
Without the assumption that \emph{R} = \emph{B}(\emph{C}), it's not so
clear this is a counterexample to those principles, even if it is a
counterexample to β. We'll return to this in Section~\ref{sec-dorr}.

\section{Properties γ and δ}\label{sec-gamma}

Assume \emph{R} does not satisfy trichotomy, but is a semiorder, and
\emph{C} = \emph{B}(\emph{R}). Then β will fail, but γ and δ will hold.
Conversely, for any \emph{C} where γ and δ hold, there is a semiorder
\emph{R} such that \emph{C} = \emph{B}(\emph{R}). We're not going to be
very interested in δ, but we will be very interested in γ.

The reason γ holds when \emph{R} is a semiorder and \emph{C} =
\emph{B}(\emph{R}) is instructive. If \emph{x} is choiceworthy among
\emph{S}, then nothing in \emph{S} is better than \emph{x}. Similarly,
if \emph{x} is choiceworthy among \emph{T}, then nothing in \emph{T} is
better than \emph{x}. So nothing in \emph{S} ∪ \emph{T} is better than
\emph{x}. So \emph{x} is choiceworthy among \emph{S} ∪
\emph{T}.\footnote{This argument doesn't appeal to \emph{R} being a
  semiorder, but if it is not, most theorists would not define \emph{B}
  this way.}

Conversely, if there are cases where \emph{C} should not satisfy γ, then
we'll have an argument that \emph{C} should not be based in some
semiorder \emph{R}. Showing that there are such cases will be one of the
main tasks of the rest of this paper.

We had two kinds of counterexamples to β, but only one of them will be
relevant here. I don't think there are any intuitive counterexamples to
γ that start with Poincaré-style reflections on the intransitivity of
indifference. But there are going to be variations on the bicycle and
pony example that generate intuitive counterexamples to γ. We'll come
back to these in Section~\ref{sec-sartre}.

It is common to say that when \emph{C} = \emph{B}(\emph{R}) for some
semiorder \emph{R}, that \emph{C} is \textbf{rationalizable}, and when
\emph{R} is a partial order, that \emph{C} is \textbf{rationalizable by
a partial order}. I find this terminology tendentious - why should
semiorders be the only things that can make \emph{C} rational? And as
we'll see in Section~\ref{sec-games}, it conflicts with what is called
\emph{rationalizable} in game theory. But it's a common enough
terminology that I wanted to mention it here.

\section{Aizerman's Property}\label{sec-aiz}

\begin{itemize}
\tightlist
\item
  Assume there is some set of orderings, each satisfying the
  assumptions.
\item
  Say \emph{x} ∈ \emph{C}(\emph{S}) iff \emph{x} is maximal according to
  one of these orderings.
\item
  Then \emph{C} satisfies Aiz.
\item
  Conversely, if \emph{C} satisfies Aiz, we can generate such a set.
\item
  This matters because it shows we can still put some substantive
  constraints on \emph{C}.
\end{itemize}

\section{Preference and Trade}\label{sec-trade}

One reason for starting with preference is that we are interested in
explaining trade, and the fundamental explanation of trade is that each
prefers what the other has got. But this only really holds in a
non-monetary economy. In a monetary economy, most trades are for money -
Chooser will pay \$100 for those shoes. At one level that means Chooser
prefers the shoes to the \$100. But \$100 has exchange value, not use
value. It's not of any value in itself, but only what it can buy. So
really what explains the trade is that Chooser takes the shoes to be one
of the choiceworthy things out of the things that cost \$100. In a
monetary economy, it's really choice, not preference, that explains most
of the trading that we see.

\section{Constraints on Preference}\label{sec-constraints}

Another reason for taking preference to be primary is that it lets us
state clearly some intuitive rational constraints, such as transitivity,
and independence.

But we can do that for \emph{C} as well.

Boring way, just do it for choices from pair-sets.

Better way, state the more general result.

(I have this in one of the files.)

\section{Degenerate Games}\label{sec-games}

\begin{itemize}
\tightlist
\item
  Mention Pearce's result here
\item
  Also mention that deletion of dominated strategies looks like path
  independence
\end{itemize}

\section{Choice Under Uncertainty}\label{sec-uncertainty}

\section{Multiple Equilibria}\label{sec-multieq}

\begin{itemize}
\tightlist
\item
  Start with game version
\item
  Then the version with a predictor, Skyrms's `nice demon'
\item
  Then go on to Spencer/Gallow dispute
\item
  Note that Spencer's argument just needs that if one is completely
  certain that X is best, then it's irrational to do otherwise
\item
  Go to a four way version that includes (a) a 7,1 option that
  dominates, and (b) a 1,2 option that is dominated by mixtures.
\item
  This violates β and γ
\end{itemize}

\section{Mixed Strategies}\label{sec-mixed}

How can mixing be uniquely rational? Answer, it's the only ratifiable
choice. Also, once one chooses it, all options are equally preferable
Choice is fundamentally an ex ante notion, while preference is an ex
post notion That's why any equation, like \emph{C} = \emph{B}(\emph{R})
must be wrong - they concern different times

\section{Multiple Attributes and Decisiveness}\label{sec-sartre}

Simplified version of Sartre's example Intuition, can choose either, but
should stick to it Not just for practical reasons (maybe cite Moss
here?) Model, (10,0) or (0,10); optimal choice includes choosing which
you care about Is that right everywhere? Maybe not, but it's something
we can model.

\section{β and Incompleteness}\label{sec-dorr}

Sartre's example can be used for small improvement argument But what
that really shows is that β fails If preference is about ex post, it
doesn't show that completeness fails Could be that once one has chosen,
one should have complete preferences If you think that completeness has
to be true because of the nature of vagueness (a la Broome) or because
of the nature of comparatives (a la Dorr), it doesn't follow that you
can't have multi-attribute choice, or β failures. You just have to think
that when preferences are defined ex post, they have to be complete. To
be sure, I don't think Broome's or Dorr et al's arguments do work. I
just want to note that it would be odd to, e.g., insist on β because of
arguments about the semantics of comparatives. Taking choice to be ex
ante and preference to be ex post means you don't have to.

\section{Bad Compromises}\label{sec-badcomp}

The resistance example Go through some options Intuitively, gamma could
fail here, just like it fails in the games Doesn't seem that this should
violate Path Independence though

\section{Levi and Sen}\label{sec-levisen}

Levi's secretaries examples Levi uses this to argue for constraint that
must be best on some precisification That is, for
pseudo-rationalizability Sen's objection The model he's using violates
Lederman's unidimensional expectations Still, Levi's result isn't
guaranteed by unidimensional expectations But it is coherent given it,
and we should take that seriously

\section{Negative Dominance}\label{sec-negdom}

Lederman's argument using the Sartre/Levi examples Intuition - choice
isn't binary, so it isn't explained by betterness Translate neg dom into
choice theory and it \emph{holds} What really motivates Lederman's
examples is not negdom, but γ. And we've seen lots of reasons to be
sceptical of γ.

\section{Conclusion}\label{conclusion}

Say something clever

\phantomsection\label{refs}
\begin{CSLReferences}{1}{0}
\bibitem[\citeproctext]{ref-Aizerman1981}
Aizerman, M., and A. Malishevski. 1981. {``General Theory of Best
Variants Choice: Some Aspects.''} \emph{IEEE Transactions on Automatic
Control} 26 (5): 1030--40. doi:
\href{https://doi.org/10.1109/TAC.1981.1102777}{10.1109/TAC.1981.1102777}.

\bibitem[\citeproctext]{ref-Armstrong1939}
Armstrong, W. E. 1939. {``The Determinateness of the Utility
Function.''} \emph{The Economic Journal} 49 (195): 453--67. doi:
\href{https://doi.org/10.2307/2224802}{10.2307/2224802}.

\bibitem[\citeproctext]{ref-Armstrong1948}
---------. 1948. {``Uncertainty and the Utility Function.''} \emph{The
Economic Journal} 58 (229): 1--10. doi:
\href{https://doi.org/10.2307/2226342}{10.2307/2226342}.

\bibitem[\citeproctext]{ref-Armstrong1950}
---------. 1950. {``A Note on the Theory of Consumer's Behaviour.''}
\emph{Oxford Economic Papers} 2 (1): 119--22. doi:
\href{https://doi.org/10.1093/oxfordjournals.oep.a041384}{10.1093/oxfordjournals.oep.a041384}.

\bibitem[\citeproctext]{ref-Arrow1951}
Arrow, Kenneth J. 1951. \emph{Social Choice and Individual Values}.
First. New York: John Wiley \& Sons. Online at
\url{https://cowles.yale.edu/sites/default/files/2022-09/m12-all.pdf}.

\bibitem[\citeproctext]{ref-Blair1976}
Blair, Douglas H, George Bordes, Jerry S Kelly, and Kotaro Suzumura.
1976. {``Impossibility Theorems Without Collective Rationality.''}
\emph{Journal of Economic Theory} 11 (3): 361--79. doi:
\href{https://doi.org/10.1016/0022-0531(76)90047-8}{10.1016/0022-0531(76)90047-8}.

\bibitem[\citeproctext]{ref-Chang1997}
Chang, Ruth. 1997\emph{.introduction to Incommensurability,
Incomparability and Practical Reason.} Cambridge, MA: Harvard University
Press.

\bibitem[\citeproctext]{ref-Chang2017}
---------. 2017. {``Hard Choices.''} \emph{Journal of the American
Philosophical Association} 3 (1): 1--21. doi:
\href{https://doi.org/10.1017/apa.2017.7}{10.1017/apa.2017.7}.

\bibitem[\citeproctext]{ref-Chernoff1954}
Chernoff, Herman. 1954. {``Rational Selection of Decision Functions.''}
\emph{Econometrica} 22 (4): 422--43. doi:
\href{https://doi.org/10.2307/1907435}{10.2307/1907435}.

\bibitem[\citeproctext]{ref-Debreu1960}
Debreu, Gerard. 1960. {``Review of \emph{Individual Choice Behavior: A
Theoretical Analysis}, by r. Duncan Luce.''} \emph{American Economic
Review} 50 (1): 186--88.

\bibitem[\citeproctext]{ref-Fara2001}
Fara, Delia Graff. 2001. {``Phenomenal Continua and the Sorites.''}
\emph{Mind} 110 (440): 905--36. doi:
\href{https://doi.org/10.1093/mind/110.440.905}{10.1093/mind/110.440.905}.
This paper was first published under the name {``Delia Graff.''}

\bibitem[\citeproctext]{ref-Gibbard2014}
Gibbard, Allan F. 2014. {``Social Choice and the Arrow Conditions.''}
\emph{Economics and Philosophy} 30 (3): 269--84. doi:
\href{https://doi.org/10.1017/S026626711400025X}{10.1017/S026626711400025X}.

\bibitem[\citeproctext]{ref-GulPesendorfer2008}
Gul, Faruk, and Wolfgang Pesendorfer. 2008. {``The Case for Mindless
Economics.''} In \emph{Foundations of Positive and Normative Economics},
edited by Andrew Caplin and Andrew Schotter, 2--40. Oxford: Oxford
University Press. doi:
\href{https://doi.org/10.1093/acprof:oso/9780195328318.003.0001}{10.1093/acprof:oso/9780195328318.003.0001}.

\bibitem[\citeproctext]{ref-Hansson2009}
Hansson, Sven Ove. 2009. {``Preference-Based Choice Functions: A
Generalized Approach.''} \emph{Synthese} 171 (2): 257--69. doi:
\href{https://doi.org/10.1007/s11229-009-9650-5}{10.1007/s11229-009-9650-5}.

\bibitem[\citeproctext]{ref-LehrerWagner1985}
Lehrer, Keith, and Carl Wagner. 1985. {``Intransitive Indifference: The
Semi-Order Problem.''} \emph{Synthese} 65: 249--56. doi:
\href{https://doi.org/10.1007/BF00869302}{10.1007/BF00869302}.

\bibitem[\citeproctext]{ref-Luce1956}
Luce, R. Duncan. 1956. {``Semiorders and a Theory of Utility
Discrimination.''} \emph{Econometrica} 24 (2): 178--91. doi:
\href{https://doi.org/10.2307/1905751}{10.2307/1905751}.

\bibitem[\citeproctext]{ref-Luce1959}
---------. 1959. \emph{Individual Choice Behavior: A Theoretical
Analysis}. New York: Wiley.

\bibitem[\citeproctext]{ref-Moulin1985}
Moulin, H. 1985. {``Choice Functions over a Finite Set: A Summary.''}
\emph{Social Choice and Welfare} 2 (2): 147--60.

\bibitem[\citeproctext]{ref-Samuelson1938}
Samuelson, Paul A. 1938. {``A Note on the Pure Theory of Consumer's
Behaviour.''} \emph{Econometrica} 5 (17): 61--71. doi:
\href{https://doi.org/10.2307/2548836}{10.2307/2548836}.

\bibitem[\citeproctext]{ref-Sen1969}
Sen, Amartya. 1969. {``Quasi-Transitivity, Rational Choice and
Collective Decisions.''} \emph{The Review of Economic Studies} 36 (3):
381--93. doi: \href{https://doi.org/10.2307/2296434}{10.2307/2296434}.

\bibitem[\citeproctext]{ref-Sen1970sec}
---------. (1970) 2017. \emph{Collective Choice and Social Welfare:} An
expanded edition. Cambridge, MA: Harvard University Press.

\end{CSLReferences}



\noindent Draft of June 2025


\end{document}
