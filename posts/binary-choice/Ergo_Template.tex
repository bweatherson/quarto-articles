\documentclass{ergoclass}
\usepackage{lipsum}
\usepackage{color}

\definecolor{mygrey}{gray}{0.6}
\newcommand{\gripsum}[1]{\textcolor{mygrey}{\lipsum[#1]}}

\title{Title}
\author{Name}
\affiliation{Affiliation}
\contact{name@affilation.com}% This field can be left blank, 

%% If there are multiple authors, use this command. It allows as many as three authors.
%% However, it absolutely should not be used for only one author.
%% The contact information can be included or left out on a contributor by contributor basis.
%
%\multiauthor{First Author}{affilation}{contact}{Second Author}{affiliation}{contact}{}{}{}

\textofabstract{%
The abstract should be included here. \gripsum{143}
}%

\articledoi{https://doi.org/---}
\volumeissueyear{1}{1}{2015}%
\setcounter{page}{1}% This sets the starting page number

\begin{document}

\maketitle

\lettrine{F}{or} articles starting without a section, this gives a drop cap. \gripsum{2}

\section{Section}
\subsection{Subsection}
\subsubsection{Subsubsection}
The familiar section and subsection commands can be used. \gripsum{3}
\subsection{Cases}
And cases can be described:
\begin{ergodescription}
\item[Trolley.] \gripsum{121}
\item[Case 1:] \gripsum{130}
\end{ergodescription}
\gripsum{4}
\begin{ergodescription}[Cases Can Be Titled]
\item[Trolley.] \gripsum{121}
\item[Case 1:] \gripsum{130}
\end{ergodescription}

\gripsum{5}
\end{document}