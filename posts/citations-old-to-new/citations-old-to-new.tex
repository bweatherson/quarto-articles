% Options for packages loaded elsewhere
\PassOptionsToPackage{unicode}{hyperref}
\PassOptionsToPackage{hyphens}{url}
%
\documentclass[
  10pt,
  letterpaper,
  DIV=11,
  numbers=noendperiod,
  twoside]{scrartcl}

\usepackage{amsmath,amssymb}
\usepackage{setspace}
\usepackage{iftex}
\ifPDFTeX
  \usepackage[T1]{fontenc}
  \usepackage[utf8]{inputenc}
  \usepackage{textcomp} % provide euro and other symbols
\else % if luatex or xetex
  \usepackage{unicode-math}
  \defaultfontfeatures{Scale=MatchLowercase}
  \defaultfontfeatures[\rmfamily]{Ligatures=TeX,Scale=1}
\fi
\usepackage{lmodern}
\ifPDFTeX\else  
    % xetex/luatex font selection
  \setmainfont[ItalicFont=EB Garamond Italic,BoldFont=EB Garamond
Bold]{EB Garamond Math}
  \setsansfont[]{Europa-Bold}
  \setmathfont[]{Garamond-Math}
\fi
% Use upquote if available, for straight quotes in verbatim environments
\IfFileExists{upquote.sty}{\usepackage{upquote}}{}
\IfFileExists{microtype.sty}{% use microtype if available
  \usepackage[]{microtype}
  \UseMicrotypeSet[protrusion]{basicmath} % disable protrusion for tt fonts
}{}
\usepackage{xcolor}
\usepackage[left=1in, right=1in, top=0.8in, bottom=0.8in,
paperheight=9.5in, paperwidth=6.5in, includemp=TRUE, marginparwidth=0in,
marginparsep=0in]{geometry}
\setlength{\emergencystretch}{3em} % prevent overfull lines
\setcounter{secnumdepth}{3}
% Make \paragraph and \subparagraph free-standing
\ifx\paragraph\undefined\else
  \let\oldparagraph\paragraph
  \renewcommand{\paragraph}[1]{\oldparagraph{#1}\mbox{}}
\fi
\ifx\subparagraph\undefined\else
  \let\oldsubparagraph\subparagraph
  \renewcommand{\subparagraph}[1]{\oldsubparagraph{#1}\mbox{}}
\fi


\providecommand{\tightlist}{%
  \setlength{\itemsep}{0pt}\setlength{\parskip}{0pt}}\usepackage{longtable,booktabs,array}
\usepackage{calc} % for calculating minipage widths
% Correct order of tables after \paragraph or \subparagraph
\usepackage{etoolbox}
\makeatletter
\patchcmd\longtable{\par}{\if@noskipsec\mbox{}\fi\par}{}{}
\makeatother
% Allow footnotes in longtable head/foot
\IfFileExists{footnotehyper.sty}{\usepackage{footnotehyper}}{\usepackage{footnote}}
\makesavenoteenv{longtable}
\usepackage{graphicx}
\makeatletter
\def\maxwidth{\ifdim\Gin@nat@width>\linewidth\linewidth\else\Gin@nat@width\fi}
\def\maxheight{\ifdim\Gin@nat@height>\textheight\textheight\else\Gin@nat@height\fi}
\makeatother
% Scale images if necessary, so that they will not overflow the page
% margins by default, and it is still possible to overwrite the defaults
% using explicit options in \includegraphics[width, height, ...]{}
\setkeys{Gin}{width=\maxwidth,height=\maxheight,keepaspectratio}
% Set default figure placement to htbp
\makeatletter
\def\fps@figure{htbp}
\makeatother

\setlength\heavyrulewidth{0ex}
\setlength\lightrulewidth{0ex}
\usepackage[automark]{scrlayer-scrpage}
\clearpairofpagestyles
\cehead{
  Brian Weatherson
  }
\cohead{
  Citations, Then and Now
  }
\ohead{\bfseries \pagemark}
\cfoot{}
\makeatletter
\newcommand*\NoIndentAfterEnv[1]{%
  \AfterEndEnvironment{#1}{\par\@afterindentfalse\@afterheading}}
\makeatother
\NoIndentAfterEnv{itemize}
\NoIndentAfterEnv{enumerate}
\NoIndentAfterEnv{description}
\NoIndentAfterEnv{quote}
\NoIndentAfterEnv{equation}
\NoIndentAfterEnv{longtable}
\NoIndentAfterEnv{abstract}
\renewenvironment{abstract}
 {\vspace{-1.25cm}
 \quotation\small\noindent\rule{\linewidth}{.5pt}\par\smallskip
 \noindent }
 {\par\noindent\rule{\linewidth}{.5pt}\endquotation}
\KOMAoption{captions}{tableheading}
\makeatletter
\@ifpackageloaded{caption}{}{\usepackage{caption}}
\AtBeginDocument{%
\ifdefined\contentsname
  \renewcommand*\contentsname{Table of contents}
\else
  \newcommand\contentsname{Table of contents}
\fi
\ifdefined\listfigurename
  \renewcommand*\listfigurename{List of Figures}
\else
  \newcommand\listfigurename{List of Figures}
\fi
\ifdefined\listtablename
  \renewcommand*\listtablename{List of Tables}
\else
  \newcommand\listtablename{List of Tables}
\fi
\ifdefined\figurename
  \renewcommand*\figurename{Figure}
\else
  \newcommand\figurename{Figure}
\fi
\ifdefined\tablename
  \renewcommand*\tablename{Table}
\else
  \newcommand\tablename{Table}
\fi
}
\@ifpackageloaded{float}{}{\usepackage{float}}
\floatstyle{ruled}
\@ifundefined{c@chapter}{\newfloat{codelisting}{h}{lop}}{\newfloat{codelisting}{h}{lop}[chapter]}
\floatname{codelisting}{Listing}
\newcommand*\listoflistings{\listof{codelisting}{List of Listings}}
\makeatother
\makeatletter
\makeatother
\makeatletter
\@ifpackageloaded{caption}{}{\usepackage{caption}}
\@ifpackageloaded{subcaption}{}{\usepackage{subcaption}}
\makeatother
\ifLuaTeX
  \usepackage{selnolig}  % disable illegal ligatures
\fi
\IfFileExists{bookmark.sty}{\usepackage{bookmark}}{\usepackage{hyperref}}
\IfFileExists{xurl.sty}{\usepackage{xurl}}{} % add URL line breaks if available
\urlstyle{same} % disable monospaced font for URLs
\hypersetup{
  pdftitle={Citations, Then and Now},
  pdfauthor={Brian Weatherson},
  hidelinks,
  pdfcreator={LaTeX via pandoc}}

\title{Citations, Then and Now}
\author{Brian Weatherson}
\date{2024}

\begin{document}
\maketitle
\begin{abstract}
This note looks at articles that were relatively widely cited soon after
publication, and asks how often they have been cited in recent years.
The main finding is that between 1980 and the late 1990s, many articles
that were widely cited at the time have largely disappeared from the
citation record in recent times.
\end{abstract}

\setstretch{1.1}
This post is about how citation patterns change. In particular, it is
about what kinds of articles are widely cited when they first come out,
but less cited in the future. The key result is that there were
surprisingly many of these articles in journals between 1980 and 1995.
The main cause of the drop in citations seems to be simple changes in
trends. In particular, so much journal attention was paid to relatively
a priori investigations into mental content. That's not nearly as large
a part of the philosophical landscape now, so the articles that focus on
it are less widely cited.

The dataset I'm using is the same as in
\href{http://brian.weatherson.org/quarto/posts/citations-raw-data/citations.html}{an
earlier post}, and I won't repeat it here. The big thing to know is that
I'm just looking at one hundred Anglophone, relatively analytic,
philosophy journals, and looking at citations in those journals to those
journals. For this study I'm going to start in 1965.

Note that because of a gap in the Web of Science data, I had to manually
add citations to \emph{Journal of Philosophy} articles from 1971 to
1974, but I don't have citations by those articles. That leads to some
weirdnesses, but I don't think it drastically affects these results.

Note also that Web of Science doesn't start indexing \emph{Analysis}
until 1975. This does affect the results substantially; there is no
point looking pre-1965 because so many of the widely cited articles are
in \emph{Analysis}. I suspect having \emph{Analysis} would make a pretty
big change to 1965-1970 as well, but I can't be sure.

\section{Study 1 - Counting widely cited articles}\label{sec-study-one}

The main focus here is on articles published between 1965 and 2014. I'm
stopping in 2014 because I want to be able to compare how articles were
cited soon after publication, with how they have been cited recently.
That requires having enough non-recent years that are `soon after
publication'. Since my dataset stops in mid-2022, that implied stopping
in 2014.

Divide that fifty year period up into ten periods of five years each, in
the obvious way. So for each decade we have the early and late half of
the decade, though we just have late 1960s and early 2010s.

For each half-decade, and each article published in it, find two values.

\begin{itemize}
\tightlist
\item
  The \textbf{early cites} to the article are citations by the end of
  the subsequent half-decade. So for articles published between 1990 and
  1994, that means citations in or before 1999.
\item
  The \textbf{late cites} to the article are citations in 2020, 2021,
  and the part of 2022 covered in the dataset.
\end{itemize}

The late cites might not seem like much of a dataset. But because there
are so many articles published now, and because citation practices have
changed so much, that includes many many citations. The dataset I have
goes back to the mid-1950s, but over a quarter of the citations are in
these two and a half years.

Having done that, rank the articles by the number of early cites they
have, using the number of late cites as a tiebreaker. The tiebreakers
are important, because citations just within journals are not very high;
the typical case is that there are lots of ties.

Then choose the top twenty articles, i.e., the twenty articles that were
most cited by the end of the subsequent half-decade. (With ties broken
by looking at what was most cited recently.) For the period 1990-1994,
Table~\ref{tbl-early-1990s} lists the twenty articles in question.

\begin{longtable}[]{@{}
  >{\raggedright\arraybackslash}p{(\columnwidth - 0\tabcolsep) * \real{1.0000}}@{}}

\caption{\label{tbl-early-1990s}The twenty articles from 1990-1994 most
widely cited at the time.}

\tabularnewline

\toprule\noalign{}
\begin{minipage}[b]{\linewidth}\raggedright
Article
\end{minipage} \\
\midrule\noalign{}
\endhead
\bottomrule\noalign{}
\endlastfoot
D Davidson (1990) ``The Structure and Content of Truth,'' \emph{Journal
Of Philosophy} 87~(6):~279-328. \\
K Neander (1991) ``Functions as Selected Effects: The Conceptual
Analyst's Defense,'' \emph{Philosophy Of Science} 58~(2):~168-184. \\
S Yablo (1992) ``Mental Causation,'' \emph{Philosophical Review}
101~(2):~245-280. \\
T Crane and DH Mellor (1990) ``There is No Question of Physicalism,''
\emph{Mind} 99~(394):~185-206. \\
JW Kim (1990) ``Supervenience as a Philosophical Concept,''
\emph{Metaphilosophy} 21~(1-2):~1-27. \\
P Kitcher (1990) ``The Division of Cognitive Labor,'' \emph{Journal Of
Philosophy} 87~(1):~5-22. \\
K Neander (1991) ``The Teleological Notion of Function,''
\emph{Australasian Journal Of Philosophy} 69~(4):~454-468. \\
P Kitcher (1992) ``The Naturalists Return,'' \emph{Philosophical Review}
101~(1):~53-114. \\
G Rosen (1990) ``Modal Fictionalism,'' \emph{Mind} 99~(395):~327-354. \\
MB Burke (1992) ``Copper Statues and Pieces of Copper: A Challenge To
the Standard Account,'' \emph{Analysis} 52~(1):~12-17. \\
S Schiffer (1992) ``Belief Ascription,'' \emph{Journal Of Philosophy}
89~(10):~499-521. \\
M McKinsey (1991) ``Anti-Individualism and Privileged Access,''
\emph{Analysis} 51~(1):~9-16. \\
M Tye (1990) ``Vague Objects,'' \emph{Mind} 99~(396):~535-557. \\
MB Burke (1994) ``Preserving the Principle of 1 Object To a Place: A
Novel Account of the Relations Among Objects, Sorts, Sortals, and
Persistance Conditions,'' \emph{Philosophy And Phenomenological
Research} 54~(3):~591-624. \\
JA Fodor (1991) ``A Modal Argument for Narrow Content,'' \emph{Journal
Of Philosophy} 88~(1):~5-26. \\
L Laudan (1990) ``Normative Naturalism,'' \emph{Philosophy Of Science}
57~(1):~44-59. \\
PA Boghossian (1990) ``The Status of Content,'' \emph{Philosophical
Review} 99~(2):~157-184. \\
WC Salmon (1994) ``Causality Without Counterfactuals,'' \emph{Philosophy
Of Science} 61~(2):~297-312. \\
H Putnam (1994) ``Sense, Nonsense, and the Senses: An Inquiry into the
Powers of the Human Mind,'' \emph{Journal Of Philosophy}
91~(9):~445-465. \\
G Forbes (1990) ``The Indispensability of Sinn,'' \emph{Philosophical
Review} 99~(4):~535-563. \\

\end{longtable}

If we look at all the articles in the dataset, just over 1\% have been
cited sixteen or more times since 2020. (Just under 1\% have been cited
seventeen or more times.) Call this top 1\% of cited articles the
\emph{widely cited} articles in recent philosophy. Our first question is
how many of these twenty articles that were the most cited at the time,
are widely cited in this sense.

The answer is just four: Stephen Yablo's paper on mental causation,
Philip Kitcher's paper on the division of cognitive labour, and Karen
Neander's two papers on functions. Table~\ref{tbl-early-1990s-expanded}
shows how often each of these articles were cited in the `early' years,
i.e., 1990-1999, and how often they are cited in the `late years', i.e.,
from 2020 to mid 2022.

\begin{longtable}[]{@{}rrl@{}}

\caption{\label{tbl-early-1990s-expanded}Early and late citations to
twenty articles from 1990-1994.}

\tabularnewline

\toprule\noalign{}
Early Cites & Late Cites & Article \\
\midrule\noalign{}
\endhead
\bottomrule\noalign{}
\endlastfoot
33 & 9 & Davidson 1990 \\
32 & 33 & Neander 1991 \\
27 & 66 & Yablo 1992 \\
26 & 7 & Crane 1990 \\
21 & 6 & Kim 1990 \\
20 & 54 & Kitcher 1990 \\
20 & 18 & Neander 1991 \\
20 & 13 & Kitcher 1992 \\
19 & 11 & Rosen 1990 \\
19 & 8 & Burke 1992 \\
18 & 8 & Schiffer 1992 \\
17 & 10 & Mckinsey 1991 \\
17 & 6 & Tye 1990 \\
17 & 5 & Burke 1994 \\
17 & 3 & Fodor 1991 \\
16 & 14 & Laudan 1990 \\
16 & 9 & Boghossian 1990 \\
16 & 8 & Salmon 1994 \\
16 & 8 & Putnam 1994 \\
16 & 4 & Forbes 1990 \\

\end{longtable}

This is extremely unusual, though as we'll see in a bit it is something
of an outlier result. If we do the same thing for each of the five year
periods, we typically see about 8-10 articles be widely cited in this
way, with the numbers rising as we get closer to the present.
Figure~\ref{fig-still-standing} shows the numbers for each half-decade.
(Note that the year on the x-axis is the start of the half-decade being
shown.)

\begin{figure}

\centering{

\includegraphics{citations-old-to-new_files/figure-pdf/fig-still-standing-1.pdf}

}

\caption{\label{fig-still-standing}How many of the twenty articles most
cited at the time are still widely cited.}

\end{figure}%

As you can see, the period 1990-1994 really stands out here. But it's a
bit crude to just look at what's above or below a threshold. Let's try
being a bit more finegrained.

\section{Study 2 - Median citations of The Twenty}\label{sec-study-two}

Instead of looking at how many of the twenty articles crossed a somewhat
arbitrarily selected threshold, we could look instead at the median
number of citations they have in the period 2020-2022. I'm using median
not mean because the means end up being largely determined by how widely
cited the one or two most cited pieces are. If we use the same twenty
articles for each five year period, and calculate the median number of
citations they have in 2020-2022, we get the results seen in
Figure~\ref{fig-median-cites}.

\begin{figure}

\centering{

\includegraphics{citations-old-to-new_files/figure-pdf/fig-median-cites-1.pdf}

}

\caption{\label{fig-median-cites}The median number of recent citations
to twenty articles most cited at the time.}

\end{figure}%

The 1990-1994 period still does poorly, but it's not as dramatic as on
Figure~\ref{fig-still-standing}.

The 1970-1974 period does surprisingly badly on this measure. That
period includes the four very widely cited articles listed in
Table~\ref{tbl-early-1970s-sample}.

\begin{longtable}[]{@{}
  >{\raggedleft\arraybackslash}p{(\columnwidth - 4\tabcolsep) * \real{0.0764}}
  >{\raggedleft\arraybackslash}p{(\columnwidth - 4\tabcolsep) * \real{0.0701}}
  >{\raggedright\arraybackslash}p{(\columnwidth - 4\tabcolsep) * \real{0.8535}}@{}}

\caption{\label{tbl-early-1970s-sample}Four very widely cited articles
from the early 1970s}

\tabularnewline

\toprule\noalign{}
\begin{minipage}[b]{\linewidth}\raggedleft
Early Cites
\end{minipage} & \begin{minipage}[b]{\linewidth}\raggedleft
Late Cites
\end{minipage} & \begin{minipage}[b]{\linewidth}\raggedright
Article
\end{minipage} \\
\midrule\noalign{}
\endhead
\bottomrule\noalign{}
\endlastfoot
22 & 88 & D Lewis (1973) ``Causation,'' \emph{Journal Of Philosophy} 70
(17): 556-567. \\
23 & 72 & H Frankfurt (1971) ``Freedom of the Will and the Concept of a
Person,'' \emph{Journal Of Philosophy} 68 (1): 5-20. \\
12 & 49 & P Benacerraf (1973) ``Mathematical Truth,'' \emph{Journal Of
Philosophy} 70 (19): 661-679. \\
12 & 48 & JJ Thomson (1971) ``A Defense of Abortion,'' \emph{Philosophy
\& Public Affairs} 1~(1):~47-66. \\
23 & 33 & H Putnam (1973) ``Meaning and Reference,'' \emph{Journal Of
Philosophy} 70 (19): 699-711. \\
11 & 23 & P Foot (1972) ``Morality as a System of Hypothetical
Imperatives,'' \emph{Philosophical Review} 81~(3):~305-316. \\
12 & 22 & D Parfit (1971) ``Personal Identity,'' \emph{Philosophical
Review} 80~(1):~3-27. \\
11 & 16 & T Burge (1973) ``Reference and Proper Names,'' \emph{Journal
Of Philosophy} 70 (14): 425-439. \\
12 & 15 & H Field (1973) ``Theory Change and the Indeterminacy of
Reference,'' \emph{Journal Of Philosophy} 70 (14): 462-481. \\
18 & 9 & H Field (1972) ``Tarski's Theory of Truth,'' \emph{Journal Of
Philosophy} 69 (13): 347-375. \\
16 & 9 & WV Quine (1970) ``On the Reasons for Indeterminacy of
Translation,'' \emph{Journal Of Philosophy} 67~(6):~178-183. \\
18 & 8 & J Kim (1973) ``Causation, Nomic Subsumption, and the Concept of
Event,'' \emph{Journal Of Philosophy} 70 (8): 217-236. \\
11 & 6 & JJ Thomson (1971) ``The Time of a Killing,'' \emph{Journal Of
Philosophy} 68 (5): 115-132. \\
15 & 5 & A Goldman (1971) ``The Individuation of Action,'' \emph{Journal
Of Philosophy} 68 (21): 761-774. \\
13 & 5 & M Devitt (1974) ``Singular Terms,'' \emph{Journal Of
Philosophy} 71 (7): 183-205. \\
12 & 5 & E Zahar (1973) ``Why Did Einsteins Programme Supersede
Lorentz's (1),'' \emph{British Journal For The Philosophy Of Science}
24~(2):~95-123. \\
11 & 4 & J Perry (1972) ``Can the Self Divide?,'' \emph{Journal Of
Philosophy} 69 (16): 463-488. \\
11 & 4 & AW Wood (1972) ``Marxian Critique of Justice,''
\emph{Philosophy \& Public Affairs} 1~(3):~244-282. \\
11 & 1 & R Rorty (1970) ``Incorrigibility as Mark of Mental,''
\emph{Journal Of Philosophy} 67~(12):~399-424. \\
15 & 0 & R Rorty (1972) ``The World Well Lost,'' \emph{Journal Of
Philosophy} 69 (19): 649-665. \\

\end{longtable}

But several other articles that were more prominent at the time,
including two by Hartry Field and two by Richard Rorty, are not nearly
as prominent in the recent literature. The large number of citations to
these four articles doesn't move the median that much.

\section{Study 3 - Rolling Periods}\label{sec-study-three}

If you look back at Table~\ref{tbl-early-1990s}, you see a lot of
articles from 1990 and 1991 in particular. This shouldn't be a surprise.
The way the twenty articles were selected was by looking at the most
cited articles by a fixed date, in this case 1999. Articles published in
1990 and 1991 had a lot more time to accumulate citations by 1999 than
articles published later in the half-decade.

One way to fix that is to get away from having round numbers in our five
year periods. For any year \emph{y} from 1965 to 2010, we can perform
the following calculations.

\begin{itemize}
\tightlist
\item
  Select the articles published between \emph{y} and *y+4.
\item
  Sort them by the number of citations they have by \emph{y}+9 (using
  citations since 2020 as a tiebreaker).
\item
  Ask how many of the top twenty on that list are widely cited recently
  (i.e., have at least sixteen citations).
\end{itemize}

If we set \emph{y} to be 1990, that gives us the results we saw in
Table~\ref{tbl-early-1990s} and Table~\ref{tbl-early-1990s-expanded}.
But we can do it for years that don't end with 0 and 5. If we do it for
all the years from 1965 to 2010, we get the results in
Figure~\ref{fig-all-still-standing}.

\begin{figure}

\centering{

\includegraphics{citations-old-to-new_files/figure-pdf/fig-all-still-standing-1.pdf}

}

\caption{\label{fig-all-still-standing}How many of the twenty articles
most cited at the time are still widely cited.}

\end{figure}%

The odd result for 1990 itself looks like an outlier here. Why are the
years around it so different?

It's easy to explain why the value for 1989 is different. The articles
listed in Table~\ref{tbl-top-1989} were published in 1989, and widely
cited.

\begin{longtable}[]{@{}
  >{\raggedleft\arraybackslash}p{(\columnwidth - 4\tabcolsep) * \real{0.1026}}
  >{\raggedleft\arraybackslash}p{(\columnwidth - 4\tabcolsep) * \real{0.0940}}
  >{\raggedright\arraybackslash}p{(\columnwidth - 4\tabcolsep) * \real{0.8034}}@{}}

\caption{\label{tbl-top-1989}Widely cited articles from 1989}

\tabularnewline

\toprule\noalign{}
\begin{minipage}[b]{\linewidth}\raggedleft
Early Cites
\end{minipage} & \begin{minipage}[b]{\linewidth}\raggedleft
Late Cites
\end{minipage} & \begin{minipage}[b]{\linewidth}\raggedright
Article
\end{minipage} \\
\midrule\noalign{}
\endhead
\bottomrule\noalign{}
\endlastfoot
32 & 36 & RG Millikan (1989) ``In Defense of Proper Functions,''
\emph{Philosophy Of Science} 56~(2):~288-302. \\
23 & 39 & GA Cohen (1989) ``On the Currency of Egalitarian Justice,''
\emph{Ethics} 99~(4):~906-944. \\
21 & 20 & PA Boghossian and JD Velleman (1989) ``Color as a Secondary
Quality,'' \emph{Mind} 98~(389):~81-103. \\
30 & 44 & RG Millikan (1989) ``Biosemantics,'' \emph{Journal Of
Philosophy} 86~(6):~281-297. \\

\end{longtable}

The range 1989-1993 includes those four articles which the range
1990-1994 does not. But what is going on with the range 1991-1995? How
does adding 1995 to the set make such a difference? It's a bit more
complicated than that. The nine articles from 1991-1995 which are widely
cited recently are shown in Table~\ref{tbl-top-1991}.

\begin{longtable}[]{@{}
  >{\raggedleft\arraybackslash}p{(\columnwidth - 4\tabcolsep) * \real{0.0541}}
  >{\raggedleft\arraybackslash}p{(\columnwidth - 4\tabcolsep) * \real{0.0495}}
  >{\raggedright\arraybackslash}p{(\columnwidth - 4\tabcolsep) * \real{0.8964}}@{}}

\caption{\label{tbl-top-1991}Widely cited articles from 1991}

\tabularnewline

\toprule\noalign{}
\begin{minipage}[b]{\linewidth}\raggedleft
Early Cites
\end{minipage} & \begin{minipage}[b]{\linewidth}\raggedleft
Late Cites
\end{minipage} & \begin{minipage}[b]{\linewidth}\raggedright
Article
\end{minipage} \\
\midrule\noalign{}
\endhead
\bottomrule\noalign{}
\endlastfoot
18 & 84 & D Lewis (1994) ``Humean Supervenience Debugged,'' \emph{Mind}
103~(412):~473-490. \\
31 & 66 & S Yablo (1992) ``Mental Causation,'' \emph{Philosophical
Review} 101~(2):~245-280. \\
17 & 61 & DC Dennett (1991) ``Real Patterns,'' \emph{Journal Of
Philosophy} 88~(1):~27-51. \\
28 & 54 & T Burge (1993) ``Content Preservation,'' \emph{Philosophical
Review} 102~(4):~457-488. \\
18 & 48 & M Johnston (1992) ``How To Speak of the Colors,''
\emph{Philosophical Studies} 68~(3):~221-263. \\
36 & 33 & K Neander (1991) ``Functions as Selected Effects: The
Conceptual Analyst's Defense,'' \emph{Philosophy Of Science}
58~(2):~168-184. \\
18 & 21 & K Neander (1995) ``Misrepresenting and Malfunctioning,''
\emph{Philosophical Studies} 79~(2):~109-141. \\
19 & 20 & M Forster and E Sober (1994) ``How To Tell When Simpler, More
Unified, or Less \emph{Ad Hoc} Theories Will Provide More Accurate
Predictions,'' \emph{British Journal For The Philosophy Of Science}
45~(1):~1-35. \\
22 & 18 & K Neander (1991) ``The Teleological Notion of Function,''
\emph{Australasian Journal Of Philosophy} 69~(4):~454-468. \\

\end{longtable}

What's happened here is that several of the articles, most notably the
Lewis and Dennett ones, did not get a huge number of citations straight
away. They counted for the period 1991-1995 but not the period 1990-1994
because they were so often cited in 2000.

There is a general methodological point here which I've never resolved
satisfactorily. Just what we mean by saying an article was cited a lot
`when it came out' is not entirely clear. Does Dennett's paper count as
frequently cited when it first came out because it started getting a lot
of citations nine years after publication? There are some borderline
judgments to be made here.

Anyway, this shows that there was something a bit unusual about the
results in Section~\ref{sec-study-one} and Section~\ref{sec-study-two}.
It was only by a very particular choice of years that the period
1990-1994 looks so unusual.

But the broader picture shows that the particular period 1990-1994 was a
bit unusual relative to its surrounds. It still leaves us with a
question about those surrounds. And we need one last study to see the
thing that most needs explaining.

\section{Study 4 - Medians in Rolling Periods}\label{sec-study-four}

As in Section~\ref{sec-study-three}, do the following calculation for
each year \emph{y}.

\begin{itemize}
\tightlist
\item
  Select the articles published between \emph{y} and *y+4.
\item
  Sort them by the number of citations they have by \emph{y}+9 (using
  citations since 2020 as a tiebreaker).
\item
  Focus on the top twenty in that list.
\end{itemize}

But instead of asking how many of these twenty are `widely cited',
instead calculate the median number of citations since 2020 for those
twenty articles. The results are shown in Figure~\ref{fig-all-median}.

\begin{figure}

\centering{

\includegraphics{citations-old-to-new_files/figure-pdf/fig-all-median-1.pdf}

}

\caption{\label{fig-all-median}The median number of recent cites for the
twenty articles most cited at the time.}

\end{figure}%

This is a really amazing graph. Every year from 1980 to 1995 is worse
than every year from 1974 to 1979, and every year from 1996 to the
present. How could this have happened?

\section{Theories}\label{sec-theories}



\noindent Published online in September 2024.

\end{document}
