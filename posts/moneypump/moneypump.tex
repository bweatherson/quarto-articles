% Options for packages loaded elsewhere
\PassOptionsToPackage{unicode}{hyperref}
\PassOptionsToPackage{hyphens}{url}
%
\documentclass[
  11pt,
  letterpaper,
  DIV=11,
  numbers=noendperiod,
  twoside]{scrartcl}

\usepackage{amsmath,amssymb}
\usepackage{setspace}
\usepackage{iftex}
\ifPDFTeX
  \usepackage[T1]{fontenc}
  \usepackage[utf8]{inputenc}
  \usepackage{textcomp} % provide euro and other symbols
\else % if luatex or xetex
  \usepackage{unicode-math}
  \defaultfontfeatures{Scale=MatchLowercase}
  \defaultfontfeatures[\rmfamily]{Ligatures=TeX,Scale=1}
\fi
\usepackage{lmodern}
\ifPDFTeX\else  
    % xetex/luatex font selection
    \setmainfont[ItalicFont=EB Garamond Italic,BoldFont=EB Garamond
Bold]{EB Garamond Math}
    \setsansfont[]{EB Garamond}
  \setmathfont[]{Garamond-Math}
\fi
% Use upquote if available, for straight quotes in verbatim environments
\IfFileExists{upquote.sty}{\usepackage{upquote}}{}
\IfFileExists{microtype.sty}{% use microtype if available
  \usepackage[]{microtype}
  \UseMicrotypeSet[protrusion]{basicmath} % disable protrusion for tt fonts
}{}
\usepackage{xcolor}
\usepackage[left=1.1in, right=1in, top=0.8in, bottom=0.8in,
paperheight=9.5in, paperwidth=7in, includemp=TRUE, marginparwidth=0in,
marginparsep=0in]{geometry}
\setlength{\emergencystretch}{3em} % prevent overfull lines
\setcounter{secnumdepth}{3}
% Make \paragraph and \subparagraph free-standing
\makeatletter
\ifx\paragraph\undefined\else
  \let\oldparagraph\paragraph
  \renewcommand{\paragraph}{
    \@ifstar
      \xxxParagraphStar
      \xxxParagraphNoStar
  }
  \newcommand{\xxxParagraphStar}[1]{\oldparagraph*{#1}\mbox{}}
  \newcommand{\xxxParagraphNoStar}[1]{\oldparagraph{#1}\mbox{}}
\fi
\ifx\subparagraph\undefined\else
  \let\oldsubparagraph\subparagraph
  \renewcommand{\subparagraph}{
    \@ifstar
      \xxxSubParagraphStar
      \xxxSubParagraphNoStar
  }
  \newcommand{\xxxSubParagraphStar}[1]{\oldsubparagraph*{#1}\mbox{}}
  \newcommand{\xxxSubParagraphNoStar}[1]{\oldsubparagraph{#1}\mbox{}}
\fi
\makeatother


\providecommand{\tightlist}{%
  \setlength{\itemsep}{0pt}\setlength{\parskip}{0pt}}\usepackage{longtable,booktabs,array}
\usepackage{calc} % for calculating minipage widths
% Correct order of tables after \paragraph or \subparagraph
\usepackage{etoolbox}
\makeatletter
\patchcmd\longtable{\par}{\if@noskipsec\mbox{}\fi\par}{}{}
\makeatother
% Allow footnotes in longtable head/foot
\IfFileExists{footnotehyper.sty}{\usepackage{footnotehyper}}{\usepackage{footnote}}
\makesavenoteenv{longtable}
\usepackage{graphicx}
\makeatletter
\newsavebox\pandoc@box
\newcommand*\pandocbounded[1]{% scales image to fit in text height/width
  \sbox\pandoc@box{#1}%
  \Gscale@div\@tempa{\textheight}{\dimexpr\ht\pandoc@box+\dp\pandoc@box\relax}%
  \Gscale@div\@tempb{\linewidth}{\wd\pandoc@box}%
  \ifdim\@tempb\p@<\@tempa\p@\let\@tempa\@tempb\fi% select the smaller of both
  \ifdim\@tempa\p@<\p@\scalebox{\@tempa}{\usebox\pandoc@box}%
  \else\usebox{\pandoc@box}%
  \fi%
}
% Set default figure placement to htbp
\def\fps@figure{htbp}
\makeatother
% definitions for citeproc citations
\NewDocumentCommand\citeproctext{}{}
\NewDocumentCommand\citeproc{mm}{%
  \begingroup\def\citeproctext{#2}\cite{#1}\endgroup}
\makeatletter
 % allow citations to break across lines
 \let\@cite@ofmt\@firstofone
 % avoid brackets around text for \cite:
 \def\@biblabel#1{}
 \def\@cite#1#2{{#1\if@tempswa , #2\fi}}
\makeatother
\newlength{\cslhangindent}
\setlength{\cslhangindent}{1.5em}
\newlength{\csllabelwidth}
\setlength{\csllabelwidth}{3em}
\newenvironment{CSLReferences}[2] % #1 hanging-indent, #2 entry-spacing
 {\begin{list}{}{%
  \setlength{\itemindent}{0pt}
  \setlength{\leftmargin}{0pt}
  \setlength{\parsep}{0pt}
  % turn on hanging indent if param 1 is 1
  \ifodd #1
   \setlength{\leftmargin}{\cslhangindent}
   \setlength{\itemindent}{-1\cslhangindent}
  \fi
  % set entry spacing
  \setlength{\itemsep}{#2\baselineskip}}}
 {\end{list}}
\usepackage{calc}
\newcommand{\CSLBlock}[1]{\hfill\break\parbox[t]{\linewidth}{\strut\ignorespaces#1\strut}}
\newcommand{\CSLLeftMargin}[1]{\parbox[t]{\csllabelwidth}{\strut#1\strut}}
\newcommand{\CSLRightInline}[1]{\parbox[t]{\linewidth - \csllabelwidth}{\strut#1\strut}}
\newcommand{\CSLIndent}[1]{\hspace{\cslhangindent}#1}

\setlength\heavyrulewidth{0ex}
\setlength\lightrulewidth{0ex}
\usepackage[automark]{scrlayer-scrpage}
\clearpairofpagestyles
\cehead{
  Brian Weatherson
  }
\cohead{
  Consistency and Resoluteness
  }
\ohead{\bfseries \pagemark}
\cfoot{}
\makeatletter
\newcommand*\NoIndentAfterEnv[1]{%
  \AfterEndEnvironment{#1}{\par\@afterindentfalse\@afterheading}}
\makeatother
\NoIndentAfterEnv{itemize}
\NoIndentAfterEnv{enumerate}
\NoIndentAfterEnv{description}
\NoIndentAfterEnv{quote}
\NoIndentAfterEnv{equation}
\NoIndentAfterEnv{longtable}
\NoIndentAfterEnv{abstract}
\renewenvironment{abstract}
 {\vspace{-1.25cm}
 \quotation\small\noindent\emph{Abstract}:}
 {\endquotation}
\newfontfamily\tfont{EB Garamond}
\addtokomafont{disposition}{\rmfamily}
\addtokomafont{title}{\normalfont\itshape}
\let\footnoterule\relax
\cehead{
        Lederman and Weatherson
        }
\KOMAoption{captions}{tableheading}
\makeatletter
\@ifpackageloaded{caption}{}{\usepackage{caption}}
\AtBeginDocument{%
\ifdefined\contentsname
  \renewcommand*\contentsname{Table of contents}
\else
  \newcommand\contentsname{Table of contents}
\fi
\ifdefined\listfigurename
  \renewcommand*\listfigurename{List of Figures}
\else
  \newcommand\listfigurename{List of Figures}
\fi
\ifdefined\listtablename
  \renewcommand*\listtablename{List of Tables}
\else
  \newcommand\listtablename{List of Tables}
\fi
\ifdefined\figurename
  \renewcommand*\figurename{Figure}
\else
  \newcommand\figurename{Figure}
\fi
\ifdefined\tablename
  \renewcommand*\tablename{Table}
\else
  \newcommand\tablename{Table}
\fi
}
\@ifpackageloaded{float}{}{\usepackage{float}}
\floatstyle{ruled}
\@ifundefined{c@chapter}{\newfloat{codelisting}{h}{lop}}{\newfloat{codelisting}{h}{lop}[chapter]}
\floatname{codelisting}{Listing}
\newcommand*\listoflistings{\listof{codelisting}{List of Listings}}
\makeatother
\makeatletter
\makeatother
\makeatletter
\@ifpackageloaded{caption}{}{\usepackage{caption}}
\@ifpackageloaded{subcaption}{}{\usepackage{subcaption}}
\makeatother

\usepackage{bookmark}

\IfFileExists{xurl.sty}{\usepackage{xurl}}{} % add URL line breaks if available
\urlstyle{same} % disable monospaced font for URLs
\hypersetup{
  pdftitle={Consistency and Resoluteness},
  pdfauthor={Harvey Lederman; Brian Weatherson},
  hidelinks,
  pdfcreator={LaTeX via pandoc}}


\title{Consistency and Resoluteness}
\author{Harvey Lederman \and Brian Weatherson}
\date{2025}

\begin{document}
\maketitle
\begin{abstract}
When does a little story in which a character loses money, or at least
comes out with less money than they might easily have had, reveal a
defect in that character's rationality? We argue that these stories are
less philosophically revealing than is often assumed. In particular, we
argue the story can only be used in this kind of argument if the
character has firm beliefs about what they will do later in the story.
This constraint is, we argue, violated in recent arguments against
imprecise credences and incomparable values.
\end{abstract}


\setstretch{1.1}
One genre of parable has played a key role in philosophical theorizing
about rationality. In some respects, the conventions of the genre are
simple: A character is portrayed, presented with certain choices, and,
in response, loses something of value, with no compensatory gain. The
inevitable moral is then drawn: the protagonist displays a rational
defect.

Recently this kind of parable has been deployed by Adam Elga
(\citeproc{ref-Elga2010}{2010}), to argue against imprecise credences,
and by Johan Gustaffsson (\citeproc{ref-Gustafsson2022}{2022},
\citeproc{ref-Gustafsson2025}{Forthcoming}), to argue against incomplete
preferences. We argue that these parables fail to ground convincing
arguments. This is interesting both for debates about the nature of
credences and values, but also for more general debates about the role
of these parables in philosophy.

Let's start with three such parables that we hope you'll agree are not
persuasive.

Dylan owns \emph{B}, but has a slight preference for \emph{A}. So they
trade \emph{B} plus ε for \emph{A}. The next day they wake up with the
opposite preference, and trade \emph{A} plus ε for \emph{B}. At the end
they are back where they started, minus 2ε. Should this story make us
think that rational people never change their tastes? Hopefully not.

Blake has credence 0.6 in \emph{p}. They pay 0.55 for a bet that pays 1
if \emph{p}, and 0 otherwise. The next morning, they wake up and find,
much to their surprise, that they now have credence 0.4 in \emph{p}.
They sell the bet for 0.45, netting a loss of 0.1. This one is a little
more controversial, but we also don't think that this shows that
changing one's beliefs is irrational.\footnote{Two important caveats
  here. One is that believers in various uniqueness theses will think
  that Blake is not perfectly rational; see Kopec and Titelbaum
  (\citeproc{ref-KopecTitelbaum2016}{2016}) for a survey of the issues
  here. But note that uniqueness is a theory of substantive rationality,
  and these stories are meant to reveal violations of structural
  rationality. So we think uniqueness shouldn't be assumed here. Another
  is that there is an argument from David Lewis
  (\citeproc{ref-Lewis1999b}{1999}) that not conditionalising will lead
  to a money pump, and Blake doesn't conditionalise. But as several
  authors have pointed out (e.g., Bradley
  (\citeproc{ref-Bradley2005}{2005})), Lewis's argument shows at most
  that any policy other than conditionalisation leads to a money pump.
  And Blake has no such policy; they just find themselves with new
  beliefs.}

Leslie is offered the following trade: If they pay 5, they will receive
a bet that pays 8 if this fair coin lands heads, and 0 if it lands
tails. Also, they have the option of paying 1 to put off their decision
until after the coin is flipped. Textbook theories of rationality say
that they should pay the 1 and take this option. But whatever way the
coin lands, and whatever they do, they would have been better off doing
that without paying 1. Here we think everyone will agree that Leslie
does nothing irrational by paying to delay the decision, even though
whatever happens, they could have done better. In general, paying for
options is not always irrational. Anyone who pays extra for a
cancellable flight or hotel does that, and these payments are not always
irrational.

These cases show, we think, that stories involving money pumps don't
reveal irrationality on the part of the protagonist if the story
involves changes of desire, or changes of belief, either through
`fickle' change (Woodard (\citeproc{ref-Woodard2022}{2022})) or through
learning. We want to suggest a similar constraint in the same
neighbourhood. The protagonist must have beliefs about what they will do
at future nodes, and these beliefs must update by conditionalisation. If
that's not true, we will argue, the money pump parable doesn't show
anything.

So consider a canonical money-pump argument against incomplete
preferences, often called the ``Single Souring Moneypump''
(\citeproc{ref-Chang1997}{Chang 1997, 11};
\citeproc{ref-Gustafsson2022}{Gustafsson 2022, 26}). The tree for this
is shown in Figure~\ref{fig-single-souring}.

\begin{figure}

\centering{

\pandocbounded{\includegraphics[keepaspectratio]{moneypump_files/figure-pdf/fig-single-souring-1.pdf}}

}

\caption{\label{fig-single-souring}The Single Souring Moneypump}

\end{figure}%

The agent is presented first with a choice between taking Up, which
results in \emph{A}, or Down, which results in a choice between option
\emph{B} and \emph{A}-. The protagonist is assumed to prefer \emph{A} to
\emph{A}-, but to have no relevant preferential relations between either
\emph{A} and \emph{B} or \emph{A}- and \emph{B} (that is, they are not
indifferent between the elements of these pairs, and do not prefer one
to the other). Plausibly, then, the agent is rationally permitted, in
their first choice, to take either option: they can take Up and get
\emph{A} or take Down and face the choice between \emph{A}- and
\emph{B}. The problem is what happens if they take this second option.
For in that case, it would seem that they are permitted again to take
\emph{A}-, but this is a loss relative to what they could have had at no
cost, i.e.~\emph{A}.

In this simple case, a normal person who made the decision to go Down,
rather than the sure thing of \emph{A}, would do so in the belief (in
fact, presumably, in the knowledge) that they would go on to choose
\emph{B}, not \emph{A}-. Indeed, it seems to us that anyone who made
this choice without the belief that they would later choose \emph{B}
would be doing something odd in the case as described. So, if such a
person did make the choice in the first place, to go down, giving
themselves the option between \emph{B} and \emph{A}-, and if they then
chose \emph{A}-, this choice would either (a) require a change in their
belief about what they would do, or (b) require that the person choose
\emph{A}- while still believing that we would choose \emph{B}. The
second of these options, (b), quickly leads to inconsistency given
natural constraints on the sort of person for whom a moneypump
dramatizes a rational defect. Provided the hero knows that they're in
fact faced with the choice, and provided they update their conditional
belief that, if they are faced with the choice, they will choose
\emph{B}, then at the moment they make the choice, they will believe
that they are choosing \emph{B}. Provided they also believe that they
are choosing what they are choosing when they choose it, they will
believe that they are choosing \emph{A}-. Given knowledge that these
were exclusive options, the hero's beliefs are inconsistent, hence
irrational.

If a moneypump only dramatizes incoherence if it is compatible with the
protagonist's beliefs about what they will do being held fixed (ruling
out (a) as an option in the above story), and we can furthermore
motivate a rational requirement that, in this case, the person who goes
Down believes that they will later choose \emph{B}, this moneypump (and
its descendants, on which, more soon) would disappear. If, however,
beliefs like this one are instead allowed to shift in the course of a
moneypump, and only their preferences must be held fixed throughout the
story, then the moneypump is genuine. So which is it? \%It takes a bit
to spell out why \emph{exactly} the latter would be rational---we'll
come back to that below---but for now the question is whether (a) is a
problem. If I change my mind in the course of the choice, would that
violate the generic conventions of the money-pump, or is this the sort
of change of belief that should be allowed?

So far we have spoken about the capacity to rationally form beliefs
about what one will do, and to have that belief evolve rationally
through the moment of choice. A different approach, which we are also
open to, would be to require that a moneypump illustrates a relevant
rational defect only if the character portrayed is capable of knowing
what they will later do. Since knowledge requires truth, if it is a
requirement of the genre that the character can choose Down only if they
know they will choose \emph{B} later, then a person who will ultimately
choose \emph{A}- will not have been an appropriate target of the
moneypump (since they will not have known that they would choose
\emph{B}). We ourselves think this diagnosis is important, and to
different degrees even find it attractive, but we will mostly focus
below on the diagnosis in terms of belief, because we expect it to be
more widely appealing to proponents of moneypumps, who are typically
interested in a notion of rationality that can be articulated in terms
of belief/credence and desire/preference alone. We'll also abbreviate
the relevant condition by saying that the person has the ability to form
beliefs about what they will do, leaving implicit the very important
additional requirement that, once formed, this belief continues to
evolve in response to the evidence.

The above discussion helps to illustrate the important distinction
between the claim that a condition is a rational requirement and the
claim that a moneypump can dramatize a rational defect only if it
depicts people who satisfy a given condition. It is at least
controversial whether, even in simple cases like the Single-Souring
Moneypump, being able to know what you will do later is a requirement of
rationality. Suppose that, unforeseen by the moneypumper or the chooser,
a blow to the head will make the chooser have a thirst for \emph{A}- in
between their choice of Down and their decision between \emph{A}- and
\emph{B}. They would then violate the putative rational requirement.
Some will conclude that the rational requirement should be rejected,
since it does not seem that the protagonist is irrational for making
their initial choice, even though that was made without knowledge of
what they would do at the later time. Others will hold that, the person
is irrational \emph{with respect to the whole sequence} of their
actions, arguing that a sequence can be irrational even if none of its
constituent choices is. (On its own, the choice of A- in the second
round would also be rational in this case, in light of what they then
prefer.) We will not attempt to settle this controversy here. Instead we
want to emphasize that this debate about what is a
\emph{rational requirement} is a different one from the question of what
background conditions on the protagonist's psychology are required if a
moneypump is to be the basis for dramatizing a rational defect. We do
not think it is at all plausible that there is a rational requirement
that people's tastes remain the same. But we do think that, in the
context of such preference changes (as in the case of the grapes or the
plane tickets), a loss no longer dramatizes a rational defect. Such
changes put the person out of bounds, as it were. The same point applies
to beliefs about what one will do: we are not claiming that it is a
rational requirement that, in every case, a person be able to form such
beliefs. We are instead claiming that it is only in cases where people
are capable of forming such beliefs that a moneypump illustrates a
rational defect.

\%The same style of argument can't be leveled against a requirement that
a person who knowingly faces this sequence of choices choose Down only
if they believe they will choose \emph{B}. If the person is, in advance,
uncertain of whether their preferences will stay the same at the later
nodes, and for this reason they don't have the relevant belief, then
it's quite a different problem than the one we saw above. In that case
you're forced to treat yourself a bit like another person, a future
randomizing device. If your preferences are ``standard'\,' (satisfying
completeness and independence) there's no problem: you should just
choose what's best by your current lights. If they're not, and say,
they're incomplete, then the question will depend on hard issues about
how people with such nonstandard preferences should handle decisions
under risk (or for that matter uncertainty), which isn't our topic here.

Before turning to our argument for this claim, we want to further
illustrate the application of our constraint to two prominent examples:
one due to Elga, and the other due to Gustafsson. The discussion will
help to develop the exact form of our thesis, and also highlight the
contrast between our own diagnosis of what goes wrong in these cases,
and a different diagnosis, advocated by proponents of resolute choice.

ELGA:

We turn now to Gustafsson's ``A behavioral moneypump\ldots'\,' This
moneypump assumes a decision-maker whose choices between incomparable
options are determined probabilistically. The agent faces a first
choice, to go Up or Down. Regardless of which way they go, a coin will
be tossed. If they have chosen Up, they receive \emph{A}- (if Heads) or
\emph{B}- (if Tails). If they choose Down, and the coin is Heads, they
will choose between B- - and A. If the coin comes up Tails they will
choose between B and A. Given arbitrary probabilities on the agent's
choices, Gustafsson shows that we can choose a biased coin in such a way
that the agent is rationally required to choose Up. But this is a
moneypump, since the agent could improve their choice in every state: if
Heads, by going Down and choosing \emph{A}; if Tails by going Down and
choosing \emph{B}.

There are two senses in which, in Gustafsson's example the protagonist
cannot rationally form beliefs about what they will do at the outset.
The first is that, they do not know how the coin will land, and so
should not form an outright belief about what they will do (if they go
Down). The second is that, they are not in a position to form a belief
about what they will do
\emph{even conditional on the coin landing Heads} (or for that matter
Tails). It is this second sense of inability that our condition is
designed to rule out: our claim is that moneypumps are effective only if
they are compatible with the person having knowledge of / belief in
certain conditionals about how they will act given the resolution of
relevant empirical uncertainty. If one can only form probabilistic
beliefs about what one will do in this conditional sense, our claim is
that the person is alienated from their future selves in a way that
defangs the moneypump. Gustafsson's behavioral moneypump is a moneypump
only for someone who is alienated from themselves in this way. If,
however, we're allowed to include requirements about what beliefs they
will have about their own choices (and the preservation of those
beliefs), then the person who chooses Down may be required to do so
believing they'll choose \emph{A} if Heads and \emph{B} if Tails. The
probabilities disappear, and the person would be irrational if they
deviated from the plan---not because they have deviated from the plan,
but because that conflicts with what they believed they would do. (The
same point can be made for the moneypump in Gustafsson 2022 TODO page
number.)

This discussion illustrates how our approach differs from the usual form
of Resolute Choice, or even from what Gustafsson calls Conservative
Choice, in that it comes ready-made with an explanation for why
deviating from the plan is irrational. Elga and Gustafsson both complain
(CITE) against the fan of resolute choice that it's not clear why having
made a plan should be a reason for a future choice. Gustafsson also
argues that it's not clear why having made a plan should even be a
tie-breaker when other things are equal. But the idea here is not that
having made the plan gives you a reason to take one action or another.
In the case of our proposal about knowledge, it's that a person who
doesn't know what they will later do is out of bounds, not in the field
of play. If they could have the knowledge, then since what one knows
must be true, they would act in the relevant way. In the case of our
proposal about belief, it's that a person who could not form and
maintain the beliefs would be out of bounds. If they could form and
maintain the beliefs, they would be moneypumped only if they violated
the \emph{consistency} of their beliefs (which is irrational in a
different way).

We think it is already of interest that the fan of incomplete beliefs or
preferences (to mention just two) could respond to moneypump arguments
against their position by advancing the idea that the opponent has got
the generic conventions wrong. The true generic conventions, they might
claim, require that beliefs about what you'll do stay the same
throughout the process, once they're formed. The opponent of
incompleteness might reject this claim about the genre, but the debate
would at least have changed subject-matter. More strongly, we think this
move would neutralize such moneypump arguments into a draw instead of a
clear loss for those who favor the rational permissibility of such
incompleteness.

But we want to go further than just suggesting this view, to give a
direct argument for it. The argument is based on the following case, a
slight variant of one from Rubinstein and Piccione, discussed by
Stalnaker (1999, p 317). A fair coin will be flipped, and you will see
the outcome. Regardless of how it comes up, you can choose to opt out or
to opt in. If you opt out, you get a payoff of 2. If you opt in, your
memory of the outcome will be erased, and you'll face a second choice:
guessing the outcome of the coin. If the coin came up Heads, and you
guess correctly, you get a payoff of 6, otherwise 0. If it comes up
Tails and you guess correctly, you get a payoff of 6, otherwise 0.

There are two optimal strategies, each of which has an expected payout
of 4. On the first, one opts out if Heads and opts in if Tails, choosing
Tails at the next choice. On the second, one opts out if Tails and opts
in if Heads, choosing Heads at the next choice. We think that the sort
of character depicted in a moneypump should be able to execute one of
these strategies, and obtain the best expectation. Either of our
proposals---that such a character must be able to know what they will
do, or that such a character must be able to form a belief about what
they will do---would explain this verdict. If the person is able to come
to know what they will do later (and remember that this is a form of
conditional knowledge---they know what they will do if the coin comes up
a certain way), then they will be able---by making the decision to
execute this strategy---to come to know that if it is Heads (say) they
will opt in and then play Heads. If they know this then plausibly, if
they know that it is Heads, they know that they will opt in and then
play Heads. Similarly, if they are able to come to believe that if it is
Heads they will opt in and then play Heads (and maintain this belief),
then if they come to know that it is Heads, they will plausibly believe
that they will opt in and then play Heads. If this belief persists, then
consistency will demand that they in fact play
Heads.\%\textbackslash footnote\{In the original version of the above
case there is an asymmetry in the second round. If you opt in, the coin
came up Heads, and you guess correctly, you get a payoff of 6, otherwise
0. If you opt in, it comes up Tails and you guess correctly, you get a
payoff of 5, otherwise 0. In such a case, we have the judgment that the
strategy if Heads, opt out, if Tails, opt in and then play Tails is the
uniquely rational one. The question is why. Our suggestion is that, if
the person plans and therefore come to believe that they will choose to
opt out if Tails and then opt in followed by Heads if Heads, this makes
the later choice of Tails (if Heads) irrational. Choosing Tails at that
node would conflict with your earlier belief that you would choose
Heads, by reasoning similar to the reasoning I gave above for the Single
Souring Moneypump. If, in forgetting the outcome of the coin toss, the
person also forgets their belief about what they would do at various
nodes, it is not clear that the optimal strategy is the uniquely
rational one. The fact that you can, in forming these plans and hence
beliefs about what you will do, determine what will be consistent with
your beliefs at later nodes, explains why you can in this case (as you
couldn't in a two-person version) choose the optimal strategy. In
choosing a plan at the start, you choose what you believe. In choosing
what you believe, you choose what will be consistent with your beliefs,
and hence what you will do if you remain rational.

\%We haven't focused on this case in the main text because we think
there are alternative stories that would deliver the same verdict about
the rational action. \%\}

This case seems to us similar to the stylized cases presented in
moneypumps. Someone who grants this, and holds that only preferences
(but nothing else) are held fixed in a moneypump, would have to concede
that even a person with the preferences of a classical expected utility
maximizer with utility linear in money would not be able to achieve the
maximum payout in this case. They might of course claim that empirical
forgetting of the kind depicted above should be ruled out, and not
allowed in a true moneypump. But this new requirement---that for
relevant empirical claims there is no loss of knowledge or belief---must
be argued for as a further feature of moneypumps. Moreover, it is at
least on a par with our own view, which says that the protagonist of a
moneypump is able to know or form and preserve a belief about what they
will do. So we contend that this case provides evidence that to be the
basis for drawing a moral of irrationality, the moneypump must be
consistent with the claim that the protagonist can form beliefs about
what they will do.

We have presented our main claim as an \emph{additional} claim, over and
above the idea that preferences should be held fixed for a moneypump to
reach its intended moral. We are, in fact, attracted to a stronger
claim, that our condition is a replacement for this more standard one.
In other words, we are open to the idea that preferences need not be
held fixed in a moneypump, provided the protagonist is capable of
knowing or having a belief about what they will do. This claim, just
like the one about preferences, would deliver the correct verdict in our
grapes and plane ticket cases. Moreover, in cases where
preference-changes are relevant to the case but not relevant to the
actions before a person (for instance, if in the single-souring
moneypump, the person came to prefer \emph{A} over \emph{B}, after
making the initial choice to go down, while holding the rest of their
preferences fixed), it does not seem to us that all bets are off about
the rationality of the subsequent action. The \emph{reason} changes of
preference matter in moneypumps seems to be that they make belief /
knowledge about what the person will do impossible.

We have focused on an understanding of moneypumps as dramatizations of
defects of rationality (CITE Christensen). A different understanding of
such arguments is that the possibility of exploitation is itself the
problem, not merely a dramatization of an underlying rational defect.
There is then a question of what forms of exploitability are
problematic. Our discussion of
\texttt{generic\ constraints\textquotesingle{}\textquotesingle{}\ could\ be\ translated\ into\ claims\ about\ the\ conditions\ under\ which\ exploitability\ is\ a\ rational\ problem.\ But\ we\ expect\ the\ translation\ to\ be\ less\ compelling\ to\ those\ who\ have\ this}realistic'\,'
understanding of moneypumps. In real life, we often do not know what we
will do (even in this conditional sense), and are not in a position to
form (rational) beliefs about it. We think there are independent reasons
to reject this way of understanding moneypumps, but here is not the
place to argue for that claim.

\phantomsection\label{refs}
\begin{CSLReferences}{1}{0}
\bibitem[\citeproctext]{ref-Bradley2005}
Bradley, Richard. 2005. {``Radical Probabilism and Bayesian
Conditioning.''} \emph{Philosophy of Science} 72 (2): 342--64. doi:
\href{https://doi.org/10.1086/432427}{10.1086/432427}.

\bibitem[\citeproctext]{ref-Chang1997}
Chang, Ruth. 1997\emph{.introduction to Incommensurability,
Incomparability and Practical Reason.} Cambridge, MA: Harvard University
Press.

\bibitem[\citeproctext]{ref-Elga2010}
Elga, Adam. 2010. {``Subjective Probabilities Should Be Sharp.''}
\emph{Philosophers' Imprint} 10: 1--11.

\bibitem[\citeproctext]{ref-Gustafsson2025}
Gustafsson, Johan E. Forthcoming. {``A Behavioural Money-Pump Argument
for Completeness.''} \emph{Theory and Decision}, Forthcoming.

\bibitem[\citeproctext]{ref-Gustafsson2022}
---------. 2022. \emph{Money-Pump Arguments}. Cambridge: Cambridge
University Press.

\bibitem[\citeproctext]{ref-KopecTitelbaum2016}
Kopec, Matthew, and Michael G. Titelbaum. 2016. {``The Uniqueness
Thesis.''} \emph{Philosophy Compass} 11 (4): 189--200. doi:
\href{https://doi.org/10.1111/phc3.12318}{10.1111/phc3.12318}.

\bibitem[\citeproctext]{ref-Lewis1999b}
Lewis, David. 1999. {``Why Conditionalize?''} In \emph{Papers in
Metaphysics and Epistemology}, 403--7. Cambridge University Press.
Originally written as a course handout in 1972.

\bibitem[\citeproctext]{ref-Woodard2022}
Woodard, Elise. 2022. {``A Puzzle about Fickleness.''} \emph{No{û}s}.
doi: \href{https://doi.org/10.1111/nous.12359}{10.1111/nous.12359}.

\end{CSLReferences}



\noindent Draft of February 2025.


\end{document}
