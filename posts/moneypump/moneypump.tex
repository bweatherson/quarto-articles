% Options for packages loaded elsewhere
% Options for packages loaded elsewhere
\PassOptionsToPackage{unicode}{hyperref}
\PassOptionsToPackage{hyphens}{url}
%
\documentclass[
  11pt,
  letterpaper,
  DIV=11,
  numbers=noendperiod,
  twoside]{scrartcl}
\usepackage{xcolor}
\usepackage[left=1.1in, right=1in, top=0.8in, bottom=0.8in,
paperheight=9.5in, paperwidth=7in, includemp=TRUE, marginparwidth=0in,
marginparsep=0in]{geometry}
\usepackage{amsmath,amssymb}
\setcounter{secnumdepth}{3}
\usepackage{iftex}
\ifPDFTeX
  \usepackage[T1]{fontenc}
  \usepackage[utf8]{inputenc}
  \usepackage{textcomp} % provide euro and other symbols
\else % if luatex or xetex
  \usepackage{unicode-math} % this also loads fontspec
  \defaultfontfeatures{Scale=MatchLowercase}
  \defaultfontfeatures[\rmfamily]{Ligatures=TeX,Scale=1}
\fi
\usepackage{lmodern}
\ifPDFTeX\else
  % xetex/luatex font selection
  \setmainfont[ItalicFont=EB Garamond Italic,BoldFont=EB Garamond
Bold]{EB Garamond Math}
  \setsansfont[]{EB Garamond}
  \setmathfont[]{Garamond-Math}
\fi
% Use upquote if available, for straight quotes in verbatim environments
\IfFileExists{upquote.sty}{\usepackage{upquote}}{}
\IfFileExists{microtype.sty}{% use microtype if available
  \usepackage[]{microtype}
  \UseMicrotypeSet[protrusion]{basicmath} % disable protrusion for tt fonts
}{}
\usepackage{setspace}
% Make \paragraph and \subparagraph free-standing
\makeatletter
\ifx\paragraph\undefined\else
  \let\oldparagraph\paragraph
  \renewcommand{\paragraph}{
    \@ifstar
      \xxxParagraphStar
      \xxxParagraphNoStar
  }
  \newcommand{\xxxParagraphStar}[1]{\oldparagraph*{#1}\mbox{}}
  \newcommand{\xxxParagraphNoStar}[1]{\oldparagraph{#1}\mbox{}}
\fi
\ifx\subparagraph\undefined\else
  \let\oldsubparagraph\subparagraph
  \renewcommand{\subparagraph}{
    \@ifstar
      \xxxSubParagraphStar
      \xxxSubParagraphNoStar
  }
  \newcommand{\xxxSubParagraphStar}[1]{\oldsubparagraph*{#1}\mbox{}}
  \newcommand{\xxxSubParagraphNoStar}[1]{\oldsubparagraph{#1}\mbox{}}
\fi
\makeatother


\usepackage{longtable,booktabs,array}
\usepackage{calc} % for calculating minipage widths
% Correct order of tables after \paragraph or \subparagraph
\usepackage{etoolbox}
\makeatletter
\patchcmd\longtable{\par}{\if@noskipsec\mbox{}\fi\par}{}{}
\makeatother
% Allow footnotes in longtable head/foot
\IfFileExists{footnotehyper.sty}{\usepackage{footnotehyper}}{\usepackage{footnote}}
\makesavenoteenv{longtable}
\usepackage{graphicx}
\makeatletter
\newsavebox\pandoc@box
\newcommand*\pandocbounded[1]{% scales image to fit in text height/width
  \sbox\pandoc@box{#1}%
  \Gscale@div\@tempa{\textheight}{\dimexpr\ht\pandoc@box+\dp\pandoc@box\relax}%
  \Gscale@div\@tempb{\linewidth}{\wd\pandoc@box}%
  \ifdim\@tempb\p@<\@tempa\p@\let\@tempa\@tempb\fi% select the smaller of both
  \ifdim\@tempa\p@<\p@\scalebox{\@tempa}{\usebox\pandoc@box}%
  \else\usebox{\pandoc@box}%
  \fi%
}
% Set default figure placement to htbp
\def\fps@figure{htbp}
\makeatother


% definitions for citeproc citations
\NewDocumentCommand\citeproctext{}{}
\NewDocumentCommand\citeproc{mm}{%
  \begingroup\def\citeproctext{#2}\cite{#1}\endgroup}
\makeatletter
 % allow citations to break across lines
 \let\@cite@ofmt\@firstofone
 % avoid brackets around text for \cite:
 \def\@biblabel#1{}
 \def\@cite#1#2{{#1\if@tempswa , #2\fi}}
\makeatother
\newlength{\cslhangindent}
\setlength{\cslhangindent}{1.5em}
\newlength{\csllabelwidth}
\setlength{\csllabelwidth}{3em}
\newenvironment{CSLReferences}[2] % #1 hanging-indent, #2 entry-spacing
 {\begin{list}{}{%
  \setlength{\itemindent}{0pt}
  \setlength{\leftmargin}{0pt}
  \setlength{\parsep}{0pt}
  % turn on hanging indent if param 1 is 1
  \ifodd #1
   \setlength{\leftmargin}{\cslhangindent}
   \setlength{\itemindent}{-1\cslhangindent}
  \fi
  % set entry spacing
  \setlength{\itemsep}{#2\baselineskip}}}
 {\end{list}}
\usepackage{calc}
\newcommand{\CSLBlock}[1]{\hfill\break\parbox[t]{\linewidth}{\strut\ignorespaces#1\strut}}
\newcommand{\CSLLeftMargin}[1]{\parbox[t]{\csllabelwidth}{\strut#1\strut}}
\newcommand{\CSLRightInline}[1]{\parbox[t]{\linewidth - \csllabelwidth}{\strut#1\strut}}
\newcommand{\CSLIndent}[1]{\hspace{\cslhangindent}#1}



\setlength{\emergencystretch}{3em} % prevent overfull lines

\providecommand{\tightlist}{%
  \setlength{\itemsep}{0pt}\setlength{\parskip}{0pt}}



 


\setlength\heavyrulewidth{0ex}
\setlength\lightrulewidth{0ex}
\usepackage[automark]{scrlayer-scrpage}
\clearpairofpagestyles
\cehead{
  Brian Weatherson
  }
\cohead{
  Consistency and Resoluteness
  }
\ohead{\bfseries \pagemark}
\cfoot{}
\makeatletter
\newcommand*\NoIndentAfterEnv[1]{%
  \AfterEndEnvironment{#1}{\par\@afterindentfalse\@afterheading}}
\makeatother
\NoIndentAfterEnv{itemize}
\NoIndentAfterEnv{enumerate}
\NoIndentAfterEnv{description}
\NoIndentAfterEnv{quote}
\NoIndentAfterEnv{equation}
\NoIndentAfterEnv{longtable}
\NoIndentAfterEnv{abstract}
\renewenvironment{abstract}
 {\vspace{-1.25cm}
 \quotation\small\noindent\emph{Abstract}:}
 {\endquotation}
\newfontfamily\tfont{EB Garamond}
\addtokomafont{disposition}{\rmfamily}
\addtokomafont{title}{\normalfont\itshape}
\let\footnoterule\relax
\cehead{
        Lederman and Weatherson
        }
\KOMAoption{captions}{tableheading}
\makeatletter
\@ifpackageloaded{caption}{}{\usepackage{caption}}
\AtBeginDocument{%
\ifdefined\contentsname
  \renewcommand*\contentsname{Table of contents}
\else
  \newcommand\contentsname{Table of contents}
\fi
\ifdefined\listfigurename
  \renewcommand*\listfigurename{List of Figures}
\else
  \newcommand\listfigurename{List of Figures}
\fi
\ifdefined\listtablename
  \renewcommand*\listtablename{List of Tables}
\else
  \newcommand\listtablename{List of Tables}
\fi
\ifdefined\figurename
  \renewcommand*\figurename{Figure}
\else
  \newcommand\figurename{Figure}
\fi
\ifdefined\tablename
  \renewcommand*\tablename{Table}
\else
  \newcommand\tablename{Table}
\fi
}
\@ifpackageloaded{float}{}{\usepackage{float}}
\floatstyle{ruled}
\@ifundefined{c@chapter}{\newfloat{codelisting}{h}{lop}}{\newfloat{codelisting}{h}{lop}[chapter]}
\floatname{codelisting}{Listing}
\newcommand*\listoflistings{\listof{codelisting}{List of Listings}}
\makeatother
\makeatletter
\@ifpackageloaded{tikz}{}{\usepackage{tikz}}
\makeatother
\makeatletter
\@ifpackageloaded{caption}{}{\usepackage{caption}}
\@ifpackageloaded{subcaption}{}{\usepackage{subcaption}}
\makeatother
\usepackage{bookmark}
\IfFileExists{xurl.sty}{\usepackage{xurl}}{} % add URL line breaks if available
\urlstyle{same}
\hypersetup{
  pdftitle={Consistency and Resoluteness},
  pdfauthor={Harvey Lederman; Brian Weatherson},
  hidelinks,
  pdfcreator={LaTeX via pandoc}}


\title{Consistency and Resoluteness}
\author{Harvey Lederman \and Brian Weatherson}
\date{2025}
\begin{document}
\maketitle
\begin{abstract}
When does a little story in which a character loses money, or at least
comes out with less money than they might easily have had, reveal a
defect in that character's rationality? We argue that these stories are
less philosophically revealing than is often assumed. In particular, we
argue the story can only be used in this kind of argument if the
character has firm beliefs about what they will do later in the story.
This constraint is, we argue, violated in recent arguments against
imprecise credences and incomparable values.
\end{abstract}


\setstretch{1.1}
One genre of parable has played a key role in philosophical theorizing
about rationality. In some respects, the conventions of the genre are
simple: A character is portrayed, presented with certain choices, and,
in response, loses something of value, with no compensatory gain. The
inevitable moral is then drawn: the protagonist displays a rational
defect.

Recently this kind of parable has been deployed by Adam Elga
(\citeproc{ref-Elga2010}{2010}), to argue against imprecise credences,
and by Johan Gustaffsson (\citeproc{ref-Gustafsson2022}{2022},
\citeproc{ref-Gustafsson2025}{Forthcoming}), to argue against incomplete
preferences. We argue that these parables fail to ground convincing
arguments. This is interesting both for debates about the nature of
credences and values, but also for more general debates about the role
of these parables in philosophy.

Let's start with three such parables that we hope you'll agree are not
persuasive.

Dylan owns \emph{B}, but has a slight preference for \emph{A}. So they
trade \emph{B} plus ε for \emph{A}. The next day they wake up with the
opposite preference, and trade \emph{A} plus ε for \emph{B}. At the end
they are back where they started, minus 2ε. Should this story make us
think that rational people never change their tastes? Hopefully not.

Blake has credence 0.6 in \emph{p}. They pay 0.55 for a bet that pays 1
if \emph{p}, and 0 otherwise. The next morning, they wake up and find,
much to their surprise, that they now have credence 0.4 in \emph{p}.
They sell the bet for 0.45, netting a loss of 0.1. This one is a little
more controversial, but we also don't think that this shows that
changing one's beliefs is irrational.\footnote{Two important caveats
  here. One is that believers in various uniqueness theses will think
  that Blake is not perfectly rational; see Kopec and Titelbaum
  (\citeproc{ref-KopecTitelbaum2016}{2016}) for a survey of the issues
  here. But note that uniqueness is a theory of substantive rationality,
  and these stories are meant to reveal violations of structural
  rationality. So we think uniqueness shouldn't be assumed here. Another
  is that there is an argument from David Lewis
  (\citeproc{ref-Lewis1999b}{1999}) that not conditionalising will lead
  to a money pump, and Blake doesn't conditionalise. But as several
  authors have pointed out (e.g., Bradley
  (\citeproc{ref-Bradley2005}{2005})), Lewis's argument shows at most
  that any policy other than conditionalisation leads to a money pump.
  And Blake has no such policy; they just find themselves with new
  beliefs.}

Leslie is offered the following trade: If they pay 5, they will receive
a bet that pays 8 if this fair coin lands heads, and 0 if it lands
tails. Also, they have the option of paying 1 to put off their decision
until after the coin is flipped. Textbook theories of rationality say
that they should pay the 1 and take this option. But whatever way the
coin lands, and whatever they do, they would have been better off doing
that without paying 1. Here we think everyone will agree that Leslie
does nothing irrational by paying to delay the decision, even though
whatever happens, they could have done better. In general, paying for
options is not always irrational. Anyone who pays extra for a
cancellable flight or hotel does that, and these payments are not always
irrational.

These cases show, we think, that stories involving money pumps don't
reveal irrationality on the part of the protagonist if the story
involves changes of desire, or changes of belief, either through
`fickle' change (Woodard (\citeproc{ref-Woodard2022}{2022})) or through
learning. We want to suggest a similar constraint in the same
neighbourhood. The protagonist must have beliefs about what they will do
at future nodes, and these beliefs must update by conditionalisation. If
that's not true, we will argue, the money pump parable doesn't show
anything.

So consider a canonical money-pump argument against incomplete
preferences, often called the ``Single Souring Moneypump''
(\citeproc{ref-Chang1997}{Chang 1997, 11};
\citeproc{ref-Gustafsson2022}{Gustafsson 2022, 26}). The tree for this
is shown in Figure~\ref{fig-single-souring}.

\begin{figure}

\centering{

\usetikzlibrary{calc}

\tikzset{
% Two node styles for game trees: solid and hollow
solid node/.style={circle,draw,inner sep=1.5,fill=black},
hollow node/.style={circle,draw,inner sep=1.5},
square node/.style={rectangle,draw, inner sep = 1, fill = black}
}

% Specify spacing for each level of the tree
%\tikzstyle{level 1}=[level distance=12mm,sibling distance=15mm]
%\tikzstyle{level 2}=[level distance=15mm,sibling distance=15mm]
%\tikzstyle{level 3}=[level distance=13mm,sibling distance=11mm]


\begin{tikzpicture}[grow=right, sloped]
\node[hollow node, draw] {}
    child {
        node[solid node, draw] {}
        child {
            node[square node, fill, inner sep=1.5pt, label=right:{\emph{B}}] {}
            edge from parent
            node[above] {}
        }
        child {
            node[square node, fill, inner sep=1.5pt, label=right:{\emph{A}-}] {}
            edge from parent
            node[below] {}
        }
        edge from parent
        node[below] {}
    }
        child {
        node[square node, fill, inner sep=1.5pt, label=right:{\emph{A}}] {}
        edge from parent
        node[above] {}
    };
\end{tikzpicture}


}

\caption{\label{fig-single-souring}The Single Souring Moneypump}

\end{figure}%

The agent, call them Evelyn, is presented at \emph{t}\textsubscript{1}
with a choice between taking Up, which results in \emph{A}, or Down,
which results in a choice (at \emph{t}\textsubscript{2}) between option
\emph{B} and \emph{A}-. Evelyn is assumed to prefer \emph{A} to
\emph{A}-, but to have no relevant preferential relations between either
\emph{A} and \emph{B} or \emph{A}- and \emph{B} (that is, they are not
indifferent between the elements of these pairs, and do not prefer one
to the other). Plausibly, then, Evelyn is rationally permitted, at
\emph{t}\textsubscript{1}, to take either option: they can take Up and
get \emph{A} or take Down and face the choice between \emph{A}- and
\emph{B}. The problem is what happens if they take this second option.
For in that case, it would seem that they are permitted at
\emph{t}\textsubscript{2} to take \emph{A}-, but this is a loss relative
to what they could have had at no cost, i.e.~\emph{A}.

In this simple case, a normal person who made the decision to go Down,
rather than the sure thing of \emph{A}, would do so in the belief (in
fact, presumably, in the knowledge) that they would go on to choose
\emph{B}, not \emph{A}-. Indeed, it seems to us that anyone who made
this choice without the belief that they would later choose \emph{B}
would be doing something odd in the case as described. So, if such a
person did make the choice in the first place, to go down, giving
themselves the option between \emph{B} and \emph{A}-, and if they then
chose \emph{A}-, this choice would either (a) require a change in their
belief about what they would do, or (b) require that the person choose
\emph{A}- while still believing that we would choose \emph{B}. The
second of these options, (b), quickly leads to inconsistency given
natural constraints on the sort of person for whom a moneypump
dramatizes a rational defect. Provided Evelyn knows that they're in fact
faced with the choice, and provided they update their conditional belief
that, if they are faced with the choice, they will choose \emph{B}, then
at the moment they make the choice, they will believe that they are
choosing \emph{B}. Provided they also believe that they are choosing
what they are choosing when they choose it, they will believe that they
are choosing \emph{A}-. Given knowledge that these were exclusive
options, Evelyn's beliefs are inconsistent, hence irrational.

If a moneypump only dramatizes incoherence if it is compatible with the
protagonist's beliefs about what they will do being held fixed (ruling
out (a) as an option in the above story), and we can furthermore
motivate a rational requirement that, in this case, the person who goes
Down believes that they will later choose \emph{B}, this moneypump (and
its descendants, on which, more soon) would disappear. If, however,
beliefs like this one are instead allowed to shift in the course of a
moneypump, and only their preferences must be held fixed throughout the
story, then the moneypump is genuine. So which is it?

The examples involving Blake and Leslie already provide reasons for
thinking that beliefs must be held fixed in a moneypump argument. So the
big issue is whether, as we've suggested in the last two paragraphs, it
is a requirement that characters like Evelyn have firm beliefs about
what they will do at later stages. We have six arguments for the view
that it is. That is, each of these arguments is intended to show that
the moneypump only shows that Evelyn's preferences are incoherent if we
can consistently add to the story that at \emph{t}\textsubscript{1}
Evelyn has a firm belief about what they will do at
\emph{t}\textsubscript{2} (should the game get that far). The first
three arguments are probably not going to convince many sceptics, but
having them on the table is helpful for understanding the later
arguments. (And they might convince some neutrals, which is our primary
aim.)\footnote{Note to Harvey: The first three are probably cuttable to
  be honest; they are as much me writing out how I came to a view like
  this as anything resembling a standard argument.}

The first argument is a somewhat basic appeal to intuition. The
moneypump argument, like all arguments, shows that some propositions are
inconsistent. The further inference that this particular proposition is
false requires a judgment that it is the least plausible of the
inconsistent set. In this case, at most what the argument shows is that
not both (1) and (2) can be correct.

\begin{enumerate}
\def\labelenumi{\arabic{enumi}.}
\tightlist
\item
  Evelyn can rationally act at \emph{t}\textsubscript{1} without having
  a firm belief about what they will do at \emph{t}\textsubscript{2}.
\item
  Evelyn can rationally have no preference between \emph{A} and
  \emph{B}, and no preference between \emph{B} and \emph{A}-, but prefer
  \emph{A} to \emph{A}-.
\end{enumerate}

It seems to us 2 is much more intuitively plausible than 1. We suspect
this won't move many people on the other side, however, so let's move on
to more arguments.

The second argument is a perhaps more basic appeal to authority. Open up
any game theory textbook and look at how it treats decisions over time.
All of the theories put forward will assume, usually without comment,
that each player has firm beliefs about what strategy they will adopt at
later times. Those strategies might be mixed strategies, so the player
might not know what move they will make. But it's an unchallenged
assumption that rational action in a dynamic game requires having a view
about what one's later self will do. Treating one's later self as a
foreign agent, or as something like the weather that might change at
random, is unheard of. Now maybe all these theorists are wrong. Or maybe
we could try to strengthen this argument by saying that naturalism
requires taking seriously the best scientific practice and this is the
best scientific practice. We'll just settle for noting that the view
we're putting forward is not a strange one; indeed in large parts of the
literature it is the most mundane orthodoxy.

The third argument is a tu quoque against Elga's deployment of this
argument in particular.\footnote{For Elga, the options \emph{A},
  \emph{B}, and \emph{A}- are bets, where \emph{A} and \emph{B} are bets
  on propositions with incomparable credences, and \emph{A}- is a bet on
  a proposition that's a little less likely than \emph{A}.} Assume that
Evelyn has no beliefs at all at \emph{t}\textsubscript{1} about what
they will do at \emph{t}\textsubscript{2}. Then, on one standard way of
understanding what an imprecise credence is, Evelyn's credences about
the proposition \emph{Evelyn will take B at t\textsubscript{2}} are
imprecise. But Elga says that imprecise credences are irrational. So
Elga has to, by his own lights, require irrationality in the very setup
of the problem. So his argument can't show that the attitudes towards
\emph{A}, \emph{B}, and \emph{A}- in particular are irrational.

The version of the argument in Gustafsson
(\citeproc{ref-Gustafsson2025}{Forthcoming}) is not subject to this
criticism. He is careful to note that Evelyn's counterpart (in the
moneypump he puts forward) has numerical credences about what they will
do at later stages. He does insist that these credences are in (0, 1);
i.e., they are not extreme. We'll come back to this assumption below,
but for now we just want that this particular objection has no force
against his version of the moneypump argument.

The fourth argument is that the principle we're suggesting here unifies,
and explains, what's going on in the cases involving Dylan, Blake, and
Leslie. There should be some story about the losses (or at least
non-maximal gains) in those cases are not evidence of irrationality, and
ideally it should be a simple, common, story. We have such a story to
offer. The parables only work if the protagonist starts with firm
beliefs about what they will do at later stages, and those beliefs only
update by conditionalisation.\footnote{That is, the only updates involve
  moving between \emph{If I get to this choice point, I will take this
  option} to \emph{Now that I'm at this choice point, I will take this
  option}.}

The fifth argument is that only with this constraint added can players
do well in certain kinds of decision problems that, intuitively,
rational players can do well in. The following example is taken (with
very minor modifications) from Stalnaker
(\citeproc{ref-Stalnaker1999}{1999, 317}), who in turn takes it from
Piccione and Rubinstein (\citeproc{ref-PiccioneRubinstein1997}{1997}).

A fair coin will be flipped, and Casey will see the outcome. Regardless
of how it comes up, theycan choose to opt out or to opt in. If they opt
out, they get 2. If they opt in, they lose the memory of the coin flip,
and face a second choice: guessing the outcome of the coin. If they
guess correctly they get 6, otherwise 0.

There are two optimal strategies, each of which has an expected payout
of 4. On the first, Casey opts out if Heads and opts in if Tails,
choosing Tails at the next choice. On the second, Casey opts out if
Tails and opts in if Heads, choosing Heads at the next choice. We think
that the sort of character depicted in a moneypump should be able to
execute one of these strategies, and obtain the best expectation. If
Casey can form beliefs about what they will do, and only update those
beliefs by conditionalisation, then they will carry out one or other of
these strategies. If it is consistent with the kind of rationality
depicted in the moneypump stories that Casey could either (a) not form a
belief at the earlier time about what they will do at the later time, or
(b) change this belief for no reason, then it would also be consistent
with this kind of rationality that at the later time they would simply
have to guess how the coin landed. It would still be rational to opt in,
because this guess has an expected return of 3, but they could not get
the return of 4 from playing either of the optimal strategies. This
seems bad, and not consistent with the kind of rationality that we think
the stories should depict.

The sixth and final argument is one that we've alluded to above. At
\emph{t}\textsubscript{3}, after the second choice is made, Evelyn has a
belief about what they do at \emph{t}\textsubscript{2}. (We're using
this awkward tenseless formulation because we want to talk about their
attitude to the proposition before, during and after
\emph{t}\textsubscript{2}, and any more natural formulation would be
misleading about two of those three times.) How did they acquire that
belief? There are three logical options:

\begin{itemize}
\tightlist
\item
  It could have been a belief they had from \emph{t}\textsubscript{1}.
  That's the view we're defending.
\item
  It could have been brought about by some arbitrary change before
  \emph{t}\textsubscript{2}. Then Evelyn is in the same category as
  Blake: they have an arbitrary change of belief during the story. That
  implies (a) that they violate the conventions of the moneypump genre,
  and (b) that they are arguably independently irrational, so any losses
  can be chalked up to this irrationality rather than a defect in their
  preferences.
\item
  It could have been something they learned at
  \emph{t}\textsubscript{2}. On this picture, Evelyn does update by
  conditionalisation, but the new evidence is that they see what they do
  at \emph{t}\textsubscript{2}, and update accordingly. This also seems
  irrational to us. Rational people are not alienated from their own
  actions in this way; they do not take the observer stance towards
  themselves. ``Oh look, I chose B!'' they say, as one might notice the
  action or another, or the involuntary motion of a body part. Choosing
  \emph{A}- is meant to be an action, a sign of irrationality according
  to the proponent of the argument, not a mere movement. That's
  inconsistent with it being something that Evelyn doesn't know will
  happen until they see it.
\end{itemize}

The only option that seems reasonable here is the first one: for the
moneypump argument to work, and Evelyn to be rational in the right kind
of way, they have to have a belief at \emph{t}\textsubscript{1} about
what they will do at \emph{t}\textsubscript{2}. If that belief isn't
accurate, the moneypump argument fails for one reason or another.

In his most recent defence of moneypump arguments, Gustafsson
(\citeproc{ref-Gustafsson2025}{Forthcoming}) suggests a reason for
rejecting everything we've just said. He argues that if one has a
disposition to choose \emph{X} over \emph{Y} with probability 1, that is
sufficient for preferring \emph{X} to \emph{Y}. He then develops a much
more complicated moneypump that assumes that the protagonist at earlier
stages has some non-extremal probability about what they will do at
later stages when faced with a choice between incomparable options.

This way of setting up the moneypump argument has some nice advantages
over the one shown in Figure~\ref{fig-single-souring}. First, it takes
seriously the question of what attitude the protagonist has at earlier
times towards their actions at later times. Second, it makes that
attitude one that could be rational given the conclusion of the
moneypump. That is, it is a numerical credence, not an incomplete or
imprecise attitude. Third, in the resulting argument, it's not just that
the protagonist \emph{could} choose suboptimal options; the first move
at least in his parable is one that the protagonist is required to make
given the standards of rationality he defends.

But all of these moves require the initial assumption that the
protagonist cannot simultaneously have probability 1 that they will
choose \emph{B} over \emph{A}-, and not have a preference for \emph{B}
over \emph{A}-. We think there are several reasons for thinking these
attitudes are consistent. To start, note that the principle Gustafsson
uses would lead to fairly dramatic changes in how we think about
equilibria. In the centipede game in Figure~\ref{fig-centipede},
orthodox analysis would say there
\textless{}\emph{Rr},\emph{R}\textgreater{} is a subgame perfect
equilibrium, and indeed is a weakly Pareto dominant equilibrium.

\begin{figure}

\centering{

\usetikzlibrary{calc}

\tikzset{
% Two node styles for game trees: solid and hollow
solid node/.style={circle,draw,inner sep=1.5,fill=black},
hollow node/.style={circle,draw,inner sep=1.5},
square node/.style={rectangle,draw, inner sep = 1, fill = black}
}

\begin{tikzpicture}[]
  \node[hollow node, label=above:{P1}] {}
    child[grow=right] {
      node[solid node, label=above:{P2}] {}
      child[grow=right] {
        node[solid node, label=above:{P1}] {}
        child[grow=right] {
          node[square node, label=right:{3, 2}] {}
          edge from parent
          node[below] {r}
        }
        child[grow=down] {
          node[square node, label=below:{3, 0}] {}
          edge from parent
          node[left] {d}
        }
        edge from parent
        node[below] {R}
      }
      child[grow=down]  {
        node[square node, label=below:{0, 2}] {}
        edge from parent
        node[left] {D}
      }
      edge from parent
      node[below] {R}
    }
    child[grow=down]  {
      node[square node, label=below:{1, 0}] {}
      edge from parent
      node[left] {d}
    };
\end{tikzpicture}

}

\caption{\label{fig-centipede}A centipede game}

\end{figure}%

On Gustafsson's view, \textless{}\emph{Rr},\emph{R}\textgreater{} cannot
be an equilibrium of \emph{this} game. To be an equilibrium, it is
required that both players are confident that P1 will play \emph{r}. If
P2 does not believe this with probability 1, they are better off playing
\emph{D} than \emph{R}. And normally an equilibrium requires that
players have common beliefs. But according to Gustafsson, if P1 believes
they will play \emph{r} with probability 1, they cannot be indifferent
between the two outcomes on the right. That in turn is inconsistent with
the way the game is written, where P1 is assigned exactly the same
payout in those two options. We think the orthodox analysis of the game
here is correct; P1 can consistently be indifferent between the two
options on the right, and believe (along with P2) that they will play
\emph{r} rather than \emph{d} if the game gets that far.

Set that case aside though, and think about another character, Hunter,
with the following disposition. Whenever they have to bet on the flip of
a fair coin, they choose Heads. According to Gustafsson's way of setting
things up, Hunter prefers a bet on Heads to a bet on Tails. This seems
implausible to us; forming such a tie-breaking resolution seems like a
different thing than forming a preference.

There are two other reasons for denying that Hunter prefers betting on
Heads to betting on Tails. A standard principle about rationality, one
that is often used in deriving credences and utilities from betting
dispositions, is that one prefers a bet on \emph{p} to a bet on \emph{q}
(with the same stakes) iff one thinks \emph{p} is more probable than
\emph{q}. If Hunter genuinely prefers betting on Heads to betting on
Tails when the coin is fair, they violate this principle. Gustafsson
also suggests that preference should satisfy a continuity assumption -
whenever one strictly prefers \emph{X} to \emph{Y}, there should be some
third option \emph{Z} that one prefers to \emph{Y}, and is less
preferred than \emph{X}. If Hunter prefers Heads to Tails in virtue of
having this disposition, they will violate this continuity constraint.
If this is the \emph{only} such tie-breaking resolution they have, there
will be nothing which they prefer to Tails, and prefer less than Heads.
So even without getting into what happens with incomparable options, the
thought that a consistent choice disposition implies a preference has
enough unhappy consequences.

The view that characters like Evelyn should have firm beliefs about what
they will do at later stages of decision trees, and that they should
only update these beliefs by conditionalisation, suggests a different
analysis of what's going on in these cases to what is common in the
literature. It is often thought that if one wants to defend the
rationality of Evelyn's preferences in light of the fact that they end
up with \emph{A}-, one should find some way in which the choice of
\emph{A}- over \emph{B} at \emph{t}\textsubscript{2} is irrational. This
thought is common ground between philosophers like Chang
(\citeproc{ref-Chang2005}{2005}), who thinks Evelyn is rational, and
philosophers like Gustafsson (\citeproc{ref-Gustafsson2022}{2022}), who
thinks they are not. So there is a lot of debate about whether
rationality requires sticking to earlier plans, as Chang
suggests\footnote{Chang's later views on this are a bit more nuanced.
  See Chang (\citeproc{ref-Chang2017}{2017}) for an updated account, and
  Doody (\citeproc{ref-Doody2019}{2019}, sect 4) for a careful analysis
  of how Chang's views on the case have changed over time.}, or whether
the rationality of a choice at a time is independent of past choices, as
Gustafsson suggests. We think this debate is looking at the wrong spot.
There are three other places we could be looking.

It could be that Evelyn is irrational, but the irrationality is at
\emph{t}\textsubscript{1}. This would be the case if, for example,
Evelyn chose to go Down at \emph{t}\textsubscript{1} while believing
that they would take \emph{A}- at \emph{t}\textsubscript{2}.

A more radical possibility is that Evelyn is irrational over the period
from \emph{t}\textsubscript{1} to \emph{t}\textsubscript{2} without
being irrational at either time. We want to at least open the
possibility that changes like Blake's involve intertemporal
irrationality even if Blake is not irrational at either time. And it
seems consistent to say that the combination of choosing Down and
\emph{A}- is irrational even if neither single act is irrational.

But the last option is the one we most want to highlight. Dylan, Blake,
and Leslie all end up with less than they could have done. But this need
not be a sign of irrationality; it might be that they simply don't
satisfy the presuppositions of the moneypump argument. Once one makes
clear what attitude Evelyn does or does not have at
\emph{t}\textsubscript{1} towards the action that will be taken at
\emph{t}\textsubscript{2}, this presupposition failure might be made
clear.

These last three explanations are not necessarily in tension. It might
be that if Evelyn believes at \emph{t}\textsubscript{1} that \emph{A}-
will be chosen at \emph{t}\textsubscript{2}, that the irrational act is
at \emph{t}\textsubscript{1}, while if they do not believe this before
\emph{t}\textsubscript{1} but arbitrarily change their belief by
\emph{t}\textsubscript{2}, they are outside the scope of moneypump
arguments. Either way, one can reject the moneypump argument without
saying that Evelyn does something irrational at
\emph{t}\textsubscript{2}.

\section*{References}\label{references}
\addcontentsline{toc}{section}{References}

\phantomsection\label{refs}
\begin{CSLReferences}{1}{0}
\bibitem[\citeproctext]{ref-Bradley2005}
Bradley, Richard. 2005. {``Radical Probabilism and Bayesian
Conditioning.''} \emph{Philosophy of Science} 72 (2): 342--64. doi:
\href{https://doi.org/10.1086/432427}{10.1086/432427}.

\bibitem[\citeproctext]{ref-Chang1997}
Chang, Ruth. 1997. {``Introduction.''} In \emph{Incommensurability,
Incomparability and Practical Reason.}, edited by Ruth Chang, 1--34.
Cambridge, MA: Harvard University Press.

\bibitem[\citeproctext]{ref-Chang2005}
---------. 2005. {``Parity, Interval Value, and Choice.''} \emph{Ethics}
115 (2): 331--50. doi:
\href{https://doi.org/10.1086/426307}{10.1086/426307}.

\bibitem[\citeproctext]{ref-Chang2017}
---------. 2017. {``Hard Choices.''} \emph{Journal of the American
Philosophical Association} 3 (1): 1--21. doi:
\href{https://doi.org/10.1017/apa.2017.7}{10.1017/apa.2017.7}.

\bibitem[\citeproctext]{ref-Doody2019}
Doody, Ryan. 2019. {``Opaque Sweetening and Transitivity.''}
\emph{Australasian Journal of Philosophy} 97 (3): 579--91. doi:
\href{https://doi.org/10.1080/00048402.2018.1520269}{10.1080/00048402.2018.1520269}.

\bibitem[\citeproctext]{ref-Elga2010}
Elga, Adam. 2010. {``Subjective Probabilities Should Be Sharp.''}
\emph{Philosophers' Imprint} 10: 1--11.

\bibitem[\citeproctext]{ref-Gustafsson2025}
Gustafsson, Johan E. Forthcoming. {``A Behavioural Money-Pump Argument
for Completeness.''} \emph{Theory and Decision}, Forthcoming. doi:
\href{https://doi.org/10.1007/s11238-025-10025-3}{10.1007/s11238-025-10025-3}.

\bibitem[\citeproctext]{ref-Gustafsson2022}
---------. 2022. \emph{Money-Pump Arguments}. Cambridge: Cambridge
University Press.

\bibitem[\citeproctext]{ref-KopecTitelbaum2016}
Kopec, Matthew, and Michael G. Titelbaum. 2016. {``The Uniqueness
Thesis.''} \emph{Philosophy Compass} 11 (4): 189--200. doi:
\href{https://doi.org/10.1111/phc3.12318}{10.1111/phc3.12318}.

\bibitem[\citeproctext]{ref-Lewis1999b}
Lewis, David. 1999. {``Why Conditionalize?''} In \emph{Papers in
Metaphysics and Epistemology}, 403--7. Cambridge University Press.
Originally written as a course handout in 1972.

\bibitem[\citeproctext]{ref-PiccioneRubinstein1997}
Piccione, Michele, and Ariel Rubinstein. 1997. {``On the Interpretation
of Decision Problems with Imperfect Recall.''} \emph{Games and Economic
Behavior} 20 (1): 3--24. doi:
\url{https://doi.org/10.1006/game.1997.0536}.

\bibitem[\citeproctext]{ref-Stalnaker1999}
Stalnaker, Robert. 1999. {``Extensive and Strategic Forms: Games and
Models for Games.''} \emph{Research in Economics} 53 (3): 293--319. doi:
\href{https://doi.org/10.1006/reec.1999.0200}{10.1006/reec.1999.0200}.

\bibitem[\citeproctext]{ref-Woodard2022}
Woodard, Elise. 2022. {``A Puzzle about Fickleness.''} \emph{No{û}s}.
doi: \href{https://doi.org/10.1111/nous.12359}{10.1111/nous.12359}.

\end{CSLReferences}



\noindent Draft of April 2025.


\end{document}
