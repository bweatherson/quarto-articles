\documentclass[
  11pt,
  letterpaper]{article}

\usepackage[authoryear,round]{natbib}
\bibliographystyle{apalike}


\usepackage[unicode, hidelinks]{hyperref}
\usepackage{url}
\usepackage{tikz}
\usetikzlibrary{calc}


\title{Consistency and Resoluteness}
\author{Anon}
\date{2025}

\begin{document}
\maketitle
\begin{abstract}
When does a little story in which a character loses money, or at least comes out with less money than they might easily have had, reveal a defect in that character's rationality? We argue that these stories are less philosophically revealing than is often assumed. In particular, we argue the story can only be used in this kind of argument if the character has firm beliefs about what they will do later in the story. This constraint is, we argue, violated in recent arguments against imprecise credences and incomparable values.
\end{abstract}

One genre of parable has played a key role in philosophical theorizing about rationality. In some respects, the conventions of the genre are simple: A character is portrayed, presented with certain choices, and loses something of value, with no compensatory gain. The inevitable moral is then drawn: the protagonist displays a rational defect.

But in another respect, the generic conventions of the parable of the moneypump are complex. Not every case in which someone loses something of value without recompense illustrates a rational defect. Lightning might strike, or unjust, whimsical gods might deprive the protagonist of some good. If the parable is to be the basis for its intended moral, strict rules must be followed about who the protagonist is, and how their losses come about. 

The aim of this paper is to argue that, if a moneypump illustrates a rational defect, it must be compatible with its protagonist having beliefs at the start about what they will do if presented with any of the choices described in the parable, and updating these beliefs by conditionalization. We will show that this thesis rules out several prominent moneypumps: against imprecise credences \citep{Elga2010} and incomplete preferences \citep{Gustafsson2022, Gustafsson2025}. Our proposal provides an alternative to more familiar responses to such arguments. For instance, unlike ``resolute choice'', it does not require that an agent's choice of a plan at an earlier time give them reasons at a later time. It also does not require that a sequence of choices be evaluated as irrational when none of its constituent decisions are.

To illustrate the general idea that moneypump arguments must obey generic conventions about the characters they depict, we begin with three stories of losses without recompense which, we hope you'll agree, do not illustrate the irrationality of their protagonists.

\begin{quote} \textbf{Changing Preferences} Adrian owns $B$, but has a slight preference for $A$. So they trade $B$ plus $\varepsilon$ for $A$. The next day they wake up with the opposite preference, and trade $A$ plus $\varepsilon$ for $B$. At the end they are back where they started, minus 2$\varepsilon$. \end{quote}

\begin{quote} \textbf{Changing Beliefs} Blake has credence 0.6 in $p$. They pay 0.5 for a bet that pays 1 if $p$, and 0 otherwise. The next morning, they wake up and find, much to their surprise, that they now have credence 0.4 in $p$. They sell the bet for 0.5, netting a loss of 0.2. \end{quote}

\begin{quote} \textbf{Waiting to Resolve Uncertainty} Cameron is offered the option to bet on a fair coin toss before the coin is tossed, or to pay 1 to bet after it is tossed (and they have observed the outcome of the toss). The bet costs 5, and if they bet correctly, they will receive 8, while if they guess incorrectly they will receive nothing. Plausibly, Cameron should pay the 1 and then, once they have seen the outcome of the coin flip, place a ``bet'' on the correct outcome. But notice that, whatever way the coin lands, Cameron would have been better off making the correct bet in advance, without paying 1. \end{quote}

In each of these three cases, the protagonist loses something of value, with no compensatory gain. But none of the stories provides evidence for a verdict of (structural) irrationality. The story of Adrian clearly is not a basis for concluding that preference changes are irrational. Similarly (though this is more controversial) the story of Blake is not a basis for concluding that changes of belief are a defect of (structural) rationality.\footnote{Those who accept various uniqueness theses will think  that Blake is not perfectly rational; see \citet{KopecTitelbaum2016} for a survey of the issues here. But uniqueness is a theory of substantive rationality, so this is no challenge to our claim that the case doesn't impugn Blake's structrual rationality. A second way of challenging our claim that Blake doesn't exhibit a defect of structural rationality might be based on an argument from David \citet{Lewis1999} that not conditionalising will lead to a money pump. But as several authors have pointed out (e.g., \citet{Bradley2005}), Lewis's argument shows at most that any policy other than conditionalisation leads to a money pump. And Blake has no such policy; they just find themselves with new beliefs.} And, finally, the story of Cameron is not a basis for concluding that paying to resolve one's uncertainty is a defect of rationality in any sense; waiting is clearly rationally permitted in this case.

These cases show that stories involving sure loss relative to what a person could have had don't reveal irrationality on the part of the protagonist if the story involves certain changes of desire, or certain changes of belief, whether because of `fickle' change (as in Blake's case) \citep{Woodard2022} or because of learning (as in Cameron's). In other words, moneypump arguments are successful in general only if they are compatible with their protagonists having relevantly stable preferences, and relevantly stable beliefs. The constraint on such arguments that we will propose is in the same neighbourhood: that protagonists must have beliefs about what they will do at future nodes, and these beliefs must update by conditionalization. In our view, stories which are incompatible with the protagonist having such beliefs make no progress at all toward illustrating a rational defect, anymore than the stories of Adrian, Blake and Cameron do.

To illustrate our constraint at work, consider a canonical moneypump argument against incomplete preferences, often called the ``Single Souring Moneypump'' \citetext{\citealp{Chang1997intro}; \citealp[p.~26]{Gustafsson2022}}. The tree for this moneypump is shown in Figure~\ref{fig-single-souring}.

\begin{figure}

\centering{

\usetikzlibrary{calc}

\tikzset{
% Two node styles for game trees: solid and hollow
solid node/.style={circle,draw,inner sep=1.5,fill=black},
hollow node/.style={circle,draw,inner sep=1.5},
square node/.style={rectangle,draw, inner sep = 1, fill = black}
}

% Specify spacing for each level of the tree
%\tikzstyle{level 1}=[level distance=12mm,sibling distance=15mm]
%\tikzstyle{level 2}=[level distance=15mm,sibling distance=15mm]
%\tikzstyle{level 3}=[level distance=13mm,sibling distance=11mm]


\begin{tikzpicture}[grow=right, sloped]
\node[hollow node, draw] {}
    child {
        node[solid node, draw] {}
        child {
            node[square node, fill, inner sep=1.5pt, label=right:{\emph{B}}] {}
            edge from parent
            node[above] {}
        }
        child {
            node[square node, fill, inner sep=1.5pt, label=right:{\emph{A}-}] {}
            edge from parent
            node[below] {}
        }
        edge from parent
        node[below] {}
    }
        child {
        node[square node, fill, inner sep=1.5pt, label=right:{\emph{A}}] {}
        edge from parent
        node[above] {}
    };
\end{tikzpicture}


}

\caption{\label{fig-single-souring}The Single Souring Moneypump}

\end{figure}

The agent, call him Dylan, is presented at $t_1$ with a choice between taking Up, which results in $A$, or Down, which results in a choice (at $t_2$) between option $B$ and $A$-. Dylan is assumed to prefer $A$ to $A$-, but to have no relevant preferential relations between either $A$ and $B$ or $A$- and $B$ (that is, he is not indifferent between the elements of these pairs, and does not prefer one to the other). (We will also assume for now that Dylan only has pure strategies available, and cannot choose mixed strategies; we return to this below.) Plausibly, Dylan is rationally permitted, at $t_1$, to take either option: he can take Up and get $A$ or take Down and face the choice between $A$- and $B$. The problem is what happens if he goes Down. For in that case, it would seem that he is also permitted at $t_2$ to take $A$- (since he doesn't disprefer it to $B$). But this is a loss relative to what they could have had at no cost, i.e.~$A$.\footnote{This is a ``non-forcing'' moneypump. For the distinction between forcing and non-forcing, see \citet[761-2]{GustafssonEspinoza2010} and \citet[27]{Gustafsson2022}.}

But let us suppose that our constraint holds: that Dylan has a belief at the start about what he will do if faced with any of the relevant choices. A typical person in this position who made the decision to go Down, rather than taking the sure thing of $A$, would do so believing that they would go on to choose $B$, not $A$-. Indeed, as we will argue below (and we think is anyway intuitively plausible), if Dylan makes the choice to go Down in this case \emph{without} the belief that they will later choose $B$, he would be doing something irrational. But if Dylan believes, when he chooses Down, that he will go on to choose $B$, but he nevertheless goes on to choose $A$-, this choice would either (a) require a change in his belief about what he would do other than updating by conditionalization, or (b) require that Dylan choose $A$- while still believing that he would choose $B$. The second of these options, (b), quickly leads to inconsistency given natural assumptions about the sort of person for whom a moneypump dramatizes a rational defect. Provided Dylan knows that he's in fact faced with the choice between $A$- and $B$, and provided he updates hisconditional belief that, if he is faced with the choice, he will choose $B$, then at the moment he makes the choice, he will believe that he is choosing $B$. Provided he also believes that he is choosing $A$- (which is after all what he is doing), and he knows that these are exclusive options, Dylan's beliefs would be inconsistent, and hence irrational. So, if a moneypump dramatizes incoherence only if it is compatible with the protagonist's beliefs about what they will do being held fixed (ruling out (a) as an option in the above story), and we can furthermore motivate our rational requirement that Dylan will go Down only if he believes that they will later choose $B$, this moneypump could not dramatize the incoherence of incomplete preferences.

We have two main arguments for the claim that a successful moneypump must be compatibility with the protagonist having beliefs about what they would do at every node.

The first is that this principle unifies and explains our earlier cases involving Adrian (Preference Change), Blake (Belief Change), and Cameron (Waiting to Resolve Uncertainty). There should be some story about why the losses (or at least non-maximal gains) in those cases are not evidence of irrationality, and ideally it should be a simple, common, story. We have such a story to offer. The parables only serve as the basis for this conclusion if the protagonist starts with beliefs about what they will do at later stages, and those beliefs only update by conditionalization, that is, moving from \emph{If I get to this choice point, I will take this option} to \emph{Now that I'm at this choice point, I will take this option}. Adrian (Preference Change) violates this constraint since they do not foresee their change of preference, and thus believes that, if offered a choice at the later time, they will stick with $A$, rather than paying $\varepsilon$ to switch to $B$. Blake (Belief Change) violates this constraint since they do not foresee their change of credence, and thus predicts that, when offered the choice of a bet at the later time, they will stick with their original choice as well. Cameron (Waiting to Resolve Uncertainty) violates the constraint because their beliefs about what they will do are sensitive to the outcome of the coin, not just to which choice they are facing at which time. Our principle correctly deems all three cases out of bounds for the application of a moneypump argument, and explains why they do not dramatize irrationality.

Our second argument is that only with this constraint added can players achieve their maximal expectation in certain kinds of decision problems that, intuitively, rational players should be able to do well in. The following example is taken (with very minor modifications) from \citet{Stalnaker1999}, who in turn takes it from \citet{PiccioneRubinstein1997}.

A fair coin will be flipped, and Ellis will see the outcome. Regardless of how the coin comes up, Ellis can choose to opt out or to opt in. If they opt out, they get 2. If they opt in, their memory of how the coin landed will be erased, and they will face a second choice: guessing the outcome of the coin. If they guess correctly, they get 6, otherwise 0.

There are two optimal strategies in this decision problem, each of which has an expected payout of 4. On the first, Ellis opts out if Heads and opts in if Tails, choosing Tails if they are offered the second choice. On the second, Ellis opts out if Tails and opts in if Heads, choosing Heads if they are offered the second choice. We think that the sort of character depicted in a moneypump should be able to execute one of these strategies, and obtain the optimal expectation. If Ellis can form beliefs about what they will do, and update those beliefs by conditionalization, then they can carry out one or other of these strategies. But if it were consistent with the conventions of a moneypump story that Ellis could either (a) not have a belief at the earlier time about what they would do if presented with further choices at the later time, or (b) change this belief other than via conditionalization, then it would also be allowed that, if Ellis opted in, they would simply have to guess how the coin landed. The best Ellis could do would be to opt in regardless of what happened, and then guess at the second decision node. This strategy has an expected return of 3, by contrast to the expected return of 4 they could have by playing either of the optimal strategies. Our claim about the generic conventions explains why there is no moneypump based on this case against classical expected utility theory: the putative ``moneypump'' is out of bounds.

We conclude the defense of our main thesis, by noting that the position we are defending is the most mundane orthodoxy in decision theory and game theory. We invite you to open any game theory textbook and look at how it treats decisions over time. All of the theories put forward will assume, usually without comment, that each player has firm beliefs about what strategy they have adopted, and so how they will act at later times. These ``actions'' might include mixed strategies, so the player might not know what moves they will make. But it's an unchallenged assumption that rational action in a dynamic game requires having a belief about what one's later self will do. Treating one's later self as an alien entity, or as something like the weather that might change at random, is almost unheard of in economics. So, while many philosophers have (implicitly) rejected our view, this is not reflective of the prevailing orthodoxy among those developing formal theories of decision-making.\footnote{This is not to say we are the only philosophers to take this line. A similar line is taken by Stalnaker: "[W]e would have an impoverished conception of a rational agent if we did not allow for intentions and plans ...the ideally rational agents that we are modelling in decision theory think of their future choices as things that call for decision."  \citep[304]{Stalnaker1999} The contrast he is drawing is with the common view that future choices call for prediction, and he understands making a decision to be a route to belief about what one will do.}

A more demanding position than the one we have defended might hold that a moneypump must be consistent with the decision-maker's having knowledge of what they would do, rather than mere belief.\footnote{Citation removed for anonymous review.} On this approach, if we suppose that if Dylan goes Down, he does so in the knowledge that he will choose $B$, we can directly derive the result that they choose $B$ (because knowledge is factive) without any assumptions about consistency. This view entails ours, so we will not argue against it here; we see proponents of it as fellow-travelers, and have simply defended the weaker version of the thesis for simplicity. 

We now turn to working out applications of our thesis to specific moneypump arguments.

We already began above to discuss how our claim allows proponents of incomplete preferences to offer a response to the Single-Souring Moneypump. In that earlier discussion, however, we left one important loose end. We did not provide a full-throated defense of the claim that it is a rational requirement that if Dylan has beliefs about what he will do at future nodes, he will choose to go Down only if he believes that he will choose $B$.\footnote{Note that even if the alternative position, which uses knowledge rather than belief is accepted, it plausibly would still suffice to argue for this rational requirement, since if this belief is required, then the person could plausibly only have knowledge in line with this belief.} It is clear that Dylan should not do so with a belief that he will later choose $A$-: in that case, he would in effect be choosing $A$- over $A$. But we need to say just a bit more about ways in which Dylan might neither believe that he will choose $A$-, nor believe that he will choose $B$.

Two such ways are already ruled out if the earlier arguments are correct. One is that Dylan has no attitude at all about what he will later do. The other is that he has some non-extreme probability about what he will do, but regard his future decision as something like the weather that he can at best have an informed guess about. But there is, perhaps, a third option we haven't yet ruled out. He might be playing a mixed strategy.

Just what mixed strategies are, and when they are rationally permissible, is a notoriously difficult question. We're hardly going to resolve this here, but we do need to lay out some of the options, for it turns out that what our argument says about Dylan and mixed strategies varies depending on just how mixed strategies are understood.

For theorists who think mixed strategies are never permissible, there is no further rational option for Dylan, so the argument is complete.

Our preferred way of understanding mixed strategies is in terms of beliefs that parties have about what a player will do. What it means to say that a player's strategy is (for example) a 60/40 mix of Up and Down at a node means that other players have credence 0.6 that, were that node reached, the player would go Up or respectively. \citep{Aumann1987}. If that's all a mixed strategy is, then since there are no other players here, again there isn't a missing option for Dylan we haven't covered.

One might, in a similar spirit, say that a mixed strategy requires that all players, including Dylan himself, have this credence.\footnote{\citet{Williams2014} distinguishes between choosing randomly, and choosing to randomise. We mean this paragraph to be about the case where Dylan chooses randomly at $t_2$, and what follows to be about the case where he chooses to randomise.} If this is all that a mixed strategy is, then again we're back in the situation where Dylan lacks a firm prediction of his future decisions, and this is something we take the earlier arguments to rule out.

But what our earlier argument does not rule out is where Dylan adopts a mixed strategy where this is understood in a more ``objective'' way. Say that to play a mixed strategy just is to use some external randomizing device.\footnote{In the terminology from Williams we mentioned in the last footnote, on this view playing a mixed strategy is choosing to randomizing, not just randomly choosing.} Now in some sense Dylan does have a firm belief at $t_1$ about what he'll do at $t_2$: he'll get out his coin, or dice, or spinner, and see what they tell him to do.

If this is what Dylan is doing, we think it's irrational for a more familiar reason. Assume that at $t_2$ he sets the randomizing device to choose $A$- with probability $p$, and at $t_1$ he believes that is what he'll do. Then he would be better off randomizing at $t_1$, choosing $A$ with probability $p$, and Down followed by $B$ with probability $1-p$. This alternative stochastically dominates the alternative of randomizing at $t_2$.

There are two natural objections to our argument here. The first is to the premise that it is a rational requirement not to choose stochastically dominated options.\footnote{For discussion of this premise see, for example, \citet{hare2010take,bader2018stochastic,ledermanmarbles,tarsneyexpected}} We are committed to this premise. But we acknowledge that some fans of incomplete preferences reject the claim that it is rationally prohibited to choose stochastically dominated options.\footnote{ \citet{bales2014decision,schoenfield2014decision}}. It is well-known that fans of such views must reconcile themselves to something quite close to moneypumps anyway, so it is not surprising that they face one here. Our own position is that the moneypump argument is successful against this particular view about rational permissibility for incomplete preferences, but that this is a problem with their view, not with incomplete preferences.

The other objection is to the claim that Dylan can do something at $t_1$, e.g., flip a coin or roll some dice, which will determine what he does at $t_2$. Why would the result of the earlier coin flip give him a reason to choose $B$ rather than $A$- at $t_2$? To this objection, we say that this ability to have a randomizing device at one time settle what one does at another time is essential to the understanding of mixed strategies the objector is working with.

Let's assume, contra everything else in this paper, that something like the approach to decision theory in \citet{vNM1944} is correct, and agents have precise numerical credences and utilities. Moreover, assume they can rationally randomize when carrying out mixed strategies. If a player rationally chooses to randomize, then (given these background assumptions) they must be indifferent between the actions over which they randomize. But if they are, then when they see the results from their randomizing device, it is unclear why they should ``follow their plan'' as opposed to choosing any of the other actions over which they were randomizing (since, by hypothesis, they were indifferent between these actions). In fact, it is not even clear why (if randomization is costless, as is standardly assumed) they should not randomize again. There is always a temporal gap between randomizing and acting, so if Dylan can randomize before $t_2$ to settle what to do at $t_2$, he can just as easily randomize before $t_1$ to settle what to do at $t_2$. Note that we don't say any of this randomization is in fact rational; we're not taking a stand on whether it is. All we are arguing is that if it is ever rational, there can be, indeed must be, some time pass between the randomization and the action.

This completes our argument for the rational requirement to choose Down only if one believes that one will choose $B$ (even if one chooses Down as the result of randomization). We note that this response to the Single-Souring Moneypump, which relies on the combination of our generic constraint on moneypumps, with the specific claim that, for such agents it is a rational requirement to go Down only if they believe they will choose $B$ at the later node, extends straightforwardly to provide a response to the \emph{forcing} moneypump for incomplete preferences due to \citet[p. 34-9]{Gustafsson2022}.\footnote{Gustafsson provides a list of principles which he claims entail the irrationality of incomplete preferences. Our position satisfies all of his premises jointly and thus shows that the argument he presents is invalid. Gustafsson operates under the assumption that specifying an agent's preferences in a decision-tree is enough to determine what actions are and are not rationally permissible in the decision tree. But this neglects that beliefs also matter. There is no fact about rational action in a decision tree, considered in abstraction from the further descriptive fact of what beliefs the agent has. In a choice between $A$ and $B$, for instance, Dylan cannot choose $A$ rationally if he also believes that he will chose $B$ (and know that these are exclusive options etc.).}

We turn next to the influential argument of \citet{Elga2010}. Elga asks us to consider someone, let's call her Finley, who has maximally imprecise credences on some proposition $H$. Elga will offer Finley two bets, one after the other:

\begin{quote}\textbf{Bet A} If $H$ is true, she loses \$10. If $H$ is false, she wins \$15.\end{quote}

\begin{quote}\textbf{Bet B} If $H$ is true, she wins \$15. If $H$ is false, she loses \$10.\end{quote}

Elga holds that it is rationally required to take at least one of these bets, and he argues that fans of imprecise credences cannot deliver this verdict. But his arguments do not succeed if our generic constraint is accepted, and if, assuming that Finley is a person who has firm beliefs about what she will do later, it is rationally required that, if Finley declines Bet A, she must do so believing that she will accept Bet B. Why should we accept that conditional though?

Our argument for it relies on two simplifications, both of which seem reasonable in the circumstances. First, since Elga's argument does not rely on mixed strategies, we'll also set them aside. In any case, we would handle them much the same way as we did in the Single Souring Moneypump. Second, we're not going to show that every theory for decision making with imprecise credences can avoid Elga's argument. To show that imprecise credences can be defended, we just need to show that at least one such decision theory avoids the trap he sets. We'll use the theory that \citet{weatherson2008decision} calls Caprice.

According to Caprice, some decisions are rational for an agent whose credences are represented by a set $R$ of credence functions iff there is some $\Pr \in R$ such that each decision is utility maximising according to $\Pr$. We take Caprice to imply the following principle: \textbf{Don't Go Irrational}.\footnote{We're not committed to the claim that Caprice actually implies \textbf{Don't Go Irrational}; if it does not, take the decision theory we're using to be Caprice plus the principles.}

\begin{quote}
    \textbf{Don't Go Irrational} If the agent believes that they will make some choices, and these choices are (collectively) rational according to Caprice, then it is irrational to make any further choice such the collection of it plus the further choices the agent believes they will make is irrational according to Caprice.

\end{quote}

Intuitively, \textbf{Don't Go Irrational} says that an individual choice that takes the agent from believing that they are doing some things that are (in fact) rational, to believing that they are doing some things that are (in fact) irrational, is itself irrational. It's a way of moving from the irrationality of a collection of actions to the irrationality of the marginal action that generates the irrationality.

Caprice is incompatible with Finley declining both bets. There is no possible $\Pr$, and hence no $\Pr$ in the set representing her imprecise credence, according to which both declines are utility maximising. So if Finley believes that she will decline the second bet, then by \textbf{Don't Go Irrational} it is irrational to decline the first bet.

Finally, we turn to Gustafsson's most recent moneypump argument against incomplete preferences \citep{Gustafsson2025}, which employs quite different considerations than we've discussed so far. As a premise in this argument, Gustafsson requires that if a person has a disposition to choose $X$ over $Y$ with probability 1, they count as ``effectively'' preferring $X$ to $Y$. He develops a (forcing) moneypump argument, which requires that the protagonist at earlier stages has some non-extremal probability about what they will do at later stages, an assumption that is justified by his claim about extremal probability ``effectively'' amounting to a preference.

But Gustafsson's claim that a disposition to choose with certainty amounts to a preference should be rejected. We will run through some reasons why. 

Consider Glenn, who has the disposition that, whenever they have to bet on the flip of a fair coin, they choose Heads. According to Gustafsson's way of setting things up, Glenn ``effectively'' prefers a bet on Heads to a bet on Tails.

But there are at least three reasons for denying that Glenn prefers betting on Heads to betting on Tails. First, it just seems implausible; forming such a tie-breaking resolution is different from forming a preference. Second, a standard principle about rationality, one that is often used in deriving credences and utilities from betting dispositions, is that one prefers a bet on $p$ to a bet on $q$ (with the same stakes) if and only if one thinks $p$ is more probable than $q$. If Glenn genuinely prefers betting on Heads to betting on Tails when the coin is fair, they violate this standard principle. Third, Gustafsson also suggests that preference should satisfy a continuity assumption---whenever one strictly prefers $X$ to $Y$, there should be some third option $Z$ that one prefers to $Y$, and is less preferred than $X$. But again, if Glenn prefers Heads to Tails in virtue of having the disposition specified above, they will violate this continuity constraint. If this is the \emph{only} such tie-breaking resolution they have, there will be nothing which they prefer to Tails, and prefer less than Heads. 

Separate from the case of Glenn, we note finally that Gustafsson's premise would also lead to fairly dramatic changes in how we think about equilibria. In the centipede game in Figure~\ref{fig-centipede}, orthodox analysis would say that $\langle Rr, R \rangle$ is a subgame perfect equilibrium, and indeed is a weakly Pareto dominant equilibrium.

\begin{figure}

\centering{

\usetikzlibrary{calc}

\tikzset{
% Two node styles for game trees: solid and hollow
solid node/.style={circle,draw,inner sep=1.5,fill=black},
hollow node/.style={circle,draw,inner sep=1.5},
square node/.style={rectangle,draw, inner sep = 1, fill = black}
}

\begin{tikzpicture}[]
  \node[hollow node, label=above:{P1}] {}
    child[grow=right] {
      node[solid node, label=above:{P2}] {}
      child[grow=right] {
        node[solid node, label=above:{P1}] {}
        child[grow=right] {
          node[square node, label=right:{3, 2}] {}
          edge from parent
          node[below] {r}
        }
        child[grow=down] {
          node[square node, label=below:{3, 0}] {}
          edge from parent
          node[left] {d}
        }
        edge from parent
        node[below] {R}
      }
      child[grow=down]  {
        node[square node, label=below:{0, 2}] {}
        edge from parent
        node[left] {D}
      }
      edge from parent
      node[below] {R}
    }
    child[grow=down]  {
      node[square node, label=below:{1, 0}] {}
      edge from parent
      node[left] {d}
    };
\end{tikzpicture}

}

\caption{\label{fig-centipede}A centipede game}

\end{figure}%

On Gustafsson's view, $\langle Rr, R \rangle$ cannot be an equilibrium of \emph{this} game. To be an equilibrium, it is required that both players are confident that P1 will play $r$. If P2 does not believe this with probability 1, they are better off playing $D$ than $R$. And normally an equilibrium requires that players have common beliefs. But according to Gustafsson, if P1 believes they will play $r$ with probability 1, they cannot be indifferent between the two outcomes on the right. That in turn is inconsistent with the way the game is written, where P1 is assigned exactly the same payout in those two options. We think the orthodox analysis of the game here is correct; P1 can consistently be indifferent between the two options on the right, and believe (along with P2) that they will play $r$ rather than $d$ if the game gets that far.

So we reject Gustafsson's extra premise. It is consistent, as we have suggested, that people have beliefs about what they will do, which do not amount to preferences between future choices.

The view of this paper is importantly different from standard responses to moneypump arguments. It is often thought that if one wants to defend the rationality of Dylan's preferences in light of the fact that he ends up with $A$-, one should find some way in which the choice of $A$- over $B$ at $t_2$ is irrational considered on its own (and irrespective of what beliefs the person has about what they will choose). This thought is common ground between philosophers like \citet{Chang2005}, who think Dylan is rational, and philosophers like \citet{Gustafsson2022}, who think he is not. So there is a lot of debate about whether rationality requires sticking to earlier plans, as Chang suggests\footnote{Chang's later views on this are a bit more nuanced. See \citet{Chang2017} and \citet[sect. 4]{Doody2019} for a careful analysis of how Chang's views on the case have changed over time.}, or whether the rationality of a choice at a time is independent of past choices, as Gustafsson suggests. For instance, both \citet[p. 8-9]{Elga2010} and \citet[p. 66-74]{Gustafsson2022} argue against fans of ``Resolute Choice'' that making a plan at an earlier time does not provide a person with a practical reason in favor of one action rather than the other at a later time.\footnote{On resolute choice in decision theory, see \citet{McClennen1990}, though there much of the focus is on preference change. In action theory, a similar theory is defended by \citet{Bratman1987} and \citet{Holton2009}. The worry that such views require that plans be reasons has been widely discussed; see \citet[Sect. 4]{sep-intention}.} We think this debate is looking in the wrong place. There are three other places we could be looking.

It could be that Dylan is irrational, but the irrationality is at $t_1$. This would be the case if, for example, Dylan chose to go Down at $t_1$ while believing that he would take $A$- at $t_2$.

A second possibility, which we haven't discussed to this point, is more radical possibility. This is that Dylan is irrational over the period from $t_1$ to $t_2$ without being irrational at either time (\citet[p. 12]{weatherson2008decision}). We want to at least open the possibility that changes like Blake's involve intertemporal irrationality even if Blake is not irrational at either time. And it seems consistent to say that the combination of choosing Down and $A$- is irrational even if neither single act is irrational.

But the last option is the one we have been most focused on throughout the paper. Adrian, Blake, Cameron and Dylan may all end up with less than they could have had. But this need not be a sign of irrationality; it might be that they simply don't satisfy the presuppositions of the moneypump argument. Once one makes clear what attitude Dylan does or does not have at $t_1$ towards the action that will be taken at $t_2$, this presupposition failure becomes clear. In particular, if we do assume that Dylan believes at $t_1$ he will choose $B$ at $t_2$, consistency of Dylan's beliefs can do the work that Resolute Choice was supposed to do, providing a rational constraint which is not a practical reason in favor of one of the actions.

These three alternative explanations are not necessarily in tension. It might be that if Dylan believes at $t_1$ that $A$- will be chosen at $t_2$, that the irrational act is at $t_1$, while if he does not believe this before $t_1$ but arbitrarily changes his belief by $t_2$, he is outside the scope of moneypump arguments. Either way, one can reject the moneypump argument without saying that Dylan does something irrational at $t_2$.


\bibliography{moneypump}

\end{document}
