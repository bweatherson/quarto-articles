% Options for packages loaded elsewhere
% Options for packages loaded elsewhere
\PassOptionsToPackage{unicode}{hyperref}
\PassOptionsToPackage{hyphens}{url}
%
\documentclass[
  11pt,
  letterpaper,
  DIV=11,
  numbers=noendperiod,
  twoside]{scrartcl}
\usepackage{xcolor}
\usepackage[left=1.1in, right=1in, top=0.8in, bottom=0.8in,
paperheight=9.5in, paperwidth=7in, includemp=TRUE, marginparwidth=0in,
marginparsep=0in]{geometry}
\usepackage{amsmath,amssymb}
\setcounter{secnumdepth}{3}
\usepackage{iftex}
\ifPDFTeX
  \usepackage[T1]{fontenc}
  \usepackage[utf8]{inputenc}
  \usepackage{textcomp} % provide euro and other symbols
\else % if luatex or xetex
  \usepackage{unicode-math} % this also loads fontspec
  \defaultfontfeatures{Scale=MatchLowercase}
  \defaultfontfeatures[\rmfamily]{Ligatures=TeX,Scale=1}
\fi
\usepackage{lmodern}
\ifPDFTeX\else
  % xetex/luatex font selection
  \setmainfont[ItalicFont=EB Garamond Italic,BoldFont=EB Garamond
Bold]{EB Garamond Math}
  \setsansfont[]{EB Garamond}
  \setmathfont[]{Garamond-Math}
\fi
% Use upquote if available, for straight quotes in verbatim environments
\IfFileExists{upquote.sty}{\usepackage{upquote}}{}
\IfFileExists{microtype.sty}{% use microtype if available
  \usepackage[]{microtype}
  \UseMicrotypeSet[protrusion]{basicmath} % disable protrusion for tt fonts
}{}
\usepackage{setspace}
% Make \paragraph and \subparagraph free-standing
\makeatletter
\ifx\paragraph\undefined\else
  \let\oldparagraph\paragraph
  \renewcommand{\paragraph}{
    \@ifstar
      \xxxParagraphStar
      \xxxParagraphNoStar
  }
  \newcommand{\xxxParagraphStar}[1]{\oldparagraph*{#1}\mbox{}}
  \newcommand{\xxxParagraphNoStar}[1]{\oldparagraph{#1}\mbox{}}
\fi
\ifx\subparagraph\undefined\else
  \let\oldsubparagraph\subparagraph
  \renewcommand{\subparagraph}{
    \@ifstar
      \xxxSubParagraphStar
      \xxxSubParagraphNoStar
  }
  \newcommand{\xxxSubParagraphStar}[1]{\oldsubparagraph*{#1}\mbox{}}
  \newcommand{\xxxSubParagraphNoStar}[1]{\oldsubparagraph{#1}\mbox{}}
\fi
\makeatother


\usepackage{longtable,booktabs,array}
\usepackage{calc} % for calculating minipage widths
% Correct order of tables after \paragraph or \subparagraph
\usepackage{etoolbox}
\makeatletter
\patchcmd\longtable{\par}{\if@noskipsec\mbox{}\fi\par}{}{}
\makeatother
% Allow footnotes in longtable head/foot
\IfFileExists{footnotehyper.sty}{\usepackage{footnotehyper}}{\usepackage{footnote}}
\makesavenoteenv{longtable}
\usepackage{graphicx}
\makeatletter
\newsavebox\pandoc@box
\newcommand*\pandocbounded[1]{% scales image to fit in text height/width
  \sbox\pandoc@box{#1}%
  \Gscale@div\@tempa{\textheight}{\dimexpr\ht\pandoc@box+\dp\pandoc@box\relax}%
  \Gscale@div\@tempb{\linewidth}{\wd\pandoc@box}%
  \ifdim\@tempb\p@<\@tempa\p@\let\@tempa\@tempb\fi% select the smaller of both
  \ifdim\@tempa\p@<\p@\scalebox{\@tempa}{\usebox\pandoc@box}%
  \else\usebox{\pandoc@box}%
  \fi%
}
% Set default figure placement to htbp
\def\fps@figure{htbp}
\makeatother


% definitions for citeproc citations
\NewDocumentCommand\citeproctext{}{}
\NewDocumentCommand\citeproc{mm}{%
  \begingroup\def\citeproctext{#2}\cite{#1}\endgroup}
\makeatletter
 % allow citations to break across lines
 \let\@cite@ofmt\@firstofone
 % avoid brackets around text for \cite:
 \def\@biblabel#1{}
 \def\@cite#1#2{{#1\if@tempswa , #2\fi}}
\makeatother
\newlength{\cslhangindent}
\setlength{\cslhangindent}{1.5em}
\newlength{\csllabelwidth}
\setlength{\csllabelwidth}{3em}
\newenvironment{CSLReferences}[2] % #1 hanging-indent, #2 entry-spacing
 {\begin{list}{}{%
  \setlength{\itemindent}{0pt}
  \setlength{\leftmargin}{0pt}
  \setlength{\parsep}{0pt}
  % turn on hanging indent if param 1 is 1
  \ifodd #1
   \setlength{\leftmargin}{\cslhangindent}
   \setlength{\itemindent}{-1\cslhangindent}
  \fi
  % set entry spacing
  \setlength{\itemsep}{#2\baselineskip}}}
 {\end{list}}
\usepackage{calc}
\newcommand{\CSLBlock}[1]{\hfill\break\parbox[t]{\linewidth}{\strut\ignorespaces#1\strut}}
\newcommand{\CSLLeftMargin}[1]{\parbox[t]{\csllabelwidth}{\strut#1\strut}}
\newcommand{\CSLRightInline}[1]{\parbox[t]{\linewidth - \csllabelwidth}{\strut#1\strut}}
\newcommand{\CSLIndent}[1]{\hspace{\cslhangindent}#1}



\setlength{\emergencystretch}{3em} % prevent overfull lines

\providecommand{\tightlist}{%
  \setlength{\itemsep}{0pt}\setlength{\parskip}{0pt}}



 


\setlength\heavyrulewidth{0ex}
\setlength\lightrulewidth{0ex}
\usepackage[automark]{scrlayer-scrpage}
\clearpairofpagestyles
\cehead{
  Brian Weatherson
  }
\cohead{
  Reply to Eaton and Pickavance
  }
\ohead{\bfseries \pagemark}
\cfoot{}
\makeatletter
\newcommand*\NoIndentAfterEnv[1]{%
  \AfterEndEnvironment{#1}{\par\@afterindentfalse\@afterheading}}
\makeatother
\NoIndentAfterEnv{itemize}
\NoIndentAfterEnv{enumerate}
\NoIndentAfterEnv{description}
\NoIndentAfterEnv{quote}
\NoIndentAfterEnv{equation}
\NoIndentAfterEnv{longtable}
\NoIndentAfterEnv{abstract}
\renewenvironment{abstract}
 {\vspace{-1.25cm}
 \quotation\small\noindent\emph{Abstract}:}
 {\endquotation}
\newfontfamily\tfont{EB Garamond}
\addtokomafont{disposition}{\rmfamily}
\addtokomafont{title}{\normalfont\itshape}
\let\footnoterule\relax
\KOMAoption{captions}{tableheading}
\makeatletter
\@ifpackageloaded{caption}{}{\usepackage{caption}}
\AtBeginDocument{%
\ifdefined\contentsname
  \renewcommand*\contentsname{Table of contents}
\else
  \newcommand\contentsname{Table of contents}
\fi
\ifdefined\listfigurename
  \renewcommand*\listfigurename{List of Figures}
\else
  \newcommand\listfigurename{List of Figures}
\fi
\ifdefined\listtablename
  \renewcommand*\listtablename{List of Tables}
\else
  \newcommand\listtablename{List of Tables}
\fi
\ifdefined\figurename
  \renewcommand*\figurename{Figure}
\else
  \newcommand\figurename{Figure}
\fi
\ifdefined\tablename
  \renewcommand*\tablename{Table}
\else
  \newcommand\tablename{Table}
\fi
}
\@ifpackageloaded{float}{}{\usepackage{float}}
\floatstyle{ruled}
\@ifundefined{c@chapter}{\newfloat{codelisting}{h}{lop}}{\newfloat{codelisting}{h}{lop}[chapter]}
\floatname{codelisting}{Listing}
\newcommand*\listoflistings{\listof{codelisting}{List of Listings}}
\makeatother
\makeatletter
\makeatother
\makeatletter
\@ifpackageloaded{caption}{}{\usepackage{caption}}
\@ifpackageloaded{subcaption}{}{\usepackage{subcaption}}
\makeatother
\usepackage{bookmark}
\IfFileExists{xurl.sty}{\usepackage{xurl}}{} % add URL line breaks if available
\urlstyle{same}
\hypersetup{
  pdftitle={Reply to Eaton and Pickavance},
  pdfauthor={Brian Weatherson},
  hidelinks,
  pdfcreator={LaTeX via pandoc}}


\title{Reply to Eaton and Pickavance}
\author{Brian Weatherson}
\date{2016}
\begin{document}
\maketitle
\begin{abstract}
Daniel Eaton and Timothy Pickavance argued that interest-relative
invariantism has a surprising and interesting consequence. They take
this consequence to be so implausible that it refutes interest-relative
invariantism. But in fact it is a consequence that any theory of
knowledge that has the resources to explain familiar puzzles (such as
Gettier cases) must have.
\end{abstract}


\setstretch{1.1}
Daniel Eaton and Timothy Pickavance note an interesting consequence of
some popular versions of interest-relative invariantism (IRI). They take
this consequence to be a \emph{reductio} of (those versions of) IRI.
I'll agree it is an interesting consequence, but argue that it cannot be
a \emph{reductio}. They take as a fixed point the following principle:

\begin{quote}
We take it to be rather obvious that one ought not be able to go from
not knowing that \emph{p} to knowing that \emph{p} by getting evidence
\emph{against} \emph{p}, nor should one be able to go from knowing that
\emph{p} to not knowing that \emph{p} by getting evidence \emph{for}
\emph{p}. ~(\citeproc{ref-EatonPickavance2015}{Eaton and Pickavance
2015, 3142}, emphasis in original)
\end{quote}

The interesting consequence they note is that some versions of IRI
violate this principle. It is possible, they argue, to come to know
\emph{p} solely by getting evidence against it, at least on common
versions of IRI.

But this is not a distinctive, or problematic, feature of IRI. It is
true of any theory of knowledge that allows for Gettier
cases.\footnote{Gettier cases should perhaps be called Dharmottara
  cases, since Dharmottara's 8th Century examples somewhat predate
  Gettier's, and are in some ways a little cleaner. Jennifer Nagel
  (\citeproc{ref-Nagel2014}{2014, 57}) has more discussion of
  Dharmottara.} Assume that our protagonist \emph{S} has evidence that
makes \emph{q} incredibly likely, and on that basis the agent believes
\emph{q}. Assume further that \emph{q} entails \emph{p}, that \emph{q}
is false, and \emph{p} is true. At least in typical such cases, \emph{S}
will not know that \emph{p}.\footnote{Is this restriction to `typical'
  cases necessary? Perhaps -- see recent work by Federico Luzzi
  (\citeproc{ref-Luzzi2010}{2010}) for a good discussion of the point.}
Now imagine that \emph{S} gets new evidence that makes \emph{q} very
unlikely, but makes \emph{r}, which also entails \emph{p}, likely
enough. If \emph{r} is true, then getting this new evidence could both
decrease the evidential probability of \emph{p}, and make it the case
that \emph{S} knows that \emph{p}. The converse is possible too. If the
agent originally believes \emph{p} on the basis of true \emph{r}, then
gets evidence that undermines \emph{r}, but makes \emph{p} more likely
by making \emph{q}, which is false and entails \emph{p}, very likely,
they will get evidence for \emph{p}, and in virtue of that lose
knowledge that \emph{p}.

This point, that in Dharmottara/Gettier cases evidence can both create
knowledge and be evidence against the proposition now known, is not
novel. Brian Weatherson (\citeproc{ref-Weatherson2014-ProbScept}{2014})
discusses some such cases, attributing them to Martin Smith. Weatherson
argues that any theory that allows for defeaters will have this
consequence, since it will always be possible for evidence to defeat a
defeater, while ever so slightly lowering the evidential probability of
\emph{p}. The same will be true of any theory that puts a safety
condition on knowledge, since it will always be possible that a new
experience will make an agent's belief safer, while ever so slightly
lowering its evidential probability.

Here's the particular way I think IRI should be implemented so that it
has the consequence Eaton and Pickavance reveal. But note that the
broader response I'm making to Eaton and Pickavance doesn't turn on any
of the details here; it turns on interests being relevant to something
that grounds the difference between knowledge and justified true belief.

I think interests matter to knowledge because they determine whether or
not practical considerations generate defeaters. Someone who believes
\emph{p}, but has different preferences over live issues conditional on
\emph{p} to what they have unconditionally, is in a way incoherent. This
incoherence defeats any claim to knowledge of \emph{p}. This kind of
defeater should be interest-relative, for if it were not, we'd say that
only things one is completely certain of can be known. If \emph{p} is
not certain, one prefers \$1 to a bet that pays \$1 iff \emph{p}. But
conditional on \emph{p}, one is indifferent between these bets. The
interest-relativity becomes relevant here; this change in preference
conditional on \emph{p} doesn't matter because one isn't really faced
with such a choice.

If interest-relativity matters to knowledge because it matters to
defeaters, or to safety, or to whatever explains Dharmottara/Gettier
cases, then we should expect IRI to have just the consequence Eaton and
Pickavance uncover. In short, any IRI theorist who thinks interests are
relevant to knowledge in ways that go beyond how they are relevant to
justified belief should view Eaton and Pickavance's results as an
interesting discovery, not any kind of \emph{reductio}.

\section*{References}\label{references}
\addcontentsline{toc}{section}{References}

\phantomsection\label{refs}
\begin{CSLReferences}{1}{0}
\bibitem[\citeproctext]{ref-EatonPickavance2015}
Eaton, Daniel, and Timothy Pickavance. 2015. {``Evidence Against
Pragmatic Encroachment.''} \emph{Philosophical Studies} 172: 3135--43.
doi:
\href{https://doi.org/10.1007/s11098-015-0461-x}{10.1007/s11098-015-0461-x}.

\bibitem[\citeproctext]{ref-Luzzi2010}
Luzzi, Federico. 2010. {``Counter-Closure.''} \emph{Australasian Journal
of Philosophy} 88 (4): 673--83. doi:
\href{https://doi.org/10.1080/00048400903341770}{10.1080/00048400903341770}.

\bibitem[\citeproctext]{ref-Nagel2014}
Nagel, Jennifer. 2014. \emph{Knowledge: A Very Short Introduction}.
Oxford: Oxford University Press.

\bibitem[\citeproctext]{ref-Weatherson2014-ProbScept}
Weatherson, Brian. 2014. {``Probability and Scepticism.''} In
\emph{Scepticism and Perceptual Justification}, edited by Dylan Dodd and
Elia Zardini, 71--86. Oxford: Oxford University Press.

\end{CSLReferences}



\noindent Published in\emph{
Philosophical Studies}, 2016, pp. 3231–3233.


\end{document}
