% Options for packages loaded elsewhere
\PassOptionsToPackage{unicode}{hyperref}
\PassOptionsToPackage{hyphens}{url}
%
\documentclass[
  10pt,
  letterpaper,
  DIV=11,
  numbers=noendperiod,
  twoside]{scrartcl}

\usepackage{amsmath,amssymb}
\usepackage{setspace}
\usepackage{iftex}
\ifPDFTeX
  \usepackage[T1]{fontenc}
  \usepackage[utf8]{inputenc}
  \usepackage{textcomp} % provide euro and other symbols
\else % if luatex or xetex
  \usepackage{unicode-math}
  \defaultfontfeatures{Scale=MatchLowercase}
  \defaultfontfeatures[\rmfamily]{Ligatures=TeX,Scale=1}
\fi
\usepackage{lmodern}
\ifPDFTeX\else  
    % xetex/luatex font selection
    \setmainfont[ItalicFont=EB Garamond Italic,BoldFont=EB Garamond
Bold]{EB Garamond Math}
    \setsansfont[]{Europa-Bold}
  \setmathfont[]{Garamond-Math}
\fi
% Use upquote if available, for straight quotes in verbatim environments
\IfFileExists{upquote.sty}{\usepackage{upquote}}{}
\IfFileExists{microtype.sty}{% use microtype if available
  \usepackage[]{microtype}
  \UseMicrotypeSet[protrusion]{basicmath} % disable protrusion for tt fonts
}{}
\usepackage{xcolor}
\usepackage[left=1in, right=1in, top=0.8in, bottom=0.8in,
paperheight=9.5in, paperwidth=6.5in, includemp=TRUE, marginparwidth=0in,
marginparsep=0in]{geometry}
\setlength{\emergencystretch}{3em} % prevent overfull lines
\setcounter{secnumdepth}{3}
% Make \paragraph and \subparagraph free-standing
\makeatletter
\ifx\paragraph\undefined\else
  \let\oldparagraph\paragraph
  \renewcommand{\paragraph}{
    \@ifstar
      \xxxParagraphStar
      \xxxParagraphNoStar
  }
  \newcommand{\xxxParagraphStar}[1]{\oldparagraph*{#1}\mbox{}}
  \newcommand{\xxxParagraphNoStar}[1]{\oldparagraph{#1}\mbox{}}
\fi
\ifx\subparagraph\undefined\else
  \let\oldsubparagraph\subparagraph
  \renewcommand{\subparagraph}{
    \@ifstar
      \xxxSubParagraphStar
      \xxxSubParagraphNoStar
  }
  \newcommand{\xxxSubParagraphStar}[1]{\oldsubparagraph*{#1}\mbox{}}
  \newcommand{\xxxSubParagraphNoStar}[1]{\oldsubparagraph{#1}\mbox{}}
\fi
\makeatother


\providecommand{\tightlist}{%
  \setlength{\itemsep}{0pt}\setlength{\parskip}{0pt}}\usepackage{longtable,booktabs,array}
\usepackage{calc} % for calculating minipage widths
% Correct order of tables after \paragraph or \subparagraph
\usepackage{etoolbox}
\makeatletter
\patchcmd\longtable{\par}{\if@noskipsec\mbox{}\fi\par}{}{}
\makeatother
% Allow footnotes in longtable head/foot
\IfFileExists{footnotehyper.sty}{\usepackage{footnotehyper}}{\usepackage{footnote}}
\makesavenoteenv{longtable}
\usepackage{graphicx}
\makeatletter
\newsavebox\pandoc@box
\newcommand*\pandocbounded[1]{% scales image to fit in text height/width
  \sbox\pandoc@box{#1}%
  \Gscale@div\@tempa{\textheight}{\dimexpr\ht\pandoc@box+\dp\pandoc@box\relax}%
  \Gscale@div\@tempb{\linewidth}{\wd\pandoc@box}%
  \ifdim\@tempb\p@<\@tempa\p@\let\@tempa\@tempb\fi% select the smaller of both
  \ifdim\@tempa\p@<\p@\scalebox{\@tempa}{\usebox\pandoc@box}%
  \else\usebox{\pandoc@box}%
  \fi%
}
% Set default figure placement to htbp
\def\fps@figure{htbp}
\makeatother
% definitions for citeproc citations
\NewDocumentCommand\citeproctext{}{}
\NewDocumentCommand\citeproc{mm}{%
  \begingroup\def\citeproctext{#2}\cite{#1}\endgroup}
\makeatletter
 % allow citations to break across lines
 \let\@cite@ofmt\@firstofone
 % avoid brackets around text for \cite:
 \def\@biblabel#1{}
 \def\@cite#1#2{{#1\if@tempswa , #2\fi}}
\makeatother
\newlength{\cslhangindent}
\setlength{\cslhangindent}{1.5em}
\newlength{\csllabelwidth}
\setlength{\csllabelwidth}{3em}
\newenvironment{CSLReferences}[2] % #1 hanging-indent, #2 entry-spacing
 {\begin{list}{}{%
  \setlength{\itemindent}{0pt}
  \setlength{\leftmargin}{0pt}
  \setlength{\parsep}{0pt}
  % turn on hanging indent if param 1 is 1
  \ifodd #1
   \setlength{\leftmargin}{\cslhangindent}
   \setlength{\itemindent}{-1\cslhangindent}
  \fi
  % set entry spacing
  \setlength{\itemsep}{#2\baselineskip}}}
 {\end{list}}
\usepackage{calc}
\newcommand{\CSLBlock}[1]{\hfill\break\parbox[t]{\linewidth}{\strut\ignorespaces#1\strut}}
\newcommand{\CSLLeftMargin}[1]{\parbox[t]{\csllabelwidth}{\strut#1\strut}}
\newcommand{\CSLRightInline}[1]{\parbox[t]{\linewidth - \csllabelwidth}{\strut#1\strut}}
\newcommand{\CSLIndent}[1]{\hspace{\cslhangindent}#1}

\setlength\heavyrulewidth{0ex}
\setlength\lightrulewidth{0ex}
\usepackage[automark]{scrlayer-scrpage}
\clearpairofpagestyles
\cehead{
  Brian Weatherson
  }
\cohead{
  No Royal Road to Relativism
  }
\ohead{\bfseries \pagemark}
\cfoot{}
\makeatletter
\newcommand*\NoIndentAfterEnv[1]{%
  \AfterEndEnvironment{#1}{\par\@afterindentfalse\@afterheading}}
\makeatother
\NoIndentAfterEnv{itemize}
\NoIndentAfterEnv{enumerate}
\NoIndentAfterEnv{description}
\NoIndentAfterEnv{quote}
\NoIndentAfterEnv{equation}
\NoIndentAfterEnv{longtable}
\NoIndentAfterEnv{abstract}
\renewenvironment{abstract}
 {\vspace{-1.25cm}
 \quotation\small\noindent\rule{\linewidth}{.5pt}\par\smallskip
 \noindent }
 {\par\noindent\rule{\linewidth}{.5pt}\endquotation}
\KOMAoption{captions}{tableheading}
\makeatletter
\@ifpackageloaded{caption}{}{\usepackage{caption}}
\AtBeginDocument{%
\ifdefined\contentsname
  \renewcommand*\contentsname{Table of contents}
\else
  \newcommand\contentsname{Table of contents}
\fi
\ifdefined\listfigurename
  \renewcommand*\listfigurename{List of Figures}
\else
  \newcommand\listfigurename{List of Figures}
\fi
\ifdefined\listtablename
  \renewcommand*\listtablename{List of Tables}
\else
  \newcommand\listtablename{List of Tables}
\fi
\ifdefined\figurename
  \renewcommand*\figurename{Figure}
\else
  \newcommand\figurename{Figure}
\fi
\ifdefined\tablename
  \renewcommand*\tablename{Table}
\else
  \newcommand\tablename{Table}
\fi
}
\@ifpackageloaded{float}{}{\usepackage{float}}
\floatstyle{ruled}
\@ifundefined{c@chapter}{\newfloat{codelisting}{h}{lop}}{\newfloat{codelisting}{h}{lop}[chapter]}
\floatname{codelisting}{Listing}
\newcommand*\listoflistings{\listof{codelisting}{List of Listings}}
\makeatother
\makeatletter
\makeatother
\makeatletter
\@ifpackageloaded{caption}{}{\usepackage{caption}}
\@ifpackageloaded{subcaption}{}{\usepackage{subcaption}}
\makeatother

\usepackage{bookmark}

\IfFileExists{xurl.sty}{\usepackage{xurl}}{} % add URL line breaks if available
\urlstyle{same} % disable monospaced font for URLs
\hypersetup{
  pdftitle={No Royal Road to Relativism},
  pdfauthor={Brian Weatherson},
  hidelinks,
  pdfcreator={LaTeX via pandoc}}


\title{No Royal Road to Relativism}
\author{Brian Weatherson}
\date{2011}

\begin{document}
\maketitle
\begin{abstract}
A reponse to \emph{Relativism and Monadic Truth}. I argue that while the
Cappelen and Hawthorne have good responses to the deductive arguments
for relativism, there are various good inductive arguments for
relativism that their view can't adequately respond to.
\end{abstract}


\setstretch{1.1}
\emph{Relativism and Monadic Truth} is a sustained attack on `analytical
relativism', as it has developed in recent years. The attack focusses on
two kinds of arguments. One is the argument from the behaviour of
operators, as developed by David Lewis (\citeproc{ref-Lewis1980b}{1980})
and David Kaplan (\citeproc{ref-Kaplan1977-KAPD}{1989}). The other kind
of argument takes off from phenomena concerning speech reports and
disagreements. Such arguments play central roles in arguments by, among
others, Andy Egan (\citeproc{ref-Egan2007-EGAEMR}{2007}), Max Kölbel
(\citeproc{ref-Kolbel2009-KLBTEF}{2008}), Peter Lasersohn
(\citeproc{ref-Lasersohn2005}{2005}), John MacFarlane
(\citeproc{ref-MacFarlane2003-MACFCA}{2003},
\citeproc{ref-MacFarlane2007-MACRAD}{2007}), Mark Richard
(\citeproc{ref-Richard2004-RICCAR}{2004}) and Tamina Stephenson
(\citeproc{ref-Stephenson2007-STEJDE}{2007}). These arguments also play
a role in a paper that I co-authored with Andy Egan and John Hawthorne
(\citeproc{ref-Egan2005-EGAEMI}{Egan, Hawthorne, and Weatherson 2005}).

As the reader of \emph{Relativism and Monadic Truth} can tell, John
Hawthorne no longer much likes the arguments of that paper, nor its
conclusions. And I think he's right to be sceptical of some of the
arguments we advanced. The objections that he and Herman Cappelen raise
to arguments for relativism from speech reports and from disagreement
are, I think, telling. But I don't think those are the best arguments
for relativism. (For what it's worth, I don't think they're even the
best arguments in the paper we co-authored.) The primary purpose of this
note will be say a little about what some of these better arguments are.
The core idea will be that although there is some data that is
\emph{consistent} with non-relativist theories, the best explanation of
this data is that a kind of relativism is true. In short, we should be
looking for inductive, not deductive, arguments for relativism. I'm
going to fill in some details of this argument, and say a little about
how it seemed to slip out of the main storyline of \emph{Relativism and
Monadic Truth}.

In Chapter 2 of \emph{Relativism and Monadic Truth}, Cappelen and
Hawthorne attempt to develop diagnostics for when an utterance type
\emph{S} has invariant content. They note that some relativist arguments
presuppose a diagnostic based on speech reports. The idea behind the
presupposed diagnostic is that if we can invariably report an utterance
of \emph{S} by \emph{A} by saying \emph{A said that S}, then \emph{S} is
semantically invariant. And they note that this diagnostic isn't
particularly reliable.

What they aim to replace it with is a diagnostic based on agreement
reports. Cappelen and Hawthorne are more careful on the details than
I'll be, but the rough idea is easy enough to understand. The diagnostic
says that if whenever \emph{A} and \emph{B} utter \emph{S}, we can
report that by saying \emph{A and B agree that S}, and the basis for our
saying this is that they made those utterances, then \emph{S}'s content
is invariant. The idea behind the test is that if there isn't a single
proposition that \emph{A} and \emph{B} endorse, then it would be odd to
say that they agree.

I don't think the diagnostic is particularly plausible. The next couple
of paragraphs won't come as much of a surprise to the authors of
\emph{Relativism and Monadic Truth}, since the ideas come from a talk
Herman Cappelen did at the Arché Summer School in July 2009. But they
are central enough to the story I'm telling that they are worth
including here. The core problem for this agreement based diagnostic is
that sometimes we can report parties as agreeing even though they don't
agree on the truth value of any proposition. So while (1) has a
disambiguation where it is true only if something like (2) is true, it
also has a disambiguation where it is true as long as something like (3)
is true:

\begin{description}
\tightlist
\item[(1)]
Alec, Pierre and Franz agree that they were lucky to be born where they
were actually born.
\item[(2)]
Alec, Pierre and Franz each consider each of Alec, Pierre and Franz
lucky to be born where they were actually born.
\item[(3)]
Alec, Pierre and Franz each consider themselves lucky to be born where
they were actually born.
\end{description}

Given that sentences like (1) could mean something like (3), there is
little reason to think that agreement diagnostics will provide us clear
evidence of sameness of content. Indeed, Cappelen and Hawthorne should
hope that this diagnostic doesn't always work, because the diagnostic
seems to entail relativism about epistemic modals. Imagine that a
detective and a psychic are both investigating a murder. They both
conclude that their evidence entails that Tatort did it, and that their
evidence is consistent with Tatort being dead. They are, however,
ignorant of each other's evidence, and indeed of the fact that the other
is working on the investigation. Still, if each utters (4), it seems we
are in a position to endorse (5).

\begin{description}
\tightlist
\item[(4)]
Tatort must be guilty, and might be dead.
\item[(5)]
The detective and the psychic agree that Tatort must be guilty, and that
he might be dead.
\end{description}

This will be very hard to explain on a contextualist theory of epistemic
modals, if we accept the agreement diagnostic. That's because there's no
proposition (other than the proposition that Tatort is guilty) that they
agree about. I think this is \emph{some} evidence in favour of
relativism, but if the contextualist wanted to argue that we should
understand (4) the same way we understand (1) (on its `distributive'
disambiguation), it would be hard to conclusively show they were wrong.

In any case, it is hard to see why we should expect there to be a
diagnostic of the kind Cappelen and Hawthorne are aiming for. Such
diagnostics are the exception, not the rule, in social sciences. There's
no simple diagnostic for whether a particular state is democratic or
not. (Is modern-day Afghanistan a democracy? What about modern-day
Alabama?) Nor is there a simple diagnostic for whether a particular rule
is a law. (Are internal revenue regulations laws?) But political science
and jurisprudence don't collapse in the absence of such diagnostics. Nor
should philosophical semantics collapse in the absence of a simple test
for context-sensitivity.

Indeed, the situation is political science and in jurisprudence is in
one sense worse than it is in semantics. We can state, admittedly in
theory-laden terms, what it is for the content of a sentence type to be
context-invariant or context-sensitive. It is much harder to state, even
in theory-laden terms, what it is for a state to be democratic, or for a
rule to be a law. The problem with thinking about the questions I asked
in the previous paragraph isn't that there's some hidden piece of
evidence we haven't yet uncovered. It's that the concepts do not have
clear application conditions, and the hard cases fall between the clear
instances and non-instances of the relevant property. In semantics we
have, to a first approximation, a mere epistemic challenge.

Even if the hunt for a diagnostic for context-sensitivity is bound to be
futile, as I think it is, that doesn't mean it is harmless. I think the
structure of Cappelen and Hawthorne's inquiry, which starts by looking
for a test and then goes on to apply it, pushes us towards the wrong
kind of argument. The effect of this structure is that we end up looking
for \emph{deductive} arguments for or against relativism, and the
absence of deductive arguments for relativism is taken to be a big
problem for the relativist. But we should have been looking for
\emph{inductive} arguments. The best case for relativism, I think, will
be a kind of inference to the best explanation. For instance, a
relativist might try to clean up this argument.

\begin{enumerate}
\def\labelenumi{\arabic{enumi}.}
\tightlist
\item
  Our best theory of mental content is that the contents of beliefs and
  desires do not satisfy Simplicity.\footnote{\emph{Simplicity} is
    Cappelen and Hawthorne's name for the conjunction of theses they
    want to defend against the relativist. For our purposes, Simplicity
    about mental content is the view that the contents of beliefs and
    desires are propositions, and these propositions are simply true or
    simply false, not merely true or false relative to some or other
    parameter. Simplicity about linguistic content is the view that
    these same propositions, the ones that are simply true or false, are
    the contents of declarative utterances.}
\item
  The role of language is to express thoughts, so if the contents of
  belief and desire do not satisfy Simplicity, the contents of sentences
  and utterances probably don't either.
\item
  Simplicity is false as a theory of linguistic content.
\end{enumerate}

This argument clearly isn't valid. That's by design; it's meant to be an
abductive argument against Simplicity about linguistic content. And of
course both premises are controversial. There's one argument for premise
1 in Lewis (\citeproc{ref-Lewis1979b}{1979}), and another in Perry
(\citeproc{ref-Perry1979-PERTPO}{1979}). Both arguments are
controversial. Indeed Cappelen and Hawthore spend some time (pages 50 to
54) responding to the Lewisian arguments, though they spend less time on
Perry's arguments.

I'm not going to try to advance the debate here over whether premise 1
is true or not. I suspect the solution will turn on much bigger issues
than can be covered in a note of this length. And that's because I think
the judgment about whether premise 1 is true will turn on quite global
features of our best theory of mental content. For instance, Daniel
Nolan (\citeproc{ref-Nolan2006-NOLSD}{2006}) argues that there are
certain desires that we cannot understand on the modal Lewis offers.
That doesn't \emph{entail} that Lewis is wrong about the nature of
belief, but it does make Lewis's theory of belief look less attractive.
From the other direction, many authors working on the Sleeping Beauty
problem, dating back to the problem's introduction to the philosophical
community in Elga (\citeproc{ref-Elga2000-ELGSBA}{2000}), have felt that
the problem was best approached in Lewis's Simplicity-unfriendly
framework. That doesn't entail Simplicity is wrong, but it is I think
evidence against it. On the other hand, Robert Stalnaker
(\citeproc{ref-Stalnaker2008-STAOKO}{2008}) has recently argued that
this is not the best framework for thinking about the Sleeping Beauty
problem, and I've argued (\citeproc{ref-Weatherson-SoSB}{Weatherson
2011}) that Stalnaker's approach lets us see things about the Sleeping
Beauty puzzle that are hidden on the standard, Lewisian, approach. So if
we're going to evaluate this kind of argument for relativism, the issues
are going to get far removed from familiar disputes about distributions
of words and phrases. That's not too surprising. In general, the hard
thing about abductive reasoning in philosophy is that we have to start
looking at all sorts of different kinds of evidence. But that's no
reason to think that the most telling arguments won't, at the end of the
day, be abductive arguments.

A quite different kind of argument comes from thinking about property
ascription and ignorance. It's a somewhat frequent occurrence that
modern science discovers that some of our thoughts seem to depend for
their truth on more variables than we realised. So it isn't true that
two accelerating objects simply have the same mass or different masses;
rather, their relative mass might be different relative to different
inertial frames. Or two colour patches might not be simply the same
colour or simply different colours. If the colours are metamers
(relative to human vision) then they will be in a good sense the same
colour relative to human vision, and different colours relative to more
discriminating detectors. Such cases raise challenges for the project of
interpreting a language.

Assume that the community uses terms like `mass'. Indeed, assume they
are sophisticated enough to distinguish mass from weight, for they know
that weight is relative to a gravitational field, and gravitational
fields vary in strength. But they are not sophisticated enough to know
that masses are relative to inertial frames. The members of this
community frequently go around saying things like ``Those two objects
have the same mass,'' referring to \emph{A} and \emph{B}. Call that
sentence \emph{M}. We assume that the members are in a particular
inertial frame, call it \emph{F}. Let's assume (just for a few
paragraphs) that the propositions that satisfy Simplicity are
structured, and assume that we can represent the relation \emph{has the
same mass as} by a somewhat unstructured relation \emph{SameMass}. (In
other words, ignore whatever internal structure \emph{SameMass} has,
since it won't be relevant to this example.) Then it seems to me that
there are three live options around.

\begin{enumerate}
\def\labelenumi{\arabic{enumi}.}
\tightlist
\item
  By \emph{M}, the speakers express the pseudo-proposition
  \emph{SameMass}(\emph{a}, \emph{b}), and this pseudo-proposition is
  not capable of being true or false, since \emph{SameMass} is a
  three-place relation (between two objects and an inertial frame) and
  only two places are specified.
\item
  By \emph{M}, the speakers express the proposition
  \emph{SameMass}(\emph{a}, \emph{b}, \emph{F}), and this proposition is
  (capable of being) true.
\item
  By \emph{M}, the speakers express the proposition
  \emph{SameMass}(\emph{a}, \emph{b}), and this is (capable of being)
  true relative to \emph{F}, although it might be false relative to some
  other inertial frame \emph{F}\^{}\{′\}.
\end{enumerate}

If option 3 is correct, then it seems Simplicity fails.\footnote{I say
  `seems' since I'm not sure exactly what it takes for there to be a
  notion of truth \emph{simpliciter}. The argument on page 96 against
  the conjunction of Simplicity, Eternalism and Temporalism suggests
  that Cappelen and Hawthorne believe the following principle: If
  \emph{p} is true in \emph{C}\textsubscript{1}, and false in
  \emph{C}\textsubscript{2}, and \emph{C}\textsubscript{1} and
  \emph{C}\textsubscript{2} both exist, then \emph{p} is not either
  simply true or simply false. It's not obvious to me why \emph{p}
  couldn't, in \emph{C}\textsubscript{1}, be simply true, but I take it
  Cappelen and Hawthorne are using `simply' in such a way as to exclude
  that. So option 3 is inconsistent with Simplicity.} So if there are
compelling arguments against options 1 and 2, and those are all the
options, then Simplicity is in trouble. And it seems the relativist
might make progress by pushing back against both of those options.

The simplest argument against option 1 is that it violates even a very
weak form of the Principle of Charity. Obviously there are very many
different kinds of charity principles. For instance, there are three
different versions endorsed in Davidson
(\citeproc{ref-Davidson1970-DAVME}{1970}), Lewis
(\citeproc{ref-Lewis1974c}{1974}) and
(\citeproc{ref-Williamson2007-WILTPO-17}{Williamson 2007} Ch. 8). But
any kind of Charity will imply that options 2 or 3 are preferable to
option 1, since option 1 will imply that the subjects don't even have
beliefs about the relative masses of objects, whereas the other options
will imply that their beliefs may well be true, and rational, and even
in some cases amount to knowledge. An alternative argument against
option 1 is that the members of that community would have been right to
take it as a Moorean fact that some things have the same
mass.\footnote{Compare the discussion of Moorean facts in
  (\citeproc{ref-Lewis1994a}{Lewis 1994, 489}).} So option 1 doesn't
look overly plausible.

One argument against premise 2 is that it is impossible for the members
of the community, given their powers of individuation, to make singular
reference to such a thing as an inertial frame. If they don't know what
an inertial frame is, then we might be sceptical of claims that they can
refer to it. (Note that the thought here isn't merely that some
individuals don't know what inertial frames are; the imagined case is
that even experts don't know about the kind of things that we would need
to put into the propositions to give them simple truth values.) Another
argument is that competent speakers of the language should be able to
identify the number of argument places in the properties they use.

Neither of the arguments just offered is completely compelling, though I
think both are at least promising avenues for research. But both
arguments do look notably weaker if we drop the assumption that the
relevant propositions are structured. In an unstructured propositions
framework, we perhaps don't need to worry about the members making
singular reference to things like inertial frames. We just need to have
the speakers pick out (in a perhaps imperfect way) the worlds in which
their beliefs are true. And in an unstructured propositions framework it
isn't clear that being unable to identify the number of arguments places
in the properties they use is any more of a sign of linguistic
incompetence than not knowing the individuals to which they refer. But
it is a commonplace of semantic externalism that speakers can refer
without knowing who it is they are referring to.

The arguments in the previous two paragraphs have been sketchy, to say
the least. But if they can be developed into compelling arguments, then
it might turn out that the case against option 2 succeeds iff
propositions are structured. In that case the argument for Simplicity
will turn on a very large question about the nature of propositions,
namely whether they are structured or not. Again, the take home lesson
is that debates in this area are not susceptible to easy resolution.

I'll end with a more narrowly linguistic abductive argument for
relativism and against Simplicity. I think you can find the core
ingredients of this argument in Egan, Hawthorne, and Weatherson
(\citeproc{ref-Egan2005-EGAEMI}{2005}), though it isn't as well
individuated as it might have been. The argument takes off from what
looks like a somewhat misleading claim in Cappelen and Hawthorne's book.
The context is a discussion of autocentric and exocentric uses of
predicates.\footnote{The terminology is from Lasersohn
  (\citeproc{ref-Lasersohn2005}{2005}).} The distinction between
autocentric and exocentric uses is important for thinking about the way
various predicates are used, though it isn't easy to give a
theory-neutral characterisation of it. Assuming contextualism, Cappelen
and Hawthorne note that it is easy to explain the distinction: ``a use
of a taste predicate is autocentric iff its truth conditions are given
by a completion that indexes the predicate to the subject'' and
exocentric iff ``its truth conditions are given by a completion that
indexes it to a person or group other than the speaker, which may,
however, include the speaker.'' (104) The core idea here is clear
enough, I hope, though as they say it requires a slightly different
gloss if we assume relativism.\footnote{As they also go on to note,
  things get very complicated in cases where the truth conditions turn
  on the nature of an idealised version of the speaker. An example of
  such a theory is the theory of value in Lewis
  (\citeproc{ref-Lewis1989b}{1989}). One of the key points that Cappelen
  and Hawthorne make, and I think it is a very good point against a lot
  of claims for relativist theories concerning predicates of personal
  taste, is that this kind of case is very common when it comes to
  evaluative language.}

The problem is what they go on to say about epistemic modals. In a
footnote they say,

\begin{quote}
{[}I{]}t is worth noting that there is a similar contrast between
autocentric and exocentric uses of epistemic modals. If I see Sally
hiding on a bus then I might in a suitable context say `She is hiding
because I might be on the bus' even though I know perfectly well that I
am not on the bus. (`Must' is harder to use exocentrically, though we
shall not undertake to explain this here.) (104n7)
\end{quote}

The parenthetical remark seems mistaken, or at least misleading, and for
an important reason. It is true that it is very hard to use `must'
exocentrically in a sentence of the form \emph{a must be F}. But that
doesn't mean that it is hard to use `must' exocentically. In fact it's
very easy. Almost any sentence of the form \emph{S believes that a must
be F} will have an exocentric use of `must'. That's because almost any
use of an epistemic modal in the scope of a propositional attitude
report will be `bound' to the subject of that report. (I put `bound' in
scare quotes because although contextualists will think of this as
literally a case of binding, non-contextualists may think something else
is going on.) This suggests an argument for relativism about epistemic
modals, one that seems to me to be quite a bit stronger than the
arguments for relativism discussed in \emph{Relativism and Monadic
Truth}.

\begin{enumerate}
\def\labelenumi{\arabic{enumi}.}
\tightlist
\item
  Unembedded uses of epistemic modals are generally autocentric (except
  in the context of explanations, like `because I might be on the bus').
\item
  Epistemic modals embedded in propositional attitude reports are
  generally exocentric.
\item
  There is a good, simple relativist explanation of these two facts.
\item
  There is no good, simple explanation of these facts consistent with
  Simplicity.
\item
  So, relativism is true, and Simplicity is false.
\end{enumerate}

Note that I'm not for a minute suggesting that there is no
Simplicity-friendly explanation of the facts to be had; just that it
won't be a very good explanation. Nor am I suggesting that the phenomena
obtain universally, rather than just in most cases. But they obtain
often enough to need explanation, and the best explanation will be
relativist. And that, I think, is a reason to like relativism.

The simplest relativist explanation of premises 1 and 2 uses the idea,
derived from Lewis (\citeproc{ref-Lewis1979b}{1979}), that contents are
\(\lambda\)-abstracts. So the content of \emph{a must be F} is roughly
\(\lambda\) \emph{x}.(\emph{x}'s evidence entails that \emph{A} is
\emph{F}). A content \(\lambda\) \emph{x}. \(\phi\)(\emph{x}) is true
relative to a person iff they are \(\phi\), and believed by a person iff
they consider themselves to be \(\phi\), under a distinctively
first-personal mode of presentation. Then a typical utterance of \emph{a
must be F} will be autocentric because if the asserter thinks it is
true, they must take themselves to satisfy \(\lambda\)
\emph{x}.(\emph{x}'s evidence entails that \emph{A} is \emph{F}). So
assuming they are speaking truly, the hearer can infer that the
speaker's evidence does indeed entail that \emph{A} is \emph{F}. But a
typical utterance of \emph{S believes that a must be F} will be true
just in case \emph{S} takes themselves to satisfy \(\lambda\)
\emph{x}.(\emph{x}'s evidence entails that \emph{A} is \emph{F}), and
hence will be true as long as \emph{S}'s evidence, or at least what
\emph{S} takes to be their evidence, entails that \emph{A} is \emph{F}.
There's no reference there to the speaker's evidence, so the use of
`must' is exocentric. There are, to be sure, many details of this
explanation that could use filling in, but what is clear is that there
is a natural path from the view that contents are \(\lambda\)-abstracts
to the data to be explained. And that explanation is inconsistent with
Simplicity.

Is there a good, simple Simplicity-friendly explanation of the data
around? I suspect there is not. There are two obvious places to look for
a Simplicity-friendly explanation. We could look for an explanation that
turns on the meaning of `must', or we could look for an explanation in
terms of salience. On closer inspection, neither avenue is particularly
promising.

It does seem likely that there is an available explanation of premise 1
in terms of the meaning of `must'. The contextualist about pronouns has
an easy explanation of why `we' almost always picks out a group that
includes the speaker. The explanation is just that it is part of the
meaning of `we' that it is a \emph{first-personal} plural pronoun, so it
is part of the meaning of `we' that the group it picks out includes the
speaker. We could argue that something similar goes on for `must'. So
\emph{a must be F} means, roughly, that \emph{x}'s evidence entails that
\emph{A} is \emph{F}, and it is part of the meaning of `must' that
\emph{x} either is the speaker, or is a group that includes the speaker.
The problem with this explanation is that it won't extend to premise 2.
And that's because meanings (in the relevant sense) don't change when we
move into embedded contexts. For example if Jones says ``Smith thinks
that we will all get worse grades than she will get'', Jones isn't
accusing Smith of having the inconsistent belief that she will get lower
grades than what she gets. Rather, the reference of `we' is still a
group that includes the \emph{speaker}, not the subject of the
propositional attitude report. On this model, you'd expect the truth
condition of \emph{S believes that a must be F} to be that \emph{S}
believes that \emph{x}'s evidence entails that \emph{A} is \emph{F},
where \emph{x} is the speaker, or a group containing the speaker. But
that's typically not at all what it means. So this kind of explanation
fails.

The problem for salience based explanations of premises 1 and 2 is that
salience is too fragile an explanatory base to explain the data. Let's
say that in general we think \emph{a must be F} means, roughly, that
\emph{x}'s evidence entails that \emph{A} is \emph{F}, and \emph{x} is
generally the most salient knower in the context. Then we'd expect that
it would be not too hard to read (6) in such a way that its truth
condition is (6a) rather than (6b), and (7) in such a way that its truth
condition is (7a) rather than (7b).

After all, (6) and (7) make Jones's evidence \emph{really salient}. That
evidence settles who the killer is! But, it seems, that isn't salient
enough to make (6a) or (7a) the preferred interpretation. That seems to
be bad news for a salience-based explanation of the way we interpret
epistemic modals.

Like all abductive arguments, this argument is far from conclusive. One
way for a proponent of Simplicity to respond to it would be to come up
with a neater explanation of premises 1 and 2 in our abductive argument,
without giving up Simplicity. Another way would be to argue that
although there is no nice Simplicity-friendly explanation of the data,
the costs of relativism are so high that we should shun the relativist
explanation on independent grounds. I don't pretend to have ready
responses to either of these moves. All I want to stress is that these
abductive arguments are generally stronger arguments for relativism than
the arguments that are, correctly, dismissed in \emph{Relativism and
Monadic Truth}. Those arguments try to take a quick path to relativism,
claiming that some data about reports, or disagreement, or syntax,
entails relativism. I doubt any such argument works, in part because of
the objections that Cappelen and Hawthorne raise. There is, as my title
says, no royal road to relativism. But I doubt there's a quick road away
from relativism either. If the relativist can explain with ease patterns
that perplex the contextualist, we have good reason to believe that
relativism is in fact true.

\section*{References}\label{references}
\addcontentsline{toc}{section}{References}

\phantomsection\label{refs}
\begin{CSLReferences}{1}{0}
\bibitem[\citeproctext]{ref-Davidson1970-DAVME}
Davidson, Donald. 1970. {``Mental Events.''} In \emph{Experience and
Theory}, edited by Lawrence Foster and J. W. Swanson, 79--101. London:
Duckworth.

\bibitem[\citeproctext]{ref-Egan2007-EGAEMR}
Egan, Andy. 2007. {``{Epistemic Modals, Relativism and Assertion}.''}
\emph{Philosophical Studies} 133 (1): 1--22. doi:
\href{https://doi.org/10.1007/s11098-006-9003-x}{10.1007/s11098-006-9003-x}.

\bibitem[\citeproctext]{ref-Egan2005-EGAEMI}
Egan, Andy, John Hawthorne, and Brian Weatherson. 2005. {``{Epistemic
Modals in Context}.''} In \emph{Contextualism in Philosophy: Knowledge,
Meaning, and Truth}, edited by Gerhard Preyer and Georg Peter, 131--70.
Oxford: Oxford University Press.

\bibitem[\citeproctext]{ref-Elga2000-ELGSBA}
Elga, Adam. 2000. {``Self-Locating Belief and the Sleeping Beauty
Problem.''} \emph{Analysis} 60 (2): 143--47. doi:
\href{https://doi.org/10.1093/analys/60.2.143}{10.1093/analys/60.2.143}.

\bibitem[\citeproctext]{ref-Kaplan1977-KAPD}
Kaplan, David. 1989. {``Demonstratives.''} In \emph{Themes from Kaplan},
edited by Joseph Almog, John Perry, and Howard Wettstein, 481--563.
Oxford: Oxford University Press.

\bibitem[\citeproctext]{ref-Kolbel2009-KLBTEF}
Kölbel, Max. 2008. {``The Evidence for Relativism.''} \emph{Synthese}
166 (2): 375--95. doi:
\href{https://doi.org/10.1007/s11229-007-9281-7}{10.1007/s11229-007-9281-7}.

\bibitem[\citeproctext]{ref-Lasersohn2005}
Lasersohn, Peter. 2005. {``Context Dependence, Disagreement and
Predicates of Personal Taste.''} \emph{Linguistics and Philosophy} 28
(6): 643--86. doi:
\href{https://doi.org/10.1007/s10988-005-0596-x}{10.1007/s10988-005-0596-x}.

\bibitem[\citeproctext]{ref-Lewis1974c}
Lewis, David. 1974. {``Radical Interpretation.''} \emph{Synthese} 27
(3-4): 331--44. doi:
\href{https://doi.org/10.1007/bf00484599}{10.1007/bf00484599}. Reprinted
in his \emph{Philosophical Papers}, Volume 1, Oxford: Oxford University
Press, 1983, 108-118. References to reprint.

\bibitem[\citeproctext]{ref-Lewis1979b}
---------. 1979. {``Attitudes \emph{de Dicto} and \emph{de Se}.''}
\emph{Philosophical Review} 88 (4): 513--43. doi:
\href{https://doi.org/10.2307/2184646}{10.2307/2184646}. Reprinted in
his \emph{Philosophical Papers}, Volume 1, Oxford: Oxford University
Press, 1983, 133-156. References to reprint.

\bibitem[\citeproctext]{ref-Lewis1980b}
---------. 1980. {``Index, Context, and Content.''} In \emph{Philosophy
and Grammar}, edited by Stig Kanger and Sven Öhman, 79--100. Dordrecht:
Reidel. Reprinted in his \emph{Papers in Philosophical Logic},
Cambridge: Cambridge University Press, 1998, 21-44. References to
reprint.

\bibitem[\citeproctext]{ref-Lewis1989b}
---------. 1989. {``Dispositional Theories of Value.''}
\emph{Aristotelian Society Supplementary Volume} 63 (1): 113--37. doi:
\href{https://doi.org/10.1093/aristoteliansupp/63.1.89}{10.1093/aristoteliansupp/63.1.89}.
Reprinted in his \emph{Papers in Ethics and Social Philosophy},
Cambridge: Cambridge University Press, 2000, 68-94. References to
reprint.

\bibitem[\citeproctext]{ref-Lewis1994a}
---------. 1994. {``Humean Supervenience Debugged.''} \emph{Mind} 103
(412): 473--90. doi:
\href{https://doi.org/10.1093/mind/103.412.473}{10.1093/mind/103.412.473}.
Reprinted in his \emph{Papers in Metaphysics and Epistemology},
Cambridge: Cambridge University Press, 1999, 224-247. References to
reprint.

\bibitem[\citeproctext]{ref-MacFarlane2003-MACFCA}
MacFarlane, John. 2003. {``Future Contingents and Relative Truth.''}
\emph{The Philosophical Quarterly} 53 (212): 321--36. doi:
\href{https://doi.org/10.1111/1467-9213.00315}{10.1111/1467-9213.00315}.

\bibitem[\citeproctext]{ref-MacFarlane2007-MACRAD}
---------. 2007. {``{Relativism and Disagreement}.''}
\emph{Philosophical Studies} 132 (1): 17--31. doi:
\href{https://doi.org/10.1007/s11098-006-9049-9}{10.1007/s11098-006-9049-9}.

\bibitem[\citeproctext]{ref-Nolan2006-NOLSD}
Nolan, Daniel. 2006. {``Selfless Desires.''} \emph{Philosophy and
Phenomenological Research} 73 (3): 665--79. doi:
\href{https://doi.org/10.1111/j.1933-1592.2006.tb00553.x}{10.1111/j.1933-1592.2006.tb00553.x}.

\bibitem[\citeproctext]{ref-Perry1979-PERTPO}
Perry, John. 1979. {``The Problem of the Essential Indexical.''}
\emph{No{û}s} 13 (1): 3--21. doi:
\href{https://doi.org/10.2307/2214792}{10.2307/2214792}.

\bibitem[\citeproctext]{ref-Richard2004-RICCAR}
Richard, Mark. 2004. {``Contextualism and Relativism.''}
\emph{Philosophical Studies} 119 (1-2): 215--42. doi:
\href{https://doi.org/10.1023/b:phil.0000029358.77417.df}{10.1023/b:phil.0000029358.77417.df}.

\bibitem[\citeproctext]{ref-Stalnaker2008-STAOKO}
Stalnaker, Robert. 2008. \emph{Our Knowledge of the Internal World}.
Oxford: Oxford University Press.

\bibitem[\citeproctext]{ref-Stephenson2007-STEJDE}
Stephenson, Tamina. 2007. {``Judge Dependence, Epistemic Modals, and
Predicates of Personal Taste.''} \emph{Linguistics and Philosophy} 30
(4): 484--525. doi:
\href{https://doi.org/10.1007/s10988-008-9023-4}{10.1007/s10988-008-9023-4}.

\bibitem[\citeproctext]{ref-Weatherson-SoSB}
Weatherson, Brian. 2011. {``Stalnaker on Sleeping Beauty.''}
\emph{Philosophical Studies} 155 (3): 445--56. doi:
\href{https://doi.org/10.1007/s11098-010-9613-1}{10.1007/s11098-010-9613-1}.

\bibitem[\citeproctext]{ref-Williamson2007-WILTPO-17}
Williamson, Timothy. 2007. \emph{{The Philosophy of Philosophy}}.
Blackwell.

\end{CSLReferences}



\noindent Published in\emph{
Analysis}, 2011, pp. 133-143.


\end{document}
