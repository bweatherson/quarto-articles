% Options for packages loaded elsewhere
% Options for packages loaded elsewhere
\PassOptionsToPackage{unicode}{hyperref}
\PassOptionsToPackage{hyphens}{url}
%
\documentclass[
  11pt,
  letterpaper,
  DIV=11,
  numbers=noendperiod,
  twoside]{scrartcl}
\usepackage{xcolor}
\usepackage[left=1.1in, right=1in, top=0.8in, bottom=0.8in,
paperheight=9.5in, paperwidth=7in, includemp=TRUE, marginparwidth=0in,
marginparsep=0in]{geometry}
\usepackage{amsmath,amssymb}
\setcounter{secnumdepth}{3}
\usepackage{iftex}
\ifPDFTeX
  \usepackage[T1]{fontenc}
  \usepackage[utf8]{inputenc}
  \usepackage{textcomp} % provide euro and other symbols
\else % if luatex or xetex
  \usepackage{unicode-math} % this also loads fontspec
  \defaultfontfeatures{Scale=MatchLowercase}
  \defaultfontfeatures[\rmfamily]{Ligatures=TeX,Scale=1}
\fi
\usepackage{lmodern}
\ifPDFTeX\else
  % xetex/luatex font selection
  \setmainfont[ItalicFont=EB Garamond Italic,BoldFont=EB Garamond
Bold]{EB Garamond Math}
  \setsansfont[]{EB Garamond}
  \setmathfont[]{Garamond-Math}
\fi
% Use upquote if available, for straight quotes in verbatim environments
\IfFileExists{upquote.sty}{\usepackage{upquote}}{}
\IfFileExists{microtype.sty}{% use microtype if available
  \usepackage[]{microtype}
  \UseMicrotypeSet[protrusion]{basicmath} % disable protrusion for tt fonts
}{}
\usepackage{setspace}
% Make \paragraph and \subparagraph free-standing
\makeatletter
\ifx\paragraph\undefined\else
  \let\oldparagraph\paragraph
  \renewcommand{\paragraph}{
    \@ifstar
      \xxxParagraphStar
      \xxxParagraphNoStar
  }
  \newcommand{\xxxParagraphStar}[1]{\oldparagraph*{#1}\mbox{}}
  \newcommand{\xxxParagraphNoStar}[1]{\oldparagraph{#1}\mbox{}}
\fi
\ifx\subparagraph\undefined\else
  \let\oldsubparagraph\subparagraph
  \renewcommand{\subparagraph}{
    \@ifstar
      \xxxSubParagraphStar
      \xxxSubParagraphNoStar
  }
  \newcommand{\xxxSubParagraphStar}[1]{\oldsubparagraph*{#1}\mbox{}}
  \newcommand{\xxxSubParagraphNoStar}[1]{\oldsubparagraph{#1}\mbox{}}
\fi
\makeatother


\usepackage{longtable,booktabs,array}
\usepackage{calc} % for calculating minipage widths
% Correct order of tables after \paragraph or \subparagraph
\usepackage{etoolbox}
\makeatletter
\patchcmd\longtable{\par}{\if@noskipsec\mbox{}\fi\par}{}{}
\makeatother
% Allow footnotes in longtable head/foot
\IfFileExists{footnotehyper.sty}{\usepackage{footnotehyper}}{\usepackage{footnote}}
\makesavenoteenv{longtable}
\usepackage{graphicx}
\makeatletter
\newsavebox\pandoc@box
\newcommand*\pandocbounded[1]{% scales image to fit in text height/width
  \sbox\pandoc@box{#1}%
  \Gscale@div\@tempa{\textheight}{\dimexpr\ht\pandoc@box+\dp\pandoc@box\relax}%
  \Gscale@div\@tempb{\linewidth}{\wd\pandoc@box}%
  \ifdim\@tempb\p@<\@tempa\p@\let\@tempa\@tempb\fi% select the smaller of both
  \ifdim\@tempa\p@<\p@\scalebox{\@tempa}{\usebox\pandoc@box}%
  \else\usebox{\pandoc@box}%
  \fi%
}
% Set default figure placement to htbp
\def\fps@figure{htbp}
\makeatother


% definitions for citeproc citations
\NewDocumentCommand\citeproctext{}{}
\NewDocumentCommand\citeproc{mm}{%
  \begingroup\def\citeproctext{#2}\cite{#1}\endgroup}
\makeatletter
 % allow citations to break across lines
 \let\@cite@ofmt\@firstofone
 % avoid brackets around text for \cite:
 \def\@biblabel#1{}
 \def\@cite#1#2{{#1\if@tempswa , #2\fi}}
\makeatother
\newlength{\cslhangindent}
\setlength{\cslhangindent}{1.5em}
\newlength{\csllabelwidth}
\setlength{\csllabelwidth}{3em}
\newenvironment{CSLReferences}[2] % #1 hanging-indent, #2 entry-spacing
 {\begin{list}{}{%
  \setlength{\itemindent}{0pt}
  \setlength{\leftmargin}{0pt}
  \setlength{\parsep}{0pt}
  % turn on hanging indent if param 1 is 1
  \ifodd #1
   \setlength{\leftmargin}{\cslhangindent}
   \setlength{\itemindent}{-1\cslhangindent}
  \fi
  % set entry spacing
  \setlength{\itemsep}{#2\baselineskip}}}
 {\end{list}}
\usepackage{calc}
\newcommand{\CSLBlock}[1]{\hfill\break\parbox[t]{\linewidth}{\strut\ignorespaces#1\strut}}
\newcommand{\CSLLeftMargin}[1]{\parbox[t]{\csllabelwidth}{\strut#1\strut}}
\newcommand{\CSLRightInline}[1]{\parbox[t]{\linewidth - \csllabelwidth}{\strut#1\strut}}
\newcommand{\CSLIndent}[1]{\hspace{\cslhangindent}#1}



\setlength{\emergencystretch}{3em} % prevent overfull lines

\providecommand{\tightlist}{%
  \setlength{\itemsep}{0pt}\setlength{\parskip}{0pt}}



 


\setlength\heavyrulewidth{0ex}
\setlength\lightrulewidth{0ex}
\usepackage[automark]{scrlayer-scrpage}
\clearpairofpagestyles
\cehead{
  Brian Weatherson
  }
\cohead{
  Defending Interest Relative Invariantism
  }
\ohead{\bfseries \pagemark}
\cfoot{}
\makeatletter
\newcommand*\NoIndentAfterEnv[1]{%
  \AfterEndEnvironment{#1}{\par\@afterindentfalse\@afterheading}}
\makeatother
\NoIndentAfterEnv{itemize}
\NoIndentAfterEnv{enumerate}
\NoIndentAfterEnv{description}
\NoIndentAfterEnv{quote}
\NoIndentAfterEnv{equation}
\NoIndentAfterEnv{longtable}
\NoIndentAfterEnv{abstract}
\renewenvironment{abstract}
 {\vspace{-1.25cm}
 \quotation\small\noindent\emph{Abstract}:}
 {\endquotation}
\newfontfamily\tfont{EB Garamond}
\addtokomafont{disposition}{\rmfamily}
\addtokomafont{title}{\normalfont\itshape}
\let\footnoterule\relax
\KOMAoption{captions}{tableheading}
\makeatletter
\@ifpackageloaded{caption}{}{\usepackage{caption}}
\AtBeginDocument{%
\ifdefined\contentsname
  \renewcommand*\contentsname{Table of contents}
\else
  \newcommand\contentsname{Table of contents}
\fi
\ifdefined\listfigurename
  \renewcommand*\listfigurename{List of Figures}
\else
  \newcommand\listfigurename{List of Figures}
\fi
\ifdefined\listtablename
  \renewcommand*\listtablename{List of Tables}
\else
  \newcommand\listtablename{List of Tables}
\fi
\ifdefined\figurename
  \renewcommand*\figurename{Figure}
\else
  \newcommand\figurename{Figure}
\fi
\ifdefined\tablename
  \renewcommand*\tablename{Table}
\else
  \newcommand\tablename{Table}
\fi
}
\@ifpackageloaded{float}{}{\usepackage{float}}
\floatstyle{ruled}
\@ifundefined{c@chapter}{\newfloat{codelisting}{h}{lop}}{\newfloat{codelisting}{h}{lop}[chapter]}
\floatname{codelisting}{Listing}
\newcommand*\listoflistings{\listof{codelisting}{List of Listings}}
\makeatother
\makeatletter
\makeatother
\makeatletter
\@ifpackageloaded{caption}{}{\usepackage{caption}}
\@ifpackageloaded{subcaption}{}{\usepackage{subcaption}}
\makeatother
\usepackage{bookmark}
\IfFileExists{xurl.sty}{\usepackage{xurl}}{} % add URL line breaks if available
\urlstyle{same}
\hypersetup{
  pdftitle={Defending Interest Relative Invariantism},
  pdfauthor={Brian Weatherson},
  hidelinks,
  pdfcreator={LaTeX via pandoc}}


\title{Defending Interest Relative Invariantism}
\author{Brian Weatherson}
\date{2011}
\begin{document}
\maketitle
\begin{abstract}
Since interest-relative invariantism (hereafter, IRI) was introduced
into contemporary epistemology in the early 2000s, it has been
criticised on a number of fronts. This paper responds to six different
criticisms of IRI launched by five different authors. And it does so by
noting that the best version of IRI is immune to the criticisms they
have launched. The `best version' in question notes three things about
IRI. First, what matters for knowledge is not strictly the \emph{stakes}
the agent faces in any decision-problem, but really the \emph{odds} at
which she has to bet. Second, IRI is a relatively weak theory; it just
says interests sometimes matter. Defenders of IRI have often derived it
from much stronger principles about reasoning, and critics have attacked
those principles, but much weaker principles would do. Third, and most
importantly, interests matter because generate certain kinds of
\emph{defeaters}. It isn't part of this version of IRI that an agent can
know something in virtue of their interests. Rather, the theory says
that whether a certain kind of consideration is a defeater to an agent's
putative knowledge that \emph{p} depends on their interests. This
matters for the intuitive plausibility of IRI. Critics have argued,
rightly, that interests don't behave in ways distinctive of grounds of
knowledge. But interests do behave like other kinds of defeaters, and
this undermines the criticisms of IRI.
\end{abstract}


\setstretch{1.1}
In recent years a number of authors have defended the
interest-relativity of knowledge and justification. Views of this form
are floated by John Hawthorne (\citeproc{ref-Hawthorne2004}{2004}), and
endorsed by Jeremy Fantl and Matthew McGrath
(\citeproc{ref-Fantl2002}{2002}, \citeproc{ref-FantlMcGrath2009}{2009}),
Jason Stanley (\citeproc{ref-Stanley2005-STAKAP}{2005}) and Brian
Weatherson (\citeproc{ref-Weatherson2005-WEACWD}{2005}). The various
authors differ quite a lot in how much interest-relativity they allow,
but what is common is the defence of interest-relativity.

These views have, quite naturally, drawn a range of criticisms. The
primary purpose of this paper is to respond to these criticisms and, as
it says on the tin, defend interest-relative invariantism, or IRI for
short. But I don't plan to defend every possible version of IRI, only a
particular one. Most of the critics of IRI have assumed that it must
have some or all of the following features.

\begin{enumerate}
\def\labelenumi{\arabic{enumi}.}
\tightlist
\item
  It is harder to know things in high-stakes situations than in
  low-stakes situations.
\item
  There is an interest-sensitive constituent of knowledge.
\item
  IRI stands and falls with some principles connecting knowledge and
  action, such as the principles found in Hawthorne and Stanley
  (\citeproc{ref-Hawthorne2008-HAWKAA}{2008}).
\end{enumerate}

My preferred version of IRI has none of these three features.\footnote{It
  is a tricky exegetical question how many of the three features here
  must be read into defences of IRI in the literature. My reading is
  that they do not have to be read in, so it is not overly original of
  me to defend a version of IRI that does away with all three. But I
  know many people disagree with that. If they're right, this paper is
  more original than I think it is, and so I'm rather happy to be wrong.
  But I'm going to mostly set these exegetical issues aside, and compare
  different theories without taking a stand on who originally
  promulgated them.}

First, it says that knowledge changes when the \textbf{odds} an agent
faces change, not when the \textbf{stakes} change. More precisely,
interests affect belief because whether someone believes \emph{p}
depends \emph{inter alia} on whether their credence in \emph{p} is high
enough that any bet on \emph{p} they actually face is a good bet. And
interests affect knowledge largely because they affect belief. Raising
the stakes of any bet on \emph{p} does not directly change whether an
agent believes \emph{p}, but changing the odds of the bets on \emph{p}
they face does change it. In practice raising the stakes changes the
odds due to the declining marginal utility of material goods. So in
practice high-stakes situations are typically long-odds situations. But
knowledge is hard in those situations because they are long-odds
situations, not because they are high-stakes situations.

So my version of IRI says that knowledge differs between these two
cases.

\begin{description}
\item[High Cost Map:]
Zeno is walking to the Mysterious Bookshop in lower Manhattan. He's
pretty confident that it's on the corner of Warren Street and West
Broadway. But he's been confused about this in the past, forgetting
whether the east-west street is Warren or Murray, and whether the
north-south street is Greenwich, West Broadway or Church. In fact he's
right about the location this time, but he isn't justified in having a
credence in his being correct greater than about 0.95. While he's
walking there, he has two options. He could walk to where he thinks the
shop is, and if it's not there walk around for a few minutes to the
nearby corners to find where it is. Or he could call up directory
assistance, pay \$1, and be told where the shop is. Since he's confident
he knows where the shop is, and there's little cost to spending a few
minutes walking around if he's wrong, he doesn't do this, and walks
directly to the shop.
\item[Low Cost Map:]
Just like the previous case, except that Zeno has a new phone with more
options. In particular, his new phone has a searchable map, so with a
few clicks on the phone he can find where the store is. Using the phone
has some very small costs. For example, it distracts him a little, which
marginally raises the likelihood of bumping into another pedestrian. But
the cost is very small compared to the cost of getting the location
wrong. So even though he is very confident about where the shop is, he
double checks while walking there.
\end{description}

I think the Map Cases are like the various cases that have been used to
motivate interest-relativity\footnote{Such as the Bank Cases in Stanley
  (\citeproc{ref-Stanley2005-STAKAP}{2005}), or the Train Cases in Fantl
  and McGrath (\citeproc{ref-Fantl2002}{2002}).} in all important
respects. I think Zeno knows where the shop is in High Cost Map, and
doesn't know in Low Cost Map. And he doesn't know in Low Cost Map
because the location of the shop has suddenly become the subject matter
of a bet at very long odds. You should think of Zeno's not checking the
location of the shop on his phone-map as a bet on the location of the
shop. If he wins the bet, he wins a few seconds of undistracted
strolling. If he loses, he has to walk around a few blocks looking for a
store. The disutility of the loss seems easily twenty times greater than
the utility of the gain, and by hypothesis the probability of winning
the bet is no greater than 0.95. So he shouldn't take the bet. Yet if he
knew where the store was, he would be justified in taking the bet. So he
doesn't know where the store is. Now this is not a case where higher
\emph{stakes} defeat knowledge. If anything, the stakes are lower in Low
Cost Map. But the relevant odds are longer, and that's what matters to
knowledge.

Second, on this version of IRI, interests matter because there are
interest-sensitive defeaters, not because interests form any kind of new
condition on knowledge, alongside truth, justification, belief and so
on. In particular, interests matter because there are interest-relative
coherence constraints on knowledge. Some coherence constraints, I claim,
are not interest-relative. If an agent believes ¬\emph{p}, that belief
defeats her purported knowledge that \emph{p}, even if the belief that
\emph{p} is true, justified, safe, sensitive and so on. It is tempting
to try to posit a further coherence condition.

\begin{description}
\item[Practical Coherence]
An agent does not know that \emph{p} if she prefers ϕ to ψ
unconditionally, but prefers ψ to ϕ conditional on \emph{p}.
\end{description}

But that is too strong. For reasons similar to those gone over at the
start of Hawthorne (\citeproc{ref-Hawthorne2004}{2004}), it would mean
we know nearly nothing. A more plausible condition is:

\begin{description}
\item[Relevant Practical Coherence]
An agent does not know that \emph{p} if she prefers ϕ to ψ
unconditionally, but prefers ψ to ϕ conditional on \emph{p}, for any ϕ,
ψ that are relevant given her interests.
\end{description}

When this condition is violated, the agent's claim to knowledge is
defeated. As we'll see below, defeaters behave rather differently to
constituents of knowledge. Some things which could not plausibly be
grounds for knowledge could be defeaters to defeaters for knowledge.

\textbf{Relevant Practical Coherence} suffices, at least among agents
who are trying to maximise expected value, to generate an
interest-relativity to knowledge. The general structure of the case
should be familiar from the existing literature. Let \emph{p} be a
proposition that is true, believed by the agent, and strongly but not
quite conclusively supported by their evidence. Let \emph{B} be a bet
that has a small positive return if \emph{p}, and a huge negative return
if ¬\emph{p} . Assume the agent is now offered the bet, and let ϕ be
declining the bet, and ψ be accepting the bet. Conditional on \emph{p},
the bet wins, so the agent prefers the small positive payout, so prefers
ψ to ϕ conditional on \emph{p}. But the bet has a massively negative
expected return, so unconditionally the agent does not want it. That is,
unconditionally she prefers ϕ to ψ. Once the bet is offered, the actions
ϕ and ψ become relevant given her interests, so by \textbf{Relevant
Practical Coherence} she no longer knows \emph{p}. So for such an agent,
knowledge is interest-relative.

Cases where knowledge is defeated because if the agent did know
\emph{p}, that would lead to problems elsewhere in their cognitive
system, have a few quirky features. In particular, whether the agent
knows \emph{p} can depend on very distant features. Consider the
following kind of case.

\begin{quote}
\textbf{Confused Student}

Con is systematically disposed to affirm the consequent. That is, if he
notices that he believes both \emph{p} and \emph{q} → \emph{p}, he's
disposed to either infer \emph{q}, or if that's impermissible given his
evidence, to ditch his belief in the conjunction of \emph{p} and
\emph{q} → \emph{p}. Con has completely compelling evidence for both
\emph{q} → \emph{p} and ¬\emph{q}. He has good but less compelling
evidence for \emph{p}. And this evidence tracks the truth of \emph{p} in
just the right way for knowledge. On the basis of this evidence, Con
believes \emph{p}. Con has not noticed that he believes both \emph{p}
and \emph{q} → \emph{p}. If he did, he's unhesitatingly drop his belief
that \emph{p}, since he'd realise the alternatives (given his
dispositions) involved dropping belief in a compelling proposition. Two
questions:

\begin{itemize}
\item
  Does Con know that \emph{p}?
\item
  If Con were to think about the logic of conditionals, and reason
  himself out of the disposition to affirm the consequent, would he know
  that \emph{p}?
\end{itemize}
\end{quote}

I think the answer to the first question is \emph{No}, and the answer to
the second question is \emph{Yes}. As it stands, Con's disposition to
affirm the consequent is a doxastic defeater of his putative knowledge
that \emph{p}. Put another way, \emph{p} doesn't cohere well enough with
the rest of Con's views for his belief that \emph{p} to count as
knowledge. To be sure, \emph{p} coheres well enough with those beliefs
by objective standards, but it doesn't cohere at all by Con's lights.
Until he changes those lights, it doesn't cohere well enough to be
knowledge. Moreover (as a referee pointed out), Con's belief is not
safe. Since he could easily have `reasoned' himself out of his belief
that \emph{p}, the belief isn't safe in the way that knowledge is safe.

I think that beliefs which violate \textbf{Relevant Practical Coherence}
fail to be knowledge for the same reason that Con's belief that \emph{p}
fails to be knowledge. In what follows, I'll make frequent use of this
analogy; many of the objections to IRI turn out to be equally strong
objections to the view that there are ever defeaters of the type Con
suffers from.

This suggests our third point. This version of IRI does not take IRI to
be a consequence of more general principles about knowledge and action.
It simply says that there exist at least one pair of cases where the
only relevant difference between agents in the two cases concerns their
interests, but one knows that \emph{p} and the other does
not.\footnote{And this is true even though \emph{p} is not a proposition
  about their interests, or something that is supported by propositions
  about their interests, and so on.} I happen to think that most of the
general principles that philosophers have used to try to derive IRI are
false. But since IRI is much weaker than those principles, that is no
reason to conclude IRI is false.\footnote{I will consider, and
  tentatively support, one principle stronger than IRI in the final
  section. But the key point is that these general principles are not
  needed to defend IRI.}

The existence of interest-relativity is then quite a weak claim. There
are plenty of stronger claims in the area we could make. I prefer, for
instance, a version of IRI where being offered bets like \emph{B}
defeats knowledge that \emph{p} even if the agent does not have the
preferences I ascribed above. (That could be because she isn't trying to
maximise expected value, or because she's messed up the expected value
calculations.) But knowledge could be interest-relative even if I'm
wrong about those cases.

So I've set out a version of IRI that lacks three features often
attributed to IRI. I haven't argued for that theory here - I do that at
much greater length in (Author Paper 1). But I hope I've done enough to
convince you that the theory is both a version of IRI, and not obviously
false. In what follows, I'll argue that the theory is immune to the
various challenges to IRI that have been put forward in the literature.
This immunity is, I think, a strong reason to prefer this version of
IRI.

\section{Experimental Objections}\label{sect:xphi}

I don't place as much weight as some philosophers do on the correlation
between the verdicts of an epistemological theory and the gut reactions
that non-experts have to tricky cases. And I don't think the best cases
for IRI relies on such a correlation holding. The best case for IRI is
that it integrates nicely with an independently supported theory of
belief, and that it lets us keep a number of plausible principles
without drifting into skepticism.\footnote{This points are expanded upon
  greatly in Weatherson (\citeproc{ref-Weatherson2012}{2012}).} But
still, it is nice to not have one's theory saying exorbitantly
counterintuitive things. Various experimental results, such as the
results in May et al. (\citeproc{ref-May2010}{2010}) and Feltz and
Zarpentine (\citeproc{ref-FeltzZarpentine2010}{2010}), might be thought
to suggest that IRI does have consequences which are counterintuitive,
or which at least run counter to the intuitions of some experimental
subjects. I'm going to concentrate on the latter set of results here,
though I think that what I say will generalise to related experimental
work. In fact, I think the experiments don't really tell against IRI,
because IRI, at least in my preferred version, doesn't make \emph{any}
unambiguous predictions about the cases at the centre of the
experiments. The reason for this is related to my insistence that we
concentrate on the odds an agent faces, not the stakes she faces.

Feltz and Zarpentine gave subjects related vignettes, such as the
following pair. (Each subject only received one of the pair.)

\begin{description}
\item[High Stakes Bridge]
John is driving a truck along a dirt road in a caravan of trucks. He
comes across what looks like a rickety wooden bridge over a yawning
thousand foot drop. He radios ahead to find out whether other trucks
have made it safely over. He is told that all 15 trucks in the caravan
made it over without a problem. John reasons that if they made it over,
he will make it over as well. So, he thinks to himself, `I know that my
truck will make it across the bridge.'
\item[Low Stakes Bridge]
John is driving a truck along a dirt road in a caravan of trucks. He
comes across what looks like a rickety wooden bridge over a three foot
ditch. He radios ahead to find out whether other trucks have made it
safely over. He is told that all 15 trucks in the caravan made it over
without a problem. John reasons that if they made it over, he will make
it over as well. So, he thinks to himself, `I know that my truck will
make it across the bridge.' (\citeproc{ref-FeltzZarpentine2010}{Feltz
and Zarpentine 2010, 696})
\end{description}

Subjects were asked to evaluate John's thought. And the result was that
27\% of the participants said that John does not know that the truck
will make it across in \textbf{Low Stakes Bridge}, while 36\% said he
did not know this in \textbf{High Stakes Bridge}. Feltz and Zarpentine
say that these results should be bad for interest-relativity views. But
it is hard to see just why this is so.

Note that the change in the judgments between the cases goes in the
direction that IRI seems to predict. The change isn't trivial, even if
due to the smallish sample size it isn't statistically significant in
this sample. But should a view like IRI have predicted a larger change?
To figure this out, we need to ask three questions.

\begin{enumerate}
\def\labelenumi{\arabic{enumi}.}
\tightlist
\item
  What are the costs of the bridge collapsing in the two cases?
\item
  What are the costs of not taking the bet, i.e., not driving across the
  bridge?
\item
  What is the rational credence to have in the bridge's sturdiness given
  the evidence John has?
\end{enumerate}

Conditional on the bridge not collapsing, the drivers presumably prefer
taking the bridge to not taking it. And the actions of taking the bridge
or going around the long way are relevant. So by \textbf{Relevant
Practical Coherence}, the drivers know the bridge will not collapse in
Low Stakes Bridge but not High Stakes Bridge if the following equation
is true. (I assume all the other conditions for knowledge are met, and
that there are no other salient instances of Relevant Practical
Coherence to consider.)

\[
\frac{C_H}{G + C_H} > x > \frac{C_L}{G + C_L}
\]

where \emph{G} is the gain the driver gets from taking a non-collapsing
bridge rather than driving around (or whatever the alternative is),
\emph{C\textsubscript{H}} is the cost of being on a collapsing bridge in
High Stakes Bridge, \emph{C\textsubscript{L}} is the cost of being on a
collapsing bridge in Low Stakes Bridge, and \emph{x} is the probability
that the bridge will collapse. I assume \emph{x} is constant between the
two cases. If that equation holds, then taking the bridge, i.e., acting
as if the bridge won't collapse, maximises expected utility in Low
Stakes Bridge but not High Stakes Bridge. So in High Stakes Bridge,
adding the proposition that the bridge won't collapse to the agent's
cognitive system produces incoherence, since the agent won't (at least
rationally) act as if the bridge won't collapse. So if the equation
holds, the agent's interests in avoiding \emph{C\textsubscript{H}}
creates a doxastic defeater in High Stakes Bridge.

But does the equation hold? Or, more relevantly, did the subjects of the
experiment believe that the equation hold? None of the four variables
has their values clearly entailed by the story, so we have to guess a
little as to what the subjects' views would be.

Feltz and Zarpentine say that the costs in ``High Stakes Bridge are very
costly---certain death---whereas the costs in Low Stakes Bridge are
likely some minor injuries and embarrassment.''
(\citeproc{ref-FeltzZarpentine2010}{Feltz and Zarpentine 2010, 702}) I
suspect both of those claims are wrong, or at least not universally
believed. A lot more people survive bridge collapses than you may
expect, even collapses from a great height.\footnote{In the West Gate
  bridge collapse in Melbourne in 1971, a large number of the victims
  were underneath the bridge; the people on top of the bridge had a
  non-trivial chance of survival. That bridge was 200 feet above the
  water, not 1000, but I'm not sure the extra height would matter
  greatly. Again from a slightly lower height, over 90\% of people on
  the bridge survived the I-35W collapse in Minneapolis in 2007.} And
once the road below a truck collapses, all sorts of things can go wrong,
even if the next bit of ground is only 3 feet away. (For instance, if
the bridge collapses unevenly, the truck could roll, and the driver
would probably suffer more than minor injuries.)

We aren't given any information as to the costs of not crossing the
bridge. But given that 15 other trucks, with less evidence than John,
have decided to cross the bridge, it seems plausible to think they are
substantial. If there was an easy way to avoid the bridge, presumably
the \emph{first} truck would have taken it. If \emph{G} is large enough,
and \emph{C\textsubscript{H}} small enough, then the only way for this
equation to hold will be for \emph{x} to be low enough that we'd have
independent reason to say that the driver doesn't know the bridge will
hold.

But what is the value of \emph{x}? John has a lot of information that
the bridge will support his truck. If I've tested something for
sturdiness two or three times, and it has worked, I won't even think
about testing it again. Consider what evidence you need before you'll
happily stand on a particular chair to reach something in the kitchen,
or put a heavy television on a stand. Supporting a weight is the kind of
thing that either fails the first time, or works fairly reliably.
Obviously there could be some strain-induced effects that cause a
subsequent failure\footnote{As I believe was the case in the I-35W
  collapse.}, but John really has a lot of evidence that the bridge will
support him.

Given those three answers, it seems to me that it is a reasonable bet to
cross the bridge. At the very least, it's no more of an unreasonable bet
than the bet I make every day crossing a busy highway by foot. So I'm
not surprised that 64\% of the subjects agreed that John knew the bridge
would hold him. At the very least, that result is perfectly consistent
with IRI, if we make plausible assumptions about how the subjects would
answer the three numbered questions above.

And as I've stressed, these experiments are only a problem for IRI if
the subjects are reliable. I can think of two reasons why they might not
be. First, subjects tend to massively discount the costs and likelihoods
of traffic related injuries. In most of the country, the risk of death
or serious injury through motor vehicle accident is much higher than the
risk of death or serious injury through some kind of crime or other
attack, yet most people do much less to prevent vehicles harming them
than they do to prevent criminals or other attackers harming
them.\footnote{See the massive drop in the numbers of students walking
  or biking to school, reported in Ham, Martin, and Kohl III
  (\citeproc{ref-Ham2008}{2008}), for a sense of how big an issue this
  is.} Second, only 73\% of these subjects in \emph{this very
experiment} said that John knows the bridge will support him in
\textbf{Low Stakes Bridge}. This is rather striking. Unless the subjects
endorse an implausible kind of scepticism, something has gone wrong with
the experimental design. But if the subjects are implausibly sceptical,
then we shouldn't require our epistemological theory to track their gut
reactions. (And if something has gone wrong with the experimental
design, then obviously can't be used as the basis for any objection.) So
given the fact that the experiment points broadly in the direction of
IRI, and that with some plausible assumptions it is perfectly consistent
with that theory, and that the subjects seem unreasonably sceptical to
the point of unreliability about epistemology, I don't think this kind
of experimental work threatens IRI.

\section{Knowledge By Indifference and By
Wealth}\label{knowledge-by-indifference-and-by-wealth}

Gillian Russell and John Doris (\citeproc{ref-RussellDoris2008}{2009})
argue that Jason Stanley's account of knowledge leads to some
implausible attributions of knowledge, and if successful their
objections would generalise to other forms of IRI. I'm going to argue
that Russell and Doris's objections turn on principles that are
\emph{prima facie} rather plausible, but which ultimately we can reject
for independent reasons.\footnote{I think the objections I make here are
  similar in spirit to those Stanley made in a comments thread on
  \href{http://el-prod.baylor.edu/certain_doubts/?p=616}{Certain
  Doubts}, though the details are new. The thread is at
  \url{http://el-prod.baylor.edu/certain_doubts/?p=616}.}

Their objection relies on variants of the kind of case Stanley uses
heavily in his (\citeproc{ref-Stanley2005-STAKAP}{2005}) to motivate a
pragmatic constraint on knowledge. Stanley considers the kinds of cases
we used to derive IRI from \textbf{Relevant Practical Coherence}. So
imagine an agent who faces a choice between accepting the status quo,
call that ϕ, and taking some giant risk, call that ψ. The giant risk in
this case will involve a huge monetary loss if ¬\emph{p}, and a small
non-monetary gain if \emph{p}. Stanley says, and I agree, that in such a
case the agent doesn't know \emph{p}, even if their belief in \emph{p}
is true, well supported by evidence, and so on. Moreover, he says, had ψ
not been a relevant option, the agent could have known \emph{p}. I
agree, and I think \textbf{Relevant Practical Coherence} explains these
intuitions well.

Russell and Doris imagine two kinds of variants on Stanley's case. In
one variant the agent doesn't care about the material loss associated
with ψ ∧ ¬\emph{p}. As I would put it, although their material wealth
would decline precipitously in that case, their utility would not,
because their utility is not tightly correlated with material wellbeing.
Given that, the agent may well prefer ψ to ϕ unconditionally, and so
would still know \emph{p}. Russell and Doris don't claim this is a
problem in itself, but they do think the conjunction of this with the
previous paragraph is a problem. As they put it, ``you should have
reservations \ldots{} about what makes the knowledge claim true: not
giving a damn, however enviable in other respects, should not be
knowledge-making.'' (\citeproc{ref-RussellDoris2008}{Russell and Doris
2009, 432}).

Their other variant involves an agent with so much money that the
material loss is trifling to them. Since the difference in utility
between having, say, eight billion dollars and seven billion dollars is
not that high, perhaps they will again prefer ψ to ϕ unconditionally, so
still know \emph{p}. But it is, allegedly, counterintuitive to have the
knowledge that \emph{p} turn on the agent's wealth. As Russell and Doris
say, ``matters are now even dodgier for practical interest accounts,
because \emph{money} turns out to be knowledge making.''
(\citeproc{ref-RussellDoris2008}{Russell and Doris 2009, 433}) And this
isn't just because wealth can purchase knowledge. As they say, ``money
may buy the \emph{instruments} of knowledge \ldots{} but here the
connection between money and knowledge seems rather too direct.''
(\citeproc{ref-RussellDoris2008}{Russell and Doris 2009, 433})

The first thing to note about this case is that indifference and wealth
aren't really producing knowledge. What they are doing is more like
defeating a defeater. Remember that the agent in question had enough
evidence, and enough confidence, that they would know \emph{p} were it
not for the practical circumstances. As I said in the introduction,
practical considerations enter debates about knowledge in part because
they are distinctive kinds of defeaters. It seems that's what is going
on here. And we have, somewhat surprisingly, independent evidence to
think that indifference and wealth do matter to defeaters.

Consider two variants on Gilbert Harman's `dead dictator' example
(\citeproc{ref-Harman1973}{Harman 1973, 75}). In the original example,
an agent reads that the dictator has died through an actually reliable
source. But there are many other news sources around, such that if the
agent read them, she would lose her belief. Even if the agent doesn't
read those sources, their presence can constitute defeaters to her
putative knowledge that the dictator died.

In our first variant on Harman's example, the agent simply does not care
about politics. It's true that there are many other news sources around
that are ready to mislead her about the dictator's demise. But she has
no interest in looking them up, nor is she at all likely to look them
up. She mostly cares about literature, and will spend her day reading
old novels. In this case, the misleading news sources are too distant,
in a sense, to be defeaters. So she still knows the dictator has died.
Her indifference towards politics doesn't generate knowledge - the
original reliable report is the knowledge generator - but her
indifference means that a would-be defeater doesn't gain traction.

It might be objected here that the agent doesn't know the dictator has
died because there are misleading reports around saying the dictator is
alive, and she is in no position to rebut them. But this is too high a
standard for knowledge. There are millions of people in Australia who
know that humans are contributing to global warming on purely
testimonial grounds. Many, perhaps even most, of these people would not
be able to answer a carefully put together argument that humans are not
contributing to global warming, such as an argument that picked various
outlying statistics to mislead the reader. And such arguments certainly
exist; the conservative parts of the media do as much as they can to
play them up. But the mere existence of such arguments doesn't defeat
the average person's testimonial knowledge about anthropogenic global
warming. Similarly, the mere existence of misleading reports does not
defeat our agent's knowledge of the dictator's death, as long as there
is no nearby world where she is exposed to the reports. (Thanks here to
an anonymous referee.)

In the second variant, the agent cares deeply about politics, and has
masses of wealth at hand to ensure that she knows a lot about it. Were
she to read the misleading reports that the dictator has survived, then
she would simply use some of the very expensive sources she has to get
more reliable reports. Again this suffices for the misleading reports
not to be defeaters. Even before the rich agent exercises her wealth,
the fact that her wealth gives her access to reports that will correct
for misleading reports means that the misleading reports are not
actually defeaters. So with her wealth she knows things she wouldn't
otherwise know, even before her money goes to work. Again, her money
doesn't generate knowledge -- the original reliable report is the
knowledge generator -- but her wealth means that a would-be defeater
doesn't gain traction.

The same thing is true in Russell and Doris's examples. The agent has
quite a bit of evidence that \emph{p}. That's why she knows \emph{p}.
There's a potential practical defeater for \emph{p}. But due to either
indifference or wealth, the defeater is immunised. Surprisingly perhaps,
indifference and/or wealth can be the difference between knowledge and
ignorance. But that's not because they can be in any interesting sense
`knowledge makers', any more than I can make a bowl of soup by
preventing someone from tossing it out. Rather, they can be things that
block defeaters, both when the defeaters are the kind Stanley talks
about, and when they are more familiar kinds of defeaters.

\section{Temporal Embeddings}\label{sect:time}

Michael Blome-Tillmann (\citeproc{ref-MBT2009}{2009}) has argued that
tense-shifted knowledge ascriptions can be used to show that his version
of Lewisian contextualism is preferable to IRI. Like Russell and Doris,
his argument uses a variant of Stanley's Bank Cases.\footnote{In the
  interests of space, I won't repeat those cases yet again here.} Let
\emph{O} be that the bank is open Saturday morning. If Hannah has a
large debt, she is in a high-stakes situation with respect to \emph{O}.
In Blome-Tillmann's version of the example, Hannah had in fact incurred
a large debt, but on Friday morning the creditor waived this debt.
Hannah had no way of anticipating this on Thursday. She has some
evidence for \emph{O}, but not enough for knowledge if she's in a
high-stakes situation. Blome-Tillmann says that this means after Hannah
discovers the debt waiver, she could say

\begin{enumerate}
\def\labelenumi{(\arabic{enumi})}
\tightlist
\item
  I didn't know \emph{O} on Thursday, but on Friday I did.
\end{enumerate}

But I'm not sure why this case should be problematic for any version of
IRI, and very unsure why it should even look like a \emph{reductio} of
IRI. As Blome-Tillmann notes, it isn't really a situation where Hannah's
stakes change. She was never actually in a high stakes situation. At
most her perception of her stakes change; she thought she was in a
high-stakes situation, then realised that she wasn't. Blome-Tillmann
argues that even this change in perceived stakes can be enough to make
(1) true if IRI is true. Now actually I agree that this change in
perception could be enough to make (1) true, but when we work through
the reason that's so, we'll see that it isn't because of anything
distinctive, let alone controversial, about IRI.

If Hannah is rational, then given her interests she won't be ignoring
¬\emph{O} possibilities on Thursday. She'll be taking them into account
in her plans. Someone who is anticipating ¬\emph{O} possibilities, and
making plans for them, doesn't know \emph{O}. That's not a distinctive
claim of IRI. Any theory should say that if a person is worrying about
¬\emph{O} possibilities, and planning around them, they don't know
\emph{O}. And that's simply because knowledge requires a level of
confidence that such a person simply does not show. If Hannah is
rational, that will describe her on Thursday, but not on Friday. So (1)
is true not because Hannah's practical situation changes between
Thursday and Friday, but because her psychological state changes, and
psychological states are relevant to knowledge.

What if Hannah is, on Thursday, irrationally ignoring ¬\emph{O}
possibilities, and not planning for them even though her rational self
wishes she were planning for them? In that case, it seems she still
believes \emph{O}. After all, she makes the same decisions as she would
as if \emph{O} were sure to be true. But it's worth remembering that if
Hannah does irrationally ignore ¬\emph{O} possibilities, she is being
irrational with respect to \emph{O}. And it's very plausible that this
irrationality defeats knowledge. That is, you can't be irrational with
respect to a proposition and know it. Irrationality excludes knowledge.
In any case, I doubt this is the natural way to read Blome-Tillmann's
example. We naturally read Hannah as being rational, and if she is
rational she won't have the right kind of confidence to count as knowing
\emph{O} on Thursday.

There's a methodological point here worth stressing. Doing epistemology
with imperfect agents often results in facing tough choices, where any
way to describe a case feels a little counterintuitive. If we simply hew
to intuitions, we risk being led astray by just focussing on the first
way a puzzle case is described to us. But once we think through Hannah's
case, we see perfectly good reasons, independent of IRI, to endorse
IRI's prediction about the case.

\section{Problematic Conjunctions}\label{sect:conj}

Blome-Tillmann offers another argument against IRI, that makes heavy use
of the notion of having enough evidence to know something. Here is how
he puts the argument. (Again I've changed the numbering and some
terminology for consistency with this paper.)

\begin{quote}
Suppose that John and Paul have exactly the same evidence, while John is
in a low-stakes situation towards \emph{p} and Paul in a high-stakes
situation towards \emph{p}. Bearing in mind that IRI is the view that
whether one knows \emph{p} depends on one's practical situation, IRI
entails that one can truly assert:

\begin{enumerate}
\def\labelenumi{\arabic{enumi}.}
\setcounter{enumi}{1}
\tightlist
\item
  John and Paul have exactly the same evidence for \emph{p}, but only
  John has enough evidence to know \emph{p}, Paul doesn't.
\end{enumerate}

(\citeproc{ref-MBT2009}{Blome-Tillmann 2009, 328--29})
\end{quote}

And this is meant to be a problem, because (2) is intuitively false.

But IRI doesn't entail any such thing. We can see this by looking at a
simpler example that illustrates the way `enough' works.

George and Ringo both have \$6000 in their bank accounts. They both are
thinking about buying a new computer, which would cost \$2000. Both of
them also have rent due tomorrow, and they won't get any more money
before then. George lives in New York, so his rent is \$5000. Ringo
lives in Syracuse, so his rent is \$1000. Clearly, (REC) and (RAC) are
true.

\begin{description}
\tightlist
\item[(REC)]
Ringo has enough money to buy the computer.
\item[(RAC)]
Ringo can afford the computer.
\end{description}

And (GEC) is true as well, though there's at least a reading of (GAC)
where it is false.

\begin{description}
\tightlist
\item[(GEC)]
George has enough money to buy the computer.
\item[(GAC)]
George can afford the computer.
\end{description}

Focus for now on (GEC). It is a bad idea for George to buy the computer;
he won't be able to pay his rent. But he has enough money to do so; the
computer costs \$2000, and he has \$6000 in the bank. So (GEC) is true.
Admittedly there are things close to (GEC) that aren't true. He hasn't
got enough money to buy the computer and pay his rent. You might say
that he hasn't got enough money to buy the computer given his other
financial obligations. But none of this undermines (GEC).

Now just like George has enough money to buy the computer, Paul has
enough evidence to know that \emph{p}. Paul can't know that \emph{p},
just like George can't buy the computer, because of his practical
situation. But that doesn't mean he doesn't have enough evidence to know
it. He clearly does have enough evidence, since he has the same evidence
John has, and John knows that \emph{p}. So, contra Blome-Tillmann, IRI
doesn't entail this problematic conjunction.

In a footnote attached to this, Blome-Tillmann offers a reformulation of
the argument.

\begin{quote}
I take it that having enough evidence to `know \emph{p}' in \emph{C}
just means having evidence such that one is in a position to `know
\emph{p}' in \emph{C}, rather than having evidence such that one `knows
\emph{p}'. Thus, another way to formulate (2) would be as follows: `John
and Paul have exactly the same evidence for \emph{p}, but only John is
in a position to know \emph{p}, Paul isn't.'
(\citeproc{ref-MBT2009}{Blome-Tillmann 2009, 329n23})
\end{quote}

Now having enough evidence to know \emph{p} isn't the same as being in a
position to know it, any more than having enough money to buy the
computer puts George in a position to buy it. So I think this is more of
a new objection than a reformulation of the previous point. But might it
be a \emph{stronger} objection? Might it be that IRI entails (PosK),
which is false?

\begin{description}
\item[(PosK)]
John and Paul have exactly the same evidence for \emph{p}, but only John
is in a position to know \emph{p}, Paul isn't.
\end{description}

Actually, it isn't a problem that IRI says that (PosK) is true. In fact,
almost any epistemological theory will imply that conjunctions like that
are true. In particular, any epistemological theory that allows for the
existence of defeaters which do not supervene on the possession of
evidence will imply that conjunctions like (PosK) are true. For example,
anyone who thinks that whether you can know that a barn-like structure
is really a barn depends on whether there are non-barns in the
neighbourhood that look like the structure you're looking at will think
that conjunctions like (PosK) are true. Again, it matters a lot that IRI
is suggesting that traditional epistemologists did not notice that there
are distinctively pragmatic defeaters. Once we see that, we'll see that
conjunctions like (PosK) are not surprising at all.

Consider again Con, and his friend Mod who is disposed to reason by
modus ponens and not by affirming the consequent. We could say that Con
and Mod have the same evidence for \emph{p}, but only Mod is in a
position to know \emph{p}. There are only two ways to deny that
conjunction. One is to interpret `position to know' so broadly that Con
is in a position to know \emph{p} because he could change his
inferential dispositions. But then we might as well say that Paul is in
a position to know \emph{p} because he could get into a different
`stakes' situation. Alternatively, we could say that Con's inferential
dispositions count as a kind of evidence against \emph{p}. But that
stretches the notion of evidence beyond a breaking point. Note that we
didn't say Con had any \emph{reason} to affirm the consequent, just that
he does. Someone might adopt, or change, a poor inferential habit
because they get new evidence. But they need not do so, and we shouldn't
count their inferential habits as evidence they have.

If that case is not convincing, we can make the same point with a simple
Gettier-style case.

\begin{quote}
\textbf{Getting the Job}

In world 1, at a particular workplace, someone is about to be promoted.
Agnetha knows that Benny is the management's favourite choice for the
promotion. And she also knows that Benny is Swedish. So she comes to
believe that the promotion will go to someone Swedish. Unsurprisingly,
management does choose Benny, so Agnetha's belief is true.

World 2 is similar, except there it is Anni-Frid who knows that Benny is
the management's favourite choice for the promotion, that Benny is
Swedish. So \emph{she} comes to believe that the promotion will go to
someone Swedish. But in this world Benny quits the workplace just before
the promotion is announced, and the management unexpectedly passes over
a lot of Danish workers to promote another Swede, namely Björn. So
Anni-Frid's belief that the promotion will go to someone Swedish is
true, but not in a way that she could have expected.
\end{quote}

In that story, I think it is clear that Agnetha and Anni-Frid have
exactly the same evidence that the job will go to someone Swedish, but
only Agnetha is in a position to know this, Anni-Frid is not. The fact
that an intermediate step is false in Anni-Frid's reasoning, but not
Agnetha's, means that Anni-Frid's putative knowledge is defeated, but
Agnetha's is not. And when that happens, we can have differences in
knowledge without differences in evidence. So it isn't an argument
against IRI that it allows differences in knowledge without differences
in evidence.

\section{Holism and Defeaters}\label{sect:holism}

The big lesson of the last few sections is that interests create
defeaters. Sometimes an agent can't know \emph{p} because adding
\emph{p} to her stock of beliefs would introduce either incoherence or
irrationality. The reason is normally that the agent faces some decision
where it is, say, bad to do ϕ, but good to do ϕ given \emph{p}. In that
situation, if she adds \emph{p}, she'll either incoherently think that
it's bad to do ϕ although it's good to do it given what is (by her
lights) true. Moreover, the IRI theorist says, being incoherent in this
way blocks knowledge, so the agent doesn't know \emph{p}.

But there are other, more roundabout, ways in which interests can mean
that believing \emph{p} would entail incoherence. One of these is
illustrated by an example alleged by Ram Neta to be hard for
interest-relative theorists to accommodate.

\begin{quote}
Kate needs to get to Main Street by noon: her life depends upon it. She
is desperately searching for Main Street when she comes to an
intersection and looks up at the perpendicular street signs at that
intersection. One street sign says ``State Street'' and the
perpendicular street sign says ``Main Street.'' Now, it is a matter of
complete indifference to Kate whether she is on State Street--nothing
whatsoever depends upon it. (\citeproc{ref-Neta2007}{Neta 2007, 182})
\end{quote}

Let's assume for now that Kate is rational; dropping this assumption
introduces mostly irrelevant complications. That is, we will assume Kate
is an expected utility maximiser. Kate will not believe she's on Main
Street. She would only have that belief if she took it to be settled
that she's on Main, and hence not worthy of spending further effort
investigating. But presumably she won't do that. The rational thing for
her to do is to get confirming (or, if relevant, confounding) evidence
for the appearance that she's on Main. If it were settled that she was
on Main, the rational thing to do would be to try to relax, and be
grateful that she had found Main Street. Since she has different
attitudes about what to do \emph{simpliciter} and conditional on being
on Main Street, she doesn't believe she's on Main Street.

So far so good, but what about her attitude towards the proposition that
she's on State Street? She has enough evidence for that proposition that
her credence in it should be rather high. And no practical issues turn
on whether she is on State. So she believes she is on State, right?

Not so fast! Believing that she's on State has more connections to her
cognitive system than just producing actions. Note in particular that
street signs are hardly basic epistemic sources. They are the kind of
evidence we should be `conservative' about in the sense of Pryor
(\citeproc{ref-Pryor2004-PRYWWW}{2004}). We should only use them if we
antecedently believe they are correct. So for Kate to believe she's on
State, she'd have to believe the street signs she can see are correct.
If not, she'd incoherently be relying on a source she doesn't trust,
even though it is not a basic source.\footnote{The caveats here about
  basic sources are to cancel any suggestion that Kate has to
  antecedently believe that any source is reliable before she uses it.
  As Pryor (\citeproc{ref-Pryor2000-PRYTSA}{2000}) notes, that view is
  problematic. The view that we only get knowledge from a street sign if
  we antecedently have reason to trust it is not so implausible.} But if
she believes the street signs are correct, she'd believe she was on
Main, and that would lead to practical incoherence. So there's no way to
coherently add the belief that she's on State Street to her stock of
beliefs. So she doesn't know, and can't know, that she's either on State
or on Main. This is, in a roundabout way, due to the high stakes Kate
faces.

Neta thinks that the best way for the interest-relative theorist to
handle this case is to say that the high stakes associated with the
proposition that Kate is on Main Street imply that certain methods of
belief formation do not produce knowledge. And he argues, plausibly,
that such a restriction will lead to implausibly sceptical results. But
that's not the only way for the interest-relative theorist to go. What
they could, and I think should, say is that Kate can't know she's on
State Street because the only grounds for that belief are intimately
connected to a proposition that, in virtue of her interests, she needs
very large amounts of evidence to believe.

\section{Non-Consequentialist Cases}\label{non-consequentialist-cases}

None of the replies yet have leaned heavily on the last of the three
points from the introduction, the fact that IRI is an existential claim.
This reply will make heavy use of that fact.

If an agent is merely trying to get the best outcome for themselves,
then it makes sense to represent them as a utility maximiser. But when
agents have to make decisions that might involve them causing harm to
others if certain propositions turn out to be true, then I think it is
not so clear that orthodox decision theory is the appropriate way to
model the agents. That's relevant to cases like this one, which Jessica
Brown has argued are problematic for the epistemological theories John
Hawthorne and Jason Stanley have recently been defending.\footnote{The
  target here is not directly the interest-relativity of their theories,
  but more general principles about the role of knowledge in action and
  assertion. But it's important to see how IRI handles the cases that
  Brown discusses, since these cases are among the strongest challenges
  that have been raised to IRI.}

\begin{quote}
A student is spending the day shadowing a surgeon. In the morning he
observes her in clinic examining patient A who has a diseased left
kidney. The decision is taken to remove it that afternoon. Later, the
student observes the surgeon in theatre where patient A is lying
anaesthetised on the operating table. The operation hasn't started as
the surgeon is consulting the patient's notes. The student is puzzled
and asks one of the nurses what's going on:

\textbf{Student}: I don't understand. Why is she looking at the
patient's records? She was in clinic with the patient this morning.
Doesn't she even know which kidney it is?

\textbf{Nurse}: Of course, she knows which kidney it is. But, imagine
what it would be like if she removed the wrong kidney. She shouldn't
operate before checking the patient's records.
(\citeproc{ref-Brown2008-BROKAP}{Brown 2008, 1144--45})
\end{quote}

It is tempting, but I think mistaken, to represent the payoff table
associated with the surgeon's choice as follows. Let \textbf{Left} mean
the left kidney is diseased, and \textbf{Right} mean the right kidney is
diseased.

\begin{longtable}[]{@{}lcc@{}}
\toprule\noalign{}
\endhead
\bottomrule\noalign{}
\endlastfoot
& \textbf{Left} & \textbf{Right} \\
\textbf{Remove left kidney} & 1 & -1 \\
\textbf{Remove right kidney} & -1 & 1 \\
\textbf{Check notes} & 1-ε & 1-ε \\
\end{longtable}

Here ε is the trivial but non-zero cost of checking the chart. Given
this table, we might reason that since the surgeon knows that she's in
the left column, and removing the left kidney is the best option in that
column, she should remove the left kidney rather than checking the
notes.

But that reasoning assumes that the surgeon does not have any
obligations over and above her duty to maximise expected utility. And
that's very implausible, since consequentialism is a fairly implausible
theory of medical ethics.\footnote{I'm not saying that consequentialism
  is wrong as a theory of medical ethics. But if it is right, so many
  intuitions about medical ethics are going to be mistaken that such
  intuitions have no evidential force. And Brown's argument relies on
  intuitions about this case having evidential value. So I think for her
  argument to work, we have to suppose non-consequentialism about
  medical ethics.}

It's not clear exactly what obligation the surgeon has. Perhaps it is an
obligation to not just know which kidney to remove, but to know this on
the basis of evidence she has obtained while in the operating theatre.
Or perhaps it is an obligation to make her belief about which kidney to
remove as sensitive as possible to various possible scenarios. Before
she checked the chart, this counterfactual was false: \emph{Had she
misremembered which kidney was to be removed, she would have a true
belief about which kidney was to be removed.} Checking the chart makes
that counterfactual true, and so makes her belief that the left kidney
is to be removed a little more sensitive to counterfactual
possibilities.

However we spell out the obligation, it is plausible given what the
nurse says that the surgeon has some such obligation. And it is
plausible that the `cost' of violating this obligation, call it , is
greater than the cost of checking the notes. So here is the decision
table the surgeon faces.

\begin{longtable}[]{@{}lcc@{}}
\toprule\noalign{}
\endhead
\bottomrule\noalign{}
\endlastfoot
& \textbf{Left} & \textbf{Right} \\
\textbf{Remove left kidney} & 1-Δ & -1-Δ \\
\textbf{Remove right kidney} & -1-Δ & 1-Δ \\
\textbf{Check notes} & 1-ε & 1-ε \\
\end{longtable}

And it isn't surprising, or a problem for an interest-relative theory of
knowledge, that the surgeon should check the notes, even if she believes
\emph{and knows} that the left kidney is the diseased one. This is not
to say that the surgeon does know that the left kidney is diseased, just
that the version of IRI being defended here is neutral on that question.

There is a very general point here. It suffices to derive IRI that we
defend principles like the following:

\begin{itemize}
\tightlist
\item
  Whenever maximising expected value is called for, one should maximise
  expected value conditional on everything one knows.
\item
  Maximising expected value is called for often enough that there exist
  the kinds of pairs of cases IRI claims exist. That's because in some
  cases, changing the options facing an agent will make it the case that
  which live option is best differs from which live option is best given
  \emph{p}, even though the agent antecedently knew \emph{p}.
\end{itemize}

But that doesn't imply that maximising expected value is always called
for. Especially in a medical case, it is hard to square an injunction
like ``Do No Harm!'' with a view that one should maximise expected
value, since maximising expected value requires treating harms and
benefits symmetrically. What would be a problem for the version of IRI
defended here was a case with the following four characteristics.

\begin{itemize}
\tightlist
\item
  Maximising expected value is called for in the case.
\item
  Conditional on \emph{p}, the action with the highest expected value is
  ϕ.
\item
  It would be wrong to do ϕ.
\item
  The agent knows \emph{p}.
\end{itemize}

It is tempting for the proponent of IRI to resist any attempted
counterexample by claiming it is not really a case of knowledge. That
might be the right thing to say in Brown's case. But IRI defenders
should remember that it is often a good move to deny that the first
condition holds. Consequentialism is not an obviously correct theory of
decision making in morally fraught situations; purported counterexamples
that rely on it can therefore be resisted.

\subsection*{References}\label{references}
\addcontentsline{toc}{subsection}{References}

\phantomsection\label{refs}
\begin{CSLReferences}{1}{0}
\bibitem[\citeproctext]{ref-MBT2009}
Blome-Tillmann, Michael. 2009. {``Contextualism, Subject-Sensitive
Invariantism, and the Interaction of {`Knowledge'}-Ascriptions with
Modal and Temporal Operators.''} \emph{Philosophy and Phenomenological
Research} 79 (2): 315--31. doi:
\href{https://doi.org/10.1111/j.1933-1592.2009.00280.x}{10.1111/j.1933-1592.2009.00280.x}.

\bibitem[\citeproctext]{ref-Brown2008-BROKAP}
Brown, Jessica. 2008. {``Knowledge and Practical Reason.''}
\emph{Philosophy Compass} 3 (6): 1135--52. doi:
\href{https://doi.org/10.1111/j.1747-9991.2008.00176.x}{10.1111/j.1747-9991.2008.00176.x}.

\bibitem[\citeproctext]{ref-Fantl2002}
Fantl, Jeremy, and Matthew McGrath. 2002. {``Evidence, Pragmatics, and
Justification.''} \emph{Philosophical Review} 111: 67--94. doi:
\href{https://doi.org/10.2307/3182570}{10.2307/3182570}.

\bibitem[\citeproctext]{ref-FantlMcGrath2009}
---------. 2009. \emph{Knowledge in an Uncertain World}. Oxford: Oxford
University Press.

\bibitem[\citeproctext]{ref-FeltzZarpentine2010}
Feltz, Adam, and Chris Zarpentine. 2010. {``Do You Know More When It
Matters Less?''} \emph{Philosophical Psychology} 23 (5): 683--706. doi:
\href{https://doi.org/10.1080/09515089.2010.514572}{10.1080/09515089.2010.514572}.

\bibitem[\citeproctext]{ref-Ham2008}
Ham, Sandra A., Sarah Martin, and Harold W. Kohl III. 2008. {``Changes
in the Percentage of Students Who Walk or Bike to School-United States,
1969 and 2001.''} \emph{Journal of Physical Activity and Health} 5 (2):
205--15. doi:
\href{https://doi.org/10.1123/jpah.5.2.205}{10.1123/jpah.5.2.205}.

\bibitem[\citeproctext]{ref-Harman1973}
Harman, Gilbert. 1973. \emph{Thought}. Princeton: Princeton University
Press.

\bibitem[\citeproctext]{ref-Hawthorne2004}
Hawthorne, John. 2004. \emph{Knowledge and Lotteries}. Oxford: Oxford
University Press.

\bibitem[\citeproctext]{ref-Hawthorne2008-HAWKAA}
Hawthorne, John, and Jason Stanley. 2008. {``{Knowledge and Action}.''}
\emph{Journal of Philosophy} 105 (10): 571--90. doi:
\href{https://doi.org/10.5840/jphil20081051022}{10.5840/jphil20081051022}.

\bibitem[\citeproctext]{ref-May2010}
May, Joshua, Walter Sinnott-Armstrong, Jay G. Hull, and Aaron Zimmerman.
2010. {``Practical Interests, Relevant Alternatives, and Knowledge
Attributions: An Empirical Study.''} \emph{Review of Philosophy and
Psychology} 1 (2): 265--73. doi:
\href{https://doi.org/10.1007/s13164-009-0014-3}{10.1007/s13164-009-0014-3}.

\bibitem[\citeproctext]{ref-Neta2007}
Neta, Ram. 2007. {``Anti-Intellectualism and the Knowledge-Action
Principle.''} \emph{Philosophy and Phenomenological Research} 75 (1):
180--87. doi:
\href{https://doi.org/10.1111/j.1933-1592.2007.00069.x}{10.1111/j.1933-1592.2007.00069.x}.

\bibitem[\citeproctext]{ref-Pryor2000-PRYTSA}
Pryor, James. 2000. {``{The Skeptic and the Dogmatist}.''} \emph{No{û}s}
34 (4): 517--49. doi:
\href{https://doi.org/10.1111/0029-4624.00277}{10.1111/0029-4624.00277}.

\bibitem[\citeproctext]{ref-Pryor2004-PRYWWW}
---------. 2004. {``{What's Wrong with Moore's Argument?}''}
\emph{Philosophical Issues} 14 (1): 349--78. doi:
\href{https://doi.org/10.1111/j.1533-6077.2004.00034.x}{10.1111/j.1533-6077.2004.00034.x}.

\bibitem[\citeproctext]{ref-RussellDoris2008}
Russell, Gillian, and John M. Doris. 2009. {``Knowledge by
Indifference.''} \emph{Australasian Journal of Philosophy} 86 (3):
429--37. doi:
\href{https://doi.org/10.1080/00048400802001996}{10.1080/00048400802001996}.

\bibitem[\citeproctext]{ref-Stanley2005-STAKAP}
Stanley, Jason. 2005. \emph{{Knowledge and Practical Interests}}. Oxford
University Press.

\bibitem[\citeproctext]{ref-Weatherson2005-WEACWD}
Weatherson, Brian. 2005. {``{Can We Do Without Pragmatic
Encroachment?}''} \emph{Philosophical Perspectives} 19 (1): 417--43.
doi:
\href{https://doi.org/10.1111/j.1520-8583.2005.00068.x}{10.1111/j.1520-8583.2005.00068.x}.

\bibitem[\citeproctext]{ref-Weatherson2012}
---------. 2012. {``Knowledge, Bets and Interests.''} In \emph{Knowledge
Ascriptions}, edited by Jessica Brown and Mikkel Gerken, 75--103.
Oxford: Oxford University Press.

\end{CSLReferences}



\noindent Published in\emph{
Logos and Episteme}, 2011, pp. 591-609.


\end{document}
