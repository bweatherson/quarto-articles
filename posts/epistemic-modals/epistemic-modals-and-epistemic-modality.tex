% Options for packages loaded elsewhere
\PassOptionsToPackage{unicode}{hyperref}
\PassOptionsToPackage{hyphens}{url}
%
\documentclass[
  10pt,
  letterpaper,
  DIV=11,
  numbers=noendperiod,
  twoside]{scrartcl}

\usepackage{amsmath,amssymb}
\usepackage{setspace}
\usepackage{iftex}
\ifPDFTeX
  \usepackage[T1]{fontenc}
  \usepackage[utf8]{inputenc}
  \usepackage{textcomp} % provide euro and other symbols
\else % if luatex or xetex
  \usepackage{unicode-math}
  \defaultfontfeatures{Scale=MatchLowercase}
  \defaultfontfeatures[\rmfamily]{Ligatures=TeX,Scale=1}
\fi
\usepackage{lmodern}
\ifPDFTeX\else  
    % xetex/luatex font selection
  \setmainfont[ItalicFont=EB Garamond Italic,BoldFont=EB Garamond
Bold]{EB Garamond Math}
  \setsansfont[]{Europa-Bold}
  \setmathfont[]{Garamond-Math}
\fi
% Use upquote if available, for straight quotes in verbatim environments
\IfFileExists{upquote.sty}{\usepackage{upquote}}{}
\IfFileExists{microtype.sty}{% use microtype if available
  \usepackage[]{microtype}
  \UseMicrotypeSet[protrusion]{basicmath} % disable protrusion for tt fonts
}{}
\usepackage{xcolor}
\usepackage[left=1in, right=1in, top=0.8in, bottom=0.8in,
paperheight=9.5in, paperwidth=6.5in, includemp=TRUE, marginparwidth=0in,
marginparsep=0in]{geometry}
\setlength{\emergencystretch}{3em} % prevent overfull lines
\setcounter{secnumdepth}{3}
% Make \paragraph and \subparagraph free-standing
\ifx\paragraph\undefined\else
  \let\oldparagraph\paragraph
  \renewcommand{\paragraph}[1]{\oldparagraph{#1}\mbox{}}
\fi
\ifx\subparagraph\undefined\else
  \let\oldsubparagraph\subparagraph
  \renewcommand{\subparagraph}[1]{\oldsubparagraph{#1}\mbox{}}
\fi


\providecommand{\tightlist}{%
  \setlength{\itemsep}{0pt}\setlength{\parskip}{0pt}}\usepackage{longtable,booktabs,array}
\usepackage{calc} % for calculating minipage widths
% Correct order of tables after \paragraph or \subparagraph
\usepackage{etoolbox}
\makeatletter
\patchcmd\longtable{\par}{\if@noskipsec\mbox{}\fi\par}{}{}
\makeatother
% Allow footnotes in longtable head/foot
\IfFileExists{footnotehyper.sty}{\usepackage{footnotehyper}}{\usepackage{footnote}}
\makesavenoteenv{longtable}
\usepackage{graphicx}
\makeatletter
\def\maxwidth{\ifdim\Gin@nat@width>\linewidth\linewidth\else\Gin@nat@width\fi}
\def\maxheight{\ifdim\Gin@nat@height>\textheight\textheight\else\Gin@nat@height\fi}
\makeatother
% Scale images if necessary, so that they will not overflow the page
% margins by default, and it is still possible to overwrite the defaults
% using explicit options in \includegraphics[width, height, ...]{}
\setkeys{Gin}{width=\maxwidth,height=\maxheight,keepaspectratio}
% Set default figure placement to htbp
\makeatletter
\def\fps@figure{htbp}
\makeatother
% definitions for citeproc citations
\NewDocumentCommand\citeproctext{}{}
\NewDocumentCommand\citeproc{mm}{%
  \begingroup\def\citeproctext{#2}\cite{#1}\endgroup}
\makeatletter
 % allow citations to break across lines
 \let\@cite@ofmt\@firstofone
 % avoid brackets around text for \cite:
 \def\@biblabel#1{}
 \def\@cite#1#2{{#1\if@tempswa , #2\fi}}
\makeatother
\newlength{\cslhangindent}
\setlength{\cslhangindent}{1.5em}
\newlength{\csllabelwidth}
\setlength{\csllabelwidth}{3em}
\newenvironment{CSLReferences}[2] % #1 hanging-indent, #2 entry-spacing
 {\begin{list}{}{%
  \setlength{\itemindent}{0pt}
  \setlength{\leftmargin}{0pt}
  \setlength{\parsep}{0pt}
  % turn on hanging indent if param 1 is 1
  \ifodd #1
   \setlength{\leftmargin}{\cslhangindent}
   \setlength{\itemindent}{-1\cslhangindent}
  \fi
  % set entry spacing
  \setlength{\itemsep}{#2\baselineskip}}}
 {\end{list}}
\usepackage{calc}
\newcommand{\CSLBlock}[1]{\hfill\break\parbox[t]{\linewidth}{\strut\ignorespaces#1\strut}}
\newcommand{\CSLLeftMargin}[1]{\parbox[t]{\csllabelwidth}{\strut#1\strut}}
\newcommand{\CSLRightInline}[1]{\parbox[t]{\linewidth - \csllabelwidth}{\strut#1\strut}}
\newcommand{\CSLIndent}[1]{\hspace{\cslhangindent}#1}

\setlength\heavyrulewidth{0ex}
\setlength\lightrulewidth{0ex}
\usepackage[automark]{scrlayer-scrpage}
\clearpairofpagestyles
\cehead{
  Brian Weatherson
  }
\cohead{
  Epistemic Modals and Epistemic Modality
  }
\ohead{\bfseries \pagemark}
\cfoot{}
\makeatletter
\newcommand*\NoIndentAfterEnv[1]{%
  \AfterEndEnvironment{#1}{\par\@afterindentfalse\@afterheading}}
\makeatother
\NoIndentAfterEnv{itemize}
\NoIndentAfterEnv{enumerate}
\NoIndentAfterEnv{description}
\NoIndentAfterEnv{quote}
\NoIndentAfterEnv{equation}
\NoIndentAfterEnv{longtable}
\NoIndentAfterEnv{abstract}
\renewenvironment{abstract}
 {\vspace{-1.25cm}
 \quotation\small\noindent\rule{\linewidth}{.5pt}\par\smallskip
 \noindent }
 {\par\noindent\rule{\linewidth}{.5pt}\endquotation}
\KOMAoption{captions}{tableheading}
\makeatletter
\@ifpackageloaded{caption}{}{\usepackage{caption}}
\AtBeginDocument{%
\ifdefined\contentsname
  \renewcommand*\contentsname{Table of contents}
\else
  \newcommand\contentsname{Table of contents}
\fi
\ifdefined\listfigurename
  \renewcommand*\listfigurename{List of Figures}
\else
  \newcommand\listfigurename{List of Figures}
\fi
\ifdefined\listtablename
  \renewcommand*\listtablename{List of Tables}
\else
  \newcommand\listtablename{List of Tables}
\fi
\ifdefined\figurename
  \renewcommand*\figurename{Figure}
\else
  \newcommand\figurename{Figure}
\fi
\ifdefined\tablename
  \renewcommand*\tablename{Table}
\else
  \newcommand\tablename{Table}
\fi
}
\@ifpackageloaded{float}{}{\usepackage{float}}
\floatstyle{ruled}
\@ifundefined{c@chapter}{\newfloat{codelisting}{h}{lop}}{\newfloat{codelisting}{h}{lop}[chapter]}
\floatname{codelisting}{Listing}
\newcommand*\listoflistings{\listof{codelisting}{List of Listings}}
\makeatother
\makeatletter
\makeatother
\makeatletter
\@ifpackageloaded{caption}{}{\usepackage{caption}}
\@ifpackageloaded{subcaption}{}{\usepackage{subcaption}}
\makeatother
\ifLuaTeX
  \usepackage{selnolig}  % disable illegal ligatures
\fi
\IfFileExists{bookmark.sty}{\usepackage{bookmark}}{\usepackage{hyperref}}
\IfFileExists{xurl.sty}{\usepackage{xurl}}{} % add URL line breaks if available
\urlstyle{same} % disable monospaced font for URLs
\hypersetup{
  pdftitle={Epistemic Modals and Epistemic Modality},
  pdfauthor={Brian Weatherson; Andy Egan},
  hidelinks,
  pdfcreator={LaTeX via pandoc}}

\title{Epistemic Modals and Epistemic Modality}
\author{Brian Weatherson \and Andy Egan}
\date{2011}

\begin{document}
\maketitle
\begin{abstract}
This chapter introduces the main themes of the volume, summarizes the
chapters in it, and looks at the various arguments that have been raised
for semantic relativism over the past decade. It concludes that two of
these arguments seem to be resistant to the anti-relativist replies that
have appeared in response to this work on relativism. One of these is an
argument from agreement. It is argued that contextualist theories about
various puzzling locutions have a hard time explaining why it is so easy
for people who would happily utter the same words to describe themselves
as agreeing, if those words were really context-sensitive. Another is an
argument concerning attitude ascriptions. It seems there are quite
different restrictions on what values the (allegedly) context-sensitive
expressions can take inside and outside of attitude ascriptions. Since
this isn't how context-sensitive terms usually behave, this phenomena
tells against contextualism, and in favour of relativism.
\end{abstract}

\setstretch{1.1}
\section{Epistemic Possibility and Other Types of
Possibility}\label{epistemic-possibility-and-other-types-of-possibility}

There is a lot that we don't know. That means that there are a lot of
possibilities that are, epistemically speaking, open. For instance, we
don't know whether it rained in Seattle yesterday. So, for us at least,
there is an epistemic possibility where it rained in Seattle yesterday,
and one where it did not. It's tempting to give a very simple analysis
of epistemic possibility:

\begin{itemize}
\tightlist
\item
  A possibility is an epistemic possibility if we do not know that it
  does not obtain.
\end{itemize}

But this is problematic for a few reasons. One issue, one that we'll
come back to, concerns the first two words. The analysis appears to
quantify over possibilities. But what are they? As we said, that will
become a large issue pretty soon, so let's set it aside for now. A more
immediate problem is that it isn't clear what it is to have de re
attitudes towards possibilities, such that we know a particular
possibility does or doesn't obtain. Let's try rephrasing our analysis so
that it avoids this complication.

\begin{itemize}
\tightlist
\item
  A possibility is an epistemic possibility if for every \emph{p} such
  that \emph{p} is true in that possibility, we do not know that
  \emph{p} is false.
\end{itemize}

If we identify \emph{possibilities} with metaphysical possibilities,
this seems to rule out too much. Let \emph{p} be any contingent claim
whose truth value we don't know. We do know, since it follows from the
meaning of \emph{actually}, that \emph{p} iff actually \emph{p} is true.
But that biconditional isn't true in any world where p's truth value
differs from its actual truth value. So the only epistemic possibilities
are ones where p's truth value is the same as it actually is. But
\emph{p} was arbitrary in this argument, so the only epistemic
possibilities are ones where \emph{every} proposition has the same truth
value as it actually does. This seems to leave us with too few epistemic
possibilities!

A natural solution is to drop the equation of \emph{possibilities} here
with metaphysical possibilities. We've motivated this by using a
proposition that is easy to know to be true, though it isn't true in
many metaphysical possibilities. There are many problems from the other
direction; that is, there are many cases where we want to say that there
is a certain kind of epistemic possibility, even though there is no
matching metaphysical possibility. We'll go through five such examples.

First, there are necessary \emph{a posteriori} claims that arise from
the nature of natural kinds. The standard example here is \emph{Water is
atomic}. That couldn't be true; necessarily, anything atomic is not
water. But until relatively recently, it was an epistemic possibility.

Second, there are claims arising from true, and hence metaphysically
necessary, identity and non-identity statements. A simple example here
is \emph{Hesperus is not Phosphorus}. This could not be true; by
necessity, these celestial bodies are identical. But it was an epistemic
possibility.

Third, there are claims about location. It isn't quite clear what
proposition one expresses by saying \emph{It's five o'clock}, but,
plausibly, the speaker is saying of a particular time that that very
time is five o'clock. It's plausible that if that's true, it's true as a
matter of necessity. (Could this very time have occurred earlier or
later? It doesn't seem like it could have.) So a false claim about what
time it is will be necessarily false. But often there will be a lot of
epistemic possibilities concerning what time it is.

Temporal location raises further matters beyond the necessary \emph{a
posteriori}. We want there to be epistemic possibilities in which it is
four o'clock, five o'clock and so on. But it isn't altogether clear
whether claims like that can be true in metaphysical possibilities. If
we identify a metaphysical possibility with a possible world, then it
isn't clear what would make it the case that it is four o'clock in a
possible world. (What time is it in this possible world?) This might
suggest there are different \emph{kinds} of facts at a metaphysical
possibility as at an epistemic possibility.

Fourth, there are issues about mathematics. Actually, there are two
kinds of puzzle cases here. One concerns propositions that are logical
consequences of our mathematical beliefs, but which we haven't figured
out yet. Twenty years ago, it certainly seemed to be an epistemic
possibility that the equation \(a^n + b^n = c^n\) had positive integer
solutions with \(n > 2\). Now we know that there are no such solutions.
Moreover, if mathematics is necessarily true, then there isn't even a
metaphysical possibility in which there are such solutions. So we
shouldn't think that there was some metaphysical possibility that twenty
years ago we hadn't ruled out. Rather, we were just unsure what
metaphysical possibilities there are.

Finally, there are issues about logic. (Some views on the nature of
logic and mathematics will deny that our fourth and fifth categories are
different.) Getting the metaphysics of taste right is hard. One option
that we think at least can't be ruled out is that intuitionist logic is
the correct logic of taste talk. That is, when it comes to taste, we
don't even know that it's true that everything either is or is not
tasty. But that doesn't mean we're committed to the existence of a
possibility where it isn't true that everything is tasty or not tasty;
if such a state isn't actual, it probably isn't possible. The liar
paradox is even harder than the metaphysics of taste. Anything should be
on the table, even the dialethist view that the Liar is both true and
false. That is, the Liar might be true and false. In saying that, we
certainly don't mean to commit to the existence of some possibility
where the Liar is true and false. We're pretty sure (but not quite
certain!) that no such possibility exists.

The last two cases might be dealt with by being more careful about what
an epistemic possibility is. There are quite simple cases in which we
want to resist the identification of epistemic possibilities with what
we don't \emph{know} to be the case. For discussion of several such
cases, see Hacking (\citeproc{ref-Hacking1967}{1967}), Teller
(\citeproc{ref-Teller1972}{1972}) and DeRose
(\citeproc{ref-DeRose1991}{1991}). If we could very easily come to know
that \emph{p} does not obtain, perhaps because \emph{p} is clearly ruled
out by things we do know, then intuitively it isn't the case that
\emph{p} is epistemically possible. If we know that if \(q\) then not
\emph{p}, and we know \(q\), then \emph{p} is not possible, even if we
haven't put conditional and antecedent together to conclude that
\emph{p} is false. So we need to put some constraints on the
epistemically possible beyond what we know to be false. Perhaps those
constraints will go so far as to rule out anything inconsistent with
what we know. In that case, it wasn't possible all along that Fermat's
Last Theorem was false. And, assuming the non-classical approaches to
taste and alethic paradoxes are incorrect, those approaches aren't even
possibly correct. We're not endorsing this position, just noting that it
is a way to rescue the idea that all epistemic possibilities are
metaphysical possibilities.

The papers in this volume that most directly address these issues are by
Frank Jackson, David Chalmers and Robert Stalnaker. Jackson argues
against the view that accounting for epistemic possibilities requires us
to think that there is a kind of possibility, conceptual possibility,
that is broader than metaphysical possibility. He briefly reviews the
reasons some people have had for taking this position, including those
we've just discussed, and some of the reasons he rejected it in ``From
Metaphysics to Ethics''. But he adds some new arguments as well against
this position, what he calls the `two space' view of possibility. One
argument says that if there is a possibility of any kind where water is
not H\textsubscript{2}O, then being water and being H\textsubscript{2}O
must be different properties by Leibniz's Law. But then we have an
implausible necessary connection between distinct properties. Another
argument turns on the difficulty of identifying the water in these
supposed conceptual possibilities that are not metaphysically possible.

David Chalmers discusses what kind of thing epistemic possibilities, or
as he calls them, `scenarios', might be. He discusses the strengths,
weaknesses and intricacies of two proposals: what he calls the
`metaphysical' and `epistemic' constructions. The metaphysical
construction is fairly familiar: it takes epistemic possibilities to be
centered possible worlds. The epistemic construction takes epistemic
possibilities to be maximally possible sentences of a specially
constructed language. The metaphysical construction requires several
assumptions before it matches up with the intuitive notion of epistemic
possibility, while the epistemic construction requires a primitive
notion of epistemic possibility. But both constructions seem to
illuminate the elusive notion of an epistemic possibility. Chalmers ends
with a discussion of several applications of his constructions in
semantics, in formal epistemology and in moral psychology.

Another place where one finds an important role for a distinctively
epistemic (or at least doxastic) sort of possibility is in theorizing
about indicative conditionals. In Robert Stalnaker's contribution, he
examines two types of accounts of indicative conditionals, which differ
in where they locate the conditionality. One view analyzes assertions of
indicative conditionals as a special sort of \emph{conditional
assertion}, and another analyzes them as an ordinary assertion of a
special sort of \emph{conditional proposition}. Stalnaker argues that
the two views are not so different as we might initially have thought.

\section{Three Approaches to Epistemic
Modals}\label{three-approaches-to-epistemic-modals}

Even when we settle the issue of what epistemic possibilities are, we
are left with many issues about how to talk about them. Speakers will
often say that something is (epistemically) possible, or that it might
be true. (It's plausible that claims that \emph{p} must be true, or that
\emph{p} is probable, are closely related to these, but we'll stick to
claims about (epistemic) possibility at least for this introduction.)
It's plausible to think that a proposition isn't possible or impossible
simpliciter, it's rather that it is possible or impossible relative to
some person, some group, some evidence or some information. Yet
statements of epistemic possibility in plain English do not make any
explicit reference to such a person, group, evidence set or information
state. One of the key issues confronting a semanticist attempting to
theorise about epistemic modals is what to do about this lack of a
reference. We'll look at three quite different approaches for dealing
with this lack: contextualist, relativist and expressivist.

\subsection{Contextualism}\label{contextualism}

Consider a particular utterance, call it \emph{u}, made by speaker
\emph{S}, of the form \emph{a might be F}, where the might here is
intuitively understood as being epistemic in character. To a first
approximation, the sentence is saying \emph{a}'s being \emph{F} is
consistent with, or not ruled out by, a certain body of knowledge. But
whose body of knowledge? Not God's, presumably, for then \emph{a might
be F} would be true iff \emph{a is F} is true, and that's implausible.
The contextualist answer is that the relevant body of knowledge is
supplied by context.

When discussing the ways in which context fills in content, some writers
will start with the pronoun \emph{I} as an example. And to some extent
it's a useful example. The sentence \emph{I am a fool} doesn't have
truth-conditional content outside of a context of utterance. But any
utterance of that sentence does express something truth conditional.
Which truth conditional sentence it expresses is dependent on facts
about the context of its use. In fact, it is dependent on just one fact,
namely who utters it. So when Andy utters \emph{I am a fool} he
expresses the proposition that Andy is a fool. And when Brian utters
\emph{I am a fool} he expresses the proposition that Brian is a fool.

So far \emph{I} is a useful example of a context-sensitive expression.
But in many ways it is an unusual example of context-sensitivity, and
focussing too much on it can lead to an overly simplistic view of how
context-sensitive terms work. In particular, \emph{I} has three
properties that are unusual for a context-sensitive expression.

\begin{itemize}
\tightlist
\item
  Its content in a context is computable from the context by a simple
  algorithm - namely the content is the speaker.
\item
  Its content does not depend on any properties of the intended audience
  of the utterance.
\item
  It behaves exactly the same way in embedded and unembedded contexts.
\end{itemize}

Some terms have none of these properties. Consider, for example,
\emph{we}.

There isn't any obvious algorithm for computing the content of a
particular use of \emph{we}. The content may depend on the intentions of
the speaker. It may depend on which people have been talked about. In
sentences of the form \emph{We are F}, different values of \emph{F}
might constrain what values can be rationally assigned to \emph{we}. And
when that is so, the interpretation of \emph{we} will (usually) be
constrained to those groups.

Perhaps most notably, it depends a lot on the audience. If \emph{S} is
talking to \emph{H}, and says \emph{We should grab some lunch}, the
content is that \emph{S} and \emph{H} should grab some lunch. And that's
the content because \emph{H} is the intended audience of the utterance.
Intended audiences can change quickly. If Andy says \emph{We will finish
the paper this afternoon, then we will go for a walk}, talking to Brian
when he utters the first conjunct, and Fido when he utters the second,
the content is that Andy and Brian will finish the paper this afternoon,
then Andy and Fido will go for a walk.

That we has neither of the first two properties is uncontroversial. What
is perhaps a little more controversial is that it does not have the
third either. When we is in an unembedded context it (usually) behaves
like a free (plural) variable. Under certain embeddings, it can behave
like a bound variable. Barbara Partee and Phillipe Schlenker offer the
following examples.

\begin{description}
\item[(5.9)]
John often comes over for Sunday brunch. Whenever someone else comes
over too, we (all) end up playing trios.
(\citeproc{ref-Partee1989}{Partee 1989})
\item[(5.10)]
Each of my colleagues is so difficult that at some point or other we've
had an argument. (\citeproc{ref-Schlenker2003}{Schlenker 2003})
\end{description}

In neither case does we contribute a group consisting of the speaker
plus some salient individuals. Indeed, in neither case does it so much
as contribute a group, since it is (or at least behaves like) a bound
variable. There's nothing in the contextualist story about we that
prevents this.

It's worthwhile reviewing these facts about we, because on the most
plausible contextualist stories about might, it too has these three
properties. The contextualist theory we have in mind says that the
content of \emph{u} is \emph{For all that group X could know using
methods M,} \emph{a} is \emph{F}. The group \emph{X} will usually
consist of the speaker and some salient others, perhaps including the
intended audience. The salient methods might include little more than
easy deduction from what is currently known, or may include some wider
kinds of investigation. (See DeRose (\citeproc{ref-DeRose1991}{1991})
for arguments that the relevant methods include more than deduction, and
that they are contextually variable.)

Now it isn't part of the contextualist theory that there is an easy
method for determining who is in \emph{X}, or what methods are in
\emph{M}. So in that respect \emph{might} is like \emph{we}. But, just
as the group denoted by we typically includes the intended audience of
the utterance, the group \emph{X} will typically include the intended
audience of \emph{u}. And the methods \emph{M} will typically include
any method that can be easily carried out. This can be used to explain
some phenomena about disagreement. So if Andy says, to Brian, \emph{a
might be F}, and Brian knows that \emph{a} is not \emph{F} (or can
easily deduce this from what he knows), Brian can disagree with what
Andy says. That is, he can disagree with the proposition that it is
consistent with what members of the conversation know that \emph{a} is
\emph{F}. And, the contextualist says, that's just what Andy did say. If
Brian presents Andy with his grounds for disagreement, Andy might well
retract what he said. Since arguments about disagreeing with utterances
like \emph{u} have been prominent in the literature, it is worth noting
that the contextualist theory can explain at least some facts about
disagreement.

Nor is it part of the contextualist theory that might behaves exactly
the same way in embedded and unembedded contexts. Indeed, like
\emph{we}, \emph{might} can behave like a bound variable. On the most
natural reading of \emph{Every pedestrian fears that they might be being
watched}, there is no single group \emph{X} such that every pedestrian
fears that for all \emph{X} (could easily) know, that pedestrian is
being watched. Rather, every pedestrian fears that for all they
themselves know, they are being watched. The naturalness of this reading
is no embarrassment to the contextualist theory, since it is a common
place that terms that usually get their values from context can also, in
the right setting, behave like bound variables.

Indeed, thinking about these parallels between context-sensitive
expressions and epistemic modals seems to provide some support for
contextualism. In his contribution to the volume, Jonathan Schaffer
argues that various features of the way epistemic modals behave in
larger sentences support the idea that an evaluator place must be
realised in the syntax. For instance, consider the natural
interpretation of ``Anytime you are going for a walk, if it might rain,
you should bring an umbrella.'' We interpret that as saying that
whenever you go for a walk, you should bring an umbrella if your
evidence at that time is consistent with rain. Schaffer interprets that
as evidence that there is hidden syntactic structure in epistemic
modals, and argues that the contextualist offers the best account of how
the hidden structure gets its semantic values.

So the contextualist has a lot of explanatory resources, and a lot of
flexibility in their theory, which are both clear virtues. But there are
some limits to the flexibility. There are some things that the
contextualist, at least as we're using the term `contextualist' is
committed to. In particular, the contextualist is committed to the
content of a particular speech act (or at least of a particular
assertion) is absolute, not assessor-relative. And they're committed to
the truth value of those contents being the same relative to any
assessor. Let's give those two commitments names.

\begin{description}
\item[(C)]
The semantic content of an assertion is the same relative to any
assessors.
\item[(T)]
The truth value of the semantic content of an assertion is the same
relative to any assessors.
\end{description}

The first of these rules out the possibility that the semantic content
of an assertion differs with respect to different groups. The second
rules out the possibility that semantic contents have assessor relative
truth values. Modern relativists have proposed theories that dispense
with these dogmas, and we'll investigate those in the next section,
after going over some of the motivations for relativism.

\subsection{Relativism}\label{relativism}

In many fields, relativism is motivated by instances of ``faultless
disagreement'', and epistemic modals are not left out of this trend.
Here is the kind of case that we used in Egan, Hawthorne, and Weatherson
(\citeproc{ref-Egan2005-EGAEMI}{2005}) to motivate relativism.

\begin{quote}
Consider the following kind of case. Holmes and Watson are using a
primitive bug to listen in on Moriarty's discussions with his underlings
as he struggles to avoid Holmes's plan to trap him. Moriarty says to his
assistant

\begin{description}
\item[(24)]
Holmes might have gone to Paris to search for me.
\end{description}

Holmes and Watson are sitting in Baker Street listening to this. Watson,
rather inexplicably, says ``That's right'' on hearing Moriarty uttering
(24). Holmes is quite perplexed. Surely Watson knows that he is sitting
right here, in Baker Street, which is definitely not in Paris.
\end{quote}

Here we have Watson somewhat surprisingly agreeing with Moriarty. In
some sense, it seems wrong for him to have done so. He should have
disagreed. Well, imagine that he did, by saying ``That's not right''.
The quick argument for relativism is that the contextualist cannot make
sense of this. Whatever group's knowledge Moriarty intended to be
talking about when he spoke, it presumably didn't include Holmes and
Watson; it just included him and his intended audience, i.e.~the
underlings. And it's true that for all they know, Holmes is in Paris. So
the content of Moriarty's utterance is true. But it seems that Watson
can properly disagree with it (and can't properly agree with it). That,
we thought, was a problem.

There are three kinds of response to this argument on behalf of the
contextualist that we think look promising. All of these responses are
discussed in Fintel and Gillies (\citeproc{ref-vonFintel2008}{2008}). We
might look harder at the denotation of the `that' in Watson's reply, we
might think again about what the relevant group is, and we might look at
other cases where the contextualist story is more promising, as a way of
motivating the first two responses. Let's look at these in turn.

Above we said that Watson disagreed with Moriarty by saying ``That's not
right''. But that's potentially reading too much into the data. What
seems correct is that Watson can say ``That's not right''. But that's
only to disagree with Moriarty if the `that' denotes what Moriarty said.
And that might not be true. It's possible that it picks out, say, the
embedded proposition that Holmes has gone to Paris. And it's fine for
Watson to disagree with that.

Even if Watson is disagreeing with the semantic content of Moriarty's
utterance, it might be that he's doing so properly, because what
Moriarty said is false. That might be the case because it might be that,
in virtue of hearing the utterance, Watson became part of the relevant
group \emph{X}. Typically speaker intentions, particularly singular
speaker intentions, are not the final word in determining the content of
a context-sensitive expression. If Brian points over his shoulder,
thinking a nice glass of shiraz is behind him, and says \emph{That is
tasty}, while in fact what he is pointing at is a vile confection of
Vegemite infused chardonnay, he's said something false. The simplest
thing to say about a case like this is that Brian intended the
denotation of `That' to be the thing he was pointing at, whatever it is.
Similarly, Moriarty might have intended the relevant group \emph{X} to
be whoever heard the utterance at that time, even if he didn't know
Watson was in that group. (Or it might be that, whatever Moriarty's
intentions, the semantic rules and conventions for `might' in English
determine that the relevant group includes everybody who heard the
utterance at the time.)

This second response would seem somewhat ad hoc were it not for a class
of examples von Fintel and Gillies describe concerning assessors from
radically different contexts. Typically the anti-contextualist
commentary on cases like these suggest that any hearer who knows that
\emph{a} is not \emph{F} can disagree with \emph{u}. But that doesn't
seem to be in general true.

\begin{quote}
Or consider the case of Detective Parker. He has been going over some
old transcripts from Al Capone's court case in the 1920s--Capone is
being asked about where some money is in relation to a particular safe:

\begin{description}
\item[(20)]
\begin{enumerate}
\def\labelenumi{\arabic{enumi}.}
\item[]
\item
  Capone: The loot might be in the safe.
\item
  Parker: ??Al was wrong/What Al said is false. The safe was cracked by
  Geraldo in the 80s and there was nothing inside.
  (\citeproc{ref-vonFintel2008}{2008, 86})
\end{enumerate}
\end{description}
\end{quote}

The knowledge of at least some hearers, such as Detective Parker, does
not seem to count for assessing the correctness of Capone's speech. A
contextualist might suggest that's because contemporaneous hearers are
in the relevant group, and later reviewers are not.

So there are definitely some contextualism-friendly lines of response
available to the argument for relativism from disagreement. But
interestingly, some of these contextualist responses do not work as well
as a response to a similar argument from agreement. Imagine that Andy,
after doing some reading on the publicly available evidence, correctly
concludes that it doesn't rule out Prince Albert Victor. He doesn't
think this is very likely, but thinks it is possible. Andy hears someone
on TV talking about the Ripper who says ``Prince Albert Victor might
have been Jack the Ripper'', and Andy says ``That's true''. Intuitively
Andy is right to agree with the TV presenter, but this is a little hard
to explain on the contextualist theory.

Note that here we can't say that Andy is agreeing because he is agreeing
with the embedded proposition, namely that Prince Albert Victor was the
Ripper. That's because he doesn't agree with that; he thinks it is an
open but unlikely possibility.

Nor does it particularly matter that Andy, as one of the people watching
the TV show, is part of the relevant group \emph{X}. All that would show
is that if Andy knew Prince Albert Victor wasn't the Ripper, the
presenter's assertion is false. But unless Andy is the group \emph{X},
the fact that Andy's knowledge, or even what is available to Andy, does
not rule out the Prince does not mean Andy should agree with the
statement. For all Andy knows, someone else watching, perhaps even
someone else the presenter intends to include in her audience, has
evidence exculpating the Prince. If that's right, then he does not know
that the proposition the contextualist says the speaker asserted is
true. But yet he seems justified in agreeing with the presenter. This
seems like a real problem for contextualism.

A quite different objection to contextualism comes from metasemantic
considerations. The most casual reflection on the intuitive content of
utterances like \emph{u} suggests there is staggeringly little rhyme or
reason to which group \emph{X} or method \emph{M} might be relevant. The
argument here isn't that the contextualist's semantic proposal is
mistaken in some way. Rather, the argument is that the accompanying
metasemantic theory, i.e.~the theory of how semantic values get fixed,
is intolerably complicated. Slightly more formally, we can argue as
follows.

\begin{enumerate}
\def\labelenumi{\arabic{enumi}.}
\tightlist
\item
  If contextualism is true, the metasemantic theory of how a particular
  use of `might' gets its semantic value is hideously complicated.
\item
  Metasemantic theories about how context-sensitive terms get their
  values on particular occasions are never hideously complicated.
\item
  So, contextualism is false.
\end{enumerate}

The problem with this argument, as Michael Glanzberg
(\citeproc{ref-Glanzberg2007}{2007}) has argued, is that premise 2 seems
to be false. There are examples of uncontroversially context-sensitive
terms, like `that', for which the accompanying metasemantic theory is,
by any standard, hideously complicated. So the prospects of getting to
relativism from metasemantic complexity are not, we think, promising.

But there is a different metasemantic motivation for relativism that we
think is a little more promising. Compare the difference between (1) and
(2).

\begin{description}
\tightlist
\item[(1)]
Those guys are in trouble, but they don't know that they are. (2)

??Those guys are in trouble, but they might not be.
\end{description}

Something has gone wrong in (2). This suggests that (2) can't be used to
express (1). That is, there's no good interpretation of (2) where those
guys are the group \emph{X}. This is a little surprising, since we've
made the guys pretty salient. Cases like this have motivated what we
called the Speaker Inclusion Constraint (hereafter SIC) in ``Epistemic
Modals in Context''. That is, in unembedded uses of `might' the group X
always includes the speaker. Now the explanation of the problem with (2)
is that for the speaker to assert the first clause, she must know that
the guys are in trouble, but if that's the case, and she's in group X,
then the second clause is false.

Now a generalisation like this doesn't look like it should be grounded
in the meaning (in some sense of `meaning') of `might'. For comparison,
it seems to be part of the meaning of `we' that it is a first-person
plural pronoun. It isn't just a metasemantic generalisation that the
speaker is always one of the group denoted by `we'. By analogy, it is
part of the meaning of `might' that the speaker is always part of the
group \emph{X}.

Further, when the meaning of a context-sensitive expression constrains
its value, those constraints still hold when the term is used as a bound
variable. For instance, it is part of the meaning of `she' that it
denotes a female individual. If Smith is male, then the semantic content
of \emph{She is happy} can't be that Smith is happy. Similarly, when
`she' is behaving like a bound variable, the only values it can take are
female individuals. So we can't use \emph{Every student fears she will
fail the test} to quantify over some students some of whom are male. And
there's no interpretation of \emph{Every class hopes we will win} where
it means that every class hopes that that class will win. Even when
under a quantifier and an attitude ascribing verb, `we' must still pick
out a group that includes the speaker. The natural generalisation is
that constraints on context supplied by meaning do not get overridden by
other parts of the sentence.

The problem for contextualists about `might' is that it doesn't behave
as you'd expect given these generalisations. In particular, the SIC
doesn't hold when `might' is in certain embeddings. So there is a
reading of \emph{Every student fears they might have failed} where it
means that every student fears that, for all they know, they failed. The
knowledge of the speaker isn't relevant here. Indeed, even if the
speaker knows that many students did not fail, this sentence can be
properly uttered. This suggests the following argument.

\begin{enumerate}
\def\labelenumi{\arabic{enumi}.}
\tightlist
\item
  If contextualism is true, then the explanation of the SIC is that it
  is part of the meaning of `might' that the relevant group X includes
  the speaker.
\item
  If it is part of the meaning of `might' that the relevant group X
  includes the speaker, then this must be true for all uses of `might',
  included embedded uses.
\item
  When `might' is used inside the scope of an attitude ascription, the
  relevant group need not include the speaker.
\item
  So, contextualism is not true.
\end{enumerate}

Premise 1 would be false if the metasemantics was allowed to be
systematic enough to explain why the SIC holds even though it is not
part of the meaning. Premise 2 would be false if we allowed `might' to
have a systematically different meaning inside and outside the scope of
attitude ascriptions. And premise 3 would be false if any attitude
ascriptions that are made are, contrary to intuition, tacitly about the
speaker's knowledge. Since none of these seems particularly plausible,
there does seem to be a problem for contextualism here.

In their contribution to this volume, Kai von Fintel and Thony Gillies
reject one of the presuppositions of the argument we've just presented.
Classical contextualism, what they call `the canon', says that context
picks out a particular group, and an utterance of `It might be that
\emph{p}' is true iff that group's information is consistent with
p.~That's what we've taken as the stalking horse in this section, and
von Fintel and Gillies are certainly right that it is the canonical
version of contextualism. Von Fintel and Gillies agree that the broad
outline of this contextualist story is correct. But they deny that
context picks out a determinate group, or a determinate body of
information. Rather, uttering an epistemic modal will `put into play' a
number of propositions of the form `For all group \emph{G} knows,
\emph{p}'. This ambiguity, or perhaps better indeterminacy, is crucial
they argue to the pragmatic role that epistemic modals play. And once we
are sensitive to it, they claim, we see that contextualism has more
explanatory resources than we'd previously assumed, and so the
motivation for relativism fades away.

In summary, there are four motivations for relativism that have been
floated in the literature. These are:

\begin{itemize}
\tightlist
\item
  Intuitions about disagreement;
\item
  Intuitions about agreement;
\item
  Arguments from metasemantic complexity; and
\item
  Arguments from semantic change in attitude ascriptions,
\end{itemize}

As noted, the third argument doesn't seem very compelling, and it is a
fairly open question whether the first works. But the second and fourth
do look like good enough arguments to motivate alternatives.

\subsection{Two Kinds of Relativism}\label{two-kinds-of-relativism}

We said above that contextualism is characterised by two theses,
repeated here for convenience.

\begin{description}
\item[(C)]
The semantic content of an assertion is the same relative to any
assessors.
\item[(T)]
The truth value of the semantic content of an assertion is the same
relative to any assessors.
\end{description}

So there are two ways to not be a relativist, deny (C) and deny (T). One
might deny both, but we'll leave that option out of our survey.

What we call \emph{content relativism} denies (C). The picture is that
contextualists were right to posit a variable \emph{X} in the structure
of an epistemic modal claim. But the contextualists were wrong to think
that \emph{X} gets its value from the context of utterance. Rather, the
value of \emph{X} is fixed in part by the context of assessment. In the
simplest (plausible) theory, \emph{X} is the speaker and the assessor.
So if Smith asserts that Jones might be happy, the content of that
assertion, relative to Andy, is that for all Smith and Andy know, Jones
is happy, while relative to Brian its content is that for all Smith and
Brian know, Jones is happy.

The primary motivation for content relativism is that it keeps quite a
bit of the contextualist picture, while allowing enough flexibility to
explain the phenomena that troubled contextualism. So for the content
relativist, contents are exactly the same kinds of propositions as the
contextualist thinks they are. So we don't need to tell a new kind of
story about what it is for a content to be true, to be accepted, etc.
Further, because we keep the variable \emph{X}, we can explain the
`bound variable' readings of epistemic modals discussed in the first
section.

A worry about content relativism is that the `metasemantic' argument
against contextualism might equally well tell against it. The worry
there was that the constraints on \emph{X} seemed to depend, in an
unhappy way, on where in the sentence it appeared. The content
relativist has a move available here. She can say that as a rule,
whenever there's a variable like \emph{X} attached to a term, and that
term is in an attitude ascription, then the variable is bound to the
subject of the ascription. This might be an interesting generalisation.
For instance, if she is a content relativist about both epistemic modals
and predicates of personal taste, she has a single explanation for why
both types of terms behave differently inside and outside attitude
ascriptions.

There are two interesting `near cousins' of content relativism. One is a
kind of \emph{content pluralism}. We might hold (a) that an assertion's
content is not relative to an assessor, but (b) some assertions have
many contents. So if \emph{S} says \emph{a might be F}, and this is
assessed by many hearers, \emph{S} asserts \emph{For all s and h know,}
\emph{a} is \emph{F}, for each \emph{H} who hears and assesses the
speech. Now when a hearer \emph{h}\textsubscript{1} does this, she'll
probably focus on one particular content of \emph{S}'s assertion, namely
that \emph{For all s and h}\textsubscript{1} know, \emph{a} is \emph{F}.
But the content pluralist accepts (while the content relativist denies)
that even relative to \emph{h}\textsubscript{1}, \emph{S}'s assertion
also had the content \emph{For all s and h}\textsubscript{2} know,
\emph{a} is \emph{F}, where \emph{h}\textsubscript{2} is a distinct
assessor.

Another near cousin is the view, defended in this volume by Kent Bach,
that the semantic content of an epistemic modal is typically not a
complete proposition. In the case just described, it might be that the
semantic content of what \emph{S} says is \emph{For all \_\_\_\_ knows,}
\emph{a} is \emph{F}, and that's not a proposition. Now a given hearer,
\emph{H}, might take \emph{S} to have communicated to them that
\emph{For all s and h know,} \emph{a} is \emph{F}, but that's not
because that's the semantic content of what \emph{S} says. It's not the
absolute content (a la contextualism), the content relative to \emph{H}
(a la content relativism) or one of the contents (a la content
pluralism).

It's a very big question how we should discriminate between these
theories. Some readers may even worry that there is no substantive
differences between the theories, they are in some sense saying the same
thing in different words. One big task for future research is to clearly
state the competing theories in the vicinity of here, and find arguments
that discriminate between them.

A quite different kind of relativism denies (T). This view says that the
content itself of an assertion can be true for some assessors, and false
for others. Such a view is not unknown in recent philosophy. In the
1970s and 1980s (and to a lesser extent in subsequent years) there was a
debate between temporalists and eternalists about propositions. The
temporalists thought that a tensed proposition, i.e.~the content of a
tensed assertion, could be true at one time and false at another. The
eternalists denied this, either taking truth to be invariant across
times, or in some cases denying that it even made sense to talk about
truth being relative to something, e.g.~a time.

Contemporary forms of truth relativism generalise the temporalist
picture. The temporalists thought that propositions are true or false
relative to a world-time pair. Modern relativists think that
propositions are true or false relative to a world-assessor pair, or
what loosely following Quine (\citeproc{ref-Quine1969}{1969}) we might
call a centered world. (Quine used this to pick out any world-place-time
triple, but since most times and places don't have assessors at them,
world-assessor pairs, or even world-assessor-time triples, are more
restricted.) For example, as a first pass at a truth-relativism about
predicates of personal taste, one might propose that the proposition
expressed by a typical utterance of `beer is tasty' will be true at any
centered world where the person at the center of the world likes the
taste of beer.

The truth relativist has an easy explanation of the data that motivated
the rejection of contextualism. Recall two puzzles for the contextualist
about terms like `tasty': that it is so easy to agree with claims about
what's tasty, and that reports of the form \emph{X} thinks that beer is
tasty are always about \emph{X}'s attitude towards beer, not about
\emph{X}'s beliefs about how the speaker finds beer.

On the first puzzle, note that if to agree with an assertion is to agree
with its propositional content, and that content is true at the center
of your world iff you find beer tasty, then to agree with an assertion
that beer is tasty, you don't have to launch an inquiry into the
sincerity of the speaker, you just have to check whether you like beer.
If you're in a world full in insincere speakers, and abundant beer,
that's relatively easy.

On the second puzzle, if propositional attitude ascriptions report the
subject's attitude towards a proposition, and if a proposition is a set
of centered worlds, then the subject's attitude towards `Beer is tasty'
should be given by their attitude towards whether that proposition is
true in their centered world. That is, it should be given by their
attitude towards beer. And that's just what we find.

The extension of all this to epistemic modals is more or less
straightforward. The simplest truth relativist theory says that an
utterance of the form \emph{a might be F} is true iff, for all the
assessor at the center of the world knows, \emph{a} is \emph{F}. As
Richard Dietz (\citeproc{ref-Dietz2008}{2008}) has pointed out, this
won't do as it stands. If the speaker knows \emph{a} is not \emph{F},
then their utterance seems like it should be false relative to everyone.
(Conversely, a speaker who knows \emph{a} is \emph{F} speaks truly,
relative to any assessor, when they say \emph{a must be F}.) If we're
convinced of this, the solution is a mild complication of the theory.
The utterance is both somewhat context-sensitive, and somewhat relative.
So \emph{S}'s utterance of \emph{a might be F} is true at a centered
world iff for all \emph{S} plus the person at the center of the world
know, \emph{a} is \emph{F}. We might want to add more complications (is
it knowledge that matters or available information, for example?) but
that's one candidate truth relativist theory.

There are three worries we might have about truth relativism. One is a
very big picture worry that the very notion of truth being relative is
misguided. This is a theme of Herman Cappelen and John Hawthorne's
\emph{Relativism and Monadic Truth} . Another is that it overgenerates
`explanations'. We can't explain cases like the Capone/Parker example.
And a third is that, by making propositions so different from what we
thought they were, we'll have to redo a lot of philosophy of language
that presupposed propositions have the same truth value for everyone. In
particular, we'll have to rethink what an assertion is. (That challenge
is addressed -- in different ways -- in recent work by John MacFarlane
and by Andy Egan.)

The strongest defence of relativism in this volume comes from John
MacFarlane. His work on tense
(\citeproc{ref-MacFarlane2003-MACFCA}{MacFarlane 2003}), and on
knowledge attributions
(\citeproc{ref-MacFarlane2005-Knowledge}{MacFarlane 2005b}), and on the
broader philosophical status of relativism and other rivals to classical
contextualism (\citeproc{ref-MacFarlane2005-MACMSO}{MacFarlane 2005a},
\citeproc{ref-MacFarlane2009-MACNC}{2009}), have been immensely
influential in the contemporary debates. Here he develops a relativistic
semantics for epistemic modals, along the lines of the proposals he has
offered elsewhere for tense and knowledge attributions. He argues that
many phenomena, several of which we've discussed in this introduction,
raise trouble for contextualism and promote relativism. These phenomena
include third-party assessments, retraction and disagreement. He argues
that only the relativist can explain the troublemaking phenomena.

\subsection{Expressivism}\label{expressivism}

So far we've looked at two of our three major approaches to epistemic
modals. The contextualist says that which proposition is asserted by an
epistemic modal depends crucially on the context of utterance. The
relativist says that the contextualist is ignoring the importance of the
context of assessment. The content relativist says that they are
ignoring the way in which the context of assessment partially determines
what is said. The truth relativist says that they are ignoring the way
in which propositions uttered have different truth values at different
contexts of assessment.

The expressivist thinks that there is a common assumption behind all of
these theories, and it is a mistaken assumption. The assumption is that
when we're in the business of putting forward epistemic modals, we're in
the business of asserting things that have truth values. The
expressivist rejects that assumption. They say that when we say \emph{a
might be F}, we're not asserting that we are uncertain about whether
\emph{a} is \emph{F}, we're expressing that uncertainty directly. The
contextualists and relativists think that in making these utterances,
we're expressing a second-order belief, i.e.~a belief about our own
knowledge, or lack thereof. The expressivists think we're expressing a
much simpler mental state: uncertainty.

One way to motivate expressivism is to start with the anti-contextualist
arguments, and then argue that relativism is not an acceptable way out.
So we might, for instance, start with the argument from agreement. The
expressivist notes that there are many ways to agree with a statement.
If Smith says `Let's have Chinese for dinner', and Jones agrees, there
need not be any proposition that Smith asserted that Jones is agreeing
to. We're happy to call all sorts of meetings of minds agreements. So
the agreement phenomena that the contextualist can't explain, the
expressivist can explain. When Smith says `Brown might be a spy', and
Jones agrees, there isn't necessarily any proposition they both accept.
Rather, their agreement consists in having a common mental state, namely
uncertainty about whether Brown is a spy.

The expressivist may then run out any number of arguments against
relativism. For instance, they might argue (against content relativism)
that it is a requirement of a speech act being an assertion that it have
a determinate content. And they might argue, perhaps motivated by
theoretical considerations about the role of assertions in conversation,
that contents which vary in truth value among hearers couldn't be
contents of assertions. If true, that would rule out truth relativism.
We're moved, perhaps by elimination as much as anything, to
expressivism.

There are more direct arguments for expressivism as well. Isaac Levi
(\citeproc{ref-Levi1996}{1996, 55}) motivated a view on which epistemic
modals don't have truth values by thinking about learning. Imagine
someone previously thought that Brown might be a spy, perhaps on quite
good grounds, then they learn that he is not a spy. If that's all they
learned, then it seems odd to say that there's something that they
previously knew, that now they don't know. It seems learning shouldn't
destroy knowledge. That's what happens in standard models for belief
revision (which were one of Levi's primary concerns) and it is
independently plausible. But if epistemic modals express propositions,
and those are true or false, then there is a proposition that the person
did know and now don't know, namely that Brown might be a spy.

There are clearly a few possible responses to this argument. For one
thing, we could make the epistemic modal claims explicitly tensed. Both
before and after the learning experience, the subject knew that Brown
might, at \emph{t}\textsubscript{1}, have been a spy, but didn't know
that Brown might, at \emph{t}\textsubscript{2}, have been a spy.
(Indeed, they learned that that was false.) Or, and this is more in
keeping with the spirit of this introduction, we might spell out the
epistemic modal claim. Before and after the learning experience, the
subject knew that it was consistent with everything the subject knew
prior to the learning experience that Brown was a spy. So there's no
information lost.

The problem with this move is that it seems to make epistemic modals
overly complex. Intuitively, it is possible for a child to grasp a
modal, and for the most natural interpretation of that modal to be
epistemic, without the child having the capacity to form second order
thoughts. (This point is one that Seth Yalcin uses in his argument for a
kind of expressivism in this volume.) This question seems like it would
be good to test empirically, though we don't know of any existing
evidence that settles the question. Introspectively, it does seem that
one can think that the cat might be in the garden without thinking about
one's own epistemic or doxastic states as such. Those kinds of
introspections might tell in favour of an approach which identifies
epistemic modality with a distinct kind of relation to content, rather
than a distinct kind of content.

Following important work by Allan Gibbard
(\citeproc{ref-Gibbard1990}{1990}), there is a natural way to formalise
an expressivist theory of epistemic modality. Identify a `context' with
a set of propositions. Sentences, whether epistemic modals or simple
sentences, are satisfied or unsatisfied relative to world-context pairs,
where a world and a context make a pair iff every proposition in the
context is true at that world. Then an epistemic modal, say \emph{Brown
might be a spy}, is satisfied by such a pair iff Brown is a spy is
consistent with everything in the context. A simple sentence, like
\emph{White is a spy} is satisfied by such a pair iff White is a spy is
true at the world. The pairing becomes useful when considering, say,
conjunctions. A conjunction is satisfied iff both conjuncts are
satisfied. So \emph{White is a spy and Brown might be} is satisfied by a
world-context pair iff White is a spy at the world, and Brown's being a
spy is consistent with the context.

So far this looks a lot like relativism. A world-context pair is just
like a centered world, with the context being what's known by the person
at the center of the world. If we apply the formalism to real-life
cases, perhaps taking the contexts to be genuine contexts in the sense
of Stalnaker (\citeproc{ref-Stalnaker1978}{1978}), the two formalisms
might look very close indeed.

But there is, or at least we hope there is, a substantive philosophical
difference between them. The expressivist has a restricted sense of what
it is to make an assertion, and of what it is for an expression to be an
expression of a truth. The expressivist most insistently does not
identify satisfaction with truth. The only sentences that are true or
false are sentences that are satisfied by a world context pair
\(\langle w,c_1 \rangle\) iff they are satisfied by every other pair
starting with the same world. The expression of such a sentence, and
perhaps only of such a sentence, constitutes an assertion. Otherwise it
constitutes some other speech act.

And this is no mere difference in how to use the words `truth',
`assertion' and so on. Nor is it even just a difference about truth and
assertion and so on. It hopefully makes a difference to what predictions
we make about the way epistemic modals embed, especially how they embed
in propositional attitude ascriptions. We used that fact to argue
against expressivism in ``Epistemic Modals in Context'', since we
thought there were in some cases more examples of successful embedding
of epistemic modals, especially in conditionals, than the expressivist
would predict. On the other hand Seth Yalcin uses facts about embedding
to argue, in his paper in this volume, in favour of expressivism. He
argues that on a non-expressivist view, we should be able to suppose
that \emph{p} is true but might not be true, and that can't be supposed.

This argument is part of the argument by elimination that Yalcin against
what he calls `descriptivism' about epistemic modals in this
contribution to the volume. He uses `descriptivism' to pick out a broad
category of theories about epistemic modals that includes both
contextualism and relativism. He argues against all descriptivist views,
and in favour of what he calls `expressivism'. He says that when someone
utters an epistemic modal, they do not describe their own knowledge (or
the knowledge of someone else), rather they express their own mental
state. Some of Yalcin's arguments for expressivism are related to
arguments against contextualism; in particular he thinks like we do that
there isn't a viable form of contextualism. But he also thinks that
there are problems for relativism, such as the difficulty in supposing
Moore paradoxical propositions. He also notes that it is a puzzle for
descriptivists to make sense of belief ascriptions involving epistemic
modals. On a descriptivist model, a sentence like `\emph{X} believes
that it might be that \emph{p}' reports the existence of a second-order
belief state. But Yalcin notes there are reasons to doubt that is right.
He develops in detail an expressivist model that avoids what he takes to
be shortcomings of descriptivist approaches.

The two papers we haven't discussed so far, those by Eric Swanson and
Stephen Yablo, are both related to this expressivist family of theories,
though their positive proposals head off in distinctive directions.

Eric Swanson's contribution locates epistemic modals within a broader
category, which he calls ``the language of subjective uncertainty''. He
also emphasizes the diversity of epistemic modal locutions, and draws
attention to the risks involved in focusing too closely on just a few
examples. In the literature so far, `might' and `must' have tended to
get the lion's share of the attention, while other sorts of epistemic
modality -- including the more explicitly quantitative sorts (`four to
one against that', `there's a 55\% chance that', etc.) -- have gone
mostly unnoticed. Swanson argues that attending to other instances of
the language of subjective uncertainty serves to undermine many of the
standard proposals about epistemic `might' and `must', and motivates a
probabilistic semantics.

Somewhat relatedly, Stephen Yablo develops a theory about epistemic
modals where their primary function is not to state facts about the
world, but to update the conversational score. Theories of this kind are
quite familiar from the dynamic semantics tradition, but Yablo notes
that the existing dynamic theories of epistemic modals are quite
implausible. One of the challenges a dynamic approach to epistemic
modals faces is to say how we should update a context (or a belief
state) with It might be that \emph{p} when the context previously was
incompatible with \emph{p}. Yablo adopts some suggestions from David
Lewis's ``A Problem about Permission'' (\citeproc{ref-Lewis1979a}{Lewis
1979}) to try and solve this puzzle.

\subsection*{References}\label{references}
\addcontentsline{toc}{subsection}{References}

\phantomsection\label{refs}
\begin{CSLReferences}{1}{0}
\bibitem[\citeproctext]{ref-DeRose1991}
DeRose, Keith. 1991. {``Epistemic Possibilities.''} \emph{Philosophical
Review} 100 (4): 581--605. doi:
\href{https://doi.org/10.2307/2185175}{10.2307/2185175}.

\bibitem[\citeproctext]{ref-Dietz2008}
Dietz, Richard. 2008. {``Epistemic Modals and Correct Disagreement.''}
In \emph{Relative Truth}, edited by Manuel Garcia-Carpintero and Max
Kölbel, 239--64. Oxford: Oxford University Press.

\bibitem[\citeproctext]{ref-Egan2005-EGAEMI}
Egan, Andy, John Hawthorne, and Brian Weatherson. 2005. {``{Epistemic
Modals in Context}.''} In \emph{Contextualism in Philosophy: Knowledge,
Meaning, and Truth}, edited by Gerhard Preyer and Georg Peter, 131--70.
Oxford: Oxford University Press.

\bibitem[\citeproctext]{ref-vonFintel2008}
Fintel, Kai von, and Anthony S. Gillies. 2008. {``CIA Leaks.''}
\emph{Philosophical Review} 117 (1): 77--98. doi:
\href{https://doi.org/10.1215/00318108-2007-025}{10.1215/00318108-2007-025}.

\bibitem[\citeproctext]{ref-Gibbard1990}
Gibbard, Allan. 1990. \emph{Wise Choices, Apt Feelings: A Theory of
Normative Judgment}. Cambridge, MA: Harvard University Press.

\bibitem[\citeproctext]{ref-Glanzberg2007}
Glanzberg, Michael. 2007. {``Context, Content and Relativism.''}
\emph{Philosophical Studies} 136 (1): 1--29. doi:
\href{https://doi.org/10.1007/s11098-007-9145-5}{10.1007/s11098-007-9145-5}.

\bibitem[\citeproctext]{ref-Hacking1967}
Hacking, Ian. 1967. {``Possibility.''} \emph{Philosophical Review} 76
(2): 343--68. doi:
\href{https://doi.org/10.2307/2183640}{10.2307/2183640}.

\bibitem[\citeproctext]{ref-Levi1996}
Levi, Isaac. 1996. \emph{For the Sake of the Argument:ramsey Test
Conditionals, Inductive Inference and Nonmonotonic Reasoning}.
Cambridge: Cambridge University Press.

\bibitem[\citeproctext]{ref-Lewis1979a}
Lewis, David. 1979. {``A Problem about Permission.''} In \emph{Essays in
Honour of {J}aakko Hintikka on the Occasion of His Fiftieth Birthday on
{J}anuary 12, 1979}, edited by Esa Saarinen, Risto Hilpinen, Illka
Niiniluoto, and Merrill Provence, 163--75. Dordrecht: Reidel. Reprinted
in his \emph{Papers in Ethics and Social Philosophy}, Cambridge:
Cambridge University Press, 2000, 20-33. References to reprint.

\bibitem[\citeproctext]{ref-MacFarlane2003-MACFCA}
MacFarlane, John. 2003. {``Future Contingents and Relative Truth.''}
\emph{The Philosophical Quarterly} 53 (212): 321--36. doi:
\href{https://doi.org/10.1111/1467-9213.00315}{10.1111/1467-9213.00315}.

\bibitem[\citeproctext]{ref-MacFarlane2005-MACMSO}
---------. 2005a. {``{Making Sense of Relative Truth}.''}
\emph{Proceedings of the Aristotelian Society} 105 (1): 321--39. doi:
\href{https://doi.org/10.1111/j.0066-7373.2004.00116.x}{10.1111/j.0066-7373.2004.00116.x}.

\bibitem[\citeproctext]{ref-MacFarlane2005-Knowledge}
---------. 2005b. {``The Assessment Sensitivity of Knowledge
Attributions.''} \emph{Oxford Studies in Epistemology} 1: 197--233.

\bibitem[\citeproctext]{ref-MacFarlane2009-MACNC}
---------. 2009. {``{Nonindexical Contextualism}.''} \emph{Synthese} 166
(2): 231--50. doi:
\href{https://doi.org/10.1007/s11229-007-9286-2}{10.1007/s11229-007-9286-2}.

\bibitem[\citeproctext]{ref-Partee1989}
Partee, Barbara. 1989. {``Binding Implicit Variables in Quantified
Contexts.''} In \emph{Papers from the Twenty-Fifth Regional Meeting of
the Chicago Linguistic Society}, edited by Caroline Wiltshire, Randolph
Graczyk, and Bradley Music. Chicago: Chicago Linguistic Society.
Reprinted in\cite{Partee2004}.

\bibitem[\citeproctext]{ref-Quine1969}
Quine, W. V. O. 1969. {``Propositional Objects.''} In \emph{Ontological
Relativity and Other Essays}, 139--60. New York: Columbia University
Press.

\bibitem[\citeproctext]{ref-Schlenker2003}
Schlenker, Philippe. 2003. {``Indexicality, Logophoricity, and Plural
Pronouns.''} In \emph{Afroasiatic Grammar II: Selected Papers from the
Fifth Conference on Afroasiatic Languages, Paris, 2000}, edited by
Jacqueline Lecarme, 409--28. Amsterdam: John Benjamins.

\bibitem[\citeproctext]{ref-Stalnaker1978}
Stalnaker, Robert. 1978. {``Assertion.''} \emph{Syntax and Semantics} 9:
315--32.

\bibitem[\citeproctext]{ref-Teller1972}
Teller, Paul. 1972. {``Epistemic Possibility.''} \emph{Philosophia} 2
(4): 303--20. doi:
\href{https://doi.org/10.1007/bf02381591}{10.1007/bf02381591}.

\end{CSLReferences}



\noindent Published in\emph{
Epistemic Modality}, 2011, pp. 1-18.

\end{document}
