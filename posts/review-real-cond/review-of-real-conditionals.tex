% Options for packages loaded elsewhere
\PassOptionsToPackage{unicode}{hyperref}
\PassOptionsToPackage{hyphens}{url}
%
\documentclass[
  10pt,
  letterpaper,
  DIV=11,
  numbers=noendperiod,
  twoside]{scrartcl}

\usepackage{amsmath,amssymb}
\usepackage{setspace}
\usepackage{iftex}
\ifPDFTeX
  \usepackage[T1]{fontenc}
  \usepackage[utf8]{inputenc}
  \usepackage{textcomp} % provide euro and other symbols
\else % if luatex or xetex
  \usepackage{unicode-math}
  \defaultfontfeatures{Scale=MatchLowercase}
  \defaultfontfeatures[\rmfamily]{Ligatures=TeX,Scale=1}
\fi
\usepackage{lmodern}
\ifPDFTeX\else  
    % xetex/luatex font selection
  \setmainfont[ItalicFont=EB Garamond Italic,BoldFont=EB Garamond
Bold]{EB Garamond Math}
  \setsansfont[]{Europa-Bold}
  \setmathfont[]{Garamond-Math}
\fi
% Use upquote if available, for straight quotes in verbatim environments
\IfFileExists{upquote.sty}{\usepackage{upquote}}{}
\IfFileExists{microtype.sty}{% use microtype if available
  \usepackage[]{microtype}
  \UseMicrotypeSet[protrusion]{basicmath} % disable protrusion for tt fonts
}{}
\usepackage{xcolor}
\usepackage[left=1in, right=1in, top=0.8in, bottom=0.8in,
paperheight=9.5in, paperwidth=6.5in, includemp=TRUE, marginparwidth=0in,
marginparsep=0in]{geometry}
\setlength{\emergencystretch}{3em} % prevent overfull lines
\setcounter{secnumdepth}{3}
% Make \paragraph and \subparagraph free-standing
\ifx\paragraph\undefined\else
  \let\oldparagraph\paragraph
  \renewcommand{\paragraph}[1]{\oldparagraph{#1}\mbox{}}
\fi
\ifx\subparagraph\undefined\else
  \let\oldsubparagraph\subparagraph
  \renewcommand{\subparagraph}[1]{\oldsubparagraph{#1}\mbox{}}
\fi


\providecommand{\tightlist}{%
  \setlength{\itemsep}{0pt}\setlength{\parskip}{0pt}}\usepackage{longtable,booktabs,array}
\usepackage{calc} % for calculating minipage widths
% Correct order of tables after \paragraph or \subparagraph
\usepackage{etoolbox}
\makeatletter
\patchcmd\longtable{\par}{\if@noskipsec\mbox{}\fi\par}{}{}
\makeatother
% Allow footnotes in longtable head/foot
\IfFileExists{footnotehyper.sty}{\usepackage{footnotehyper}}{\usepackage{footnote}}
\makesavenoteenv{longtable}
\usepackage{graphicx}
\makeatletter
\def\maxwidth{\ifdim\Gin@nat@width>\linewidth\linewidth\else\Gin@nat@width\fi}
\def\maxheight{\ifdim\Gin@nat@height>\textheight\textheight\else\Gin@nat@height\fi}
\makeatother
% Scale images if necessary, so that they will not overflow the page
% margins by default, and it is still possible to overwrite the defaults
% using explicit options in \includegraphics[width, height, ...]{}
\setkeys{Gin}{width=\maxwidth,height=\maxheight,keepaspectratio}
% Set default figure placement to htbp
\makeatletter
\def\fps@figure{htbp}
\makeatother

\setlength\heavyrulewidth{0ex}
\setlength\lightrulewidth{0ex}
\usepackage[automark]{scrlayer-scrpage}
\clearpairofpagestyles
\cehead{
  Brian Weatherson
  }
\cohead{
  Review of “Real Conditionals”
  }
\ohead{\bfseries \pagemark}
\cfoot{}
\makeatletter
\newcommand*\NoIndentAfterEnv[1]{%
  \AfterEndEnvironment{#1}{\par\@afterindentfalse\@afterheading}}
\makeatother
\NoIndentAfterEnv{itemize}
\NoIndentAfterEnv{enumerate}
\NoIndentAfterEnv{description}
\NoIndentAfterEnv{quote}
\NoIndentAfterEnv{equation}
\NoIndentAfterEnv{longtable}
\NoIndentAfterEnv{abstract}
\renewenvironment{abstract}
 {\vspace{-1.25cm}
 \quotation\small\noindent\rule{\linewidth}{.5pt}\par\smallskip
 \noindent }
 {\par\noindent\rule{\linewidth}{.5pt}\endquotation}
\KOMAoption{captions}{tableheading}
\makeatletter
\@ifpackageloaded{caption}{}{\usepackage{caption}}
\AtBeginDocument{%
\ifdefined\contentsname
  \renewcommand*\contentsname{Table of contents}
\else
  \newcommand\contentsname{Table of contents}
\fi
\ifdefined\listfigurename
  \renewcommand*\listfigurename{List of Figures}
\else
  \newcommand\listfigurename{List of Figures}
\fi
\ifdefined\listtablename
  \renewcommand*\listtablename{List of Tables}
\else
  \newcommand\listtablename{List of Tables}
\fi
\ifdefined\figurename
  \renewcommand*\figurename{Figure}
\else
  \newcommand\figurename{Figure}
\fi
\ifdefined\tablename
  \renewcommand*\tablename{Table}
\else
  \newcommand\tablename{Table}
\fi
}
\@ifpackageloaded{float}{}{\usepackage{float}}
\floatstyle{ruled}
\@ifundefined{c@chapter}{\newfloat{codelisting}{h}{lop}}{\newfloat{codelisting}{h}{lop}[chapter]}
\floatname{codelisting}{Listing}
\newcommand*\listoflistings{\listof{codelisting}{List of Listings}}
\makeatother
\makeatletter
\makeatother
\makeatletter
\@ifpackageloaded{caption}{}{\usepackage{caption}}
\@ifpackageloaded{subcaption}{}{\usepackage{subcaption}}
\makeatother
\ifLuaTeX
  \usepackage{selnolig}  % disable illegal ligatures
\fi
\IfFileExists{bookmark.sty}{\usepackage{bookmark}}{\usepackage{hyperref}}
\IfFileExists{xurl.sty}{\usepackage{xurl}}{} % add URL line breaks if available
\urlstyle{same} % disable monospaced font for URLs
\hypersetup{
  pdftitle={Review of ``Real Conditionals''},
  pdfauthor={Brian Weatherson},
  hidelinks,
  pdfcreator={LaTeX via pandoc}}

\title{Review of ``Real Conditionals''}
\author{Brian Weatherson}
\date{2003}

\begin{document}
\maketitle
\begin{abstract}
Review of William Lycan, ``Real Conditionals''. Oxford: Clarendon Press,
2001.
\end{abstract}

\setstretch{1.1}
Over the last two decades, William Lycan's work on the semantics of
conditionals has been distinguished by his careful attention to the
connection between syntax and semantics, and more generally by his
impeccible methodology. Lycan takes compositionality seriously, so in
his theory the meaning of `even if', for example, should be a
combination of the meaning of `even' and the meaning of `if'. After
reading his work, it's hard to take seriously work which does not share
this methodology.

Lycan's semantics for conditionals makes central use of what he calls
`events'. An event is not a possible world, for it need not be complete
or consistent. It is more like what Barwise and Perry call a
`situation'. Conditionals are quantifiers over events, as follows:

\begin{itemize}
\tightlist
\item
  P if Q = P in any event in which Q
\item
  P only if Q = P in no event other than one in which Q
\item
  P even if Q = P in any event including any in which Q (17)
\end{itemize}

The quantifiers here are contextually restricted. Lycan includes in the
semantic analysis a predicate of events~\emph{R}, whose role is to
restrict the quantifiers over events. An event satisfies~\emph{R}~only
if it is `envisaged', which is similar to saying it is a `real' or
`relevant' possibility. The value of~\emph{R}~changes frequently;
sometimes it even changes mid-sentence. This fact is appealed to
frequently in explaining some surprising behaviour of conditionals. For
example, the invalidity of antecedent-strengthening:~\emph{if p then
r},~\emph{so if p and q then r}, is explained by saying the class of
events relevant to the truth of the conclusion may be larger than the
class of events relevant to the truth of the premise. In particular, at
least one event in which~\emph{p}~\emph{and q}~is relevant to the
conclusion, but no such event need be relevant to the premise. A similar
explanation is given for the failure of transitivity and contraposition.

The quantifier domain must include some non-actual events or
conditionals will turn into material implications. Surprisingly, Lycan
says that sometimes the quantifier includes only non-actual events. In
these cases, it is possible that all (relevant)~\emph{p}-events can
be~\emph{q}-events, even though~\emph{p}~is true but not~\emph{q}. That
is, in these cases modus ponens is invalid. Lycan argues persuasively
that the case against modus ponens is at least as strong as the case
against antecedent-strengthening, contraposition and modus tollens.

There is an extended discussion of `even', which is necessary for
providing a theory of `even if'. Lycan first suggests that~\emph{Even
Grannie was sober}~means~\emph{Everyone, including Grannie},~\emph{was
sober}. The quantifier domain includes everyone no less likely than
Grannie to be sober. After discussing some counterexamples, Lycan
suggests that instead it means~\emph{Everyone plus Grannie was sober},
where the quantifer ranges over everyone whom you would expect to be
sober. Lycan is committed to `even' being a quantifier because of its
syntactic similarity to `only', and because of the ``initial
plausibility of \ldots{} universally quantified paraphrases'' (121) of
sentences involving `even'. The discussion here is fascinating, but not
conclusive. It isn't clear, for example, that `even' and `only' have the
same syntactic role. Compare~\emph{Even supposing Jack were here, he
wouldn't help}~with *\emph{Only supposing Jack were here, he wouldn't
help}.

Lycan also includes a helpful discussion of how his theory handles Allan
Gibbard's `Riverboat Puzzle' and related cases. It is troubling, for
those who don't analyse conditionals as material implications, that
sometimes one speaker can say~\emph{If p, q}, another can say~\emph{If
p, not q}, and both seem to be speaking truly. Lycan argues we should
accept this troubling consequence, but explain it by
making~\emph{R}~sensitive to epistemic considerations.

As well as these points, Lycan raises some powerful objections to `No
Truth Value' theories of conditionals, and against the extensive use of
probability theory in semantics. The book concludes with two appendicies
on `non-conditional' conditionals, such as~\emph{If you're hungry,
there's biscuits on the sideboard}.

There's a lot to like about this book, not least it's witty, even
charming, style. Lycan considers more examples, from more diverse
sources, than most writers. The theory he presents is innovative and at
least aims to be comprehensive. And of course there are some good
arguments for it. Despite this there are, as always, occasional grounds
for complaint.

Although Lycan is very careful to get the syntax of `if' right, and
proves that unlike `and' and `or' it is not a co-ordinating conjunction,
it is not so clear that the syntactic evidence provides distinctive
support for his semantic theory. If it's consistent with the syntax to
say~\emph{p if q}~means~\emph{All relevant q-events are p-events}, it's
consistent with the syntax to say that it means~\emph{All nearby
q-worlds are p-worlds}. So the syntactic argument for preferring Lycan's
theory, to, say, Stalnaker's, is not obviously overpowering. Lycan
suggests that we can naturally paraphrase conditionals as
quantifications over events, but since he is using~\emph{event}~`in a
slightly uncommon way' (17) it is not obvious what support this gives
for his theory.

There are few reasons to favour the use of events rather than worlds in
the analysis. The fact that events can be incomplete seems to only cause
complications for the theory. The fact that they can be inconsistent is
used to rescue some intuitions about conditionals with impossible
antecedents, but many would argue those intuitions should be discarded.

But the main worry is that Lycan needs to say more about some key
notions, particularly about his~\emph{R}~and about validity. In the
discussion of the Riverboat Puzzle, Lycan says, ``I do not have a good
enough intuitive handle on my own notion of `relevance' to provide a
crushing answer {[}to a question about why certain events are not
covered by~\emph{R}{]}.'' (173)~~Lycan says that for an event to
be~\emph{R}, ``the utterer must have it at least tacitly in mind as a
live prospect.'' (19) All events in~\emph{R}~are `envisaged', to use the
term he lands on. But ``there is somethig slightly artificial or
stylized about `envisaging' \ldots{} `Envisaging' is not a
purely~\emph{de facto}~cognitive or other psychological state.'' (30)
The upshot is that the envisaged possibilities are some, but not always
all, of those that are (possibly tacitly) regarded as live. Just which
possibilities then? We are never given a specific account. Any account
we do get is, as in the above quote, almost immediately qualified.
Since~\emph{R}~does so much work, the reader is probably owed a little
more here. (This point is made at greater length in Ken Turner's
excellent review of~\emph{Real Conditionals}~in the~\emph{Journal of
Pragmatics}~forthcoming.)

We are also never specifically given an account of validity. We are told
that several argument forms, from antecedent-strengthening to modus
ponens, are invalid. This seems to mean that one could assert their
premises then reject, or a least decline to assert, their conclusion.
It's important that this process of assertion and rejection take place
in real time, because the value of~\emph{R}~needs to change for the
arguments to be invalid. Lycan has some arguments that this conception
of validity is the philosophically interesting one, but this deserves
more treatment. The logical reforms it draws in go well beyond the logic
of conditionals. On Lycan's approach,~\emph{All swans are white, so all
Australian swans are white}~is, presumably, invalid, since the scope of
the quantifier could change from premise to conclusion. And
contraposition fails for valid arguments. Contraposed modus
ponens:~\emph{p, not q, so not if p, q}~is valid, but modus ponens is
not.

None of this is to deny that~\emph{Real Conditionals}~is a great
contribution to the literature, and if it causes more theorists to pay
serious attention to Lycan's Event Theory, that would be an excellent
consequence.

\vspace{1cm}



\noindent Published in\emph{
Philosophical Review}, 2003, pp. 609-611.

\end{document}
