% Options for packages loaded elsewhere
% Options for packages loaded elsewhere
\PassOptionsToPackage{unicode}{hyperref}
\PassOptionsToPackage{hyphens}{url}
%
\documentclass[
  11pt,
  letterpaper,
  DIV=11,
  numbers=noendperiod,
  twoside]{scrartcl}
\usepackage{xcolor}
\usepackage[left=1.1in, right=1in, top=0.8in, bottom=0.8in,
paperheight=9.5in, paperwidth=7in, includemp=TRUE, marginparwidth=0in,
marginparsep=0in]{geometry}
\usepackage{amsmath,amssymb}
\setcounter{secnumdepth}{3}
\usepackage{iftex}
\ifPDFTeX
  \usepackage[T1]{fontenc}
  \usepackage[utf8]{inputenc}
  \usepackage{textcomp} % provide euro and other symbols
\else % if luatex or xetex
  \usepackage{unicode-math} % this also loads fontspec
  \defaultfontfeatures{Scale=MatchLowercase}
  \defaultfontfeatures[\rmfamily]{Ligatures=TeX,Scale=1}
\fi
\usepackage{lmodern}
\ifPDFTeX\else
  % xetex/luatex font selection
  \setmainfont[ItalicFont=EB Garamond Italic,BoldFont=EB Garamond
SemiBold]{EB Garamond Math}
  \setsansfont[]{EB Garamond}
  \setmathfont[]{Garamond-Math}
\fi
% Use upquote if available, for straight quotes in verbatim environments
\IfFileExists{upquote.sty}{\usepackage{upquote}}{}
\IfFileExists{microtype.sty}{% use microtype if available
  \usepackage[]{microtype}
  \UseMicrotypeSet[protrusion]{basicmath} % disable protrusion for tt fonts
}{}
\usepackage{setspace}
% Make \paragraph and \subparagraph free-standing
\makeatletter
\ifx\paragraph\undefined\else
  \let\oldparagraph\paragraph
  \renewcommand{\paragraph}{
    \@ifstar
      \xxxParagraphStar
      \xxxParagraphNoStar
  }
  \newcommand{\xxxParagraphStar}[1]{\oldparagraph*{#1}\mbox{}}
  \newcommand{\xxxParagraphNoStar}[1]{\oldparagraph{#1}\mbox{}}
\fi
\ifx\subparagraph\undefined\else
  \let\oldsubparagraph\subparagraph
  \renewcommand{\subparagraph}{
    \@ifstar
      \xxxSubParagraphStar
      \xxxSubParagraphNoStar
  }
  \newcommand{\xxxSubParagraphStar}[1]{\oldsubparagraph*{#1}\mbox{}}
  \newcommand{\xxxSubParagraphNoStar}[1]{\oldsubparagraph{#1}\mbox{}}
\fi
\makeatother


\usepackage{longtable,booktabs,array}
\usepackage{calc} % for calculating minipage widths
% Correct order of tables after \paragraph or \subparagraph
\usepackage{etoolbox}
\makeatletter
\patchcmd\longtable{\par}{\if@noskipsec\mbox{}\fi\par}{}{}
\makeatother
% Allow footnotes in longtable head/foot
\IfFileExists{footnotehyper.sty}{\usepackage{footnotehyper}}{\usepackage{footnote}}
\makesavenoteenv{longtable}
\usepackage{graphicx}
\makeatletter
\newsavebox\pandoc@box
\newcommand*\pandocbounded[1]{% scales image to fit in text height/width
  \sbox\pandoc@box{#1}%
  \Gscale@div\@tempa{\textheight}{\dimexpr\ht\pandoc@box+\dp\pandoc@box\relax}%
  \Gscale@div\@tempb{\linewidth}{\wd\pandoc@box}%
  \ifdim\@tempb\p@<\@tempa\p@\let\@tempa\@tempb\fi% select the smaller of both
  \ifdim\@tempa\p@<\p@\scalebox{\@tempa}{\usebox\pandoc@box}%
  \else\usebox{\pandoc@box}%
  \fi%
}
% Set default figure placement to htbp
\def\fps@figure{htbp}
\makeatother


% definitions for citeproc citations
\NewDocumentCommand\citeproctext{}{}
\NewDocumentCommand\citeproc{mm}{%
  \begingroup\def\citeproctext{#2}\cite{#1}\endgroup}
\makeatletter
 % allow citations to break across lines
 \let\@cite@ofmt\@firstofone
 % avoid brackets around text for \cite:
 \def\@biblabel#1{}
 \def\@cite#1#2{{#1\if@tempswa , #2\fi}}
\makeatother
\newlength{\cslhangindent}
\setlength{\cslhangindent}{1.5em}
\newlength{\csllabelwidth}
\setlength{\csllabelwidth}{3em}
\newenvironment{CSLReferences}[2] % #1 hanging-indent, #2 entry-spacing
 {\begin{list}{}{%
  \setlength{\itemindent}{0pt}
  \setlength{\leftmargin}{0pt}
  \setlength{\parsep}{0pt}
  % turn on hanging indent if param 1 is 1
  \ifodd #1
   \setlength{\leftmargin}{\cslhangindent}
   \setlength{\itemindent}{-1\cslhangindent}
  \fi
  % set entry spacing
  \setlength{\itemsep}{#2\baselineskip}}}
 {\end{list}}
\usepackage{calc}
\newcommand{\CSLBlock}[1]{\hfill\break\parbox[t]{\linewidth}{\strut\ignorespaces#1\strut}}
\newcommand{\CSLLeftMargin}[1]{\parbox[t]{\csllabelwidth}{\strut#1\strut}}
\newcommand{\CSLRightInline}[1]{\parbox[t]{\linewidth - \csllabelwidth}{\strut#1\strut}}
\newcommand{\CSLIndent}[1]{\hspace{\cslhangindent}#1}



\setlength{\emergencystretch}{3em} % prevent overfull lines

\providecommand{\tightlist}{%
  \setlength{\itemsep}{0pt}\setlength{\parskip}{0pt}}



 


\setlength\heavyrulewidth{0ex}
\setlength\lightrulewidth{0ex}
\usepackage[automark]{scrlayer-scrpage}
\clearpairofpagestyles
\cehead{
  Brian Weatherson
  }
\cohead{
  Relativism
  }
\ohead{\bfseries \pagemark}
\cfoot{}
\makeatletter
\newcommand*\NoIndentAfterEnv[1]{%
  \AfterEndEnvironment{#1}{\par\@afterindentfalse\@afterheading}}
\makeatother
\NoIndentAfterEnv{itemize}
\NoIndentAfterEnv{enumerate}
\NoIndentAfterEnv{description}
\NoIndentAfterEnv{quote}
\NoIndentAfterEnv{equation}
\NoIndentAfterEnv{longtable}
\NoIndentAfterEnv{abstract}
\renewenvironment{abstract}
 {\vspace{-1.25cm}
 \quotation\small\noindent\emph{Abstract}:}
 {\endquotation}
\newfontfamily\tfont{EB Garamond}
\addtokomafont{disposition}{\rmfamily}
\addtokomafont{title}{\normalfont\itshape}
\let\footnoterule\relax

\makeatletter
\renewcommand{\@maketitle}{%
  \newpage
  \null
  \vskip 2em%
  \begin{center}%
  \let \footnote \thanks
    {\itshape\huge\@title \par}%
    \vskip 0.5em%  % Reduced from default
    {\large
      \lineskip 0.3em%  % Reduced from default 0.5em
      \begin{tabular}[t]{c}%
        \@author
      \end{tabular}\par}%
    \vskip 0.5em%  % Reduced from default
    {\large \@date}%
  \end{center}%
  \par
  }
\makeatother
\RequirePackage{lettrine}

\renewenvironment{abstract}
 {\quotation\small\noindent\emph{Abstract}:}
 {\endquotation\vspace{-0.02cm}}
\cehead{
   Patrick Shirreff and Brian Weatherson
    }
\KOMAoption{captions}{tableheading}
\makeatletter
\@ifpackageloaded{caption}{}{\usepackage{caption}}
\AtBeginDocument{%
\ifdefined\contentsname
  \renewcommand*\contentsname{Table of contents}
\else
  \newcommand\contentsname{Table of contents}
\fi
\ifdefined\listfigurename
  \renewcommand*\listfigurename{List of Figures}
\else
  \newcommand\listfigurename{List of Figures}
\fi
\ifdefined\listtablename
  \renewcommand*\listtablename{List of Tables}
\else
  \newcommand\listtablename{List of Tables}
\fi
\ifdefined\figurename
  \renewcommand*\figurename{Figure}
\else
  \newcommand\figurename{Figure}
\fi
\ifdefined\tablename
  \renewcommand*\tablename{Table}
\else
  \newcommand\tablename{Table}
\fi
}
\@ifpackageloaded{float}{}{\usepackage{float}}
\floatstyle{ruled}
\@ifundefined{c@chapter}{\newfloat{codelisting}{h}{lop}}{\newfloat{codelisting}{h}{lop}[chapter]}
\floatname{codelisting}{Listing}
\newcommand*\listoflistings{\listof{codelisting}{List of Listings}}
\makeatother
\makeatletter
\makeatother
\makeatletter
\@ifpackageloaded{caption}{}{\usepackage{caption}}
\@ifpackageloaded{subcaption}{}{\usepackage{subcaption}}
\makeatother
\usepackage{bookmark}
\IfFileExists{xurl.sty}{\usepackage{xurl}}{} % add URL line breaks if available
\urlstyle{same}
\hypersetup{
  pdftitle={Relativism},
  pdfauthor={Brian Weatherson; Patrick Shirreff},
  hidelinks,
  pdfcreator={LaTeX via pandoc}}


\title{Relativism}
\author{Brian Weatherson \and Patrick Shirreff}
\date{2017}
\begin{document}
\maketitle
\begin{abstract}
Relativism is the view that the truth of a sentence is relative both to
a context of utterance and to a context of assessment. That the truth of
a sentence is relative to a context of utterance is uncontroversial in
contemporary semantics. This chapter focuses on three points: whether
the version of contextualism is vulnerable to the disagreement and
retraction arguments, and if so, whether these problems can be avoided
by a more sophisticated contextualist theory. The points include:
whether relativism really does avoid the four problems posed for the
other theories; and whether there are other theories that also avoid the
problems, without running into the problems facing relativism or
problems of their own. The chapter concentrates on two families of views
that have been called relativist: Relativism about propositional truth;
and Relativism about utterance truth.
\end{abstract}


\setstretch{1.1}
Relativism, in the sense we're interested in here, is the view that the
truth of a sentence is relative both to a context of utterance and a
context of assessment. That the truth of a sentence is relative to a
context of utterance is uncontroversial in contemporary semantics. If
the authors of this paper were to both utter the sentence ``I'm
Canadian'', then one of us would say something true, and the other
something false. And that's because the truth of the utterance ``I'm
Canadian'' is sensitive to a feature of the context of utterance, namely
the identity of the speaker. And that in turn is explained by the fact
that the proposition expressed by that sentence is different in
different contexts. Relativists deny that this simple story can explain
all the ways in which context language.

A core motivation for relativism comes from looking at the problems with
other views. So let's start by thinking about what might be happening
when a speaker says ``This is tasty''. (Call this utterance, which we'll
come back to a bit, U, and its speaker S.) The demonstrative `this' is
context-sensitive, but let's assume it is clear what is being referred
to, and think about what contribution the predicate is making. There are
two natural proposals that are simple, easy to fit into familiar
frameworks, and most likely wrong.

First, the predicate might pick out an objective property of tastiness,
and when S utters U, she ascribes this property to a thing. This
objective view has a number of problems. First, there is a metaphysical
problem; just what is this objective tastiness? Second, there is an
epistemological problem. Typically, a speaker like S could be prepared
to utter U after taking one bite of the thing. But for most plausible
answers to the first question, it is unclear how they could know that
the object had that property. The two problems reinforce each other; The
more plausible answers to the metaphysical question make the property
sensitive to very abtruse conditions, such as how idealised observers
would react to ingesting the substance. But that makes it even more
unlikely that typical utterances of U satisfy the epistemological
requirement. (The points we're making here are well known, but our
presentation owes a lot to Lasersohn
(\citeproc{ref-Lasersohn2005}{2005}).)

Second, the predicate might be context-sensitive. For concreteness,
let's focus on a very simple contextualist theory of `tasty'. An
utterance of U is true, relative to a context \emph{c}, iff the speaker
in \emph{c} likes the taste of the referent of `this' in \emph{c}. So
``This is tasty'' and ``I like the taste of this'' express something
very close to the same proposition. (At least they express propositions
that are necessarily materially equivalent, but the relation between the
two is probably closer than that.) Again there are two problems, both of
them to do with the different dynamics of ``This is tasty'' and ``I like
the taste of this''. A hearer H, who does not like the taste of this
substance, could more readily disagree with ``This is tasty'' than with
``I like the taste of this''. If H doesn't like its taste, he'll
nevertheless concede that S's utterance of ``I like the taste of this''
is true, but won't concede that ``This is tasty'' is true. And a similar
phenomenon occurs when S herself changes her mind. If later she dislikes
the taste of the substance, she will be disposed to retract her
utterance of ``This is tasty'', but not of ``I like the taste of this''.
So both disagreement and retraction data pose problems for the
contextualist. (On disagreement, see especially Tamina Stephenson
(\citeproc{ref-Stephenson2007}{2007}). On retraction, and most
everything else we'll talk about in this paper, see John MacFarlane
(\citeproc{ref-MacFarlane2014}{2014}).)

If the objectivist and contextualist solutions fail, the relativist
suggests that they have a useful alternative. Say that truth is doubly
relativised, both to contexts of utterance and to context of assessment.
So the utterance, or perhaps just the proposition expressed by it, is
true relative to a context of assessment \emph{c\textsubscript{a}} and
context of utterance \emph{c\textsubscript{u}} iff the agent at
\emph{c\textsubscript{a}} likes the taste of the denotation of `this' in
\emph{c\textsubscript{u}}. So now the truth of the utterance is relative
both to the context of utterance and the context of assessment. This
solves the metaphysical problem; talk about what a person likes is
unproblematic. It solves at least a version of the epistemological
problem; a speaker knows what they like, so can make utterances that are
true in their context. It solves a version of the disagreement problem;
if H doesn't like the taste of the stuff, he can truly say that what S
said is not true, since it isn't true in his context. And it explains
the retraction data, since once S changes her taste, what she said is no
longer true relative to her new context, and hence should be retracted.

As everywhere else in philosophy, arguments and claims can be and are
contested. In the quick case for relativism we've described so far,
there are at least four points where one could readily disagree.

\begin{enumerate}
\def\labelenumi{\arabic{enumi}.}
\tightlist
\item
  Whether objectivism is really vulnerable to the combination of the
  metaphysical and the epistemological arguments.
\item
  Whether this version of contextualism is vulnerable to the
  disagreement and retraction arguments, and if so, whether these
  problems can be avoided by a more sophisticated contextualist theory.
\item
  Whether relativism really does avoid the four problems posed for the
  other theories.
\item
  Whether there are other theories that also avoid the problems, without
  running into the problems facing relativism or problems of their own.
\end{enumerate}

In this entry, we'll focus on the last three points, since they are more
widely discussed in the literature than the first one. And indeed, the
last three points are extremely actively debated in recent years. We
won't try to take sides in those debates, though we will note that on no
point is the final picture nearly as rosy for the relativist as this
initial sketch may suggest.

We've so far focussed on one particular predicate, `tasty', though the
points we've been making generalise to most predicates of personal
taste. But this is far from the only area where relativist theories have
been proposed. There has been a lot of discussion of relativist theories
of epistemic modals. The arguments here are fairly similar to the
arguments about predicates of personal taste (though see the discussion
below about syntax and control). And MacFarlane has argued for
relativist treatments of future contingents, and knowledge ascriptions.
We will not spend much time on those constructions explicitly, but the
issues they raise are broadly similar to the issues raised by predicates
of personal taste, and by epistemic modals.

We will largely bypass two quite different areas where relativist
theories have been proposed. Mark Richard
(\citeproc{ref-Richard2008}{2008}) has argued for a relativist treatment
of comparative adjectives. And Weatherson
(\citeproc{ref-Weatherson2011-NoRoyal}{2011}) argued for relativist
treatments of areas of certain areas discourse where common assumptions
about the area are false, and a relativist treatment might be the most
charitable fix. As interesting as we find these proposals, they haven't
occasioned nearly as much discussion as the proposals discussed above,
and so we'll set them aside.

What we will do is first clarify some of the many ways in which the term
`relativism' has been used in recent debates, then review some technical
material about indices and contexts that is essential for understanding
some relativist views, then look at the main line of recent debate
concerning relativism, the one centered on issues about retraction and
disagreement, and finally look at some syntactic evidence for
relativism.

\section{Varieties of Relativism}\label{varietiesofrelativism}

As we noted in the very first paragraph, it is uncontroversial that
sentence truth is sensitive to the context of utterance. It is extremely
contentious just how many such sentences are sensitive to the context of
utterance. (See, for instance (\citeproc{ref-DeRose2009}{DeRose 2009};
\citeproc{ref-HarmanThomson1996}{Harman and Thomson 1996};
\citeproc{ref-Cappelen2005}{Cappelen and Lepore 2005}) for some of the
disputes.) But that there is some sensitivity here is uncontroversial.
The view that a sentence type's truth is sensitive to its context of
utterance is sometimes called ``indexical relativism''
(\citeproc{ref-Kolbel2004}{Kölbel 2004};
\citeproc{ref-Einheuser2008}{Einheuser 2008}). For some purposes this is
a useful name. In particular, the view that the content of moral
predicates, and hence the truth value of ascriptions of moral
predicates, is sensitive to the context of utterance does seem like a
form of moral relativism, as that term is standardly understood. It is,
for instance, the view that Harman (\citeproc{ref-Harman1975}{1975})
defends in a self-proclaimed defence of moral relativism. (That this was
traditionally known as relativism is stressed in Cappelen and Huvenes
(\citeproc{ref-CappelenHuvenes2014}{2014}), who draw some interesting
conclusions from this fact.) But for present purposes we want to clearly
exclude those views, and focus on much more radical proposals that have
been recently made.

There are two (overlapping) families of views that have been called
relativist, and which we will be concentrating on for the bulk of this
entry. They are:

\begin{itemize}
\tightlist
\item
  Relativism about propositional truth - Whether a proposition is true
  is not an absolute fact. Propositions can be true relative to some
  contexts of assessment, and false relative to others.
\item
  Relativism about utterance truth - Whether an utterance is true is not
  an absolute fact. Utterances can be true relative to some contexts of
  assessment, and false relative to others. It isn't always clear
  whether a particular author, in defending relativism, is primarily
  defending the first or second of these claims. But it is natural to
  interpret most relativists as defending relativism about propositional
  truth. This is most clearly true for Kölbel
  (\citeproc{ref-Kolbel2002}{2002}) and Egan
  (\citeproc{ref-Egan2007-EGAEMR}{2007}), but there are few relativists
  that it is hard to interpret as taking this to be the primary focus.
  The one big exception, however, is the most prominent defender of
  relativism in the contemporary literature, John MacFarlane
  (\citeproc{ref-MacFarlane2014}{2014}). He takes relativism about
  utterance truth to be the only genuine form of relativism, though he
  also endorses relativism about propositional truth. We are not going
  to engage with the dispute over what is \emph{really} a relativist
  view. Instead we'll just describe a pair of views that are relativist
  in one of these senses but not the other, to explain how the two
  senses come apart.
\end{itemize}

First, let's think about how to make propositional truth, but not
utterance truth, relative. Assume that Suzy actually swims, but that she
might not have. Then, on a natural way of thinking about modality, the
proposition \emph{Suzy swims} will be true relative to the actual world,
but false relative to any possible world in which she does not. So
propositional truth is relative, with the relativity being to possible
worlds.

Cappelen and Hawthorne (\citeproc{ref-CappelenHawthorne2009}{2009})
argue that the reasoning of the previous paragraph misunderstands the
nature of truth and modality. The proposition \emph{Suzy swims} is, they
say, simply true. It might have been false. But that doesn't mean it is
merely true relative to some world or other; it just means that it might
have had a different property than it actually has. Schaffer
(\citeproc{ref-Schaffer2012}{2012}) argues that the things we say and
think, i.e., propositions, typically have worlds in their content, so
what we express by the sentence ``Suzy swims'' is the proposition
\emph{Suzy swims at @}, and that proposition, like all propositions, has
a non-relative truth value.

Even if all that is wrong, and the contents of sentences like ``Suzy
swims'' are propositions that are true in some worlds, false in others,
there is a reason not to call this `relativism'. After all, other worlds
don't exist, or at least don't exist in the way ours does. So it isn't
true that the proposition expressed by ``Suzy swims'' is true relative
to something and false relative to something else; there isn't a
something else for it to be false relative to.

Taking the last point on board, a serious relativism about propositional
truth should say that there is a proposition \emph{p}, and two relata
\emph{a} and \emph{b} such that \emph{p} is true relative to \emph{a},
and false relative to \emph{b}. One way to do that, as seen in for
example Kaplan (\citeproc{ref-Kaplan1977}{1989}), is to make truth
relative to times, and believe that other times exist. (The contrast
here is with the eternalist view of propositions defended by Evans
(\citeproc{ref-Evans1985}{1985}).) But the radical view that has been
central to many recent versions of relativism has been to say that truth
is relative to world, time, individual triples. So there is a
proposition, for example, that is true relative to a triple
\(\langle\)\emph{w},~\emph{t},~\emph{a}\(\rangle\) iff in \emph{w}, the
individual \emph{a} disapproves of murder at \emph{t}. Perhaps, on a
relativist framework, this just is the proposition that murder is wrong.
The picture of propositions here owes something to the discussion of de
se beliefs in Lewis (\citeproc{ref-Lewis1979b}{1979}), since Lewis
thought the contents of such beliefs were sets of such triples. But
Lewis put this forward as a theory about mental content; the move to
extending it to linguistic content is more recent.

On this view, the contents of utterances will be propositions that can
be true relative to one individual (in a world, at a time) and false
relative to another. But the utterances themselves will naturally be
thought to have absolute truth value. Think back to the view that
propositional truth is just relative to worlds. Then an utterance of
``Suzy swims'' will, intuitively, be true iff Suzy swims in the world
the utterance is made. When we think about a counterfactual utterance of
``Suzy swims'' in a world where she does not, we don't think the
utterance is true just because its content is actually true. So although
this is a form of relativism about propositional truth, it is not yet a
form of relativism about utterance truth. MacFarlane, who takes
utterance relativism to be central, calls this view `non-indexical
contextualism', with the name meant to highlight that it isn't, by his
lights, genuinely relativist.

Now turn to relativism about utterance truth. A simple way to implement
this is to say that the propositional content of an utterance is
relative to an assessor. So consider a very simple, and surely false,
view about `you' in English. It holds that the content of an utterance
of `you', relative to a context of assessment, is the agent of that
context. And it says that the utterance is true at a context of
assessment iff the proposition it expresses relative to that context is
(absolutely) true. So an utterance of ``You swim'' will express a
different proposition relative to contexts with different agents. And
since some of those propositions will be true, and some false, we have a
version of relativism about utterance truth. But this view is consistent
with the view that propositional truth is absolute, not relative even to
worlds.

So the two versions of relativism are dissociable. But it is also
possible to hold them simultaneously, as MacFarlane
(\citeproc{ref-MacFarlane2014}{2014}) does. To set up MacFarlane's view,
it is helpful to review the framework against which it was developed.

\section{Index, Context and Content}\label{indexcontextandcontent}

The existence of indexical terms, like `I', means that we can't give a
theory for truth conditions of sentences as such. The sentence-type
\emph{I am in Ann Arbor now} doesn't have truth conditions; only
occurrences of this sentence do. So a natural move is to build up a
theory that assigns to each term a function from contexts to contents,
and use that to provide a theory of which contexts a sentence is true
in. It turns out that there are reasons to endorse more complications
than this. In particular, sentences like (1) show the need for
relativising truth conditions relative to both a context and an
\emph{index}.

\begin{quote}
(1) It might have been the case that my actual parents never met
\end{quote}

That's true. But think of how we might naturally provide truth
conditions for it. First, we define the context-relative truth
conditions for ``My actual parents never met.''. The context will
provide a world for `actually' (i.e., the actual world), and an agent
for `my' (i.e., the speaker). Then, we say that the modal operator
shifts the world of evaluation. Intuitively, \emph{Might p} is true iff
\emph{p} is true in some possible world. So we need some way to see if
the content of ``My actual parents never met'' is true in some other
world. But that's hard to do, since (given origin essentialism), there
will be no context in which ``My actual parents never met'' can be truly
uttered, so its content will be the function that maps every context
into FALSE. And that's the same content as, for example, ``Two plus two
is five'', and so it can't be possibly true.

There are actually two related problems here. One is that while we need
to look around the other worlds to see whether some thing is true, the
nature of that thing is fixed by the actual context. So if the sentence
is uttered by Sasha Obama in this world, the content is something like
\emph{It might have been the case that Barack Obama and Michelle Obama
never met}. We need to fix the values of the contextually sensitive
terms (`my' and `actual') even when they appear inside operators. The
second problem is that something as coarse-grained as a function is very
hard to `shift' in just one respect.

There is a well-known solution to this, developed primarily by Hans Kamp
(\citeproc{ref-Kamp1971}{1971}), then put to important philosophical use
by David Lewis (\citeproc{ref-Lewis1980b}{1980}) and David Kaplan
(\citeproc{ref-Kaplan1977}{1989}). Say that our semantic theory will
assign a function from both contexts and indices to terms. Ultimately,
each sentence will get as its semantic value a function from contexts
and indices to truth values. Lewis describes the distinction between
contexts and indices thus,

\begin{quote}
A context is a location -- time, place, and possible world -- where a
sentence is said. It has countless features, determined by the character
of the location. An index is an n-tuple of features of context, but not
necessarily features that go together in any possible context. Thus an
index might consist of a speaker, a time before his birth, a worlds
where he never lived at all, and so on.
~(\citeproc{ref-Lewis1980b}{Lewis 1980, 79})
\end{quote}

indices are structured, and so they can be `improper'; it might consist
of things that don't go together. If the index contains just worlds and
individuals, the index could be
\(\langle\)\emph{w},~Sasha~Obama\(\rangle\), even if Sasha doesn't exist
in \emph{w}. Such an index will play a central role in making (1), as
uttered by Sasha Obama, true. A context, on the other hand, will by its
nature be proper; it will pick out a world, and a time in that world,
and a person existing at that time, and so on.

Indices are, unlike contexts, `shiftable'. There can be sentential
operators that say the truth value of the whole sentence is given by
looking at the truth value of the embedded sentence at some different
index. Assume, for example, that there is a world parameter in the
index, so the index is \(\langle\)\emph{w},~\emph{x}\(\rangle\), where
\emph{x} contains everything else in the index. Then \emph{Might p} will
be true relative to context \emph{c} and index \(\langle\)w,~
\emph{x}\(\rangle\) just in case for some \emph{w}′, \emph{p} is true
relative to context \emph{c} and index
\(\langle\)\emph{w}′,~\emph{x}\(\rangle\). Note that in this definition,
we do \emph{not} shift \emph{c}; so if \emph{c} determines that the
referent of `I' is Sasha Obama, that stays even if we `shift' the index
to a world where Sasha does not exist.

The separation between context and index was developed to solve some
technical problems, but it can do some philosophical work. The theory
associates each sentence with a function from contexts and indices to
truth values. Equivalently, it associates sentence-context pairs with
functions from indices to truth values. So it is natural to say that the
content of an utterance is (or is at least intimately connected to) the
function associated with the pair consisting of the sentence uttered and
the context it was uttered in. If the index is simply a world, then the
contents will be functions from worlds to truth-values, as defended by
classical contextualists such as Robert Stalnaker
(\citeproc{ref-Stalnaker1984}{1984}). David Lewis
(\citeproc{ref-Lewis1980b}{1980}) objected that assigning contents at
just this stage was arbitrary, why not associate contents with functions
from context to truth values instead? François Recanati
(\citeproc{ref-Recanati2007}{2007}) responds well to Lewis's arguments,
although his defence requires making indices much more complex. In
particular, he includes parameters for times and individuals in the
index, leading him to support relativism about propositional truth.

With this background, we can more easily describe two distinctive
features of MacFarlane's views. First, he makes indices very complicated
indeed. The index might include, among other things, an information set
(for interpreting epistemic and deontic modals), a standard of taste
(for interpreting predicates of personal taste), relevant alternatives
(for interpreting knowledge ascriptions), a time (for handling future
contingents), and perhaps many more things. For simplicity, we'll call
these things collectively a \emph{perspective}, and say that indices are
world-perspective pairs. (Though note that our terminology here requires
perspectives to be structured entities, and for them to be potentially
improper.) If the propositional content of an utterance is (intimately
connected to) a function from indices to truth-values, then the one
proposition will be true relative to one perspective and false relative
to another.

So far we only have relativism about propositional truth, or what
MacFarlane calls non-indexical contextualism. MacFarlane argues for a
second revision to contextualist orthodoxy. Assume that we have a
particular utterance \emph{U} of a sentence \(S\) in context of
utterance \emph{c\textsubscript{u}}, and that utterance is being
assessed in context \emph{c\textsubscript{a}}. And assume that
\emph{w}(\emph{c}) is the world of context \emph{c}, and
\emph{p}(\emph{c}) is the perspective of context \emph{c}. Sentence
\emph{S} is associated with a function \emph{f\textsubscript{S}} from
context-index pairs, i.e., context-world-perspective triples, to truth
values. The relativist about propositional truth, but not utterance
truth, says that \emph{U} is true iff
\emph{f\textsubscript{S}}(\emph{c\textsubscript{u}},~\emph{w}(\emph{c\textsubscript{u}}),~\emph{p}(\emph{c\textsubscript{u}}))
= TRUE. MacFarlane's relativist\footnote{Strictly speaking, to get
  MacFarlane's exact view which kinds of views are relativist we need to
  complicate things even more. MacFarlane explicitly leaves open the
  possibility that we have a non-indexical contextualist view about some
  terms, and a relativist view about others, and expressly says that
  such a view is a form of relativism. So what we should say is that
  perspectives are sub-divided into those features where the context of
  utterance is relevant for utterance truth, and those where the context
  of assessment is relevant to utterance truth. If the first set of
  features of \emph{p\textsubscript{u}}, and the second set is
  \emph{p\textsubscript{a}}, then the utterance \emph{U} of sentence
  \emph{S} in context \emph{c\textsubscript{u}} will be true, relative
  to \emph{c\textsubscript{a}}, iff
  \emph{f\textsubscript{S}}(\emph{c\textsubscript{u}},~\emph{w}(\emph{c\textsubscript{u}}),~\emph{p\textsubscript{u}}(\emph{c\textsubscript{u}}),~\emph{p\textsubscript{a}}(\emph{c\textsubscript{a}}))
  = TRUE. MacFarlane's own view seems to be that
  \emph{p\textsubscript{u}} will be empty, so the simplified view we've
  given in the text is a fair representation of how he thinks actual
  natural languages work. But there are possible languages where the
  complications noted here are relevant.} says that \emph{U} is true iff
\emph{f\textsubscript{S}}(\emph{c\textsubscript{u}},~\emph{w}(\emph{c\textsubscript{u}}),~\emph{p}(\emph{c\textsubscript{a}}))
= TRUE. Crucially, we assess the utterance itself, and not just its
content, by the perspective of the assessor. MacFarlane argues that this
move allows a better understanding of disagreement and retraction, which
were central to the phenomena that motivated relativism. So let's turn
to how well relativism explains the phenomena it was designed to
explain.

\section{Retraction and Disagreement}\label{retractionanddisagreement}

\subsection{Retraction}\label{retraction}

Contextualists have a difficult time explaining why we retract earlier
claims involving predicates of personal taste or epistemically modals
assertions when our perspective has changed in the intervening time.
Consider this example:

\begin{quote}
Kim (age 8): Lunchables are delicious.\\
Kim (age 26 being reminded of previous assertion): I take it back/I was
wrong/what I said was false. Lunchables aren't delicious.
\end{quote}

When Kim retracts her earlier assertion, it is natural to use any of the
three forms of retraction listed here. It isn't natural to say either of
the following things:

\begin{itemize}
\tightlist
\item
  Lunchables were delicious back then, but they aren't delicious any
  more.
\item
  When I said that back then, I only meant that they were delicious to
  me back then.
\end{itemize}

There is a contrast with how clearly context sensitive terms like `here'
are used. It isn't natural to use any of the three forms of retraction
we attribute to Kim below.

\begin{quote}
Kim (in a café): It is pleasant here.\\
Kim (in an oil refinery, reminded of previous assertion): I take it
back/I was wrong/What I said was false. It isn't pleasant here.
\end{quote}

On the other hand, it is natural to say things like:

\begin{itemize}
\tightlist
\item
  It was pleasant where we were, but it isn't pleasant where we are now.
\item
  When I said that back then, I only meant it was pleasant where we
  were.
\end{itemize}

Predicates of personal taste, like `delicious', don't behave like
explicitly context-sensitive expressions like `pleasant here'. We see
the same pattern with epistemic modals.

\begin{quote}
Hakeem: It might be snowing outside.\\
Fabian: No/that's wrong/that's false, it can't be snowing. I just looked
out the window and there were clear skies and the sun was out.\\
Hakeem: Really? Then I guess I was mistaken.
\end{quote}

If \emph{It might be that p} is true iff the speaker's knowledge is
compatible with \emph{p}, as it does on a simple contextualist theory,
then none of this conversation makes any sense. But, argue relativists,
it is perfectly natural. A natural move here is to argue that the
problem with simple contextualism isn't the contextualism, but the
simplicity. Exploring the moves that can be made here will take us too
far afield; MacFarlane (\citeproc{ref-MacFarlane2014}{2014}) goes
through a number of possible alternative contextualist theories and
shows how examples like the ones involving Kim, Hakeem and Fabian can be
generated to raise problems for each of them. Instead, let's look at how
the relativist handles the cases.

According to MacFarlane, relativists are committed to the following
principle about retraction:

\begin{quote}
The speaker ought to retract the assertion if she has good grounds for
thinking that its content is false (as assessed from the perspective she
currently occupies).
\end{quote}

Given that Kim and Hakeem are evaluating their earlier utterances from
new perspectives, perspectives where the assertions are now judged to be
false, they both should retract that earlier assertion because they now
take them to be false. But arguably this principle overgenerates.
Consider this example from Fintel and Gillies
(\citeproc{ref-vonFintelGillies2008}{2008}).

\begin{quote}
Alex: The keys might be in the drawer.\\
Billy: (Looks in the drawer, agitated) They're not. Why did you say
that?\\
Alex: Look, I didn't say that they were in the drawer. I said they might
be there -- and they might have been. Sheesh.
~(\citeproc{ref-vonFintelGillies2008}{Fintel and Gillies 2008, 81})
\end{quote}

The lesson they draw from the example is that retraction is somewhat
voluntary. Despite what relativists claim, it is not always true that
`might' claims are retracted or rejected in light of new evidence.
Instead, what seems to be true is that that solipsistic readings for the
modals are virtually always available. They say that the relativist
can't easily explain this data. MacFarlane
(\citeproc{ref-MacFarlane2014}{2014}, Ch. 10) responds by saying, in
effect, that epistemic modal claims sometimes mean what contextualists
say they mean, and this is compatible with relativism.

\subsection{Disagreement and Agreement}\label{disagreementandagreement}

The data about retraction are closely related to a phenomenon that
classically was central to debates about relativism. A traditional
argument for relativism is that it is necessary to explain `faultless
disagreement'. If one person says Vegemite is tasty, and another says
that it is not, the alleged datum is that they are disagreeing, but
neither is wrong. Objective treatments of taste save the phenomenon of
disagreement, but lose the faultlessness. Contextualist, or otherwise
subjective, treatments of taste save the faultlessness, but lose the
disagreement. Relativism was alleged to keep both. Faultless
disagreement plays a big role in Kölbel
(\citeproc{ref-Kolbel2002}{2002}), but it has dropped out of the recent
literature somewhat. But the focus on disagreement remains, with the
central claim being that contextualists cannot explain why some
conversations are genuinely disagreements. The following is how
MacFarlane puts the point using an example from predicates of personal
taste:

\begin{quote}
If the truth of my claim that a food is ``tasty'' depends on how it
strikes me, while the truth of your claim that the same food is ``not
tasty'' depends on how it strikes you, then our claims are compatible,
and we do not disagree in making them. But it seems that we do
disagree--even if we are aware that the source of our disagreement is
our differing tastes. ~(\citeproc{ref-MacFarlane2014}{MacFarlane 2014,
8})
\end{quote}

This is no argument against any kind of objectivist treatment of
predicates of personal taste. The issue is solely whether relativism or
contextualism does a better job of explaining the facts about
disagreement. We'll look at a couple of reasons for thinking that the
problem is less pressing for contextualists than it might first seem,
then an interesting attempt to resolve the problems that remain for
contextualism, then at some reasons for doubting that relativism helps
solve the remaining problems.

\subsection{Clarifying the Data}\label{clarifyingthedata}

The puzzle for contextualism is supposed to be that there is a big
difference between the felicity of Mark's reply in the following two
cases.

\begin{quote}
Sally: This chilli is tasty.\\
Mark: I disagree. It's too hot.
\end{quote}

\begin{quote}
Sally: I'm from Barcelona.\\
Mark: I disagree. \#I'm from Oslo.
\end{quote}

Assume Sally and Mark are both being sincere. So Sally does like the
chilli and is from Barcelona, and Mark doesn't like it and is from Oslo.
In neither case can Mark sincerely repeat the words that Sally uttered.
Indeed, he can sincerely utter the negations of the sentences Sally
uttered. But in the first case, this seems to amount to a disagreement,
and in the second case it does not. If the context-sensitivity of
`tasty' and `I' is explained the same way, this is mysterious.

But note that there are other examples that do not seem to be amenable
to a relativist treatment where Mark can express disagreement with
Sally.

\begin{quote}
Sally: I like this chilli.\\
Mark: I disagree. It is too hot.
\end{quote}

What seems to be going on there is that Mark is disagreeing with an
attitude that Sally has, but not with any proposition she expressed.
After all, Mark presumably agrees with the proposition that Sally likes
the chilli. That's why he knows he is disagreeing with her. And that's
the content of what he uttered. So some disagreements that are triggered
by assertions are not with the content of what is asserted. This might
be the germ of an idea for how contextualists can explain the data about
disagreement, one nicely developed by Torfinn Huvenes
(\citeproc{ref-Huvenes2012}{2012}). He argues that in a lot of cases of
disagreements involving taste, what we see is not a disagreement about
any content, but a disagreement of attitudes.

\begin{quote}
Two parties disagree just in case there is something towards which they
have conflicting attitudes. This sometimes means that there is a content
that one party accepts and the other rejects, but that does not always
have to be the case. Just as two parties may have conflicting beliefs,
they may also have conflicting desires or preferences.
~(\citeproc{ref-Huvenes2012}{Huvenes 2012, 178})
\end{quote}

If Huvenes is right, then we can't draw any conclusions for semantics
from the facts about disagreement.

There is a further problem with using facts about disagreement to argue
against contextualism. Consider the following dialogue.

\begin{quote}
Sally: Joe might be in Boston.\\
Mark: I disagree; he's definitely in China.
\end{quote}

If \emph{Joe might be in Boston} just means \emph{For all I know, Joe is
in Boston}, then it is prima facie unclear why Mark is disagreeing.
After all, it is consistent with Joe's being in China that for all Sally
knows, Joe is in Boston. But, say some contextualists, Mark need not
disagree with the whole content of Sally's utterance. He may just be
disagreeing with the \emph{prejacent} of the modal, i.e., that Joe is in
Boston.

Most of the terms of disagreement that have been used in examples
motivating anti-contextlism can be seen to target something other than
the entire proposition ~(\citeproc{ref-vonFintelGillies2008}{Fintel and
Gillies 2008, 83}). This is a particular problem for arguments for
relativism involving epistemic modals. It's possible to say ``I
disagree'', ``that's false'', ``no'', or ``you're mistaken'' and
disagree with the prejacent of someone's modal claim, not necessarily
the whole claim.

One potential avenue for the relativist to get around this worry is to
limit the range of disagreement markers that count as expressing the
right kind of disagreement. For example, Tamina Stephenson
(\citeproc{ref-Stephenson2007}{2007}) limits the terms of disagreement
in the examples she uses to ``no'' and ``nuh-uh''. And John MacFarlane
(\citeproc{ref-MacFarlane2014}{2014, 11}) discusses disagreements that
more explicitly target the entire asserted proposition such as ``the
proposition you expressed is false'' and ``what you asserted is false''.
The issue with the two former disagreement markers is that they can
explicitly target the embedded clause of an expression. Indeed,
Stephenson herself provides a clear example of this phenomenon.

\begin{quote}
Mary: How's the cake?\\
Sam: I think it's tasty.\\
Sue: Nuh-uh, it isn't tasty at all!
~(\citeproc{ref-Stephenson2007}{Stephenson 2007, 512})
\end{quote}

The issue with the disagreement markers that MacFarlane uses is that
they have become too technical to do the type of work they need to do.
Relativism is meant to be an empirical thesis that relied on natural
language data to back it up. We have moved well outside of the realm of
natural language and the types of natural language intuitions about the
acceptability of sentences that we can get when we move to ``the
proposition you expressed is false'' and ``what you asserted is false''.

Note that the two contextualist responses we've described here are
rather complementary. The point Huvenes makes, that disagreements can
involve attitudes other than belief, seems best served to defuse
arguments from disagreements concerning predicates of personal taste.
And the point that von Fintel and Gillies make, that disagreements can
target prejacents, seems best served to defuse arguments from
disagreements concerning epistemic modals. It is possible there are
replies to this last point; Weatherson and Egan
(\citeproc{ref-WeathersonEgan2011}{2011}) for example suggest that
examples involving agreements are invulnerable to the response that von
Fintel and Gillies make. But as it stands, the dominant trend in the
literature seems to be in the direction of thinking the contextualist
has the resources to answer these relativist arguments.

\subsection{Presuppositions and Common
Context}\label{presuppositionsandcommoncontext}

There is a more radical, and perhaps more concessive, response to the
disagreement arguments available to the contextualist. Dan López de Sa
(\citeproc{ref-LopezDeSa2008b}{2008}) develops a contextualist theory
that explains the disagreement data by positing that for many
context-sensitive terms, there is a presupposition that users of the
term are in the same context. (Note that López de Sa calls his theory an
``indexical relativist'' theory, but it is a kind of contextualist
theory in the way the terms are being used here.)

It isn't true in general that there is a presupposition of commonality
of context. If two speakers both say ``I am happy'', they should be
interpreted as putting forward different propositions. And that is
because, in the relevant sense, they are in different contexts. But,
perhaps, many terms are not like this. In particular, cases where there
appears to be a problem with explaining the phenomena involving
disagreement, perhaps this is not so. So consider the stock example
López de Sa uses, a variant on one that may seem familiar by now.

\begin{quote}
Hannah: Homer Simpson is funny.\\
Sarah: I disagree. Homer is not funny.
\end{quote}

If `funny' denotes a different property when Hannah and Sarah use it,
there is no disagreement about propositional content here. And the
contextualist account of predicates of personal taste would predict that
it could, and perhaps often will, denote a different propertty on
different occasions of usage. But it seems that there is a disagreement
here, and arguably even one about propositional content. The solution
López de Sa offers is that in any conversation, there is a
presupposition that we are applying the same aesthetic standards. The
model he uses is a suggestion made by David Lewis
(\citeproc{ref-Lewis1989b}{1989}) in defending a contextualist treatment
of claims about value.

\begin{quote}
Wouldn't you hear them saying `value for me and my mates' or value for
the likes of you'? Wouldn't you think they'd stop arguing after one
speaker says X is a value and the other says it isn't? -- Not
necessarily. They might always presuppose, with more or less confidence
(well-founded or otherwise), that whatever relativity there is won't
matter in this conversation. ~(\citeproc{ref-Lewis1989b}{Lewis 1989,
84})
\end{quote}

Here is how López de Sa develops the point.

\begin{quote}
According to the approach, `is funny' triggers a presupposition of
commonality to the effect that both Hannah and Sarah are similar with
respect to humour. Thus, in any non-defective conservation where Hannah
uttered `Homer is funny' and Sarah replied `No, it is not,' it would
indeed be common ground that Hannah and Sarah are relevantly alike, and
thus that they are contradicting each other. After all, provided they
are alike, either both Hannah and Sarah are amused by Homer or they are
not. ~(\citeproc{ref-LopezDeSa2008b}{López de Sa 2008, 305})
\end{quote}

This is an ingenious idea, but there are a few hurdles to be cleared
before it could be declared a full solution to the problem. First, it
needs a way to deal with eavesdroppers. If Hannah writes ``Homer Simpson
is funny'' on a scrap of paper, and later Sarah chances upon that paper,
she can still say that she disagrees. But it is very odd to think there
is a presupposition of commonality of taste with anyone who chances upon
one's writings. Second, it needs to account for cases where the
presupposition is expressly cancelled. The Hannah/Sarah dialogue feels
natural even in cases where it has been made explicit that Hannah and
Sarah have completely different tastes, and they are displaying their
differences for the amusement of their friends
~(\citeproc{ref-MacFarlane2014}{MacFarlane 2014, 131--32}). Third, it
doesn't quite capture the idea that Hannah and Sarah are contradicting
each other. When Sarah disagrees, it shows that either there is a
proposition one accepts and the other rejects, or that the context is
defective. If we thought the data was that there was a contradiction
between what they say, López de Sa's approach can't explain that. And
finally, it isn't obvious how to extend this theory to other terms for
which relativism seems promising. It is one thing to say that
conversations about humour presuppose a common standard for humour. It
is much less plausible to say that conversations about what might be the
case presuppose that the parties to the conversation know the same
things. None of these hurdles seem impossible to clear, but they do
raise doubts about whether presuppositions can solve all the
contextualists' problems with disagreement.

\subsection{Relativism and
Disagreement}\label{relativismanddisagreement}

If, after all that, we conclude that the contextualist still has a
problem with disagreement, it's fair to ask whether the relativist does
any better. Let's think again about López de Sa's example of Hannah and
Sarah.

\begin{quote}
Hannah: Homer Simpson is funny.\\
Sarah: I disagree. Homer is not funny.
\end{quote}

A simple relativist theory assigns truth conditions to Hannah's
utterance relative to a context of utterance, her own, and an index that
consists of a world-perspective pair. (Remember that we're using
`perspective' to cover everything other than a world that a relativist
may want to put into an index, and that it will be a structured entity.)
The content of Hannah's utterance will be set by her context. So it will
be (or at least will determine) a function from indices, i.e.,
world-perspective pairs, to truth-values. Call the world Hannah and
Sarah occupy \emph{w}, and their perspectives \emph{p\textsubscript{H}}
and \emph{p\textsubscript{S}}. Then the proposition that Homer Simpson
is funny will be true relative to
\(\langle\)\emph{w},~\emph{p\textsubscript{H}}\(\rangle\) and false
relative to \(\langle\)\emph{w},~\emph{p\textsubscript{S}}\(\rangle\).
So there is a proposition Hannah accepts and Sarah rejects, and so they
disagree. Doesn't think mean that the relativist can explain the sense
in which Hannah and Sarah disagree?

Not so fast. Consider another case involving One, who lives in
\emph{w}\textsubscript{1} where Mars has one moon, and Two, who lives in
\emph{w}\textsubscript{2} where Mars has two moons. They make the
following utterances:

\begin{quote}
One: Mars has one moon.\\
Two: Mars has two moons.
\end{quote}

Now most theorists would say that One and Two have expressed
propositions that cannot be true together. (As noted earlier, Schaffer
(\citeproc{ref-Schaffer2012}{2012}) disagrees, though not in a way that
helps relativism.) But they don't disagree. Among other things, they
both think that the other speaks truly.

Hannah doesn't just think that the proposition that Homer Simpson is
funny is true relative to
\(\langle\)\emph{w},~\emph{p\textsubscript{H}}\(\rangle\); she thinks it
is simply true. That is because she occupies (for want of a better word)
the index \(\langle\)\emph{w},~\emph{p\textsubscript{H}}\(\rangle\). And
one doesn't just think that the proposition that Mars has one moon is
true relative to
\(\langle\)\emph{w}\textsubscript{1},~\emph{p}\textsubscript{1}\(\rangle\),
she thinks it is simply true. But in doing so, she need not be in
disagreement with someone who occupies a different index, such as Two,
and who thinks it is false. It isn't easy to read off the existence of
disagreement from the endorsement of conflicting propositions when the
parties occupy different indices. And that raises doubts about whether
the relativist has really explained the disagreement. (The Mars example
is from MacFarlane (\citeproc{ref-MacFarlane2014}{2014, 128}). Both
Dreier (\citeproc{ref-Dreier2009}{2009}) and Francén
(\citeproc{ref-Francen2010}{2010}) raise doubts about whether the
relativist can explain disagreement.)

Part of MacFarlane's response to this is to insist that the different
elements of an index are treated differently. If we are relativists
about propositional truth, but not utterance truth, it is natural to
treat the two elements the same. A speaker, we'll then say, speaks truly
iff the proposition they express is true relative to the context they
occupy. But that isn't MacFarlane's view. He holds that an utterance is
true, i.e., that a speaker speaks truly, relative to a context of
assessment iff the proposition they express is true relative to the
world they occupy, and the perspective of the context of assessment.
This gives him a way to distinguish the Hannah/Sarah case from the
One/Two case. There is still a lot more work to do to turn this into a
full theory of disagreement, and chapter 5 of MacFarlane's book has a
very careful study of the varieties of disagreement that are possible on
a relativst theory, and how they can be used to explain the data. We're
not going to attempt to evaluate the success of these responses, but
rather conclude by noting that even if the contextualist has work to do
to explain the phenomena involving disagreement, so does the relativist.

\subsection{Problems with the Data}\label{problemswiththedata}

Most of the arguments in the literature have started with intuitions
about disagreeement. We don't think there is anything wrong with this in
principle; indeed, it is what we've done so far. But when the intuitions
get a little shaky, as they are in a few cases we've described so far,
it is worth checking them more carefully, against a broader range of
informants. And when that is done, it isn't clear that the data help the
relativists as much as the relativists have claimed. Knobe and Yalcin
(\citeproc{ref-KnobeYalcin2014}{2014}) provide evidence that the
following claim, which we'll call (J), isn't as empirically justified as
the relativists have made it out to be.

\begin{quote}
(J) Competent speaker/hearers tend to judge a present-tense bare
epistemic possibility claim (BEP) true only if the prejacent is
compatible with their information (whether or not they are the producer
of that utterance); otherwise the BEP is judged false.
\end{quote}

They argue that many relativists, in particular Egan and MacFarlane, are
committed to this claim. Although Egan and MacFarlane differ on several
points (Egan takes relativism about propositional truth to be primary,
MacFarlane relativism about utterance truth), it does seem true that (J)
is important to both of them. As Knobe and Yalcin put it,

\begin{quote}
Egan and MacFarlane are both clearly animated by the thought that
``people tend to assess epistemic modal claims for truth in light of
what they (the assessors) know, even if they realize that they know more
than the speaker (or relevant group) did at the time of utterance''
(MacFarlane 2011: 160; see also Egan 2007: 2-5, the section entitled
``Motivation for relativism: eavesdroppers'').
~(\citeproc{ref-KnobeYalcin2014}{Knobe and Yalcin 2014, 3--4})
\end{quote}

This isn't what their data showed. The subjects were shown speakers
whose evidence strongly, but falsely, suggested that Fat Tony was dead.
They overwhelmingly said that an utterance of ``Fat Tony is dead'' is
false, but most said an utterance of ``Fat Tony might be dead'' was
true. (Though it is worth noting that the responses displayed a
considerably ambivalence; the answers weren't in line with what either a
contextualist or a relativist would straightforwardly predict.) The
subjects did say that it would be correct for the speaker who said ``Fat
Tony might be dead'' to retract that utterance once it was clear Fat
Tony was alive. But this typically wasn't because they thought the
earlier utterance was false.

The point of this study was not to directly target intuitions about
disagreement, but rather inter-contextual judgments and felicity of
retraction. But the issues are closely related. If subjects who know Fat
Tony is alive don't judge that an utterance of ``Fat Tony might be
dead'' is false, then either they don't disagree with such an utterer,
or the disagreement is, as Huvenes suggests, not the kind that motivates
altering our semantics in the direction relativists suggest.

As Knobe and Yalcin are careful to note, even if relativists were
completely wrong about inter-contextual evaluation of utterances, about
disagreement, and about retraction, there are still other arguments for
relativism. We'll end with one other such argument, due primarily to
Tamina Stephenson (\citeproc{ref-Stephenson2007}{2007}).

\section{Control and Syntax}\label{controlandsyntax}

There is a striking semantic/syntactic phenomenon that epistemic modals
and predicates of personal taste share. It's easiest to describe the
phenomenon if we assume a contextualist semantics, though we'll
eventually use the puzzle to cast doubt on that semantics. So assume,
for now, that \emph{It must be that p} is true iff \emph{p} is
guaranteed to be true by the (possibly idealised) knowledge of \emph{X},
where \emph{X} is an individual or group supplied by context. (And
perhaps the amount of idealisation is context-sensitive too.) And assume
that \emph{F is tasty} is true iff \emph{F} tastes good to (the possibly
idealised version of) \emph{X}, where \emph{X} is again supplied by
context.

When `might' or `tasty' are not embedded, then \emph{X} seems like it
has to include the speaker, and perhaps not much more than that. Even
making some other taster or knower salient does not suffice to change
the value of \emph{X}. For instance, if one utters ``Joe is a great cook
and connoisseur of fine food. The meals he prepares are always tasty,''
the last sentence is not naturally construed as saying that \emph{Joe}
always likes the taste of what Joe cooks, but that the speaker and her
hearer do (or will). Or if, to borrow an example from
~(\citeproc{ref-Weatherson2011-NoRoyal}{Weatherson 2011}), we say
``Jones must know who the killer is,'' the relevant \emph{X} for
interpreting the `must' consists of the speaker and her hearers, not
Jones. There is something strange about this; it normally isn't that
hard to make others relevant in a way that makes them the value of a
contextually filled variable.

Or, perhaps, it isn't normally that hard with one key exception. Some
context-sensitive terms have, as part of their meaning, that their
extension includes the speaker. Such terms include `I' and `we'. Perhaps
`must' and `tasty' are like `I' and `we' in this respect. Except, and
here is Stephenson's key insight, there is a big difference between
`I'/`we' and `must'/`tasty'. The former still pick out the speaker (and
perhaps those near her) under the sceopt of an attitude verb. The latter
do not. Consider the natural interpretations of these sentences.

\begin{itemize}
\tightlist
\item
  Joe thinks my gumbo is tasty.
\item
  Joe thinks we must have stolen the candy.
\end{itemize}

In each case, the personal pronouns (`my', `we') have their customary
denotations. It is the speaker's gumbo that is being praised, and the
speaker and her friends who are being suspected of theft. But `tasty'
and `must' do not behave like that. It isn't that Joe thinks the speaker
likes the taste of her gumbo, but that Joe himself does. And it isn't
that Joe thinks the speaker's evidence entails her guilt, but that Joe's
evidence does.

Stephenson points out that we get even more dramatic results with more
complex sentences. Consider these two sentences.

\begin{itemize}
\tightlist
\item
  Mary thinks that Sam thinks it must be raining.
\item
  Mary thinks that Sam must think it is raining.
  ~(\citeproc{ref-Stephenson2007}{Stephenson 2007, 490})
\end{itemize}

The value of \emph{X} for the first `must' has to be Sam, and for the
second `must' it has to be Mary. In neither case is it the speaker, and
in neither case is it even particularly optional how to interpret the
`must'. Compare \emph{Mary thinks that Jane likes her house}, which is
naturally read as three-way ambiguous. (The house might be Jane's, or
Mary's, or a contextually supplied third party's.) The general point
here is that the \emph{X} values that the contextualist posits behave
rather unlike other implicit or explicit context-sensitive terms.

The relativist explanation of this is that epistemic modals, and
predicates of personal taste are, to use Stephenson's phrase,
``inherently judge-dependent''. That is, they are inherently dependent
on some perspective for their truth. In the terms we've been using so
far, we need a perpective in the index, and not just in the context, to
explain the truth of these claims. When this is combined with a natural
view that attitude predicates obligatorily shift the perspective (or
judge) parameter, then it just naturally falls out that epistemic modals
must take on the perspective of the immediate subjective of the attitude
verb. Such a semantics is defended by Stephenson, and by Peter Lasersohn
(\citeproc{ref-Lasersohn2005}{2005}).

It falls out of this semantics that there cannot be ``exocentric'' uses
of bare epistemic modals. These are uses of bare epistemic modals that
take on the perspective of someone other than the speaker. (A similar
point applies to predicates of personal taste.) This is a nice
explanation of the fact that it is very hard to generate these
exocentric uses. But it arguably overgenerates; there are cases where it
seems we do get the exocentric reading. Here is one case from Egan,
Hawthorne, and Weatherson (\citeproc{ref-Egan2005-EGAEMI}{2005}).

\begin{quote}
Ann is planning a surprise party for Bill. Unfortunately, Chris has
discovered the surprise and told Bill all about it. Now Bill and Chris
are having fun watching Ann try to set up the party without being
discovered. Currently Ann is walking past Chris's apartment carrying a
large supply of party hats. She sees a bus on which Bill frequently
rides home, so she jumps into some nearby bushes to avoid being spotted.
Bill, watching from Chris's window, is quite amused, but Chris is
puzzled and asks Bill why Ann is hiding in the bushes. Bill says, ``I
might be on that bus'' ~(\citeproc{ref-Egan2005-EGAEMI}{Egan, Hawthorne,
and Weatherson 2005, 140})
\end{quote}

The natural reading of what Bill says is that for all \emph{Ann} knows,
Bill is on the bus. And this is predicted to be impossible on the type
of relativist view that gets us the correct results in the obligatory
control cases. Stephenson suggests that there is an ellipsis in Bill's
sentence. It should really be understood as:

\begin{itemize}
\tightlist
\item
  Ann is hiding in the bushes because I might be on that bus.
\end{itemize}

So in a sense, it isn't a bare epistemic modal; it is in a
`because'-clause. Moreover, suggests Stephenson, we should take
`because'-clauses to express something like a person's conscious
reasoning or rationale. So in this case `because' acts like an attitude
verb and shift the perspective (or judge) of the epistemic modal. (A
similar move, in response to a similar objection, is defended by John
MacFarlane (\citeproc{ref-MacFarlane2014}{2014, 272ff}).)

The relativist needs two premises here for the defence to work. The
first is that all these exocentric uses are either in the scope of an
attitude verb, or are in an explanatory context. The second is that it
is fair to treat `because' as sufficiently like an attitude verb for
these purposes. A contextualist could well object to either assumption.
But even if they grant both assumptions, a contextualist may want to
simply resist the whole line of reasoning. At most what these arguments
about control show is that `might', and `tasty', behave rather
differently to other context-sensitive terms. There is nothing
inconsistent about just accepting that as a surprising fact. And given
how radical a thesis relativism seems to many, accepting relativism as
an explanation for these facts about control could well be an excessive
reaction. The issues here have not been worked out in nearly as much
detail as the issues concerning disagreement and retraction, and we are
a long way from having a full accounting of the costs and benefits of
the possible dialectical moves.

\subsection*{References}\label{references}
\addcontentsline{toc}{subsection}{References}

\phantomsection\label{refs}
\begin{CSLReferences}{1}{0}
\bibitem[\citeproctext]{ref-CappelenHawthorne2009}
Cappelen, Herman, and John Hawthorne. 2009. \emph{Relativism and Monadic
Truth}. Oxford: Oxford University Press.

\bibitem[\citeproctext]{ref-CappelenHuvenes2014}
Cappelen, Herman, and Torfinn Tomesen Huvenes. 2014. {``Relative
Truth.''} In \emph{The Oxford Handbook of Truth}, edited by Michael
Glanzberg. Oxford: Oxford University Press.

\bibitem[\citeproctext]{ref-Cappelen2005}
Cappelen, Herman, and Ernest Lepore. 2005. \emph{Insensitive Semantics:
A Defence of Semantic Minimalism and Speech Act Pluralism}. Oxford:
Blackwell.

\bibitem[\citeproctext]{ref-DeRose2009}
DeRose, Keith. 2009. \emph{The Case for Contextualism: Knowledge,
Skepticism and Context}. Oxford: Oxford.

\bibitem[\citeproctext]{ref-Dreier2009}
Dreier, James. 2009. {``Relativism (and Expressivism) and the Problem of
Disagreement.''} \emph{Philosophical Perspectives} 23: 79--110. doi:
\href{https://doi.org/10.1111/j.1520-8583.2009.00162.x}{10.1111/j.1520-8583.2009.00162.x}.

\bibitem[\citeproctext]{ref-Egan2007-EGAEMR}
Egan, Andy. 2007. {``{Epistemic Modals, Relativism and Assertion}.''}
\emph{Philosophical Studies} 133 (1): 1--22. doi:
\href{https://doi.org/10.1007/s11098-006-9003-x}{10.1007/s11098-006-9003-x}.

\bibitem[\citeproctext]{ref-Egan2005-EGAEMI}
Egan, Andy, John Hawthorne, and Brian Weatherson. 2005. {``{Epistemic
Modals in Context}.''} In \emph{Contextualism in Philosophy: Knowledge,
Meaning, and Truth}, edited by Gerhard Preyer and Georg Peter, 131--70.
Oxford: Oxford University Press.

\bibitem[\citeproctext]{ref-Einheuser2008}
Einheuser, Iris. 2008. {``Three Forms of Truth-Relativism.''} In
\emph{Relativising Utterance Truth}, edited by Manuel Garcia-Carpintero
and Max Kölbel, 187--203. Oxford: Oxford University Press.

\bibitem[\citeproctext]{ref-Evans1985}
Evans, Gareth. 1985. {``Does Tense Logic Rest on a Mistake?''} In
\emph{Collected Papers}, 343--63. Oxford: Clarendon Press.

\bibitem[\citeproctext]{ref-vonFintelGillies2008}
Fintel, Kai von, and Anthony S. Gillies. 2008. {``CIA Leaks.''}
\emph{Philosophical Review} 117 (1): 77--98. doi:
\href{https://doi.org/10.1215/00318108-2007-025}{10.1215/00318108-2007-025}.

\bibitem[\citeproctext]{ref-Francen2010}
Francén, Ragnar. 2010. {``No Deep Disagreement for New Relativists.''}
\emph{Philosohical Studies} 151 (1): 19--37. doi:
\href{https://doi.org/10.1007/s11098-009-9414-6}{10.1007/s11098-009-9414-6}.

\bibitem[\citeproctext]{ref-Harman1975}
Harman, Gilbert. 1975. {``Moral Relativism Defended.''}
\emph{Philosophical Review} 84 (1): 3--22. doi:
\href{https://doi.org/10.2307/2184078}{10.2307/2184078}.

\bibitem[\citeproctext]{ref-HarmanThomson1996}
Harman, Gilbert, and Judith Jarvis Thomson. 1996. \emph{Moral Relativism
and Moral Objectivity}. Cambridge, MA: Blackwell.

\bibitem[\citeproctext]{ref-Huvenes2012}
Huvenes, Torfinn Thomesen. 2012. {``Varieties of Disagreement and
Predicates of Taste.''} \emph{Australasian Journal of Philosophy} 90
(1): 167--81. doi:
\href{https://doi.org/10.1080/00048402.2010.550305}{10.1080/00048402.2010.550305}.

\bibitem[\citeproctext]{ref-Kamp1971}
Kamp, Hans. 1971. {``Formal Properties of {`Now'}.''} \emph{Theoria} 37:
227--74.

\bibitem[\citeproctext]{ref-Kaplan1977}
Kaplan, David. 1989. {``Demonstratives.''} In \emph{Themes from Kaplan},
edited by Joseph Almog, John Perry, and Howard Wettstein, 481--563.
Oxford: Oxford University Press.

\bibitem[\citeproctext]{ref-KnobeYalcin2014}
Knobe, Joshua, and Seth Yalcin. 2014. {``Epistemic Modals and Context:
Experimental Data.''} \emph{Semantics and Pragmatics} 7 (10): 1--21.
doi: \href{https://doi.org/10.3765/sp.7.10}{10.3765/sp.7.10}.

\bibitem[\citeproctext]{ref-Kolbel2002}
Kölbel, Max. 2002. \emph{Truth Without Objectivity}. London: Routledge.

\bibitem[\citeproctext]{ref-Kolbel2004}
---------. 2004. {``Indexical Relativism Versus Genuine Relativism.''}
\emph{International Journal of Philosophical Studies} 12 (3): 297--313.
doi:
\href{https://doi.org/10.1080/0967255042000243966}{10.1080/0967255042000243966}.

\bibitem[\citeproctext]{ref-Lasersohn2005}
Lasersohn, Peter. 2005. {``Context Dependence, Disagreement and
Predicates of Personal Taste.''} \emph{Linguistics and Philosophy} 28
(6): 643--86. doi:
\href{https://doi.org/10.1007/s10988-005-0596-x}{10.1007/s10988-005-0596-x}.

\bibitem[\citeproctext]{ref-Lewis1979b}
Lewis, David. 1979. {``Attitudes \emph{de Dicto} and \emph{de Se}.''}
\emph{Philosophical Review} 88 (4): 513--43. doi:
\href{https://doi.org/10.2307/2184646}{10.2307/2184646}. Reprinted in
his \emph{Philosophical Papers}, Volume 1, Oxford: Oxford University
Press, 1983, 133-156. References to reprint.

\bibitem[\citeproctext]{ref-Lewis1980b}
---------. 1980. {``Index, Context, and Content.''} In \emph{Philosophy
and Grammar}, edited by Stig Kanger and Sven Öhman, 79--100. Dordrecht:
Reidel. Reprinted in his \emph{Papers in Philosophical Logic},
Cambridge: Cambridge University Press, 1998, 21-44. References to
reprint.

\bibitem[\citeproctext]{ref-Lewis1989b}
---------. 1989. {``Dispositional Theories of Value.''}
\emph{Aristotelian Society Supplementary Volume} 63 (1): 113--37. doi:
\href{https://doi.org/10.1093/aristoteliansupp/63.1.89}{10.1093/aristoteliansupp/63.1.89}.
Reprinted in his \emph{Papers in Ethics and Social Philosophy},
Cambridge: Cambridge University Press, 2000, 68-94. References to
reprint.

\bibitem[\citeproctext]{ref-LopezDeSa2008b}
López de Sa, Dan. 2008. {``Presuppositions of Commonality.''} In
\emph{Relativising Utterance Truth}, edited by Manuel Garcia-Carpintero
and Max Kölbel, 297--310. Oxford University Press.

\bibitem[\citeproctext]{ref-MacFarlane2014}
MacFarlane, John. 2014. \emph{Assessment Sensitivity}. Oxford: Oxford
University Press.

\bibitem[\citeproctext]{ref-Recanati2007}
Recanati, François. 2007. \emph{Perspectival Thought: A Plea for
(Moderate) Relativism}. Oxford: Oxford University Press.

\bibitem[\citeproctext]{ref-Richard2008}
Richard, Mark. 2008. \emph{When Truth Gives Out}. Oxford: Oxford
University Press.

\bibitem[\citeproctext]{ref-Schaffer2012}
Schaffer, Jonathan. 2012. {``Necessitarian Propositions.''}
\emph{Synthese} 189 (1): 119--62. doi:
\href{https://doi.org/10.1007/s11229-012-0097-8}{10.1007/s11229-012-0097-8}.

\bibitem[\citeproctext]{ref-Stalnaker1984}
Stalnaker, Robert. 1984. \emph{Inquiry}. Cambridge, MA: MIT Press.

\bibitem[\citeproctext]{ref-Stephenson2007}
Stephenson, Tamina. 2007. {``Judge Dependence, Epistemic Modals, and
Predicates of Personal Taste.''} \emph{Linguistics and Philosophy} 30
(4): 487--525. doi:
\href{https://doi.org/10.1007/s10988-008-9023-4}{10.1007/s10988-008-9023-4}.

\bibitem[\citeproctext]{ref-Weatherson2011-NoRoyal}
Weatherson, Brian. 2011. {``No Royal Road to Relativism.''}
\emph{Analysis} 71 (1): 133--43. doi:
\href{https://doi.org/10.1093/analys/anq060}{10.1093/analys/anq060}.

\bibitem[\citeproctext]{ref-WeathersonEgan2011}
Weatherson, Brian, and Andy Egan. 2011. {``Epistemic Modals and
Epistemic Modality.''} In \emph{Epistemic Modality}, edited by Andy Egan
and Brian Weatherson, 1--18. Oxford: Oxford University Press.

\end{CSLReferences}



\noindent Published in\emph{
Blackwell Companion to the Philosophy of Language}, 2017, pp. 787-803.


\end{document}
