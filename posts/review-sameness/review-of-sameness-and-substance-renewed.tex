% Options for packages loaded elsewhere
\PassOptionsToPackage{unicode}{hyperref}
\PassOptionsToPackage{hyphens}{url}
%
\documentclass[
  10pt,
  letterpaper,
  DIV=11,
  numbers=noendperiod,
  twoside]{scrartcl}

\usepackage{amsmath,amssymb}
\usepackage{setspace}
\usepackage{iftex}
\ifPDFTeX
  \usepackage[T1]{fontenc}
  \usepackage[utf8]{inputenc}
  \usepackage{textcomp} % provide euro and other symbols
\else % if luatex or xetex
  \usepackage{unicode-math}
  \defaultfontfeatures{Scale=MatchLowercase}
  \defaultfontfeatures[\rmfamily]{Ligatures=TeX,Scale=1}
\fi
\usepackage{lmodern}
\ifPDFTeX\else  
    % xetex/luatex font selection
  \setmainfont[ItalicFont=EB Garamond Italic,BoldFont=EB Garamond
Bold]{EB Garamond Math}
  \setsansfont[]{Europa-Bold}
  \setmathfont[]{Garamond-Math}
\fi
% Use upquote if available, for straight quotes in verbatim environments
\IfFileExists{upquote.sty}{\usepackage{upquote}}{}
\IfFileExists{microtype.sty}{% use microtype if available
  \usepackage[]{microtype}
  \UseMicrotypeSet[protrusion]{basicmath} % disable protrusion for tt fonts
}{}
\usepackage{xcolor}
\usepackage[left=1in, right=1in, top=0.8in, bottom=0.8in,
paperheight=9.5in, paperwidth=6.5in, includemp=TRUE, marginparwidth=0in,
marginparsep=0in]{geometry}
\setlength{\emergencystretch}{3em} % prevent overfull lines
\setcounter{secnumdepth}{3}
% Make \paragraph and \subparagraph free-standing
\ifx\paragraph\undefined\else
  \let\oldparagraph\paragraph
  \renewcommand{\paragraph}[1]{\oldparagraph{#1}\mbox{}}
\fi
\ifx\subparagraph\undefined\else
  \let\oldsubparagraph\subparagraph
  \renewcommand{\subparagraph}[1]{\oldsubparagraph{#1}\mbox{}}
\fi


\providecommand{\tightlist}{%
  \setlength{\itemsep}{0pt}\setlength{\parskip}{0pt}}\usepackage{longtable,booktabs,array}
\usepackage{calc} % for calculating minipage widths
% Correct order of tables after \paragraph or \subparagraph
\usepackage{etoolbox}
\makeatletter
\patchcmd\longtable{\par}{\if@noskipsec\mbox{}\fi\par}{}{}
\makeatother
% Allow footnotes in longtable head/foot
\IfFileExists{footnotehyper.sty}{\usepackage{footnotehyper}}{\usepackage{footnote}}
\makesavenoteenv{longtable}
\usepackage{graphicx}
\makeatletter
\def\maxwidth{\ifdim\Gin@nat@width>\linewidth\linewidth\else\Gin@nat@width\fi}
\def\maxheight{\ifdim\Gin@nat@height>\textheight\textheight\else\Gin@nat@height\fi}
\makeatother
% Scale images if necessary, so that they will not overflow the page
% margins by default, and it is still possible to overwrite the defaults
% using explicit options in \includegraphics[width, height, ...]{}
\setkeys{Gin}{width=\maxwidth,height=\maxheight,keepaspectratio}
% Set default figure placement to htbp
\makeatletter
\def\fps@figure{htbp}
\makeatother

\setlength\heavyrulewidth{0ex}
\setlength\lightrulewidth{0ex}
\usepackage[automark]{scrlayer-scrpage}
\clearpairofpagestyles
\cehead{
  Brian Weatherson
  }
\cohead{
  Review of “Sameness and Substance Renewed”
  }
\ohead{\bfseries \pagemark}
\cfoot{}
\makeatletter
\newcommand*\NoIndentAfterEnv[1]{%
  \AfterEndEnvironment{#1}{\par\@afterindentfalse\@afterheading}}
\makeatother
\NoIndentAfterEnv{itemize}
\NoIndentAfterEnv{enumerate}
\NoIndentAfterEnv{description}
\NoIndentAfterEnv{quote}
\NoIndentAfterEnv{equation}
\NoIndentAfterEnv{longtable}
\NoIndentAfterEnv{abstract}
\renewenvironment{abstract}
 {\vspace{-1.25cm}
 \quotation\small\noindent\rule{\linewidth}{.5pt}\par\smallskip
 \noindent }
 {\par\noindent\rule{\linewidth}{.5pt}\endquotation}
\KOMAoption{captions}{tableheading}
\makeatletter
\@ifpackageloaded{caption}{}{\usepackage{caption}}
\AtBeginDocument{%
\ifdefined\contentsname
  \renewcommand*\contentsname{Table of contents}
\else
  \newcommand\contentsname{Table of contents}
\fi
\ifdefined\listfigurename
  \renewcommand*\listfigurename{List of Figures}
\else
  \newcommand\listfigurename{List of Figures}
\fi
\ifdefined\listtablename
  \renewcommand*\listtablename{List of Tables}
\else
  \newcommand\listtablename{List of Tables}
\fi
\ifdefined\figurename
  \renewcommand*\figurename{Figure}
\else
  \newcommand\figurename{Figure}
\fi
\ifdefined\tablename
  \renewcommand*\tablename{Table}
\else
  \newcommand\tablename{Table}
\fi
}
\@ifpackageloaded{float}{}{\usepackage{float}}
\floatstyle{ruled}
\@ifundefined{c@chapter}{\newfloat{codelisting}{h}{lop}}{\newfloat{codelisting}{h}{lop}[chapter]}
\floatname{codelisting}{Listing}
\newcommand*\listoflistings{\listof{codelisting}{List of Listings}}
\makeatother
\makeatletter
\makeatother
\makeatletter
\@ifpackageloaded{caption}{}{\usepackage{caption}}
\@ifpackageloaded{subcaption}{}{\usepackage{subcaption}}
\makeatother
\ifLuaTeX
  \usepackage{selnolig}  % disable illegal ligatures
\fi
\usepackage{bookmark}

\IfFileExists{xurl.sty}{\usepackage{xurl}}{} % add URL line breaks if available
\urlstyle{same} % disable monospaced font for URLs
\hypersetup{
  pdftitle={Review of ``Sameness and Substance Renewed''},
  pdfauthor={Brian Weatherson},
  hidelinks,
  pdfcreator={LaTeX via pandoc}}

\title{Review of ``Sameness and Substance Renewed''}
\author{Brian Weatherson}
\date{2002}

\begin{document}
\maketitle
\begin{abstract}
Review of David Wiggins, ``Sameness and Substance Renewed''. Cambridge:
Cambridge University Press, 2001.
\end{abstract}

\setstretch{1.1}
\emph{Sameness and Substance Renewed} (hereafter, 2001) is, in effect, a
second edition of Wiggins's 1980 book \emph{Sameness and Substance}
(hereafter, 1980), which in turn expanded and corrected some ideas in
his 1967 \emph{Identity and Spatio-Temporal Continuity} (hereafter,
1967). All three books have similar aims. The first is to argue,
primarily against Geach, that identity is absolute not relative. The
second is to argue that, despite this, whenever an identity claim
\emph{a}~=~\emph{b} is true, there is a sortal \emph{f} such that
\emph{a} is the same \emph{f} as \emph{b}. The biggest difference
between 1967 and the two later books is that the later books contain
much more detail on what a sortal must be if this claim, called
\textbf{D}, is to be both correct and philosophically interesting. The
third aim is to apply the first two conclusions to the topic of personal
identity.

For the bulk of 2001, most of the changes to 1980 are confined to
footnotes, the bulk of these consisting of a citation of, and
occasionally a brief comment upon, a post 1980 publication that bears on
Wiggins's approach to the topic. This changes in the last two chapters.
The penultimate chapter is a mostly new discussion of the determinacy of
identity. In the final chapter, on personal identity, Wiggins retracts
many of the claims made in the matching chapter of 1980, and raises some
interesting objections to Parfit's account of personal identity. Apart
from those, the major changes to 1980 are stylistic. No longer is the
material considered more peripheral printed in smaller type, though to
make up for this there are frequent exhortations to skip these sections.
And the longer notes of the 1980 edition are now mostly incorporated
into the text, several prefixed with advice that they not be read.

In the preface Wiggins says that it is a matter of no concern whether
1980 and 2001 are the same book. But it is of concern to me, twice over.
First, it makes a large difference to what kind of review should be
written whether this is an old book reissued or a new book. Secondly,
the difficulty in answering this question draws out some problems
Wiggins's theory faces when we try to apply it outside the realms of
physics and biology. On the first problem, the reader who hopes here to
find a comprehensive discussion of the literature on identity post-1980
as it strikes David Wiggins will be disappointed. Three examples should
help to illustrate this.

In 1967 Wiggins held, quite sensibly, that a statue is not identical to
the bronze that it is made of, but rather is \emph{constituted} by that
bronze. This was an important move in his response to Geach. If identity
is absolute, and the statue is identical with the bronze, then we can't
say that when the statue is remoulded into a vase, we have the same lump
of bronze but a different artwork. In 2001 he says much the same thing.
This still seems like good common sense, but the problem is that in the
intervening 34 years there has been a mass of work on constitution, most
of it concluding that the constitution relation is much more problematic
than we had originally thought. In a genuinely new work, or even perhaps
in a revised old work, this material should have been addressed.

In 1980 Wiggins makes it quite clear he dislikes perdurantist theories
of persistence that hold that an object persists from
\emph{t}­\textsubscript{1} to \emph{t}\textsubscript{2} by having
instantaneous temporal parts at every time in
{[}\emph{t}\textsubscript{1}, \emph{t}\textsubscript{2}{]}. Just why he
dislikes perdurantism is unclear, since all his arguments are directed
against the conjunction of this view with the striking, and not
especially popular, view that our ordinary names refer to these
instantaneous objects. In 2001 it is still clear he dislikes
perdurantism. But we find little on the barrage of arguments
perdurantists have offered in the last 21 years in support of their
position. All we find is a footnote expressing agreement with Mark
Johnston's response to Lewis's `problem of temporary intrinsics'
argument, and a citation of a paper expanding upon said agreement. Given
that this book largely reprints previously available material, including
more detail here would not have been absurd, and saying something about
other arguments quite appropriate.

The third example is a little more serious. In the book's new chapter,
he outlines approvingly Evans's proof that identity is always
determinate identity, and cites (without outlining) a proof by
Williamson that distinctness is always determinate. From these proofs he
quite naturally concludes that the prospects for indeterminate
identities are pretty grim. But he doesn't engage with those who
maintain that, despite all this, there really are indeterminate
identities. It would have been worthwhile, for instance, to see a more
direct engagement between his views and those of, say, Terrence Parsons,
who over the last 15 years has developed a rather detailed theory on
which indeterminate identity is possible. It will probably turn out that
Wiggins's position is entirely correct, and Parsons's position basically
mistaken, but that's no reason to not take Parsons more seriously.

Apart from this oversight, there is one rather odd feature in the
discussion of determinacy. Wiggins says that ``it can be \emph{perfectly
determinate} which mountain \emph{x} is without \emph{x}'s extent being
determinate.'' (166) The idea is that it can be determinate that
\emph{x} is, say, this mountain, while it is indeterminate whether, say,
that foothill is part of \emph{x}. Such a position always feels strained
to me, but it is certainly not unfamiliar. But it is very hard to see
how it is meant to fit in with Wiggins's picture of the role of sortal
concepts, such as \emph{mountain}. On page 70 he says a sortal concept
is such that grasp of it determines ``what changes \emph{x} tolerates
without there ceasing to exist such a thing as \emph{x}.'' (He actually
says `substance-concept', not `sortal concept' there, but these phrases
seem to be used synonymously.) The trouble should be apparent. For it to
be determinate what \emph{x} is, presumably just is for it to be
determinate that \emph{x} is this \emph{mountain}. That is, it is
determinate that \emph{x} falls under the sortal \emph{mountain}, and
that sortal must determine persistence conditions, else it fails to be a
sortal. But that means \emph{x}'s persistence conditions are determined.
So there is a determinate fact, perhaps unknown and perhaps even
unknowable, about how far in the future one can go without leaving
\emph{x} behind. (I assume here that if \emph{x}'s temporal extent is
\emph{determined} by which sortal it falls under, then \emph{x}'s
temporal extent is \emph{determinate}. I imagine some will deny this
claim, but it looks like a platitude to me.) On the other hand, it seems
that it can be determinate what \emph{x} is even though the conditions
of \emph{x}'s spatial persistence, conditions that determine how far
westward one can go without leaving \emph{x} behind, are not
determinate. Just how this asymmetry is to be tolerated is not
explained.

Much of the interest in this edition will focus on the new material on
personal identity, and I shall say more about this below. But it is
worth going over the central claims of the earlier part of the book. The
crucial principle is called \textbf{D}(ii). The derivation of it appeals
crucially to \textbf{D}(i). (Both definitions, and the commentary, from
page 64.)

\begin{description}
\tightlist
\item[D(i)]
(\emph{x})(\emph{t}){[}(\emph{x} exists at \emph{t})~→ (∃\emph{g})
(\emph{g}(\emph{x})~at \emph{t}){]}.
\item[D(ii)]
(\emph{x})(∃\emph{g})(\emph{t}) {[}(\emph{x} exists at \emph{t})~→
(\emph{g}(\emph{x})~at \emph{t}){]}.
\end{description}

`\emph{x}' ranges over three-dimensional continuants, `\emph{t}' over
times and `\emph{g}' over sortals. The argument for \textbf{D}(i) is
that for any object at any time there is an answer to the Aristotelian
question \emph{What is it}? This answer is a sortal so, as just noted,
it must determine principles of persistence. It must also determine ``a
prnciple of \emph{activity}, a principle of \emph{functioning} or a
principle of \emph{operation}''. (72) If the last claim looks
disjunctive, that's because it is. Sortals for living objects determine
principles of activity, sortals for artifacts determine principles of
functioning. Just what philosophically interesting features these
principles share is never satisfactorily explained. So there's a
suspicion that there is no decent concept of \emph{sortal} that covers
the kinds of things living creatures are and the kinds of things
artifacts are. In slogan form \emph{sortal} isn't a sortal. Two other
considerations reinforce that suspicion.

First, there are objects that don't naturally fall under any known
sortal. Just looking at the computer I'm now using, there is the latch
that holds the lid down when it's closed, the button that opens the CD
tray, the brightness control, and the stick that plays some (but not
all) the functional roles of a mouse. It's far from obvious that any of
these falls under a sortal, at least if a sortal must determine
persistence conditions and a principle of functioning.

Secondly, there's a tension between the kinds of sortals Wiggins thinks
appropriate for artifacts and what he says about persistence. Artifact
sortals are, he says, functional kinds. These sortals are meant to
determine persistence conditions. Whether an object has persisted is not
meant to depend on extrinsic, or external, factors. This is the upshot
of his \emph{Only a and b} rule (96), which is important in ruling out
`best deserver' theories of persistence. That rule says that we don't
need to consider objects other than \emph{a} or \emph{b} to determine
whether \emph{a} is \emph{b}. So whether \emph{a}, the boy genius is
\emph{b}, the Nobel Prize winning author, cannot depend on the existence
or otherwise of a person more closely continuous with \emph{a} than
\emph{b} happens to be. Hence Nozick's theory of personal identity,
which rejects this, cannot be true. But to determine whether \emph{a}´,
the brightness control on my computer at \emph{t}\textsubscript{1} is
\emph{b}´, the volume control on Jack's computer at
\emph{t}\textsubscript{2}, we need to see which sortals \emph{a}´ and
\emph{b}´ fall under, to see whether \emph{a}´ has persisted. That will
depend, in part, on whether \emph{a}´ still falls under that sortal,
whatever it is. Such a sortal will sort by functional role, and whether
that functional role is fulfilled at all times between
\emph{t}\textsubscript{1} and \emph{t}\textsubscript{2} will be
determined by things other than \emph{a}´ and \emph{b}´. The point
generalises: in most cases whether an object continues to fill a
functional role often turns on the existence, and behaviour, of
\emph{other} objects. The natural conclusion to be drawn here is that
\textbf{D}(i) might be false for artifacts.

Clearly \textbf{D}(i) doesn't entail \textbf{D}(ii). But, bracketing our
concerns about its truth, it might still be usable in an abductive
argument for \textbf{D}(ii). Wiggins does just this. The argument, and
this is I think the \emph{only} argument for \textbf{D}(ii), is that it
best explains our widespread agreement over whether an object has
survived. Wiggins says that, ``Our capacity for massive agreement about
this is much more remarkable than our occasional disagreement,'' (66)
and given \textbf{D}(i), \textbf{D}(ii) is the best explanation of this.
Two replies. First, the agreement is not all that widespread, even in
actual cases. Think, for example, about the range of disagreement over
whether a corpse is a thing that once lived, and hence the disagreement
over whether Auntie is buried behind the back shed, or Auntie no longer
exists. Secondly, there is a better explanation -- the perdurantist
explanation given by Lewis. This not only explains why there is
agreement just where there is agreement, but why there is disagreement
where there is disagreement. The story is familiar. All sorts of
continuants (fusions of temporal parts) exist, but we choose which ones
to refer to and quantify over because of our particular interests, and
those choices are codified in our linguistic practices. When dealing
with historically familiar situations, membership of linguistic, and
more broadly cultural, communities commits us to common answers. Since
most situations are historically familiar, we have agreement in most
cases. When we deal with new cases, either generated by new technology
or new imaginativeness, and our interests do not pick out a clearly
preferable continuant, we do not have agreement. If this is right, we
will agree about the familiar and disagree about the unfamiliar.
Happily, this is exactly what we find -- we all know what the
persistence conditions for cows, pigs and chicken are, we do not know
the persistence conditions for corporate entities or pieces of software
or even relocated football teams in perfectly everyday cases.

In 1980, the chapter on personal identity had two main aims. The first
was to argue that Lockean considerations about continuity of memory
could be used in an account of personal identity, even if they would
have to be used as reference fixers rather than as constituents of a
reductive analysis. The opponents here were those followers of Bishop
Butler, most prominently Anthony Flew, who held that any such
consideration would be hopelessly circular. The second aim was to argue
that it is a conceptual truth that persons are animals. The targets here
were (unnamed) philosophers who wanted to provide a complete functional
analysis of a person. The objections were mainly political but since it
is unclear whether the philosophy of personal identity should be part of
metaphysics or ethics, that's perfectly acceptable.

In 2001, the aims have changed. Wiggins retracts everything he said
against Butler and Flew, and then spends most of the chapter in a
lengthy discussion of Parfit's theory of personal identity. Butler's
original complaint against Locke was that the concept of memory needs a
pre-existing concept of personal identity to be applied so it cannot be
used in a non-circular account of identity. In the bluntest version of
this complaint, it is held that `A remembers X-ing' is properly
represented as `A remembers A X-ing', which requires an identity between
the referents of the two occurences of `A'. Wiggins's complaint, in
1980, was that this position costs us some vital distinctions. We want
to distinguish, for example, `A imagines being an elephant' from `A
imagines A being an elephant'. In the first case only, A imagines
something possible. And what holds for imagining, he thought, holds for
remembering. Wiggins now rejects the last step here. Remembering, unlike
imagining, has a tie to the truth. A can only remember X-ing if A in
fact X-ed.~This last claim might not be part of the logical form of `A
remembers X-ing', but there is a close relation between the two. Wiggins
calls the relationship presupposition, but the name here doesn't matter
much. So Butler, and Flew, were right -- appeals to memory in a theory
of personal identity are hopelessly circular, because they presuppose
that debates about identity through time are resolved.

The rest of the chapter outlines concerns with Parfit's theory of
personal identity, based on his concept of quasi-memory, and with the
intuitions behind some cases that support Parfit's theory. Quasi-memory,
unlike memory, need not be factive, but what is quasi-remembered must
have happened somewhere, to someone. Wiggins launches a barrage of
attacks on this idea, of which the following three seem most telling.
First, quasi-memory could not be defined (and is not defined) as memory
minus factiveness, because conceptual subtraction is undefined in the
absence of conceptual analysis, and we don't have a conceptual analysis
here. Secondly, the concept of quasi-memory may seem to make sense for
quite general, \emph{de dicto} memories, but it runs into trouble with
\emph{de re}, or even with more specific memories. If I am to
quasi-remember climbing Big Ben on my sixteenth birthday, must someone
have climbed Big Ben on my sixteenth birthday? Or perhaps on their
sixteenth birthday? Thirdly, even if Parfit can define the concept of
\emph{accurate quasi-memory}, that won't get us the general concept of
quasi-memory, because again conceptual subtraction doesn't make sense.

There are some good points here for followers of Parfit to consider.
They are followed by some interesting considerations about why we might
rethink our intuitions about personal identity in cases involving brain
swaps. Those interested in personal identity debates, and particularly
Wiggins's and Parfit's contributions, should pay close attention here.
These will not be the only people to whom this new volume has interest.
\emph{Sameness and Substance} was an important statement of a rather
commonsensical solution to some of the hardest questions in metaphysics,
a solution with which everyone working in the field should be well
acquainted. And those who have not previously read it closely will find
that the stylistic changes (having a uniform font size, incorporating
the longer notes into the text) make the renewed version of
\emph{Sameness and Substance} much more accessible than the original.

\vspace{1cm}



\noindent Published in \emph{Notre Dame Philosophical Reviews}, 2002 \\
\url{https://ndpr.nd.edu/news/sameness-and-substance-renewed/}

\end{document}
