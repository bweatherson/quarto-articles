% Options for packages loaded elsewhere
% Options for packages loaded elsewhere
\PassOptionsToPackage{unicode}{hyperref}
\PassOptionsToPackage{hyphens}{url}
%
\documentclass[
  11pt,
  letterpaper,
  DIV=11,
  numbers=noendperiod,
  twoside]{scrartcl}
\usepackage{xcolor}
\usepackage[left=1.1in, right=1in, top=0.8in, bottom=0.8in,
paperheight=11in, paperwidth=8.5in, includemp=TRUE, marginparwidth=0in,
marginparsep=0in]{geometry}
\usepackage{amsmath,amssymb}
\setcounter{secnumdepth}{3}
\usepackage{iftex}
\ifPDFTeX
  \usepackage[T1]{fontenc}
  \usepackage[utf8]{inputenc}
  \usepackage{textcomp} % provide euro and other symbols
\else % if luatex or xetex
  \usepackage{unicode-math} % this also loads fontspec
  \defaultfontfeatures{Scale=MatchLowercase}
  \defaultfontfeatures[\rmfamily]{Ligatures=TeX,Scale=1}
\fi
\usepackage{lmodern}
\ifPDFTeX\else
  % xetex/luatex font selection
  \setmainfont[ItalicFont=EB Garamond Italic,BoldFont=EB Garamond
Bold]{EB Garamond Math}
  \setsansfont[]{EB Garamond}
  \setmathfont[]{Garamond-Math}
\fi
% Use upquote if available, for straight quotes in verbatim environments
\IfFileExists{upquote.sty}{\usepackage{upquote}}{}
\IfFileExists{microtype.sty}{% use microtype if available
  \usepackage[]{microtype}
  \UseMicrotypeSet[protrusion]{basicmath} % disable protrusion for tt fonts
}{}
\usepackage{setspace}
% Make \paragraph and \subparagraph free-standing
\makeatletter
\ifx\paragraph\undefined\else
  \let\oldparagraph\paragraph
  \renewcommand{\paragraph}{
    \@ifstar
      \xxxParagraphStar
      \xxxParagraphNoStar
  }
  \newcommand{\xxxParagraphStar}[1]{\oldparagraph*{#1}\mbox{}}
  \newcommand{\xxxParagraphNoStar}[1]{\oldparagraph{#1}\mbox{}}
\fi
\ifx\subparagraph\undefined\else
  \let\oldsubparagraph\subparagraph
  \renewcommand{\subparagraph}{
    \@ifstar
      \xxxSubParagraphStar
      \xxxSubParagraphNoStar
  }
  \newcommand{\xxxSubParagraphStar}[1]{\oldsubparagraph*{#1}\mbox{}}
  \newcommand{\xxxSubParagraphNoStar}[1]{\oldsubparagraph{#1}\mbox{}}
\fi
\makeatother


\usepackage{longtable,booktabs,array}
\usepackage{calc} % for calculating minipage widths
% Correct order of tables after \paragraph or \subparagraph
\usepackage{etoolbox}
\makeatletter
\patchcmd\longtable{\par}{\if@noskipsec\mbox{}\fi\par}{}{}
\makeatother
% Allow footnotes in longtable head/foot
\IfFileExists{footnotehyper.sty}{\usepackage{footnotehyper}}{\usepackage{footnote}}
\makesavenoteenv{longtable}
\usepackage{graphicx}
\makeatletter
\newsavebox\pandoc@box
\newcommand*\pandocbounded[1]{% scales image to fit in text height/width
  \sbox\pandoc@box{#1}%
  \Gscale@div\@tempa{\textheight}{\dimexpr\ht\pandoc@box+\dp\pandoc@box\relax}%
  \Gscale@div\@tempb{\linewidth}{\wd\pandoc@box}%
  \ifdim\@tempb\p@<\@tempa\p@\let\@tempa\@tempb\fi% select the smaller of both
  \ifdim\@tempa\p@<\p@\scalebox{\@tempa}{\usebox\pandoc@box}%
  \else\usebox{\pandoc@box}%
  \fi%
}
% Set default figure placement to htbp
\def\fps@figure{htbp}
\makeatother


% definitions for citeproc citations
\NewDocumentCommand\citeproctext{}{}
\NewDocumentCommand\citeproc{mm}{%
  \begingroup\def\citeproctext{#2}\cite{#1}\endgroup}
\makeatletter
 % allow citations to break across lines
 \let\@cite@ofmt\@firstofone
 % avoid brackets around text for \cite:
 \def\@biblabel#1{}
 \def\@cite#1#2{{#1\if@tempswa , #2\fi}}
\makeatother
\newlength{\cslhangindent}
\setlength{\cslhangindent}{1.5em}
\newlength{\csllabelwidth}
\setlength{\csllabelwidth}{3em}
\newenvironment{CSLReferences}[2] % #1 hanging-indent, #2 entry-spacing
 {\begin{list}{}{%
  \setlength{\itemindent}{0pt}
  \setlength{\leftmargin}{0pt}
  \setlength{\parsep}{0pt}
  % turn on hanging indent if param 1 is 1
  \ifodd #1
   \setlength{\leftmargin}{\cslhangindent}
   \setlength{\itemindent}{-1\cslhangindent}
  \fi
  % set entry spacing
  \setlength{\itemsep}{#2\baselineskip}}}
 {\end{list}}
\usepackage{calc}
\newcommand{\CSLBlock}[1]{\hfill\break\parbox[t]{\linewidth}{\strut\ignorespaces#1\strut}}
\newcommand{\CSLLeftMargin}[1]{\parbox[t]{\csllabelwidth}{\strut#1\strut}}
\newcommand{\CSLRightInline}[1]{\parbox[t]{\linewidth - \csllabelwidth}{\strut#1\strut}}
\newcommand{\CSLIndent}[1]{\hspace{\cslhangindent}#1}



\setlength{\emergencystretch}{3em} % prevent overfull lines

\providecommand{\tightlist}{%
  \setlength{\itemsep}{0pt}\setlength{\parskip}{0pt}}



 


\setlength\heavyrulewidth{0ex}
\setlength\lightrulewidth{0ex}
\usepackage[automark]{scrlayer-scrpage}
\clearpairofpagestyles
\cehead{
  Brian Weatherson
  }
\cohead{
  Infinity and Intransitivity
  }
\ohead{\bfseries \pagemark}
\cfoot{}
\makeatletter
\newcommand*\NoIndentAfterEnv[1]{%
  \AfterEndEnvironment{#1}{\par\@afterindentfalse\@afterheading}}
\makeatother
\NoIndentAfterEnv{itemize}
\NoIndentAfterEnv{enumerate}
\NoIndentAfterEnv{description}
\NoIndentAfterEnv{quote}
\NoIndentAfterEnv{equation}
\NoIndentAfterEnv{longtable}
\NoIndentAfterEnv{abstract}
\renewenvironment{abstract}
 {\vspace{-1.25cm}
 \quotation\small\noindent\emph{Abstract}:}
 {\endquotation}
\newfontfamily\tfont{EB Garamond}
\addtokomafont{disposition}{\rmfamily}
\addtokomafont{title}{\normalfont\itshape}
\let\footnoterule\relax
\setlength\heavyrulewidth{0ex}
\setlength\lightrulewidth{0ex}
\usepackage[automark]{scrlayer-scrpage}
\clearpairofpagestyles
\cehead{
  Anon
  }
\cohead{
  Infinity and Intransitivity
  }
\ohead{\bfseries \pagemark}
\cfoot{}
\author{Anon}
\renewenvironment{abstract}
 {\vspace{1.25cm}
 \quotation\small\noindent\emph{Abstract}:}
 {\endquotation}
\usepackage{xcolor}
\newcommand{\textred}[1]{\textcolor{red}{#1}}
\newcommand{\textblue}[1]{\textcolor{blue}{#1}}
\KOMAoption{captions}{tableheading}
\makeatletter
\@ifpackageloaded{caption}{}{\usepackage{caption}}
\AtBeginDocument{%
\ifdefined\contentsname
  \renewcommand*\contentsname{Table of contents}
\else
  \newcommand\contentsname{Table of contents}
\fi
\ifdefined\listfigurename
  \renewcommand*\listfigurename{List of Figures}
\else
  \newcommand\listfigurename{List of Figures}
\fi
\ifdefined\listtablename
  \renewcommand*\listtablename{List of Tables}
\else
  \newcommand\listtablename{List of Tables}
\fi
\ifdefined\figurename
  \renewcommand*\figurename{Figure}
\else
  \newcommand\figurename{Figure}
\fi
\ifdefined\tablename
  \renewcommand*\tablename{Table}
\else
  \newcommand\tablename{Table}
\fi
}
\@ifpackageloaded{float}{}{\usepackage{float}}
\floatstyle{ruled}
\@ifundefined{c@chapter}{\newfloat{codelisting}{h}{lop}}{\newfloat{codelisting}{h}{lop}[chapter]}
\floatname{codelisting}{Listing}
\newcommand*\listoflistings{\listof{codelisting}{List of Listings}}
\makeatother
\makeatletter
\makeatother
\makeatletter
\@ifpackageloaded{caption}{}{\usepackage{caption}}
\@ifpackageloaded{subcaption}{}{\usepackage{subcaption}}
\makeatother
\usepackage{bookmark}
\IfFileExists{xurl.sty}{\usepackage{xurl}}{} % add URL line breaks if available
\urlstyle{same}
\hypersetup{
  pdftitle={Infinity and Intransitivity},
  pdfauthor={Anon},
  hidelinks,
  pdfcreator={LaTeX via pandoc}}


\title{Infinity and Intransitivity}
\author{Anon}
\date{2025}
\begin{document}
\maketitle
\begin{abstract}
Jacob Nebel has an example that he uses to argue that a kind of
impartiality is impossible when the world population is infinite. I
think it shows something else, namely that infinite worlds give us
another reason to reject the transitivity of indifference.
\end{abstract}


\setstretch{1.1}
The most famous result in social choice theory, the impossibility
theorem due to Kenneth Arrow (\citeproc{ref-Arrow1951}{1951}), shows
that if some structural conditions are met, any function for aggregating
social preferences must treat people asymmetrically. In Arrow's case,
the asymmetry is very strong; one person must be a dictator. In a
forthcoming paper in \emph{Noûs}, Jacob Nebel
(\citeproc{ref-Nebel2025}{2025}) makes an argument with the same kind of
structure. He shows that if certain structural conditions are met,
anyone trying to act benevolently in the interest of people in general
must treat people asymmetrically. In Nebel's case the result is not that
there must be anything like a dictator, just that there must be
asymmetric treatment.

The usual way to prove Arrow's theorem is to start with the plausible
structural assumptions, and show that they entail the existence of a
dictator. Nevertheless, it is not a particularly common inference from
Arrow's theorem that, since the structural conditions are clearly true,
we should have a dictator. Still, it is interesting to ask just which of
the structural conditions should fail.

Nebel does argue that the structural conditions are true, and hence we
should embrace asymmetry. I'm going to argue that would be a mistake,
the same kind of mistake someone would be making if they accepted the
structural conditions in Arrow's proof. As with Arrow's theorem, we
should take the proof as a \emph{reductio} of the assumptions.

The assumption I think fails in Nebel's case is that indifference is
transitive. Amartya Sen (\citeproc{ref-Sen1970sec}{{[}1970{]} 2017})
noted that this is also a load-bearing assumption in Arrow's theorem,
and explored what would a theory that rejected it would look like. It
turns out in that case rejecting the transitivity of indifference alone
is not enough to get out of an intolerable result
(\citeproc{ref-Gibbard2014}{Gibbard 2014}). Something similar will be
true here; once transitivity of indifference goes some other principles
will have to go too. But the idea that Sen had in response to Arrow's
result is, I'll argue, the right response to Nebel's result.

The point is not just to defend the possibility of impartiality from
Nebel's argument, important though that is. The central point is that
cases like this provide a new kind of reason to reject the transitivity
of indifference.

\section{Nebel's Paradox}\label{sec-nebel}

This section describes the argument that Nebel gives against the
possibility of impartial beneficence.\footnote{The main example in this
  argument draws on Goodsell (\citeproc{ref-Goodsell2021}{2021}).} We
have a world that includes a countable infinity of people,
\emph{a}\textsubscript{1}, \emph{a}\textsubscript{2}, \ldots,
\emph{b}\textsubscript{1}, \emph{b}\textsubscript{2}, \ldots.\footnote{Actually
  all that matters is that these people are in the world. There could be
  other people who are not affected by any actions in what follows. I'll
  come back to this point.} Each of these people has one of two possible
welfare states, which we'll denote as 0 and 1.\footnote{So while we will
  be interested in paradoxes involving unbounded good, none of these
  paradoxes are relevant to the maximisation of expected welfare for any
  individual.} We'll say an outcome is any assignment of a probability
distribution of welfare payouts to each individual.\footnote{This means
  that we're not going to consider what happens if the world population
  changes; that raises a whole host of different concerns.}

One of four lotteries will take place. The lotteries are shown in
Table~\ref{tbl-nebel}. In each lottery, there are two random devices
that contribute to the determination of the state``. The first is a fair
{red} coin which will be flipped once. The second is a fair coin that
will be flipped indefinitely until it lands heads. We'll write {H} and
{T} for the {red} coin landing heads and tails respectively, and
H\textsubscript{\emph{i}} for the proposition that the other coin will
be flipped \emph{i} times. (For completeness, say that
H\textsubscript{1} also includes the possibility that it lands tails
every time.) So we have the states shown in the top row of
Table~\ref{tbl-nebel}, with their probability below them. In each cell,
we show the people who get payout 1 in that state from that lottery;
everyone else gets payout 0.

\begin{longtable}[]{@{}
  >{\raggedleft\arraybackslash}p{(\linewidth - 16\tabcolsep) * \real{0.1351}}
  >{\centering\arraybackslash}p{(\linewidth - 16\tabcolsep) * \real{0.1081}}
  >{\centering\arraybackslash}p{(\linewidth - 16\tabcolsep) * \real{0.1081}}
  >{\centering\arraybackslash}p{(\linewidth - 16\tabcolsep) * \real{0.1081}}
  >{\centering\arraybackslash}p{(\linewidth - 16\tabcolsep) * \real{0.1081}}
  >{\centering\arraybackslash}p{(\linewidth - 16\tabcolsep) * \real{0.1081}}
  >{\centering\arraybackslash}p{(\linewidth - 16\tabcolsep) * \real{0.1081}}
  >{\centering\arraybackslash}p{(\linewidth - 16\tabcolsep) * \real{0.1081}}
  >{\centering\arraybackslash}p{(\linewidth - 16\tabcolsep) * \real{0.1081}}@{}}
\caption{Nebel's example}\label{tbl-nebel}\tabularnewline
\toprule\noalign{}
\begin{minipage}[b]{\linewidth}\raggedleft
Lottery
\end{minipage} & \begin{minipage}[b]{\linewidth}\centering
{H}H\textsubscript{1}
\end{minipage} & \begin{minipage}[b]{\linewidth}\centering
{T}H\textsubscript{1}
\end{minipage} & \begin{minipage}[b]{\linewidth}\centering
{H}H\textsubscript{2}
\end{minipage} & \begin{minipage}[b]{\linewidth}\centering
{T}H\textsubscript{2}
\end{minipage} & \begin{minipage}[b]{\linewidth}\centering
\ldots{}
\end{minipage} & \begin{minipage}[b]{\linewidth}\centering
{H}H\textsubscript{\emph{k}}
\end{minipage} & \begin{minipage}[b]{\linewidth}\centering
{T}H\textsubscript{\emph{k}}
\end{minipage} & \begin{minipage}[b]{\linewidth}\centering
\ldots{}
\end{minipage} \\
\midrule\noalign{}
\endfirsthead
\toprule\noalign{}
\begin{minipage}[b]{\linewidth}\raggedleft
Lottery
\end{minipage} & \begin{minipage}[b]{\linewidth}\centering
{H}H\textsubscript{1}
\end{minipage} & \begin{minipage}[b]{\linewidth}\centering
{T}H\textsubscript{1}
\end{minipage} & \begin{minipage}[b]{\linewidth}\centering
{H}H\textsubscript{2}
\end{minipage} & \begin{minipage}[b]{\linewidth}\centering
{T}H\textsubscript{2}
\end{minipage} & \begin{minipage}[b]{\linewidth}\centering
\ldots{}
\end{minipage} & \begin{minipage}[b]{\linewidth}\centering
{H}H\textsubscript{\emph{k}}
\end{minipage} & \begin{minipage}[b]{\linewidth}\centering
{T}H\textsubscript{\emph{k}}
\end{minipage} & \begin{minipage}[b]{\linewidth}\centering
\ldots{}
\end{minipage} \\
\midrule\noalign{}
\endhead
\bottomrule\noalign{}
\endlastfoot
\textbf{Probability} & 1/4 & 1/4 & 1/8 & 1/8 & \ldots{} &
1/2\textsuperscript{\emph{k}+1} & 1/2\textsuperscript{\emph{k}+1} &
\ldots{} \\
\emph{P} & \emph{a}\textsubscript{1}, \emph{a}\textsubscript{2} &
\emph{a}\textsubscript{1}, \emph{a}\textsubscript{2} &
\emph{a}\textsubscript{1} \ldots{} \emph{a}\textsubscript{4} &
\emph{a}\textsubscript{1} \ldots{} \emph{a}\textsubscript{4} & \ldots{}
& \emph{a}\textsubscript{1} \ldots{}
\emph{a}\textsubscript{2\textsuperscript{\emph{k}}} &
\emph{a}\textsubscript{1} \ldots{}
\emph{a}\textsubscript{2\textsuperscript{\emph{k}}} & \ldots{} \\
\emph{Q} & \emph{a}\textsubscript{1}, \emph{a}\textsubscript{2} &
\emph{b}\textsubscript{1}, \emph{b}\textsubscript{2} &
\emph{a}\textsubscript{1} \ldots{} \emph{a}\textsubscript{4} &
\emph{b}\textsubscript{1} \ldots{} \emph{b}\textsubscript{4} & \ldots{}
& \emph{a}\textsubscript{1} \ldots{}
\emph{a}\textsubscript{2\textsuperscript{\emph{k}}} &
\emph{b}\textsubscript{1} \ldots{}
\emph{b}\textsubscript{2\textsuperscript{\emph{k}}} & \ldots{} \\
\emph{P}ʹ & \emph{a}\textsubscript{1}, \emph{a}\textsubscript{2} &
\emph{a}\textsubscript{1} \ldots{} \emph{a}\textsubscript{4} &
\emph{a}\textsubscript{1}, \emph{a}\textsubscript{2} &
\emph{a}\textsubscript{1} \ldots{} \emph{a}\textsubscript{8} & \ldots{}
& \emph{a}\textsubscript{1}, \emph{a}\textsubscript{2} &
\emph{a}\textsubscript{1} \ldots{}
\emph{a}\textsubscript{2\textsuperscript{\emph{k}+1}} & \ldots{} \\
\emph{Q}ʹ & \emph{a}\textsubscript{1}, \emph{b}\textsubscript{1} &
\emph{a}\textsubscript{1}, \emph{a}\textsubscript{2},
\emph{b}\textsubscript{1}, \emph{b}\textsubscript{2} &
\emph{a}\textsubscript{2}, \emph{b}\textsubscript{2} &
\emph{a}\textsubscript{1} \ldots{} \emph{a}\textsubscript{4},
\emph{b}\textsubscript{1} \ldots{} \emph{b}\textsubscript{4} & \ldots{}
& \emph{a}\textsubscript{\emph{k}-1}, \emph{b}\textsubscript{\emph{k}-1}
& \emph{a}\textsubscript{1} \ldots{}
\emph{a}\textsubscript{2\textsuperscript{\emph{k}}},
\emph{b}\textsubscript{1} \ldots{}
\emph{b}\textsubscript{2\textsuperscript{\emph{k}}} & \ldots{} \\
\end{longtable}

Now assume that that for each person \emph{x} we have relations of
strict welfare preference ≻\textsubscript{\emph{x}} and welfare
indifference \textasciitilde{}\textsubscript{\emph{x}}. We also have a
social welfare ordering, i.e., a pair of relations of strict preference
≻\textsubscript{∀} and indifference \textasciitilde{}\textsubscript{∀}.
In Nebel's discussion, these are the preference orderings of a
benevolent person. I think it's more intuitive to think of them as the
preference orderings of a good state, but not much turns on this. Either
way, they are a social ordering, in just the sense that Arrow was
interested in, and they should relate to the individual orderings in
some principled way.

We'll make the following assumptions:

\begin{itemize}
\tightlist
\item
  No matter the subscript, \textasciitilde{} is symmetric and ≻ is
  asymmetric;
\item
  No matter the subscript, ≻ is transitive;
\item
  For any individual \emph{x}, \textasciitilde{}\textsubscript{\emph{x}}
  is transitive;
\item
  No matter the subscript, for any options \emph{A} and \emph{B},
  \emph{A}~≻~\emph{B}, \emph{A}~\textasciitilde~\emph{B} and
  \emph{B}~≻~\emph{A} are exclusive and exhaustive.\footnote{The last
    assumption I'll denote below as \textbf{I-completeness}.}
\item
  Every individual prefers a higher probability of getting 1 to a lower
  probability, and is indifferent between any two states with the same
  probability of getting 1.
\end{itemize}

Nebel then derives a contradiction from the following principles.

\begin{description}
\tightlist
\item[Transitivity of Social Indifference]
If \emph{A}~\textasciitilde{}\textsubscript{∀}~\emph{B} and
\emph{B}~\textasciitilde{}\textsubscript{∀}~\emph{C} then
\emph{A}~\textasciitilde{}\textsubscript{∀}~\emph{C}.
\item[Pareto Indifference]
If \emph{A}~\textasciitilde{}\textsubscript{\emph{x}}~\emph{B} for all
\emph{x}, then \emph{A}~\textasciitilde{}\textsubscript{∀}~\emph{B}.
\item[Pareto Strict Preference]
If \emph{A}~≻\textsubscript{\emph{x}}~\emph{B} for all \emph{x}, then
\emph{A}~≻\textsubscript{∀}~\emph{B}.
\item[Pairwise Anonymity]
For outcome \emph{O} and individual \emph{x}, write
\emph{O\textsubscript{x}} for the outcome individual \emph{x} gets in
\emph{O}. For any outcomes \emph{A} and \emph{B} if there are two
individuals \emph{i} and \emph{j} such that
\emph{A\textsubscript{i}}~\textasciitilde{}\textsubscript{\emph{i}}
\emph{B}\textsubscript{j} and
\emph{A\textsubscript{j}}~\textasciitilde{}\textsubscript{\emph{j}}
\emph{B}\textsubscript{i}, and for all other individuals \emph{h},
\emph{A\textsubscript{h}}~\textasciitilde{}\textsubscript{\emph{h}}
\emph{B}\textsubscript{h}, then
\emph{A}~\textasciitilde{}\textsubscript{∀} \emph{B}.
\item[State Indifference]
For any two lotteries \emph{A} and \emph{B}, and any partition of the
state space, if in each state
\emph{A}~\textasciitilde{}\textsubscript{∀} \emph{B}, then prior to the
lottery taking place, \emph{A}~\textasciitilde{}\textsubscript{∀}
\emph{B}.
\end{description}

The derivation involves proving \emph{Q}ʹ~≻\textsubscript{∀} \emph{Q},
\emph{Q}~\textasciitilde{}\textsubscript{∀} \emph{P},
\emph{P}~\textasciitilde{}\textsubscript{∀} \emph{P}ʹ and
\emph{P}ʹ~\textasciitilde{}\textsubscript{∀} \emph{Q}ʹ.

First, to prove \emph{Q}ʹ~≻\textsubscript{∀} \emph{Q}. Note that each
individual has a higher probability of getting 1 in \emph{Q}ʹ than
\emph{Q}. So \emph{Q}ʹ~≻\textsubscript{\emph{x}}~\emph{Q} for all
\emph{x}, so by Pareto Strict Preference, \emph{Q}ʹ~≻\textsubscript{∀}
\emph{Q}.

Second, to prove \emph{Q}~\textasciitilde{}\textsubscript{∀} \emph{P}.
In each result of the lottery, the same number of people get 1. So in
each state, a finite number of applications of Pairwise Anonymity plus
one use of Pareto Indifference we have
\emph{Q}~\textasciitilde{}\textsubscript{∀} \emph{P} conditional on each
state of the lottery. So by Lottery Indifference,
\emph{Q}~\textasciitilde{}\textsubscript{∀} \emph{P}.

Third, to prove \emph{P}~\textasciitilde{}\textsubscript{∀} \emph{P}ʹ.
Each individual has the same probability of getting 1 in each lottery,
so each individual is indifferent between the lotteries, so by Pareto
Indifference, \emph{P}~\textasciitilde{}\textsubscript{∀} \emph{P}ʹ.

Fourth, to prove \emph{P}ʹ~\textasciitilde{}\textsubscript{∀} \emph{Q}ʹ.
Here it is the same reasoning as in step 2. No matter what lottery
outcome we're in, the same number of people get payout 1 in \emph{P}ʹ as
\emph{Q}ʹ, so by many (but in each state a finite number of)
applications of Pairwise Anonymity plus one use of Pareto Indifference
we have \emph{Q}~\textasciitilde{}\textsubscript{∀} \emph{P} conditional
on each state of the lottery. So by Lottery Indifference,
\emph{Q}~\textasciitilde{}\textsubscript{∀} \emph{P}.

Finally, by two applications of Transitivity of Indifference we get from
\emph{Q}~\textasciitilde{}\textsubscript{∀} \emph{P},
\emph{P}~\textasciitilde{}\textsubscript{∀} \emph{P}ʹ, and
\emph{P}ʹ~\textasciitilde{}\textsubscript{∀} \emph{Q}ʹ to
\emph{Q}~\textasciitilde{}\textsubscript{∀} \emph{Q}ʹ, contradicting
\emph{Q}ʹ~≻\textsubscript{∀} \emph{Q}.

Nebel argues that the culprit here is Pairwise Anonymity. He argues that
his means that a certain kind of impartial benevolence is impossible.
I'm going to argue that it's really the transitivity assumption we
should give up. The short version of my argument is going to be that (a)
there are situations where the first four steps of this proof seem just
like good reasoning, and (b) if we accept the transitivity of
indifference, then not only can we prove the social welfare ordering
does not treat everyone alike, we can prove that it has arbitrarily
large differences in how it treats people. The latter seems
unacceptable, particularly when we are thinking about how the state
should allocate resources and risks across a population.

\section{Background Ideology}\label{sec-ideology}

There are two ways of presenting welfare orderings that get used in the
literature, and they end up being helpful for different purposes.

One is to start with notions of \emph{better than} and \emph{exactly as
good as}, where the latter is assumed to be an equivalence relation.
This is I think the standard way to do things in contemporary
philosophy. But there is an older notation that I will use here, because
I think it turns out to be more useful.

Following Sen (\citeproc{ref-Sen1970sec}{{[}1970{]} 2017}), I'll write
\emph{xRy} for the claim that \emph{x} is at least as good as \emph{y}.
More precisely, I'll understand it as the claim that \emph{x} is
choiceworthy when it and \emph{y} are the available options. Using
\emph{R} we can define relations of strict indifference, \emph{P}, and
indifference, \emph{I}, as the asymmetric and symmetric parts of
\emph{R}. We could also, if we like, define a relation of equality
\emph{E}, where \emph{xEy} means that \emph{x} and \emph{y} stand in the
same \emph{R}-relations with respect to some space of
options.\footnote{Formally, \emph{xEy} is true iff \emph{xRx} and for
  all \emph{z}, \emph{xRz}~\leftrightarrow *yRz\emph{, and
  }zRx\emph{~\leftrightarrow *zRy}.} So defined, \emph{E} is clearly an
equivalence relation. But while \emph{I} is by definition symmetric, it
isn't obvious it is transitive.\footnote{It also isn't obvious it is
  reflexive. I'll come back to this presently. Rejecting the
  transitivity of \emph{I} does not give us a reason to reject the
  transitivity of \emph{P}, as advocated by, e.g., Temkin
  (\citeproc{ref-Temkin1987}{1987}).}

The different notations naturally suggest slightly different
understandings of what it is for a preference relation to be
\emph{complete}. If we start with a strict better than ≻, and a relation
of exactly as good as that's assumed to be an equivalent relation, it's
natural to write a relation of being preferred or equal as ≽. And then
it's natural to understand completeness as what I'll call
\textbf{E-completeness}.

\begin{description}
\tightlist
\item[E-completeness]
\emph{x}~≽~\emph{y}~∨~\emph{y}~≽~\emph{x}
\end{description}

In Nebel's paper he talks about how dropping completeness might be a way
out of the problem, and by this he means, as I think most philosophers
would these days, E-completeness. If we start with \emph{R} as our basic
notion, it's more natural to define completeness as what I'll call
\textbf{I-completeness}.

\begin{description}
\tightlist
\item[I-completeness]
\emph{xRy}~∨~\emph{yRx}
\end{description}

Both E-completeness and I-completeness can be stated as trichotomy
theses.\footnote{The language of trichotomy is taken from Chang
  (\citeproc{ref-Chang2002}{2002}).} E-completeness is equivalent to
\emph{xPy}~∨~\emph{xEy}~∨~\emph{yPx}, while I-completeness is equivalent
to \emph{xPy}~∨~\emph{xIy}~∨~\emph{yPx}.

Assuming \emph{P} is transitive, E-completeness is entailed by
I-completeness and the transitivity of \emph{I}. A reason to prefer
starting with notions like \emph{R} and \emph{I} is that the last line
suggests there are two quite distinct ways for E-completeness to fail.
One is that I-completeness fails, and the other is that \emph{I} is
intransitive. And it might be useful to readily distinguish these two
failure cases.

It is very common to assume \emph{I} is complete, at least as an
idealising assumption. There are really two prominent ways for \emph{I}
to fail to be complete. One is that the thinker simply has no attitude
about (at least one of) \emph{x} and \emph{y}, perhaps because they
don't have the capacity to think about them. Think about a subject who
has never heard of Porepunkah or Tidbinbilla, and ask what preferences
they have over a holiday in one or the other.\footnote{I don't mean ask
  them; that might cause them to form attitudes. I mean ask whether they
  prefer one or the other before hearing about either. The reason I
  included \emph{xRx} in the formal definition of \emph{E} is so we
  wouldn't say \emph{E} holds in cases where the agent has no attitude
  towards either relata.} It's natural to say that any positive claim
about what preferences they have is false, so I-completeness fails.
I-completeness might also fail if there are prudential dilemmas, where
neither choice is acceptable.

The cases where I-completeness fails are very different from the cases
where \emph{I} appears to be intransitive. In the next section we'll
look at some cases from the literature that have been argued to be
counterexamples to transitivity.

\section{Why Intransitivity?}\label{sec-intransitivity}

One reason to reject intransitivity, first discussed by Wallace
Armstrong (\citeproc{ref-Armstrong1939}{1939}), starts with the
assumption that perceptual discriminability is intransitive. There may
be a Sorites series of cups of coffee with sugar where the subject
cannot tell any two adjacent cups apart, and so is indifferent between
any pair, but the last one is too sweet for the chooser, so they
strictly prefer the first cup to the last cup. This looks like a good
argument against the transitivity of indifference iff perceptual
indiscriminability really is intransitive. If it is transitive, as Delia
Graff Fara (\citeproc{ref-Fara2001}{2001}) argues, then it is less
compelling.

A second reason traces back to an example by Gerard Debreu
(\citeproc{ref-Debreu1960}{1960}). Let \emph{x} and \emph{y} be very
different options that the chooser is indifferent between, in the sense
that they could happily choose either. Now let
\emph{x}\textsuperscript{+} be something that strictly improves
\emph{x}, e.g., it is \emph{x} plus a dollar.\footnote{Debreu describes
  an example like this but not this argument. I haven't been able to
  find anything I'd happily say is the first instance of this argument.}
Plausibly we have \emph{xIy} and \emph{yIx}\textsuperscript{+}, but we
definitely do not have \emph{xIx}\textsuperscript{+}. This is the kind
of example that motivated Sen (\citeproc{ref-Sen1970sec}{{[}1970{]}
2017}) to look for what happens to Arrow's impossibility theorem when we
drop the assumption that indifference in the social ordering is
transitive.

A third reason involves non-numerical probability. There is a long line
of arguments from Keynes (\citeproc{ref-Keynes1921}{1921}) to
(\citeproc{ref-HajekHawthorneEtc}{\textbf{HajekHawthorneEtc?}}) arguing
that some propositions do not have numerical probabilities. Let \emph{p}
be such a proposition, \emph{x} a bet that pays 1 if \emph{p} and 0
otherwise, and \emph{y} a constant return of some value that the
probability of \emph{p} is neither strictly greater than nor strictly
less than. If we model non-numerical probability by using sets of
probability functions, and say that \emph{xRy} iff \emph{x} has as high
an expected return as \emph{y} according to at least one of the
functions in that set, then \emph{I} will be intransitive.

One important strand in Sen's work on social choice functions was to
investigate whether dropping the assumption of transitivity of
indifference in the social ordering provided a natural way out of
Arrow's impossibility result. My impression is that most theorists think
this part of his project was unsuccessful. Once Gibbard
(\citeproc{ref-Gibbard2014}{2014}) proved (in a note first circulated in
1969) that dropping this assumption just meant that oligarchic as well
as dictatorial decision rules were compatible with the other
assumptions, this was not seen as a particularly promising path forward.

The argument in this paper will not involve Arrow's theorem, but it will
follow Sen's work in one respect. I'm going to be looking at, indeed
endorsing, arguments that even if indifference is transitive for
individual preferences and welfare, it is not transitive in the social
ordering.

\section{Nebel's Paradox Revisited}\label{sec-nebel-revisited}

In this section I'll set out one intuitive argument for thinking that
the culprit in Nebel's paradox, the false assumption, is the
transitivity of indifference. Imagine that the world contains a
countable infinity of people, whose welfare levels are 0 or 1, as in
Nebel's original case. In that world B is a benevolent person, who wants
what is best for others. B is approached one day by an angel who
convinces B that she has the ability to carry out lotteries like
\emph{P} and \emph{Q}, and can be trusted to make the payouts.

The angel tells B that she has identified infinitely many people B is
not acquainted and who will, if left unattended, have welfare level 0.
(I assume B is only acquainted with, and has the capacity for singular
thought about, finitely many people, so there will be infinitely many he
has no singular thoughts about.) The angel has divided these people into
six classes. Within each of those six classes, the angel has numbered
each of the people \emph{a}\textsubscript{1}, \emph{a}\textsubscript{2},
\ldots, \emph{b}\textsubscript{1}, \emph{b}\textsubscript{2}, \ldots.
Each of the classes is associated with a pair of lotteries from
\emph{P}, \emph{Q}, \emph{P}ʹ and \emph{Q}ʹ. For each such pair, the
angel will ask B which lottery he wants carried out, and will ensure
that the people the angel will ask B which lottery should be carried
out, and then after flipping the {red} coin once and the black coin
sufficiently often, increase the welfare of the winners.

I don't have good intuitions about what B should do when presented with
the pairs (\emph{P}, \emph{Q}ʹ), and (\emph{Q}, \emph{P}ʹ). But the
arguments from Section~\ref{sec-nebel} that B should be indifferent when
presented with (\emph{P}, \emph{Q}), (\emph{P}, \emph{P}ʹ), and
(\emph{P}ʹ, \emph{Q}ʹ) seem compelling, as does the argument that B
should definitely choose \emph{Q}ʹ rather than \emph{Q}. If one is at
all antecedently sympathetic to the arguments from Keynes, Armstrong,
Debreu, Sen, and so on that indifference is sometimes intransitive, it
is very natural to think this is just another such case. The alternative
is that B should, with just the information the angel has presented,
form a sufficiently asymmetric attitude towards the people who happen to
have some labels rather than another than the arguments for indifference
are blocked. That doesn't seem particularly plausible to me.

While I think the intuitions here are reasonably good, as intuitions go,
it would be good to have something more solid to rely on. That's what
I'll try to do over the next two sections.

\section{Intransitivity and Principles}\label{sec-iandt}

If we accept that indifference can be intransitive, a lot of otherwise
plausible sounding principles must be rejected, or at least qualified.
Start with the principle, which Nebel uses of Pareto Indifference.

Pareto Indiffernce says that if everyone is indifferent between what
they receive in \emph{w}\textsubscript{1} and \emph{w}\textsubscript{2},
then \emph{w}\textsubscript{1} is just as good as
\emph{w}\textsubscript{2}. Now consider four possible outcomes,
\emph{c}, \emph{c}\textsuperscript{+}, \emph{d}, and
\emph{d}\textsuperscript{+}, with \emph{c}\textsuperscript{+}\emph{Pc},
\emph{d}\textsuperscript{+}\emph{Pd}, and indifference between the other
four options. The mere possibility of intransitivity doesn't entail that
such a quadruple exists. But it seems plausible that it should if
Debreu-style small improvement cases are possible. For example, let
\emph{c} be the Paris holiday, \emph{d} be the Rome holiday, and the
\textsuperscript{+} variants better by a dollar. Now consider the two
worlds described in Table~\ref{tbl-pareto-indifference}. Everyone except
\emph{a}\textsubscript{1} and \emph{a}\textsubscript{2} gets the same
outcome in \emph{w}\textsubscript{1} and \emph{w}\textsubscript{2}, and
the social welfare function regards \emph{a}\textsubscript{1} and
\emph{a}\textsubscript{2} as equally important.\footnote{Even we
  accepted that it was impossible for the social welfare function to
  treat everyone equally, it should be possible to find two people who
  are treated equally.}

\begin{longtable}[]{@{}cccc@{}}
\caption{A counterexample to Pareto
Indifference}\label{tbl-pareto-indifference}\tabularnewline
\toprule\noalign{}
& \emph{w}\textsubscript{1} & \emph{w}\textsubscript{2} &
\emph{w}\textsubscript{3} \\
\midrule\noalign{}
\endfirsthead
\toprule\noalign{}
& \emph{w}\textsubscript{1} & \emph{w}\textsubscript{2} &
\emph{w}\textsubscript{3} \\
\midrule\noalign{}
\endhead
\bottomrule\noalign{}
\endlastfoot
\emph{a}\textsubscript{1} & \emph{c} & \emph{d}\textsuperscript{+} &
\emph{c}\textsuperscript{+} \\
\emph{a}\textsubscript{2} & \emph{d} & \emph{c}\textsuperscript{+} &
\emph{d}\textsuperscript{+} \\
\end{longtable}

Since \emph{w}\textsubscript{2} is exactly as good as
\emph{w}\textsubscript{3}, and \emph{w}\textsubscript{1} is worse than
\emph{w}\textsubscript{3}, it's plausible that \emph{w}\textsubscript{1}
is worse than \emph{w}\textsubscript{2}. Also, the net outcome of
\emph{w}\textsubscript{2} is the same as the net outcome of
\emph{w}\textsubscript{1}, holidays in Paris and Rome for two citizens,
plus two dollars. But every individual is indifferent between what they
have in \emph{w}\textsubscript{2} as is in \emph{w}\textsubscript{1}. So
we have a counterexample to Pareto Indifference.

A similar example shows that State Indifference fails.\footnote{This
  example is similar to the `sweetening' example described by Casper
  (\citeproc{ref-Hare2010}{\textbf{Hare2010?}}).} Now just consider
\emph{a}\textsubscript{1}, and imagine that they have to choose between
lotteries \emph{L} and \emph{L}\textsuperscript{+}, as shown in
Table~\ref{tbl-state-indifference}. The outcome of the lotteries depends
on whether the {[}red{]}\{text-red\} coin from earlier lands
{[}Heads{]}\{text-red\} or {[}Tails{]}\{text-red\}

\begin{longtable}[]{@{}ccc@{}}
\caption{A counterexample to State
Indifference}\label{tbl-state-indifference}\tabularnewline
\toprule\noalign{}
& {[}H{]}\{text-red\} & {[}T{]}\{text-red\} \\
\midrule\noalign{}
\endfirsthead
\toprule\noalign{}
& {[}H{]}\{text-red\} & {[}T{]}\{text-red\} \\
\midrule\noalign{}
\endhead
\bottomrule\noalign{}
\endlastfoot
\emph{L} & \emph{c} & \emph{d} \\
\emph{L}\textsuperscript{+} & \emph{d}\textsuperscript{+} &
\emph{c}\textsuperscript{+} \\
\emph{L}-alt & \emph{c}\textsuperscript{+} &
\emph{d}\textsuperscript{+} \\
\end{longtable}

However the coin lands, \emph{a}\textsubscript{1} is indifferent between
the outcomes of \emph{L} and \emph{L}\textsuperscript{+}. But since
\emph{a}\textsubscript{1} strictly prefers \emph{L}-alt to \emph{L} in
every state, and \emph{L}\textsuperscript{+} is stochastically
equivalent to \emph{L}-alt, it seems plausible that they should prefer
\emph{L}\textsuperscript{+} to \emph{L}. This violates State
Indifference.

\subsection*{References}\label{references}
\addcontentsline{toc}{subsection}{References}

\phantomsection\label{refs}
\begin{CSLReferences}{1}{0}
\bibitem[\citeproctext]{ref-Armstrong1939}
Armstrong, W. E. 1939. {``The Determinateness of the Utility
Function.''} \emph{The Economic Journal} 49 (195): 453--67. doi:
\href{https://doi.org/10.2307/2224802}{10.2307/2224802}.

\bibitem[\citeproctext]{ref-Arrow1951}
Arrow, Kenneth J. 1951. \emph{Social Choice and Individual Values}. New
York: John Wiley \& Sons.

\bibitem[\citeproctext]{ref-Chang2002}
Chang, Ruth. 2002. {``The Possibility of Parity.''} \emph{Ethics} 112
(4): 659--88. doi:
\href{https://doi.org/10.1086/339673}{10.1086/339673}.

\bibitem[\citeproctext]{ref-Debreu1960}
Debreu, Gerard. 1960. {``Review of \emph{Individual Choice Behavior: A
Theoretical Analysis}, by {R. Duncan Luce}.''} \emph{American Economic
Review} 50 (1): 186--88.

\bibitem[\citeproctext]{ref-Fara2001}
Fara, Delia Graff. 2001. {``Phenomenal Continua and the Sorites.''}
\emph{Mind} 110 (440): 905--36. doi:
\href{https://doi.org/10.1093/mind/110.440.905}{10.1093/mind/110.440.905}.
This paper was first published under the name {``Delia Graff.''}

\bibitem[\citeproctext]{ref-Gibbard2014}
Gibbard, Allan F. 2014. {``Social Choice and the Arrow Conditions.''}
\emph{Economics and Philosophy} 30 (3): 269--84. doi:
\href{https://doi.org/10.1017/S026626711400025X}{10.1017/S026626711400025X}.

\bibitem[\citeproctext]{ref-Goodsell2021}
Goodsell, Zachary. 2021. {``A St Petersburg Paradox for Risky Welfare
Aggregation.''} \emph{Analysis} 81 (3): 420--26. doi:
\href{https://doi.org/10.1093/analys/anaa079}{10.1093/analys/anaa079}.

\bibitem[\citeproctext]{ref-Keynes1921}
Keynes, John Maynard. 1921. \emph{Treatise on Probability}. London:
Macmillan.

\bibitem[\citeproctext]{ref-Nebel2025}
Nebel, Jacob M. 2025. {``Infinite Ethics and the Limits of
Impartiality.''} No{û}s. 2025. doi:
\href{https://doi.org/10.1111/nous.70010}{10.1111/nous.70010}.

\bibitem[\citeproctext]{ref-Sen1970sec}
Sen, Amartya. (1970) 2017. \emph{Collective Choice and Social Welfare:}
An expanded edition. Cambridge, MA: Harvard University Press. doi:
\href{https://doi.org/10.4159/9780674974616}{10.4159/9780674974616}.

\bibitem[\citeproctext]{ref-Temkin1987}
Temkin, Larry S. 1987. {``Intransitivity and the Mere Addition
Paradox.''} \emph{Philosophy \& Public Affairs} 16 (2): 138--87.

\end{CSLReferences}



Draft of September 2025.


\end{document}
