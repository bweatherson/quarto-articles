% Options for packages loaded elsewhere
\PassOptionsToPackage{unicode}{hyperref}
\PassOptionsToPackage{hyphens}{url}
\PassOptionsToPackage{dvipsnames,svgnames,x11names}{xcolor}
%
\documentclass[
  11pt,
  letterpaper,
  DIV=11,
  numbers=noendperiod]{scrartcl}

\usepackage{amsmath,amssymb}
\usepackage{iftex}
\ifPDFTeX
  \usepackage[T1]{fontenc}
  \usepackage[utf8]{inputenc}
  \usepackage{textcomp} % provide euro and other symbols
\else % if luatex or xetex
  \ifXeTeX
    \usepackage{mathspec} % this also loads fontspec
  \else
    \usepackage{unicode-math} % this also loads fontspec
  \fi
  \defaultfontfeatures{Scale=MatchLowercase}
  \defaultfontfeatures[\rmfamily]{Ligatures=TeX,Scale=1}
\fi
\usepackage{lmodern}
\ifPDFTeX\else  
    % xetex/luatex font selection
  \setmainfont[Scale = MatchLowercase]{Scala Pro}
  \setsansfont[]{Scala Sans Pro}
  \ifXeTeX
    \setmathfont(Digits,Latin,Greek)[]{Scala Pro}
  \else
    \setmathfont[]{Scala Pro}
  \fi
\fi
% Use upquote if available, for straight quotes in verbatim environments
\IfFileExists{upquote.sty}{\usepackage{upquote}}{}
\IfFileExists{microtype.sty}{% use microtype if available
  \usepackage[]{microtype}
  \UseMicrotypeSet[protrusion]{basicmath} % disable protrusion for tt fonts
}{}
\makeatletter
\@ifundefined{KOMAClassName}{% if non-KOMA class
  \IfFileExists{parskip.sty}{%
    \usepackage{parskip}
  }{% else
    \setlength{\parindent}{0pt}
    \setlength{\parskip}{6pt plus 2pt minus 1pt}}
}{% if KOMA class
  \KOMAoptions{parskip=half}}
\makeatother
\usepackage{xcolor}
\setlength{\emergencystretch}{3em} % prevent overfull lines
\setcounter{secnumdepth}{3}
% Make \paragraph and \subparagraph free-standing
\ifx\paragraph\undefined\else
  \let\oldparagraph\paragraph
  \renewcommand{\paragraph}[1]{\oldparagraph{#1}\mbox{}}
\fi
\ifx\subparagraph\undefined\else
  \let\oldsubparagraph\subparagraph
  \renewcommand{\subparagraph}[1]{\oldsubparagraph{#1}\mbox{}}
\fi


\providecommand{\tightlist}{%
  \setlength{\itemsep}{0pt}\setlength{\parskip}{0pt}}\usepackage{longtable,booktabs,array}
\usepackage{calc} % for calculating minipage widths
% Correct order of tables after \paragraph or \subparagraph
\usepackage{etoolbox}
\makeatletter
\patchcmd\longtable{\par}{\if@noskipsec\mbox{}\fi\par}{}{}
\makeatother
% Allow footnotes in longtable head/foot
\IfFileExists{footnotehyper.sty}{\usepackage{footnotehyper}}{\usepackage{footnote}}
\makesavenoteenv{longtable}
\usepackage{graphicx}
\makeatletter
\def\maxwidth{\ifdim\Gin@nat@width>\linewidth\linewidth\else\Gin@nat@width\fi}
\def\maxheight{\ifdim\Gin@nat@height>\textheight\textheight\else\Gin@nat@height\fi}
\makeatother
% Scale images if necessary, so that they will not overflow the page
% margins by default, and it is still possible to overwrite the defaults
% using explicit options in \includegraphics[width, height, ...]{}
\setkeys{Gin}{width=\maxwidth,height=\maxheight,keepaspectratio}
% Set default figure placement to htbp
\makeatletter
\def\fps@figure{htbp}
\makeatother
% definitions for citeproc citations
\NewDocumentCommand\citeproctext{}{}
\NewDocumentCommand\citeproc{mm}{%
  \begingroup\def\citeproctext{#2}\cite{#1}\endgroup}
\makeatletter
 % allow citations to break across lines
 \let\@cite@ofmt\@firstofone
 % avoid brackets around text for \cite:
 \def\@biblabel#1{}
 \def\@cite#1#2{{#1\if@tempswa , #2\fi}}
\makeatother
\newlength{\cslhangindent}
\setlength{\cslhangindent}{1.5em}
\newlength{\csllabelwidth}
\setlength{\csllabelwidth}{3em}
\newenvironment{CSLReferences}[2] % #1 hanging-indent, #2 entry-spacing
 {\begin{list}{}{%
  \setlength{\itemindent}{0pt}
  \setlength{\leftmargin}{0pt}
  \setlength{\parsep}{0pt}
  % turn on hanging indent if param 1 is 1
  \ifodd #1
   \setlength{\leftmargin}{\cslhangindent}
   \setlength{\itemindent}{-1\cslhangindent}
  \fi
  % set entry spacing
  \setlength{\itemsep}{#2\baselineskip}}}
 {\end{list}}
\usepackage{calc}
\newcommand{\CSLBlock}[1]{\hfill\break#1\hfill\break}
\newcommand{\CSLLeftMargin}[1]{\parbox[t]{\csllabelwidth}{\strut#1\strut}}
\newcommand{\CSLRightInline}[1]{\parbox[t]{\linewidth - \csllabelwidth}{\strut#1\strut}}
\newcommand{\CSLIndent}[1]{\hspace{\cslhangindent}#1}

\setlength\heavyrulewidth{0ex}
\setlength\lightrulewidth{0ex}
\makeatletter
\def\@maketitle{%
\newpage
\null
\vskip 2em%
\begin{center}%
\let \footnote \thanks
  {\LARGE \@title \par}%
  \vskip 1.5em%
  {\large
    \lineskip .5em%
    \begin{tabular}[t]{c}%
      \@author
    \end{tabular}\par}%
  %\vskip 1em%
  %{\large \@date}%
\end{center}%
\par
\vskip 1.5em}
\makeatother 
\KOMAoption{captions}{tableheading}
\makeatletter
\@ifpackageloaded{caption}{}{\usepackage{caption}}
\AtBeginDocument{%
\ifdefined\contentsname
  \renewcommand*\contentsname{Table of contents}
\else
  \newcommand\contentsname{Table of contents}
\fi
\ifdefined\listfigurename
  \renewcommand*\listfigurename{List of Figures}
\else
  \newcommand\listfigurename{List of Figures}
\fi
\ifdefined\listtablename
  \renewcommand*\listtablename{List of Tables}
\else
  \newcommand\listtablename{List of Tables}
\fi
\ifdefined\figurename
  \renewcommand*\figurename{Figure}
\else
  \newcommand\figurename{Figure}
\fi
\ifdefined\tablename
  \renewcommand*\tablename{Table}
\else
  \newcommand\tablename{Table}
\fi
}
\@ifpackageloaded{float}{}{\usepackage{float}}
\floatstyle{ruled}
\@ifundefined{c@chapter}{\newfloat{codelisting}{h}{lop}}{\newfloat{codelisting}{h}{lop}[chapter]}
\floatname{codelisting}{Listing}
\newcommand*\listoflistings{\listof{codelisting}{List of Listings}}
\makeatother
\makeatletter
\makeatother
\makeatletter
\@ifpackageloaded{caption}{}{\usepackage{caption}}
\@ifpackageloaded{subcaption}{}{\usepackage{subcaption}}
\makeatother
\ifLuaTeX
  \usepackage{selnolig}  % disable illegal ligatures
\fi
\IfFileExists{bookmark.sty}{\usepackage{bookmark}}{\usepackage{hyperref}}
\IfFileExists{xurl.sty}{\usepackage{xurl}}{} % add URL line breaks if available
\urlstyle{same} % disable monospaced font for URLs
\hypersetup{
  pdftitle={Assertion, Knowledge and Action},
  pdfauthor={Ishani Maitra; Brian Weatherson},
  colorlinks=true,
  linkcolor={black},
  filecolor={Maroon},
  citecolor={Blue},
  urlcolor={Blue},
  pdfcreator={LaTeX via pandoc}}

\title{Assertion, Knowledge and Action}
\author{Ishani Maitra \and Brian Weatherson}
\date{2010-03-23}

\begin{document}
\maketitle
It is widely believed that the mere truth of \emph{p} is insufficient
for \emph{p} to be properly assertable, even if \emph{p} is relevant to
current conversation. If a speaker simply guessed that \emph{p} is true,
then she shouldn't say \emph{p}, for example. There is some dissent from
this view (e.g., Weiner (\citeproc{ref-Weiner2005}{2005})), but it is
something close to orthodoxy in the current literature on assertion that
something further is needed. The most common `something else' is
\emph{knowledge}: a speaker shouldn't say \emph{p} unless they know
\emph{p}. This view is nowadays commonly associated with Timothy
Williamson (\citeproc{ref-Williamson1996-WILKAA}{1996},
\citeproc{ref-Williamson2000-WILKAI}{2000}), but it has historical
antecedents tracing back at least to Max Black's
(\citeproc{ref-Black1952}{1952}) paper ``Saying and
Disbelieving''.\footnote{Timothy Williamson
  (\citeproc{ref-Williamson2000-WILKAI}{Williamson 2000} Ch. 11) has the
  clearest statement of the view we're considering here. It is also
  defended by Keith DeRose (\citeproc{ref-DeRose2002}{2002}). Both
  DeRose and John Hawthorne (\citeproc{ref-Hawthorne2004}{2004}) deploy
  it extensively as a constraint on theories of knowledge. Jason Stanley
  (\citeproc{ref-Stanley2008-STAKAC}{2008}) argues for an even stronger
  constraint: that we should only assert \emph{p} if we are certain that
  \emph{p}. Igor Douven (\citeproc{ref-Douven2006}{2006}) argues that
  truth is neither sufficient nor necessary, so the norm should be
  assert only what is rationally credible. Kent Bach
  (\citeproc{ref-Bach2010}{2007}) and Frank Hindriks
  (\citeproc{ref-Hindriks2007}{2007}) both suggest that the only real
  norm governing \emph{assertion} is belief, but that since knowledge is
  a norm of belief, we shouldn't generally assert what we do not know.
  In this paper we're not concerned with the question of whether the
  rule \emph{Assert only what you know} holds solely in virtue of the
  normative nature of assertion itself, as Williamson thinks, or partly
  in virtue of norms applying to related states like belief, as Bach and
  Hindriks suggest, but rather whether the rule is even a good rule.}
Call Williamson's position \emph{The Knowledge Rule}.

\begin{description}
\item[The Knowledge Rule]
Assert that \emph{p} only if you know that \emph{p}.
\end{description}

This paper aims to raise trouble for The Knowledge Rule, and several
related positions, by focussing on a particular kind of assertion. We'll
be looking at assertions about what is to be done. The boldest statement
of our position is that if an agent should do \emph{X}, then that agent
is in a position to say that they should do \emph{X}. (We'll qualify
this a little below, but it's helpful to start with the bold position.)
We argue, following Williamson's `anti-luminosity' arguments, that its
being true that \emph{X} is the thing to do for an agent doesn't entail
that that agent knows it's the thing to do.\footnote{We'll use the
  expressions `thing to do' and `what to do' interchangeably throughout
  the paper. By \emph{X is what to do}, we mean \emph{X} ought to be
  done, all things considered. We take no position on whether \emph{X}'s
  being what to do entails its being the morally right thing to do. That
  may be the case, but nothing we say in this paper depends on its being
  so.} If both these claims are true, then there will be cases where it
is fine to assert that \emph{X} is what to do, even though the agent
doesn't know this. So, The Knowledge Rule is mistaken. Slightly more
formally, we'll be interested in arguments of this structure.

\begin{quote}
\emph{Master Argument (First Attempt)}

\begin{enumerate}
\def\labelenumi{\arabic{enumi}.}
\item
  If act \emph{X} is what to do for agent \emph{S}, then \emph{S} can
  properly assert that \emph{X} is what to do (assuming that this
  assertion is relevant to the current conversation).
\item
  It is possible that \emph{X} is what to do for \emph{S,} even though
  \emph{S} is not in a position to know this.
\item
  So, it is possible that \emph{S} can properly assert that \emph{X} is
  what to do even though she does not know, and is not even in a
  position to know, that \emph{X} is what to do.
\end{enumerate}
\end{quote}

In section 1, we'll motivate premise 1 with a couple of vignettes. In
section 2, we'll qualify that premise and make it more plausible. In
section 3, we'll motivate premise 2. In section 4, we'll look at one of
the positive arguments for The Knowledge Rule, the argument from Moore's
paradox, and conclude that it is of no help. In section 5, we'll look at
what could be put in place of The Knowledge Rule, and suggest two
alternatives.

\begin{description}
\item[The Evidence Responsiveness Rule]
Assert that \emph{p} only if your attitude towards \emph{p} is properly
responsive to the evidence you have that bears on \emph{p}.
\item[The Action Rule]
Assert that \emph{p} only if acting as if \emph{p} is true is the thing
for you to do.
\end{description}

We're not going to argue for these rules in detail; that would take a
much longer paper. Nor are we going to decide between them. What we are
going to suggest is that these rules have the virtues that are commonly
claimed for The Knowledge Rule, but lack The Knowledge Rule's
problematic consequences when it comes to assertions about what to do.

\section{Speaking about What to Do}\label{speaking-about-what-to-do}

We start by motivating premise 1 of the Master Argument with a couple of
examples. Both cases are direct counterexamples to The Knowledge Rule,
but we're interested in the first instance in what the cases have in
common. After presenting the vignettes, we offer three distinct
arguments to show that, in such cases, it \emph{is} proper for the
speakers to assert what they do assert, even though they don't know it
to be true.

\subsection*{Going to War}\label{going-to-war}
\addcontentsline{toc}{subsection}{Going to War}

Imagine that a country, Indalia, finds itself in a situation in which
the thing for it to do, given the evidence available to its leaders, is
to go to war against an enemy. (Those pacifists who think it is never
right to go to war won't like this example, but we think war can at
least sometimes be justified.) But it is a close call. Had the evidence
been a bit weaker, had the enemy been a little less murderous, or the
risk of excessive civilian casualties a little higher, it would have
been preferable to wait for more evidence, or use non-military measures
to persuade the enemy to change its ways. So, while going to war is the
thing to do, the leaders of Indalia can't know this. We'll come back to
this in section 2, but the crucial point here is that knowledge has a
safety constraint, and any putative knowledge here would violate this
constraint.

Our leaders are thus in a delicate position here. The Prime Minister of
Indalia decides to launch the war, and gives a speech in the House of
Commons setting out her reasons. All the things she says in the speech
are true, and up to her conclusion they are all things that she knows.
She concludes with (1).

Now (1) is also true, and the Prime Minister believes it, but it is not
something she knows. So, the Prime Minister violates The Knowledge Rule
when she asserts (1). But it seems to us that she doesn't violate any
norms in making this assertion. We'll have a lot more to say about why
this is so in a few paragraphs. But first, here's a less dramatic case
that is also a counterexample to The Knowledge Rule, one that involves
prudential judgments rather than moral judgments.

\subsection*{Buying Flood Insurance}\label{buying-flood-insurance}
\addcontentsline{toc}{subsection}{Buying Flood Insurance}

Raj and Nik are starting a small business. The business is near a river
that hasn't flooded in recent memory, but around which there isn't much
flood protection. They could buy flood insurance which would be useful
in a flood, naturally, but would be costly in the much more likely event
that there is not a flood. Raj has done the calculations of the
likelihood of a flood, the amount this would damage the business, the
utility loss of not having this damage insured, and the utility loss of
paying flood insurance premiums. He has concluded that buying flood
insurance is the thing to do. As it happens, this was a good conclusion
to draw: it does, in fact, maximise his (and Nik's) expected utility
over time. (It doesn't maximise their actual utility, as there actually
won't be a flood over the next twelve months. So, the insurance premium
is an expense they could have avoided. But that doesn't seem
particularly relevant for prudential evaluation. Prudential buyers of
insurance should maximise expected utility, not actual utility. Or so we
must say unless we want to be committed to the view that everyone who
buys an insurance policy and doesn't make a claim on it is imprudent.)

But again, it's a close call. If there had been a little less evidence
that a flood was a realistic possibility, or the opportunity cost of
using those dollars on insurance premiums had been a little higher, or
the utility function over different outcomes a little different, it
would have been better to forego flood insurance. That suggests that
safety considerations make it the case that Raj doesn't know that buying
flood insurance is the thing to do, though in fact it is.

Let's now assume Raj has done everything he should do to investigate the
costs and benefits of flood insurance. We can imagine a conversation
between him and Nik going as follows.

\begin{quote}
Nik: Should we get flood insurance?\\
Raj: I don't know. Hold on; I'm on the phone.\\
Nik: Who are you calling?\\
Raj: The insurance agent. I'm buying flood insurance.
\end{quote}

There is clearly a pragmatic tension in Raj's actions here. But given
The Knowledge Rule, there's little else he can do. It would be a serious
norm violation to say nothing in response to Nik's question. And given
that he can't say ``Yes'' without violating The Knowledge Rule, he has
to say ``I don't know''. Moreover, since by hypothesis buying flood
insurance is the thing to do in his situation, he can't not buy the
insurance without doing the wrong thing. So, given The Knowledge Rule,
he's doing the best he can. But it's crazy to think that this is the
best he can do.

We think that these cases are problems for The Knowledge Rule. In
particular, we think that in each case, there is a non-defective
assertion of something that is not known. It seems to us intuitively
clear that those assertions are non-defective, but for those who don't
share this intuition, we have three independent arguments. The arguments
focus on \textbf{Going to War}, but they generalize easily enough to
\textbf{Buying Flood Insurance}.

\subsection*{Argument One: ``That was your first
mistake''}\label{argument-one-that-was-your-first-mistake}
\addcontentsline{toc}{subsection}{Argument One: ``That was your first
mistake''}

Imagine that the Prime Minister has a philosophical advisor. And the
advisor's job is to inform the Prime Minister whenever she violates a
norm, and stay silent otherwise. If The Knowledge Rule is correct, then
the advisor should stay silent as the Prime Minister orders the
invasion, silent as the Prime Minister sets out the reasons for the
invasion, then speak up at the very last line of the speech. That
strikes us as absurd. It's particularly absurd when you consider that
the last line of the speech is supported by what came earlier in the
speech, and the Prime Minister believes it, and asserts it, because it
is well supported by what came earlier in the speech. Since we think
this couldn't be the right behaviour for the advisor, we conclude that
there's no norm violation in the Prime Minister asserting (1).

We've heard two replies to this kind of argument. According to one sort
of reply, The Knowledge Rule is not meant to be an
`all-things-considered' norm. The defender of The Knowledge Rule can say
that the Prime Minister's assertion is defective because it violates
that rule, but allow that it is nevertheless all-things-considered
proper, because some other norm outweighs The Knowledge Rule on this
occasion. We agree that The Knowledge Rule is not intended to be an
all-things-considered norm. But even keeping clearly in mind the
distinction between being defective in some respect and being defective
all-things-considered, it is still deeply unintuitive to say that the
Prime Minister's assertion is defective in a respect. That is, we don't
think the philosophical advisor should speak up just at the very end of
the Prime Minister's speech even if she's meant to observe \emph{all}
the norm violations (rather than just the all-things-considered norm
violations).

Perhaps the defender of The Knowledge Rule needn't just appeal to an
intuition here. Another reply we've heard starts from the premise that
the Prime Minister's assertion would be better, in a certain respect, if
she \emph{knew} that it was true. Therefore, there is a respect in which
that assertion is defective, just as The Knowledge Rule requires. To
this second reply, our response is that the premise is true, but the
reasoning is invalid. Saying why requires reflecting a bit on the nature
of norms.

There are lots of ways for assertions to be better. It is better,
\emph{ceteris paribus}, for assertions to be funny rather than unfunny.
It is better for assertions to be sensitive rather than insensitive. (We
mean this both in the Nozickian sense, i.e., an assertion is sensitive
iff it wouldn't have been made if it weren't true, and in the Hallmark
greeting card sense.) It is better for speakers to be certain of the
truth of their assertions than for them to be uncertain. But these facts
don't imply that humour, sensitivity, or certainty are norms of
assertion, for it doesn't follow that assertions that lack humour (or
sensitivity or certainty) are \emph{always} defective. Similarly, the
fact that it is better to know what you say than not doesn't imply that
asserting what you don't know is always defective. In slogan form:
\emph{Not every absence of virtue is a vice}. We think knowledge is a
virtue of assertions. (In fact, we think that pretty much every norm of
assertion that has been proposed in the literature picks out a virtue of
assertion.) What we deny is that the absence of knowledge is (always) a
vice. Since not every absence of virtue is a vice, one can't argue that
the Prime Minister's assertion is \emph{defective} by arguing it could
have been better. And that's why the argument being considered is
invalid.

\subsection*{Argument Two: ``Actions speak louder than
words''}\label{argument-two-actions-speak-louder-than-words}
\addcontentsline{toc}{subsection}{Argument Two: ``Actions speak louder
than words''}

It's a bit of folk wisdom that actions speak louder than words. It isn't
crystal clear just what this wisdom amounts to, but we think one aspect
of it is that an agent incurs more normative commitments by \emph{doing}
\emph{X} than by \emph{talking} about \emph{X}. But if The Knowledge
Rule is right, then this piece of wisdom is in this aspect
back-to-front. According to that rule, an agent incurs a greater
normative commitment by \emph{saying} that \emph{X} is what to do than
they do by just doing \emph{X}. If they do \emph{X}, and \emph{X} is
indeed what to do, then they've satisfied all of their normative
commitments. If, by contrast, they say that \emph{X} is what to do, then
not only must \emph{X} be what to do, but they must know this fact as
well. This strikes us as completely back-to-front. We conclude that
there is nothing improper about asserting that \emph{X} is what to do
(as the Prime Minister does), when \emph{X} is in fact what to do.

\subsection*{Argument Three: ``What else could I
do?''}\label{argument-three-what-else-could-i-do}
\addcontentsline{toc}{subsection}{Argument Three: ``What else could I
do?''}

Here's a quite different argument that \emph{Going to War} \emph{is} a
counterexample to The Knowledge Rule.

\begin{enumerate}
\def\labelenumi{\arabic{enumi}.}
\item
  If ending the speech the way she did was a norm violation, there is a
  better way for the Prime Minister to end her speech.
\item
  There is no better way for the Prime Minister to end the speech
  without saying something that she does not know to be true.
\item
  So, ending the speech the way she did was not a norm violation.
\item
  So, The Knowledge Rule is subject to counterexample.
\end{enumerate}

Premise 1 is a kind of `ought-implies-can' principle, and as such, it
isn't completely obvious that it is true. But when we've presented this
argument to various groups, the focus has always been on premise two.
The common complaint has been that the Prime Minister could have ended
the speech in one of the following ways, thereby complying with The
Knowledge Rule.

\begin{itemize}
\item
  I've decided that going to war is the thing to do in the
  circumstances.
\item
  I believe that going to war is the thing to do in the circumstances.
\item
  It seems to me that going to war is the thing to do in the
  circumstances.
\end{itemize}

Our first reply to this suggestion is that we'd fire a speechwriter who
recommended that a Prime Minister end such a speech in such a weaselly
way, so this hardly counts as a criticism of premise 2. Our more serious
reply is that even if the Prime Minister ended the speech this way,
she'd \emph{still} violate The Knowledge Rule. To see why this is so, we
need to pay a little closer attention to what The Knowledge Rule says.

Note that The Knowledge Rule is \emph{not} a rule about what kind of
declarative utterance you can properly make. An actor playing Hamlet
does not violate The Knowledge Rule if he fails to check, before
entering the stage, whether something is indeed rotten in the state of
Denmark. The rule is a rule about what one \emph{asserts}. And just as
you can assert \emph{less} than you declaratively utter (e.g., on
stage), you can also assert \emph{more} than you declaratively
utter.\footnote{The points we're about to make are fairly familiar by
  now, but for more detail, see Cappelen and Lepore
  (\citeproc{ref-Cappelen2005}{2005}), which played an important role in
  reminding the philosophy of language community of their significance.}
For instance, someone who utters \emph{The F is G} in a context in which
it is common ground that \emph{a} is the \emph{F} typically asserts both
that the \emph{F} is \emph{G}, and that \emph{a} is \emph{G}. Similarly,
someone who utters \emph{I think that S} typically asserts both asserts
that they have a certain thought, and asserts the content of that
thought. We can see this is so by noting that we can properly challenge
an utterance of \emph{I think that S} by providing reasons that \emph{S}
is false, even if these are not reasons that show that the speaker does
not (or at least did not) have such a thought. In the context of her
speech of the House of Commons, even if the Prime Minister were to end
with one of the options above, she would still assert the same thing she
would assert by uttering (1) in the circumstances, and she'd still be
right to make such an assertion.

\section{Bases for Action and
Assertion}\label{bases-for-action-and-assertion}

One might worry that premise 1 in our master argument is mistaken, in
the following way. We said that if \emph{X} is the thing to do for
\emph{S}, then \emph{S} can say that \emph{X} is what to do. But one
might worry about cases where \emph{S} makes a lucky guess about what is
to be done. Above we imagined that Raj had taken all of the factors
relevant to buying flood insurance into account. But imagine a different
case, one involving Raj*, Raj's twin in a similar possible world. Raj*
decides to buy flood insurance because he consults his Magic 8-Ball.
Then, even if buying flood insurance would still maximize his expected
utility, it doesn't seem right for Raj* to say that buying flood
insurance is what to do.

Here is a defence of premise 1 that seems initially attractive, though
not, we think, ultimately successful. The Magic 8-ball case isn't a
clear counterexample to premise 1, it might be argued, because it isn't
clear that buying flood insurance for these reasons is the thing for
Raj* to do. On one hand, we do have the concept of doing the right thing
for the wrong reasons, and maybe that is the right way to describe what
Raj* does if he follows the ball's advice. But it isn't clearly a
correct way to describe Raj*. It's not true, after all, that he's
maximising actual utility. (Remember that there will be no claims on the
policy he buys.) And it isn't clear how to think about expected utility
maximisation when the entrepreneur in question relies on the old Magic
8-Ball for decision making. And we certainly want to say that there's
something wrong about this very decision when made using the Magic
8-Ball. So, perhaps we could say that buying flood insurance isn't what
to do for Raj* in this variant example, because he has bad reasons.

But this seems like a tendentious defence of the first premise. Worse
still, it is an unnecessary defence. What we really want to focus on are
cases where people do the right thing for the right reasons. Borrowing a
leaf from modern epistemology, we'll talk about actions having a basis.
As well as there being a thing to do in the circumstances (or, more
plausibly, a range of things to do), there is also a correct
\emph{basis} for doing that thing (or, more plausibly, a range of
correct bases). What we care about is when \emph{S} does \emph{X} on
basis \emph{B}, and doing \emph{X} on basis \emph{B} is the thing to do
in \emph{S}'s situation. Using this notion of a basis for action, we can
restate the main argument.

\begin{quote}
\emph{Master Argument (Corrected)}

\begin{enumerate}
\def\labelenumi{\arabic{enumi}.}
\item
  If doing \emph{X} on basis \emph{B} is what to do for agent \emph{S},
  then \emph{S} can properly, on basis \emph{B}, assert that \emph{X} is
  what to do (assuming this is relevant to the conversation).
\item
  It is possible that doing \emph{X} on basis \emph{B} is what to do for
  \emph{S}, even though \emph{S} is not in a position to know, and
  certainly not in a position to know on basis \emph{B}, that \emph{X}
  is what to do.
\item
  So, it is possible that \emph{S} properly can assert that \emph{X} is
  what to do, even though she does not know, and is not even in a
  position to know, that \emph{X} is what to do.
\end{enumerate}
\end{quote}

We endorse this version of the master argument. Since its conclusion is
the denial of The Knowledge Rule, we conclude that The Knowledge Rule is
mistaken. But we perhaps haven't said enough about premise 2 to seal the
argument. The next section addresses that issue.

\section{Marginal Wars}\label{marginal-wars}

The argument for premise 2 is just a simple application of Williamson's
anti-luminosity reasoning. (The canonical statement of this reasoning is
in Williamson (\citeproc{ref-Williamson2000-WILKAI}{2000} Ch. 4)).
Williamson essentially argues as follows, for many different values of
\emph{p}. There are many ways for \emph{p} to be true, and many ways for
it to be false. Some of the ways in which \emph{p} can be true are
extremely similar to ways in which it can be false. If one of those ways
is the actual way in which \emph{p} is true, then to know that \emph{p}
we have to know that situations very similar to the actual situation do
not obtain. But in general we can't know that. So, some of the ways in
which \emph{p} can be true are not compatible with our knowing that
\emph{p} is true. In Williamson's nice phrase, \emph{p} isn't
\emph{luminous}, where a luminous proposition is one that can be known
(by a salient agent) whenever it is true. The argument of this paragraph
is called `an anti-luminosity argument', and we think that many
instances of it are sound.

There is a crucial epistemic premise in the middle of that argument:
that we can't know something if it is false in similar situations. There
are two ways that we could try to motivate this premise. First, we could
try to motivate it with the help of conceptual considerations about the
nature of knowledge. That's the approach that Williamson takes. But his
approach is controversial. It is criticised by Sainsbury
(\citeproc{ref-Sainsbury1996}{1995}) and Weatherson
(\citeproc{ref-Weatherson2004-WEALMT}{2004}) on the grounds that his
safety principle goes awry in some special cases. Sainsbury focuses on
mathematical knowledge, Weatherson on introspective knowledge. But the
cases in which we're most interested in this paper -- Indalia going to
war, Raj and Nik buying flood insurance -- don't seem to fall into
either of these problem categories. Nevertheless, rather than pursue
this line, we'll consider a different approach to motivating this
premise.

The second motivation for the epistemic premise comes from details of
the particular cases. In the two cases on which we're focusing, the
agents simply lack fine discriminatory capacities. They can't tell some
possibilities apart from nearby possibilities. That is, they can't know
whether they're in one world or in some nearby world. That's not because
it's conceptually impossible to know something that fine, but simply an
unfortunate fact about their setup. If they can't know that they're not
in a particular nearby world in which \(\neg\)\emph{p}, they can't know
\emph{p}. Using variants of \emph{Going to War}, we'll describe a few
ways this could come about.

The simplest way for this to come about is if war-making is the thing to
do given what we know, but some of the crucial evidence consists of
facts that we know, but don't know that we know. Imagine that a crucial
piece of Indalia's case for war comes from information from an Indalian
spy working behind enemy lines. As it turns out, the spy is reliable, so
the leaders of Indalia can acquire knowledge from her testimony. But she
could easily enough have been unreliable. She could, for instance, have
been bought off by the enemy's agents. As it happens, the amount of
money that would have taken was outside the budget the enemy has
available for counterintelligence. But had the spy been a little less
loyal, or the enemy a little less frugal with the counterintelligence
budget, she could easily have been supplying misinformation to Indalia.
So, while the spy is a safe knowledge source, the Indalian leaders don't
\emph{know} that she is safe. They don't, for instance, know the size of
the enemy's counterintelligence budget, or how much it would take to buy
off their spy, so for all they know, she is very much at risk of being
bought off.

In this case, if the spy tells the Indalian leaders that \emph{p}, they
come to know that \emph{p}, and they can discriminate \emph{p} worlds
from \(\neg\)\emph{p} worlds. But they don't know that they know that
\emph{p}, so for all they know, they don't know \emph{p}. And for some
\emph{p} that they learn from the spy, if they don't know \emph{p}, then
going to war isn't the thing for them to do in the circumstances. So,
given that they don't know the spy is reliable, they don't know that
going to war is the thing for them to do. But the spy really is
reliable, so they do know \emph{p}, so going to war is indeed the thing
for them to do.

Or consider a slightly less fanciful case, involving statistical
sampling. Part of the Prime Minister's case for starting the war was
that the enemy was killing his own citizens. Presumably she meant that
he was killing them in large numbers. (Every country with capital
punishment kills its own citizens, but arguably that isn't a sufficient
reason to invade.) In practice, our knowledge of the scope of this kind
of governmental killing comes from statistical sampling. And this
sampling has a margin of error. Now imagine that the Indalian leaders
know that a sample has been taken, and that it shows that the enemy has
killed \emph{n} of his citizens, with a margin of error of \emph{m}. So,
assuming there really are \emph{n} killings, they know that the enemy
has killed between \emph{n} - \emph{m} and \emph{n} + \emph{m} of his
citizens. Since knowing that he's killed \emph{n} - \emph{m} people is
sufficient to make going to war the thing to do, the war can be properly
started.

But now let's think about what the Indalian leaders know that they know
in this case. The world where the enemy has killed \emph{n} - \emph{m}
people is consistent with their knowledge. And their margin of error on
estimates of how many the enemy has killed is \emph{m}. So, if that
world is actual, they don't know the enemy has killed more than \emph{n}
- 2\emph{m} of his citizens. And that knowledge might not be enough to
make going to war the thing to do, especially if \emph{m} is large.
(Think about the case where \emph{m} = \emph{n}/2, for instance.) So,
there's a world consistent with their knowledge (the \emph{n} - \emph{m}
killings world), in which they don't know enough about what the enemy is
doing to make going to war the thing to do. In general, if there's a
world consistent with your knowledge where \emph{p} is false, you don't
know \emph{p}. Letting \emph{p} be \emph{Going to war is what to do}, it
follows then that they don't know that going to war is what to do, even
though it actually is the thing to do.

Another way we could have a borderline war is a little more
controversial. Imagine a case where the leaders of Indalia know all the
salient descriptive facts about the war. They know, at least well enough
for present purposes, what the costs and benefits of the war might be.
But it is a close call whether the war is the thing to do given those
costs and benefits. Perhaps different plausible moral theories lead to
different conclusions. Or perhaps the leaders know what the true moral
theory is, but that theory offers ambiguous advice. We can imagine a
continuum of cases where the true theory says war is clearly what to do
at one end, clearly not what to do at another, and a lot of murky space
between. Unless we are willing to give up on classical logic, we must
think that somewhere there is a boundary between the cases where it is
and isn't what to do, and it seems in cases near the boundary even a
true belief about what to do will be unsafe. That is, even a true belief
will be based on capacities that can't reliably discriminate situations
where going to war is what to do from cases where it isn't.

We've found, when discussing this case with others, that some people
find this outcome quite intolerable. They think that there must be some
epistemic constraints on war-making. And we agree. They go on to think
that these constraints will be incompatible with the kind of cases we
have in mind that make premise 2 true. And here we disagree. It's worth
going through the details here, because they tell us quite a bit about
the nature of epistemic constraints on action.

Consider all principles of the form

\begin{description}
\item[(KW)]
Going to war is \emph{N1} only if the war-maker knows that going to war
is \emph{N2}.
\end{description}

where \emph{N1} and \emph{N2} are normative statuses, such as being the
thing to do, being right, being good, being just, being utility
increasing, and so on. All such principles look like epistemic
constraints on war-making, broadly construed. One principle of this form
would be that going to war is right only if the war-maker knows that
going to war is just. That would be an epistemic constraint on
war-making, and a plausible one. Another principle of this form would be
that going to war is the thing to do only if the war-maker knows that
going to war increases actual utility. That would be a very strong
epistemic constraint on war-making, one that would rule out pretty much
every actual war, and one that is consistent with the anti-luminosity
argument with which we started this section. So, the anti-luminosity
argument is consistent with there being quite strong epistemic
constraints on war-making.

What the anti-luminosity argument is \emph{not} consistent with is there
being any true principle of the form (KW) where \emph{N1} equals
\emph{N2}. In particular, it isn't consistent with the principle that
going to war is the thing to do only if the war maker knows that it is
the thing to do. But that principle seems quite implausible, because of
cases where going to war is, but only barely, the thing to do. More
generally, the following luminosity of action principle seems wrong for
just about every value of \emph{X}.

\begin{description}
\item[(LA)]
\emph{X} is the thing for \emph{S} to do only if \emph{S} knows that
\emph{X} is the thing for her to do.
\end{description}

Not only is (LA) implausible, things look bad for The Knowledge Rule if
it has to rely on (LA) being true. None of the defenders of The
Knowledge Rule has given us an argument that (LA) is true. One of them
has given us all we need to show that (LA) is false! It doesn't look
like the kind of principle that The Knowledge Rule should have to depend
upon. So, defending The Knowledge Rule here looks hopeless.

Note that given premise 1 of the Master Argument, as corrected,
\emph{every} instance of (LA) has to be true for The Knowledge Rule to
be universally true. Let's say that you thought (LA) was true when
\emph{X} is \emph{starting a war}, but not when \emph{X} is \emph{buying
flood insurance}. Then we can use the case of Raj and Nik to show that
The Knowledge Rule fails, since Raj can say that buying flood insurance
is what to do in a case where it is what to do, but he doesn't know
this.

One final observation about the anti-luminosity argument. Given the way
Williamson presents the anti-luminosity argument, it can appear that in
all but a few cases, if \emph{p}, the salient agent can know that
\emph{p}. After all, the only examples Williamson gives are cases that
are only picked out by something like the Least Number Theorem. So, one
might think that while luminosity principles are false, they are
approximately true. More precisely, one might think that in all but a
few weird cases near the borderline, if \emph{p}, then a salient agent
is in a position to know \emph{p}. If so, then the failures of
luminosity aren't of much practical interest, and hence the failures of
The Knowledge Rule we've pointed out aren't of much practical interest.

We think this is all mistaken. Luminosity failures arise because agents
have less than infinite discriminatory capacities. The worse the
discriminatory capacities, the greater the scope for luminosity
failures. When agents have very poor discriminatory capacities, there
will be very many luminosity failures. This is especially marked in
decision-making concerning war. The fog of war is thick. There is very
much that we don't know, and what we do know is based on evidence that
is murky and ephemeral. There is very little empirical information that
we know that we know. If there are certain actions (such as starting a
war) that are proper only if we know a lot of empirical information, the
general case will be that we cannot know that these actions are correct,
even when they are. This suggests that luminosity failures, where an
action is correct but not known to be correct, or a fact is known but
not known to be known, are not philosophical curiosities. In
epistemically challenging environments, like a war zone, they are
everyday facts of life.

\section{Moore's Paradox}\label{moores-paradox}

There is a standard argument for The Knowledge Rule that goes as
follows. First, if the Knowledge Rule did not hold, then certain Moore
paradoxical assertions would be acceptable. In particular, it would be
acceptable to assert \emph{q, but I don't know that q}.\footnote{(\citeproc{ref-Williamson2000-WILKAI}{Williamson
  2000}), for instance, shows the strength of this argument.} But
second, Moore paradoxical assertions are never acceptable. Hence, The
Knowledge Rule holds. We reject both premises of this argument.

To reject the first premise, it suffices to show that some rule other
than The Knowledge Rule can explain the unacceptability of Moore
paradoxical assertions. Consider, for example, The Undefeated Reason
rule.

\begin{description}
\item[The Undefeated Reason Rule]
Assert that \emph{p} only if you have an undefeated reason to believe
that \emph{p}.
\end{description}

The Undefeated Reason Rule says that \emph{q but I don't know that q}
can be asserted only if the speaker has an undefeated reason to believe
it. That means the speaker has an undefeated reason to believe each
conjunct. That means that the speaker has an undefeated reason to
believe that they don't know \emph{q}. But in every case where it is
unacceptable to both assert \emph{q} and assert that you don't know
\emph{q}, the speaker's undefeated reason to believe they don't know
\emph{q} will be a defeater for her belief that \emph{q}. If you have
that much evidence that you don't know \emph{q}, that will in general
defeat whatever reason you have to believe \emph{q}.

We don't claim that The Undefeated Reason Rule is correct. (In fact, we
prefer the rules we'll discuss in section 5.) We do claim that it
provides an alternative explanation of the unacceptability of instances
of \emph{q but I don't know that q}. So, we claim that it undermines the
first premise of Williamson's argument from that unacceptability to The
Knowledge Rule.

We also think that Williamson's explanation of Moore paradoxicality
over-generates. There is generally something odd about saying \emph{q
but I don't know that q}. We suspect that the best explanation for why
this is odd will be part of a broader explanation that also explains,
for instance, why saying \emph{I promise to do X, but I'm not actually
doing to do X} is also defective. Williamson's explanation isn't of this
general form. He argues that saying \emph{q but I don't know that q} is
defective because it is defective in \emph{every} context to both assert
\emph{q} and assert that you don't know that \emph{q}. But we don't
think that it is always defective to make both of these
assertions.\footnote{This is why we hedged a little two paragraphs ago
  about what precisely The Undefeated Reason Rule explains. We suspect
  that many in the literature have misidentified the explicandum.} In
particular, if a speaker is asked whether \emph{q} is true, and whether
they know that \emph{q}, it can be acceptable to reply affirmatively to
the first question, but negatively to the second one. If so, then the
second premise of Williamson's argument from Moore paradoxicality is
also false.

Imagine that the Indalian Prime Minister is a philosopher in her spare
time. After the big speech to Parliament she goes to her Peninsula
Reading Group. It turns out Michael Walzer and Tim Williamson are there,
and have questions about the speech.

\begin{quote}
TW: Do you agree that knowledge requires safety?\\
PM: Yes, yes I do.\\
TW: And do you agree that your belief that going to war is the thing to
do is not safe?\\
PM: Right again.\\
TW: So, you don't know that going to war is the thing to do?\\
PM: You're right, I don't.\\
MW: But is it the thing to do?\\
PM: Yes.
\end{quote}

The Prime Minister's answers in this dialogue seem non-defective to us.
But if Williamson's explanation of why Moore paradoxical utterances are
defective is correct, her answers should seem defective. So,
Williamson's explanation over-generates. Whether or not it is true that
all assertions of sentences of the form \emph{q but I don't know that q}
are defective, it isn't true that there is a defect in any performance
that includes both an assertion of \emph{q} and an assertion of the
speaker's ignorance as to whether \emph{q}. The Prime Minister's
performance in her reading group is one such performance. So, the
explanation of Moore paradoxicality cannot be that any such performance
would violate a norm governing assertion.

To sum up, then, we've argued that The Knowledge Rule (a) fails to be
the only explanation of Moore paradoxicality, and (b) misclassifies
certain performances that are a little more complex than simple
conjunctive assertions as defective. So, there's no good argument from
Moore paradoxicality to The Knowledge Rule.

\section{Action and Assertion}\label{action-and-assertion}

If we're right, there's a striking asymmetry between certain kinds of
assertions. In the war example, early in her speech, the Prime Minister
says (2).

That's not the kind of thing she could properly say if it could easily
have been false given her evidence. And like many assertions, this is
not an assertion whose appropriateness is guaranteed by its truth.
Asserting (2) accuses someone of murder, and you can't properly make
such accusations without compelling reasons, even if they happen to be
true. On the other hand, we say, the truth of (1) does (at least when it
is accepted on the right basis) suffice to make it properly assertable.

There's a similar asymmetry in the flood insurance example. In that
example, (3) is true, but neither Raj nor Nik knows it.

Again, in these circumstances, this isn't the kind of thing Raj can
properly say. Even though (3) is true, it would be foolhardy for Raj to
make such a claim without very good reasons. By contrast, again, we say
that Raj can properly assert that the thing to do, in their
circumstances, is to buy flood insurance, even though he does not know
this.

There are two directions one could go at this point. If we're right, any
proposed theory of the norms governing assertion must explain the
asymmetry. Theories that cannot explain it, like The Knowledge Rule, or
the Certainty Rule proposed by Jason Stanley
(\citeproc{ref-Stanley2008-STAKAC}{2008}), or the Rational Credibility
Rule proposed by Igor Douven (\citeproc{ref-Douven2006}{2006}), are
thereby refuted.

\begin{description}
\item[The Certainty Rule]
Assert only what is certain.
\item[The Rational Credibility Rule]
Assert only what is rationally credible.
\end{description}

The Certainty Rule fails since the Prime Minister is not certain of (1).
And the Prime Minister can't be certain of (1), since certainty requires
safety just as much as knowledge does.

It's a little harder to show our example refutes The Rational
Credibility Rule. Unlike knowledge, a safety constraint is not built
into the concept of rational credibility. (Since rational credibility
does not entail truth, in Douven's theory, it can hardly entail truth in
nearby worlds.) But we think that safety constraints may still apply to
rational credibility in some particular cases. If you aren't very good
at judging building heights of tall buildings to a finer grain than 10
meters, then merely looking at a building that is 84 meters tall does
not make it rationally credible for you that the building is more than
80 meters tall. In general, if your evidence does not give you much
reason to think you are not in some particular world where \emph{p} is
false, and you didn't have prior reason to rule that world out, then
\emph{p} isn't rationally credible. So, when evidence doesn't
discriminate between nearby possibilities, and \emph{p} is false in
nearby possibilities, \emph{p} isn't rationally credible.

And that, we think, is what happens in our two examples. Just as someone
looking at an 84 meter building can't rationally credit that it is more
than 80 meters tall, unless they are abnormally good at judging heights,
agents for whom \emph{X} is just barely the thing to do can't rationally
credit that \emph{X} is the thing to do. By The Rational Credibility
Rule, they can't say \emph{X} is the thing to do. But they can say that;
that's what our examples show. So, The Rational Credibility Rule must be
wrong.

But we can imagine someone pushing in the other direction, perhaps with
the help of this abductive argument.

\begin{enumerate}
\def\labelenumi{\arabic{enumi}.}
\item
  A speaker can only assert things like (2) or (3) if they know them to
  be true.
\item
  The best explanation of premise 1 of this argument is The Knowledge
  Rule.
\item
  So, The Knowledge Rule is correct.
\end{enumerate}

This isn't a crazy argument. Indeed, it seems to us that it is implicit
in some of the better arguments for The Knowledge Rule. But we think it
fails. And it fails because there are alternative explanations of the
first premise, explanations that don't make mistaken predictions about
the Prime Minister's speech. For instance, we might have some kind of
Evidence Responsiveness Rule.

\begin{description}
\item[The Evidence Responsiveness Rule]
Assert that \emph{p} only if your attitude towards \emph{p} is properly
responsive to the evidence you have that bears on \emph{p}.
\end{description}

Given how much can be covered by `properly', this is more of a schema
than a rule. Indeed, it is a schema that has The Knowledge Rule as one
of its precisifications. In \emph{Knowledge and Its Limits}, Williamson
first argues that assertion is ``governed by a non-derivative evidential
rule'' (249), and then goes on to argue that the proper form of that
rule is The Knowledge Rule. We agree with the first argument, and
disagree with the second one.\footnote{Actually, our agreement with
  Williamson here is a bit more extensive than the text suggests.
  Williamson holds that part of what makes a speech act an
  \emph{assertion} as opposed to some other kind of act is that it is
  governed by The Knowledge Rule. Although many philosophers agree with
  Williamson that The Knowledge Rule is true, this fascinating claim
  about the metaphysics of speech acts has been largely ignored.
  Translating Williamson's work into the terminology of this paper,
  we're inclined to agree that a speech act is an assertion partly in
  virtue of being responsive to evidence in the right way. But filling
  in the details on this part of the story would take us too far from
  the main storyline of this paper.}

Note that even a fairly weak version of The Evidence Responsiveness Rule
would explain what is going on with cases like (1) and (2). Starting a
war is a serious business. You can't properly do it \emph{unless} your
views about the war are evidence responsive in the right way. You can't,
that is, correctly \emph{guess} that starting the war is the thing to
do. You \emph{can} correctly guess that starting the war will be utility
maximizing. And you can correctly guess that starting the war would be
what to choose if you reflected properly on the evidence you have, and
the moral significance of the choices in front of you. But you simply
can't guess that starting the war is what to do, and be right. If you're
merely guessing that starting a war is thing to do, then you're wrong to
start that war. So, if (1) is true, and the Prime Minister believes it,
her belief simply \emph{must} be evidence responsive. Then, by The
Evidence Responsiveness Rule, she can assert it.

For most assertions, however, this isn't the case. Even if it's true
that it will rain tomorrow, the Prime Minister's could believe that
without her belief being evidence responsive. In general, \emph{p} does
not entail that \emph{S} even believes that \emph{p}, let alone that
this belief of \emph{S}'s is evidence responsive. But in cases like (1),
this entailment does hold, and that's what explains the apparent
asymmetry that we started this section with.

The Evidence Responsiveness Rule also handles so called `lottery
propositions' nicely. If you know that the objective chance of \emph{p}
being true is \emph{c}, where \emph{c} is less than 1, it will seem odd
in a lot of contexts to simply assert \emph{p}. In his arguments for The
Knowledge Rule, Williamson makes a lot of this fact. In particular, he
claims that the best explanation for this is that we can't know that
\emph{p} on purely probabilistic grounds. This has proven to be one of
the most influential arguments for The Knowledge Rule in the literature.
But some kind of Evidence Responsiveness Rule seems to handle lottery
cases even more smoothly. In particular, an Evidence Responsiveness Rule
that allows for what constitutes `proper' responsiveness to be sensitive
to the interests of the conversational participants will explain some
odd features concerning lottery propositions and assertability.

In the kind of cases that motivate Williamson, we can't say \emph{p}
where it is objectively chancy whether \emph{p}, and the chance of
\emph{p} is less than 1. But there's one good sense in which such an
assertion would not be properly responsive to the evidence. After all,
in such a case there's a nearby world, with all the same laws, and with
all the same past fatcs, and in which the agent has all the same
evidence, in which \emph{p} is false. And the agent knows all this. That
doesn't look like the agent is being properly responsive to her
evidence.

On the other hand, we might suspect that Williamson's arguments
concerning lottery propositions overstate the data. Consider this old
story from David Lewis (\citeproc{ref-Lewis1996b}{1996}).\footnote{We've
  slightly modified the case. Lewis says we can say that we \emph{know}
  Bill will never be rich. That seems to us to be a much more
  controversial than what we've included here.}

\begin{quote}
Pity poor Bill! He squanders all his spare cash on the pokies, the
races, and the lottery. He will be a wage slave all his days \ldots{} he
will never be rich. (\citeproc{ref-Lewis1996b}{Lewis 1996, 443} in
reprint)
\end{quote}

These seem like fine assertions. One explanation of the appropriateness
of those assertions combines The Knowledge Rule with contextualism about
assertion.\footnote{The combination is slightly trickier to state than
  would be ideal. The explanation we have in mind is that \emph{S} can
  properly assert \emph{p} only if \emph{S} can truly say \emph{I know
  that p}, where `know' in this utterance is context sensitive.} But
contextualism has many weaknesses, as shown in Hawthorne (2004) and
Stanley (2005). A less philosophically loaded explanation of Lewis's
example is that proper responsiveness comes in degrees, and for purposes
of talking about Bill, knowing that it's overwhelmingly likely that he's
doomed to wage slavery is evidence enough to assert that he'll never be
rich. The details of this explanation obviously need to be filled in,
but putting some of the sensitivity to conversational standards, or
practical interests, into the norms of assertion seems to be a simpler
explanation of the data than a contextualist explanation. (It would be
\emph{a priori} quite surprising if the norms of proper assertion were
not context-sensitive, or interests-sensitive. The norms of
appropriateness for most actions are sensitive to context and
interests.) So The Evidence Responsiveness Rule seems more promising
here than The Knowledge Rule.

A harder kind of case for The Knowledge Rule concerns what we might call
`academic assertions'. This kind of case is discussed in Douven
(\citeproc{ref-Douven2006}{2006}) and in Maitra
(\citeproc{ref-MaitraANG}{2010}). In academic papers, we typically make
assertions that we do not know. We don't know that most of the things
we've said here are true. (Before the last sentence we're not sure we
knew that any of the things we said were true.) But that's because
knowledge is a bad standard for academic discourse. Debate and
discussion would atrophy if we had to wait until we had knowledge before
we could present a view. So, it seems that assertion can properly outrun
knowledge in academic debate.

Again, a context-sensitive version of The Evidence Responsiveness Rule
explains the data well. Although you don't need to \emph{know} things to
assert them in philosophy papers, you have to have evidence for them. We
couldn't have just spent this paper insisting louder and louder that The
Knowledge Rule is false. We needed to provide evidence, and hopefully
we've provided a lot of it. In some contexts, such as testifying in
court, you probably need more evidence than what we've offered to ground
assertions. But in dynamic contexts of inquiry, where atrophy is to be
feared more than temporary mistakes, the standards are lower. Good
evidence, even if not evidence beyond any reasonable doubt, or even if
not enough for knowledge, suffices for assertion. That's the standard we
typically hold academic papers to. Like with lotteries, we think the
prospects of explaining these apparently variable standards in terms of
a norm of assertion that is context-sensitive are greater than the
prospects for explaining them in terms of contextually sensitive
knowledge ascriptions.

Here's a different and somewhat more speculative proposal idea for a
rule that also explains the asymmetry we started this section with. We
call it the Action Rule.

\begin{description}
\item[The Action Rule]
Assert that \emph{p} only if acting as if \emph{p} is true is the thing
for you to do.
\end{description}

We take the notion of acting as if something is true from Stalnaker
(\citeproc{ref-Stalnaker1973-STAP-5}{1973}). Intuitively, to act as if
\emph{p} is true is to build \emph{p} into one's plans, or to take
\emph{p} for granted when acting. This, note, is not the same as using
\emph{p} as a basis for action. When Raj buys flood insurance, he acts
as if buying flood insurance is the thing to do. But the fact that
buying flood insurance is the thing to do isn't the basis for his
action. (Since he does not know this, one might suspect it wouldn't be a
good basis.) Instead his basis is what he knows about the river, and his
business, and its vulnerability to flooding. When an agent is trying to
maximise the expected value of some variable (e.g., utility, profit,
etc.), then to act as if \emph{p} is true is simply to maximise the
conditional expected value of that variable, in particular, to maximise
the expected value of that variable conditional on \emph{p}. Even when
one is not maximising any expected value, we can still use the same
idea. To act as if \emph{p} is to take certain conditional obligations
or permissions you have -- in particular, those obligations or
permissions that are conditional on \emph{p} -- to be actual obligations
or permissions.

To see how The Action Rule generates the intended asymmetry, we'll need
a bit of formalism. Here are the terms that we will use.

\begin{itemize}
\item
  \emph{X} denotes an action, agent, circumstance triple
  \(\langle\)\emph{X\textsubscript{Action}, X\textsubscript{Agent},
  X\textsubscript{Circumstance}}\(\rangle\). We take such triples to
  have a truth value\emph{.} \emph{X} is true iff
  \emph{X\textsubscript{Agent}} performs \emph{X\textsubscript{Action}}
  in \emph{X\textsubscript{Circumstance}}.
\item
  ThingToDo(\emph{X}) means that \emph{X} is the thing to do for
  \emph{X\textsubscript{Agent}} in \emph{X\textsubscript{Circumstance}}.
\item
  Act(\emph{S,p}) means that agent \emph{S} acts as if \emph{p} is true.
\item
  Assert(\emph{S},\emph{p}) means that agent \emph{S} can properly
  assert that \emph{p}.
\end{itemize}

So, The Action Rule is this.

\begin{quote}
Assert(\emph{S,p}) \({\rightarrow}\) ThingToDo(Act(\emph{S,p}))
\end{quote}

In our derivations, the following equivalence will be crucial.

\begin{quote}
Act(\emph{X\textsubscript{Agent},}ThingToDo(\emph{X}))
\({\leftrightarrow}\) \emph{X}
\end{quote}

That is, acting as if \emph{X} is what to do (in your circumstances) is
simply to do \emph{X} (in those circumstances). And in doing \emph{X},
you're acting as if \emph{X} is what to do (in your circumstances). We
take this equivalence to be quite resilient; in particular, it holds
under operators like `ThingToDo'. So, adding that operator to the
previous equivalence, we get another equivalence.

\begin{quote}
ThingToDo(Act(\emph{X\textsubscript{Agent},}ThingToDo(\emph{X})))
\({\leftrightarrow}\) ThingToDo(\emph{X})
\end{quote}

If we substitute ThingToDo(\emph{X}) for \emph{p} in The Action Rule, we
get this.

\begin{quote}
Assert(\emph{X\textsubscript{Agent},}ThingToDo(\emph{X}))
\({\rightarrow}\)
ThingToDo(Act(\emph{X\textsubscript{Agent},}ThingToDo(\emph{X})))
\end{quote}

But by the equivalence we derived earlier, that's equivalent to the
following.

\begin{quote}
Assert(\emph{X\textsubscript{Agent},}ThingToDo(\emph{X}))
\({\rightarrow}\) ThingToDo(\emph{X})
\end{quote}

So, we get the nice result that The Action Rule is trivially satisfied
for any true claim about what is to be done. That is, for the special
case where \emph{p} is \emph{X is the thing for you to do}, The Action
Rule just reduces to something like the Truth Rule. And so we get a nice
explanation of why the Prime Minister and Raj can properly make their
assertions about what to do in their respective
circumstances.\footnote{The derivation here is deliberately simplified
  in one way. We haven't included anything about the \emph{bases} for
  action or assertion. We don't think being sensitive to bases in the
  formalism would make a material change, but it would obscure the
  structure of the argument.}

To explain the other side of the asymmetry with which we began this
section, note that these biconditionals do not hold where \emph{p} is an
arbitrary proposition, and \emph{S} an arbitrary agent.

\begin{quote}
ThingToDo(Act(\emph{S,p})) \({\leftrightarrow}\) \emph{p}

Act(\emph{S},ThingToDo(Act(\emph{S},\emph{p}))) \({\leftrightarrow}\)
\emph{p}
\end{quote}

To see this, let \emph{p} be the proposition expressed by (4). To act as
if this is true is to, \emph{inter alia}, not buy flood insurance. If
there won't be a flood, buying flood insurance is throwing away money,
and when you're running a business, throwing away money isn't the thing
to do. In symbols, \emph{Act(}Raj and Nik,\emph{p)} is equivalent to
\emph{Raj and Nik don't buy flood insurance}. But not buying flood
insurance is not the thing to do. The prudent plan is to buy flood
insurance. So, \emph{ThingToDo(Act(}Raj and Nik,\emph{p))} is false,
even though \emph{p} is true. So, the first biconditional fails. Since
Raj and Nik do go on to buy flood insurance, i.e., since they don't act
as if \emph{ThingToDo(Act(}Raj and Nik,\emph{p))}, the left-hand-side of
the second biconditional is also false. But again, the right-hand-side
is true. So, that biconditional is false as well. And without those
biconditionals, The Action Rule doesn't collapse into
\emph{Assert(S},\emph{p)} \({\rightarrow}\) \emph{p}.

We have thus far argued that The Action Rule can provide an explanation
for the asymmetry we noted at the beginning of this section.\footnote{This
  explanation makes some interestingly different predictions from the
  explanation in terms of The Evidence Responsiveness Rule. Suppose that
  for relatively trivial decisions, like where to go for a walk on a
  nice summer day, one can correctly guess that \emph{X} is the thing to
  do. Then the Evidence Responsiveness Rule would suggest that the truth
  of claims about where to go for a walk is not sufficient grounds for
  their assertability, while the Action Rule would still imply that
  truth is sufficient grounds for assertability. We're not sure that
  this supposition -- that for relatively trivial decisions, one can
  correctly guess that \emph{X} is the thing to do -- is coherent, nor
  what to say about assertability judgments in (imagined) cases where
  the supposition holds. So, we're not sure we can really use this to
  discriminate between the two proposed explanations. Nevertheless, it
  is interesting to note how the explanations come apart. Thanks here to
  Susanna Schellenberg.} This is not, however, meant to be anything like
a complete defence of that rule. That would require a lot more than
we've provided here. But we do think that the Action Rule can explain a
lot of the phenomena that are meant to motivate The Knowledge Rule, as
well as some phenomena The Knowledge Rule struggles with.But we do think
The Action Rule has some virtues. We'll close with a discussion of how
it explains the two kinds of cases that we argued that The Evidence
Responsiveness Rule handles well.

To see this, consider first `lottery propositions'. If you know that the
objective chance of \emph{p} being true is \emph{c}, where \emph{c} is
less than 1, it will seem odd in a lot of contexts to simply assert
\emph{p}. In his arguments for The Knowledge Rule, Williamson makes a
lot of this fact. In particular, he claims that the best explanation for
this is that we can't know that \emph{p} on purely probabilistic
grounds. This has proven to be one of the most influential arguments for
The Knowledge Rule in the literature.

We suggest that The Action Rule can offers an alternative a nice
explanation for why it's often defective to assert lottery propositions.
Note first that inIn a lot of cases, it isn't rational for us to act on
\emph{p} when we have only purely probabilistic evidence for it,
especially when acting on \emph{p} amounts to betting on \emph{p} at
sufficiently unfavourable odds. This point is something of a staple of
the `interest-relative-invariantism' literature on knowledge.\footnote{See,
  for instance, Fantl and McGrath (\citeproc{ref-Fantl2002}{2002}),
  Hawthorne (\citeproc{ref-Hawthorne2004}{2004}), Stanley
  (\citeproc{ref-Stanley2005-STAKAP}{2005}), and Weatherson
  (\citeproc{ref-Weatherson2005-WEACWD}{2005}).} To take a mundane case,
imagine that you're cleaning up your desk, and you come across some
lottery tickets. Most are for lotteries that have passed, that you know
you lost. One ticket, however, is for a future lottery, which you know
you have very little chance of winning. In such a case, to act as if the
ticket for the future lottery would lose would be to throw it out along
with the other tickets. But that would be irrational, and not at all how
we'd act in such a case. That is to say, in such a case, we don't (and
shouldn't, rationally speaking) act as if the ticket for the future
lottery will lose, even though we take that outcome to be highly
probable.

If acting as if a lottery proposition is true isn't the thing to do,
then The Action Rule will say that asserting such a proposition
defective. Therefore, we think that The Action Rule can capture why in
many cases you can't in general assert lottery propositions.

A harder kind of case for The Knowledge Rule concerns what we might call
`academic assertions'. This kind of case is discussed in Douven
(\citeproc{ref-Douven2006}{2006}) and in Maitra
(\citeproc{ref-MaitraANG}{2010}). In academic papers, we typically make
assertions that we do not know. We don't know that most of the things
we've said here are true. (Before the last sentence we're not sure we
knew that any of the things we said were true.) But that's because
knowledge is a bad standard for academic discourse. Debate and
discussion would atrophy if we had to wait until we had knowledge before
we could present a view. So, it seems that assertion can properly outrun
knowledge in academic debate.

Academic assertions raised a problem for The Knowledge Rule because
proper assertion in the context of inquiry can outrun knowledge. But
note that action in such a context can also properly outrun knowledge.
It would slow down learning dramatically if people didn't engage in
various projects that really only make sense if some hypothesis is true.
So, academics will study in archives, conduct experiments, write papers,
etc. etc., and do so on the basis of reasons they no more know than we
know the truth of the speculative claims of this paper. And this is all
to the good; the alternative is a vastly inferior alternative to
academia as we know it. So, in some fields, action requires much less
than knowledge. Happily, in those fields, assertion also requires much
less than knowledge. Indeed, the shortfalls in the two cases seem to
parallel nicely. And this parallel is neatly captured by The Action
Rule.

As we said, none of this is a knockdown case for The Action Rule. Our
primary purpose is to argue against The Knowledge Rule. As long as the
Action Rule is plausible, we have defeated the abductive argument for
The Knowledge Rule that was discussed at the start of this section, and
we think we've done enough to show it is plausible. We also hope we've
made a successful case for moving the study of assertability away from
rules like The Knowledge Rule, and instead have it be more tightly
integrated with our best theories about evidence and action.\footnote{We'd
  like to thank Matthew Benton, Jessica Brown, Andy Egan, and Susanna
  Schellenberg, as well as an audience at the Bellingham Summer
  Philosophy Conference, for helpful discussion of earlier drafts of
  this paper.}

\subsection*{References}\label{references}
\addcontentsline{toc}{subsection}{References}

\phantomsection\label{refs}
\begin{CSLReferences}{1}{0}
\bibitem[\citeproctext]{ref-Bach2010}
Bach, Kent. 2007. {``Knowledge in and Out of Context.''} In
\emph{Knowledge and Skepticism}, edited by Joseph Keim Campbell, Michael
O'Rourke, and Harry S. Silverstein, 105--36. Cambridge, MA: MIT Press.

\bibitem[\citeproctext]{ref-Black1952}
Black, Max. 1952. {``Saying and Disbelieving.''} \emph{Analysis} 13 (2):
25--33. \url{https://doi.org/10.1093/analys/13.2.25}.

\bibitem[\citeproctext]{ref-Cappelen2005}
Cappelen, Herman, and Ernest Lepore. 2005. \emph{Insensitive Semantics:
A Defence of Semantic Minimalism and Speech Act Pluralism}. Oxford:
Blackwell.

\bibitem[\citeproctext]{ref-DeRose2002}
DeRose, Keith. 2002. {``Assertion, Knowledge and Context.''}
\emph{Philosophical Review} 111 (2): 167--203.
\url{https://doi.org/10.2307/3182618}.

\bibitem[\citeproctext]{ref-Douven2006}
Douven, Igor. 2006. {``Assertion, Knowledge and Rational Credibility.''}
\emph{Philosophical Review} 115 (4): 449--85.
\url{https://doi.org/10.1215/00318108-2006-010}.

\bibitem[\citeproctext]{ref-Fantl2002}
Fantl, Jeremy, and Matthew McGrath. 2002. {``Evidence, Pragmatics, and
Justification.''} \emph{Philosophical Review} 111: 67--94.
\url{https://doi.org/10.2307/3182570}.

\bibitem[\citeproctext]{ref-Hawthorne2004}
Hawthorne, John. 2004. \emph{Knowledge and Lotteries}. Oxford: Oxford
University Press.

\bibitem[\citeproctext]{ref-Hindriks2007}
Hindriks, Frank. 2007. {``The Status of the Knowledge Account of
Assertion.''} \emph{Linguistics and Philosophy} 30 (3): 393--406.
\url{https://doi.org/10.1007/s10988-007-9019-5}.

\bibitem[\citeproctext]{ref-Lewis1996b}
Lewis, David. 1996. {``Elusive Knowledge.''} \emph{Australasian Journal
of Philosophy} 74 (4): 549--67.
\url{https://doi.org/10.1080/00048409612347521}.

\bibitem[\citeproctext]{ref-MaitraANG}
Maitra, Ishani. 2010. {``Assertion, Norms and Games.''} In
\emph{Assertion: New Philosophical Essays}, edited by Jessica Brown and
Herman Cappelen, 277--96. Oxford: Oxford University Press.

\bibitem[\citeproctext]{ref-Sainsbury1996}
Sainsbury, Mark. 1995. {``Vagueness, Ignorance and Margin for Error.''}
\emph{British Journal for the Philosophy of Science} 46: 589--601.
\url{https://doi.org/10.1093/bjps/46.4.589}.

\bibitem[\citeproctext]{ref-Stalnaker1973-STAP-5}
Stalnaker, Robert. 1973. {``{Presuppositions}.''} \emph{Journal of
Philosophical Logic} 2 (4): 447--57.
\url{https://doi.org/10.1007/bf00262951}.

\bibitem[\citeproctext]{ref-Stanley2005-STAKAP}
Stanley, Jason. 2005. \emph{{Knowledge and Practical Interests}}. Oxford
University Press.

\bibitem[\citeproctext]{ref-Stanley2008-STAKAC}
---------. 2008. {``{Knowledge and Certainty}.''} \emph{Philosophical
Issues} 18 (1): 35--57.
\url{https://doi.org/10.1111/j.1533-6077.2008.00136.x}.

\bibitem[\citeproctext]{ref-Weatherson2004-WEALMT}
Weatherson, Brian. 2004. {``Luminous Margins.''} \emph{Australasian
Journal of Philosophy} 82 (3): 373--83.
\url{https://doi.org/10.1080/713659874}.

\bibitem[\citeproctext]{ref-Weatherson2005-WEACWD}
---------. 2005. {``{Can We Do Without Pragmatic Encroachment?}''}
\emph{Philosophical Perspectives} 19 (1): 417--43.
\url{https://doi.org/10.1111/j.1520-8583.2005.00068.x}.

\bibitem[\citeproctext]{ref-Weiner2005}
Weiner, Matthew. 2005. {``Must We Know What We Say.''}
\emph{Philosophical Review} 114 (2): 227--51.
\url{https://doi.org/10.1215/00318108-114-2-227}.

\bibitem[\citeproctext]{ref-Williamson1996-WILKAA}
Williamson, Timothy. 1996. {``{Knowing and Asserting}.''}
\emph{Philosophical Review} 105 (4): 489--523.
\url{https://doi.org/10.2307/2998423}.

\bibitem[\citeproctext]{ref-Williamson2000-WILKAI}
---------. 2000. \emph{{Knowledge and its Limits}}. Oxford University
Press.

\end{CSLReferences}



\end{document}
