% Options for packages loaded elsewhere
% Options for packages loaded elsewhere
\PassOptionsToPackage{unicode}{hyperref}
\PassOptionsToPackage{hyphens}{url}
%
\documentclass[
  11pt,
  letterpaper,
  DIV=11,
  numbers=noendperiod,
  twoside]{scrartcl}
\usepackage{xcolor}
\usepackage[left=1.1in, right=1in, top=0.8in, bottom=0.8in,
paperheight=9.5in, paperwidth=7in, includemp=TRUE, marginparwidth=0in,
marginparsep=0in]{geometry}
\usepackage{amsmath,amssymb}
\setcounter{secnumdepth}{3}
\usepackage{iftex}
\ifPDFTeX
  \usepackage[T1]{fontenc}
  \usepackage[utf8]{inputenc}
  \usepackage{textcomp} % provide euro and other symbols
\else % if luatex or xetex
  \usepackage{unicode-math} % this also loads fontspec
  \defaultfontfeatures{Scale=MatchLowercase}
  \defaultfontfeatures[\rmfamily]{Ligatures=TeX,Scale=1}
\fi
\usepackage{lmodern}
\ifPDFTeX\else
  % xetex/luatex font selection
  \setmainfont[ItalicFont=EB Garamond Italic,BoldFont=EB Garamond
SemiBold]{EB Garamond Math}
  \setsansfont[]{EB Garamond}
  \setmathfont[]{Garamond-Math}
\fi
% Use upquote if available, for straight quotes in verbatim environments
\IfFileExists{upquote.sty}{\usepackage{upquote}}{}
\IfFileExists{microtype.sty}{% use microtype if available
  \usepackage[]{microtype}
  \UseMicrotypeSet[protrusion]{basicmath} % disable protrusion for tt fonts
}{}
\usepackage{setspace}
% Make \paragraph and \subparagraph free-standing
\makeatletter
\ifx\paragraph\undefined\else
  \let\oldparagraph\paragraph
  \renewcommand{\paragraph}{
    \@ifstar
      \xxxParagraphStar
      \xxxParagraphNoStar
  }
  \newcommand{\xxxParagraphStar}[1]{\oldparagraph*{#1}\mbox{}}
  \newcommand{\xxxParagraphNoStar}[1]{\oldparagraph{#1}\mbox{}}
\fi
\ifx\subparagraph\undefined\else
  \let\oldsubparagraph\subparagraph
  \renewcommand{\subparagraph}{
    \@ifstar
      \xxxSubParagraphStar
      \xxxSubParagraphNoStar
  }
  \newcommand{\xxxSubParagraphStar}[1]{\oldsubparagraph*{#1}\mbox{}}
  \newcommand{\xxxSubParagraphNoStar}[1]{\oldsubparagraph{#1}\mbox{}}
\fi
\makeatother


\usepackage{longtable,booktabs,array}
\usepackage{calc} % for calculating minipage widths
% Correct order of tables after \paragraph or \subparagraph
\usepackage{etoolbox}
\makeatletter
\patchcmd\longtable{\par}{\if@noskipsec\mbox{}\fi\par}{}{}
\makeatother
% Allow footnotes in longtable head/foot
\IfFileExists{footnotehyper.sty}{\usepackage{footnotehyper}}{\usepackage{footnote}}
\makesavenoteenv{longtable}
\usepackage{graphicx}
\makeatletter
\newsavebox\pandoc@box
\newcommand*\pandocbounded[1]{% scales image to fit in text height/width
  \sbox\pandoc@box{#1}%
  \Gscale@div\@tempa{\textheight}{\dimexpr\ht\pandoc@box+\dp\pandoc@box\relax}%
  \Gscale@div\@tempb{\linewidth}{\wd\pandoc@box}%
  \ifdim\@tempb\p@<\@tempa\p@\let\@tempa\@tempb\fi% select the smaller of both
  \ifdim\@tempa\p@<\p@\scalebox{\@tempa}{\usebox\pandoc@box}%
  \else\usebox{\pandoc@box}%
  \fi%
}
% Set default figure placement to htbp
\def\fps@figure{htbp}
\makeatother





\setlength{\emergencystretch}{3em} % prevent overfull lines

\providecommand{\tightlist}{%
  \setlength{\itemsep}{0pt}\setlength{\parskip}{0pt}}



 


\setlength\heavyrulewidth{0ex}
\setlength\lightrulewidth{0ex}
\usepackage[automark]{scrlayer-scrpage}
\clearpairofpagestyles
\cehead{
  Brian Weatherson
  }
\cohead{
  Review of “Theories of Vagueness”
  }
\ohead{\bfseries \pagemark}
\cfoot{}
\makeatletter
\newcommand*\NoIndentAfterEnv[1]{%
  \AfterEndEnvironment{#1}{\par\@afterindentfalse\@afterheading}}
\makeatother
\NoIndentAfterEnv{itemize}
\NoIndentAfterEnv{enumerate}
\NoIndentAfterEnv{description}
\NoIndentAfterEnv{quote}
\NoIndentAfterEnv{equation}
\NoIndentAfterEnv{longtable}
\NoIndentAfterEnv{abstract}
\renewenvironment{abstract}
 {\vspace{-1.25cm}
 \quotation\small\noindent\emph{Abstract}:}
 {\endquotation}
\newfontfamily\tfont{EB Garamond}
\addtokomafont{disposition}{\rmfamily}
\addtokomafont{title}{\normalfont\itshape}
\let\footnoterule\relax

\makeatletter
\renewcommand{\@maketitle}{%
  \newpage
  \null
  \vskip 2em%
  \begin{center}%
  \let \footnote \thanks
    {\itshape\huge\@title \par}%
    \vskip 0.5em%  % Reduced from default
    {\large
      \lineskip 0.3em%  % Reduced from default 0.5em
      \begin{tabular}[t]{c}%
        \@author
      \end{tabular}\par}%
    \vskip 0.5em%  % Reduced from default
    {\large \@date}%
  \end{center}%
  \par
  }
\makeatother
\RequirePackage{lettrine}

\renewenvironment{abstract}
 {\quotation\small\noindent\emph{Abstract}:}
 {\endquotation\vspace{-0.02cm}}
\KOMAoption{captions}{tableheading}
\makeatletter
\@ifpackageloaded{caption}{}{\usepackage{caption}}
\AtBeginDocument{%
\ifdefined\contentsname
  \renewcommand*\contentsname{Table of contents}
\else
  \newcommand\contentsname{Table of contents}
\fi
\ifdefined\listfigurename
  \renewcommand*\listfigurename{List of Figures}
\else
  \newcommand\listfigurename{List of Figures}
\fi
\ifdefined\listtablename
  \renewcommand*\listtablename{List of Tables}
\else
  \newcommand\listtablename{List of Tables}
\fi
\ifdefined\figurename
  \renewcommand*\figurename{Figure}
\else
  \newcommand\figurename{Figure}
\fi
\ifdefined\tablename
  \renewcommand*\tablename{Table}
\else
  \newcommand\tablename{Table}
\fi
}
\@ifpackageloaded{float}{}{\usepackage{float}}
\floatstyle{ruled}
\@ifundefined{c@chapter}{\newfloat{codelisting}{h}{lop}}{\newfloat{codelisting}{h}{lop}[chapter]}
\floatname{codelisting}{Listing}
\newcommand*\listoflistings{\listof{codelisting}{List of Listings}}
\makeatother
\makeatletter
\makeatother
\makeatletter
\@ifpackageloaded{caption}{}{\usepackage{caption}}
\@ifpackageloaded{subcaption}{}{\usepackage{subcaption}}
\makeatother
\usepackage{bookmark}
\IfFileExists{xurl.sty}{\usepackage{xurl}}{} % add URL line breaks if available
\urlstyle{same}
\hypersetup{
  pdftitle={Review of ``Theories of Vagueness''},
  pdfauthor={Brian Weatherson},
  hidelinks,
  pdfcreator={LaTeX via pandoc}}


\title{Review of ``Theories of Vagueness''}
\author{Brian Weatherson}
\date{2003}
\begin{document}
\maketitle
\begin{abstract}
Review of Rosanna Keefe, ``Theories of Vagueness''. Cambridge: Cambridge
University Press, 2000.
\end{abstract}


\setstretch{1.1}
Many philosophers, I suspect, are partial to supervaluational theories
of vagueness. And with good reason. Its rivals all seem to promise
metaphysical mysteries concerning hitherto unnoticed, and perhaps
unnoticeable, sharp boundaries around our concepts, or radical revision
in our logical practices. And not only have philosophers been so
tempted. The texts are a little unclear, but it seems several economists
can be read as adopting supervaluational solutions to the difficulties
raised by vagueness in economic concepts. Given its popularity, and
plausibility, supervaluationism deserves a book-length defence. Yet this
is the first such book in the philosophical canon.

And it is a fine defence of supervaluationism, though I doubt it is
entirely successful. I ended up, a little contrariwise, feeling less
convinced in the hegemony of supervaluational approaches than I was when
I started. In part this was because Keefe was so clear in setting out
where some rival approaches, especially degree-based approaches, failed
that I felt she had inadvertently pointed out where her opponents should
move next. Keefe's positive theory is fairly familiar, though she often
marshalls new arguments for it, so this review will not dwell on
exposition but concentrate on Keefe's arguments.

The book effectively divides into three parts. The first two chapters
involve some scene-setting, and a discussion of the various
methodological principles adopted. The next four chapters are attacks on
non-supervaluational theories. And the final two chapters defend Keefe's
preferred version of supervaluationism. Starting with methodology seemed
to be a good idea, but the discussion didn't break much new ground. It
turns out, surprisingly enough, that reflective equilibrium is useful in
theorising about vagueness too.

The first rival theory to be examined is epistemicism, and Keefe
presents two main arguments. One of these is fairly familiar,
epistemicists have no theory about how predicates get the particular
precise extensions that they do. This is true, but then
supervaluationists aren't exactly flush with theories about how
predicates get the particular imprecise extensions that they do either.
The other criticism is more interesting. Epistemicism posits a sharp
boundary between the tall and not-tall, but we don't know where this
boundary is. It is a mystery why we do not have this knowledge, one that
epistemicists try to solve by showing we could not know where the
boundary is. But there are other mysteries too. For some reason, we
don't try to find out where the boundary is, and we don't have beliefs
about where it is. That we couldn't get this kind of knowledge doesn't
explain these omissions. Like Hobbes trying to square the circle, we all
try impossible things sometimes. So epistemicism has more explaining to
do than it has hitherto done.

Keefe spends two chapters attacking theories based on degrees of truth.
There are several related objections raised to these theories, but
fundamentally they all boil down to the problem of false precision. If
there is no fact of the matter as to whether Kylie is happy is true or
not, then there is no fact of the matter as to whether it is true to
degree 0.314. The most natural formulation of degree theories assumes
there are facts of this latter form. More complicated formulations are
either incoherent or incomplete. Keefe is good at working through the
various moves that have been attempted to avoid this problem, and
showing that none of them work. But at times she seemed too content to
refute theories that had appeared in the literature, rather than
anticipating future challengers. One particular challenge seemed
particularly worthy of consideration. At one point Keefe introduces a
new connective ≥\textsubscript{T} to mean true to the same or a greater
degree. She notes that most extant degree theorists are committed to a
connectedness principle for ≥\textsubscript{T}: either
\emph{p}~≥\textsubscript{T}~\emph{q} or
\emph{q}~≥\textsubscript{T}~\emph{p}. But this principle is implausible
given their other commitments. At this point it seemed relevant to
wonder how well a theory that dropped all talk of degrees of truth and
just took this connective ≥\textsubscript{T} as primitive could avoid
Keefe's objections. Indeed, at the equivalent point in his discussion in
\emph{Vagueness}, Timothy Williamson considers some arguments against
just this position, but his discussion is rather brief. One can't reply
to every \emph{possible} response, but this one seemed so apposite, I
would have liked to see Keefe's response to it.

Keefe holds that a sentence is true iff it is true on all admissible
precisifications, i.e.~that truth is supertruth. She says little about
what makes a precisification acceptable, except that it must respect
penumbral connections, and that admissibility is a vague matter. One
consequence of this is that schema (T): S is true iff S is not always
true. Keefe suggests that the standard arguments for (T) are circular,
because they assume that there are no truth-value gaps. And, following
van Fraassen, she notes something similar to (T) is true, and this is
good enough.

She holds that an argument is valid iff it preserves truth,
i.e.~supertruth. Hence we have S~\(\dashv \vdash\)~S is true.This
interderivability might explain why we, mistakenly, think (T) is true.
There is a familiar problem with this move, one stressed by Williamson.
On all precisifications, all theorems of classical logic are true, so
these all end up being true. So to that extent supervaluationism
preserves classical logic. But not all admissible inference rules of
classical logic preserve supertruth. In particular, the deduction
theorem is no longer admissible. We can't infer (2) from (1):

\begin{enumerate}
\def\labelenumi{\arabic{enumi}.}
\tightlist
\item
  S \(\vdash\) S is true.
\item
  \(\vdash\) S ⊃ S is true.
\end{enumerate}

Keefe responds by noting that something similar to the deduction theorem
is true, and this might explain our mistaken attachment to it. Keefe
assumes the language contains an operator D, read `definitely', such
that DA means A is supertrue. Then the following rule is admissible:

\begin{quote}
(⊃I*) From A,B \(\vdash\) C, infer B \(\vdash\) DA ⊃ C
\end{quote}

We think the standard ⊃ introduction rule, the deduction theorem, is
acceptable because we mistake it for this one. Keefe notes we can set
out similar kinds of rules for argument by cases (∨-elimination) and
reductio ad absurdum (¬-introduction). These are intended to be small
deviations from classical logic, but they strip the proof theory of much
of its power. Keefe's rules are insufficient to prove p ⊃ r \(\vdash\)
(p ∧ q) ⊃ r, or p ⊃ p.~One might question the value of inference rules
with such little power.

There is little on the specific problems associated with the problem of
the many for supervaluationism. Stephen Schiffer\footnote{``Two Issues
  of Vagueness'' Monist 81: 193-214.} recently proposed a variant on the
problem, suggesting that the standard supervaluational solution
misclassifies some speech reports. Keefe replies that Schiffer's
argument doesn't seem to go through if we adopt Davidson's paratactic
theory of speech reports. Well, maybe it doesn't, but if
supervaluationism requires the paratactic theory of speech reports, that
seems highly relevant to its ultimate success, but Keefe merely assigns
it a footnote.

There is a little more on the most obvious difficulty for
supervaluationism, that it verifies some strange existentials. Consider
a Sorites series of objects arranged with respect to F-ness, each a
little more F than its predecessor, with the extremes being clearly F
and not-F respectively. Then the sentence \emph{There is a pair of
adjacent objects such that one is F and the other is not} is supertrue.
Keefe notes that we can distinguish here between the truth of the
existential and the truth of an instance. And she notes that pragmatic
theories due to Fine and Tappenden might explain our unwillingness to
assert the existential, even if it is true. Still, it is would have been
nice to see some discussion on whether this is a particular problem when
F is a phenomenal predicate, or when it is `ineradicably vague', as
Dummett and other think predicates like `sort of nearby' are.

The major innovation in the book is its treatment of higher-order
vagueness. Keefe notes the following argument raises a serious
difficulty for supervaluationism here

\begin{quote}
According to the theory, a sentence is true simpliciter iff it is true
on all complete and admissible specifications {[}i.e.~on all
precisifications{]}. But for any sentence, either it is true on all
complete and admissible specifications (hence true simpliciter) or not
(hence borderline or false). So there is no scope for avoiding sharp
boundaries to the borderline cases or for accommodating borderline
borderline cases. (202)
\end{quote}

Keefe's response is to claim that the argument assumes, falsely, that
there is ``a precise and unique set of complete and admissible
specifications.'' (202) Keefe denies this and instead develops a theory
of higher-order vagueness based on supervaluating the concept of
admissibility. But it is not clear where the argument does assume this.
After all, the argument makes no mention of sets. And it is a little
unclear just why this assumption should be false. Keefe argues that
there is no such set because complete and admissible specification is
vague, just as there is no precise and unique set of tall things because
tall is vague. But even though complete and admissible specification is
vague, on every precisification there is still a unique set of complete
and admissible specifications, so it is true that there is such a set,
so there is one. (Keefe accepts the S is true therefore S direction of
(T).)

It is also unclear how the vagueness of the term complete and admissible
specification is relevant. The supervaluational idea was that a sentence
is true iff it is true on all specifications (precisifications) that are
complete and admissible, not iff it is true on all specifications
satisfying the term complete and admissible. It is a use/mention
confusion to hold the latter view. But if the former is correct, the
vagueness of any term, even `admissible', is irrelevant to the above
argument. So there still seems to be work to do on higher-order
vagueness.

I've focussed on the negatives here, but this shouldn't overshadow how
many good parts this book has. The coverage of the literature is
peerless, the writing is always crisp and clear, and in many places
Keefe's arguments move the debate in interesting new directions. It
would be a great book to teach from, and indeed I would already be doing
so, if only it were available in paperback. I do hope the publishers
correct this problem shortly.

\vspace{1cm}



\noindent Published in\emph{
Philosophy and Phenomenological Research}, 2003, pp. 491-494.


\end{document}
