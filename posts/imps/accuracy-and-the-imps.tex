% Options for packages loaded elsewhere
\PassOptionsToPackage{unicode}{hyperref}
\PassOptionsToPackage{hyphens}{url}
%
\documentclass[
  11pt,
  letterpaper,
  DIV=11,
  numbers=noendperiod,
  twoside]{scrartcl}

\usepackage{amsmath,amssymb}
\usepackage{setspace}
\usepackage{iftex}
\ifPDFTeX
  \usepackage[T1]{fontenc}
  \usepackage[utf8]{inputenc}
  \usepackage{textcomp} % provide euro and other symbols
\else % if luatex or xetex
  \usepackage{unicode-math}
  \defaultfontfeatures{Scale=MatchLowercase}
  \defaultfontfeatures[\rmfamily]{Ligatures=TeX,Scale=1}
\fi
\usepackage{lmodern}
\ifPDFTeX\else  
    % xetex/luatex font selection
    \setmainfont[ItalicFont=EB Garamond Italic,BoldFont=EB Garamond
Bold]{EB Garamond Math}
    \setsansfont[]{EB Garamond}
  \setmathfont[]{Garamond-Math}
\fi
% Use upquote if available, for straight quotes in verbatim environments
\IfFileExists{upquote.sty}{\usepackage{upquote}}{}
\IfFileExists{microtype.sty}{% use microtype if available
  \usepackage[]{microtype}
  \UseMicrotypeSet[protrusion]{basicmath} % disable protrusion for tt fonts
}{}
\usepackage{xcolor}
\usepackage[left=1.1in, right=1in, top=0.8in, bottom=0.8in,
paperheight=9.5in, paperwidth=7in, includemp=TRUE, marginparwidth=0in,
marginparsep=0in]{geometry}
\setlength{\emergencystretch}{3em} % prevent overfull lines
\setcounter{secnumdepth}{3}
% Make \paragraph and \subparagraph free-standing
\makeatletter
\ifx\paragraph\undefined\else
  \let\oldparagraph\paragraph
  \renewcommand{\paragraph}{
    \@ifstar
      \xxxParagraphStar
      \xxxParagraphNoStar
  }
  \newcommand{\xxxParagraphStar}[1]{\oldparagraph*{#1}\mbox{}}
  \newcommand{\xxxParagraphNoStar}[1]{\oldparagraph{#1}\mbox{}}
\fi
\ifx\subparagraph\undefined\else
  \let\oldsubparagraph\subparagraph
  \renewcommand{\subparagraph}{
    \@ifstar
      \xxxSubParagraphStar
      \xxxSubParagraphNoStar
  }
  \newcommand{\xxxSubParagraphStar}[1]{\oldsubparagraph*{#1}\mbox{}}
  \newcommand{\xxxSubParagraphNoStar}[1]{\oldsubparagraph{#1}\mbox{}}
\fi
\makeatother


\providecommand{\tightlist}{%
  \setlength{\itemsep}{0pt}\setlength{\parskip}{0pt}}\usepackage{longtable,booktabs,array}
\usepackage{calc} % for calculating minipage widths
% Correct order of tables after \paragraph or \subparagraph
\usepackage{etoolbox}
\makeatletter
\patchcmd\longtable{\par}{\if@noskipsec\mbox{}\fi\par}{}{}
\makeatother
% Allow footnotes in longtable head/foot
\IfFileExists{footnotehyper.sty}{\usepackage{footnotehyper}}{\usepackage{footnote}}
\makesavenoteenv{longtable}
\usepackage{graphicx}
\makeatletter
\newsavebox\pandoc@box
\newcommand*\pandocbounded[1]{% scales image to fit in text height/width
  \sbox\pandoc@box{#1}%
  \Gscale@div\@tempa{\textheight}{\dimexpr\ht\pandoc@box+\dp\pandoc@box\relax}%
  \Gscale@div\@tempb{\linewidth}{\wd\pandoc@box}%
  \ifdim\@tempb\p@<\@tempa\p@\let\@tempa\@tempb\fi% select the smaller of both
  \ifdim\@tempa\p@<\p@\scalebox{\@tempa}{\usebox\pandoc@box}%
  \else\usebox{\pandoc@box}%
  \fi%
}
% Set default figure placement to htbp
\def\fps@figure{htbp}
\makeatother
% definitions for citeproc citations
\NewDocumentCommand\citeproctext{}{}
\NewDocumentCommand\citeproc{mm}{%
  \begingroup\def\citeproctext{#2}\cite{#1}\endgroup}
\makeatletter
 % allow citations to break across lines
 \let\@cite@ofmt\@firstofone
 % avoid brackets around text for \cite:
 \def\@biblabel#1{}
 \def\@cite#1#2{{#1\if@tempswa , #2\fi}}
\makeatother
\newlength{\cslhangindent}
\setlength{\cslhangindent}{1.5em}
\newlength{\csllabelwidth}
\setlength{\csllabelwidth}{3em}
\newenvironment{CSLReferences}[2] % #1 hanging-indent, #2 entry-spacing
 {\begin{list}{}{%
  \setlength{\itemindent}{0pt}
  \setlength{\leftmargin}{0pt}
  \setlength{\parsep}{0pt}
  % turn on hanging indent if param 1 is 1
  \ifodd #1
   \setlength{\leftmargin}{\cslhangindent}
   \setlength{\itemindent}{-1\cslhangindent}
  \fi
  % set entry spacing
  \setlength{\itemsep}{#2\baselineskip}}}
 {\end{list}}
\usepackage{calc}
\newcommand{\CSLBlock}[1]{\hfill\break\parbox[t]{\linewidth}{\strut\ignorespaces#1\strut}}
\newcommand{\CSLLeftMargin}[1]{\parbox[t]{\csllabelwidth}{\strut#1\strut}}
\newcommand{\CSLRightInline}[1]{\parbox[t]{\linewidth - \csllabelwidth}{\strut#1\strut}}
\newcommand{\CSLIndent}[1]{\hspace{\cslhangindent}#1}

\usepackage{mathrsfs}
\setlength\heavyrulewidth{0ex}
\setlength\lightrulewidth{0ex}
\usepackage[automark]{scrlayer-scrpage}
\clearpairofpagestyles
\cehead{
  James M. Joyce and Brian Weatherson
  }
\cohead{
  Accuracy and the Imps
  }
\ohead{\bfseries \pagemark}
\cfoot{}
\makeatletter
\newcommand*\NoIndentAfterEnv[1]{%
  \AfterEndEnvironment{#1}{\par\@afterindentfalse\@afterheading}}
\makeatother
\NoIndentAfterEnv{itemize}
\NoIndentAfterEnv{enumerate}
\NoIndentAfterEnv{description}
\NoIndentAfterEnv{quote}
\NoIndentAfterEnv{equation}
\NoIndentAfterEnv{longtable}
\NoIndentAfterEnv{abstract}
\renewenvironment{abstract}
 {\vspace{-1.25cm}
 \quotation\small\noindent\rule{\linewidth}{.5pt}\par\smallskip
 \noindent }
 {\par\noindent\rule{\linewidth}{.5pt}\endquotation}
\KOMAoption{captions}{tableheading}
\makeatletter
\@ifpackageloaded{caption}{}{\usepackage{caption}}
\AtBeginDocument{%
\ifdefined\contentsname
  \renewcommand*\contentsname{Table of contents}
\else
  \newcommand\contentsname{Table of contents}
\fi
\ifdefined\listfigurename
  \renewcommand*\listfigurename{List of Figures}
\else
  \newcommand\listfigurename{List of Figures}
\fi
\ifdefined\listtablename
  \renewcommand*\listtablename{List of Tables}
\else
  \newcommand\listtablename{List of Tables}
\fi
\ifdefined\figurename
  \renewcommand*\figurename{Figure}
\else
  \newcommand\figurename{Figure}
\fi
\ifdefined\tablename
  \renewcommand*\tablename{Table}
\else
  \newcommand\tablename{Table}
\fi
}
\@ifpackageloaded{float}{}{\usepackage{float}}
\floatstyle{ruled}
\@ifundefined{c@chapter}{\newfloat{codelisting}{h}{lop}}{\newfloat{codelisting}{h}{lop}[chapter]}
\floatname{codelisting}{Listing}
\newcommand*\listoflistings{\listof{codelisting}{List of Listings}}
\makeatother
\makeatletter
\makeatother
\makeatletter
\@ifpackageloaded{caption}{}{\usepackage{caption}}
\@ifpackageloaded{subcaption}{}{\usepackage{subcaption}}
\makeatother

\usepackage{bookmark}

\IfFileExists{xurl.sty}{\usepackage{xurl}}{} % add URL line breaks if available
\urlstyle{same} % disable monospaced font for URLs
\hypersetup{
  pdftitle={Accuracy and the Imps},
  pdfauthor={James M. Joyce; Brian Weatherson},
  hidelinks,
  pdfcreator={LaTeX via pandoc}}


\title{Accuracy and the Imps\thanks{Thanks to Alejandro Pérez Carballo,
Richard Pettigrew, and the participants in the Arché Epistemology
Seminar for helpful comments.}}
\author{James M. Joyce \and Brian Weatherson}
\date{2019}

\begin{document}
\maketitle
\begin{abstract}
Recently several authors have argued that accuracy-first epistemology
ends up licensing problematic epistemic bribes. They charge that it is
better, given the accuracy-first approach, to deliberately form one
false belief if this will lead to forming many other true beliefs. We
argue that this is not a consequence of the accuracy-first view. If one
forms one false belief and a number of other true beliefs, then one is
committed to many other false propositions, e.g., the conjunction of
that false belief with any of the true beliefs. Once we properly account
for all the falsehoods that are adopted by the person who takes the
bribe, it turns out that the bribe does not increase accuracy.
\end{abstract}


\setstretch{1.1}
\section{Accuracy, Bribes and Scoring
Rules}\label{accuracybribesandscoringrules}

Belief aims at the truth. So at least in some sense, an agent is doing
better at believing the closer they are to the truth. When applied to
individual beliefs, this generates epistemic advice that is literally
platitudinous: if you know that a change in your attitude towards
\emph{p} will make your attitude towards \emph{p} more accurate, make
that change! When applied to collective bodies of belief though, the
advice turns out to be more contentious. Call \textbf{epistemic
consequentialism} the view that if an agent knows that a change in their
overall belief state will make their belief state more accurate, they
should make that change, if they have the power to do so.

Hilary Greaves (\citeproc{ref-Greaves2013}{2013}) has recently argued
that epistemic consequentialism is false because it licences certain
epistemic `bribes', and these should not be licenced. We'll argue that
the best forms of epistemic consequentialism do not licence some of
these bribes after all.\footnote{Though they do licence others; see
  section 2.4 for more discussion.} Here is the key case Greaves
uses.\footnote{Greaves has four other cases, but the Imps case is the
  only one that is a problem for all forms of consequentialism she
  discusses. Similar cases have suggested by Selim Berker
  (\citeproc{ref-Berker2013b}{2013a}, \citeproc{ref-Berker2013a}{2013b})
  and C. S. Jenkins (\citeproc{ref-Jenkins2007}{2007}), but we'll focus
  on Greaves's discussion since she engages more fully with the
  literature on scoring rules. We'll return briefly to Berker's
  discussion in section 2.}

\begin{quote}
Emily is taking a walk through the Garden of Epistemic Imps. A child
plays on the grass in front of her. In a nearby summerhouse are \emph{n}
further children, each of whom may or may not come out to play in a
minute. They are able to read Emily's mind, and their algorithm for
deciding whether to play outdoors is as follows. If she forms degree of
belief 0 that there is now a child before her, they will come out to
play. If she forms degree of belief 1 that there is a child before her,
they will roll a fair die, and come out to play iff the outcome is an
even number. More generally, the summerhouse children will play with
chance (1-\emph{q}(\emph{C}\textsubscript{0})/2), where
\emph{q}(\emph{C}\textsubscript{0}) is the degree of belief Emily adopts
in the proposition \emph{C}\textsubscript{0} that there is now a child
before her. Emily's epistemic decision is the choice of credences in the
proposition \emph{C}\textsubscript{0} that there is now a child before
her, and, for each \emph{j}~=~1, \ldots, \emph{n} the proposition
\emph{C\textsubscript{j}} that the \emph{j}th summerhouse child will be
outdoors in a few minutes' time.
\end{quote}

\begin{quote}
\ldots{} if Emily can just persuade herself to ignore her evidence for
\emph{C}\textsubscript{0}, and adopt (at the other extreme) credence 0
in \emph{C}\textsubscript{0}, then, by adopting degree of belief 1 in
each \emph{C\textsubscript{j}} (\emph{j}~=~1, \ldots{} , 10), she can
guarantee a perfect match to the remaining truths. Is it epistemically
rational to accept this `epistemic bribe'? Greaves
(\citeproc{ref-Greaves2013}{2013, 918})
\end{quote}

The epistemic consequentialist says that it is best to have credences
that are as accurate as possible. We will focus on believers who assign
probabilistically coherent credences (degrees of belief) to the
propositions in some ``target set'' \(\mathscr{X}\), and we will think
of the ``degree of fit'' between her beliefs and the truth as being
measured by a strictly proper scoring rule. This is a function
\textbf{I}\textsubscript{\(\mathscr{X}\)} which associates each pair
\(\langle\)\textbf{cred},~@\(\rangle\) consisting of a credence function
\textbf{cred} whose domain includes \(\mathscr{X}\) and a consistent
truth-value assignment @ for elements of \(\mathscr{X}\) with a
non-negative real number
\textbf{I}\textasciitilde{}\(\mathscr{X}\)(@,~\textbf{cred}).
Intuitively, \textbf{I}\textsubscript{\(\mathscr{X}\)} measures the
inaccuracy of the credences that cred assigns to the propositions in
\(\mathscr{X}\) when their truth-values are as described by @. Note that
higher \textbf{I}\textsubscript{\(\mathscr{X}\)}-values indicate higher
levels of epistemic disutility, so that lower is better from a
consequentialist perspective. One popular scoring rule is the Brier
score, which identifies inaccuracy with the average squared distance
between credences and truth-values. (Greaves calls this the `quadratic
scoring rule', which is a useful description too.) More formally, we
have:

\[
\mathbf{Brier}_{\mathscr{X}}(@, \mathbf{cred}) = \frac{1}{|\mathscr{X}|}\sum_{X \in \mathscr{X}} (\mathbf{cred}(X) - @(X))^2
\]

where \(|\mathscr{X}|\) is the number of propositions in \(\mathscr{X}\)
and @(\emph{X}) is either zero or one depending upon whether X is true
or false.

Another common score is the logarithmic rule, which defines inaccuracy
as:

\[
\mathbf{Log}_{\mathscr{X}}(@, \mathbf{cred}) = \frac{1}{|\mathscr{X}|}\sum_{X \in \mathscr{X}} -\text{log}(\mathbf{cred}(X)) \cdot @(X)
\]

For now we will follow Greaves in assuming that our epistemic
consequentialist uses the Brier score to measure epistemic disutility,
but we will relax that assumption in a little while.

Now let's think about the `bribe' that Greaves offers, from the point of
view of the epistemic consequentialist. The choices are to have one of
two credal states, which we'll call \textbf{cred\textsubscript{1}} and
\textbf{cred\textsubscript{2}}. We'll say \textbf{cred\textsubscript{1}}
is the one that best tracks the initial evidence, so
\textbf{cred\textsubscript{1}}(\emph{C}\textsubscript{0})~=~1, and
\textbf{cred\textsubscript{1}}(\emph{C\textsubscript{i}})~=~0.5 for
\emph{i}~∈~\{1, \ldots, 10\}. And \textbf{cred\textsubscript{2}} is the
credence Emily adopts if she accepts the bribe, so
\textbf{cred\textsubscript{2}}(\emph{C}\textsubscript{0})~=~0, while
\textbf{cred\textsubscript{2}}(\emph{C\textsubscript{i}})~=~1 for
\emph{i}~∈~\{1, \ldots, 10\}. Which state is better?

Thinking like an epistemic consequentialist, you might ask which state
is more accurate? It seems like that would be
\textbf{cred\textsubscript{2}}. While \textbf{cred\textsubscript{1}}
gets \emph{C}\textsubscript{0} exactly right it does not do very well on
the other propositions. In contrast, while
\textbf{cred\textsubscript{2}} gets \emph{C}\textsubscript{0} exactly
wrong, it is perfect on the other ten propositions. So overall,
\textbf{cred\textsubscript{2}} looks to have better epistemic
consequences: when compared to being right about one proposition and off
by 0.5 on ten others, being right on ten is surely worth one false
belief. The Brier score seems to bear this out. If we let
\(\mathscr{X}\), the target set, consist of \emph{C}\textsubscript{0},
\emph{C}\textsubscript{1}, \ldots, \emph{C}\textsubscript{10}, then we
have

\[
\begin{aligned}
\mathbf{Brier}_\mathscr{X}(\mathbf{cred_1}, @) &= \frac{1}{11}[(1-\mathbf{cred_1}(C_0))^2 + \sum_{i = 1}^{10} (@(C_i) - \frac{1}{2}) ^2] = \frac{10}{44} \\
\mathbf{Brier}_\mathscr{X}(\mathbf{cred_2}, @) &= \frac{1}{11}[(1-\mathbf{cred_2}(C_0))^2 + \sum_{i = 1}^{10} (@(C_i) - \mathbf{cred_2}(C_i)) ^2] = \frac{1}{11} 
\end{aligned}
\]

So, it seems that a good epistemic consequentialist will take the bribe.
But, doesn't that seem like the height of epistemic irresponsibility? It
means choosing to believe that \emph{C}\textsubscript{0} is certainly
false when you have conclusive evidence for thinking that it is true. If
you see the child on the lawn in front of you, how can you sanction
believing she is not there?

As Greaves admits, intuitions are divided here. Some consequentialists
might think that ``epistemic bribes'' are at least sometimes worth
taking, while those of a more deontological bent will always find such
trade-offs ``beyond the pale'' ~(\citeproc{ref-Berker2013b}{Berker
2013a, 363}). We will largely sidestep these contentious issues here,
though our argument will offer comfort to epistemic consequentialists
who feel queasy about accepting the bribe offered in Imps. We contend
that, when inaccuracy is measured properly, the consequences of adopting
the \textbf{cred\textsubscript{2}} credences are strictly worse than the
consequences of adopting \textbf{cred\textsubscript{1}}.

The basic problem is that Imps cherry-picks propositions in a way no
consequentialist should condone. Its persuasive force rests on the
assumption that, for purposes of epistemic evaluation, nothing matters
except the accuracies of the credences assigned to propositions in the
target set \(\mathscr{X}\). But \(\mathscr{X}\) is the wrong target! By
confining attention to it Greaves ignores the many other credences to
which Emily becomes committed as a consequence of adopting
\textbf{cred\textsubscript{1}} or \textbf{cred\textsubscript{2}}. Any
(coherent) agent who invests credence zero in \emph{C}\textsubscript{0}
must also invest credence zero in any proposition
\emph{C}\textsubscript{0}~∧~\emph{Y}, where \emph{Y} is any conjunction
or disjunction of elements from \(\mathscr{X}\). Likewise, anyone who
invests credence one in \emph{C\textsubscript{n}} must invest credence
one in any proposition \emph{C\textsubscript{n}}~∨~\emph{Y}, where
\emph{Y} is any conjunction or disjunction from \(\mathscr{X}\). In the
current context (where the probabilities of the various
\emph{C\textsubscript{i}} are independent), when Emily adopts a credence
function over \(\mathscr{X}\) she commits to having a credence for (i)
every atomic proposition
±\emph{C}\textsubscript{0}~∧~±\emph{C}\textsubscript{1}~∧~±\emph{C}\textsubscript{2}~∧~\ldots~∧~±\emph{C}\textsubscript{10},
where `±' can be either an affirmation or a negation, and (ii) every
disjunction of these atomic propositions. In short, she commits to
having credences over the whole Boolean algebra
\(\mathscr{A}_\mathscr{X}\) generated by \(\mathscr{X}\). Since each
event of a child coming out is independent, adopting
\textbf{cred\textsubscript{1}} will commit her to setting
\textbf{cred\textsubscript{1}}(±\emph{C}\textsubscript{0}~∧~±\emph{C}\textsubscript{1}~∧~±\emph{C}\textsubscript{2}~∧~\ldots~∧~±\emph{C}\textsubscript{10})
= \(\frac{1}{1024}\) when \emph{C}\textsubscript{0} is affirmed, and 0
when it is negated. While adopting \textbf{cred\textsubscript{2}}
commits her to setting
\textbf{cred\textsubscript{2}}(±\emph{C}\textsubscript{0}~∧~±\emph{C}\textsubscript{1}~∧~±\emph{C}\textsubscript{2}~∧~\ldots~∧~±\emph{C}\textsubscript{10})
equal to 1 when \emph{C}\textsubscript{0} is negated and the rest of the
\textbf{C\textsubscript{i}} are affirmed, and to 0 otherwise. In this
way, each of these probability assignments over the 2048 atoms determine
a definite probability for every one of the 2\textsuperscript{2048}
propositions in \(\mathscr{A}_\mathscr{X}\).

It is our view that consequentialists should reject any assessment of
epistemic utility that fails to take the accuracies of \emph{all} these
credences into account. All are consequences of adopting
\textbf{cred\textsubscript{1}} or \textbf{cred\textsubscript{2}}, and so
all should be part of any consequentialist evaluation of the quality of
those credal states. The right ``target set'' to use when computing
epistemic disutility is not \(\mathscr{X}\) but
\(\mathscr{A}_\mathscr{X}\). If we don't do that, we ignore most of the
ways in which \textbf{cred\textsubscript{1}} and
\textbf{cred\textsubscript{2}} differ in accuracy. If Emily takes the
bribe, she goes from having credence 0.5 in
\emph{C}\textsubscript{0}~↔~C\textsubscript{1} to having credence 0 in
it. And that's unfortunate, because the chance of
\emph{C}\textsubscript{0}~↔~C\textsubscript{1} goes from 0.5 to 1. This
is another proposition, as well as \emph{C}\textsubscript{0}, that Emily
acquires a false belief in by taking the bribe. Of course, there are
other propositions not counted that go the other way. Originally, Emily
has a credence of 0.25 in
\emph{C}\textsubscript{1}~∧~C\textsubscript{2}, and its chance is also
0.25. After taking the bribe, this has a chance of 1, and her credence
in it is 1. That's an improvement in accuracy. So there are a host of
both improvements and deteriorations that are as yet unaccounted for. We
should account for them, and making the target set be
\(\mathscr{A}_\mathscr{X}\) does that.

When seen from this broader perspective, it turns out the seeming
superiority of \textbf{cred\textsubscript{2}} over
\textbf{cred\textsubscript{1}} evaporates. The rest of this section (and
the appendix) is dedicated to demonstrating this. We'll make the
calculations a little easier on ourselves by relying on a theorem
concerning Brier scores for coherent agents. Assume, as is the case
here, that Emily's credences are defined over an atomic Boolean algebra
of propositions. The atoms are the `worlds', or states that are
maximally specific with respect to the puzzle at hand. In this case
there are 2048 states, which we'll label \emph{s}\textsubscript{0}
through \emph{s}\textsubscript{2047}. In \emph{s\textsubscript{k}}, the
first child is on the lawn iff \emph{k}~⩽~1023, and summerhouse child
\emph{i} comes out iff the (\emph{i}~+~1)th digit in the binary
expansion of \emph{k} is 1. Let \(\mathscr{S}_\mathscr{X}\) be the set
of all these states. That's not a terrible target set; as long as Emily
is probabilistically coherent it is comprehensive. The theorem in
question says that for any credence function \textbf{cred} defined over
a partition of states \(\mathscr{S}\), and over the algebra
\(\mathscr{A}\) generated by those states,

\begin{description}
\item[Theorem-1]
\textbf{Brier}\textsubscript{\(\mathscr{A}\)}(\textbf{cred}, @) =
\(\frac{|\mathscr{S}|}{4}\)\textbf{Brier}\textsubscript{\(\mathscr{S}\)}(\textbf{cred},
@)
\end{description}

(The proof of this is in the appendix.) So whichever credence function
is more accurate with respect to \(\mathscr{S}_{\mathscr{X}}\) will be
more accurate with respect to \(\mathscr{A}_{\mathscr{X}}\). So let's
just work out
\textbf{Brier}\textsubscript{\(\mathscr{S}_{\mathscr{X}}\)} for
\textbf{cred\textsubscript{1}} and \textbf{cred\textsubscript{2}} at the
actual world.

First, \textbf{cred\textsubscript{1}} will appropriately assign credence
0 to each \emph{s\textsubscript{k}} (\emph{k}~∈~\{0,~\ldots,~1023\}).
Then it assigns credence 1/1024 to every other
\emph{s\textsubscript{k}}. For 1023 of these, that is off by 1/1024,
contributing 1/2\textsuperscript{20} to the Brier score. And for 1 of
them, namely @, it is off by 1023/1024, contributing
\(\frac{1023^2}{2^{20}}\). So we get:

\[
\begin{aligned}
\mathbf{Brier}_{\mathscr{S}_{\mathscr{X}}}(\mathbf{cred_1}, @) &= \frac{1}{2048} [1024 \cdot 0 + 1023 \cdot \frac{1}{2^{20}} + \frac{1023^2}{2^{20}}] \\
&= \frac{1}{2048} \cdot \frac{1023 + 1023 ^2}{2^{20}} \\
&= \frac{1}{2048} \cdot \frac{1023 \cdot 1024}{2^{20}} \\
&= \frac{1}{2048} \cdot \frac{1023}{1024} \\
&= \frac{2^{10}-1}{2^{21}}
\end{aligned}
\]

It's a bit easier to work out
\textbf{Brier}\textsubscript{\(\mathscr{S}_{\mathscr{X}}\)}(\textbf{cred\textsubscript{2}},~\emph{s}\textsubscript{2047}).
(We only need to work out the Brier score for that state, because by the
setup of the problem, Emily knows that's the state she'll be in if she
adopts \textbf{cred\textsubscript{2}}). There are 2048 elements in
\(\mathscr{S}_{\mathscr{X}}\). And \textbf{cred\textsubscript{2}}
assigns the perfectly accurate credence to 2046 of them, and is
perfectly inaccurate on 2, namely \emph{s}\textsubscript{1023}, which it
assigns credence 1, and \emph{s}\textsubscript{2047} which it assigns
credence 0. So we have

\[
\begin{aligned}
\mathbf{Brier}_{\mathscr{S}_{\mathscr{X}}}(\mathbf{cred_2}, s_{2047}) &= \frac{1}{2048} (2046 \cdot 0 + 1 + 1) \\
&= \frac{1}{1024} \\
&= \frac{2^{11}}{2^{21}}
\end{aligned}
\]

In fact, it isn't even close. If Emily adopts
\textbf{cred\textsubscript{2}} she becomes a little more than twice as
inaccurate.

It is tedious to calculate
\textbf{Brier}\textsubscript{\(\mathscr{A}_{\mathscr{X}}\)}(\textbf{cred\textsubscript{1}},
@) directly, but it is enlightening to work through the calculation of
\textbf{Brier}\textsubscript{\(\mathscr{A}_{\mathscr{X}}\)}(\textbf{cred\textsubscript{2}},
\emph{s}\textsubscript{2047}). Note that there are two crucial states
out of the 2048: \emph{s}\textsubscript{2047}, the actual state where
all children come out, and state \emph{s}\textsubscript{1023} where
child 0 does not come out, but the other 10 children all do. There are
\(2^{2^{11}-2}\) propositions in each of the following four sets:

\begin{enumerate}
\def\labelenumi{\arabic{enumi}.}
\tightlist
\item
  \(\{p: s_{2047} \vDash p\) and \(s_{1023} \vDash p\}\)
\item
  \(\{p: s_{2047} \vDash p\) and \(s_{1023} \nvDash p\}\)
\item
  \(\{p: s_{2047} \nvDash p\) and \(s_{1023} \vDash p\}\)
\item
  \(\{p: s_{2047} \nvDash p\) and \(s_{1023} \nvDash p\}\)
\end{enumerate}

If Emily takes the bribe, she will have perfect accuracy with respect to
all the propositions in class 1 (which are correctly believed to be
true), and all the propositions in class 4 (which are correctly believed
to be false). But she will be perfectly inaccurate with respect to all
the propositions in class 2 (which are incorrectly believed to be
false), and all the propositions in class 3 (which are incorrectly
believed to be true). So she is perfectly accurate on half the
propositions, and perfectly inaccurate on half of them, so one's average
inaccuracy is 0.5~·~0 + 0.5~·~1 = 0.5. And that's an enormous
inaccuracy. It is, in fact, as inaccurate as one can possibly be while
maintaining probabilistic coherence.

\begin{description}
\tightlist
\item[Theorem-2]
When inaccuracy over \(\mathscr{A}\) is measured using the Brier score,
the least accurate credal states are those which assign credence 1 to
some false atom of \(\mathscr{A}\).
\end{description}

(The proof is in the appendix.) So taking the bribe is not a good deal,
even by consequentialist lights. And that isn't too surprising; taking
the bribe makes Emily have maximally inaccurate credences on half of the
possible propositions about the children.

So far we have followed Greaves in assuming that inaccuracy is measured
by the quadratic, or Brier, rule. It turns out that we can drop that
assumption. We actually only need some very weak conditions on accuracy
rules to get the result that Greaves style bribes are bad deals, though
the proof of this becomes a trifle more complicated.

Let \(\mathscr{A}\) be an algebra of propositions generated by a
partition of 2\emph{N} atoms \emph{a}\textsubscript{1}, \ldots,
a\textsubscript{2\emph{N}}. Suppose \emph{a}\textsubscript{1} is the
truth, and consider two probability functions, \emph{P} and \emph{Q}
defined in \(\mathscr{A}\). \emph{P} assigns all its mass to the first
\emph{N} atoms, so that \(P(a_k) = 0\) for all
\emph{k}~\textgreater~\emph{N}. We also assume that \emph{P} assigns
some positive probability to the true atom \emph{a}\textsubscript{1}.
\emph{Q} assigns all its mass to the false atom
\emph{a}\textsubscript{2\emph{N}}. Note that this will be a good model
of any case where an agent is offered a bribe of the form: drop the
positive confidence you have in proposition \emph{p}\textsubscript{0},
instead assign it credence 0, and you'll be guaranteed a maximally
accurate credence in \emph{j} other logically independent propositions
\emph{p}\textsubscript{1},~\ldots,~\emph{p\textsubscript{j}}. The only
other assumptions needed to get the model to work are that
\emph{p}\textsubscript{0} is actually true, and
\emph{N}~=~2\textsuperscript{\emph{j}}.

Imagine that the accuracy of a probability function π over
\(\mathscr{A}\) is measured by a proper scoring rule of the form

\[
\mathbf{I}(a_n, \pi) = 2^{-2N}\sum_{X \in \mathscr{A}} \mathbf{i}(v_n(X), \pi(X))
\]

where \(v_n(X)\) is \emph{X}s truth value when \emph{a\_n} is the true
atom, and \textbf{i} is a score that gives the accuracy of π(\emph{X})
in the event that \emph{X}s truth value is \(v_n(X)\). We shall assume
that this score has the following properties.

\begin{description}
\item[Truth Directedness]
The value of \textbf{i}(1,~\emph{p}) decreases monotonically as \emph{p}
increases. The value of \textbf{i}(0,~\emph{p}) increases monotonically
as \emph{p} decreases.
\item[Extensionality]
\(\mathbf{i}(v_n(X), \pi(X))\) is a function only of the truth-value and
the probability; the identity of the proposition does not matter.
\item[Negation Symmetry]
\(\mathbf{i}(v_n(¬ X), \pi(¬ X)) = \mathbf{i}(v_n(X), \pi(X))\) for all
\emph{x}, \emph{n}, π.
\item[Theorem-3]
Given these assumptions, \emph{P}'s accuracy strictly exceeds
\emph{Q}'s.
\end{description}

Again, the proof is in the appendix.

Theorem-3 ensures that taking the deal that Greaves offers in Imps will
reduce Emily's accuracy relative to any proper scoring rule satisfying
Truth Directedness, Extensionality and Negation Symmetry. To see why,
think of Emily's credences as being defined over an algebra generated by
the atoms
±\emph{C}\textsubscript{0}~∧~±\emph{C}\textsubscript{1}~∧~±\emph{C}\textsubscript{2}~∧~\ldots~∧~±\emph{C}\textsubscript{10},
where \emph{C}\textsubscript{0} is understood to be true. Since Emily is
convinced of \emph{C}\textsubscript{0} and believes that every other
\emph{C\textsubscript{n}} has some chance of occurring, and since the
various \emph{C\textsubscript{n}} are independent of one another, her
credence function \textbf{cred\textsubscript{1}} will assigns a positive
probability to each atom, including the true atom (whichever that might
be). Now, let \emph{Q} be a credence function that places all its weight
on some false atom
¬\emph{C}\textsubscript{0}~∧~±\emph{C}\textsubscript{1}~∧~±\emph{C}\textsubscript{2}~∧~\ldots~∧~±\emph{C}\textsubscript{10}.
Theorem-3 tells us that Emily's \textbf{cred\textsubscript{1}} is more
accurate than \emph{Q}, and that this is true no matter which
\emph{C}\textsubscript{0} atom is true or which
¬\emph{C}\textsubscript{0} atom \emph{Q} regards as certain. By taking
the bribe Emily will guarantee the truth of
\emph{C}\textsubscript{0}~∧~C\textsubscript{1}~∧~\ldots~∧~\emph{C}\textsubscript{10},
but the cost will be that she must adopt the
\textbf{cred\textsubscript{2}} credences, which assign probability one
to the false atom ¬
\emph{C}\textsubscript{0}~∧~C\textsubscript{1}~∧~\ldots~∧~\emph{C}\textsubscript{10}.
Extensionality ensures that any two credence functions that assign
probability one to a false atom will have the same inaccuracy score, and
that this score will not depend on which atom happens to be the true
one. The upshot is that \textbf{cred\textsubscript{2}} will have the
same inaccuracy when Emily accepts the bribe as \emph{Q} does when she
rejects it. Thus, since \textbf{cred\textsubscript{1}} is more accurate
than \emph{Q}, it is also more accurate than
\textbf{cred\textsubscript{2}}, which means that Emily should reject the
bribe in order to promote credal accuracy.

We do not want to oversell this conclusion. Strictly speaking, we have
only shown that consequentialists should reject epistemic bribes when
doing so requires them to go from being confident in a truth to being
certain of some maximally specific falsehood. This is a rather special
situation, and there are nearby cases to which our results do not apply,
and in which consequentialists may sanction bribe-taking. For example,
if Emily only has to cut her credence for \emph{C}\textsubscript{0} in
half, say from ½ to ¼, to secure knowledge of
\emph{C}\textsubscript{1}~∧~\ldots~∧~\emph{C}\textsubscript{10}, then
Theorem-3 offers us no useful advice. Indeed, depending on the scoring
rule and the nature of the bribe, we suspect that believers will often
be able to improve accuracy by changing their credences in ways not
supported by their evidence, especially when these changes affect the
truth-values of believed propositions. The only thing we insist upon is
that, in all such cases, credal accuracy should be measured over all
relevant propositions, not just over a select salient few. But that's
something that is independently plausible. Perhaps it might be
pragmatically justified to become more accurate on salient propositions
at the expense of becoming very inaccurate over hard to state compounds
of those propositions, but it is never epistemically justified.

\section{Four Caveats}\label{fourcaveats}

\subsection{Greaves's Imps Argument May Work Against Some Forms of
Consequentialism}\label{greavessargumentmayworkagainstsomeformsofconsequentialism}

We said above that no consequentialist should accept Greaves's setup of
the Imps puzzle, since they should not accept an inaccuracy measure that
ignores some kind of introduced inaccuracy. That means that, for all we
have said, Greaves's argument works against those consequentialists who
do not agree with us over the suitability of target sets that are
neither algebras or partitions. And, at least outside philosophy, some
theorists do seem to disagree with us.

For instance, it is common in meteorology to find theorists who measure
the accuracy of rain forecasts over an \emph{n} day period by just
looking at the square of the distance between the probability of rain
and the truth about rain on each day. To pick an example almost
literally at random, Mark Roulston (\citeproc{ref-Roulston2007}{2007})
defends the use of the Brier score, calculated just this way, as a
measure of forecast accuracy. So Greaves's target, while not including
all consequentialists, does include many real theorists.

That said, it seems there are more mundane reasons to not like this
approach to measuring the accuracy of weather forecasts. Consider this
simple case. Ankita and Bojan are issuing forecasts for the week that
include probabilities of rain. They each think that there is a 0\%
chance of rain most days. But Ankita thinks there will be one short
storm come through during the week, while Bojan issues a 0\% chance of
rain forecast for each day. Ankita thinks the storm is 75\% likely to
come on Wednesday, so there's a 75\% chance of rain that day, and 25\%
likely to come Thursday, so there's a 25\% chance of rain that day.

As it happens, the storm comes on Thursday. So over the course of the
week, Bojan's forecast is more accurate than Ankita's. Bojan is
perfectly accurate on 6 days, and off by 1 on Thursday. Ankita is
perfectly accurate on 5 days, and gets an inaccuracy score of
0.75\textsuperscript{2} = 0.5625 on Wednesday and Thursday, which adds
up to more than Bojan's inaccuracy. But this feels wrong. There is a
crucial question that Ankita was right about and Bojan was wrong about,
namely will there be a storm in the middle of the week. Ankita's
forecast only looks less accurate because we aren't measuring accuracy
with respect to this question. So even when we aren't concerned with
magical cases like Greaves's, there is a good reason to measure accuracy
comprehensively, i.e., with respect to an algebra or a partition.

\subsection{Separateness of
Propositions}\label{separatenessofpropositions}

There is a stronger version of the intuition behind the Imps case that
we simply reject. The intuition is well expressed by Selim Berker
(\citeproc{ref-Berker2013b}{2013a, 365}, emphasis in original)

\begin{quote}
The more general point is this: when determining the epistemic status of
a belief in a given proposition, it is epistemically irrelevant whether
or not that belief conduces (either directly or indirectly) toward the
promotion of true belief and the avoidance of false belief in
\emph{other} propositions beyond the one in question.
\end{quote}

Let's put that to the test by developing the Ankita and Bojan story a
little further. They have decided to include, in the next week's
forecast, a judgment on the credibility of rain. Bojan thinks the
evidence is rather patchy. And he has been reading Glenn Shafer
(\citeproc{ref-Shafer1976}{1976}), and thinks that when the evidence is
patchy, credences in propositions and their negations need not add to 1.
So if \emph{p} is the proposition \emph{It will rain next week}, Bojan
has a credence of 0.4 in both \emph{p} and ¬\emph{p}.

Ankita thinks that's crazy, and suggests that there must be something
deeply wrong with the Shafer-based theory that Bojan is using. But Bojan
is able to easily show that the common arguments against Shafer's theory
are blatantly question begging ~(\citeproc{ref-Maher1997}{Maher 1997};
\citeproc{ref-Weatherson1999}{Weatherson 1999}). So Ankita tries a new
tack. She has been reading Joyce (\citeproc{ref-Joyce1998}{1998}), from
which she got the following idea. She argues that Bojan will be better
off from the point of view of accuracy in having credence 0.5 in each of
\emph{p} and \emph{¬ p} than in having credence 0.4 in each. As it
stands, one of Bojan's credences will be off by 0.4, and the other by
0.6, for a Brier score of
(0.4\textsuperscript{2}~+~0.6\textsuperscript{2})/2~=~0.26, whereas
switching would give him a Brier score of
(0.5\textsuperscript{2}~+~0.5\textsuperscript{2})/2~=~0.25.

But Bojan resists. He offers two arguments in reply.

First, he says, for all Ankita knows, one of his credences might be best
responsive to the evidence. And it is wrong, always and everywhere, to
change a credence away from one that is best supported by the evidence
in order to facilitate an improvement in global accuracy. That, says
Bojan, is a violation of the ``separateness of
propositions''~(\citeproc{ref-Berker2013b}{Berker 2013a}).

Second, he says, even by Ankita's accuracy-based lights, this is a bad
idea. After all, he will be making one of his credences less accurate in
order to make an improvement in global accuracy. And that's again a
violation of the separateness of propositions. It's true that he won't
be making himself more inaccurate in one respect so as to secure
accuracy in another, as in the bribes case. But he will be following
advice that is motivated by the aim of becoming, in total, more
accurate, at the expense of accuracy for some beliefs.

We want to make two points in response. First, if the general point that
Berker offers is correct, then these are perfectly sound replies by
Bojan. Although Bojan is not literally in a bribe case, like Emily, he
is being advised to change some credences because the change will make
his overall credal state better, even if it makes it locally worse in
one place. It does not seem to matter whether he can identify which
credence gets made worse. Berker argues that the trade offs that
epistemic consequentialism makes the same mistake ethical
consequentialism makes; it authorises inappropriate trade-offs. But in
the ethical case, it doesn't matter whether the agent can identify who
is harmed by the trade-off. If it is wrong to harm an identifiable
person for the greater good, it is wrong to harm whoever satisfies some
description in order to produce the greater good.

So if the analogy with anti-consequentialism in ethics goes through,
Bojan is justified in rejecting Ankita's advice. After all there is,
according to Berker, a rule against making oneself doxastically worse in
one spot for the gain of an overall improvement. And that's what Bojan
would do if he took Ankita's advice. But, we say, Bojan is not justified
in rejecting Ankita's advice. In fact, Ankita's advice is sound advice,
and Bojan would do well to take it. So Berker's general point is wrong.

Our second point is a little more contentious. We suspect that if Bojan
has a good reason to resist this move of Ankita's, he has good reason to
resist all attacks on his Shafer-based position. So if Berker's general
point is right, it means there is nothing wrong with Bojan's
anti-probabilist position. Now we haven't argued for this; to do so
would require going through all the arguments for probabilism and seeing
whether they can be made consistent with Berker's general point. But our
suspicion is that none of them can be, since they are all arguments that
turn on undesirable properties of global features of non-probabilistic
credal states. So if Berker is right, probabilism is wrong, and we think
it is not wrong.

\subsection{Is this Consequentialism?}\label{isthisconsequentialism}

So far we've acquiesed with the general idea that Greaves's and Berker's
target should be called \emph{consequentialism}. But there are reasons
to be unhappy with this label. In general, a consequentialist theory
allows agents to make things worse in the here and now, in return for
future gains. A consequentialist about prudential decision making, in
the sense of Hammond (\citeproc{ref-Hammond1988}{1988}), will recommend
exercise and medicine taking. And they won't be moved by the fact that
the exercise hurts and the medicine is foul-tasting. It is worth
sacrificing the welfare of the present self for the greater welfare of
later selves.

Nothing like that is endorsed, as far as we can tell, by any of the
existing `epistemic consequentialists'. Certainly the argument that
Ankita offers Bojan does not rely on this kind of reasoning. In
particular, epistemic consequentialists do not say that it is better to
make oneself doxastically worse off now in exchange for greater goods
later. Something like that deal is offered to the reader of Descartes
(\citeproc{ref-DescartesMeditations}{1641/1996}), but it isn't as
popular nowadays.

Rather, the rule that is endorsed is \emph{Right now, have the credences
that best track the truth!} This isn't clearly a form of
consequentialism, since it really doesn't care about the
\emph{consequences} of one's beliefs. It does say that it is fine to
make parts of one's doxastic state worse in order to make the whole
better. That's what would happen if Bojan accepted Ankita's advice. But
that's very different from doing painful exercise, or drinking
unpleasant medicine. (Or, for that matter, to withdrawing belief in any
number of truths.)

When Greaves tries to flesh out epistemic consequentialism, she compares
it to evidential and causal versions of prudential decision theory. But
it seems like the right comparison might be to something we could call
\emph{constitutive} decision theory. The core rule, remember, is that
agents should form credences that constitute being maximally accurate,
not that cause them to be maximally accurate.

The key point here is not the terminological one about who should be
called consequentialist. Rather, it is that the distinction between
causation and constitution is very significant here, and comparing
epistemic utility theory to prudential utility theory can easily cause
it to be lost. Put another way, we have no interest in defending someone
who wants to defend a causal version of epistemic utility theory, and
hence thinks it could be epistemically rational to be deliberately
inaccurate now in order to be much more accurate tomorrow. We do want to
defend the view that overall accuracy right now is a prime epistemic
goal.

\subsection{Other Bribes}\label{otherbribes}

As already noted, we have not offered a general purpose response to
bribery based objections to epistemic consequentialism. All we've shown
is that some popular examples of this form of objection misfire, because
they offer bribes that are bad by the consequentialists' own lights. But
there could be bribes that are immune to our objection.

For example, imagine that Ankita has, right now, with credence 0.9 in
\emph{D}\textsubscript{0}, and 0.5 in \emph{D}\textsubscript{1}. These
are good credences to have, since she knows those are the chances of
\emph{D}\textsubscript{0} and \emph{D}\textsubscript{1}. She's then
offered an epistemic bribe. If she changes her credence in
\emph{D}\textsubscript{0} to 0.91, the chance of
\emph{D}\textsubscript{1} will become 1, and she can have credence 1 in
\emph{D}\textsubscript{1}. Taking this bribe will increase her accuracy.

We could imagine the anti-consequentialist arguing as follows.

\begin{enumerate}
\def\labelenumi{\arabic{enumi}.}
\tightlist
\item
  If epistemic consequentialism is true, Ankita is epistemically
  justified in accepting this bribe.
\item
  Ankita is not epistemically justified in accepting this bribe.
\item
  So, epistemic consequentialism is not true.
\end{enumerate}

We're not going to offer a reply to this argument here; that is a task
for a much longer paper. There are some reasons to resist premise one.
It isn't clear that it is conceptually possible to accept the bribe. (It
really isn't clear that it is practically possible, but we're not sure
whether that's a good reply on the part of the consequentialist.) And it
isn't clear that the argument for premise one properly respects the
distinction between causation and constitution we described in the last
section.

Even if those arguments fail, the intuitive force of premise two is not
as strong as the intuition behind Greaves's, or Berker's, anti-bribery
intuitions. And that's one of the main upshots of this paper. It's
commonly thought that for the consequentialist, in any field, everything
has its price. The result we proved at the end of section one shows this
isn't true. It turns out that no good epistemic consequentialist should
accept a bribe that leads them to believing an atomic proposition they
have conclusive evidence is false, no matter how strong the inducements.
Maybe one day there will be a convincing bribery based case that
epistemic consequentialism is unacceptably corrupting of the epistemic
soul. But that case hasn't been made yet, because we've shown a limit on
how corrupt the consequentialist can be.

\section*{Appendix: Proofs of Theorems 1, 2,
3}\label{appendix-proofs-of-theorems-1-2-3}
\addcontentsline{toc}{section}{Appendix: Proofs of Theorems 1, 2, 3}

\begin{description}
\tightlist
\item[Theorem-1]
Brier\(_{\mathscr{A}}(\mathbf{c},@) = \frac{N}{4}\text{Brier}_{\mathscr{S}}(\mathbf{c},@)\)
where\\
\[
\text{Brier}_{\mathscr{S}}(\mathbf{c},@) = \frac{\sum_{s \in \mathscr{S}} (@(s) - c(s))^2}{N}
\]
\end{description}

To prove this we rely on a series of lemmas.

Let \(\mathscr{A}\) be the algebra generated by a finite partition of
states \(\mathscr{S}= \{s_1, s_2, \dots, s_N\}\). @ is a truth-value
assignment for propositions in \(\mathscr{A}\). For simplicity, assume
\(s_1\) is the true state, so that @(\(s_1\)) = 1 and @(\(s_n\)) = 0 for
\(n > 1\). The credence function \textbf{c} assigns values of
\(c_1, c_2, \dots, c_{N-1}, c_N\) to the elements of \(\mathscr{S}\),
where \(\sum^{N}_{n=1} c_n = 1\) in virtue of coherence.

It will be convenient to start by partitioning \(\mathscr{A}\) into four
``quadrants''. Let \(B\) range over all disjunctions with disjunctions
drawn from \(\mathscr{B}= \{s_2, s_3, \dots, s_{N-1}\}\) (including the
empty disjunction, i.e., the logical contradition \(\bot\)). Then,
\(\mathscr{A}\) can be split into four disjoint parts:

\begin{itemize}
\tightlist
\item
  \(\mathscr{A}_1 = \{B \vee s_1 \vee s_N: B\) a disjunction of the
  elements of \(\mathscr{B}\}\)
\item
  \(\mathscr{A}_2 = \{B \vee s_1: B\) a disjunction of the elements of
  \(\mathscr{B}\}\)
\item
  \(\mathscr{A}_3 = \{B \vee s_N: B\) a disjunction of the elements of
  \(\mathscr{B}\}\)
\item
  \(\mathscr{A}_4 = \{B: B\) a disjunction of the elements of
  \(\mathscr{B}\}\)
\end{itemize}

Notice that:

\begin{itemize}
\tightlist
\item
  \(\mathscr{A}_1 \cup \mathscr{A}_2\) contains all and only the true
  propositions in \(\mathscr{A}\).
\item
  \(\mathscr{A}_3 \cup \mathscr{A}_4\) contains all and only the false
  propositions in \(\mathscr{A}\).
\item
  \(\mathscr{A}_1\) and \(\mathscr{A}_4\) are \emph{complementary} sets,
  i.e., all elements of \(\mathscr{A}_4\) are negations of elements of
  \(\mathscr{A}_1\), and conversely.\\
\item
  \(\mathscr{A}_2\) and \(\mathscr{A}_3\) are also complementary.
\item
  \(\mathscr{A}_1 \cup \mathscr{A}_4\) is the subalgebra of
  \(\mathscr{A}\) generated by
  \(\{s_1 \vee s_N, s_2, s_3, \dots, s_{N-1}\}\).
\end{itemize}

All four quadrants have the same cardinality of \(2^{N-2}\).

For an additive scoring rule
\(\mathbf{I}(\mathbf{c}, @) = \sum_{A \in \mathscr{A}}\mathbf{i}(\mathbf{c}(A), @(A))\)
and \(j = 1, 2, 3, 4\), define
\(\mathbf{I}_j = \sum_{A \in \mathscr{A}_j}\mathbf{i}(\mathbf{c}(A), @(A))\),
and note that
\(\mathbf{I}(\mathbf{c}, @) = 2^{-N}(\mathbf{I}_1 + \mathbf{I}_2 + \mathbf{I}_3 + \mathbf{I}_4)\).

\begin{description}
\tightlist
\item[Lemma-1.1]
If \(\textbf{I}\) is negation symmetric, i.e., if
\(\mathbf{i}(\mathbf{c}(\neg A), @(\neg A)) = \mathbf{i}(\mathbf{c}(A), @(A))\)
for all \(A\), then \(\mathbf{I}_1 = \mathbf{I}_4\) and
\(\mathbf{I}_2 = \mathbf{I}_3\), and
\(\mathbf{I}(\mathbf{c},@) = 2^{1-N}(\mathbf{I}_2 + \mathbf{I}_4)\).
\end{description}

\emph{Proof}\\
This is a direct consequence of the fact that \(\mathscr{A}_1\) is
complementary to \(\mathscr{A}_4\) and that \(\mathscr{A}_2\) is
complementary to \(\mathscr{A}_3\) since this allows us to write

\[
\begin{aligned}
\mathbf{I}_1(\mathbf{c},@) &= \sum_{A \in \mathscr{A}_1} \mathbf{i}(\mathbf{c}(A), @(A)) = \sum_{A \in \mathscr{A}_1} \mathbf{i}(\mathbf{c}(\neg A), @(\neg A)) = \mathbf{I}_4(\mathbf{c},@). \\
\mathbf{I}_3(\mathbf{c},@) &= \sum_{A \in \mathscr{A}_3} \mathbf{i}(\mathbf{c}(A), @(A)) = \sum_{A \in \mathscr{A}_3} \mathbf{i}(\mathbf{c}(\neg A), @(\neg A)) = \mathbf{I}_2(\mathbf{c},@). \text{ QED}\\
\end{aligned}
\]

\noindent Applying Lemma 1.1 with \textbf{I} = \textbf{Brier} we get

\[
\begin{aligned}
(\#)\quad \mathbf{Brier}_{\mathscr{A}}(\mathbf{c}, @) &= 2^{1-N} \sum_{A \in \mathscr{A}} (@(A) - c(A))^2 \\
 &= 2^{1-N} \sum_B [(1-c_1)^2 - 2(1-c_1)\mathbf{c}(B) + \mathbf{c}(B)^2]
 \end{aligned}
\]

since

\[
\begin{aligned}
\mathbf{Brier}_2 &= \sum_B[1 - \mathbf{c}(B \vee s_1)]^2 &&= \sum_B[(1 - c_1) - \mathbf{c}(B)]^2 \\
& &&= \sum_B [(1-c_1)^2 - 2(1-c_1)\mathbf{c}(B) + \mathbf{c}(B)^2] \\
\mathbf{Brier}_4 &= \sum_B \mathbf{c}(B)^2 && \quad
\end{aligned}
\]

\begin{description}
\tightlist
\item[Lemma-1.2]
\[
 \begin{aligned}
 (\sum_{n=2}^{N-1} c_n)^2 &= \sum_{n=2}^{N-1} c{_n}^2 + 2 \sum_{n=2}^{N-2} \sum_{j>n}^{N-1}c_nc_j
 \end{aligned}
 \]
\end{description}

Proof by induction. Easy.

\begin{description}
\tightlist
\item[Lemma-1.3]
\[
  \begin{aligned}
   \mathbf{Brier}_{\mathscr{S}}(\mathbf{c},@) &= \frac{2}{N}[(1-c_1)^2 + \sum_{n=2}^{N-1} c{_n}^2  - (1-c_1)(\sum_{n=2}^{N-1}c_n) + \sum_{n=2}^{N-2} \sum_{j>n}^{N-1}c_nc_j]
   \end{aligned}
   \]
\end{description}

\emph{Proof}\\
Using the definition of the Brier score and the fact that \(s_1\) is
true, we have

\[
\begin{aligned}
\mathbf{Brier}_{\mathscr{S}}(\mathbf{c},@) &= \frac{1}{N}[(1 - c_1)^2 + \sum_{n=2}^{N-1} c{_n}^2 + (1 - \sum_{n=1}^{N-1}c_n)^2] \\
&= \frac{1}{N}[(1 - c_1)^2 + \sum_{n=2}^{N-1} c{_n}^2 + ((1 -c_1) - \sum_{n=2}^{N-1}c_n)^2] \\
&= \frac{1}{N}[(1 - c_1)^2 + \sum_{n=2}^{N-1} c{_n}^2 + (1 -c_1)^2 - 2(1 - c_1) \sum_{n=2}^{N-1}c_n + (\sum_{n=2}^{N-1}c_n)^2] \\
&= \frac{1}{N}[(1 - c_1)^2 + \sum_{n=2}^{N-1} c{_n}^2 + (1 -c_1)^2 - 2(1 - c_1) \sum_{n=2}^{N-1}c_n  \\
&\quad \quad +\sum_{n=2}^{N-1} c{_n}^2 + 2 \sum_{n=2}^{N-2} \sum_{j>n}^{N-1}c_nc_j] \quad \text{(Lemma-1.2)}
\end{aligned}
\]

\noindent Then grouping like terms and factoring out 2 yields the
desired result. QED

\begin{description}
\tightlist
\item[Lemma-1.4]
\[
 \begin{aligned}
 \sum_{n=2}^{N-1}c_n &= 2^{3-N}\sum_{B \in \mathscr{B}}\mathbf{c}(B)
 \end{aligned}
 \]
\end{description}

\emph{Proof}\\
For each \(n = 2, 3, \dots, N-1\), each \(s_n\) appears in half of the
\(2^{N-2}\) disjunctions with disjuncts drawn from \(\mathscr{B}\). As a
result, each \(c_n\) appears as a summand \(2^{N-3}\) times among the
sums that express the various \(\mathbf{c}(B)\). So
\(\sum_{B \in \mathscr{B}}\mathbf{c}(B) = 2^{N-3}\sum_{n=2}^{N-1}c_n\).
QED

\begin{description}
\tightlist
\item[Lemma-1.5]
\[
  \begin{aligned}
  \sum_{B \in \mathscr{B}}\mathbf{c}(B)^2 &= 2^{N-3}[\sum_{n=2}^{N-1} c{_n}^2 + \sum_{n=2}^{N-2} \sum_{j>n}^{N-1}c_nc_j]
  \end{aligned}
  \]
\end{description}

\emph{Proof}\\
We proceed by induction starting with the first meaningful case of
\(N=4\), where calculation shows
\(\sum_B\mathbf{c}(B)^2 = (c_2 + c_3)^2 + c{_2}^2 + c{_3}^2 = 2[c{_2}^2 + c{_3}^2 + c_2c_3]\).
Now, assume the identity holds for disjunctions \(B\) of elements of
\(\mathscr{B}\) and show that it holds for disjunctions \(A\) of
elements of \(\mathscr{B}\cup \{s_N\}\).

\[
\begin{aligned}
\sum_A\mathbf{c}(A)^2 &= \sum_B\mathbf{c}(B)^2 + \sum_B\mathbf{c}(B \vee s_N)^2 \\
&= \sum_B\mathbf{c}(B)^2 + \sum_B(\mathbf{c}(B)^2 + 2c_N\mathbf{c}(B) + c{_N}^2)\\
&= 2\sum_B\mathbf{c}(B)^2 + 2c_N\sum_B\mathbf{c}(B) + \sum_Bc{_N}^2 \\
&= 2 \cdot 2^{N-3}[\sum_{n=2}^{N-1} c{_n}^2 + \sum_{n=2}^{N-2} \sum_{j>n}^{N-1}c_nc_j] + 2c_N\sum_B\mathbf{c}(B) + \sum_Bc{_N}^2 &&\text{(Induction Hypothesis)}\\
&= 2^{N-2}[\sum_{n=2}^{N-1} c{_n}^2 + \sum_{n=2}^{N-2} \sum_{j>n}^{N-1}c_nc_j] + 2^{N-2}c_N\sum_{n=2}^{N-1}c_n + \sum_Bc{_N}^2 &&\text{(Lemma-1.4)} \\
&= 2^{N-2}[\sum_{n=2}^{N-1} c{_n}^2 + \sum_{n=2}^{N-2} \sum_{j>n}^{N-1}c_nc_j] + 2^{N-2}c_N\sum_{n=2}^{N-1}c_n + 2^{N-2}c{_N}^2 &&\text{Since $|\mathscr{B}| = 2^{N-2}$} \\
&= 2^{N-2}[\sum_{n=2}^{N} c{_n}^2 + \sum_{n=2}^{N-1} \sum_{j>n}^{N}c_nc_j] && \text{ QED}
\end{aligned}
\]

\noindent Plugging the results of the last two lemmas into Lemma-1.3
produces a result of

\[
\begin{aligned}
\mathbf{Brier}_{\mathscr{S}}(\textbf{c},@) &= \frac{2}{N}[(1-c_1)^2 + 2^{3-N}\sum_{B \in \mathscr{B}}\mathbf{c}(B)^2 - 2^{3-N}(1-c_1)\sum_{B \in \mathscr{B}}\mathbf{c}(B)] \\
&= \frac{2}{N}\sum_{B \in \mathscr{B}}[2^{2-N}(1-c_1)^2 + 2^{3-N}\mathbf{c}(B)^2 - 2^{3-N}(1-c_1)\mathbf{c}(B)] \\
&= \frac{2^{3-N}}{N}\sum_{B \in \mathscr{B}}[(1 - c_1)^2 + 2\mathbf{c}(B)^2 - 2(1-c_1)\mathbf{c}(B)]
\end{aligned}
\]

\noindent Comparing this to (\#) we see that it is just \(\frac{N}{4}\)
times Brier\(_\mathscr{S}(\mathbf{c},@)\), as we aimed to prove. QED.

\begin{description}
\tightlist
\item[Theorem-2]
When inaccuracy over \(\mathscr{A}\) is measured using the Brier score,
the least accurate credal states are those which assign credence 1 to
some false atom of \(\mathscr{A}\).
\end{description}

\emph{Proof}\\
As before, suppose that \(@(s_1) = 1\), and let \textbf{c} be a credence
function that assigns credence 1 to some false atom
\(s_2, s_3,..., s_N\) of \(\mathscr{A}\). In light of Theorem-1 it
suffices to show that
\(\mathbf{Brier}_\mathscr{S}(\mathbf{c}, @) > \mathbf{Brier}_\mathscr{S}(\mathbf{b}, @)\)
where \textbf{b} does not assign credence 1 to any false atom. Start by
noting that for any credence function \(\pi\) defined on the atoms of
\(\mathscr{A}\) one has

\[
\begin{aligned}
\mathbf{Brier}_\mathscr{S}(\pi,@) &= \frac{1}{N}[(1-\pi_1)^2 + \sum_{n=2}^{N-1}\pi{_n}^2+ (1-\sum_{n=1}^{N-1}\pi{_n})^2] \\
&= \frac{1}{N}[1 - 2\pi_1 + \sum_{n=1}^{N-1}\pi{_n}^2+ (1-\sum_{n=1}^{N-1}\pi{_n})^2]
\end{aligned}
\]

\noindent But, since each \(\pi_n \in [0, 1]\) is non-negative, it
follows that
\(\pi_1 \geq \pi{_1}^2, \pi_2 \geq \pi{_2}^2, \dots, \pi_N \geq \pi{_N}^2\)
with the inequality strict in each case unless \(\pi_n\) is either 1 or
0.

This means that the sum
\(\sum_{n=1}^{N-1}\pi{_n}^2+ (1-\sum_{n=1}^{N-1}\pi{_n})^2\) is less
than or equal to 1, with equality if and only if exactly one of the
atoms \(s_n\) is assigned probability 1 (and the rest have probability
zero). As a result,
\textbf{Brier}\(_\mathscr{S}(\pi, @) \leq \frac{2}{N}(1 - \pi_1)\) with
equality if and only if exactly one of the atoms \(s_n\) is assigned
probability 1. So, there are three relevant cases:

\begin{itemize}
\tightlist
\item
  If \(\pi\) assigns some false atom probability 1,
  \textbf{Brier}\(_\mathscr{S}(\pi, @) = \frac{2}{N}\cdot(1 - 0) = \frac{2}{N}\).
\item
  If \(\pi\) assigns the true atom probability 1,
  \textbf{Brier}\(_\mathscr{S}(\pi, @) = \frac{2}{N}\cdot(1 - 1) = 0\).
\item
  If \(\pi\) does not assign any atom probability 1,
  \textbf{Brier}\(_\mathscr{S}(\pi, @) < \frac{2}{N}\cdot(1 - c_1) \leq \frac{2}{N}\).
\end{itemize}

So, since \textbf{c} fits case (i) and \textbf{b} fits case (ii) or
(iii) we have the desired result. QED

\begin{description}
\tightlist
\item[Theorem-3]
Let \(\mathscr{A}\) be an algebra of propositions generated by atoms
\(a_1, ..., a_{2N}\), where \(a_1\) is the truth. Let \emph{p} and \(Q\)
be probability functions defined on \(\mathscr{A}\). \emph{p} assigns
all its mass to the first \(N\) atoms, so that
\(P(a_1 \vee \dots \vee a_N) = 1\), and it also assigns some positive
probability to \(a_1\). \(Q\) assigns all its mass to the false atom
\(a_{2N}\), so that \(Q(a_{2N}) = 1\). Then, for any proper score
\textbf{I} satisfying Truth-directedness, Extensionality and Negation
Symmetry we have \(\mathbf{I}(v_1, P) < \mathbf{I}(v_1, Q)\) where
\(v_1\) is the truth-value assignment associated with \(a_1\) (i.e.,
where \(v_1(X) = 1\) if and only if \(a_1\) entails \emph{x}).
\end{description}

\emph{Proof}\\
We can divide the algebra \(\mathscr{A}\) into four quadrants

\[
\begin{aligned}
\mathscr{A}^1 &= \{X \in \mathscr{A}: a_1 \vDash X \text{ and } a_{2N} \vDash X\} \\
\mathscr{A}^2 &= \{X \in \mathscr{A}: a_1 \vDash X \text{ and } a_{2N} \nvDash X\} \\
\mathscr{A}^3 &= \{X \in \mathscr{A}: a_1 \nvDash X \text{ and } a_{2N} \vDash X\} \\
\mathscr{A}^4 &= \{X \in \mathscr{A}: a_1 \nvDash X \text{ and } a_{2N} \nvDash X\}
\end{aligned}
\]

\noindent We know the following:

\begin{itemize}
\tightlist
\item
  \(Q\) is maximally accurate on \(\mathscr{A}^1 \cup \mathscr{A}^4\).
  Every proposition in \(\mathscr{A}^1\) is true, and \(Q\) assigns it a
  probability of 1. Every proposition in \(\mathscr{A}^4\) is false, and
  \(Q\) assigns it a probability of 0.
\item
  \(Q\) is maximally inaccurate on \(\mathscr{A}^2 \cup \mathscr{A}^3\).
  Every proposition in \(\mathscr{A}^2\) is true, and \(Q\) assigns it a
  probability of 0. Every proposition in \(\mathscr{A}^3\) is false, and
  \(Q\) assigns it a probability of 1.
\item
  \emph{p} is maximally accurate on
  \(\mathscr{A}^3 \cup \mathscr{A}^4\). Every proposition in
  \(\mathscr{A}^3 \cup \mathscr{A}^4\) is false, and \emph{p} assigns it
  a probability of 0.
\item
  Each quadrant has \(2^{2N-2}\) elements.
\end{itemize}

\begin{description}
\tightlist
\item[Lemma-3.1]
When \(a_1\) is true, the accuracy score of \emph{p} over the
propositions in \(\mathscr{A}^1\) is identical to the accuracy score of
\emph{p} over the propositions in \(\mathscr{A}^2\).
\end{description}

\emph{Proof}\\
\noindent Note first that the function
\(F: \mathscr{A}^1 \rightarrow \mathscr{A}^2\) that takes \emph{x} to
\(X \wedge \neg a_{2N}\) is a bijection of \(\mathscr{A}^1\) onto
\(\mathscr{A}^2\). Since every proposition in
\(\mathscr{A}^1 \cup \mathscr{A}^2\) is true, we can then write the
respective accuracy scores of \(\mathscr{A}^1\) and \(\mathscr{A}^2\) as

\[
\begin{aligned}
\mathbf{I}_{\mathscr{A}^1}(a_1, P) &= 2^{2-2N} \cdot \sum_{X \in \mathscr{A}^1} \mathbf{I}(1, P(X)) \\
\mathbf{I}_{\mathscr{A}^2}(a_1, P) &= 2^{2-2N} \cdot \sum_{X \in \mathscr{A}^1} \mathbf{I}(1, P(X \wedge \neg a_{2N}))
\end{aligned}
\]

\noindent Note: \emph{x} ranges over \(\mathscr{A}^1\) in both
summations. But since \(P(a_{2N}) = 0\) we have
\(P(X) = P(X \wedge a_{2N})\) for each \emph{x} in \(\mathscr{A}^1\).
Since \textbf{I} is extensional, this means that
\(\mathbf{I}(1, P(X)) = \mathbf{I}(1, P(X \wedge a_{2N}))\) for each
\emph{x} in \(\mathscr{A}^1\). And, it follows that
\(\mathbf{I}_{\mathscr{A}^1}(a_1, P)\) and
\(\mathbf{I}_{\mathscr{A}^2}(a_1, P)\) are identical. (Note that even if
\(P(a_{2N}) > 0\), Truth-directedness entails that
\(\mathbf{I}_{\mathscr{A}^1}(a_1, P) < \mathbf{I}_{\mathscr{A}^2}(a_1, P)\).)

\begin{description}
\tightlist
\item[Lemma-3.2]
When \(a_1\) is true, the accuracy score of \(Q\) over \(\mathscr{A}^2\)
is identical to the accuracy score of \(Q\) over \(\mathscr{A}^3\).
\end{description}

\emph{Proof}\\
To see this, note first that the function
\(G: \mathscr{A}^2 \rightarrow \mathscr{A}^3\) that takes \emph{x} to
\(G(X) = \neg X\) is a bijection (i.e., the negation of everything in
\(\mathscr{A}^2\) is in \(\mathscr{A}^3\) and vice-versa). This,
together with the fact that \(\mathscr{A}^2\) contains only truths and
\(\mathscr{A}^3\) contains only falsehoods, lets us write

\[
\begin{aligned}
\mathbf{I}_{\mathscr{A}^2}(a_1, Q) &= 2^{2-2N} \cdot \sum_{X \in \mathscr{A}^2} \mathbf{I}(1, Q(X)) \\
\mathbf{I}_{\mathscr{A}^3}(a_1, Q) &= 2^{2-2N} \cdot \sum_{X \in \mathscr{A}^2} \mathbf{I}(0, Q(\neg X))
\end{aligned}
\]

\noindent But since \textbf{I} is negation symmetric,
\(\mathbf{I}(1, Q(X)) = \mathbf{I}(0, Q(\neg X))\) for every \emph{x},
which means that
\(\mathbf{I}_{\mathscr{A}^2}(a_1, Q) = \mathbf{I}_{\mathscr{A}^3}(a_1, Q)\).
(Note that this proof made no assumptions about \(Q\) except that it was
a probability.)

\begin{description}
\tightlist
\item[Lemma-3.3]
If \(P(a_1) > 0\), the accuracy score of \emph{p} over \(\mathscr{A}^2\)
is strictly less than the accuracy score of \(Q\) over
\(\mathscr{A}^2\).
\end{description}

\emph{Proof}\\
Since \(Q(X) = 0\) everywhere on \(\mathscr{A}^2\) we have
\[\begin{aligned}
\mathbf{I}_{\mathscr{A}^2}(a_1, P) &= 2^{2-2N} \cdot \sum_{X \in \mathscr{A}^2} \mathbf{I}(1, P(X)) \\
\mathbf{I}_{\mathscr{A}^2}(a_1, Q) &= 2^{2-2N} \cdot \sum_{X \in \mathscr{A}^2} \mathbf{I}(1, 0) 
\end{aligned}
\]

\noindent But, by Truth Directedness
\(\mathbf{I}(1, 0) > \mathbf{I}(1, P(X))\) since \(P(a_1) > 0\) implies
that \(P(X) > 0\) for all \(X \in \mathscr{A}^2\). Thus
\(\mathbf{I}_{\mathscr{A}^2}(a_1, Q) > \mathbf{I}_{\mathscr{A}^2}(a_1, P)\).

To complete the proof of the theorem we need only note that

\[
\begin{aligned}
\mathbf{I}_{\mathscr{A}}(a_1, P) &= \frac{\mathbf{I}_{\mathscr{A}^1}(a_1, P)}{4} + \frac{\mathbf{I}_{\mathscr{A}^2}(a_1, P)}{4} &\text{(since }P\text{ is perfect on }\mathscr{A}^3 \cup \mathscr{A}^4) \\
&= \frac{\mathbf{I}_{\mathscr{A}^2}(a_1, P)}{2} &\text{Lemma-3.1} \\
&< \frac{\mathbf{I}_{\mathscr{A}^2}(a_1, Q)}{2} &\text{Lemma-3.3} \\
&= \frac{\mathbf{I}_{\mathscr{A}^2}(a_1, Q)}{4} + \frac{\mathbf{I}_{\mathscr{A}^3}(a_1, Q)}{4} &\text{Lemma-3.2} \\
&= \mathbf{I}_{\mathscr{A}}(a_1, Q) &\text{(since }Q\text{ is perfect on }\mathscr{A}^1 \cup \mathscr{A}^4)
\end{aligned}
\]

\subsection*{References}\label{references}
\addcontentsline{toc}{subsection}{References}

\phantomsection\label{refs}
\begin{CSLReferences}{1}{0}
\bibitem[\citeproctext]{ref-Berker2013b}
Berker, Selim. 2013a. {``Epistemic Teleology and the Separateness of
Propositions.''} \emph{Philosophical Review} 122 (3): 337--93. doi:
\href{https://doi.org/10.1215/00318108-2087645}{10.1215/00318108-2087645}.

\bibitem[\citeproctext]{ref-Berker2013a}
---------. 2013b. {``The Rejection of Epistemic Consequentialism.''}
\emph{Philosophical Issues} 23 (1): 363--87. doi:
\href{https://doi.org/10.1111/phis.12019}{10.1111/phis.12019}.

\bibitem[\citeproctext]{ref-DescartesMeditations}
Descartes, René. 1641/1996. \emph{Meditations on First Philosophy, {Tr.
John Cottingham}}. Cambridge: Cambridge University Press.

\bibitem[\citeproctext]{ref-Greaves2013}
Greaves, Hilary. 2013. {``Epistemic Decision Theory.''} \emph{Mind} 122
(488): 915--52. doi:
\href{https://doi.org/10.1093/mind/fzt090}{10.1093/mind/fzt090}.

\bibitem[\citeproctext]{ref-Hammond1988}
Hammond, Peter J. 1988. {``Consequentialist Foundations for Expected
Utility.''} \emph{Theory and Decision} 25 (1): 25--78. doi:
\href{https://doi.org/10.1007/BF00129168}{10.1007/BF00129168}.

\bibitem[\citeproctext]{ref-Jenkins2007}
Jenkins, C. S. 2007. {``Entitlement and Rationality.''} \emph{Synthese}
157 (1): 25--45. doi:
\href{https://doi.org/10.1007/s11229-006-0012-2}{10.1007/s11229-006-0012-2}.

\bibitem[\citeproctext]{ref-Joyce1998}
Joyce, James M. 1998. {``A Non-Pragmatic Vindication of Probabilism.''}
\emph{Philosophy of Science} 65 (4): 575--603. doi:
\href{https://doi.org/10.1086/392661}{10.1086/392661}.

\bibitem[\citeproctext]{ref-Maher1997}
Maher, Patrick. 1997. {``Depragmatised Dutch Book Arguments.''}
\emph{Philosophy of Science} 64 (2): 291--305. doi:
\href{https://doi.org/10.1086/392552}{10.1086/392552}.

\bibitem[\citeproctext]{ref-Roulston2007}
Roulston, Mark S. 2007. {``Performance Targets and the Brier Score.''}
\emph{Meterological Applications} 14: 185--94. doi:
\href{https://doi.org/10.1002/met.21}{10.1002/met.21}.

\bibitem[\citeproctext]{ref-Shafer1976}
Shafer, Glenn. 1976. \emph{A Mathematical Theory of Evidence}.
Princeton: Princeton University Press.

\bibitem[\citeproctext]{ref-Weatherson1999}
Weatherson, Brian. 1999. {``Begging the Question and Bayesians.''}
\emph{Studies in the History and Philosophy of Science Part A} 30:
687--97.

\end{CSLReferences}



\noindent Published in\emph{
Logos and Episteme}, 2019, pp. 263-282.


\end{document}
