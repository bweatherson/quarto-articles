% Options for packages loaded elsewhere
\PassOptionsToPackage{unicode}{hyperref}
\PassOptionsToPackage{hyphens}{url}
%
\documentclass[
  10pt,
  letterpaper,
  DIV=11,
  numbers=noendperiod,
  twoside]{scrartcl}

\usepackage{amsmath,amssymb}
\usepackage{setspace}
\usepackage{iftex}
\ifPDFTeX
  \usepackage[T1]{fontenc}
  \usepackage[utf8]{inputenc}
  \usepackage{textcomp} % provide euro and other symbols
\else % if luatex or xetex
  \usepackage{unicode-math}
  \defaultfontfeatures{Scale=MatchLowercase}
  \defaultfontfeatures[\rmfamily]{Ligatures=TeX,Scale=1}
\fi
\usepackage{lmodern}
\ifPDFTeX\else  
    % xetex/luatex font selection
    \setmainfont[ItalicFont=EB Garamond Italic,BoldFont=EB Garamond
Bold]{EB Garamond Math}
    \setsansfont[]{Europa-Bold}
  \setmathfont[]{Garamond-Math}
\fi
% Use upquote if available, for straight quotes in verbatim environments
\IfFileExists{upquote.sty}{\usepackage{upquote}}{}
\IfFileExists{microtype.sty}{% use microtype if available
  \usepackage[]{microtype}
  \UseMicrotypeSet[protrusion]{basicmath} % disable protrusion for tt fonts
}{}
\usepackage{xcolor}
\usepackage[left=1in, right=1in, top=0.8in, bottom=0.8in,
paperheight=9.5in, paperwidth=6.5in, includemp=TRUE, marginparwidth=0in,
marginparsep=0in]{geometry}
\setlength{\emergencystretch}{3em} % prevent overfull lines
\setcounter{secnumdepth}{3}
% Make \paragraph and \subparagraph free-standing
\makeatletter
\ifx\paragraph\undefined\else
  \let\oldparagraph\paragraph
  \renewcommand{\paragraph}{
    \@ifstar
      \xxxParagraphStar
      \xxxParagraphNoStar
  }
  \newcommand{\xxxParagraphStar}[1]{\oldparagraph*{#1}\mbox{}}
  \newcommand{\xxxParagraphNoStar}[1]{\oldparagraph{#1}\mbox{}}
\fi
\ifx\subparagraph\undefined\else
  \let\oldsubparagraph\subparagraph
  \renewcommand{\subparagraph}{
    \@ifstar
      \xxxSubParagraphStar
      \xxxSubParagraphNoStar
  }
  \newcommand{\xxxSubParagraphStar}[1]{\oldsubparagraph*{#1}\mbox{}}
  \newcommand{\xxxSubParagraphNoStar}[1]{\oldsubparagraph{#1}\mbox{}}
\fi
\makeatother


\providecommand{\tightlist}{%
  \setlength{\itemsep}{0pt}\setlength{\parskip}{0pt}}\usepackage{longtable,booktabs,array}
\usepackage{calc} % for calculating minipage widths
% Correct order of tables after \paragraph or \subparagraph
\usepackage{etoolbox}
\makeatletter
\patchcmd\longtable{\par}{\if@noskipsec\mbox{}\fi\par}{}{}
\makeatother
% Allow footnotes in longtable head/foot
\IfFileExists{footnotehyper.sty}{\usepackage{footnotehyper}}{\usepackage{footnote}}
\makesavenoteenv{longtable}
\usepackage{graphicx}
\makeatletter
\newsavebox\pandoc@box
\newcommand*\pandocbounded[1]{% scales image to fit in text height/width
  \sbox\pandoc@box{#1}%
  \Gscale@div\@tempa{\textheight}{\dimexpr\ht\pandoc@box+\dp\pandoc@box\relax}%
  \Gscale@div\@tempb{\linewidth}{\wd\pandoc@box}%
  \ifdim\@tempb\p@<\@tempa\p@\let\@tempa\@tempb\fi% select the smaller of both
  \ifdim\@tempa\p@<\p@\scalebox{\@tempa}{\usebox\pandoc@box}%
  \else\usebox{\pandoc@box}%
  \fi%
}
% Set default figure placement to htbp
\def\fps@figure{htbp}
\makeatother
% definitions for citeproc citations
\NewDocumentCommand\citeproctext{}{}
\NewDocumentCommand\citeproc{mm}{%
  \begingroup\def\citeproctext{#2}\cite{#1}\endgroup}
\makeatletter
 % allow citations to break across lines
 \let\@cite@ofmt\@firstofone
 % avoid brackets around text for \cite:
 \def\@biblabel#1{}
 \def\@cite#1#2{{#1\if@tempswa , #2\fi}}
\makeatother
\newlength{\cslhangindent}
\setlength{\cslhangindent}{1.5em}
\newlength{\csllabelwidth}
\setlength{\csllabelwidth}{3em}
\newenvironment{CSLReferences}[2] % #1 hanging-indent, #2 entry-spacing
 {\begin{list}{}{%
  \setlength{\itemindent}{0pt}
  \setlength{\leftmargin}{0pt}
  \setlength{\parsep}{0pt}
  % turn on hanging indent if param 1 is 1
  \ifodd #1
   \setlength{\leftmargin}{\cslhangindent}
   \setlength{\itemindent}{-1\cslhangindent}
  \fi
  % set entry spacing
  \setlength{\itemsep}{#2\baselineskip}}}
 {\end{list}}
\usepackage{calc}
\newcommand{\CSLBlock}[1]{\hfill\break\parbox[t]{\linewidth}{\strut\ignorespaces#1\strut}}
\newcommand{\CSLLeftMargin}[1]{\parbox[t]{\csllabelwidth}{\strut#1\strut}}
\newcommand{\CSLRightInline}[1]{\parbox[t]{\linewidth - \csllabelwidth}{\strut#1\strut}}
\newcommand{\CSLIndent}[1]{\hspace{\cslhangindent}#1}

\setlength\heavyrulewidth{0ex}
\setlength\lightrulewidth{0ex}
\usepackage[automark]{scrlayer-scrpage}
\clearpairofpagestyles
\cehead{
  Brian Weatherson
  }
\cohead{
  Games and the Reason-Knowledge Principle
  }
\ohead{\bfseries \pagemark}
\cfoot{}
\makeatletter
\newcommand*\NoIndentAfterEnv[1]{%
  \AfterEndEnvironment{#1}{\par\@afterindentfalse\@afterheading}}
\makeatother
\NoIndentAfterEnv{itemize}
\NoIndentAfterEnv{enumerate}
\NoIndentAfterEnv{description}
\NoIndentAfterEnv{quote}
\NoIndentAfterEnv{equation}
\NoIndentAfterEnv{longtable}
\NoIndentAfterEnv{abstract}
\renewenvironment{abstract}
 {\vspace{-1.25cm}
 \quotation\small\noindent\rule{\linewidth}{.5pt}\par\smallskip
 \noindent }
 {\par\noindent\rule{\linewidth}{.5pt}\endquotation}
\KOMAoption{captions}{tableheading}
\makeatletter
\@ifpackageloaded{caption}{}{\usepackage{caption}}
\AtBeginDocument{%
\ifdefined\contentsname
  \renewcommand*\contentsname{Table of contents}
\else
  \newcommand\contentsname{Table of contents}
\fi
\ifdefined\listfigurename
  \renewcommand*\listfigurename{List of Figures}
\else
  \newcommand\listfigurename{List of Figures}
\fi
\ifdefined\listtablename
  \renewcommand*\listtablename{List of Tables}
\else
  \newcommand\listtablename{List of Tables}
\fi
\ifdefined\figurename
  \renewcommand*\figurename{Figure}
\else
  \newcommand\figurename{Figure}
\fi
\ifdefined\tablename
  \renewcommand*\tablename{Table}
\else
  \newcommand\tablename{Table}
\fi
}
\@ifpackageloaded{float}{}{\usepackage{float}}
\floatstyle{ruled}
\@ifundefined{c@chapter}{\newfloat{codelisting}{h}{lop}}{\newfloat{codelisting}{h}{lop}[chapter]}
\floatname{codelisting}{Listing}
\newcommand*\listoflistings{\listof{codelisting}{List of Listings}}
\makeatother
\makeatletter
\makeatother
\makeatletter
\@ifpackageloaded{caption}{}{\usepackage{caption}}
\@ifpackageloaded{subcaption}{}{\usepackage{subcaption}}
\makeatother

\usepackage{bookmark}

\IfFileExists{xurl.sty}{\usepackage{xurl}}{} % add URL line breaks if available
\urlstyle{same} % disable monospaced font for URLs
\hypersetup{
  pdftitle={Games and the Reason-Knowledge Principle},
  pdfauthor={Brian Weatherson},
  hidelinks,
  pdfcreator={LaTeX via pandoc}}


\title{Games and the Reason-Knowledge Principle}
\author{Brian Weatherson}
\date{2012}

\begin{document}
\maketitle
\begin{abstract}
A potential counterexample to Hawthorne and Stanley's Reason-Knowledge
Principle
\end{abstract}


\setstretch{1.1}
John Hawthorne and Jason Stanley
(\citeproc{ref-Hawthorne2008-HAWKAA}{2008}), defend what they call the
``Reason-Knowledge Principle''.

\begin{quote}
Where one's choice is \emph{p}-dependent, it is appropriate to treat the
proposition that \emph{p} as a reason for acting iff you know that
\emph{p}. (578)
\end{quote}

There have been many attempts in the literature to show that this leads
to implausible \emph{actions}. As Jonathan Ichikawa
(\citeproc{ref-Ichikawa2012}{2012}) shows, most of these attempts rest
on further, and arguably false, assumptions about the connection between
reasons and action. Relatedly, most of those responses concern the role
of knowledge and reasons in decision-making. I'll argue that we can
formulate a sharper problem for the principle if we focus on
game-playing, and say exactly which extra assumptions we are making.

The Reason-Knowledge Principle should have the following implications,
at least for cases where \(S\)'s aim is to produce the best outcome.

\begin{description}
\tightlist
\item[(1)]
If \(S\) knows that \(\varphi\) and \(\psi\) will produce the same
outcome, and \(S\) must choose \(\varphi\) or \(\psi\), then it is
rationally permissible for \(S\) to choose \(\psi\).
\item[(2)]
If \(S\) knows that \(\varphi\) and \(\psi\) will produce the same
outcome if \emph{p}, and \(\varphi\) will produce a better outcome if
\(\neg p\), then it is rationally permissible for \(S\) to choose
\(\psi\) iff she knows \emph{p}.
\end{description}

The point of (1) is that \(S\) can use her knowledge that \(\varphi\)
and \(\psi\) will produce the same outcome to justify making an
arbitrary choice between \(\varphi\) and \(\psi\). And the point of (2)
is that the Reason-Knowledge Principle suggests only knowledge that
\emph{p} could justify ignoring the fact that \(\psi\) does worse than
\(\varphi\) if \(\neg p\).

Define a \textbf{symmetric} game as having these features:

\begin{itemize}
\tightlist
\item
  The game is purely co-operative; each player gets the same payoffs;
\item
  Each player knows nothing about the other save that it is common
  knowledge the players are rational, and hence know what each other's
  rational requirements are;
\item
  Each player has the same moves available; and,
\item
  The payoffs are a function of just which moves are made, not of who
  makes them.
\end{itemize}

Assume \(A\) and \(B\) are playing a symmetric game, and it is common
knowledge which symmetric game they are playing. Then the following
premise seems hard to dispute:

\begin{description}
\tightlist
\item[(3)]
It is rationally required for \(A\) to play \(\varphi\) iff \(A\) knows
\(B\) will play \(\varphi\).
\end{description}

What makes (3) so compelling is that we can derive it from (4), (5) and
(6).

\begin{description}
\tightlist
\item[(4)]
\(A\) knows that \(B\) will play \(\varphi\) iff \(A\) knows that any
rational player will play \(\varphi\).
\item[(5)]
If \(A\) knows any rational player will play \(\varphi\), then \(A\) is
rationally required to play \(\varphi\).
\item[(6)]
If \(A\) is rationally required to play \(\varphi\), then \(A\) knows
that any rational player will play \(\varphi\).
\end{description}

We get (4) from the fact that \(A\) knows nothing about \(B\) save that
she is rational. We get (5) by the factivity of knowledge. And we get
(6) by the requirement that the players are rational, and hence know
what rationality requires of each player. And these three together
entail (3). So (3) is true, and (1) and (2) are entailed by the
Reason-Knowledge Principle. Unfortunately, (1), (2) and (3) are
inconsistent, as we'll now show.

Informally, in this game \(A\) and \(B\) must each play either a green
or red card. I will capitalise \(A\)'s moves, i.e., \(A\) can play GREEN
or RED, and italicise \(B\)'s moves, i.e., \(B\) can play \emph{green}
or \emph{red}. If two green cards, or one green card and one red card
are played, each player gets \$1. If two red cards are played, each gets
nothing. Each cares just about their own wealth, so getting \$1 is worth
1 util. All of this is common knowledge. More formally, here is the game
table, with \(A\) on the row and \(B\) on the column.

\begin{longtable}[]{@{}lcc@{}}
\toprule\noalign{}
\endhead
\bottomrule\noalign{}
\endlastfoot
& \emph{green} & \emph{red} \\
GREEN & 1, 1 & 1, 1 \\
RED & 1, 1 & 0, 0 \\
\end{longtable}

Assume \(A\) knows \(B\) will play \emph{green}. By (3), it is
rationally required that \(A\) plays GREEN. But \(A\) can use this
knowledge of \(B\) to deduce that GREEN and RED have the same payoff. So
by (1), it is rationally permissible to play RED. Contradiction.

Now assume \(A\) does not know \(B\) will play \emph{green}. By (3), it
is not a rational requirement that \(A\) plays GREEN. But \(A\) knows
that GREEN does better than RED unless \(B\) plays \emph{green}. And
since she doesn't know \(B\) plays \emph{green}, by (2), she's required
to play GREEN. Contradiction.

So either assuming that \(A\) does or does not know that it is
rationally required for \(B\) to play \emph{green} leads to a
contradiction given (1), (2) and (3). So these three premises are
inconsistent. Since (3) is true, that means (1) or (2) is false. And
since the Reason-Knowledge principle entails those two premises, one of
which is false, the Reason-Knowledge Principle is false.

I'm not entirely sure which of (1) and (2) is false; both of them do
feel plausible. I suspect the problem is (1). Assume \(A\) deduces from
premises she believes that rational players will play a green card.
Perhaps she agrees with Robert Stalnaker
(\citeproc{ref-Stalnaker1998}{1998}) that rationality requires avoiding
weakly dominated options. Then she knows it doesn't matter to her
outcome whether she plays GREEN or RED; she will get \$1 either way. But
if she plays RED, she is incoherent; she is doing something she thinks
no rational player does. And perhaps this incoherence is a bad thing in
itself. Niko Kolodny (\citeproc{ref-Kolodny2005}{2005}) argues that
incoherence is not bad in itself; Jacob Ross
(\citeproc{ref-Ross2012}{2012}) argues that it is. The suggestion that
(1) is the false premise favours Ross's view over Kolodny's. But this
conclusion is very speculative; the main thing I wanted to note was the
problem this game raises for the Reason-Knowledge Principle.

\section*{References}\label{references}
\addcontentsline{toc}{section}{References}

\phantomsection\label{refs}
\begin{CSLReferences}{1}{0}
\bibitem[\citeproctext]{ref-Hawthorne2008-HAWKAA}
Hawthorne, John, and Jason Stanley. 2008. {``{Knowledge and Action}.''}
\emph{Journal of Philosophy} 105 (10): 571--90. doi:
\href{https://doi.org/10.5840/jphil20081051022}{10.5840/jphil20081051022}.

\bibitem[\citeproctext]{ref-Ichikawa2012}
Ichikawa, Jonathan. 2012. {``Experimentalist Pressure Against
Traditional Methodology.''} \emph{Philosophical Psychology} 25 (5):
743--65. doi:
\href{https://doi.org/10.1080/09515089.2011.625118}{10.1080/09515089.2011.625118}.

\bibitem[\citeproctext]{ref-Kolodny2005}
Kolodny, Niko. 2005. {``Why Be Rational?''} \emph{Mind} 114 (455):
509--63. doi:
\href{https://doi.org/10.1093/mind/fzi509}{10.1093/mind/fzi509}.

\bibitem[\citeproctext]{ref-Ross2012}
Ross, Jacob. 2012. {``All Roads Lead to Violations of Countable
Additivity.''} \emph{Philosophical Studies} 161 (3): 381--90. doi:
\href{https://doi.org/10.1007/s11098-011-9744-z}{10.1007/s11098-011-9744-z}.

\bibitem[\citeproctext]{ref-Stalnaker1998}
Stalnaker, Robert. 1998. {``Belief Revision in Games: Forward and
Backward Induction.''} \emph{Mathematical Social Sciences} 36 (1):
31--56. doi:
\href{https://doi.org/10.1016/S0165-4896(98)00007-9}{10.1016/S0165-4896(98)00007-9}.

\end{CSLReferences}



\noindent Published in\emph{
The Reasoner}, 2012, pp. 6-7.


\end{document}
