% Options for packages loaded elsewhere
% Options for packages loaded elsewhere
\PassOptionsToPackage{unicode}{hyperref}
\PassOptionsToPackage{hyphens}{url}
%
\documentclass[
  11pt,
  letterpaper,
  DIV=11,
  numbers=noendperiod,
  twoside]{scrartcl}
\usepackage{xcolor}
\usepackage[left=1.1in, right=1in, top=0.8in, bottom=0.8in,
paperheight=9.5in, paperwidth=7in, includemp=TRUE, marginparwidth=0in,
marginparsep=0in]{geometry}
\usepackage{amsmath,amssymb}
\setcounter{secnumdepth}{3}
\usepackage{iftex}
\ifPDFTeX
  \usepackage[T1]{fontenc}
  \usepackage[utf8]{inputenc}
  \usepackage{textcomp} % provide euro and other symbols
\else % if luatex or xetex
  \usepackage{unicode-math} % this also loads fontspec
  \defaultfontfeatures{Scale=MatchLowercase}
  \defaultfontfeatures[\rmfamily]{Ligatures=TeX,Scale=1}
\fi
\usepackage{lmodern}
\ifPDFTeX\else
  % xetex/luatex font selection
  \setmainfont[ItalicFont=EB Garamond Italic,BoldFont=EB Garamond
SemiBold]{EB Garamond Math}
  \setsansfont[]{EB Garamond}
  \setmathfont[]{Garamond-Math}
\fi
% Use upquote if available, for straight quotes in verbatim environments
\IfFileExists{upquote.sty}{\usepackage{upquote}}{}
\IfFileExists{microtype.sty}{% use microtype if available
  \usepackage[]{microtype}
  \UseMicrotypeSet[protrusion]{basicmath} % disable protrusion for tt fonts
}{}
\usepackage{setspace}
% Make \paragraph and \subparagraph free-standing
\makeatletter
\ifx\paragraph\undefined\else
  \let\oldparagraph\paragraph
  \renewcommand{\paragraph}{
    \@ifstar
      \xxxParagraphStar
      \xxxParagraphNoStar
  }
  \newcommand{\xxxParagraphStar}[1]{\oldparagraph*{#1}\mbox{}}
  \newcommand{\xxxParagraphNoStar}[1]{\oldparagraph{#1}\mbox{}}
\fi
\ifx\subparagraph\undefined\else
  \let\oldsubparagraph\subparagraph
  \renewcommand{\subparagraph}{
    \@ifstar
      \xxxSubParagraphStar
      \xxxSubParagraphNoStar
  }
  \newcommand{\xxxSubParagraphStar}[1]{\oldsubparagraph*{#1}\mbox{}}
  \newcommand{\xxxSubParagraphNoStar}[1]{\oldsubparagraph{#1}\mbox{}}
\fi
\makeatother


\usepackage{longtable,booktabs,array}
\usepackage{calc} % for calculating minipage widths
% Correct order of tables after \paragraph or \subparagraph
\usepackage{etoolbox}
\makeatletter
\patchcmd\longtable{\par}{\if@noskipsec\mbox{}\fi\par}{}{}
\makeatother
% Allow footnotes in longtable head/foot
\IfFileExists{footnotehyper.sty}{\usepackage{footnotehyper}}{\usepackage{footnote}}
\makesavenoteenv{longtable}
\usepackage{graphicx}
\makeatletter
\newsavebox\pandoc@box
\newcommand*\pandocbounded[1]{% scales image to fit in text height/width
  \sbox\pandoc@box{#1}%
  \Gscale@div\@tempa{\textheight}{\dimexpr\ht\pandoc@box+\dp\pandoc@box\relax}%
  \Gscale@div\@tempb{\linewidth}{\wd\pandoc@box}%
  \ifdim\@tempb\p@<\@tempa\p@\let\@tempa\@tempb\fi% select the smaller of both
  \ifdim\@tempa\p@<\p@\scalebox{\@tempa}{\usebox\pandoc@box}%
  \else\usebox{\pandoc@box}%
  \fi%
}
% Set default figure placement to htbp
\def\fps@figure{htbp}
\makeatother


% definitions for citeproc citations
\NewDocumentCommand\citeproctext{}{}
\NewDocumentCommand\citeproc{mm}{%
  \begingroup\def\citeproctext{#2}\cite{#1}\endgroup}
\makeatletter
 % allow citations to break across lines
 \let\@cite@ofmt\@firstofone
 % avoid brackets around text for \cite:
 \def\@biblabel#1{}
 \def\@cite#1#2{{#1\if@tempswa , #2\fi}}
\makeatother
\newlength{\cslhangindent}
\setlength{\cslhangindent}{1.5em}
\newlength{\csllabelwidth}
\setlength{\csllabelwidth}{3em}
\newenvironment{CSLReferences}[2] % #1 hanging-indent, #2 entry-spacing
 {\begin{list}{}{%
  \setlength{\itemindent}{0pt}
  \setlength{\leftmargin}{0pt}
  \setlength{\parsep}{0pt}
  % turn on hanging indent if param 1 is 1
  \ifodd #1
   \setlength{\leftmargin}{\cslhangindent}
   \setlength{\itemindent}{-1\cslhangindent}
  \fi
  % set entry spacing
  \setlength{\itemsep}{#2\baselineskip}}}
 {\end{list}}
\usepackage{calc}
\newcommand{\CSLBlock}[1]{\hfill\break\parbox[t]{\linewidth}{\strut\ignorespaces#1\strut}}
\newcommand{\CSLLeftMargin}[1]{\parbox[t]{\csllabelwidth}{\strut#1\strut}}
\newcommand{\CSLRightInline}[1]{\parbox[t]{\linewidth - \csllabelwidth}{\strut#1\strut}}
\newcommand{\CSLIndent}[1]{\hspace{\cslhangindent}#1}



\setlength{\emergencystretch}{3em} % prevent overfull lines

\providecommand{\tightlist}{%
  \setlength{\itemsep}{0pt}\setlength{\parskip}{0pt}}



 


\setlength\heavyrulewidth{0ex}
\setlength\lightrulewidth{0ex}
\usepackage[automark]{scrlayer-scrpage}
\clearpairofpagestyles
\cehead{
  Brian Weatherson
  }
\cohead{
  The Role of Naturalness in Lewis’s Theory of Meaning
  }
\ohead{\bfseries \pagemark}
\cfoot{}
\makeatletter
\newcommand*\NoIndentAfterEnv[1]{%
  \AfterEndEnvironment{#1}{\par\@afterindentfalse\@afterheading}}
\makeatother
\NoIndentAfterEnv{itemize}
\NoIndentAfterEnv{enumerate}
\NoIndentAfterEnv{description}
\NoIndentAfterEnv{quote}
\NoIndentAfterEnv{equation}
\NoIndentAfterEnv{longtable}
\NoIndentAfterEnv{abstract}
\renewenvironment{abstract}
 {\vspace{-1.25cm}
 \quotation\small\noindent\emph{Abstract}:}
 {\endquotation}
\newfontfamily\tfont{EB Garamond}
\addtokomafont{disposition}{\rmfamily}
\addtokomafont{title}{\normalfont\itshape}
\let\footnoterule\relax

\makeatletter
\renewcommand{\@maketitle}{%
  \newpage
  \null
  \vskip 2em%
  \begin{center}%
  \let \footnote \thanks
    {\itshape\huge\@title \par}%
    \vskip 0.5em%  % Reduced from default
    {\large
      \lineskip 0.3em%  % Reduced from default 0.5em
      \begin{tabular}[t]{c}%
        \@author
      \end{tabular}\par}%
    \vskip 0.5em%  % Reduced from default
    {\large \@date}%
  \end{center}%
  \par
  }
\makeatother
\RequirePackage{lettrine}

\renewenvironment{abstract}
 {\quotation\small\noindent\emph{Abstract}:}
 {\endquotation\vspace{-0.02cm}}
\KOMAoption{captions}{tableheading}
\makeatletter
\@ifpackageloaded{caption}{}{\usepackage{caption}}
\AtBeginDocument{%
\ifdefined\contentsname
  \renewcommand*\contentsname{Table of contents}
\else
  \newcommand\contentsname{Table of contents}
\fi
\ifdefined\listfigurename
  \renewcommand*\listfigurename{List of Figures}
\else
  \newcommand\listfigurename{List of Figures}
\fi
\ifdefined\listtablename
  \renewcommand*\listtablename{List of Tables}
\else
  \newcommand\listtablename{List of Tables}
\fi
\ifdefined\figurename
  \renewcommand*\figurename{Figure}
\else
  \newcommand\figurename{Figure}
\fi
\ifdefined\tablename
  \renewcommand*\tablename{Table}
\else
  \newcommand\tablename{Table}
\fi
}
\@ifpackageloaded{float}{}{\usepackage{float}}
\floatstyle{ruled}
\@ifundefined{c@chapter}{\newfloat{codelisting}{h}{lop}}{\newfloat{codelisting}{h}{lop}[chapter]}
\floatname{codelisting}{Listing}
\newcommand*\listoflistings{\listof{codelisting}{List of Listings}}
\makeatother
\makeatletter
\makeatother
\makeatletter
\@ifpackageloaded{caption}{}{\usepackage{caption}}
\@ifpackageloaded{subcaption}{}{\usepackage{subcaption}}
\makeatother
\usepackage{bookmark}
\IfFileExists{xurl.sty}{\usepackage{xurl}}{} % add URL line breaks if available
\urlstyle{same}
\hypersetup{
  pdftitle={The Role of Naturalness in Lewis's Theory of Meaning},
  pdfauthor={Brian Weatherson},
  hidelinks,
  pdfcreator={LaTeX via pandoc}}


\title{The Role of Naturalness in Lewis's Theory of Meaning}
\author{Brian Weatherson}
\date{2013}
\begin{document}
\maketitle
\begin{abstract}
Many writers have held that in his later work, David Lewis adopted a
theory of predicate meaning such that the meaning of a predicate is the
most natural property that is (mostly) consistent with the way the
predicate is used. That orthodox interpretation is shared by both
supporters and critics of Lewis's theory of meaning, but it has recently
been strongly criticised by Wolfgang Schwarz. In this paper, I accept
many of Schwarz's criticisms of the orthodox interpretation, and add
some more. But I also argue that the orthodox interpretation has a grain
of truth in it, and seeing that helps us appreciate the strength of
Lewis's late theory of meaning.
\end{abstract}


\setstretch{1.1}
It is sometimes claimed (e.g., by (\citeproc{ref-Sider2001-SIDCOP}{Sider
2001a}, \citeproc{ref-Sider4D}{2001b};
\citeproc{ref-Stalnaker2004-JACLOI-2}{Stalnaker 2004};
\citeproc{ref-Williams2007}{Williams 2007};
\citeproc{ref-Weatherson2003-WEAWGA}{Weatherson 2003})) that David
Lewis's theory of predicate meaning assigns a central role to
naturalness.\footnote{Holton (\citeproc{ref-Holton2003Lewis}{2003}) is
  more nuanced, but does tell a similar story in the context of
  discussing Lewis's account of (potential) semantic indeterminacy.
  Weatherson (\citeproc{ref-Weatherson2010-VaI}{2010}) follows Holton in
  this respect.} Some of the people who claim this also say that the
theory they attribute to Lewis is true. The authors I have mentioned
aren't as explicit as each other about exactly which theory they are
attributing to Lewis, but the rough intuitive idea is that the meaning
of a predicate is the most natural property that is more-or-less
consistent with the usage of the predicate. Call this kind of
interpretation the `orthodox' interpretation of Lewis.\footnote{As some
  further evidence for how orthodox the `orthodox' interpretation is,
  note that Williams (\citeproc{ref-Williams2007}{2007}) is a prize
  winning essay published with two commentaries in the
  \emph{Philosophical Review}. That paper takes the orthodox
  interpretation as its starting point, and neither of the commentaries
  (Bays (\citeproc{ref-Bays2007}{2007}) and Hawthorne
  (\citeproc{ref-Hawthorne2007}{2007})) criticise this starting point.}
Recently Wolfgang Schwarz (\citeproc{ref-Schwarz2009}{2009, 209ff}) has
argued that the orthodox interpretation is a misinterpretation, and
actually naturalness plays a much smaller role in Lewis's theory of
meaning than is standardly assumed.\footnote{Schwarz
  (\citeproc{ref-Schwarz2006}{2006}) develops his criticism of orthodoxy
  in more detail, and in English, but it is as yet unpublished.}
Simplifying a lot, one key strand in Schwarz's interpretation is that
naturalness plays no role in the theory of meaning in
(\citeproc{ref-Lewis1969a}{Lewis 1969},
\citeproc{ref-Lewis1975b}{1975}), since Lewis hadn't formulated the
concept yet, and Lewis didn't abandon that theory of meaning, since he
never announced he was abandoning it, so naturalness doesn't play
anything like the role orthodoxy assigns to it.

In this article I attempt to steer a middle ground between these two
positions. I'm going to defend the following parcel of theses. These are
all exegetical claims, but I'm also interested in defending most of the
theses that I ultimately attribute to Lewis, so getting clear on just
what Lewis meant is of more than historical interest.

\begin{enumerate}
\def\labelenumi{\arabic{enumi}.}
\tightlist
\item
  Naturalness matters to Lewis's (post-1983) theory of sentence meaning
  only insofar as it matters to his theory of rationality, and the
  theory of rationality matters to the (pre- and post-1983) theory of
  meaning.
\item
  Naturalness might play a slightly larger role in Lewis's theory of
  word meaning, but it isn't nearly as significant as the orthodox view
  suggests.
\item
  When we work through Lewis's theory of word and sentence meaning, we
  see that the orthodox interpretation assigns to Lewis a theory that
  isn't his theory of meaning, but is by his lights a useful heuristic.
\item
  An even better heuristic than `meaning = use plus naturalness' would
  be `meaning = predication plus naturalness', but even this would be a
  fallible heuristic, not a theory.
\item
  When correctly interpreted, Lewis's theory is invulnerable to the
  challenges put forward in Williams
  (\citeproc{ref-Williams2007}{2007}).
\end{enumerate}

I'm going to start by saying a little about the many roles naturalness
plays in Lewis's philosophy, and about his big picture views on thought
and meaning. Then I'll offer a number of arguments against the orthodox
interpretation of Lewis's theory of sentence meaning. After that, I'll
turn to Lewis's theory of word meaning, where it is harder to be quite
clear about just what the theory is, and how much it might have changed
once natural properties were added to the metaphysics. An appendix
discusses some interpretative questions that arise if we are sceptical
that any one division of properties can do all the work that Lewis has
the natural/non-natural division do.

\section{How Naturalness Enters The Theory of
Meaning}\label{how-naturalness-enters-the-theory-of-meaning}

Most of the core elements of David Lewis's philosophy were present, at
least in outline, from his earliest work. The big exception is the
theory of natural properties introduced in Lewis
(\citeproc{ref-Lewis1983e}{1983}). As he says in that paper, he had
previously believed that ``set theory applied to possibilia is all the
theory of properties that anyone could ever need''
(\citeproc{ref-Lewis1983e}{Lewis 1983, 377n}). Once he introduces this
new concept of naturalness, Lewis puts it to all sorts of work
throughout his philosophy. I'm rather sceptical that there is any one
feature of properties that can do all the varied jobs Lewis wants
naturalness to do, but the grounds for, and consequences of, this
scepticism are a little orthogonal to the main theme of this paper, so
I've set it aside.

As the orthodox interpretation stresses, Lewis has naturalness do some
work in this theory of content. That he does think there's a connection
between naturalness and content is undeniable from the most casual
reading of his post-1983 work. But just how they are connected is less
obvious. To spell out these connections, let's start with three Lewisian
themes.

\begin{itemize}
\tightlist
\item
  Facts about linguistic meaning are to be explained in terms of facts
  about minds. In particular, to speak a language \(\mathcal{L}\) is to
  have a convention of being truthful and trusting in \(\mathcal{L}\)
  (\citeproc{ref-Lewis1969a}{Lewis 1969},
  \citeproc{ref-Lewis1975b}{1975}). And to have such a convention is a
  matter of having certain beliefs and desires. So mental content is
  considerably prior to linguistic content in a Lewisian theory.
  Moreover, Lewis's theory of linguistic content is, in the first
  instance, a theory of \emph{sentence} meaning, not a theory of
  \emph{word} meaning. \footnote{These points are stressed by Wolfgang
    Schwarz (\citeproc{ref-Schwarz2006}{2006},
    \citeproc{ref-Schwarz2009}{2009}). He also notes that in ``Putnam's
    Paradox'' Lewis explicitly sets these parts of his theory aside so
    he can discuss Putnam's arguments on grounds most favourable to
    Putnam. As Schwarz says, this should make us suspicious of the
    central role ``Putnam's Paradox'' plays in defences of the orthodox
    interpretation. We will return to this point in the section on
    textual evidence for and against orthodoxy.

    A referee notes, correctly, that the phrase `in the first instance'
    is doing a lot of work here. That's right; we'll return in much more
    detail below to Lewisian theories of word meaning, and what role
    naturalness plays in them.}
\item
  The principle of charity plays a central role in Lewis's theory of
  mental content (\citeproc{ref-Lewis1974c}{Lewis 1974},
  \citeproc{ref-Lewis1994b}{1994}). To a first approximation, a creature
  believes that \emph{p} iff the best interpretation of the creature's
  behavioural dispositions includes the attribution of the belief that
  \emph{p} to the creature. And, ceteris paribus, it is better to
  interpret a creature so that it is more rather than less rational. It
  will be pretty important for what follows that Lewis adopts a
  principle of charity that highlights \emph{rationality}, not
  \emph{truth}. It is also important to Lewis that we don't just
  interpret the individual creature, but creatures of a kind
  (\citeproc{ref-Lewis1980c}{Lewis 1980}). I'm not going to focus on the
  social externalist features of Lewis's theory of mental states, but I
  think they assist the broader story I want to tell.
\item
  Lewis's theory of mental content has it that mental contents are (what
  most of us would call) properties, not (what most of us would call)
  propositions (\citeproc{ref-Lewis1979b}{Lewis 1979}). So a theory of
  natural properties can easily play a role in the theory of mental
  content, since mental contents are properties. If you think mental
  contents are propositions, the connection between naturalness and
  mental content will be more indirect. Just how indirect it is will
  depend on what your theory of propositions is. But if mental contents
  are Lewisian propositions, the connection may be very indirect indeed.
  After all, propositions that we might pick out with sentences
  containing words that denote very unnatural properties, such as
  \emph{All emeroses are gred}, might be intuitively very natural.
\end{itemize}

Now let's see why we might end up with naturalness in the theory of
meaning. An agent has certain dispositions. For instance, after seeing a
bunch of green emeralds, and no non-green emeralds, in a large and
diverse range of environments, she has a disposition to say ``All
emeralds are green''. In virtue of what is she speaking a language in
which ``green'' means green, and not grue? (Note that when \emph{I} use
``grue'', I mean a property that only differs from greenness among
objects which it is easy to tell that neither our agent, nor any of her
interlocutors, could possibly be acquainted with at the time she makes
the utterance in question.)

Let's say that \(\mathcal{L}\)\textsubscript{1} is English, i.e., a
language in which ``green'' means green, and
\(\mathcal{L}\)\textsubscript{2} a language which is similar to
\(\mathcal{L}\)\textsubscript{1} except that ``green'' means grue. Our
question is, what makes it the case that the agent is speaking
\(\mathcal{L}\)\textsubscript{1} and not
\(\mathcal{L}\)\textsubscript{2}? That is, what makes it the case that
the agent has adopted the convention of being truthful and trusting in
\(\mathcal{L}\)\textsubscript{1}, and not the convention or being
truthful and trusting in \(\mathcal{L}\)\textsubscript{2}?

We assumed that the agent has seen a lot of emeralds which are both
green and grue. To a first approximation, it is more charitable to
attribute to the agent the belief that all emeralds are green than the
belief that all emeralds are grue because greenness is more natural than
gruesomeness. As Lewis says, ``The principles of charity will impute a
bias towards believing things are green rather than grue''
(\citeproc{ref-Lewis1983e}{1983, 375}). And for Lewis, charity requires
imputing more reasonable interpretations. But why is it more charitable
to attribute beliefs about greenness to beliefs about grueness? I think
it is because we need more evidence to rationally form a belief that
some class of things are all grue than we need to form a belief that
everything in that class is green. And that's because, ceteris paribus,
we need more evidence to rationally form a belief that all \emph{F}s are
\emph{G}s than that all \emph{F}s are \emph{H}s when \emph{G} is less
natural than \emph{H}. The agent has, we might assume, sufficient
evidence to rationally believe that all emeralds are green, but not
sufficient evidence to believe that all emeralds are grue.

So the first two Lewisian themes notes above, the reduction of
linguistic meaning to mental content, and the centrality of a
rationality-based principle of charity, push us towards thinking that
naturalness is closely connected to mental content and hence to
linguistic meaning. And it has pushed us towards thinking that if
naturalness is connected to meaning, it is via this connection I've
posited between naturalness and rational belief. Note that Lewis doesn't
ever endorse anything like that general a connection, but I suspect he
had something like this in mind when he wrote the sentence I quoted in
the previous paragraph. We'll come back to this interpretative question
at some length below.

But the argument I offered was a bit quick, because I ignored the third
Lewisian theme: beliefs are relations to properties, not propositions.
On Lewis's theory, to believe that all emeralds are green is to
self-ascribe the property of being in a world where all emeralds are
green. So if a certain body of evidence makes it possible for the agent
to rationally believe that all emeralds are green, but not for her to
believe that all emeralds are grue, and that's because rationality is
constitutively connected to naturalness, then that must be because the
first of the following properties is more natural than the second:

\begin{itemize}
\tightlist
\item
  Being in a world where all emeralds are green
\item
  Being in a world where all emeralds are grue
\end{itemize}

That could still be true, though it is notable how far removed we are
from the intuitions that motivate the distinctions between more and less
natural properties. It's not like there is some sense, intuitively, in
which things that have the first property form a more unified class than
things that have the second property.

So it's plausible that naturalness is connected to mental content, at
least as long as naturalness is connected to rational belief. And since
mental content is connected to linguistic content, we're now in the
vicinity of the orthodox interpretation. But I don't think the orthodox
interpretation can be right. I'll give four reasons for this, starting
with the textual evidence for and against it.

\section{Textual Evidence about Sentence
Meaning}\label{textual-evidence-about-sentence-meaning}

There is some \emph{prima facie} textual evidence for the orthodox
interpretation. But looking more careful at the context of these texts
not just undermines the support the text gives to the orthodox
interpretation, but actually tells against it. (This part of the paper
is indebted even more than the rest to Wolfgang Schwarz's work, and
could be easily skipped by those familiar with that work.)

I'll focus on the last seven pages of ``New Work for a Theory of
Universals''. This is the part of ``New Work'' that uses the notion of
naturalness, as introduced in the paper, to respond to Putnam's
model-theoretic arguments for massive indeterminacy of meaning. Lewis
actually responds to Putnam twice over. First, he responds to Putnam
directly, by showing how adding naturalness to a use-based theory of
sentence meaning avoids the `just more theory' objection that's central
to Putnam's argument. And when Lewis describes this direct response, he
says things that sound a lot like the orthodox interpretation.

\begin{quote}
I would instead propose that the saving constraint concerns the referent
- not the referrer, and not the causal channels between the two. It
takes two to make a reference, and we will not find the constraint if we
look for it always on the wrong side of the relationship. Reference
consists in part of what we do in language or thought when we refer, but
in part it consists in eligibility of the referent. And this eligibility
to be referred to is a matter of natural properties.
(\citeproc{ref-Lewis1983e}{Lewis 1983, 371})
\end{quote}

But after this direct response is finished, Lewis notes that he has
conceded quite a lot to Putnam in making the response.

\begin{quote}
You might well protest that Putnam's problem is misconceived, wherefore
no need has been demonstrated for resources to solve it. \ldots{} Where
are the communicative intentions and the mutual expectations that seem
to have so much to do with what we mean? In fact, where is thought?
\ldots I think the point is well taken, but I think it doesn't matter.
If the problem of intentionality is rightly posed there will still be a
threat of radical indeterminacy, there will still be a need for saving
constraints, there will still be a remedy analogous to Merrill's
suggested answer to Putnam, and there will still be a need for natural
properties. (\citeproc{ref-Lewis1983e}{Lewis 1983, 373})
\end{quote}

I noted earlier that Schwarz makes much of a similar passage in
``Putnam's Paradox'', and I think he is right to do so. Here's a crucial
quote from that paper.

\begin{quote}
I shall acquiesce in Putnam's linguistic turn: I shall discuss the
semantic interpretation of language rather than the assignment of
content to attitudes, thus ignoring the possibility that the latter
settles the former. It would be better, I think, to start with the
attitudes and go on to language. But I think that would relocate, rather
than avoid, the problem; wherefore I may as well discuss it on Putnam's
own terms. (\citeproc{ref-Lewis1984b}{Lewis 1984, 222})
\end{quote}

That passage ends with a footnote where he says the final section of
``New Work'' contains a version of how the `relocated' problem would be
solved. So let's turn back to that. The following long portmanteau quote
from pages 373 to 375 captures, I think, the heart of my interpretation.

\begin{quote}
The problem of assigning content to functionally characterised states is
to be solved by means of constraining principles. Foremost among these
are principles of fit. \ldots A state typically caused by round things
before the eyes is a good candidate for interpretation as the visual
experience of confronting something round; and its typical impact on the
states interpreted as systems of belief ought to be interpreted as the
exogenous addition of a belief that one is confronting something round,
with whatever adjustment that addition calls for. \ldots Call two worlds
equivalent iff they are alike in respect of the subject's evidence and
behaviour, and note that any decent world is equivalent inter alia to
horrendously counterinductive worlds and to worlds where everything
unobserved by the subject is horrendously nasty. \ldots We can
interchange equivalent worlds ad lib and preserve fit. So, given any
fitting and reasonable interpretation, we can transform it into an
equally fitting perverse interpretation by swapping equivalent worlds
around \ldots If we rely on principles of fit to do the whole job, we
can expect radical indeterminacy of interpretation. We need further
constraints, of the sort called principles of (sophisticated) charity,
or of `humanity'. {[}A footnote here refers to ``Radical
Interpretation''.{]} Such principles call for interpretations according
to which the subject has attitudes that we would deem reasonable for one
who has lived the life that he has lived. (Unlike principles of crude
charity, they call for imputations of error if he has lived under
deceptive conditions.) These principles select among conflicting
interpretations that equally well conform to the principles of fit. They
impose \emph{apriori} -- albeit defeasible - presumptions about what
sorts of things are apt to be believed and desired \ldots{}\textbf{It is
here that we need natural properties.} The principles of charity will
impute a bias toward believing that things are green rather than grue
\ldots In short, they will impute eligible content \ldots They will
impute other things as well, but it is the imputed eligibility that
matters to us at present. (\citeproc{ref-Lewis1983e}{Lewis 1983,
373--75}, my emphasis)
\end{quote}

I think that does a reasonably clear job of supporting the
interpretation I set out in the introduction over the orthodox
interpretation. Naturalness matters to linguistic meaning all right. But
the chain of influence is very long and indirect. Naturalness constrains
what is reasonable, reasonableness constrains charitable
interpretations, charitable interpretations constrain mental content,
and mental content constrains linguistic content. Without naturalness at
the first step, we get excessive indeterminacy of content. With it, the
Putnamian problems are solved. But there's no reason to think
naturalness has any more direct role to play at any level in the theory
of linguistic content.

In short, Lewis changed what he thought about rationality when he
adopted the theory of natural properties. Since rationality was a part
of his theory of mental content, and mental content determines
linguistic content, this change had downstream consequences for what he
said about linguistic content. But there wasn't any other way his theory
of linguistic content changed, nor, contra orthodoxy, any direct link
between naturalness and predicate meaning.

Moreover, when we look at the closest thing to a worked example in Lewis
(\citeproc{ref-Lewis1983e}{1983}), we don't get any motivation for the
orthodox interpretation. Here's the example he uses, which concerns
mental content. Let \emph{f} be any mapping from worlds to worlds such
that the agent has the same evidence and behaviour in \emph{w} and
\emph{f}(\emph{w}). Extend \emph{f} to a mapping from sets of worlds to
sets of worlds in the following (standard) way: \emph{f}(\emph{S}) =
\{\emph{f}(\emph{w}): \emph{w}~∈~\emph{S}\}. Then the agent's behaviour
will be rationalised by her evidence just as much if she has credence
function \emph{C} and value function \emph{V}, as if she has credence
function \emph{C}′ and value function \emph{V}′, where
\emph{C}′(\emph{f}(\emph{S})) = \emph{C}(\emph{S}), and
\emph{V}′(\emph{f}(\emph{S})) = \emph{V}(\emph{S}). To relate this back
to the familiar Goodmanian puzzle, let \emph{f} map any world where all
emeralds are green to nearest world where all emeralds are grue, and
vice versa, and map any other world to itself. Then the above argument
will say that the agent's behaviour is rationalised by her evidence just
as much as if her credences are \emph{C} as if they are \emph{C}′. That
is, her behaviour is rationalised by her evidence just as much if she
gives very high credence to all emeralds being green as to all emeralds
being grue. So understanding charity merely as rationalizing behaviour
leaves us without a way to say that the agent believes unobserved
emeralds are green and not grue.

Lewis's solution is to say that charity requires more than that. In
particular, it requires that we assign natural rather than unnatural
beliefs to agents where that is possible. I've argued above that this
makes perfect sense if we connect naturalness with rationality. The
crucial thing to note here is that this all happens a long time before
we can set out the way that a sentence is used, since the way a sentence
is used on Lewis's theory of linguistic content includes the beliefs
that are formed on hearing it. So the discussion in ``New Work''
suggests that naturalness matters for content, but not in a way that can
be easily factorised out. And that's exactly what I think is the best
way to understand Lewis's theory.

\section{Textual Evidence and Naturalness and
Rationality}\label{textual-evidence-and-naturalness-and-rationality}

A major part of my argument above was that naturalness affected Lewis's
theory of rationality. In particular, once he had naturalness to work
with, he seemed to think that it was more rational to project natural
rather than unnatural properties. The textual evidence for this is, I'll
admit, fragmentary. But it is fairly widespread. Let's start with a
quote we've already seen.

\begin{quote}
The principles of charity will impute a bias toward believing that
things are green rather than grue (\citeproc{ref-Lewis1983e}{Lewis 1983,
375})
\end{quote}

As noted above, I assume this isn't a special feature of green and grue,
but rather that there is a general principle in favour of projecting
natural properties. But it would be good to have more evidence for that.

Lewis returns to the example of the believer in grue emeralds a few
times. Here is one version of the story in \emph{Plurality}.

\begin{quote}
We think that some sorts of belief and desire \ldots{} would be
unreasonable in a strong sense \ldots{} utterly unintelligible and
nonsensical. Think of the man who, for no special reason, expects
unexamined emeralds to be grue. \ldots{} What makes the perversely
twisted assignment of content incorrect, however well it fits the
subject's behaviour, is exactly that it assigns ineligible, unreasonable
content when a more eligible assignment would have fit behaviour equally
well. (\citeproc{ref-Lewis1986a}{Lewis 1986, 38--39})
\end{quote}

And a little later, when replying to Kaplan's paradox, he says,

\begin{quote}
Given a fitting assignment, we can scramble it into an equally fitting
but perverse alternative assignment. Therefore a theory of content needs
a second part: as well as principles of fit, we need `principles of
humanity', which create a presumption in favour of some sorts of content
and against others. (\citeproc{ref-Lewis1986a}{Lewis 1986, 107})
\end{quote}

He returns to this point again in ``Reduction of Mind''.

\begin{quote}
{[}Folk psychology{]} sets presumptive limits on what our contents of
belief and desire can be. Self-ascribed properties may be `far from
fundamental', I said -- but not \emph{too} far. Especially gruesome
gerrymanders are \emph{prima facie} ineligible to be contents of belief
and desire. In short, folk psychology says that we make sense. It
credits us with a modicum of rationality in our acting, believing and
desiring. (\citeproc{ref-Lewis1994b}{Lewis 1994, 320})
\end{quote}

The running thread through these last three quotes is that our theory of
mental content rules out gruesome assignments, and it does this because
assigning rationality is constitutive of correctly interpreting. This
can only work if naturalness is connected to rationality. I've
attributed a stronger claim to Lewis, that not only is naturalness
connected to rationality, but that the connection goes through
projection.\footnote{The view I'm attributing to Lewis is endorsed by
  one prominent supporter of the orthodox interpretation, namely Ted
  Sider. See his (\citeproc{ref-Sider2012}{2011, 35ff}).}

One piece of evidence for that is that Lewis says, in ``Meaning Without
Use'' that Kripkenstein's challenge was ``formerly Goodman's challenge''
(\citeproc{ref-Lewis1992a}{Lewis 1992, 109}). He goes on to say that the
solution to this challenge (or should that be `these challenges')
involves ``carrying more baggage of primitive distinctions or
ontological commitments than some of us might have hoped''
(\citeproc{ref-Lewis1992a}{Lewis 1992, 110}). A footnote on that
sentence cites ``New Work'', in case it isn't obvious that the baggage
here is the distinction between natural and unnatural properties. So
somehow, Lewis thinks that natural properties help solve Goodman's
puzzle. I think that the simplest such solution is the right one to
attribute to Lewis; natural properties are \emph{prima facie} more
eligible to be projected.

A referee noted that this passage is a little odd; it appears to simply
conflate a meta-semantical paradox with an epistemological paradox. But
I think that just shows how much, for Lewis, meta-semantical questions
are epistemological questions. Words get their meanings in virtue of our
conventions. Our conventions consist of our beliefs and desires. And
facts about rationality are, in part, constitutive of what we believe
and desire.

Finally, consider the way in which the papers on natural properties are
introduced in \emph{Papers in Metaphysics and Epistemology}. Lewis says
that ``I had been persuaded by Goodman and others that all properties
were equal: it was hopeless to try to distinguish `natural' properties
from gruesomely gerrymandered, disjunctive properties.''
(\citeproc{ref-Lewis1999a}{Lewis 1999, 1--2}) A footnote refers to
\emph{Fact, Fiction and Forecast}. Of course, the point of ``New Work''
is that Lewis abandons this, explicitly Goodmanian, view. Now that he
had learned property egalitarianism from Goodman of course doesn't show
that once he became a property inegalitarian, he applied this to
Goodman's own paradox. But it does seem striking that the only citation
of an egalitarian view is of \emph{Fact, Fiction and Forecast}. I take
that to be some, inconclusive, evidence that Lewis did indeed think
natural properties were related to Goodman's paradox.

Ultimately, it seems the textual evidence is this. There are many
different occasions where Lewis makes clear there is a connection
between naturalness and rationality, and in particular, between
naturalness and the kind of rationality that is relevant to content
assignment. There are hints that this connection goes via naturalness
playing a role in solving Goodman's paradox. Notably, there is no other
obvious way in which naturalness could connect to rationality. At least,
I can neither think of another connection, nor see any evidence for
another connection in the Lewis corpus. So I conclude, a little
tentatively, that Lewis thought natural properties had a role to play in
solving Goodman's paradox.

\section{Word Meaning and
Naturalness}\label{word-meaning-and-naturalness}

In ``Languages and Language'', Lewis doesn't say that human linguistic
practices merely determine truth conditions for the spoken sentences.
That is, our linguistic practices don't merely determine which
\textbf{language}, in Lewis's sense, we speak. They also determine, to
some extent, a \textbf{grammar}, which specifies the truth conditional
contribution of the various parts of the sentence. The grammar
determines the ``fine structure of meaning''
(\citeproc{ref-Lewis1975b}{Lewis 1975, 177}) of a sentence or phrase.

In comments on an earlier draft of this paper, an anonymous referee
stressed that naturalness could enter directly into a theory of meaning
once we stopped focussing on sentence meaning, and started looking on
word meaning. I don't mean to say the referee was endorsing any
particular role for naturalness in the theory of word meaning. But the
point that we need to say more about the Lewisian approach to word
meaning before we conclude that naturalness is only indirectly related
to meaning is right. And I'm grateful for the encouragement to discuss
it further.

Lewis has a short discussion of grammars in ``Languages and Language'',
and another in ``Radical Interpretation''. It's worth looking at both of
these in turn. I'll take ``Languages and Language'' first, since even
though it has a slightly later publication date, in the respects we're
discussing here it closely resembles the theory in \emph{Convention}.

On pages 177-8 of that paper, Lewis notes three ways in which there may
be indeterminacy in the grammar.

\begin{enumerate}
\def\labelenumi{\arabic{enumi}.}
\tightlist
\item
  A subject's behavioural dispositions and anatomy might underdetermine
  their beliefs and desires.
\item
  The beliefs and desires might underdetermine the truth conditions of
  their language.
\item
  The truth conditions of the language might underdetermine the meanings
  of the individual words.
\end{enumerate}

While Lewis does not think the second is actually a source of
indeterminacy, he does think that the third is.

\begin{quote}
My present discussion has been directed at the middle step \ldots{} I
have said \ldots{} that the beliefs and desires of the subject and his
fellows are such as to comprise a fully determinate convention of
truthfulness and trust in some definite language. \ldots{} I am inclined
to share in Quine's doubts about the determinacy of the third step.
(\citeproc{ref-Lewis1975b}{Lewis 1975, 178})
\end{quote}

Lewis gives reasons for this inclination a few paragraphs earlier. He
says that while we can say what it is for a community to speak one
language rather than another, we can't say what it is for a community to
speak one grammar rather than another. He says that we don't have any
objective measures for evaluating grammars. And he says Quine's examples
of indeterminacy of reference show that languages can have multiple good
grammars, even if these disagree radically about the meaning of some
constituents.

Notably, Lewis doesn't take to show that there is anything wrong with
the notion of word meaning. He says it would be ``absurd'' (177) to
conclude that. His conclusion here is more one of modesty rather than
philosophical scepticism. We don't know how to extend the theory of
sentence meaning he offers to a theory of word meaning, so we should do
what we can without talking about word meaning.

The approach in ``Radical Interpretation'' has a bit more of a hint for
how to restore semantic determinacy. The subject matter of that paper is
how to solve for the mental and linguistic contents of a speaker, called
Karl, given the physical facts about them. Lewis uses \textbf{M} for ``a
specification, in our language, of the meanings of expressions of Karl's
language.'' (\citeproc{ref-Lewis1974c}{Lewis 1974, 333}) He lists a
number of constraints on a solution, including early versions of his
principles of constitutive rationality. But the most notable constraint,
from our perspective, is this:

\begin{quote}
\emph{The Principle of Generativity} constrains \textbf{M}: \textbf{M}
should assign truth conditions to the sentences of Karl's language in a
way that is at least finitely specifiable, and preferably also
reasonably uniform and simple. (\citeproc{ref-Lewis1974c}{Lewis 1974,
339})
\end{quote}

There's something very odd about this. Lewis, in 1974, didn't have a
theory of what made an assignment simple. He needed his theory of
natural properties to do that. Or, at least, once he had the theory of
natural properties, it did all the work he ever wanted out of an account
of simplicity.

Be that as it may, it does suggest that Lewis did think that simplicity
of assignments could be used as a way of cutting down the third kind of
semantic indeterminacy discussed in ``Languages and Language''. He
doesn't think it would generate a fully determinate interpretation of
Karl's language.

\begin{quote}
It seems hopeless to deny, in the face of such examples as those in
{[}Quine's ``Ontological Relativity'', pp.~30-39{]}, that the truth
conditions of full sentences in \textbf{M} do not sutfice to determine
the rest of \textbf{M}: the parsings and the meanings of the
constituents of sentences. At least, that is so unless there is
something more than our Principle of Generativity to constrain this
auxiliary syntactic and semantic apparatus.
(\citeproc{ref-Lewis1974c}{Lewis 1974, 342--43})
\end{quote}

It's notable that some of the examples Quine gives in ``Ontological
Relativity'' are not cases where the alternative meanings are by any
measure equally natural. This positive allusion to Quine's examples
suggests a link to this comment in ``Languages and Language''

\begin{quote}
We should regard with suspicion any method that purports to settle
objectively whether, in some tribe, ``gavagai'' is true of temporally
continuant rabbits or time-slices thereof. You can give their language a
good grammar of either kind---and that's that.
(\citeproc{ref-Lewis1975b}{Lewis 1975, 177})
\end{quote}

Note that he doesn't say `equally' good. And note also how this
contrasts with the attitude he takes towards the prospects of
indeterminacy in sentence meaning. I earlier quoted him saying that part
of the point of ``Languages and Language'' was to show how the second
type of indeterminacy didn't arise. He ends ``Radical Interpretation''
with this `credo'.

\begin{quote}
Could indeterminacy of beliefs, desires, and truth conditions also arise
because two different solutions both fit all the constraints perfectly?
Here is the place to hold the line. This sort of indeterminacy has not
been shown by convincing examples, and neither could it be shown--to
me--by proof. \emph{Credo}: if ever you prove to me that all the
constraints we have yet found could permit two perfect solutions,
differing otherwise than in the auxiliary apparatus of \textbf{M}, then
you will have proved that we have not yet found all the constraints.
(\citeproc{ref-Lewis1974c}{Lewis 1974, 343})
\end{quote}

So that's where things stood before 1983. Lewis thought he had a theory
that eliminated, or at least minimised, indeterminacy at the level of
truth conditions. But he didn't think his theory eliminated
indeterminacy, even quite radical indeterminacy, in word meanings. And
he didn't seem bothered by this aspect of the theory; indeed, he thought
Quine's arguments showed that we shouldn't eliminate this kind of
indeterminacy.

This attitude towards Quinean arguments for indeterminacy is obviously a
striking contrast to the forcefulness, and rapidity, with which he
responded to Putnam's arguments for indeterminacy. That shouldn't be too
surprising once we attend to Lewis's threefold distinction between kinds
of indeterminacy. Quine was arguing that indeterminacy of the third kind
was rampant. Putnam was arguing that indeterminacy of the second kind
was rampant. And, as Lewis announced in ``Radical Interpretation'', he
wasn't going to believe any such argument.

Still, we might wonder whether the resources he brought to bear in
responding to Putnam also help respond to Quine. Or, perhaps more
importantly for exegetical reasons, we might wonder whether Lewis
thought they were useful in responding to Quine. The evidence from ``New
Work'' seems to suggest a negative answer to the latter question. Lewis
never says that one of the things you can do with the distinction
between natural and unnatural properties is respond to arguments for
Quinean indeterminacy. And that's despite the fact that ``New Work'' has
a very survey-like feel; the bulk of the paper is a long list of
philosophical work that a theory of universals can do.

In ``Putnam's Paradox'' there is a brief footnote on Quine's arguments
for indeterminacy. It reads

\begin{quote}
It is not clear how much indeterminacy might be expected to remain. For
instance, what of Quine's famous example? His rabbit-stages, undetached
rabbit parts, and rabbit-fusion seem only a little, if any, less
eligible than rabbits themselves. (\citeproc{ref-Lewis1984b}{Lewis 1984,
228n})
\end{quote}

As I've stressed repeatedly, following Schwarz, taking the disclaimers
at the start of ``Putnam's Paradox'' seriously means that we have to be
careful in interpreting what Lewis says about how words acquire
determinate meaning in that paper. But even before we adjust for the
disclaimers, this is hardly a ringing rejection of Quine's indeterminacy
arguments. The contrast to Lewis's attitude towards Putnam's arguments
is striking. Since it is the very same contrast that we saw in both
``Languages and Language'' and ``Radical Interpretation'', I think it is
fair to assume that he continued to think Quine's arguments were
considerably stronger than Putnam's.

But there is, perhaps, a change of view in ``Meaning Without Use''.
Here's the problem Lewis addresses at the end of that paper. Let
\(\mathcal{L}\)\textsubscript{1} once again be English as we currently
understand it, and let \(\mathcal{L}\)\textsubscript{3} be just like
English, except that it doesn't assign any truth conditions to sentences
over a thousand words long.\footnote{If you think sentences with a
  thousand words are too easy to understand for the argument of this
  paragraph, make the threshold higher; as long as the threshold is
  finite, it won't affect the argument.} Do our actual linguistic
practices manifest a convention of trust in
\(\mathcal{L}\)\textsubscript{1}, or trust in
\(\mathcal{L}\)\textsubscript{3}? Lewis argues that it is more like a
convention of trust in \(\mathcal{L}\)\textsubscript{3}. If someone
utters a very long sentence, we expect some kind of performance error,
at best. We don't, in general, believe what they say. So the theory of
``Languages and Language'' seems to predict that these long sentences
have no truth conditions. But that's wrong, so the theory must be
corrected.

Lewis's correction appeals, it seems, to natural properties in fixing a
grammar. He says that linguistic practice determines truth conditions
for a fragment of the language that is widely used. Those truth
conditions determine meanings of words. This determination requires
natural properties; without them the Quinean problems multiply
indefinitely. We then use those word meanings to determine the meaning
of unused sentences. A long footnote suggests that the procedure might
not be restricted to unused sentences. As long as there is a large
enough fragment in which there are conventions of truthfulness and
trust, we can extrapolate from that to other parts of the language that
are used.

This is a marked deviation from anything Lewis had said until then. From
the earliest writings, he had stressed a step-by-step approach to
content determination. Behavioural dispositions plus physical and
biological constraints determine mental content; mental content
determines sentence meaning; and sentence meaning determines word
meaning. In ``Meaning Without Use'', it seemed the last two steps were
being somewhat merged.

But we shouldn't overstate how much the third step was allowed to
encroach on the second. Lewis does think we need to rule out `bent'
grammars, which don't assign any truth conditions to sentences over a
thousand words long, or which give sentences different meanings to what
we'd expect if the word `cabbage' appears forty times. But he doesn't
think we need to rule out any `straight' grammar, which includes ``any
grammar that any linguist would actually propose.''
(\citeproc{ref-Lewis1992a}{Lewis 1992, 109})

So Lewis's focus here is to rule out unnatural \emph{compositional
rules}, not unnatural assignments of content to individual words. The
reference to linguists here might be useful. Linguists tend to spend
much more time on compositional rules than they do on the contents on
individual predicates. Notably, Quine didn't argue for indeterminacy by
positing indeterminacy in the compositional rules of the language; his
non-standard interpretations all share a standard syntax. If we posit
that Lewis thought that there was little syntactic indeterminacy in the
language, like there is little indeterminacy at the level of truth
conditions of sentences, we can tell a story that doesn't involve too
many unsignalled changes of view. Here's how I would tell that story in
some more detail.

Lewis's early view, expressed clearly in ``Radical Interpretation'' and
``Languages and Language'', and not retracted before, I think, 1992, has
the following parts:

\begin{enumerate}
\def\labelenumi{\arabic{enumi}.}
\tightlist
\item
  Conventions of truthfulness and trust determine (very sharply) truth
  conditions for sentences in a speaker's language.
\item
  Any reasonably good grammar, i.e., assignment of word meanings and
  compositional rules, that is consistent with the truth conditions is
  not determinately wrong. There is potentially substantial
  indeterminacy in the meaning of any given word, because there are many
  reasonably good grammars consistent with the truth conditions.
\end{enumerate}

After 1983, `simplicity' was understood in terms of naturalness, but
otherwise the story doesn't change a lot.

The later view, which goes by somewhat more quickly in ``Meaning Without
Use'', has the following parts:

\begin{enumerate}
\def\labelenumi{\arabic{enumi}.}
\tightlist
\item
  Conventions of truthfulness and trust in (the bulk of) the used
  fragment of the language determine truth conditions for that fragment.
\item
  Naturalness considerations determine the compositional rules for the
  language by extrapolation from that grammar.
\item
  Word meanings are determined, so far as they are determinate, by the
  truth conditions for sentences, plus the compositional rules.
\item
  Truth conditions for sentences outside the used fragment are
  determined by the word meanings and the compositional rules.
\end{enumerate}

Neither of these views look much like the orthodox view. Remember that
the orthodox view has it that considerations of naturalness can be used
to resolve debates in metaphysics. That's certainly the use that Sider
(\citeproc{ref-Sider2001-SIDCOP}{2001a}) makes of the orthodox view. But
on the early view, simplicity considerations only come in after the
truth conditions for every sentence have been determined, and hence so
that all debates are settled. And on the later view, simplicity
considerations primarily are used to settle truth conditions for unused,
or at least unusual, sentences.

Now if you thought the salient fragment in point 1 of the later view was
small, and if you thought naturalness had a major role to play in step 3
of the later view, you would get back to something like the orthodox
view. But I don't see the textual evidence for either of those
positions. Lewis says that ``the used fragment is large and varied.''
(\citeproc{ref-Lewis1992a}{Lewis 1992, 110}) It doesn't look like he is
positing wholesale changes to his view on the determination of truth
conditions. He is positing some changes; the last two pages of the paper
are clearly marked as deviations from his earlier position. But both the
examples he uses and the rhetoric around them suggests that the bulk of
the changes happen at point 2. Naturalness considerations constraint the
syntax of a language much more tightly than they constrain the
assignment of meaning to a given word. In sum, at no point in the
evolution of his views did Lewis seem to endorse the orthodox
interpretation, even as a theory of word meaning.

\section{An Argument for the Orthodox
Interpretation}\label{an-argument-for-the-orthodox-interpretation}

So far I've argued that there is no solid textual support for the
orthodox interpretation. My rival interpretation relied on there being a
connection between naturalness and induction, and as we've just seen,
there is some textual evidence for this. But perhaps there is a more
indirect way to motivate the orthodox interpretation of Lewis. The
orthodox interpretation attributes to Lewis a theory that is quite
attractive as a theory of semantic determinacy and indeterminacy. Call
that theory the \textbf{U\&N Theory}, short for the \textbf{U}se plus
\textbf{N}aturalness theory of meaning. Since Lewis was clearly looking
for such a theory when he discussed naturalness in the context of his
theory of content, it is reasonably charitable to attribute the
\textbf{U\&N Theory}~to him, as the orthodox interpretation does.

My response to this will be in three parts. First, I'll argue in this
section that my rival interpretation attributes to Lewis a theory of
semantic determinacy and indeterminacy that does just as well at
capturing the facts Lewis wanted a theory to capture, so there's no
charity based reason to attribute the \textbf{U\&N Theory}~to him (And,
as we saw in the previous section, there's no direct textual reason to
attribute it to him either.) Second, the \textbf{U\&N Theory}~is subject
to the criticisms in Williams (\citeproc{ref-Williams2007}{2007}), while
the theory I attribute to Lewis is not. Third, the \textbf{U} part of
the \textbf{U\&N Theory}~is hopelessly vague; it isn't clear how to say
what `use' is on a Lewisian theory that makes it suitable to add to
naturalness to deliver meanings. Either use is so thick that naturalness
is unneeded, or it is so thin that naturalness won't be sufficient to
set meaning. So actually it isn't particularly charitable to attribute
this theory to him.

Still, let's start with the attractions of the \textbf{U\&N Theory}. On
the one hand, agents are inclined to say ``All emeralds are green'' both
in situations where they've seen a lot of green emeralds (and no
non-green ones) and in situations where they've seen a lot of grue
emeralds (and no non-grue ones). That's because, of course, those are
exactly the same situations. So at first glance, it doesn't look like
the way in which ``green'' is used will determine whether it means green
or grue. On the other hand, once we add a requirement that terms have a
relatively natural meaning, we do get this to fall out as a result.
Moreover we can even see how this falls out of a recognisably Lewisian
approach to meaning.

Consider again our agent who says ``All emeralds are green'' after
seeing a lot of emeralds that are both green and grue. And remember that
for her to speak a language, she must typically conform to conventions
of truthfulness and trust in that language. Now if the agent was
speaking \(\mathcal{L}\)\textsubscript{2}, she would have to think that
she's doing an OK job of being truthful in
\(\mathcal{L}\)\textsubscript{2} by saying ``All emeralds are green''.
But that would be crazy. Why should she think that all emeralds are grue
given her evidence base? To attribute to her that belief would be to
gratuitously attribute irrational beliefs to her. And on Lewis's
picture, gratuitous attributions of irrationality are false. So the
agent doesn't have that belief. So she's not speaking
\(\mathcal{L}\)\textsubscript{2}.

Things are even clearer from the perspective of hearers. A hearer of
``All emeralds are green'' would be completely crazy to come to believe
that all emeralds are grue. The hearer knows, after all, that the
speaker has no acquaintance with the emeralds that would have to be blue
for all emeralds to be grue. So the hearer knows that this utterance
could not be sufficient evidence to believe that all emeralds are grue.
Yet if she speaks \(\mathcal{L}\)\textsubscript{2}, she is disposed to
believe that all emeralds are grue on hearing ``All emeralds are
green''. She isn't irrational, or at least we shouldn't assign
irrationality to her so quickly, so she doesn't speak
\(\mathcal{L}\)\textsubscript{2}.

So it looks like in this one case at least, we have a case where use
plus naturalness gives us the right theory. Agents are disposed to use
``green'' to describe emeralds that are green/grue. But the fact that
greenness is more natural than gruesomeness makes it more appropriate to
attribute to them a convention according to which ``All emeralds are
green'' means that all emeralds are green and not that all emeralds are
grue.

But more carefully, what we should say is that the \textbf{U\&N
Theory}~gives us the right result in this case. It doesn't follow that
it will work in all cases, or anything like it. And it doesn't follow
that it works for the right reasons. As we'll see, neither of those
claims are true. In fact, just re-reading the last three paragraphs
should undermine the second claim. Because we just saw a derivation that
the agents are not speaking \(\mathcal{L}\)\textsubscript{2}, that
didn't even appeal to the \textbf{U\&N Theory}. Rather, that derivation
simply used the theory of meaning in \emph{Convention} and the theory of
mental content in ``Radical Interpretation''. It's true that the latter
theory assigns a special role to rationality, and the theory of
rationality we used has, among other things, a role for natural
properties, but that is very different to the idea that naturalness
feeds directly into the theory of meaning in the way the orthodox
interpretation says. As I said at the start, I think the best
interpretation of Lewis is that he changed his theory of
\emph{rationality} in 1983, but that's the only change to his theory of
\emph{meaning}.

Put another way, these reflections on ``green'' and ``grue'' are
consistent with the view that the \textbf{U\&N Theory}~is a false
\emph{theory}, but a useful \emph{heuristic}. It's a useful heuristic
because it agrees with the true Lewisian theory in core cases, and is
much easier to apply. That's exactly what I think the \textbf{U\&N
Theory}~is, both as a matter of fact, and as a matter of Lewis
interpretation.

\section{Indeterminacy and Radically Deviant
Interpretations}\label{indeterminacy-and-radically-deviant-interpretations}

If the \textbf{U\&N Theory}~is a heuristic not a theory, we should
expect that it will break down in extreme cases. That's exactly what we
see in the cases discussed in Williams
(\citeproc{ref-Williams2007}{2007}). Those cases highlight the fact that
a Lewisian theorist needs to be careful that we don't end up concluding
that normal people, such as the agent in our example who says ``All
emeralds are green'', speak \(\mathcal{L}\)\textsubscript{4}.
\(\mathcal{L}\)\textsubscript{4} is a language in which all sentences
express claims about a particular mathematical model (essentially a
Henkin model of the sentence the agent accepts), and it is set up in
such a way that ordinary English sentences come out true, and about very
natural parts of the model. On the \textbf{U\&N Theory}, it could easily
turn out that ordinary speakers are speaking
\(\mathcal{L}\)\textsubscript{4}, since the assigned meanings are so
natural. We can see this isn't a consequence of \emph{Lewis's} theory by
working through the case from first principles. I have two arguments
here, the first of them relying on some slightly contentious claims
about the epistemology of mathematics, the second less contentious.

Assume, for reductio, that ordinary speakers are speaking
\(\mathcal{L}\)\textsubscript{4}. So, for instance, when O'Leary says
``The beer is in the fridge'', what he says is that a certain
complicated mathematical model has a certain property. (And indeed it
has that property.) Now this won't be a particularly rational thing for
O'Leary to say unless he knows more mathematics than ordinary folks like
him ordinarily do. So if O'Leary has adopted a convention of
truthfulness and trust in \(\mathcal{L}\)\textsubscript{4}, then
uttering ``The beer is in the fridge'' would be irrational, even if he
is standing in front of the open fridge, looking at the beer. That's a
gratuitous assignment of irrationality, and gratuitous assignments of
irrationality are false, so O'Leary doesn't speak
\(\mathcal{L}\)\textsubscript{4}.

Perhaps that is too quick. After all, the mathematical claim that
\(\mathcal{L}\)\textsubscript{4} associates with ``The beer is in the
fridge'' is a necessary truth. And Lewis's theory of content is
intentional, not hyper-intentional. So O'Leary does know it is true.
(And when he is standing in front of the fridge, there's even a sense
that he knows that ``The beer is in the fridge'' expresses a truth, if
\(\mathcal{L}\)\textsubscript{4} is really his language.) I think that's
probably not the right sense of ``rational'', and I'm not altogether
sure how much hostility to hyper-intensionalism we should attribute to
Lewis. But so as to avoid these questions, it's easier to consider a
different argument that focusses attention on O'Leary's audience.

When O'Leary says ``The beer is in the fridge'', Daniels hears him, and
then walks to the fridge. Why does Daniels make such a walk? Well, he
wants beer, and believes it is in the fridge. That looks like a nice
rational explanation. But why does he believe the beer is in the fridge?
I say it's because he's (rationally) adopted a convention of
truthfulness and trust in \(\mathcal{L}\)\textsubscript{1}, and so he
rationally comes to believe the beer is in the fridge when O'Leary says
``The beer is in the fridge''. On the assumption that O'Leary and
Daniels speak \(\mathcal{L}\)\textsubscript{4}, none of this story goes
through. But we must have some rational explanation of why O'Leary's
statement makes Daniels walk to the fridge. So O'Leary and Daniels must
not be speaking \(\mathcal{L}\)\textsubscript{4}.

Michael Morreau pointed out (when I presented this talk at CSMN) that
the preceding argument may be too quick. Perhaps there is a way of
rationalising Daniels's actions upon hearing O'Leary's words consistent
with the idea that they both speak \(\mathcal{L}\)\textsubscript{4}.
Perhaps, for instance, Daniels's walking to the fridge constitutes
saying something in a complicated sign language, and that thing is the
rational reply to what O'Leary said. If this kind of response works, and
I have no reason to think it won't, the solution is to increase the
costs to Daniels of performing such a reply. For instance, not too long
ago I heard Mayor Bloomberg say ``Lower Manhattan is being evacuated
because of the impending hurricane'', and I (and my family) packed up
and evacuated from Lower Manhattan. Even if one could find an
interpretation of our actions in evacuating that made them constitute
the assertion of a sensible reply to Bloomberg's mathematical assertion
in \(\mathcal{L}\)\textsubscript{4}, it would be irrational to think I
made such an assertion. Evacuating ahead of a storm with an infant is
not fun - if it was that hard to make mathematical assertions, I
wouldn't make them! And I certainly wouldn't make them in reply to
someone who wouldn't even see my gestures. So I think at least some of
the actions that are rationalised by testimony, interpreted as sentences
of \(\mathcal{L}\)\textsubscript{1}, are not rationalised by testimony,
interpreted as \(\mathcal{L}\)\textsubscript{4}. By the kind of appeal
to the principle of charity we have used a lot already, that means that
\(\mathcal{L}\)\textsubscript{4} is not the language most people speak.

The central point here is that when we are ruling out particularly
deviant interpretations of some speakers, we have to make heavy use of
the requirement that the interpretation of their shared language
rationalises what they do. In part that means it must rationalise why
they utter the strings that they do in fact utter. And when we're
considering this, we should remember the role of naturalness in a theory
of rationality. But it also means that it must rationalise why people
respond to various strings with non-linguistic actions, such as walking
to the fridge, or evacuating Lower Manhattan. Naturalness has less of a
role to play here, but the Lewisian theory still gets the right answers
provided we apply it carefully. Since the Lewisian theory gets the right
answers, and the \textbf{U\&N Theory}~gets the wrong answers, it follows
that the \textbf{U\&N Theory}~isn't Lewis's theory, and so orthodoxy is
wrong.

\section{What is the Use of a
Predicate?}\label{what-is-the-use-of-a-predicate}

We concluded the last section with an argument that Lewis isn't
vulnerable to the claim that his theory assigns complicated mathematical
claims as the meanings of ordinary English sentences. That
interpretation, we argued, is inconsistent with the way those sentences
are used. In particular, it is inconsistent with the way that
\emph{hearers} use sentences to guide their actions.

So far so good, we might think. But notice how much has been packed into
the notion of use to get us this far. In identifying the use O'Leary
makes of ``The beer is in the fridge'', we have to say a lot about
O'Leary's beliefs and desires. And in identifying the use Daniels makes
of it, we \emph{primarily} talk about the sentence's effects on
Daniels's beliefs and desires. That is, just saying how the sentence is
used requires saying a lot about mental states of speakers. And that
will often require appealing to constitutive rationality; we say that
Daniels's beliefs about the fridge changed because we need to
rationalise his fridge-directed behaviour.

And this should all make us suspicious about the prospects for
identifying meaning (in a Lewisian theory) with use plus naturalness.
The argument above that naturalness mattered to meaning relied on the
idea that naturalness matters because it affects which states are
rational, and hence which states are actualised. A belief that all
emeralds are grue is unnatural, so it is hard to hold. And since it is
hard to hold, it is hard to think one is conforming to a convention of
truthfulness in a language if one utters sentences that mean, in that
language, that all emeralds are grue. That's why it is wrong,
\emph{ceteris paribus}, to interpret people as speaking about grueness.

But now consider what happened when we were talking about Daniels and
O'Leary. Even to say how they were using the sentence ``The beer is in
the fridge'', we had to say what they believed before and after the
sentence was uttered. In other words, their mental states were
constitutive of the way the sentence was used. Now add in the extra
premise, argued for above, that naturalness matters to Lewis's theory of
linguistic content because, and only because, it matters to his theory
of mental content. (And it only matters to mental content because it
matters to the principle of charity that Lewis uses.) If mental states,
and their changes, are part of how the sentences are used, it will be
rather misleading to say that meaning is determined by use plus
naturalness. A better thing to say is that meaning is determined by use,
and that some key parts of use, i.e., mental states of speakers and
hearers, are determined in part by naturalness.

So I'm sceptical of the \textbf{U\&N Theory}. We can put the argument of
the last few paragraphs as a dilemma. There are richer and thinner ways
of identifying the use to which a sentence is put. A thin way might, for
instance, just focus on the observable state of the part of the physical
world in which the sentence is uttered. A rich way might include
include, inter alia, the use that is made of the sentence in the
management of belief and the generation of rational action. If we adopt
the thin way of thinking about use, then adding naturalness won't be
enough to say what makes it the case that O'Leary and Daniels are
speaking \(\mathcal{L}\)\textsubscript{1} rather than
\(\mathcal{L}\)\textsubscript{4}. If we adopt the rich way of thinking
about use, then the role that naturalness plays in the theory of meaning
has been incorporated into the metaphysics of use. Neither way makes the
\textbf{U\&N Theory}~true while assigning naturalness an independent
role. This dilemma isn't just an argument that we shouldn't attribute
the \textbf{U\&N Theory}~to Lewis; it is an argument against anyone
adopting that theory.

\section{From Theory to Applied
Semantics}\label{from-theory-to-applied-semantics}

So far we've argued that Lewis's semantic theory did not look a lot like
the orthodox interpretation. It's true that he thought the way a
sentence was used was of primary importance in determining its meaning.
And it's true that he thought naturalness mattered to meaning. But that
wasn't because naturalness came in to resolve the indeterminacy left in
a use-based theory of meaning. Rather, it was because naturalness was in
a part of the theory of mental content, and specifying the mental states
of speakers and hearers is part of specifying how the sentence is used.

But note that these considerations apply primarily to investigations at
a very high level of generality, such as when we're trying to solve the
problems described in ``Radical Interpretation''. They don't apply to
investigations into applied semantics. Let's say we are trying to figure
out what O'Leary and Daniels mean by ``green''. And assume that we are
taking for granted that they are speaking a language which is, in most
respects, like English. This is hardly unusual in ordinary work in
applied semantics. If we are writing a paper on the semantics of colour
terms, a paper like, say, ``Naming the Colours'', we don't concern
ourselves with the possibility that every sentence in the language
refers to some complicated mathematical claim or other.

Now given those assumptions, we can identify a moderately thin notion of
use. We know that O'Leary uses ``green'' to describe things that are, by
appearance, both green and grue. We also know that when O'Leary makes
such a description, Daniels expects the object will be both green and
grue. So focus on a notion of use such that the \emph{use} of a
predicate just is a function of which objects speakers will typically
apply the predicate to, and which properties hearers take those objects
to have once they hear the predication. If we wanted to be more precise,
we could call this notion of `use' simply \emph{predication}. When we
are doing applied semantics, especially when we are trying to figure out
the meaning of predicates, we typically know which objects a speaker is
disposed to predicate a predicate of, and that's the salient feature of
use. (This is why I said the most accurate heuristic would be meaning is
predication plus naturalness; predication is the bit of use we care
about in this context.)

This identification of use wouldn't make any sense if we were engaged in
theorising at a much more abstract level. If we are doing radical
interpretation, then we have to take non-semantic inputs, and solve
simultaneously for the values of the subject term and the predicate term
in a (simple) sentence. But when we are just doing applied semantics,
and working just on the meaning of a term like ``green'' in a
well-functioning language, we can presuppose facts about the denotation
of the subject term in sentences like \emph{S is green}, and presuppose
facts about what is the subject and what is the predicate in that
sentence, and then we can look at which properties hearers come to
associate with that very object on hearing that sentence.

Now that we have a notion of use that's distinct from naturalness, we
can ask whether it is plausible that predicate meaning is use (in that
sense) plus naturalness. And, quite plausibly, the answer is yes. The
arguments in Sider (\citeproc{ref-Sider2001-SIDCOP}{2001a}) and
Weatherson (\citeproc{ref-Weatherson2003-WEAWGA}{2003}) in favour of
this theory look like, at the very least, good arguments that the theory
does the right job in resolving Kripkensteinian problems. The theory is
immune to objections based on radical re-interpretations of the
language, as in Williams (\citeproc{ref-Williams2007}{2007}), because
those will be inconsistent with the use so defined. And the theory fits
nicely into Lewis's broader theory of meaning, i.e., his metasemantics,
which is in turn well motivated. So I think there are good reasons to
hold that when we're doing applied semantics, the \textbf{U\&N
Theory}~delivers the right verdicts, and delivers them for Lewisian
reasons. That's the heart of what's true about the \textbf{U\&N Theory},
even if it isn't a fully general theory of meaning.

\subsection*{References}\label{references}
\addcontentsline{toc}{subsection}{References}

\phantomsection\label{refs}
\begin{CSLReferences}{1}{0}
\bibitem[\citeproctext]{ref-Bays2007}
Bays, Timothy. 2007. {``The Problem with Charlie: Some Remarks on
Putnam, Lewis and Williams.''} \emph{Philosophical Review} 116 (3):
401--25. doi:
\href{https://doi.org/10.1215/00318108-2007-003}{10.1215/00318108-2007-003}.

\bibitem[\citeproctext]{ref-Hawthorne2007}
Hawthorne, John. 2007. {``Craziness and Metasemantics.''}
\emph{Philosophical Review} 116 (3): 427--40. doi:
\href{https://doi.org/10.1215/00318108-2007-004}{10.1215/00318108-2007-004}.

\bibitem[\citeproctext]{ref-Holton2003Lewis}
Holton, Richard. 2003. {``David Lewis's Philosophy of Language.''}
\emph{Mind and Language} 18 (3): 286--95. doi:
\href{https://doi.org/10.1111/1468-0017.00228}{10.1111/1468-0017.00228}.

\bibitem[\citeproctext]{ref-Lewis1969a}
Lewis, David. 1969. \emph{Convention: A Philosophical Study}. Cambridge:
Harvard University Press.

\bibitem[\citeproctext]{ref-Lewis1974c}
---------. 1974. {``Radical Interpretation.''} \emph{Synthese} 27 (3-4):
331--44. doi:
\href{https://doi.org/10.1007/bf00484599}{10.1007/bf00484599}. Reprinted
in his \emph{Philosophical Papers}, Volume 1, Oxford: Oxford University
Press, 1983, 108-118. References to reprint.

\bibitem[\citeproctext]{ref-Lewis1975b}
---------. 1975. {``Languages and Language.''} In \emph{Minnesota
Studies in the Philosophy of Science}, 7:3--35. Minneapolis: University
of Minnesota Press. Reprinted in his \emph{Philosophical Papers}, Volume
1, Oxford: Oxford University Press, 1983, 163-188. References to
reprint.

\bibitem[\citeproctext]{ref-Lewis1979b}
---------. 1979. {``Attitudes \emph{de Dicto} and \emph{de Se}.''}
\emph{Philosophical Review} 88 (4): 513--43. doi:
\href{https://doi.org/10.2307/2184646}{10.2307/2184646}. Reprinted in
his \emph{Philosophical Papers}, Volume 1, Oxford: Oxford University
Press, 1983, 133-156. References to reprint.

\bibitem[\citeproctext]{ref-Lewis1980c}
---------. 1980. {``Mad Pain and Martian Pain.''} In \emph{Readings in
the Philosophy of Psychology}, edited by Ned Block, I:216--32.
Cambridge: Harvard University Press. Reprinted in his
\emph{Philosophical Papers}, Volume 1, Oxford: Oxford University Press,
1983, 122-130. References to reprint.

\bibitem[\citeproctext]{ref-Lewis1983e}
---------. 1983. {``New Work for a Theory of Universals.''}
\emph{Australasian Journal of Philosophy} 61 (4): 343--77. doi:
\href{https://doi.org/10.1080/00048408312341131}{10.1080/00048408312341131}.
Reprinted in his \emph{Papers in Metaphysics and Epistemology},
Cambridge: Cambridge University Press, 1999, 8-55. References to
reprint.

\bibitem[\citeproctext]{ref-Lewis1984b}
---------. 1984. {``Putnam's Paradox.''} \emph{Australasian Journal of
Philosophy} 62 (3): 221--36. doi:
\href{https://doi.org/10.1080/00048408412340013}{10.1080/00048408412340013}.
Reprinted in his \emph{Papers in Metaphysics and Epistemology},
Cambridge: Cambridge University Press, 1999, 56-77. References to
reprint.

\bibitem[\citeproctext]{ref-Lewis1986a}
---------. 1986. \emph{On the Plurality of Worlds}. Oxford: Blackwell
Publishers.

\bibitem[\citeproctext]{ref-Lewis1992a}
---------. 1992. {``Meaning Without Use: Reply to {H}awthorne.''}
\emph{Australasian Journal of Philosophy} 70 (1): 106--10. doi:
\href{https://doi.org/10.1080/00048408112340093}{10.1080/00048408112340093}.
Reprinted in his \emph{Papers in Ethics and Social Philosophy},
Cambridge: Cambridge University Press, 2000, 145-151. References to
reprint.

\bibitem[\citeproctext]{ref-Lewis1994b}
---------. 1994. {``Reduction of Mind.''} In \emph{A Companion to the
Philosophy of Mind}, edited by Samuel Guttenplan, 412--31. Oxford:
Blackwell. doi:
\href{https://doi.org/10.1017/CBO9780511625343.019}{10.1017/CBO9780511625343.019}.
Reprinted in his \emph{Papers in Metaphysics and Epistemology}, 1999,
291-324, Cambridge: Cambridge University Press. References to reprint.

\bibitem[\citeproctext]{ref-Lewis1999a}
---------. 1999. \emph{Papers in Metaphysics and Epistemology}.
Cambridge: Cambridge University Press.

\bibitem[\citeproctext]{ref-Schwarz2006}
Schwarz, Wolfgang. 2006. {``Lewisian Meaning Without Naturalness.''}
Draft, January 4, 2006. Downloaded from
\url{http://www.umsu.de/words/magnetism.pdf}.

\bibitem[\citeproctext]{ref-Schwarz2009}
---------. 2009. \emph{David Lewis: Metaphysik Und Analyse}. Paderborn:
Mentis-Verlag.

\bibitem[\citeproctext]{ref-Sider2001-SIDCOP}
Sider, Theodore. 2001a. {``Criteria of Personal Identity and the Limits
of Conceptual Analysis.''} \emph{Philosophical Perspectives} 15:
189--209. doi:
\href{https://doi.org/10.1111/0029-4624.35.s15.10}{10.1111/0029-4624.35.s15.10}.

\bibitem[\citeproctext]{ref-Sider4D}
---------. 2001b. \emph{Four-Dimensionalism}. Oxford: Oxford University
Press.

\bibitem[\citeproctext]{ref-Sider2012}
---------. 2011. \emph{Writing the Book of the World}. Oxford: Oxford
University Press.

\bibitem[\citeproctext]{ref-Stalnaker2004-JACLOI-2}
Stalnaker, Robert. 2004. {``{Lewis on Intentionality}.''}
\emph{Australasian Journal of Philosophy} 82 (1): 199--212. doi:
\href{https://doi.org/10.1080/713659796}{10.1080/713659796}.

\bibitem[\citeproctext]{ref-Weatherson2003-WEAWGA}
Weatherson, Brian. 2003. {``{What Good Are Counterexamples?}''}
\emph{Philosophical Studies} 115 (1): 1--31. doi:
\href{https://doi.org/10.1023/A:1024961917413}{10.1023/A:1024961917413}.

\bibitem[\citeproctext]{ref-Weatherson2010-VaI}
---------. 2010. {``Vagueness as Indeterminacy.''} In \emph{Cuts and
Clouds: Vaguenesss, Its Nature and Its Logic}, edited by Richard Dietz
and Sebastiano Moruzzi, 77--90. Oxford: Oxford University Press.

\bibitem[\citeproctext]{ref-Williams2007}
Williams, J. Robert G. 2007. {``Eligibility and Inscrutability.''}
\emph{Philosophical Review} 116: 361--99. doi:
\href{https://doi.org/10.1215/00318108-2007-002}{10.1215/00318108-2007-002}.

\end{CSLReferences}



\noindent Published in\emph{
Journal for the History of Analytical Philosophy}, 2013, pp. 1-19.


\end{document}
