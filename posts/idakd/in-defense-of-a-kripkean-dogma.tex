% Options for packages loaded elsewhere
\PassOptionsToPackage{unicode}{hyperref}
\PassOptionsToPackage{hyphens}{url}
%
\documentclass[
  10pt,
  letterpaper,
  DIV=11,
  numbers=noendperiod,
  twoside]{scrartcl}

\usepackage{amsmath,amssymb}
\usepackage{setspace}
\usepackage{iftex}
\ifPDFTeX
  \usepackage[T1]{fontenc}
  \usepackage[utf8]{inputenc}
  \usepackage{textcomp} % provide euro and other symbols
\else % if luatex or xetex
  \usepackage{unicode-math}
  \defaultfontfeatures{Scale=MatchLowercase}
  \defaultfontfeatures[\rmfamily]{Ligatures=TeX,Scale=1}
\fi
\usepackage{lmodern}
\ifPDFTeX\else  
    % xetex/luatex font selection
  \setmainfont[ItalicFont=EB Garamond Italic,BoldFont=EB Garamond
Bold]{EB Garamond Math}
  \setsansfont[]{Europa-Bold}
  \setmathfont[]{Garamond-Math}
\fi
% Use upquote if available, for straight quotes in verbatim environments
\IfFileExists{upquote.sty}{\usepackage{upquote}}{}
\IfFileExists{microtype.sty}{% use microtype if available
  \usepackage[]{microtype}
  \UseMicrotypeSet[protrusion]{basicmath} % disable protrusion for tt fonts
}{}
\usepackage{xcolor}
\usepackage[left=1in, right=1in, top=0.8in, bottom=0.8in,
paperheight=9.5in, paperwidth=6.5in, includemp=TRUE, marginparwidth=0in,
marginparsep=0in]{geometry}
\setlength{\emergencystretch}{3em} % prevent overfull lines
\setcounter{secnumdepth}{3}
% Make \paragraph and \subparagraph free-standing
\ifx\paragraph\undefined\else
  \let\oldparagraph\paragraph
  \renewcommand{\paragraph}[1]{\oldparagraph{#1}\mbox{}}
\fi
\ifx\subparagraph\undefined\else
  \let\oldsubparagraph\subparagraph
  \renewcommand{\subparagraph}[1]{\oldsubparagraph{#1}\mbox{}}
\fi


\providecommand{\tightlist}{%
  \setlength{\itemsep}{0pt}\setlength{\parskip}{0pt}}\usepackage{longtable,booktabs,array}
\usepackage{calc} % for calculating minipage widths
% Correct order of tables after \paragraph or \subparagraph
\usepackage{etoolbox}
\makeatletter
\patchcmd\longtable{\par}{\if@noskipsec\mbox{}\fi\par}{}{}
\makeatother
% Allow footnotes in longtable head/foot
\IfFileExists{footnotehyper.sty}{\usepackage{footnotehyper}}{\usepackage{footnote}}
\makesavenoteenv{longtable}
\usepackage{graphicx}
\makeatletter
\def\maxwidth{\ifdim\Gin@nat@width>\linewidth\linewidth\else\Gin@nat@width\fi}
\def\maxheight{\ifdim\Gin@nat@height>\textheight\textheight\else\Gin@nat@height\fi}
\makeatother
% Scale images if necessary, so that they will not overflow the page
% margins by default, and it is still possible to overwrite the defaults
% using explicit options in \includegraphics[width, height, ...]{}
\setkeys{Gin}{width=\maxwidth,height=\maxheight,keepaspectratio}
% Set default figure placement to htbp
\makeatletter
\def\fps@figure{htbp}
\makeatother
% definitions for citeproc citations
\NewDocumentCommand\citeproctext{}{}
\NewDocumentCommand\citeproc{mm}{%
  \begingroup\def\citeproctext{#2}\cite{#1}\endgroup}
\makeatletter
 % allow citations to break across lines
 \let\@cite@ofmt\@firstofone
 % avoid brackets around text for \cite:
 \def\@biblabel#1{}
 \def\@cite#1#2{{#1\if@tempswa , #2\fi}}
\makeatother
\newlength{\cslhangindent}
\setlength{\cslhangindent}{1.5em}
\newlength{\csllabelwidth}
\setlength{\csllabelwidth}{3em}
\newenvironment{CSLReferences}[2] % #1 hanging-indent, #2 entry-spacing
 {\begin{list}{}{%
  \setlength{\itemindent}{0pt}
  \setlength{\leftmargin}{0pt}
  \setlength{\parsep}{0pt}
  % turn on hanging indent if param 1 is 1
  \ifodd #1
   \setlength{\leftmargin}{\cslhangindent}
   \setlength{\itemindent}{-1\cslhangindent}
  \fi
  % set entry spacing
  \setlength{\itemsep}{#2\baselineskip}}}
 {\end{list}}
\usepackage{calc}
\newcommand{\CSLBlock}[1]{\hfill\break\parbox[t]{\linewidth}{\strut\ignorespaces#1\strut}}
\newcommand{\CSLLeftMargin}[1]{\parbox[t]{\csllabelwidth}{\strut#1\strut}}
\newcommand{\CSLRightInline}[1]{\parbox[t]{\linewidth - \csllabelwidth}{\strut#1\strut}}
\newcommand{\CSLIndent}[1]{\hspace{\cslhangindent}#1}

\setlength\heavyrulewidth{0ex}
\setlength\lightrulewidth{0ex}
\usepackage[automark]{scrlayer-scrpage}
\clearpairofpagestyles
\cehead{
  Brian Weatherson
  }
\cohead{
  In Defense of a Kripkean Dogma
  }
\ohead{\bfseries \pagemark}
\cfoot{}
\makeatletter
\newcommand*\NoIndentAfterEnv[1]{%
  \AfterEndEnvironment{#1}{\par\@afterindentfalse\@afterheading}}
\makeatother
\NoIndentAfterEnv{itemize}
\NoIndentAfterEnv{enumerate}
\NoIndentAfterEnv{description}
\NoIndentAfterEnv{quote}
\NoIndentAfterEnv{equation}
\NoIndentAfterEnv{longtable}
\NoIndentAfterEnv{abstract}
\renewenvironment{abstract}
 {\vspace{-1.25cm}
 \quotation\small\noindent\rule{\linewidth}{.5pt}\par\smallskip
 \noindent }
 {\par\noindent\rule{\linewidth}{.5pt}\endquotation}
\cehead{
       Ichikawa, Maitra, and Weatherson
        }
\KOMAoption{captions}{tableheading}
\makeatletter
\@ifpackageloaded{caption}{}{\usepackage{caption}}
\AtBeginDocument{%
\ifdefined\contentsname
  \renewcommand*\contentsname{Table of contents}
\else
  \newcommand\contentsname{Table of contents}
\fi
\ifdefined\listfigurename
  \renewcommand*\listfigurename{List of Figures}
\else
  \newcommand\listfigurename{List of Figures}
\fi
\ifdefined\listtablename
  \renewcommand*\listtablename{List of Tables}
\else
  \newcommand\listtablename{List of Tables}
\fi
\ifdefined\figurename
  \renewcommand*\figurename{Figure}
\else
  \newcommand\figurename{Figure}
\fi
\ifdefined\tablename
  \renewcommand*\tablename{Table}
\else
  \newcommand\tablename{Table}
\fi
}
\@ifpackageloaded{float}{}{\usepackage{float}}
\floatstyle{ruled}
\@ifundefined{c@chapter}{\newfloat{codelisting}{h}{lop}}{\newfloat{codelisting}{h}{lop}[chapter]}
\floatname{codelisting}{Listing}
\newcommand*\listoflistings{\listof{codelisting}{List of Listings}}
\makeatother
\makeatletter
\makeatother
\makeatletter
\@ifpackageloaded{caption}{}{\usepackage{caption}}
\@ifpackageloaded{subcaption}{}{\usepackage{subcaption}}
\makeatother
\ifLuaTeX
  \usepackage{selnolig}  % disable illegal ligatures
\fi
\IfFileExists{bookmark.sty}{\usepackage{bookmark}}{\usepackage{hyperref}}
\IfFileExists{xurl.sty}{\usepackage{xurl}}{} % add URL line breaks if available
\urlstyle{same} % disable monospaced font for URLs
\hypersetup{
  pdftitle={In Defense of a Kripkean Dogma},
  pdfauthor={Jonathan Jenkins Ichikawa; Ishani Maitra; Brian Weatherson},
  hidelinks,
  pdfcreator={LaTeX via pandoc}}

\title{In Defense of a Kripkean Dogma}
\author{Jonathan Jenkins Ichikawa \and Ishani Maitra \and Brian
Weatherson}
\date{2012}

\begin{document}
\maketitle
\begin{abstract}
A reply to some empirical arguments against Kripkean meta-semantics.
\end{abstract}

\setstretch{1.1}
In ``Against Arguments from Reference'' (\citeproc{ref-MMNS2009}{Mallon
et al. 2009}), Ron Mallon, Edouard Machery, Shaun Nichols, and Stephen
Stich (hereafter, MMNS) argue that recent experiments concerning
reference undermine various philosophical arguments that presuppose the
correctness of the causal-historical theory of reference. We will argue
three things in reply. First, the experiments in question---concerning
Kripke's Gödel/Schmidt example---don't really speak to the dispute
between descriptivism and the causal-historical theory; though the two
theories are empirically testable, we need to look at quite different
data than MMNS do to decide between them. Second, the Gödel/Schmidt
example plays a different, and much smaller, role in Kripke's argument
for the causal-historical theory than MMNS assume. Finally, and
relatedly, even if Kripke \emph{is} wrong about the Gödel/Schmidt
example---indeed, even if the causal-historical theory is not the
correct theory of names for some human languages---that does not,
contrary to MMNS's claim, undermine uses of the causal-historical theory
in philosophical research projects.

\section{Experiments and Reference}\label{experiments-and-reference}

MMNS start with some by now famous experiments concerning reference and
mistaken identity. The one they focus on, and which we'll focus on too,
is a variant of Kripke's Gödel/Schmidt example. Here is the question
they gave to subjects.

\begin{quote}
Suppose that John has learned in college that Gödel is the man who
proved an important mathematical theorem, called the incompleteness of
arithmetic. John is quite good at mathematics and he can give an
accurate statement of the incompleteness theorem, which he attributes to
Gödel as the discoverer. But this is the only thing that he has heard
about Gödel. Now suppose that Gödel was not the author of this theorem.
A man called ``Schmidt'' whose body was found in Vienna under mysterious
circumstances many years ago, actually did the work in question. His
friend Gödel somehow got hold of the manuscript and claimed credit for
the work, which was thereafter attributed to Gödel. Thus he has been
known as the man who proved the incompleteness of arithmetic. Most
people who have heard the name `Gödel' are like John; the claim that
Gödel discovered the incompleteness theorem is the only thing they have
ever heard about Gödel. When John uses the name `Gödel,' is he talking
about:

\begin{description}
\tightlist
\item[(A)]
The person who really discovered the incompleteness of arithmetic?
\end{description}

or

\begin{description}
\tightlist
\item[(B)]
The person who got hold of the manuscript and claimed credit for the
work? (\citeproc{ref-MMNS2009}{Mallon et al. 2009}: 341)
\end{description}
\end{quote}

The striking result is that while a majority of American subjects answer
(B), consistently with Kripke's causal-historical theory of names, the
majority of Chinese subjects answer (A).\footnote{Note that a causal
  descriptivist about names will also say that the correct answer to
  this question is (B). So the experiment isn't really testing
  descriptivism as such versus Kripke's causal-historical theory, but
  some particular versions of descriptivism against Kripke's theory.
  These versions of descriptivism say that names refer to the satisfiers
  of (generally non-linguistic) descriptions that the name's user
  associates with the name. One such version is `famous deeds'
  descriptivism, and the descriptions MMNS use are typically famous
  deeds; nevertheless, that seems inessential to their experiments. When
  we use `descriptivism' in this paper, we'll mean any such version of
  descriptivism. Thanks here to an anonymous referee.} To the extent
that Kripke's theory is motivated by the universality of intuitions in
favour of his theory in cases like this one, Kripke's theory is
undermined.

There are now a number of challenges to this argument in the literature.
Before developing our own challenge, we'll briefly note five extant
ones, which all strike us as at least approximately correct. \footnote{The
  third objection relies on an empirical assumption that may be
  questionable. It assumes that the subject of the experiment associates
  the same description with `Gödel' as John does. A subject who (a) is a
  descriptivist and (b) associates with the name `Gödel' the description
  `the man who proved the compatibility of time travel and general
  relativity', can also make sense of the vignette, \emph{contra} Martí.
  So perhaps the objection could be resisted. But we think this
  empirical assumption is actually fairly plausible. Unless the
  experimental subjects were being picked from a very biased sample, the
  number of subjects who are familiar with Gödel's work on closed
  time-like curves is presumably vanishingly small! We're grateful here
  to an anonymous referee.}

\begin{description}
\item[(1)]
Kripke's theory is a theory of \emph{semantic} reference. When asked who
John is talking about, it is natural that many subjects will take this
to be a question about \emph{speaker} reference. And nothing in Kripke's
theory denies that \emph{John} might refer to the person who proved the
incompleteness of arithmetic, even if his word refers to someone else.
(\citeproc{ref-Ludwig2007}{Ludwig 2007};
\citeproc{ref-Deutsch2009}{Deutsch 2009})
\item[(2)]
Kripke's argument relies on the \emph{fact} that `Gödel' refers to
Gödel, not to the universality or otherwise of intuitions about what it
refers to. That some experimental subjects don't appreciate this fact
doesn't make it any less of a fact. (\citeproc{ref-Deutsch2009}{Deutsch
2009})
\item[(3)]
If the subjects genuinely were descriptivists, it isn't clear how they
could make sense of the vignette, since the name `Gödel' is frequently
used in the vignette itself to refer to the causal origin of that name,
not to the prover of the incompleteness or arithmetic.

On a related point, Martí doesn't mention this, but subjects who aren't
descriptivists should also object to the vignette, since in the story
John doesn't learn Gödel proved the incompleteness of arithmetic, at
least not if `learn' is factive. (\citeproc{ref-Marti2009}{Martı́ 2009})
\item[(4)]
The experiment asks subjects for their judgments about a metalinguistic,
and hence somewhat theoretical, question about the mechanics of
reference. It's better practice to observe how people actually refer,
rather than asking them what they think about reference.
(\citeproc{ref-Marti2009}{Martı́ 2009}; \citeproc{ref-Devitt2010}{Devitt
2011})
\item[(5)]
Intuitions about the Gödel/Schmidt case play at best a limited role in
Kripke's broader arguments, so experimental data undermining their
regularity do not cast serious doubt on Kripke's theory of reference.
(\citeproc{ref-Devitt2010}{Devitt 2011})
\end{description}

We think challenges (1)-(3) work. Something like (4) should work too,
although it requires some qualification. Consider, for instance, what
happens in syntax. It's true, of course, that we don't go around asking
ordinary speakers whether they think \emph{Lectures on Government and
Binding} was an advance over \emph{Aspects}. Or, if we did, we wouldn't
think it had much evidential value. But that's not because ordinary
speaker judgments are irrelevant to syntax. On the contrary, judgments
about whether particular strings constitute well-formed sentences are an
important part of our evidence.\footnote{This point suggests Martí's
  criticism of MMNS as stated overshoots. She wants to dismiss arguments
  from metalinguistic intuitions altogether. But intuitions about
  well-formedness \emph{are} metalinguistic intuitions, and they are a
  key part of the syntactician's toolkit. Martí concedes something like
  this point, but claims that the cases are not on a par, because syntax
  concerns a normative issue and reference does not. We're quite
  suspicious that there's such a striking distinction between the kind
  of subject-matter studied by syntacticians and semanticists. Devitt's
  version of this point is more modest and does not obviously commit to
  this exaggeration.} But they are not our only evidence, or even our
primary evidence; we also use corpus data about which words and phrases
are actually used, and many syntacticians take such usage evidence to
trump evidence from metasemantic intuitions.\footnote{Here's one example
  where testing intuitions and examining the corpus may lead to
  different answers. Many people think, perhaps because they've picked
  up something from a bad style guide, that the sentence `Whenever
  someone came into Bill's shop, he greeted them with a smile', contains
  one or two syntactic errors. (It uses a possessive as the antecedent
  of a pronoun, and it uses `them' as a bound singular variable.) Even
  if most subjects in a survey said such a sentence was not a
  well-formed sentence of English, corpus data could be used to show
  that it is. Certainly the existence of a survey showing that users in,
  say, Scotland and New Jersey give different answers when asked about
  whether the sentence is grammatical would not show that there's a
  syntactic difference between the dialects spoken in Scotland and New
  Jersey. You'd also want to see how the sentences are \emph{used}.}
Even when we do seek such intuitive answers, perhaps because there isn't
enough corpus data to settle the usage issue, the questions might be
about cases that are quite different to the cases we primarily care
about. So we might ask a lot about speakers' judgments concerning
questions even if we care primarily about the syntax of declarative
sentences.

If what Kripke (\citeproc{ref-Kripke1980}{1980}) says in \emph{Naming
and Necessity} (hereafter, NN) is right, then we should expect something
similar in the case of reference. Kripke anticipates that some people
will disagree with him about some of the examples, and offers a few
replies. (Our discussion here largely draws on footnote 36 of NN.) Part
of his reply is a version of point 1 above; those disagreements may well
be over speaker reference, not semantic reference. That reply is
correct; it's hard for us to hear a question about who someone is
talking about as anything but a question about speaker reference. He
goes on to note that his theory makes empirical predictions about how
names are used.

\begin{quote}
If I mistake Jones for Smith, I may \emph{refer} (in an appropriate
sense) to Jones when I say that Smith is raking the leaves \ldots{}
Similarly, if I erroneously think that Aristotle wrote such-and-such
passage, I may perhaps sometimes use `Aristotle' to \emph{refer} to the
actual author of the passage \ldots{} In both cases, I will withdraw my
original statement, and my original use of the name, if apprised of the
facts. (NN 86n)
\end{quote}

This seems entirely right. There's some sense in which John, in MMNS's
vignette, is referring to Gödel and some sense in which he's referring
to Schmidt. Just thinking about the particular utterance he makes using
`Gödel' won't help much in teasing apart speaker reference and semantic
reference. What we should look to are patterns of---or if they're not
available, intuitions about---withdrawals of statements containing
disputed names. To use the example Kripke gives here, consider a speaker
who (a) associates with the name `Aristotle' only the description `the
author of \emph{The Republic}', (b) truly believes that a particular
passage in \emph{The Republic} contains a quantifier scope fallacy, and
(c) is a descriptivist. She might say ``Aristotle commits a quantifier
scope fallacy in this passage.'' When she's informed that the passage
was written by Plato, she'll no longer utter those very words, but
she'll still insist that the sentence she uttered was literally true.
That's because she'll claim that in that sentence `Aristotle' just
referred to the author of the passage, and that person did commit a
quantifier scope fallacy. A non-descriptivist will take back the claim
expressed, though she might insist that what she \emph{intended} to say
was true.

So to show that subjects in different parts of the world really have
descriptivist intuitions about the Gödel/Schmidt case, we might ask
about whether they think John should withdraw, or clarify, his earlier
statements if apprised of the facts. Or we might ask whether they would
withdraw, or clarify, similar statements they had made if apprised of
the facts. Or, even better, we might test whether in practice people in
different parts of the world really do withdraw their prior claims at
different rates when apprised of the facts about a Gödel/Schmidt case.
Kripke is right that given descriptivism, a speaker shouldn't feel
obliged to withdraw the original statement when apprised of the facts,
but given the causal-historical theory, they should. So there are
experiments that we could run which would discriminate between
descriptivist and causal-historical approaches, but we don't think the
actual experiment MMNS run does so.

In its broad terms, we agree with Devitt's challenge (5), although we
understand the role of the Gödel/Schmidt case rather differently than he
does. We turn now to this question.

\section{Gödel's Role in Naming and
Necessity}\label{guxf6dels-role-in-naming-and-necessity}

In the first section we argued that the experimental data MMNS offer do
not show that the correct account of the Gödel/Schmidt example is
different in different dialects. In this section we want to argue that
there's very little one \emph{could} show about the Gödel/Schmidt
example that would bear on the broader question of what the correct
theory of reference is. To see this, let's review where the
Gödel/Schmidt example comes up in \emph{Naming and Necessity}.

In the first lecture, Kripke argues, via the modal argument, that names
can't be synonymous with descriptions. The reason is that in modal
contexts, substituting a name for an individuating description alters
truth values. So a pure descriptivism that treats names and descriptions
as synonymous is off the table. What's left, thinks Kripke, is what
Soames calls ``weak descriptivism'' (\citeproc{ref-Soames2003}{Soames
2003, vols. II, 356}). This is the view that although names are not
synonymous with descriptions, and do not abbreviate descriptions, they
do have their reference fixed by descriptions.

Here is the way Kripke introduces the picture that he is attacking.

\begin{quote}
The picture is this. I want to name an object. I think of some way of
describing it uniquely and then I go through, so to speak, a sort of
mental ceremony: By `Cicero' I shall mean the man who denounced Cataline
\ldots{} {[}M{]}y intentions are given by first, giving some condition
which uniquely determines an object, then using a certain word as a name
for the object determined by this condition. (NN 79)
\end{quote}

The Gödel/Schmidt example, or at least the version of it that MMNS
discuss, comes up in Kripke's attack on one of the consequences of this
picture of naming. (A variant on the example, where no one proves the
incompleteness of arithmetic, is used to attack another consequence of
the theory.) So the role of the Gödel/Schmidt example is to undermine
this picture of names and naming.

But note that it is far from the only attack on this picture. Indeed, it
is not even the first attack. Kripke's first argument is that for most
names, most users of the name cannot give an individuating description
of the bearer of the name. In fact, those users cannot even give a
description of the bearer that is individuating \emph{by their own
lights}. The best they can do for `Cicero' is `a Roman orator' and the
best they can do for `Feynman' is `a famous physicist'. (NN 81) But it
isn't that these users think that there was only one Roman orator, or
that there is only one famous physicist. It's just that they don't know
any more about the bearers of these names they possess. The important
point here is that Kripke starts with some examples where the best
description a speaker can associate with a name is a description that
isn't individuating \emph{even by the speakers' own lights}. And he
thinks that descriptivists can't explain how names work in these cases.

Now perhaps we'll get new experimental evidence that even in these
cases, some experimental subjects have descriptivist intuitions. Some
people might intuit that if a speaker does not know of any property that
distinguishes Feynman from Gell-Mann, their name `Feynman' is
indeterminate in reference between Feynman from Gell-Mann. We're not
sure what such an experiment would tell us about the metaphysics of
reference, but maybe someone could try undermining Kripke's argument
this way. But that's not what MMNS found; their experiments don't bear
on what Kripke says about `Feynman', and hence don't bear on his primary
argument against weak descriptivism.

Some philosophers will hold that although the picture Kripke describes
here, i.e., weak descriptivism, can't be right in general for
Feynman/Gell-Mann reasons, it could be true in some special cases. We
agree. So does Kripke. The very next sentence after the passage quoted
above says, ``Now there may be some cases in which we actually do
this.'' (NN 79) And he proceeds to describe three real life cases
(concerning `Hesperus', `Jack the Ripper' and `Neptune') where the
picture is plausibly correct. But he thinks these cases are rare. In
particular, we shouldn't think that the existence of an individuating
description is sufficient reason to believe that we are in such a case.
That, at last, is the point of the Gödel/Schmidt example. His conclusion
from that example is that weak descriptivism isn't correct even in those
special cases of names where the speaker possesses a description that
she \emph{takes} to be individuating.\footnote{The Gödel/Schmidt example
  is also distinctive in another way, in that the description in
  question actually applies to the referent of the name, and indeed
  speakers actually know this. But the flow of the text around the
  example (especially on page 84) suggests Kripke intends the example to
  make the same point as is made by other examples, such as the
  Peano/Dedekind case (in which the possessed description doesn't
  actually apply to the referent of the name). So this is probably not
  crucial to the point the example makes. We'll return below to the
  issue of just what this example shows. The key point is that the more
  distinctive the example is, the \emph{less} that would follow if
  Kripke were wrong about the example; he might only be wrong about
  examples with just those distinctive features.}

Michael Devitt (\citeproc{ref-Devitt2010}{2011}) also argues that MMNS
exaggerate the importance of the Gödel/Schmidt case. He identifies a
number of Kripke's other arguments (including the Feynman one we
mention) that he takes to be more central, and, like us, he argues that
MMNS's results do not cast doubt on these arguments. We agree, noting
only two points of difference. First, as suggested above, although the
Gödel/Schmidt case is not the only or the most central motivation for
Kripke's theory of reference, we do think that it plays a distinctive
role, compared with that of, for instance, the Feynman case. It refutes
even the weak version of weak descriptivism according to which, in the
special case in which subjects do possess individuating descriptions,
those descriptions determine reference. We think the Gödel/Schmidt case
(together with the Peano/Dedekind case) form the basis of the only
argument in \emph{Naming and Necessity} against this weak weak
descriptivism. (On a closely related point, we, unlike Devitt, take the
Gödel/Schmidt case to be addressing a quantitative question about how
common descriptive names are, not the qualitative question about whether
the causal-historical theory is true at all; we'll expand on this point
below.) Second, Devitt expresses some scepticism about the Gödel/Schmidt
judgment on the grounds that the relevant case is somewhat
`fanciful'---actual cases, Devitt suggests, are better to be trusted.
While there is surely some truth in the suggestion that intuitions about
esoteric and complicated cases can be less trustworthy than those about
everyday ones, we see little reason for concern in this instance; the
Gödel case does not describe a scenario we should expect to find trouble
thinking about.

Our reconstruction of the structure of Kripke's argument should make it
clear how \emph{unimportant} the Gödel/Schmidt example is to the broader
theoretical questions. If Kripke were wrong about the Gödel/Schmidt
case, that would at most show that there are a few more descriptive
names than we thought there were. But since the existence of some
descriptive names is consistent with the causal-historical theory of
reference, the existence of a few more is too. All the Gödel/Schmidt
example is used for in \emph{Naming and Necessity} is to show that the
number of descriptive names in English is not just small, it is
\emph{very} small. But the truth of the causal-historical theory of
reference doesn't turn on whether there are few descriptive names, or
very few descriptive names.

Once we see that the Gödel/Schmidt example concerns a quantitative
question (are descriptive names rare or very rare?) rather than a
qualitative question (is the causal-historical theory correct?), we can
see some limitations of the experiment MMNS rely on. The case that MMNS
describes to their subjects has several distinctive features, and it
isn't clear that we'd be justified in drawing conclusions from it about
cases that lack those features. Here is one such feature. The subject of
the vignette (John) acquires the name `Gödel' at the same time as he
acquires an individuating description of Gödel. Suppose it turned out
that, in some dialects at least, that would be sufficient for the name
to be a descriptive name; i.e., for it to be a name whose reference is
fixed by a description somehow attached to that name. If this conjecture
is true, then descriptive names are a little more common than Kripke
thinks they are, but not a lot more common. Now we don't actually think
this conjecture is true. And for the reasons given in section 1 we don't
think this experiment is evidence for it. What we do think is that (a)
it's hard to see how studying reactions to cases like the Gödel/Schmidt
example could show more than that some such claim about the prevalence
of descriptive names is true, and (b) such claims are not inconsistent
with the causal-historical theory.

We've argued that even if Kripke is wrong about the Gödel/Schmidt
example, that doesn't undermine the arguments for the main conclusions
of \emph{Naming and Necessity}. A natural inference from this is that
experiments about the Gödel/Schmidt example can't undermine those
conclusions. We think the natural inference is correct. A referee has
suggested that this is too quick. After all, if we have experimental
evidence that Kripke is wrong about the Gödel/Schmidt case, we might
have some grounds for suspicion about the other cases that Kripke uses
in the arguments for more central conclusions. That is, if MMNS are
right about the Gödel/Schmidt case, that doesn't give us a
\emph{deductive} argument against the other anti-descriptivist moves,
but it might give us an \emph{inductive} argument against them. This is
an important worry, but we think it can be adequately responded to.

The first thing to note is that it would be foolish to fall back to a
general scepticism about human judgment just because people disagree in
their intuitive reactions to some tricky cases. This point is well
argued by Timothy Williamson in his
(\citeproc{ref-Williamson2007-WILTPO-17}{2007} Ch. 6). If there's a
worry here, it must be because the evidence about the Gödel/Schmidt
example supports a more modest generalisation about judgments about
cases, but that generalisation is nevertheless strong enough to
undermine Kripke's other arguments. We doubt such a generalisation
exists.

It can't be that the experiments about the Gödel/Schmidt example show
that intuitive judgments about reference are systematically mistaken.
Most of our intuitions in this field are surely correct. For instance,
our intuitions that `Kripke' refers to Kripke and not Obama, and that
`Obama' refers to Obama and not Kripke, are correct. (And experiments
like the ones MMNS ran don't give us any reason at all to doubt that.)
And we could produce many more examples like that. At most, the
experiments can show us that there are spots of inaccuracy in a larger
pool of correct judgments.

It might be argued that we should be sceptical of intuitions about
reference in counterfactual cases. The correct judgments cited in the
previous paragraph are all about real cases, but the Gödel/Schmidt
example is not a real case. Now we don't think that the experiments do
undermine all intuitions about reference in counterfactual cases, but
even if they did, that wouldn't affect the Kripkean argument. That's
because the central argument against descriptivism at the start of
Lecture II involves real cases. The heavy lifting is done by cases where
speakers don't think they have an individuating description to go along
with names they use (e.g., `Feynman' and `Gell-Mann'), or they believe
they have an individuating description, but that description involves
some kind of circularity (e.g., `Einstein', `Cicero'). It seems to us
that these cases are much more like the cases where we know people have
accurate intuitions about reference (e.g., `Obama' refers to Obama),
than they are like cases where there is some dispute about their
accuracy (e.g., `Gödel' would refer to Gödel even if Schmidt had proved
the incompleteness of arithmetic). So there's no reason to doubt the
intuitions that underlie these central Kripkean arguments. And so
there's no reason from these experiments to doubt the anti-descriptivist
conclusions Kripke draws from them.

\section{Reference in Philosophy}\label{reference-in-philosophy}

If the data about the Gödel/Schmidt example don't undermine the
causal-historical theory of reference, then presumably they don't
undermine philosophical uses of that theory. But we think MMNS overstate
the role that theories of reference play in philosophical theorising,
and we'll end by saying something about this.

One simple reaction to MMNS's argument is to say that at most they show
that the causal-historical theory of reference is not true of some
dialects. But, a philosopher might say, they are not writing in such a
dialect, and the causal-historical theory is true of their dialect. And
that's all they needed for their argument. MMNS anticipate this
objection, and reply to it in section 3.3 of their paper. The reply is,
in essence, that such a picture would make a mess of communication. If
we posit dialectical variation to explain different reactions to the
Gödel/Schmidt example, and to other examples, then we cannot know what
dialect someone is speaking without knowing how they respond to these
examples. And plainly we don't need to quiz people in detail about
philosophical examples in order to communicate with them.

We offer three replies.

First, at least one of us is on record raising in principle suspicions
about this kind of argument Maitra (\citeproc{ref-Maitra2007}{2007}).
The take-home message from that paper is that communication is a lot
easier than many theorists have supposed, and requires much less
pre-communicative agreement. It seems to us that the reply MMNS offer
here is susceptible to the arguments in that paper, but for reasons of
space we won't rehearse those arguments in detail.

Second, it's one thing to think that variation in \emph{reference}
between dialects leads to communication breakdown, it's another thing
altogether to think that variation in \emph{meta-semantics} leads to
such breakdown. A little fable helps make this clear. In some parts of
Melbourne, `Gödel' refers to Gödel because of the causal chains between
the users of the name and the great mathematician. In other parts,
`Gödel' refers to Gödel because the speakers use it as a descriptive
name, associated with the description `the man who proved the
incompleteness of arithmetic'. Kevin doesn't know which area he is in
when he sees a plaque over a door saying ``Gödel lived here''. It seems
to us that Kevin can understand the sign completely without knowing how
`Gödel' got its reference. Indeed, he even knows what proposition the
sign expresses. So meta-semantic variation between dialects need not
lead to communicative failure, even when hearers don't know which
dialect is being used.

Third, if MMNS's argument succeeds, it seems to us that it shows
descriptivist theories, including the weak weak descriptivism that
Kripke is arguing against with the Gödel/Schmidt example, are doomed.
(The arguments in this paragraph are not original. Similar arguments are
used frequently in, e.g., Fodor and Lepore
(\citeproc{ref-FodorLepore1992}{1992}).) It's a platitude that different
people know different things. Barring a miracle, that means different
people will associate different descriptions with different names. If
there is widespread use of descriptive names, that means there will be
widespread differences in which descriptions are associated with which
names. And that will produce at least as much communicative difficulty
as having some people be causal-historical theorists and some people be
descriptivists. In short, if MMNS's argument against `referential
pluralism' is sound, there is an equally sound argument against
descriptivism. And note that this argument doesn't rely on any thought
experiments about particular cases. It doesn't even rely on thought
experiments about names like `Einstein', where there isn't any evidence
that Kripke is wrong about how those names work.

Dialectically, the situation is this. MMNS have offered an argument from
the possibility of communicating under conditions of ignorance about
one's interlocutor's knowledge. Similar arguments have been offered
against descriptivism. If such arguments are successful, then
descriptivism is false, and there's no problem with philosophers making
arguments from the falsity of descriptivism. If such arguments are
unsuccessful, then MMNS haven't shown that it is wrong for philosophers
to assume that the causal-historical theory is the right theory for
\emph{their} dialect, even if some other people are descriptivists. And,
as MMNS concede, as long as the philosophers themselves speak a
causal-historical theory dialect, the uses of the causal-historical
theory in philosophy seem appropriate. The only way this argument could
fail is if MMNS's argument from the possibility of communicating under
conditions of ignorance about one's interlocutor's knowledge is stronger
than the analogous arguments against descriptivism. But we see no reason
to believe that is so. If anything, it seems like a weaker argument,
because of the considerations arising from our fable about Kevin and the
`Gödel lived here' sign.

So we don't think MMNS have a good reply to the philosopher who insists
that they only need the causal-historical theory to be true of
\emph{their} dialect. But in fact we think that philosophers rarely even
assume that much.

Let's consider one of the examples that they cite: Richard Boyd's use of
the causal-historical theory of reference in developing and defending
his version of ``Cornell Realism'' in his
(\citeproc{ref-Boyd1988}{1988}). Here's one way one could try and argue
for moral realism from the causal-historical theory.

\begin{enumerate}
\def\labelenumi{\arabic{enumi}.}
\tightlist
\item
  The causal-historical theory of reference is the correct theory of
  reference for all words in all dialects (or at least our dialect).
\item
  So, it is the correct theory for `good'.
\end{enumerate}

But that's not Boyd's actual argument. And that's a good thing, because
the first premise is implausible. Someone defending it has to explain
descriptive names like `Neptune', logical terms like `and', empty
predicates like `witch', and so on. And Boyd's not in that business. His
argument is subtler. Boyd uses the causal-historical theory for two
purposes. First, he uses the development of a naturalistically
acceptable theory of reference as part of a long list of developments in
post-positivist philosophy that collectively constitute a
``distinctively realist conception of the central issues in the
philosophy of science'' (\citeproc{ref-Boyd1988}{Boyd 1988, 188}).
Second, he uses the causal-historical theory of reference, as it applies
to natural kind terms, as part of a story about how we can know a lot
about kinds that are not always easily observable
(\citeproc{ref-Boyd1988}{Boyd 1988, 195--96}). By analogy, he suggests
that we should be optimistic that a naturalistically acceptable moral
theory exists, and that it is consistent with us having a lot of moral
knowledge.

Once we look at the details of Boyd's argument, we see that it is an
argument that duelling intuitions about the Gödel/Schmidt example simply
can't touch. In part that's because Boyd cares primarily about natural
kind terms, not names. But more importantly it is because, as we noted
in section 2, the only point that's at issue by the time Kripke raises
the Gödel/Schmidt example is the \emph{number} of descriptive names.
Just looking at the arguments Kripke raises before that example gives us
more than enough evidence to use in the kind of argument Boyd is making.

It would take us far beyond the length of a short reply to go through
every philosophical use of the causal-historical theory that MMNS
purport to refute in this much detail. But we think that the kind of
response we've used here will frequently work. That is, we think few, if
any, of the arguments they attack use the parts of the causal-historical
theory that Kripke is defending with the Gödel/Schmidt example, and so
even if that example fails, it wouldn't undermine those theories.

\subsection*{References}\label{references}
\addcontentsline{toc}{subsection}{References}

\phantomsection\label{refs}
\begin{CSLReferences}{1}{0}
\bibitem[\citeproctext]{ref-Boyd1988}
Boyd, Richard. 1988. {``How to Be a Moral Realist.''} In \emph{Essays in
Moral Realism}, edited by Geoffrey Sayre-McCord, 181--228. Ithaca:
Cornell University Press.

\bibitem[\citeproctext]{ref-Deutsch2009}
Deutsch, Max. 2009. {``Experimental Philosophy and the Theory of
Reference.''} \emph{Mind and Language} 24 (4): 445--66. doi:
\href{https://doi.org/10.1111/j.1468-0017.2009.01370.x}{10.1111/j.1468-0017.2009.01370.x}.

\bibitem[\citeproctext]{ref-Devitt2010}
Devitt, Michael. 2011. {``Experimental Semantics.''} \emph{Philosophy
and Phenomenological Research} 82 (2): 418--35. doi:
\href{https://doi.org/ppr201182222}{ppr201182222}.

\bibitem[\citeproctext]{ref-FodorLepore1992}
Fodor, Jerry A., and Ernest Lepore. 1992. \emph{Holism: A Shopper's
Guide}. Cambridge: Blackwell.

\bibitem[\citeproctext]{ref-Kripke1980}
Kripke, Saul. 1980. \emph{Naming and Necessity}. Cambridge: Harvard
University Press.

\bibitem[\citeproctext]{ref-Ludwig2007}
Ludwig, Kirk. 2007. {``The Epistemology of Thought Experiments: First
Person Versus Third Person Approaches.''} \emph{Midwest Studies in
Philosophy} 31 (1): 128--59. doi:
\href{https://doi.org/10.1111/j.1475-4975.2007.00160.x}{10.1111/j.1475-4975.2007.00160.x}.

\bibitem[\citeproctext]{ref-Maitra2007}
Maitra, Ishani. 2007. {``How and Why to Be a Moderate Contextualist.''}
In \emph{Context Sensitivity and Semantic Minimalism: New Essays on
Semantics and Pragmatics}, edited by Gerhard Preyer and Georg Peter,
111--32. Oxford: Oxford University Press.

\bibitem[\citeproctext]{ref-MMNS2009}
Mallon, Ron, Eduoard Machery, Shaun Nichols, and Stephen Stich. 2009.
{``Against Arguments from Reference.''} \emph{Philosophy and
Phenomenological Research} 79 (2): 332--56. doi:
\href{https://doi.org/10.1111/j.1933-1592.2009.00281.x}{10.1111/j.1933-1592.2009.00281.x}.

\bibitem[\citeproctext]{ref-Marti2009}
Martı́, Genoveva. 2009. {``Against Semantic Multi-Culturalism.''}
\emph{Analysis} 69 (1): 42--48. doi:
\href{https://doi.org/10.1093/analys/ann007}{10.1093/analys/ann007}.

\bibitem[\citeproctext]{ref-Soames2003}
Soames, Scott. 2003. \emph{Philosophical Analysis in the Twentieth
Century}. Princeton: Princeton University Press.

\bibitem[\citeproctext]{ref-Williamson2007-WILTPO-17}
Williamson, Timothy. 2007. \emph{{The Philosophy of Philosophy}}.
Blackwell.

\end{CSLReferences}



\noindent Published in\emph{
Philosophy and Phenomenological Research}, 2012, pp. 56-68.

\end{document}
