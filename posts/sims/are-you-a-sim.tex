% Options for packages loaded elsewhere
\PassOptionsToPackage{unicode}{hyperref}
\PassOptionsToPackage{hyphens}{url}
%
\documentclass[
  10pt,
  letterpaper,
  DIV=11,
  numbers=noendperiod,
  twoside]{scrartcl}

\usepackage{amsmath,amssymb}
\usepackage{setspace}
\usepackage{iftex}
\ifPDFTeX
  \usepackage[T1]{fontenc}
  \usepackage[utf8]{inputenc}
  \usepackage{textcomp} % provide euro and other symbols
\else % if luatex or xetex
  \usepackage{unicode-math}
  \defaultfontfeatures{Scale=MatchLowercase}
  \defaultfontfeatures[\rmfamily]{Ligatures=TeX,Scale=1}
\fi
\usepackage{lmodern}
\ifPDFTeX\else  
    % xetex/luatex font selection
  \setmainfont[ItalicFont=EB Garamond Italic,BoldFont=EB Garamond
Bold]{EB Garamond Math}
  \setsansfont[]{Europa-Bold}
  \setmathfont[]{Garamond-Math}
\fi
% Use upquote if available, for straight quotes in verbatim environments
\IfFileExists{upquote.sty}{\usepackage{upquote}}{}
\IfFileExists{microtype.sty}{% use microtype if available
  \usepackage[]{microtype}
  \UseMicrotypeSet[protrusion]{basicmath} % disable protrusion for tt fonts
}{}
\usepackage{xcolor}
\usepackage[left=1in, right=1in, top=0.8in, bottom=0.8in,
paperheight=9.5in, paperwidth=6.5in, includemp=TRUE, marginparwidth=0in,
marginparsep=0in]{geometry}
\setlength{\emergencystretch}{3em} % prevent overfull lines
\setcounter{secnumdepth}{3}
% Make \paragraph and \subparagraph free-standing
\ifx\paragraph\undefined\else
  \let\oldparagraph\paragraph
  \renewcommand{\paragraph}[1]{\oldparagraph{#1}\mbox{}}
\fi
\ifx\subparagraph\undefined\else
  \let\oldsubparagraph\subparagraph
  \renewcommand{\subparagraph}[1]{\oldsubparagraph{#1}\mbox{}}
\fi


\providecommand{\tightlist}{%
  \setlength{\itemsep}{0pt}\setlength{\parskip}{0pt}}\usepackage{longtable,booktabs,array}
\usepackage{calc} % for calculating minipage widths
% Correct order of tables after \paragraph or \subparagraph
\usepackage{etoolbox}
\makeatletter
\patchcmd\longtable{\par}{\if@noskipsec\mbox{}\fi\par}{}{}
\makeatother
% Allow footnotes in longtable head/foot
\IfFileExists{footnotehyper.sty}{\usepackage{footnotehyper}}{\usepackage{footnote}}
\makesavenoteenv{longtable}
\usepackage{graphicx}
\makeatletter
\def\maxwidth{\ifdim\Gin@nat@width>\linewidth\linewidth\else\Gin@nat@width\fi}
\def\maxheight{\ifdim\Gin@nat@height>\textheight\textheight\else\Gin@nat@height\fi}
\makeatother
% Scale images if necessary, so that they will not overflow the page
% margins by default, and it is still possible to overwrite the defaults
% using explicit options in \includegraphics[width, height, ...]{}
\setkeys{Gin}{width=\maxwidth,height=\maxheight,keepaspectratio}
% Set default figure placement to htbp
\makeatletter
\def\fps@figure{htbp}
\makeatother
% definitions for citeproc citations
\NewDocumentCommand\citeproctext{}{}
\NewDocumentCommand\citeproc{mm}{%
  \begingroup\def\citeproctext{#2}\cite{#1}\endgroup}
\makeatletter
 % allow citations to break across lines
 \let\@cite@ofmt\@firstofone
 % avoid brackets around text for \cite:
 \def\@biblabel#1{}
 \def\@cite#1#2{{#1\if@tempswa , #2\fi}}
\makeatother
\newlength{\cslhangindent}
\setlength{\cslhangindent}{1.5em}
\newlength{\csllabelwidth}
\setlength{\csllabelwidth}{3em}
\newenvironment{CSLReferences}[2] % #1 hanging-indent, #2 entry-spacing
 {\begin{list}{}{%
  \setlength{\itemindent}{0pt}
  \setlength{\leftmargin}{0pt}
  \setlength{\parsep}{0pt}
  % turn on hanging indent if param 1 is 1
  \ifodd #1
   \setlength{\leftmargin}{\cslhangindent}
   \setlength{\itemindent}{-1\cslhangindent}
  \fi
  % set entry spacing
  \setlength{\itemsep}{#2\baselineskip}}}
 {\end{list}}
\usepackage{calc}
\newcommand{\CSLBlock}[1]{\hfill\break\parbox[t]{\linewidth}{\strut\ignorespaces#1\strut}}
\newcommand{\CSLLeftMargin}[1]{\parbox[t]{\csllabelwidth}{\strut#1\strut}}
\newcommand{\CSLRightInline}[1]{\parbox[t]{\linewidth - \csllabelwidth}{\strut#1\strut}}
\newcommand{\CSLIndent}[1]{\hspace{\cslhangindent}#1}

\setlength\heavyrulewidth{0ex}
\setlength\lightrulewidth{0ex}
\usepackage[automark]{scrlayer-scrpage}
\clearpairofpagestyles
\cehead{
  Brian Weatherson
  }
\cohead{
  Are You a Sim?
  }
\ohead{\bfseries \pagemark}
\cfoot{}
\makeatletter
\newcommand*\NoIndentAfterEnv[1]{%
  \AfterEndEnvironment{#1}{\par\@afterindentfalse\@afterheading}}
\makeatother
\NoIndentAfterEnv{itemize}
\NoIndentAfterEnv{enumerate}
\NoIndentAfterEnv{description}
\NoIndentAfterEnv{quote}
\NoIndentAfterEnv{equation}
\NoIndentAfterEnv{longtable}
\NoIndentAfterEnv{abstract}
\renewenvironment{abstract}
 {\vspace{-1.25cm}
 \quotation\small\noindent\rule{\linewidth}{.5pt}\par\smallskip
 \noindent }
 {\par\noindent\rule{\linewidth}{.5pt}\endquotation}
\KOMAoption{captions}{tableheading}
\makeatletter
\@ifpackageloaded{caption}{}{\usepackage{caption}}
\AtBeginDocument{%
\ifdefined\contentsname
  \renewcommand*\contentsname{Table of contents}
\else
  \newcommand\contentsname{Table of contents}
\fi
\ifdefined\listfigurename
  \renewcommand*\listfigurename{List of Figures}
\else
  \newcommand\listfigurename{List of Figures}
\fi
\ifdefined\listtablename
  \renewcommand*\listtablename{List of Tables}
\else
  \newcommand\listtablename{List of Tables}
\fi
\ifdefined\figurename
  \renewcommand*\figurename{Figure}
\else
  \newcommand\figurename{Figure}
\fi
\ifdefined\tablename
  \renewcommand*\tablename{Table}
\else
  \newcommand\tablename{Table}
\fi
}
\@ifpackageloaded{float}{}{\usepackage{float}}
\floatstyle{ruled}
\@ifundefined{c@chapter}{\newfloat{codelisting}{h}{lop}}{\newfloat{codelisting}{h}{lop}[chapter]}
\floatname{codelisting}{Listing}
\newcommand*\listoflistings{\listof{codelisting}{List of Listings}}
\makeatother
\makeatletter
\makeatother
\makeatletter
\@ifpackageloaded{caption}{}{\usepackage{caption}}
\@ifpackageloaded{subcaption}{}{\usepackage{subcaption}}
\makeatother
\ifLuaTeX
  \usepackage{selnolig}  % disable illegal ligatures
\fi
\IfFileExists{bookmark.sty}{\usepackage{bookmark}}{\usepackage{hyperref}}
\IfFileExists{xurl.sty}{\usepackage{xurl}}{} % add URL line breaks if available
\urlstyle{same} % disable monospaced font for URLs
\hypersetup{
  pdftitle={Are You a Sim?},
  pdfauthor={Brian Weatherson},
  hidelinks,
  pdfcreator={LaTeX via pandoc}}

\title{Are You a Sim?}
\author{Brian Weatherson}
\date{2003}

\begin{document}
\maketitle
\begin{abstract}
Nick Bostrom argues that if we accept some plausible assumptions about
how the future will unfold, we should believe we are probably not
humans. The argument appeals crucially to an indifference principle
whose content is unclear. I set out four possible interpretations of the
principle, none of which can be used to support Bostrom's argument. On
the first two interpretations the principle is false; on the third it
does not entail the conclusion; and on the fourth it only entails the
conclusion given an auxiliary hypothesis which we have no reason to
believe.
\end{abstract}

\setstretch{1.1}
In Will Wright's delightful game \href{http://thesims.ea.com/}{\emph{The
Sims}}, the player controls a neighbourhood full of people,
affectionately called sims. The game has no scoring system, or winning
conditions. It just allows players to create, and to some extent
participate in, an interesting mini-world. Right now the sims have
fairly primitive psychologies, but we can imagine this will be improved
as the game evolves. The game is very popular now, and it seems
plausible that it, and the inevitable imitators, will become even more
popular as its psychological engine becomes more realistic. Since each
human player creates a neighbourhood with many, many sims in it, in time
the number of sims in the world will vastly outstrip the number of
humans.

Let's assume that as the sims become more and more complex, they will
eventually acquire conscious states much like yours or mine. I do not
want to argue for or against this assumption, but it seems plausible
enough for discussion purposes. I'll reserve the term Sim, with a
capital S, for a sim that is conscious. By similar reasoning to the
above, it seems in time the number of Sims in the world will far
outstrip the number of humans, unless humanity either (a) stops
existing, or (b) runs into unexpected barriers to computing power or (c)
loses interest in these kinds of simulators. I think none of these is
likely, so I think that over time the ratio of Sims to humans will far
exceed 1:1.

Nick Bostrom (\citeproc{ref-Bostrom2003}{2003}) argues that given all
that, we should believe that we are probably Sims. Roughly, the argument
is that we know that most agents with conscious states somewhat like
ours are Sims. And we don't have any specific evidence that tells on
whether we are a Sim or a human. So the credence we each assign to
\emph{I'm a Sim} should equal our best guess as to the percentage of
human-like agents that are Sims, which is far above ½. As Glenn Reynolds
put it, ``Is it live, or is it Memorex? Statistically, it's probably
Memorex. Er, and so are you, actually.''\footnote{Original post at
  http://www.instapundit.com/archives/003465.php\#003465. Reynolds's
  comment wasn't directly about Bostrom, but it bore the ancestral of
  the relation \emph{refers} to Bostrom's paper.} (Is it worrying that
we used the assumption that we are human to generate this statistical
argument? Not necessarily; if we are Sims then the Sims:humans ratio is
probably even higher, so what we know is a lower bound on the proportion
of human-like agents that are Sims.) Less roughly, the argument appeals
crucially to the following principle:

\begin{description}
\tightlist
\item[(\#)]
\emph{Cr}(\emph{Sim}~\textbar~\emph{f\textsubscript{Sim}}~=~\emph{x})~=~\emph{x}
\end{description}

Here \emph{Cr} is a rational credence function. I will adopt David
Lewis's theory of \emph{de se} belief, and assume that the credence
function is defined over properties, rather than propositions Lewis
(\citeproc{ref-Lewis1979b}{1979}). Whenever I use a term that normally
stands for a proposition inside the scope of \emph{Cr}, it stands for
the property of being in a world where that proposition is true. So
\emph{f\textsubscript{Sim}}~=~\emph{x} stands for the property of being
in a world where 100\emph{x}\% of the human-like agents are Sims.

As Bostrom notes, the main reason for believing (\#) is that it is an
instance of a plausible general principle, which I'll call (\#\#).

\begin{description}
\tightlist
\item[(\#\#)]
∀ɸ:
\emph{Cr}(ɸ~\textbar~\emph{f}\textsubscript{ɸ}~=~\emph{x})~=~\emph{x}
\end{description}

Bostrom does not formulate this more general principle, but it is clear
that he intends something like it to be behind his argument, for many of
the defences of (\#) involve substituting some other property in place
of \emph{Sim} in statements like (\#). So I will focus here on whether
anything like (\#\#) is plausibly true, and whether it supports (\#).
There are many ways we could interpret (\#\#), depending on whether we
take \emph{Cr} to be a rational agent's current credences, or in some
sense the prior credences before they are affected by some particular
evidence, and on whether we take the quantifier to be restricted or
unrestricted. Five particular interpretations stand out as being worth
considering. None of these, however, provides much reason to believe
(\#), at least on the reading Bostrom wants to give it. In that reading
(\#) the credence function represents the current credences of an agent
much like you or me. If (\#) isn't interpreted that way, it can't play
the dialectical role Bostrom wants it to play. On two of the
interpretations, (\#\#) is false, on two others it may be true but
clearly does not entail (\#), and on the fifth it only entails (\#) if
we make an auxiliary assumption which is far from obviously true.

For ease of exposition, I will assume that \emph{Cr} describes in some
way the credences at some time of a particular rational human-like
agent, Rat, who is much like you or me, except that she is perfectly
rational.

\section{First Interpretation}\label{first-interpretation}

\emph{Cr} in (\#\#) measures Rat's current credences, and the quantifier
in (\#\#) is unrestricted. On this interpretation, (\#\#) is clearly
false, as Bostrom notes. Rat may well know that the proportion of
human-like agents that are like spaghetti westerns is rather low, while
rationally being quite confident that she likes spaghetti westerns. For
any property ɸ where Rat has some particular information about whether
he is one of the ɸs or not, that information, and not general facts
about the proportion of human-like agents that are ɸ, can (indeed
should) guide Rat's credences. So those substitution instances of (\#\#)
are false.

\section{Second Interpretation}\label{second-interpretation}

Just like the first interpretation, except that we restrict the
quantifier range so that it only ranges over properties such that Rat
does not know whether she possesses them. This interpretation seems to
be hinted at by Bostrom when he says, ``the bland indifference principle
expressed by (\#) prescribes indifference only between hypotheses about
which observer you are, when you have no information about which of
these observers you are.'' Even given this restriction, (\#\#) is still
false, as the following example shows.

Assume that Rat knows that \emph{f\textsubscript{Sim}} \textgreater{}
0.9, which Bostrom clearly takes to be consistent with rationality. And
assume also that Rat, being a normal human-like agent, knows some fairly
specific, and fairly distinctive facts about her conscious life. If Rat
is anything like you or me, she will have experiences that he can be
fairly sure are unique to her. Last night, for instance, while Rat was
listening to Go-Betweens bootlegs, watching baseball, drinking beer,
rocking in his rocking chair and thinking about Bostrom's simulation
argument, she stubbed her toe in a moderately, but not excessively,
painful way. Few people will have done all these things at once, and
none in quite that way. Let \emph{C} be the property of ever having had
an experience almost just like that. Rat knows he is a \emph{C}. She is
very confident, though not certain, that she is the only human-like
\emph{C}. Let a suman be the property of being \emph{C} and human, or
not-\emph{C} and a Sim. For much of the paper we're going to be
concerned with the following two properties.

\begin{quote}
\emph{x} is a \textbf{suman} =\textsubscript{df} \emph{x} is a human
\emph{C} or a Sim who is not a \emph{C}.

\emph{x} is a \textbf{him} =\textsubscript{df} \emph{x} is a Sim
\emph{C} or a human who is not a \emph{C}.
\end{quote}

We are following Bostrom in assuming that Rat does not know whether she
is a Sim so she does not know whether she is a suman. But given that
almost no one is \emph{C}, it follows that
\emph{f\textsubscript{suman}}~≈~\emph{f\textsubscript{Sim}}. Hence
\emph{f\textsubscript{suman}}~\textgreater{} 0.85, for if it is less
than \emph{f\textsubscript{Sim}}, it is not much less. But if
\emph{Cr}(a suman)~\textgreater~0.85, and
\emph{Cr}(\emph{Sim})~\textgreater~0.9, and Rat is coherent, it follows
that \emph{Cr}(\emph{C})~\textless~0.25. But we assumed that Rat knew
that she was a \emph{C}, and however knowledge and credence are to be
connected, it is inconceivable that one could know something while one's
credence in it is less than ¼. Hence it must be false that
\emph{Cr}(\emph{C})~\textless~¼, but we inferred that from given facts
about the story and (\#\#), as interpreted here. Hence (\#\#), as
interpreted here, is false.

\section{Third Interpretation}\label{third-interpretation}

One natural response ot the previous objection is that there shoul dbe
some way of restricting (\#\#) so that it does not apply to properties
like being a suman. Intuitively, the response is that even though Rat
doesn't know whether she is a suman, she knows something that is
relevant to whether she is a suman, namely that she is a \emph{C}. The
problem with this response is that any formal restriction on (\#\#) that
implements this intuition ends up giving us a version so weak that it
doesn't entail (\#).

The idea is that what went wrong in the previous case is that even
though Rat does not know whether she is a suman, she knows something
relevant to this. In particular, she knows that if she is a suman, she
is one of the sumans that is human, rather than one of the ones that is
a Sim. Our third interpretation avoids the difficulties this raises by
restricting the quantifier in (\#\#) even further. Say that a property ɸ
is in the domain of the quantifier iff (a) Rat does not know whether she
is ɸ, and (b) there is no more specific property ɸ\textsubscript{′} such
that Rat knows that if she is ɸ, then she is
ɸ\textsubscript{′}.\footnote{I think it is this interpretation of (\#\#)
  that Adam Elga implicitly appeals to in his solution to the Sleeping
  Beauty problem Elga (\citeproc{ref-Elga2000-ELGSBA}{2000}).} This will
rule out the applicability of (\#\#) to properties like a suman.
Unfortunately, it will also rule out the applicability of (\#\#) to
properties like \emph{being a Sim}. For Rat knows that if she is a Sim,
then she is a Sim that is also a \emph{C}. So now (\#\#) doesn't entail
(\#).

This kind of problem will arise for any attempt to put a purely formal
restriction on (\#\#). The problem is that, as Goodman noted in a quite
different context (\citeproc{ref-Goodman1955}{Goodman 1955}), there is
no formal distinction between the `normal' properties, being a human and
being a sim, and the `deviant' properties, being a suman and being a
him. The following four biconditionals are all conceptual truths, and
hence must all receive credence 1.

If the obvious truth of (1a) implies that Rat cannot apply (\#\#) to the
property o being a suman once she knows that she is a \emph{C}, for (1a)
makes that evidence look clrarly relevant to the issue of whether she is
suman, then similar reasoning suggests that the obvious truth of (2a)
implies that Rat cannot apply (\#\#) to the properties of being a human
once she knows that she is a \emph{C}, for (2a) makes that evidence look
clearly relevant to the issue of whether she is human. The point is that
a restriction on (\#\#) that is to deliver (\#) must fine some
epistemologically salient distinction between the property of being
human and the property of being suman if it is to rule out one
application of (\#\#) without ruling out the other, and if we only
consider formal constraints, we won't find such a restriction. Our final
attempt to justify (\#) from something like (\#\#) attempts to avoid
this problem by appealing directly to the nature of Rat's evidence.

\section{Fourth Interpretation}\label{fourth-interpretation}

The problems with the three interpretations of (\#\#) so far have been
that they applied \emph{after} Rat found out something distinctive about
herself, that she was a \emph{C}. Perhaps (\#\#) is really a constraint
on \emph{prior} credence functions. \emph{A priori}, Rat's credences
should be governed by an unrestricted version of (\#\#). We then have
the following argument for (\#). (As noted above, (\#) is a constraint
on current credences, so it is not immediately entailed by a constraint
on prior credences such as (\#\#) under its current interpretation.)

\begin{description}
\tightlist
\item[P1]
\emph{A priori}, Rat's conditional credence in her being a Sim given
that \emph{f\textsubscript{Sim}} is \emph{x} is \emph{x}.
\item[P2]
All of Rat's evidence is probabilistically independent of the property
of being a Sim.
\item[C]
Rat's current conditional credence in her being a Sim given that
\emph{f\textsubscript{Sim}} is \emph{x} is \emph{x}.
\end{description}

This interpretation may be reasonably faithful to what Bostrom had in
mind. The argument just sketched looks similar enough to what he hints
at in the following quote: ``More generally, if we knew that a fraction
\emph{x} of all observers with human-type experiences live in
simulations, and we don't have any information that indicate that our
own particular experiences are any more or less likely than other
human-type experiences to have been implemented \emph{in vivo} rather
than \emph{in machina}, then our credence that we are in a simulation
should equal \emph{x}.'' So it's not unreasonable to conclude that he is
committed to P2, and intends it to be used in the argument that you
should give high credence to being a Sim.\footnote{Jamie Dreier pointed
  out to me that what Bostrom says here is slightly more complicated
  than what I, hopefully charitably, attribute to him. A literal reading
  of Bostrom's passage suggests he intends the following principle.

  (B): ∀\emph{e}: \emph{Cr}(\emph{e}~\textbar~\emph{Human})~-
  \emph{Cr}(\emph{e}~\textbar~\emph{Sim})~=
  \emph{Cr}(\emph{e}~\textbar~\emph{Human})~-
  \emph{Cr}(\emph{e}~\textbar~\emph{Sim})

  The quantifier here ranges over possible experiences \emph{e},
  \emph{e} is the actual experience Rat has, and \emph{Cr} is the
  credence function at the `time' when Rat merely knows that he is
  human-like and \emph{f\textsubscript{Sim}} is greater than 0.9. I
  suggested a simpler assumption:

  (I):
  \emph{Cr}(\emph{Human}~\textbar~\emph{e})~=~\emph{Cr}(\emph{Sim}~\textbar~\emph{e})

  Bostrom needs something a little stronger than (I) to get his desired
  conclusion, for he needs this to hold not just for Rat's experience
  \emph{e}, but for your experience and mine as well. But we will not
  press that point. Given that point, though, (I) is all he needs. And
  presumably the reason he adopts (B) is because it looks like it
  entails (I). And indeed it does entail (I) given some fairly innocuous
  background assumptions.} Further, this version of (\#\#), where it is
restricted to prior credences, does not look unreasonable. So if P2 is
true, an argument for (\#) might just succeed. So the issue now is just
whether P2 is true.

Why might we reject P2? Any of the following three reasons might do.
First, Rat's evidence might be constituted by more than her conscious
phenomenal states. This reply has an externalist and an internalist
version. On the externalist version, Rat's perceptual evidence is
constituted in part by the objects she is perceiving. Just as seeing a
dagger and hallucinating a dagger provide different evidence, so does
seeing a dagger and sim-seeing a sim-dagger. For reasns Williamson
notes, a Sim may not know that she has different evidence to someone
seeing a dagger when she sim-sees a sim-dagger, but that does not imply
that she does not have different evidence unless one also assumes,
implausibly, that agents know exactly what their evidence is Williamson
(\citeproc{ref-Williamson2000-WILSAE-2}{2000}). On the internalist
version, our evidence is constituted by our sensory irritations, just as
Quine said it is (\citeproc{ref-Quine1973}{Quine 1973}). If Rat's
evidence includes the fact that her eyes are being irritated
thus-and-so, his credence conditional on that that she is human should
be 1, for if she were a Sim she could not have this evidence because she
would not have eyes. She may, depending on the kind of Sim she is, have
sim-eyes, but sim-eyes are not eyes. So Bostrom needs an argument that
evidence supervenes on conscious experiences, and he doesn't clearly
have one. This is not to say that no such argument could exist. For
example, Laurence BonJour provides some intriguing grounds for thinking
that our fundamental evidence does consist in certain kinds of conscious
states, namely occurrent beliefs (\citeproc{ref-BonJour1999}{BonJour
1999}), but we're a long way from knowing that the supervenience claims
holds. And if the supervenience claim does not hold, then even if Sims
and humans have the same kind of \emph{experiences}, they may not have
the same kind of \emph{evidence}. And if that is true, it is open to us
to hold that Rat's non-experiential evidence entails that she is not a
Sim (as both Williamson and Quine suggest), so her evidence will not be
independent of the question of whether she is a Sim.

Secondly, even if every one of Rat's experiences is probabilistically
independent of the hypothesis that she is a Sim, that doesn't give us a
sufficient reason to believe that her total evidence is so independent.
Just because \emph{e}\textsubscript{1} and \emph{e}\textsubscript{2} are
both probabilistically independent of \emph{H}, the conjunction
\emph{e}\textsubscript{1}~∧~\emph{e}\textsubscript{2} might not be
independent of \emph{H}. So possibly our reasons for accepting P2
involve a tacit scope confusion.\footnote{Thanks to Jamie Dreier for
  reminding me of this point.}

Finally, we might wonder just why we'd even think that Rat's evidence is
probabilistically independent of the hypothesis that she is human. To be
sure, her evidence does not entail that she is human. But that cannot be
enough to show that it is probabilistically independent. For the
evidence also does not entail that she is suman. And if P2 is true, then
the evidence must have quite a bit of bearing on whether she is suman.
For Rat's prior credence in being suman is above 0.9 but apparently her
posterior credence in it should be below 0.15. So the mere fact that the
evidence does not entail that she is human cannot show that it is
probabilistically independent of her being human, for the same reasoning
would show it is probabilistically independent of his being suman.

More generally, we still need a distinction here between the property of
being human and the property of being suman that shows why ordinary
evidence should be independent of the first property but not the second.
One might think the distinction can reside in the fact that \emph{being
human} is a natural property, while \emph{being suman} is gruesome. The
lesson of Goodman's riddle of induction is that we have to give a
privileged position in our epistemic framework to natural properties
like \emph{being human}, and this explains the distinction. This
response gets the status of privileged and gruesome properties
back-to-front. The real lesson of Goodman's riddle is that credences in
hypotheses involving natural properties should be distinctively
\emph{sensitive} to new evidence. Our evidence should make us quite
confident that all emeralds are green, while giving us little reason to
think that all emeralds are grue. What P2 says is that a rather natural
hypothesis, that Rat is human, is \emph{insensitive} to all the evidence
Rat has, while a rather gruesome hypothesis, that Rat is suman, is
\emph{sensitive} to this evidence. The riddle of induction gives us no
reason to believe that should happen.

It seems, though this is a little speculative, that the only reason for
accepting P2 involves a simple fallacy. It is true that we have no
reason to think that some evidence, say \emph{C}, is more or less likely
given that Rat is human rather than a Sim. But from this we should
\emph{not} conclude that we \emph{have} a reason to think it is not more
or less likely given that Rat is human rather than a Sim, which is what
P2 requires. Indeed, drawing this kind of conclusion will quickly lead
to a contradiction, for we can use the same `reasoning' to conclude that
we have a reason to think her evidence is not more or less likely given
that Rat is a suman rather than a him.

\section{Conclusion}\label{conclusion}

Nothing I have said here implies that Rat should have a high credence in
her being human. But it does make one argument that she should not have
a high credence in this look rather tenuous. Further, it is quite
plausible that if there is no good reason not to give high credence to a
hypothesis, then it is rationally permissible to give it such a high
credence. It may not be rationally mandatory to give it such a high
credence, but it is permissible. If Rat is very confident that she is
human, even while knowing that most human-like beings are Sims, she has
not violated any norms of reasoning, and hence is not thereby
irrational. In that respect she is a bit like you and me.

\subsection*{References}\label{references}
\addcontentsline{toc}{subsection}{References}

\phantomsection\label{refs}
\begin{CSLReferences}{1}{0}
\bibitem[\citeproctext]{ref-BonJour1999}
BonJour, Laurence. 1999. {``Foundationalism and the External World.''}
\emph{Philosophical Perspectives} 13: 229--49. doi:
\href{https://doi.org/10.1111/0029-4624.33.s13.11}{10.1111/0029-4624.33.s13.11}.

\bibitem[\citeproctext]{ref-Bostrom2003}
Bostrom, Nick. 2003. {``Are You Living in a Computer Simulation?''}
\emph{The Philosophical Quarterly} 53 (211): 243--55. doi:
\href{https://doi.org/10.1111/1467-9213.00309}{10.1111/1467-9213.00309}.

\bibitem[\citeproctext]{ref-Elga2000-ELGSBA}
Elga, Adam. 2000. {``Self-Locating Belief and the Sleeping Beauty
Problem.''} \emph{Analysis} 60 (2): 143--47. doi:
\href{https://doi.org/10.1093/analys/60.2.143}{10.1093/analys/60.2.143}.

\bibitem[\citeproctext]{ref-Goodman1955}
Goodman, Nelson. 1955. \emph{Fact, Fiction and Forecast}. Cambridge:
Harvard University Press.

\bibitem[\citeproctext]{ref-Lewis1979b}
Lewis, David. 1979. {``Attitudes \emph{de Dicto} and \emph{de Se}.''}
\emph{Philosophical Review} 88 (4): 513--43. doi:
\href{https://doi.org/10.2307/2184646}{10.2307/2184646}. Reprinted in
his \emph{Philosophical Papers}, Volume 1, Oxford: Oxford University
Press, 1983, 133-156. References to reprint.

\bibitem[\citeproctext]{ref-Quine1973}
Quine, W. V. O. 1973. \emph{The Roots of Reference}. La Salle: Open
Court.

\bibitem[\citeproctext]{ref-Williamson2000-WILSAE-2}
Williamson, Timothy. 2000. {``{Scepticism and Evidence}.''}
\emph{Philosophy and Phenomenological Research} 60 (3): 613--28. doi:
\href{https://doi.org/10.2307/2653819}{10.2307/2653819}.

\end{CSLReferences}



\noindent Published in\emph{
Philosophical Quarterly}, 2003, pp. 425-431.

\end{document}
