% Options for packages loaded elsewhere
\PassOptionsToPackage{unicode}{hyperref}
\PassOptionsToPackage{hyphens}{url}
%
\documentclass[
  10pt,
  letterpaper,
  DIV=11,
  numbers=noendperiod,
  twoside]{scrartcl}

\usepackage{amsmath,amssymb}
\usepackage{setspace}
\usepackage{iftex}
\ifPDFTeX
  \usepackage[T1]{fontenc}
  \usepackage[utf8]{inputenc}
  \usepackage{textcomp} % provide euro and other symbols
\else % if luatex or xetex
  \usepackage{unicode-math}
  \defaultfontfeatures{Scale=MatchLowercase}
  \defaultfontfeatures[\rmfamily]{Ligatures=TeX,Scale=1}
\fi
\usepackage{lmodern}
\ifPDFTeX\else  
    % xetex/luatex font selection
  \setmainfont[ItalicFont=EB Garamond Italic,BoldFont=EB Garamond
Bold]{EB Garamond Math}
  \setsansfont[]{Europa-Bold}
  \setmathfont[]{Garamond-Math}
\fi
% Use upquote if available, for straight quotes in verbatim environments
\IfFileExists{upquote.sty}{\usepackage{upquote}}{}
\IfFileExists{microtype.sty}{% use microtype if available
  \usepackage[]{microtype}
  \UseMicrotypeSet[protrusion]{basicmath} % disable protrusion for tt fonts
}{}
\usepackage{xcolor}
\usepackage[left=1in, right=1in, top=0.8in, bottom=0.8in,
paperheight=9.5in, paperwidth=6.5in, includemp=TRUE, marginparwidth=0in,
marginparsep=0in]{geometry}
\setlength{\emergencystretch}{3em} % prevent overfull lines
\setcounter{secnumdepth}{3}
% Make \paragraph and \subparagraph free-standing
\ifx\paragraph\undefined\else
  \let\oldparagraph\paragraph
  \renewcommand{\paragraph}[1]{\oldparagraph{#1}\mbox{}}
\fi
\ifx\subparagraph\undefined\else
  \let\oldsubparagraph\subparagraph
  \renewcommand{\subparagraph}[1]{\oldsubparagraph{#1}\mbox{}}
\fi


\providecommand{\tightlist}{%
  \setlength{\itemsep}{0pt}\setlength{\parskip}{0pt}}\usepackage{longtable,booktabs,array}
\usepackage{calc} % for calculating minipage widths
% Correct order of tables after \paragraph or \subparagraph
\usepackage{etoolbox}
\makeatletter
\patchcmd\longtable{\par}{\if@noskipsec\mbox{}\fi\par}{}{}
\makeatother
% Allow footnotes in longtable head/foot
\IfFileExists{footnotehyper.sty}{\usepackage{footnotehyper}}{\usepackage{footnote}}
\makesavenoteenv{longtable}
\usepackage{graphicx}
\makeatletter
\def\maxwidth{\ifdim\Gin@nat@width>\linewidth\linewidth\else\Gin@nat@width\fi}
\def\maxheight{\ifdim\Gin@nat@height>\textheight\textheight\else\Gin@nat@height\fi}
\makeatother
% Scale images if necessary, so that they will not overflow the page
% margins by default, and it is still possible to overwrite the defaults
% using explicit options in \includegraphics[width, height, ...]{}
\setkeys{Gin}{width=\maxwidth,height=\maxheight,keepaspectratio}
% Set default figure placement to htbp
\makeatletter
\def\fps@figure{htbp}
\makeatother
% definitions for citeproc citations
\NewDocumentCommand\citeproctext{}{}
\NewDocumentCommand\citeproc{mm}{%
  \begingroup\def\citeproctext{#2}\cite{#1}\endgroup}
\makeatletter
 % allow citations to break across lines
 \let\@cite@ofmt\@firstofone
 % avoid brackets around text for \cite:
 \def\@biblabel#1{}
 \def\@cite#1#2{{#1\if@tempswa , #2\fi}}
\makeatother
\newlength{\cslhangindent}
\setlength{\cslhangindent}{1.5em}
\newlength{\csllabelwidth}
\setlength{\csllabelwidth}{3em}
\newenvironment{CSLReferences}[2] % #1 hanging-indent, #2 entry-spacing
 {\begin{list}{}{%
  \setlength{\itemindent}{0pt}
  \setlength{\leftmargin}{0pt}
  \setlength{\parsep}{0pt}
  % turn on hanging indent if param 1 is 1
  \ifodd #1
   \setlength{\leftmargin}{\cslhangindent}
   \setlength{\itemindent}{-1\cslhangindent}
  \fi
  % set entry spacing
  \setlength{\itemsep}{#2\baselineskip}}}
 {\end{list}}
\usepackage{calc}
\newcommand{\CSLBlock}[1]{\hfill\break\parbox[t]{\linewidth}{\strut\ignorespaces#1\strut}}
\newcommand{\CSLLeftMargin}[1]{\parbox[t]{\csllabelwidth}{\strut#1\strut}}
\newcommand{\CSLRightInline}[1]{\parbox[t]{\linewidth - \csllabelwidth}{\strut#1\strut}}
\newcommand{\CSLIndent}[1]{\hspace{\cslhangindent}#1}

\setlength\heavyrulewidth{0ex}
\setlength\lightrulewidth{0ex}
\usepackage[automark]{scrlayer-scrpage}
\clearpairofpagestyles
\cehead{
  Brian Weatherson
  }
\cohead{
  Knowledge, Bets and Interests
  }
\ohead{\bfseries \pagemark}
\cfoot{}
\makeatletter
\newcommand*\NoIndentAfterEnv[1]{%
  \AfterEndEnvironment{#1}{\par\@afterindentfalse\@afterheading}}
\makeatother
\NoIndentAfterEnv{itemize}
\NoIndentAfterEnv{enumerate}
\NoIndentAfterEnv{description}
\NoIndentAfterEnv{quote}
\NoIndentAfterEnv{equation}
\NoIndentAfterEnv{longtable}
\NoIndentAfterEnv{abstract}
\renewenvironment{abstract}
 {\vspace{-1.25cm}
 \quotation\small\noindent\rule{\linewidth}{.5pt}\par\smallskip
 \noindent }
 {\par\noindent\rule{\linewidth}{.5pt}\endquotation}
\KOMAoption{captions}{tableheading}
\makeatletter
\@ifpackageloaded{caption}{}{\usepackage{caption}}
\AtBeginDocument{%
\ifdefined\contentsname
  \renewcommand*\contentsname{Table of contents}
\else
  \newcommand\contentsname{Table of contents}
\fi
\ifdefined\listfigurename
  \renewcommand*\listfigurename{List of Figures}
\else
  \newcommand\listfigurename{List of Figures}
\fi
\ifdefined\listtablename
  \renewcommand*\listtablename{List of Tables}
\else
  \newcommand\listtablename{List of Tables}
\fi
\ifdefined\figurename
  \renewcommand*\figurename{Figure}
\else
  \newcommand\figurename{Figure}
\fi
\ifdefined\tablename
  \renewcommand*\tablename{Table}
\else
  \newcommand\tablename{Table}
\fi
}
\@ifpackageloaded{float}{}{\usepackage{float}}
\floatstyle{ruled}
\@ifundefined{c@chapter}{\newfloat{codelisting}{h}{lop}}{\newfloat{codelisting}{h}{lop}[chapter]}
\floatname{codelisting}{Listing}
\newcommand*\listoflistings{\listof{codelisting}{List of Listings}}
\makeatother
\makeatletter
\makeatother
\makeatletter
\@ifpackageloaded{caption}{}{\usepackage{caption}}
\@ifpackageloaded{subcaption}{}{\usepackage{subcaption}}
\makeatother
\ifLuaTeX
  \usepackage{selnolig}  % disable illegal ligatures
\fi
\usepackage{bookmark}

\IfFileExists{xurl.sty}{\usepackage{xurl}}{} % add URL line breaks if available
\urlstyle{same} % disable monospaced font for URLs
\hypersetup{
  pdftitle={Knowledge, Bets and Interests},
  pdfauthor={Brian Weatherson},
  hidelinks,
  pdfcreator={LaTeX via pandoc}}

\title{Knowledge, Bets and Interests}
\author{Brian Weatherson}
\date{2012}

\begin{document}
\maketitle
\begin{abstract}
This paper argues that the interest-relativity of knowledge cannot be
explained by the interest-relativity of belief. The discussion starts
with an argument that knowledge plays a key pair of roles in decision
theory. It is then argued that knowledge cannot play that role unless
knowledge is interest-relative. The theory of the interest-relativity of
belief is reviewed and revised. That theory can explain some of the
cases that are used to suggest knowledge is interest-relative. But it
can't explain some cases involving ignorance, or mistake, about the odds
at which a bet is offered. The paper ends with an argument that these
cases require positing interest-relative defeaters, which affect whether
an agent knows something without affecting whether she believes it, or
is justified in believing it.
\end{abstract}

\setstretch{1.1}
When you pick up a volume like this one, which describes itself as being
about `knowledge ascriptions', you probably expect to find it full of
papers on epistemology, broadly construed. And you'd probably expect
many of those papers to concern themselves with cases where the
interests of various parties (ascribers, subjects of the ascriptions,
etc.) change radically, and this affects the truth values of various
ascriptions. And, at least in this paper, your expectations will be
clearly met.

But here's an interesting contrast. If you'd picked up a volume of
papers on `belief ascriptions', you'd expect to find a radically
different menu of writers and subjects. You'd expect to find a lot of
concern about names and demonstratives, and about how they can be used
by people not entirely certain about their denotation. More generally,
you'd expect to find less epistemology, and much more mind and language.
I haven't read all the companion papers to mine in this volume, but I
bet you won't find much of that here.

This is perhaps unfortunate, since belief ascriptions and knowledge
ascriptions raise at least some similar issues. Consider a kind of
contextualism about belief ascriptions, which holds that (L) can be
truly uttered in some contexts, but not in others, depending on just
what aspects of Lois Lane's psychology are relevant in the
conversation.\footnote{The reflections in the next few paragraphs are
  inspired by some comments by Stalnaker in his
  (\citeproc{ref-Stalnaker2008}{2008}), though I don't want to suggest
  the theory I'll discuss is actually Stalnaker's.}

\begin{description}
\tightlist
\item[(L)]
Lois Lane believes that Clark Kent is vulnerable to kryptonite.
\end{description}

We could imagine a theorist who says that whether (L) can be uttered
truly depends on whether it matters to the conversation that Lois Lane
might not recognise Clark Kent when he's wearing his Superman uniform.
And, this theorist might continue, this isn't because `Clark Kent' is a
context-sensitive expression; it is rather because `believes' is
context-sensitive. Such a theorist will also, presumably, say that
whether (K) can be uttered truly is context-sensitive.

\begin{description}
\tightlist
\item[(K)]
Lois Lane knows that Clark Kent is vulnerable to kryptonite.
\end{description}

And so, our theorist is a kind of contextualist about knowledge
ascriptions. But they might agree with approximately none of the
motivations for contextualism about knowledge ascriptions put forward by
Cohen (\citeproc{ref-Cohen1988}{1988}), DeRose
(\citeproc{ref-DeRose1995}{1995}) or Lewis
(\citeproc{ref-Lewis1996b}{1996}). Rather, they are a contextualist
about knowledge ascriptions solely because they are contextualist about
belief ascriptions like (L).

Call the position I've just described \textbf{doxastic contextualism}
about knowledge ascriptions. It's a kind of contextualism all right; it
says that (K) is context sensitive, and not merely because of the
context-sensitivity of any term in the `that'-clause. But it explains
the contextualism solely in terms of the contextualism of belief
ascriptions. The more familiar kind of contextualism about knowledge
ascriptions we'll call \textbf{non-doxastic contextualism}. Note that
the way we're classifying theories, a view that holds that (K) is
context-sensitive both because (L) is context-sensitive \emph{and}
because Cohen \emph{et al} are correct is a version of non-doxastic
contextualism. The label `non-doxastic' is being used to mean that the
contextualism isn't solely doxastic, rather than as denying
contextualism about belief ascriptions.

We can make the same kind of division among interest-relative
invariantist, or IRI, theories of knowledge ascriptions. Any kind of IRI
will say that there are sentences of the form \emph{S knows that p}
whose truth depends on the interests, in some sense, of \emph{S}. But we
can divide IRI theories up the same way that we divide up contextualist
theories.

\begin{description}
\tightlist
\item[Doxastic IRI]
Knowledge ascriptions are interest-relative, but their
interest-relativity traces solely to the interest-relativity of the
corresponding belief ascriptions.
\item[Non-Doxastic IRI]
Knowledge ascriptions are interest-relative, and their
interest-relativity goes beyond the interest-relativity of the
corresponding belief ascriptions.
\end{description}

Again, a theory that holds both that belief ascriptions are
interest-relative, and that some of the interest-relativity of knowledge
ascriptions is not explained by the interest-relativity of belief
ascriptions, will count as a version of non-doxastic IRI. I'm going to
defend a view from this class here.

In my -Weatherson (\citeproc{ref-Weatherson2005-WEACWD}{2005}) I tried
to motivate Doxastic IRI. It isn't completely trivial to map my view
onto the existing views in the literature, but the idea was to renounce
contextualism and all its empty promises, and endorse a position that's
usually known as `strict invariantism' about these classes of
statements:

\begin{itemize}
\tightlist
\item
  \emph{S} is justified in having credence \emph{x} in \emph{p};
\item
  If \emph{S} believes that \emph{p}, she knows that \emph{p};
\end{itemize}

while holding that the interests of S are relevant to the truth of
statements from these classes:

\begin{itemize}
\tightlist
\item
  \emph{S} believes that \emph{p};
\item
  \emph{S} justifiably believes that \emph{p};
\item
  \emph{S} knows that \emph{p}.
\end{itemize}

But I didn't argue for all of that. What I argued for was Doxastic IRI
about ascriptions of justified belief, and I hinted that the same
arguments would generalise to knowledge ascriptions. I now think those
hints were mistaken, and want to defend Non-Doxastic IRI about knowledge
ascriptions.\footnote{Whether Doxastic or Non-Doxastic IRI is true about
  justified belief ascriptions turns on some tricky questions about what
  to say when a subject's credences are nearly, but not exactly
  appropriate given her evidence. Space considerations prevent a full
  discussion of those cases here. Whether I can hold onto the strict
  invariantism about claims about justified credences depends, I now
  think, on whether an interest-neutral account of evidence can be
  given. Discussions with Tom Donaldson and Jason Stanley have left me
  less convinced than I was in 2005 that this is possible, but this is
  far too big a question to resolve here.} My change of heart has been
prompted by cases like those Jason Stanley
(\citeproc{ref-Stanley2005-STAKAP}{2005}) calls `Ignorant High Stakes'
cases.\footnote{I mean here the case of Coraline, to be discussed in
  section 3 below. Several people have remarked in conversation that
  Coraline doesn't look to them like a case of Ignorant High Stakes.
  This isn't surprising; Coraline is better described as being
  \emph{mistaken} than \emph{ignorant}, and she's mistaken about odds
  not stakes. If they're right, that probably means my argument for
  Non-Doxastic IRI is less like Stanley's, and hence more original, than
  I think it is. So I don't feel like pressing the point! But I do want
  to note that \emph{I} thought the Coraline example was a variation on
  a theme Stanley originated.} But to see why these cases matter, it
will help to start with why I think some kind of IRI must be true.

Here's the plan of attack. In Section~\ref{sec-1}, I'm going to argue
that knowledge plays an important role in decision theory. In
particular, I'll argue (a) that it is legitimate to write something onto
a decision table iff the decision maker knows it to be true, and (b) it
is legitimate to leave a possible state of the world off a decision
table iff the decision maker knows it not to obtain. I'll go on to argue
that this, plus some very plausible extra assumptions about the
rationality of certain possible choices, implies that knowledge is
interest-relative. In Section~\ref{sec-2} I'll summarise and extend the
argument from Weatherson (\citeproc{ref-Weatherson2005-WEACWD}{2005})
that belief is interest-relative. People who are especially interested
in the epistemology rather than the theory of belief may skip this. But
I think this material is important; most of the examples of
interest-relative knowledge in the literature can be explained by the
interest-relativity of belief. I used to think all such cases could be
explained. Section~\ref{sec-3} describes why I no longer think that.
Reflections on cases like the Coraline example suggests that there are
coherence constraints on knowledge that go beyond the coherence
constraints on justified true belief. The scope of these constraints is,
I'll argue, interest-relative. So knowledge, unlike belief or justified
belief, has interest-relative defeaters. That's inconsistent with
Doxastic IRI, so Doxastic IRI is false.

\section{The Interest-Relativity of Knowledge}\label{sec-1}

\subsection{The Struction of Decision
Problems}\label{the-struction-of-decision-problems}

Professor Dec is teaching introductory decision theory to her
undergraduate class. She is trying to introduce the notion of a dominant
choice. So she introduces the following problem, with two states,
\emph{S}\textsubscript{1} and \emph{S}\textsubscript{2}, and two
choices, \emph{C}\textsubscript{1} and \emph{C}\textsubscript{2}, as is
normal for introductory problems.

\begin{longtable}[]{@{}lcc@{}}
\toprule\noalign{}
~ & ~\emph{S}\textsubscript{1} ~ & ~\emph{S}\textsubscript{2} \\
\midrule\noalign{}
\endhead
\bottomrule\noalign{}
\endlastfoot
\emph{C}\textsubscript{1}~ & ~-\$200~ & ~\$1000 \\
\emph{C}\textsubscript{2}~ & ~-\$100~ & ~\$1500 \\
\end{longtable}

She's hoping that the students will see that \emph{C}\textsubscript{1}
and \emph{C}\textsubscript{2} are bets, but \emph{C}\textsubscript{2} is
clearly the better bet. If \emph{S}\textsubscript{1} is actual, then
both bets lose, but \emph{C}\textsubscript{2} loses less money. If
\emph{S}\textsubscript{2} is actual, then both bets win, but
\emph{C}\textsubscript{2} wins more. So \emph{C}\textsubscript{2} is
better. That analysis is clearly wrong if the state is causally
dependent on the choice, and controversial if the states are
evidentially dependent on the choices. But Professor Dec has not given
any reason for the students to think that the states are dependent on
the choices in either way, and in fact the students don't worry about
that kind of dependence.

That doesn't mean, however, that the students all adopt the analysis
that Professor Dec wants them to. One student, Stu, is particularly
unwilling to accept that \emph{C}\textsubscript{2} is better than
\emph{C}\textsubscript{1}. He thinks, on the basis of his experience,
that when more than \$1000 is on the line, people aren't as reliable
about paying out on bets. So while \emph{C}\textsubscript{1} is
guaranteed to deliver \$1000 if \emph{S}\textsubscript{2}, if the agent
bets on \emph{C}\textsubscript{2}, she might face some difficulty in
collecting on her money.

Given the context, i.e., that they are in an undergraduate decision
theory class, it seems that Stu has misunderstood the question that
Professor Dec intended to ask. But it is a little harder than it first
seems to specify just exactly what Stu's mistake is. It isn't that he
thinks Professor Dec has \emph{misdescribed} the situation. It isn't
that he thinks the agent won't collect \$1500 if she chooses
\emph{C}\textsubscript{2} and is in \emph{S}\textsubscript{2}. He just
thinks that she \emph{might} not be able to collect it, so the expected
payout might really be a little less than \$1500.

But Stu is not the only problem that Professor Dec has. She also has
trouble convincing Dom of the argument. He thinks there should be a
third state added, \emph{S}\textsubscript{3}. In
\emph{S}\textsubscript{3}, there is a vengeful God who is about to end
the world, and take everyone who chose \emph{C}\textsubscript{1} to
heaven, while sending everyone who chose \emph{C}\textsubscript{2} to
hell. Since heaven is better than hell, \emph{C}\textsubscript{2} does
not dominate \emph{C}\textsubscript{1}; it is worse in
\emph{S}\textsubscript{3}. If decision theory is to be useful, we must
say something about why we can leave states like
\emph{S}\textsubscript{3} off the decision table.

So in order to teach decision theory, Professor Dec has to answer two
questions.\footnote{If we are convinced that the right decision is the
  one that maximises expected utility, there is a sense in which these
  questions collapse. For the expected utility theorist, we can solve
  Dom's question by making sure the states are logically exhaustive, and
  making the `payouts' in each state be expected payouts. But the theory
  that the correct decision is the one that maximises expected utility,
  while plausibly true, is controversial. It shouldn't be assumed when
  we are investigating the semantics of decision tables.}

\begin{enumerate}
\def\labelenumi{\arabic{enumi}.}
\item
  What makes it legitimate to write something on the decision table,
  such as the `\$1500' we write in the bottom right cell of Dec's table?
\item
  What makes it legitimate to leave something off a decision table, such
  as leaving Dom's state \emph{S}\textsubscript{3} off the table?
\end{enumerate}

Let's start with a simpler problem that helps with both questions. Alice
is out of town on a holiday, and she faces the following decision choice
concerning what to do with a token in her hand.

\begin{longtable}[]{@{}rc@{}}
\toprule\noalign{}
\textbf{Choice}~ & ~\textbf{Outcome} \\
\midrule\noalign{}
\endhead
\bottomrule\noalign{}
\endlastfoot
Put token on table~ & ~Win \$1000 \\
Put token in pocket~ & ~Win nothing \\
\end{longtable}

This looks easy, especially if we've taken Professor Dec's class.
Putting the token on the table dominates putting the token in her
pocket. It returns \$1000, versus no gain. So she should put the token
on the table.

I've left Alice's story fairly schematic; let's fill in some of the
details. Alice is on holiday at a casino. It's a fair casino; the
probabilities of the outcomes of each of the games is just what you'd
expect. And Alice knows this. The table she's standing at is a roulette
table. The token is a chip from the casino worth \$1000. Putting the
token on the table means placing a bet. As it turns out, it means
placing a bet on the roulette wheel landing on 28. If that bet wins she
gets her token back and another token of the same value. There are many
other bets she could make, but Alice has decided not to make all but one
of them. Since her birthday is the 28\textsuperscript{th}, she is
tempted to put a bet on 28; that's the only bet she is considering. If
she makes this bet, the objective chance of her winning is
\(\frac{1}{38}\), and she knows this. As a matter of fact she will win,
but she doesn't know this. (This is why the description in the table I
presented above is truthful, though frightfully misleading.) As you can
see, the odds on this bet are terrible. She should have a chance of
winning around \(\frac{1}{2}\) to justify placing this bet.\footnote{Assuming
  Alice's utility curve for money curves downwards, she should be
  looking for a slightly higher chance of winning than \(\frac{1}{2}\)
  to place the bet, but that level of detail isn't relevant to the story
  we're telling here.} So the above table, which makes it look like
placing the bet is the dominant, and hence rational, option, is
misleading.

Just how is the table misleading though? It isn't because what is says
is false. If Alice puts the token on the table she wins \$1000; and if
she doesn't, she stays where she is. It isn't, or isn't just, that Alice
doesn't believe the table reflects what will happen if she places the
bet. As it turns out, Alice is smart, so she doesn't form beliefs about
chance events like roulette wheels. But even if she did, that wouldn't
change how misleading the table is. The table suggests that it is
rational for Alice to put the token on the table. In fact, that is
irrational. And it would still be irrational if Alice believes,
\emph{irrationally}, that the wheel will land on 28.

A better suggestion is that the table is misleading because Alice
doesn't \emph{know} that it accurately depicts the choice she faced. If
she did know that these were the outcomes to putting the token on the
table versus in her pocket, it seems it would be rational for her to put
it on the table. If we take it as tacit in a presentation of a decision
problem that the agent knows that the table accurately depicts the
outcomes of various choices in different states, then we can tell a
plausible story about what the miscommunication between Professor Dec
and her students was. Stu was assuming that if the agent wins \$1500,
she might not be able to easily collect. That is, he was assuming that
the agent does not know that she'll get \$1500 if she chooses
\emph{C}\textsubscript{2} and is in state \emph{S}\textsubscript{2}.
Professor Dec, if she's anything like other decision theory professors,
will have assumed that the agent did know exactly that. And the
miscommunication between Professor Dec and Dom also concerns knowledge.
When Dec wrote that table up, she was saying that the agent knew that
\emph{S}\textsubscript{1} or \emph{S}\textsubscript{2} obtained. And
when she says it is best to take dominating options, she means that it
is best to take options that one knows to have better outcomes. So here
are the answers to Stu and Dom's challenges.

\begin{enumerate}
\def\labelenumi{\arabic{enumi}.}
\tightlist
\item
  It is legitimate to write something on the decision table, such as the
  `\$1500' we write in the bottom right cell of Dec's table, iff the
  decision maker knows it to be true.
\item
  It is legitimate to leave something off a decision table, such as
  leaving Dom's state \emph{S}\textsubscript{3} off the table, iff the
  decision maker knows it not to obtain.
\end{enumerate}

Perhaps those answers are not correct, but what we can clearly see by
reflecting on these cases is that the standard presentation of a
decision problem presupposes not just that the table states what will
happen, but the agent stands in some special doxastic relationship to
the information explicitly on the table (such as that Alice will get
\$1500 if \emph{C}\textsubscript{2} and \emph{S}\textsubscript{2}) and
implied by where the table ends (such as that \emph{S}\textsubscript{3}
will not happen). Could that relationship be weaker than knowledge? It's
true that it is hard to come up with clear counterexamples to the
suggestion that the relationship is merely justified true belief. But I
think it is somewhat implausible to hold that the standard presentation
of an example merely presupposes that the agent has a justified true
belief that the table is correct, and does not in addition know that the
table is correct.

My reasons for thinking this are similar to one of the reasons Timothy
Williamson (\citeproc{ref-Williamson2000-WILKAI}{Williamson 2000} Ch. 9)
gives for doubting that one's evidence is all that one justifiably truly
believes. To put the point in Lewisian terms, it seems that knowledge is
a much more \emph{natural} relation than justified true belief. And when
ascribing contents, especially contents of tacitly held beliefs, we
should strongly prefer to ascribe more rather than less natural
contents.\footnote{I'm here retracting some things I said a few years
  ago in a paper on philosophical methodology
  (\citeproc{ref-Weatherson2003-WEAWGA}{Weatherson 2003}). There I
  argued that identifying knowledge with justified true belief would
  give us a theory on which knowledge was more natural than a theory on
  which we didn't identify knowledge with any other epistemic property.
  I now think that is wrong for a couple of reasons. First, although
  it's true (as I say in the earlier paper) that knowledge can't be
  primitive or perfectly natural, this doesn't make it less natural than
  justification, which is also far from a fundamental feature of
  reality. Indeed, given how usual it is for languages to have a simple
  representation of knowledge, we have some evidence that it is very
  natural for a term from a special science. Second, I think in the
  earlier paper I didn't fully appreciate the point (there attributed to
  Peter Klein) that the Gettier cases show that the property of being a
  justified true belief is not particularly natural. In general, when
  \emph{F} and \emph{G} are somewhat natural properties, then so is the
  property of being \emph{F} ∧ \emph{G}. But there are exceptions,
  especially in cases where these are properties that a whole can have
  in virtue of a part having the property. In those cases, a whole that
  has an \emph{F} part and a \emph{G} part will be \(F \wedge G\), but
  this won't reflect any distinctive property of the whole. And one of
  the things the Gettier cases show is that the properties of
  \emph{being justified} and \emph{being true}, as applied to belief,
  fit this pattern.

  Note that even if you think that philosophers are generally too quick
  to move from instinctive reactions to the Gettier case to abandoning
  the justified true belief theory of knowledge, this point holds up.
  What is important here is that on sufficient reflection, the Gettier
  cases show that some justified true beliefs are not knowledge, and
  that the cases in question also show that being a justified true
  belief is not a particularly natural or unified property. So the point
  I've been making in the last this footnote is independent of the point
  I wanted to stress in ``What Good are Counterexamples?'', namely, that
  philosophers in some areas (especially epistemology) are
  insufficiently reformist in their attitude towards our intuitive
  reactions to cases.}

So the `special doxastic relationship' is not weaker than knowledge.
Could it be stronger? Could it be, for example, that the relationship is
certainty, or some kind of iterated knowledge? Plausibly in some
game-theoretic settings it is stronger -- it involves not just knowing
that the table is accurate, but knowing that the other player knows the
table is accurate. In some cases, the standard treatment of games will
require positing even more iterations of knowledge. For convenience, it
is sometimes explicitly stated that iterations continue indefinitely, so
each party knows the table is correct, and knows each party knows this,
and knows each party knows that, and knows each party knows \emph{that},
and so on. An early example of this in philosophy is in the work by
David Lewis (\citeproc{ref-Lewis1969a}{1969}) on convention. But it is
usually acknowledged (again in a tradition extending back at least to
Lewis) that only the first few iterations are actually needed in any
problem, and it seems a mistake to attribute more iterations than are
actually used in deriving solutions to any particular game.

The reason that would be a mistake is that we want game theory, and
decision theory, to be applicable to real-life situations. There is very
little that we know, and know that we know, and know we know we know,
and so on indefinitely (\citeproc{ref-Williamson2000-WILKAI}{Williamson
2000} Ch. 4). There is, perhaps, even less that we are certain of. If we
only could say that a person is making a particular decision when they
stand in these very strong relationships to the parameters of the
decision table, then people will almost never be making the kinds of
decision we study in decision theory. Since decision theory and game
theory are not meant to be that impractical, I conclude that the
`special doxastic relationship' cannot be that strong. It could be that
in some games, the special relationship will involve a few iterations of
knowledge, but in decision problems, where the epistemic states of
others are irrelevant, even that is unnecessary, and simple knowledge
seems sufficient.

It might be argued here that we shouldn't expect to apply decision
theory directly to real-life problems, but only to idealised versions of
them, so it would be acceptable to, for instance, require that the
things we put in the table are, say, things that have probability
exactly 1. In real life, virtually nothing has probability 1. In an
idealisation, many things do. But to argue this way seems to involve
using `idealisation' in an unnatural sense. There is a sense in which,
whenever we treat something with non-maximal probability as simply given
in a decision problem that we're ignoring, or abstracting away from,
some complication. But we aren't \emph{idealising}. On the contrary,
we're modelling the agent as if they were irrationally certain in some
things which are merely very very probable.

So it's better to say that any application of decision theory to a
real-life problem will involve ignoring certain (counterfactual) logical
or metaphysical possibilities in which the decision table is not
actually true. But not any old abstraction will do. We can't ignore just
anything, at least not if we want a good model. Which abstractions are
acceptable? The response I've offered to Dom's challenge suggests an
answer to this: we can abstract away from any possibility in which
something the agent actually knows is false. I don't have a knock-down
argument that this is the best of all possible abstractions, but nor do
I know of any alternative answer to the question which abstractions are
acceptable which is nearly as plausible.

We might be tempted to say that we can abstract away from anything such
that the difference between its probability and 1 doesn't make a
difference to the ultimate answer to the decision problem. More
carefully, the idea would be that we can have the decision table
represent that \emph{p} iff \emph{p} is true and treating Pr(\emph{p})
as 1 rather than its actual value doesn't change what the agent should
do. I think this is the most plausible story one could tell about
decision tables if one didn't like the knowledge first story that I
tell. But I also don't think it works, because of cases like the
following.

Luc is lucky; he's in a casino where they are offering better than fair
odds on roulette. Although the chance of winning any bet is , if Luc
bets \$10, and his bet wins, he will win \$400. (That's the only bet on
offer.) Luc, like Alice, is considering betting on 28. As it turns out,
28 won't come up, although since this is a fair roulette wheel, Luc
doesn't know this. Luc, like most agents, has a declining marginal
utility for money. He currently has \$1,000, and for any amount of money
\emph{x}, Luc gets utility \emph{u}(\emph{x}) =
\emph{x}\textsuperscript{\(\frac{1}{2}\)} out of having \emph{x}. So
Luc's current utility (from money) is, roughly, 31.622. If he bets and
loses, his utility will be, roughly, 31.464. And if he bets and wins,
his utility will be, roughly, 37.417. So he stands to gain about 5.794,
and to lose about 0.159. So he stands to gain about 36.5 as much as he
stands to lose. Since the odds of winning are less than
\(\frac{1}{36.5}\), his expected utility goes down if he takes the bet,
so he shouldn't take it. Of course, if the probability of losing was 1,
and not merely \(\frac{37}{38}\), he shouldn't take the bet too. Does
that mean it is acceptable, in presenting Luc's decision problem, to
leave off the table any possibility of him winning, since he won't win,
and setting the probability of losing to 1 rather than \(\frac{37}{38}\)
doesn't change the decision he should make? Of course not; that would
horribly misstate the situation Luc finds himself in. It would
misrepresent how sensitive Luc's choice is to his utility function, and
to the size of the stakes. If Luc's utility function was
\emph{u}(\emph{x}) = \emph{x}\textsuperscript{\(\frac{3}{4}\)}, then he
should take the bet. If his utility function is unchanged, but the bet
was \$1 against \$40, rather than \$10 against \$400, he should take the
bet. Leaving off the possibility of winning hides these facts, and badly
misrepresents Luc's situation.

I've argued that the states we can `leave off' a decision table are the
states that the agent knows not to obtain. The argument is largely by
elimination. If we can only leave off things that have probability 1,
then decision theory would be useless; but it isn't. If we say we can
leave off things if setting their probability at 1 is an acceptable
idealisation, we need a theory of acceptable idealisations. If this is
to be a rival to my theory, the idealisation had better not be it's
acceptable to treat anything known as having probability 1. But the most
natural alternative idealisation badly misrepresents Luc's case. If we
say that what can be left off is not what's known not to obtain, but
what is, say, justifiably truly believed not to obtain, we need an
argument for why people would naturally use such an unnatural standard.
This doesn't even purport to be a conclusive argument, but these
considerations point me towards thinking that knowledge determines what
we can leave off.

I also cheated a little in making this argument. When I described Alice
in the casino, I made a few explicit comments about her information
states. And every time, I said that she \emph{knew} various
propositions. It seemed plausible at the time that this is enough to
think those propositions should be incorporated into the table we use to
represent her decision. That's some evidence against the idea that more
than knowledge, perhaps iterated knowledge or certainty, is needed
before we add propositions to the decision table.

\subsection{From Decision Theory to
Interest-Relativity}\label{from-decision-theory-to-interest-relativity}

This way of thinking about decision problems offers a new perspective on
the issue of whether we should always be prepared to bet on what we
know.\footnote{This issue is of course central to the plotline in
  Hawthorne (\citeproc{ref-Hawthorne2004}{2004}).} To focus intuitions,
let's take a concrete case. Barry is sitting in his apartment one
evening when he hears a musician performing in the park outside. The
musician, call her Beth, is one of Barry's favourite musicians, so the
music is familiar to Barry. Barry is excited that Beth is performing in
his neighbourhood, and he decides to hurry out to see the show. As he
prepares to leave, a genie appears an offers him a bet.\footnote{Assume,
  perhaps implausibly, that the sudden appearance of the genie is
  evidentially irrelevant to the proposition that the musician is Beth.
  The reasons this may be implausible are related to the arguments in
  (\citeproc{ref-RunyonGuysDolls}{Runyon 1992, 14--15}). Thanks here to
  Jeremy Fantl.} If he takes the bet, and the musician is Beth, then the
genie will give Barry ten dollars. On the other hand, if the musician is
not Beth, he will be tortured in the fires of hell for a millenium.
Let's put Barry's options in table form.

\begin{longtable}[]{@{}lcc@{}}
\toprule\noalign{}
~ & ~\textbf{Musician is Beth}~ & ~\textbf{Musician is not Beth} \\
\midrule\noalign{}
\endhead
\bottomrule\noalign{}
\endlastfoot
\textbf{Take Bet} ~ & ~ Win \$10 ~ & ~ 1000 years of torture \\
\textbf{Decline Bet}~ & ~ Status quo ~ & ~ Status quo \\
\end{longtable}

Intuitively, it is extremely irrational for Barry to take the bet.
People do make mistakes about identifying musicians, even very familiar
musicians, by the strains of music that drift up from a park. It's not
worth risking a millenium of torture for \$10.

But it also seems that we've misstated the table. Before the genie
showed up, it seemed clear that Barry knew that the musician was Beth.
That was why he went out to see her perform. (If you don't think this is
true, make the sounds from the park clearer, or make it that Barry had
some prior evidence that Beth was performing which the sounds from the
park remind him of. It shouldn't be too hard to come up with an
evidential base such that (a) in normal circumstances we'd say Barry
knew who was performing, but (b) he shouldn't take this genie's bet.)
Now our decision tables should reflect the knowledge of the agent making
the decision. If Barry knows that the musician is Beth, then the second
column is one he knows will not obtain. So let's write the table in the
standard form.

\begin{longtable}[]{@{}lcl@{}}
\toprule\noalign{}
~ & ~\textbf{Musician is Beth}~ & ~ \\
\midrule\noalign{}
\endhead
\bottomrule\noalign{}
\endlastfoot
\textbf{Take Bet} ~ & ~ Win \$10 ~ & ~ \\
\textbf{Decline Bet}~ & ~ Status quo ~ & ~ \\
\end{longtable}

And it is clear what Barry's decision should be in this situation.
Taking the bet dominates declining it, and Barry should take dominating
options.

What has happened? It is incredibly clear that Barry should decline the
bet, yet here we have an argument that he should take the bet. If you
accept that the bet should be declined, then it seems to me that there
are three options available.

\begin{enumerate}
\def\labelenumi{\arabic{enumi}.}
\tightlist
\item
  Barry never knew that the musician was Beth.
\item
  Barry did know that the musician was Beth, but this knowledge was
  destroyed by the genie's offer of the bet.
\item
  States of the world that are known not to obtain should still be
  represented in decision problems, so taking the bet is not a
  dominating option.
\end{enumerate}

The first option is basically a form of scepticism. If the take-away
message from the above discussion is that Barry doesn't know the
musician is Beth, we can mount a similar argument to show that he knows
next to nothing.\footnote{The idea that interest-relativity is a way of
  fending off scepticism is a very prominent theme in Fantl and McGrath
  (\citeproc{ref-FantlMcGrath2009}{2009}).} And the third option would
send us back into the problems about interpreting and applying decision
theory that we spent the first few pages trying to get out of.

So it seems that the best solution here, or perhaps the least bad
solution, is to accept that knowledge is interest-relative. Barry did
know that the musician was Beth, but the genie's offer destroyed that
knowledge. When Barry was unconcerned with bets at extremely long odds
on whether the musician is Beth, he knows Beth is the musician. Now that
he is interested in those bets, he doesn't know that.\footnote{On the
  version of IRI I'm defending, Barry is free to be interested in
  whatever he likes. If he started wondering about whether it would be
  rational to take such a bet, he loses the knowledge that Beth is the
  musician, even if there is no genie and the bet isn't offered. The
  existence of the genie's offer makes the bet a practical interest;
  merely wondering about the genie's offer makes the bet a cognitive
  interest. But both kinds of interests are relevant to knowledge.}

The argument here bears more than a passing resemblance to the arguments
in favour of interest-relativity that are made by Hawthorne, Stanley,
and Fantl and McGrath. But I think the focus on decision theory shows
how we can get to interest-relativity with very weak
premises.\footnote{As they make clear in their
  (\citeproc{ref-Hawthorne2008-HAWKAA}{2008}), Hawthorne and Stanley are
  interested in defending relatively strong premises linking knowledge
  and action independently of the argument for the interest-relativity
  of knowledge. What I'm doing here is showing how that conclusion does
  not rest on anything nearly as strong as the principles they believe,
  and so there is plenty of space to disagree with their general
  principles, but accept interest-relativity. The strategy here isn't a
  million miles from the point noted in Fantl and McGrath
  (\citeproc{ref-FantlMcGrath2009}{2009, 72n14}) when they note that
  much weaker premises than the ones they endorse imply a failure of
  `purism'.} In particular, the only premises I've used to derive an
interest-relative conclusion are:

\begin{enumerate}
\def\labelenumi{\arabic{enumi}.}
\tightlist
\item
  Before the genie showed up, Barry knew the musician was Beth.
\item
  It's rationally permissible, \emph{in cases like Barry's}, to take
  dominating options.
\item
  It's always right to model decision problems by including what the
  agent knows in the `framework'. That is, our decision tables should
  include what the agent knows about the payoffs in different states,
  and leave off any state the agent knows not to obtain.
\item
  It is rationally impermissible for Barry to take the genie's offered
  bet.
\end{enumerate}

The second premise there is \emph{much} weaker than the principles
linking knowledge and action defended in previous arguments for
interest-relativity. It isn't the claim that one can always act on what
one knows, or that one can only act on what one knows, or that knowledge
always (or only) provides reason to act. It's just the claim that in one
very specific type of situation, in particular when one has to make a
relatively simple bet, which affects nobody but the person making the
bet, it's rationally permissible to take a dominating option. In
conjunction with the third premise, it entails that \emph{in those kind
of cases}, the fact that one knows taking the bet will lead to a better
outcome suffices for making acceptance of the bet rationally
permissible. It doesn't say anything about what else might or might not
make acceptance rationally permissible. It doesn't say anything about
what suffices for rationally permissibility in other kinds of cases,
such as cases where someone else's interests are at stake, or where
taking the bet might violate a deontological constraint, or any other
way in which real-life choices differ from the simplest decision
problems.\footnote{I have more to say about those cases in section 2.2.}
It doesn't say anything about any other kind of permissibility, e.g.,
moral permissibility. But it doesn't need to, because we're only in the
business of proving that there is \emph{some} interest-relativity to
knowledge, and an assumption about practical rationality in some range
of cases suffices to prove that.\footnote{Also note that I'm not taking
  as a premise any claim about what Barry knows after the bet is
  offered. A lot of work on interest-relativity has used such premises,
  or premises about related intuitions. This seems like a misuse of the
  method of cases to me. That's not because we should never use
  intuitions about cases, just that these cases are too hard to think
  that snap judgments about them are particularly reliable. In general,
  we can know a lot about cases by quickly reflecting on them.
  Similarly, we know a lot about which shelves are level and which are
  uneven by visual inspection, i.e., `eyeballing'. But when different
  eyeballs disagree, it's time to bring in other tools. That's the
  approach of this paper. I don't have a story about why the various
  eyeballs disagree about cases like Barry's; that seems like a task
  best undertaken by a psychologist not a philosopher
  (\citeproc{ref-Ichikawa2009}{Ichikawa 2009}).}

The case of Barry and Beth also bears some relationship to one of the
kinds of case that have motivated contextualism about knowledge. Indeed,
it has been widely noted in the literature on interest-relativity that
interest-relativity can explain away many of the puzzles that motivate
contextualism. And there are difficulties that face any contextualist
theory (\citeproc{ref-Weatherson2006-WEAQC}{Weatherson 2006}). So I
prefer an \emph{invariantist} form of interest-relativity about
knowledge. That is, my view is a form of interest-relative-invariantism,
or IRI.\footnote{This is obviously not a full argument against
  contextualism; that would require a much longer paper than this.}

Now everything I've said here leaves it open whether the
interest-relativity of knowledge is a natural and intuitive theory, or
whether it is a somewhat unhappy concession to difficulties that the
case of Barry and Beth raise. I think the former is correct, and
interest-relativity is fairly plausible on its own merits, but it would
be consistent with my broader conclusions to say that in fact the
interest-relative theory of knowledge is very implausible and
counterintuitive. If we said that, we could still justify the
interest-relative theory by noting that we have on our hands here a
paradoxical situation, and any option will be somewhat implausible. This
consideration has a bearing on how we should think about the role of
intuitions about cases, or principles, in arguments that knowledge is
interest-relative. Several critics of the view have argued that the view
is counter-intuitive, or that it doesn't accord with the reactions of
non-expert judges.\footnote{See, for instance, Blome-Tillmann
  (\citeproc{ref-MBT2009}{2009}), or Feltz and Zarpentine
  (\citeproc{ref-FeltzZarpentine2010}{2010}).} In a companion paper,
``Defending Interest-Relative Invariantism'', I note that those
arguments usually misconstrue what the consequences of interest-relative
theories of knowledge are. But even if they don't, I don't think there's
any quick argument that if interest-relativity is counter-intuitive, it
is false. After all, the only alternatives that seem to be open here are
very counter-intuitive.

Finally, it's worth noting that if Barry is rational, he'll stop (fully)
believing that the musician is Beth once the genie makes the offer.
Assuming the genie allows this, it would be very natural for Barry to
try to acquire more information about the singer. He might walk over to
the window to see if he can see who is performing in the park. So this
case leaves it open whether the interest-relativity of knowledge can be
explained fully by the interest-relativity of belief. I used to think it
could be; I no longer think that. To see why this is so, it's worth
rehearsing how the interest-relative theory of belief runs.

\section{The Interest-Relativity of Belief}\label{sec-2}

\subsection{Interests and Functional
Roles}\label{interests-and-functional-roles}

The previous section was largely devoted to proving an existential
claim: there is \emph{some} interest-relativity to knowledge. Or, if you
prefer, it proved a negative claim: the best theory of knowledge is
\emph{not} interest-neutral. But this negative conclusion invites a
philosophical challenge: what is the best explanation of the
interest-relativity of knowledge? My answer is in two parts. Part of the
interest-relativity of knowledge comes from the interest-relativity of
belief, and part of it comes from the fact that interests generate
certain kinds of doxastic defeaters. It's the second part, the part that
is new to this paper, that makes the theory a version of non-doxastic
IRI.

Here's my theory of belief. \emph{S} believes that \emph{p} iff
conditionalising on \emph{p} doesn't change \emph{S}'s answer to any
relevant question. I'm using `relevance' here in a non-technical sense;
I say a lot more about how to cash out the notion in my
(\citeproc{ref-Weatherson2005-WEACWD}{2005}). The key thing to note is
that relevance is interest-relative, so the theory of belief is
interest-relative. There is a bit more to say about what kind of
\emph{questions} are important for this definition of belief. In part
because I've changed my mind a little bit on this since the earlier
paper, I'll spend a bit more time on it. The following four kinds of
questions are the most important.

\begin{itemize}
\tightlist
\item
  How probable is \emph{q}?
\item
  Is \emph{q} or \emph{r} more probable?
\item
  How good an idea is it to do \(\phi\)?
\item
  Is it better to do \(\phi\) or \(\psi\)?
\end{itemize}

The theory of belief says that someone who believes that \emph{p}doesn't
change their answer to any of these questions upon conditionalising on
\emph{p}. Putting this formally, and making the restriction to relevant
questions explicit, we get the following theorems of our theory of
belief.\footnote{In the last two lines, I use \emph{U}(\(\phi\)) to
  denote the expected utility of \(\phi\), and
  \emph{U}(\(\phi\)~\textbar~\emph{p}) to denote the expected utility of
  \(\phi\) conditional on \emph{p}. It's often easier to write this as
  simply \emph{U}(\(\phi \wedge\) \emph{p}), since the utility of
  \(\phi\) conditional on \emph{p} just is the utility of doing \(\phi\)
  in a world where \emph{p} is true. That is, it is the utility of
  \(\phi \wedge\) \emph{p} being realised. But we get a nicer symmetry
  between the probabilistic principles and the utility principles if we
  use the explicitly conditional notation for each.}

\begin{description}
\tightlist
\item[BAP]
For all relevant \emph{q}, \emph{x}, if \emph{p} is believed then
Pr(\emph{q})~=~\emph{x} iff Pr(\emph{q}~\textbar~\emph{p})~=~\emph{x}.
\item[BCP]
For all relevant \emph{q}, \emph{r}, if \emph{p} is believed then
Pr(\emph{q}) \(\geq\) Pr(\emph{r}) iff Pr(\emph{q}~\textbar~\emph{p})
\(\geq\) Pr(\emph{r}~\textbar~\emph{p}).
\item[BAU]
For all relevant \(\phi\), \emph{x}, if \emph{p} is believed then
\emph{U}(\(\phi\))~=~\emph{x} iff
\emph{U}(\(\phi\)~\textbar~p)~=~\emph{x}.
\item[BCU]
For all relevant \(\phi, \psi\), if \emph{p} is believed then
\emph{U}(\(\phi\)) \(\geq\) U(\(\psi\)) iff
\emph{U}(\(\phi\)~\textbar~p) \(\geq\)
\emph{U}(\(\psi\)~\textbar~\emph{p}).
\end{description}

In the earlier paper I focussed on \textbf{BAU} and \textbf{BCU}. But
\textbf{BAP} and \textbf{BCP} are important as well. Indeed, focussing
on them lets us derive a nice result.

Charlie is trying to figure out exactly what the probability of \emph{p}
is. That is, for any \emph{x} \(\in [0, 1]\), whether
Pr(\emph{p})~=~\emph{x} is a relevant question. Now Charlie is well
aware that Pr(\emph{p}~\textbar~\emph{p}) = 1. So unless Pr(\emph{p}) =
1, Charlie will give a different answer to the questions \emph{How
probable is p?} and \emph{Given p, how probable is p?}. So unless
Charlie holds that Pr(\emph{p}) is 1, she won't count as believing that
\emph{p}. One consequence of this is that Charlie can't reason, ``The
probability of \emph{p} is exactly 0.978, so \emph{p}.'' That's all to
the good, since that looks like bad reasoning. And it looks like bad
reasoning even though in some circumstances Charlie can rationally
believe propositions that she (rationally) gives credence 0.978 to.
Indeed, in some circumstances she can rationally believe something
\emph{in virtue} of it being 0.978 probable.

That's because the reasoning in the previous paragraph assumes that
every question of the form \emph{Is the probability of p equal to x?} is
relevant. In practice, fewer questions than that will be relevant. Let's
say that the only questions relevant to Charlie are of the form
\emph{What is the probability of p to one decimal place?}. And assume
that no other questions become relevant in the course of her inquiry
into this question.\footnote{This is probably somewhat unrealistic. It's
  hard to think about whether Pr(\emph{p}) is closer to 0.7 or 0.8
  without raising to salience questions about, for example, what the
  second decimal place in Pr(\emph{p}) is. This is worth bearing in mind
  when coming up with intuitions about the cases in this paragraph.}
Charlie decides that to the first decimal place, Pr(\emph{p}) = 1.0,
i.e., Pr(\emph{p})~\textgreater~0.95\$. That is compatible with simply
believing that \emph{p}. And that seems right; if for practical
purposes, the probability of \emph{p} is indistinguishable from 1, then
the agent is confident enough in \emph{p} to believe it.

So there are some nice features of this theory of belief. Indeed, there
are several reasons to believe it. It is, I have argued, the best
functionalist account of belief. I'm not going to argue for
functionalism about the mind, since the argument would take at least a
book. (The book in question might look a lot like Braddon-Mitchell and
Jackson (\citeproc{ref-DBMJackson2007}{2007}).) But I do think
functionalism is true, and so the best functionalist theory of belief is
the best theory of belief.

The argument for this theory of belief in my
(\citeproc{ref-Weatherson2005-WEACWD}{2005}) rested heavily on the flaws
of rival theories. We can see those flaws by looking at a tension that
any theory of the relationship between belief and credence must
overcome. Each of the following three principles seems to be plausible.

\begin{enumerate}
\def\labelenumi{\arabic{enumi}.}
\tightlist
\item
  If \emph{S} has a greater credence in \emph{p} than in \emph{q}, and
  she believes \emph{q}, then she believes \emph{p} as well; and if her
  credences in both \emph{p} and \emph{q} are rational, and her belief
  in \emph{q} is rational, then so is her belief in \emph{p}.
\item
  If \emph{S} rationally believes \emph{p} and rationally believes
  \emph{q}, then it is open to her to rationally believe
  \emph{p}~∧~\emph{q} without changing her credences.
\item
  \emph{S} can rationally believe \emph{p} while having credence of less
  than 1 in \emph{p}.
\end{enumerate}

But these three principles, together with some principles that are
genuinely uncontroversial, entail an absurd result. By 3, there is some
\emph{p} such that \emph{Cr}(\emph{p})~=~\emph{x}~\textless~1, and
\emph{p} is believed. (\emph{Cr} is the function from any proposition to
our agent's credence in that propositions.) Let \emph{S} know that a
particular fair lottery has \emph{l} tickets, where
\emph{l}~\textgreater~\(\frac{1}{1-x}\). The uncontroversial principle
we'll use is that in such a case \emph{S}'s credence that any given
ticket will lose should be \(\frac{l-1}{l}\). Since
\(\frac{l-1}{l}\)~\textgreater~\emph{x}, it follows by 1 that \emph{S}
believes of each ticket that it will lose. Since her credences are
rational, these beliefs are rational. By repeated applications of 2
then, the agent can rationally believe that each ticket will lose. But
she rationally gives credence 0 to the proposition that each ticket will
lose. So by 1 she can rationally believe any proposition in which her
credence is greater than 0. This is absurd.\footnote{See Sturgeon
  (\citeproc{ref-Sturgeon2008-STURAT}{2008}) for discussion of a similar
  puzzle for anyone trying to tell a unified story of belief and
  credence.}

I won't repeat all the gory details here, but one of the consequences of
the discussion in Weatherson
(\citeproc{ref-Weatherson2005-WEACWD}{2005}) was that we could hold on
to 3, and onto restricted versions of 1 and 2. In particular, if we
restricted 1 and 2 to relevant propositions (in some sense) they became
true, although the unrestricted version is false. A key part of the
argument of the earlier paper was that this was a better option than the
more commonly taken option of holding on to unrestricted versions of 1
and 3, at the cost of abandoning 2 even in clear cases. But one might
wonder why I'm holding so tightly on to 3. After all, there is a
functionalist argument that 3 is false.

A key functional role of credences is that if an agent has credence
\emph{x} in \emph{p} she should be prepared to buy a bet that returns 1
util if \emph{p}, and 0 utils otherwise, iff the price is no greater
than \emph{p} utils. A key functional role of belief is that if an agent
believes \emph{p}, and recognises that \(\phi\) is the best thing to do
given \emph{p}, then she'll do \(\phi\). Given \emph{p}, it's worth
paying any price up to 1 util for a bet that pays 1 util if \emph{p}. So
believing \emph{p} seems to mean being in a functional state that is
like having credence 1 in \emph{p}.

But this argument isn't quite right. If we spell out more carefully what
the functional roles of credence and belief are, a loophole emerges in
the argument that belief implies credence 1. The interest-relative
theory of belief turns out to exploit that loophole. What's the
difference, in functional terms, between having credence \emph{x} in
\emph{p}, and having credence \emph{x}~+~\(\varepsilon\) in \emph{p}?
Well, think again about the bet that pays 1 util if \emph{p}, and 0
utils otherwise. And imagine that bet is offered for
\emph{x}~+~\(\frac{\varepsilon}{2}\) utils. The person whose credence is
\emph{x} will decline the offer; the person whose credence is
\emph{x}~+~\(\varepsilon\) will accept it. Now it will usually be that
no such bet is on offer.\footnote{There are exceptions, especially in
  cases where \emph{p} concerns something significant to financial
  markets, and the agent trades financial products. If you work through
  the theory that I'm about to lay out, one consequence is that such
  agents should have very few unconditional beliefs about
  financially-sensitive information, just higher and lower credences. I
  think that's actually quite a nice outcome, but I'm not going to rely
  on that in the argument for the view.} No matter; as long as one agent
is \emph{disposed} to accept the offer, and the other agent is not, that
suffices for a difference in credence.

The upshot of that is that differences in credences might be, indeed
usually will be, constituted by differences in dispositions concerning
how to act in choice situations far removed from actuality. I'm not
usually in a position of having to accept or decline a chance to buy a
bet for 0.9932 utils that the local coffee shop is currently open. Yet
whether I would accept or decline such a bet matters to whether my
credence that the coffee shop is open is 0.9931 or 0.9933. This isn't a
problem with the standard picture of how credences work. It's just an
observation that the high level of detail embedded in the picture relies
on taking the constituents of mental states to involve many
dispositions.

One of the crucial features of the theory of belief I'm defending is
that what an agent believes is in general \emph{insensitive} to such
abtruse dispositions, although it is very sensitive to dispositions
about practical matters. It's true that if I believe that \emph{p}, and
I'm rational enough, I'll act as if \emph{p} is true. Is it also true
that if I believe \emph{p}, I'm disposed to act as if \emph{p} is true
no matter what choices are placed in front of me? The theory being
defended here says no, and that seems plausible. As we say in the case
of Barry and Beth, Barry can believe that \emph{p}, but be disposed to
\emph{lose that belief} rather than act on it if odd choices, like that
presented by the genie, emerge.

This suggests the key difference between belief and credence 1. For a
rational agent, a credence of 1 in \emph{p} means that the agent is
disposed to answer a wide range of questions the same way she would
answer that question conditional on \emph{p}. That follows from the fact
that these four principles are trivial theorems of the orthodox theory
of expected utility.\footnote{The presentation in this section, as in
  the earlier paper, assumes at least a weak form of consequentialism in
  the sense of Hammond (\citeproc{ref-Hammond1988}{1988}). This was
  arguably a weakness of the earlier paper. We'll return to the issue of
  what happens in cases where the agent doesn't, and perhaps shouldn't,
  maximise expected utility, at the end of the section.}

\begin{description}
\tightlist
\item[C1AP]
For all \emph{q}, \emph{x} if Pr(\emph{p})~=~1, then
Pr(\emph{q})~=~\emph{x}~iff Pr(\emph{q}~\textbar~\emph{p})~=~\emph{x}.
\item[C1CP]
For all \emph{q}, \emph{r} if Pr(\emph{p})~=~1 then Pr(\emph{q})
\(\geq\) Pr(\emph{r}) iff Pr(\emph{q}~\textbar~\emph{p}) \(\geq\)
Pr(\emph{r}~\textbar~\emph{p}).
\item[C1AU]
For all \(\phi\), \emph{x}, if Pr(\emph{p}) = 1 then
\emph{U}(\(\phi\))~=~\emph{x} iff
\emph{U}(\(\phi\)~\textbar~p)~=~\emph{x}.
\item[C1CP]
For all \(\phi, \psi\), if Pr(\emph{p}) = 1 then \emph{U}(\(\phi\))
\(\geq\) U(\(\psi\)) iff \emph{U}(\(\phi\)~\textbar~p) \(\geq\)
U(\(\psi\)~\textbar~p).
\end{description}

Those look a lot like the theorems of the theory of belief that we
discussed above. But note that these claims are \emph{unrestricted},
whereas in the theory of belief, we restricted attention to relevant
actions, propositions, utilities and probabilities. That turns out to be
the difference between belief and credence 1. Since that difference is
interest-relative, belief is interest-relative.

I used to think that that was all the interest-relativity we needed in
epistemology. Now I don't, for reasons that I'll go through in section
three. (Readers who care more about the theory of knowledge than the
theory of belief may want to skip ahead to that section.) But first I
want to clean up some loose ends in the acount of belief.

\section{Two Caveats}\label{two-caveats}

The theory sketched so far seems to me right in the vast majority of
cases. It fits in well with a broadly functionalist view of the mind,
and it handles difficult cases, like that of Charlie, nicely. But it
needs to be supplemented and clarified a little to handle some other
difficult cases. In this section I'm going to supplement the theory a
little to handle what I call `impractical propositions', and say a
little about morally loaded action.

Jones has a false geographic belief: he believes that Los Angeles is
west of Reno.\footnote{I'm borrowing this example from Fred Dretske, who
  uses it to make some interesting points about dispositional belief.}
This isn't because he's ever thought about the question. Rather, he's
just disposed to say ``Of course'' if someone asks, ``Is Los Angeles
west of Reno?'' That disposition has never been triggered, because no
one's ever bothered to ask him this. Call the proposition that Los
Angeles is west of Reno \emph{p}.

The theory given so far will get the right result here: Jones does
believe that \emph{p}. But it gets the right answer for an odd reason.
Jones, it turns out, has very little interest in American geography
right now. He's a schoolboy in St Andrews, Scotland, getting ready for
school and worried about missing his schoolbus. There's no inquiry he's
currently engaged in for which \emph{p} is even close to relevant. So
conditionalising on \emph{p} doesn't change the answer to any inquiry
he's engaged in, but that would be true no matter what his credence in
\emph{p} is.

There's an immediate problem here. Jones believes \emph{p}, since
conditionalising on \emph{p} doesn't change the answer to any relevant
inquiry. But for the very same reason, conditionalising on
\(\neg\)\emph{p} doesn't change the answer to any relevant inquiry. It
seems our theory has the bizarre result that Jones believes
\(\neg\)\emph{p} as well. That is both wrong and unfair. We end up
attributing inconsistent beliefs to Jones simply because he's a harried
schoolboy who isn't currently concerned with the finer points of
geography of the American southwest.

Here's a way out of this problem in four relatively easy
steps.\footnote{The recipe here is similar to that given in Weatherson
  (\citeproc{ref-Weatherson2005-WEACWD}{2005}), but the motivation is
  streamlined. Thanks to Jacob Ross for helpful suggestions here.}
First, we say that which questions are relevant questions is not just
relative to the agent's interests, but also relevant to the proposition
being considered. A question may be relevant relative to \emph{p}, but
not relative to \emph{q}. Second, we say that relative to \emph{p}, the
question of whether \emph{p} is more probable than \(\neg\)\emph{p} is a
relevant question. Third, we infer from that that an agent only believes
\emph{p} if their credence in \emph{p} is greater than their credence in
\(\neg\)\emph{p}, i.e., if their credence in \emph{p} is greater than
\(\frac{1}{2}\). Finally, we say that when the issue is whether the
subject believes that \emph{p}, the question of whether \emph{p} is more
probable than \(\neg\)\emph{p} is not only relevant on its own, but it
stays being a relevant question conditional on any \emph{q} that is
relevant to the subject. In the earlier paper
(\citeproc{ref-Weatherson2005-WEACWD}{Weatherson 2005}) I argue that
this solves the problem raised by impractical propositions in a smooth
and principled way.

That's the first caveat. The second is one that isn't discussed in the
earlier paper. If the agent is merely trying to get the best outcome for
themselves, then it makes sense to represent them as a utility
maximiser. And within orthodox decision theory, it is easy enough to
talk about, and reason about, conditional utilities. That's important,
because conditional utilities play an important role in the theory of
belief offered at the start of this section. But if the agent faces
moral constraints on her decision, it isn't always so easy to think
about conditional utilities.

When agents have to make decisions that might involve them causing harm
to others if certain propositions turn out to be true, then I think it
is best to supplement orthodox decision theory with an extra assumption.
The assumption is, roughly, that for choices that may harm others,
expected value is absolute value. It's easiest to see what this means
using a simple case of three-way choice. The kind of example I'm
considering here has been used for (slightly) different purposes by
Frank Jackson (\citeproc{ref-Jackson1991}{1991}).

The agent has to do \(\varphi\) or \(\psi\). Failure to do either of
these will lead to disaster, and is clearly unacceptable. Either
\(\varphi\) or \(\psi\) will avert the disaster, but one of them will be
moderately harmful and the other one will not. The agent has time before
the disaster to find out which of \(\varphi\) and \(\psi\) is harmful
and which is not for a nominal cost. Right now, her credence that
\(\varphi\) is the harmful one is, quite reasonably, \(\frac{1}{2}\). So
the agent has three choices:

\begin{itemize}
\tightlist
\item
  Do \(\varphi\);
\item
  Do \(\psi\); or
\item
  Wait and find out which one is not harmful, and do it.
\end{itemize}

We'll assume that other choices, like letting the disaster happen, or
finding out which one is harmful and doing it, are simply out of
consideration. In any case, they are clearly dominated options, so the
agent shouldn't do them. Let \emph{p} be the propostion that \(\varphi\)
is the harmful one. Then if we assume the harm in question has a
disutility of 10, and the disutility of waiting to act until we know
which is the harmful one is 1, the values of the possible outcomes are
as follows:

\begin{longtable}[]{@{}lcc@{}}
\toprule\noalign{}
~ & ~\emph{p}~ & ~\(\neg\)\emph{p} \\
\midrule\noalign{}
\endhead
\bottomrule\noalign{}
\endlastfoot
\textbf{Do} \(\varphi\) ~ & ~-10~ & ~ 0 \\
\textbf{Do} \(\psi\) ~ & ~ 0 ~ & ~ -10 \\
Find out which is harmful~ & ~-1 ~ & ~ -1 \\
\end{longtable}

Given that Pr(\emph{p}) = \(\frac{1}{2}\), it's easy to compute that the
expected value of doing either \(\varphi\) or \(\psi\) is -5, while the
expected value of finding out which is harmful is -1, so the agent
should find out which thing is to be done before acting. So far most
consequentialists would agree, and so probably would most
non-consequentialists for most ways of fleshing out the abstract example
I've described.\footnote{Some consequentialists say that what the agent
  should do depends on whether \emph{p} is true. If \emph{p} is true,
  she should do \(\psi\), and if \emph{p} is false she should do
  \(\varphi\). As we'll see, I have reasons for thinking this is rather
  radically wrong.}

But most consequentialists would also say something else about the
example that I think is not exactly true. Just focus on the column in
the table above where \emph{p} is true. In that column, the highest
value, 0, is alongside the action \emph{Do} \(\psi\). So you might think
that conditional on \emph{p}, the agent should do \(\psi\). That is, you
might think the conditional expected value of doing \(\psi\),
conditional on \emph{p} being true, is 0, and that's higher than the
conditional expected value of any other act, conditional on \emph{p}. If
you thought that, you'd certainly be in agreement with the orthodox
decision-theoretic treatment of this problem.

In the abstract statement of the situation above, I said that one of the
options would be \emph{harmful}, but I didn't say who it would be
harmful to. I think this matters. I think what I called the orthodox
treatment of the situation is correct when the harm accrues to the
person making the decision. But when the harm accrues to another person,
particularly when it accrues to a person that the agent has a duty of
care towards, then I think the orthodox treatment isn't quite right.

My reasons for this go back to Jackson's original discussion of the
puzzle. Let the agent be a doctor, the actions \(\varphi\) and \(\psi\)
be her prescribing different medication to a patient, and the harm a
severe allergic reaction that the patient will have to one of the
medications. Assume that she can run a test that will tell her which
medication the patient is allergic to, but the test will take a day.
Assume that the patient will die in a month without either medication;
that's the disaster that must be averted. And assume that the patient is
is some discomfort that either medication would relieve; that's the
small cost of finding out which medication is risk. Assume finally that
there is no chance the patient will die in the day it takes to run the
test, so the cost of running the test is really nominal.

A good doctor in that situation will find out which medication the
patient is allergic to before ascribing either medicine. It would be
\emph{reckless} to ascribe a medicine that is unnecessary and that the
patient might be allergic to. It is worse than reckless if the patient
is actually allergic to the medicine prescribed, and the doctor harms
the patient. But even if she's lucky and prescribes the `right'
medication, the recklessness remains. It was still, it seems, the wrong
thing for her to do.

All of that is in Jackson's discussion of the case, though I'm not sure
he'd agree with the way I'm about the incorporate these ideas into the
formal decision theory. Even under the assumption that \emph{p},
prescribing \(\psi\) is still wrong, because it is reckless. That should
be incorporated into the values we ascribe to different actions in
different circumstances. The way I do it is to associate the value of
each action, in each circumstance, with its actual expected value. So
the decision table for the doctor's decision looks something like this.

\begin{longtable}[]{@{}lcc@{}}
\toprule\noalign{}
~ & ~\emph{p}~ & ~\(\neg\)\emph{p} \\
\midrule\noalign{}
\endhead
\bottomrule\noalign{}
\endlastfoot
\textbf{Do} \(\varphi\) ~ & ~-5 ~ & ~ -5 \\
\textbf{Do} \(\psi\) ~ & ~-5 ~ & ~ -5 \\
Find out which is harmful~ & ~-1 ~ & ~ -1 \\
\end{longtable}

In fact, the doctor is making a decision under certainty. She knows that
the value of prescribing either medicine is -5, and the value of running
the tests is -1, so she should run the tests.

In general, when an agent has a duty to maximise the expected value of
some quantity \emph{q}, then the value that goes into the agent's
decision table in a cell is \emph{not} the value of \emph{q} in the
world-action pair the agent represents. Rather, it's the expected value
of \emph{q} given that world-action pair. In situations like this one
where the relevant facts (e.g., which medicine the patient is allergic
to) don't affect the evidence the agent has, the decision is a decision
under \emph{certainty}. This is all as things should be. When you have
obligations that are drawn in terms of the expected value of a variable,
the actual values of that variable cease to be directly relevant to the
decision problem.

Similar morals carry across to theories that offer a smaller role to
expected utility in determining moral value. In particular, it's often
true that decisions where it is uncertain what course of action will
produce the best outcome might still, in the morally salient sense, be
decisions under certainty. That's because the uncertainty doesn't impact
how we should weight the different possible outcomes, as in orthodox
utility theory, but how we should value them. That's roughly what I
think is going on in cases like this one, which Jessica Brown has argued
are problematic for the epistemological theories John Hawthorne and
Jason Stanley have recently been defending.\footnote{The target here is
  not directly the interest-relativity of their theories, but more
  general principles about the role of knowledge in action and
  assertion. Since my theories are close enough, at least in
  consequences, to Hawthorne and Stanley's, it is important to note how
  my theory handles the case.}

\begin{quote}
A student is spending the day shadowing a surgeon. In the morning he
observes her in clinic examining patient A who has a diseased left
kidney. The decision is taken to remove it that afternoon. Later, the
student observes the surgeon in theatre where patient A is lying
anaesthetised on the operating table. The operation hasn't started as
the surgeon is consulting the patient's notes. The student is puzzled
and asks one of the nurses what's going on:

\textbf{Student}: I don't understand. Why is she looking at the
patient's records? She was in clinic with the patient this morning.
Doesn't she even know which kidney it is?

\textbf{Nurse}: Of course, she knows which kidney it is. But, imagine
what it would be like if she removed the wrong kidney. She shouldn't
operate before checking the patient's records.
(\citeproc{ref-Brown2008-BROKAP}{Brown 2008, 1144--45})
\end{quote}

It is tempting, but for reasons I've been going through here mistaken,
to represent the surgeon's choice as follows. Let \textbf{Left} mean the
left kidney is diseased, and \textbf{Right} mean the right kidney is
diseased.

\begin{longtable}[]{@{}lcc@{}}
\toprule\noalign{}
~ & ~ \textbf{Left} ~ & ~ \textbf{Right} \\
\midrule\noalign{}
\endhead
\bottomrule\noalign{}
\endlastfoot
\textbf{Remove left kidney} ~ & ~ 1 ~ & ~ -1 \\
\textbf{Remove right kidney}~ & ~ -1 ~ & ~ 1 \\
\textbf{Check notes} ~ & ~1-\(\varepsilon\)~ & ~1-\(\varepsilon\) \\
\end{longtable}

Here \(\varepsilon\) is the trivial but non-zero cost of checking the
chart. Given this table, we might reason that since the surgeon knows
that she's in the left column, and removing the left kidney is the best
option in that column, she should remove the left kidney rather than
checking the notes.

But that reasoning assumes that the surgeon does not have any epistemic
obligations over and above her duty to maximise expected utility. And
that's very implausible. It's totally implausible on a
non-consequentialist moral theory. A non-consequentialist may think that
some people have just the same obligations that the consequentialist
says they have -- legislators are frequently mentioned as an example --
but surely they wouldn't think \emph{surgeons} are in this category. And
even a consequentialist who thinks that surgeons have special
obligations in terms of their institutional role should think that the
surgeon's obligations go above and beyond the obligation every agent has
to maximise expected utility.

It's not clear exactly what the obligation the surgeon has. Perhaps it
is an obligation to not just know which kidney to remove, but to know
this on the basis of evidence she has obtained while in the operating
theatre. Or perhaps it is an obligation to make her belief about which
kidney to remove as sensitive as possible to various possible scenarios.
Before she checked the chart, this counterfactual was false: \emph{Had
she misremembered which kidney was to be removed, she would have a true
belief about which kidney was to be removed.} Checking the chart makes
that counterfactual true, and so makes her belief that the left kidney
is to be removed a little more sensitive to counterfactual
possibilities.

However we spell out the obligation, it is plausible given what the
nurse says that the surgeon has some such obligation. And it is
plausible that the `cost' of violating this obligation, call it
\(\delta\) is greater than the cost of checking the notes. So here is
the decision table the surgeon faces.

\begin{longtable}[]{@{}lcc@{}}
\toprule\noalign{}
~ & ~ \textbf{Left} ~ & ~ \textbf{Right} \\
\midrule\noalign{}
\endhead
\bottomrule\noalign{}
\endlastfoot
\textbf{Remove left kidney} ~ & ~ 1-\(\delta\) ~ & ~ -1-\(\delta\) \\
\textbf{Remove right kidney}~ & ~ -1-\(\delta\) ~ & ~ 1-\(\delta\) \\
\textbf{Check notes} ~ & ~1-\(\varepsilon\)~ & ~1-\(\varepsilon\) \\
\end{longtable}

And it isn't surprising, or a problem for an interest-relative theory of
knowledge or belief, that the surgeon should check the notes, even if
she believes \emph{and knows} that the left kidney is the diseased one.

\section{Interest-Relative Defeaters}\label{sec-3}

As I said at the top, I've changed my view from Doxastic IRI to
Non-Doxastic IRI. The change of heart is occasioned by cases like the
following, where the agent is mistaken, and hence ignorant, about the
odds at which she is offered a bet on \emph{p}. In fact the odds are
much longer than she thinks. Relative to what she stands to win, the
stakes are too high.

\subsection{The Coraline Example}\label{the-coraline-example}

The problem for Doxastic IRI arises because of cases like that of
Coraline. Here's what we're going to stipulate about Coraline.

\begin{itemize}
\tightlist
\item
  She knows that \emph{p} and \emph{q} are independent, so her credence
  in any conjunction where one conjunct is a member of \{\emph{p},
  \(\neg\)\emph{p}\} and the other is a member of \{\emph{q},
  \(\neg\)\emph{q}\} will be the product of her credences in the
  conjuncts.
\item
  Her credence in \emph{p} is 0.99, just as the evidence supports.
\item
  Her credence in \emph{q} is also 0.99. This is unfortunate, since the
  rational credence in \emph{q} given her evidence is 0.01.
\item
  The only relevant question for her which is sensitive to \emph{p} is
  whether to take or decline a bet with the following payoff
  structure.\footnote{I'm more interested in the abstract structure of
    the case than in whether any real-life situation is modelled by just
    this structure. But it might be worth noting the rough kind of
    situation where this kind of situation can arise. So let's say
    Coraline has a particular bank account that is uninsured, but which
    currently paying 10\% interest, and she is deciding whether to
    deposit another \$1000 in it. Then \emph{p} is the proposition that
    the bank will not collapse, and she'll get her money back, and
    \emph{q} is the proposition that the interest will stay at 10\%. To
    make the model exact, we have to also assume that if the interest
    rate on her account doesn't stay at 10\%, it falls to 0.1\%. And we
    have to assume that the interest rate and the bank's collapse are
    probabilistically independent. Neither of these are at all
    realistic, but a realistic case would simply be more complicated,
    and the complications would obscure the philosophically interesting
    point.} (Assume that the marginal utility of money is close enough
  to constant that expected dollar returns correlate more or less
  precisely with expected utility returns.)
\end{itemize}

\begin{longtable}[]{@{}lccc@{}}
\toprule\noalign{}
~ & ~\emph{p} ∧ \emph{q}~ & ~\emph{p} \(\wedge \neg\)\emph{q}~ &
~\(\neg\)\emph{p} \\
\midrule\noalign{}
\endhead
\bottomrule\noalign{}
\endlastfoot
\textbf{Take bet} ~ & ~ 100 ~ & ~ 1 ~ & ~ 1000 \\
\textbf{Decline bet}~ & ~ 0 ~ & ~ 0 ~ & ~ 0 \\
\end{longtable}

As can be easily computed, the expected utility of taking the bet given
her credences is positive, it is just over \$89. And Coraline takes the
bet. She doesn't compute the expected utility, but she is sensitive to
it.\footnote{If she did compute the expected utility, then one of the
  things that would be salient for her is the expected utility of the
  bet. And the expected utility of the bet is different to its expected
  utility given \emph{p}. So if that expected utility is salient, she
  doesn't believe \emph{p}. And it's going to be important to what
  follows that she \emph{does} believe \emph{p}.} That is, had the
expected utility given her credences been close to 0, she would have not
acted until she made a computation. But from her perspective this looks
like basically a free \$100, so she takes it. Happily, this all turns
out well enough, since \emph{p} is true. But it was a dumb thing to do.
The expected utility of taking the bet given her evidence is negative,
it is a little under -\$8. So she isn't warranted, given her evidence,
in taking the bet.

\subsection{What Coraline Knows and What She
Believes}\label{what-coraline-knows-and-what-she-believes}

Assume, for \emph{reductio}, that Coraline knows that \emph{p}. Then the
choice she faces looks like this.

\begin{longtable}[]{@{}lcc@{}}
\toprule\noalign{}
~ & ~\emph{q}~ & ~\(\neg\)\emph{q} \\
\midrule\noalign{}
\endhead
\bottomrule\noalign{}
\endlastfoot
\textbf{Take bet} ~ & ~100~ & ~ 1 \\
\textbf{Decline bet}~ & ~ 0 ~ & ~ 0 \\
\end{longtable}

Since taking the bet dominates declining the bet, she should take the
bet if this is the correct representation of her situation. She
shouldn't take the bet, so by \emph{modus tollens}, that can't be the
correct representation of her situation. If she knew \emph{p}, that
would be the correct representation of her situation. So, again by
\emph{modus tollens}, she doesn't know \emph{p}.

Now let's consider three possible explanations of why she doesn't know
that \emph{p}.

\begin{enumerate}
\def\labelenumi{\arabic{enumi}.}
\tightlist
\item
  She doesn't have enough evidence to know that \emph{p}, independent of
  the practical stakes.
\item
  In virtue of the practical stakes, she doesn't believe that \emph{p};
\item
  In virtue of the practical stakes, she doesn't justifiably believe
  that \emph{p}, although she does actually believe it.
\item
  In virtue of the practical stakes, she doesn't know that \emph{p},
  although she does justifiably believe it.
\end{enumerate}

I think option 1 is implausibly sceptical, at least if applied to all
cases like Coraline's. I've said that the probability of \emph{p} is
0.99, but it should be clear that all that matters to generating a case
like this is that \emph{p} is not completely certain. Unless knowledge
requires certainty, we'll be able to generate Coraline-like cases where
there is sufficient evidence for knowledge. So that's ruled out.

Option 2 is basically what the Doxastic IRI theorist has to say. If
Coraline has enough evidence to know \emph{p}, but doesn't know \emph{p}
due to practical stakes, then the Doxastic IRI theorist is committed to
saying that the practical stakes block \emph{belief} in \emph{p}. That's
the Doxastic IRI position; stakes matter to knowledge because they
matter to belief.

But that's also an implausible description of Coraline's situation. She
is very confident that \emph{p}. Her confidence is grounded in the
evidence in the right way. She is insensitive in her actual
deliberations to the difference between her evidence for \emph{p} and
evidence that guarantees \emph{p}. She would become sensitive to that
difference if someone offered her a bet that she knew was a 1000-to-1
bet on \emph{p}, but she doesn't know that's what is on offer. In short,
there is no difference between her unconditional attitudes, and her
attitudes conditional on \emph{p}, when it comes to any live question.
That's enough, I think, for belief. So she believes that \emph{p}. And
that's bad news for the Doxastic IRI theorist; since it means here that
stakes matter to knowledge without mattering to belief. I conclude,
reluctantly, that Doxastic IRI is false.

\subsection{Stakes as Defeaters}\label{stakes-as-defeaters}

That still leaves two options remaining, what I've called options 3 and
4 above. Option 3, if suitably generalised, says that knowledge is
practically sensitive because the justification condition on belief is
practically sensitive. Option 4 says that practical considerations
impact knowledge directly. As I read them, Jeremy Fantl and Matthew
McGrath defend a version of Option 3. In the next and last subsection,
I'll argue against that position. But first I want to sketch what a
position like option 4 would look like.

Knowledge, unlike justification, requires a certain amount of internal
coherence among mental states. Consider the following story from David
Lewis:

\begin{quote}
I speak from experience as the repository of a mildly inconsistent
corpus. I used to think that Nassau Street ran roughly east-west; that
the railroad nearby ran roughly north-south; and that the two were
roughly parallel. (\citeproc{ref-Lewis1982c}{Lewis 1982, 436})
\end{quote}

I think in that case that Lewis doesn't know that Nassau Street runs
roughly east-west. (From here on, call the proposition that Nassau
Street runs roughly east-west \emph{N}.) If his belief that it does was
acquired and sustained in a suitably reliable way, then he may well have
a justified belief that \emph{N}. But the lack of coherence with the
rest of his cognitive system, I think, defeats any claim to knowledge he
has.

Coherence isn't just a requirement on belief; other states can cohere or
be incoherent. Assume Lewis corrects the incoherence in his beliefs, and
drops the belief that Nassau Street the railway are roughly parallel.
Still, if Lewis believed that \emph{N}, preferred doing \(\varphi\) to
doing \(\psi\) conditional on \emph{N}, but actually preferred doing
\(\psi\) to doing \(\varphi\), his cognitive system would also be in
tension. That tension could, I think, be sufficient to defeat a claim to
know that \emph{N}.

And it isn't just a requirement on actual states; it can be a
requirement on rational states. Assume Lewis believed that \emph{N},
preferred doing \(\varphi\) to doing \(\psi\) conditional on \emph{N},
and preferred doing \(\varphi\) to doing \(\psi\), but should have
preferred doing \(\psi\) to doing \(\varphi\) given his interests. Then
I think the fact that the last preference is irrational, plus the fact
that were it corrected there would be incoherence in his cognitive
states defeats the claim to know that \emph{N}.

A concrete example of this helps make clear why such a view is
attractive, and why it faces difficulties. Assume there is a bet that
wins \$2 if \emph{N}, and loses \$10 if not. Let \(\varphi\) be taking
that bet, and \(\psi\) be declining it. Assume Lewis shouldn't take that
bet; he doesnt have enough evidence to do so. Then he clearly doesn't
know that \emph{N}. If he knew that \emph{N}, \(\varphi\) would dominate
\(\psi\), and hence be rational. But it isn't, so \emph{N} isn't known.
And that's true whether Lewis's preferences between \(\varphi\) and
\(\psi\) are rational or irrational.

Attentive readers will see where this is going. Change the bet so it
wins a penny if \emph{N}, and loses \$1,000 if not. Unless Lewis's
evidence that \emph{N} is incredibly strong, he shouldn't take the bet.
So, by the same reasoning, he doesn't know that \emph{N}. And we're back
saying that knowledge requires incredibly strong evidence. The solution,
I say, is to put a pragmatic restriction on the kinds of incoherence
that matter to knowledge. Incoherence with respect to irrelevant
questions, such as whether to bet on \emph{N} at extremely long odds,
doesn't matter for knowledge. Incoherence (or coherence obtained only
through irrationality) does. The reason, I think, that Non-Doxastic IRI
is true is that this coherence-based defeater is sensitive to practical
interests.

The string of cases about Lewis and \emph{N} has ended up close to the
Coraline example. We already concluded that Coraline didn't know
\emph{p}. Now we have a story about why - belief that \emph{p} doesn't
cohere sufficiently well with what she should believe, namely that it
would be wrong to take the bet. If all that is correct, just one
question remains: does this coherence-based defeater also defeat
Coraline's claim to have a justified belief that \emph{p}? I say it does
not, for three reasons.

First, her attitude towards \emph{p} tracks the evidence perfectly. She
is making no mistakes with respect to \emph{p}. She is making a mistake
with respect to \emph{q}, but not with respect to \emph{p}. So her
attitude towards \emph{p}, i.e.~belief, is justified.

Second, talking about beliefs and talking about credences are simply two
ways of modelling the very same things, namely minds. If the agent both
has a credence 0.99 in \emph{p}, and believes that \emph{p}, these are
not two different states. Rather, there is one state of the agent, and
two different ways of modelling it. So it is implausible to apply
different valuations to the state depending on which modelling tools we
choose to use. That is, it's implausible to say that while we're
modelling the agent with credences, the state is justified, but when we
change tools, and start using beliefs, the state is unjustified. Given
this outlook on beliefs and credences, it is natural to say that her
belief is justified. Natural, but not compulsory, for reasons Jeremy
Fantl pointed out to me.\footnote{The following isn't Fantl's example,
  but I think it makes much the same point as the examples he suggested.}
We don't want a metaphysics on which persons and philosophers are
separate entities. Yet we can say that someone is a good person but a
bad philosopher. Normative statuses can differ depending on which
property of a thing we are considering. That suggests it is at least
coherent to say that one and the same state is a good credence but a bad
belief. But while this may be coherent, I don't think it is well
motivated, and it is natural to have the evaluations go together.

Third, we don't \emph{need} to say that Coraline's belief in \emph{p} is
unjustified in order to preserve other nice theories, in the way that we
do need to say that she doesn't know \emph{p} in order to preserve a
nice account of how we understand decision tables. It's this last point
that I think Fantl and McGrath, who say that the belief is unjustified,
would reject. So let's conclude with a look at their arguments.

\subsection{Fantl and McGrath on
Interest-Relativity}\label{fantl-and-mcgrath-on-interest-relativity}

Fantl and McGrath's argue for the principle (JJ), which entails that
Coraline is not justified in believing \emph{p}.

\begin{description}
\tightlist
\item[(JJ)]
If you are justified in believing that \emph{p}, then \emph{p} is
warranted enough to justify you in \(\varphi\)-ing, for any \(\varphi\).
(\citeproc{ref-FantlMcGrath2009}{Fantl and McGrath 2009, 99})
\end{description}

In practice, what this means is that there can't be a salient \emph{p},
\(\varphi\) such that:

\begin{itemize}
\tightlist
\item
  The agent is justified in believing \emph{p};
\item
  The agent is not warranted in doing \(\varphi\); but
\item
  If the agent had more evidence for \emph{p}, and nothing else, the
  agent would be be warranted in doing \(\varphi\).
\end{itemize}

That is, once you've got enough evidence, or warrant, for justified
belief in \emph{p}, then you've got enough evidence for \emph{p} as
matters for any decision you face. This seems intuitive, and Fantl and
McGrath back up its intuitiveness with some nicely drawn examples. But I
think it is false, and the Coraline example shows it is false. Coraline
isn't justified in taking the bet, and is justified in believing
\emph{p}, but more evidence for \emph{p} would suffice for taking the
bet. So Coraline's case shows that (JJ) is false. But there are a number
of possible objections to that position. I'll spend the rest of this
section, and this paper, going over them.\footnote{Thanks here to a long
  blog comments thread with Jeremy Fantl and Matthew McGrath for making
  me formulate these points much more carefully. The original thread is
  at
  \url{http://tar.weatherson.org/2010/03/31/do-justified-beliefs-justify-action/}.}

\emph{Objection}: The following argument shows that Coraline is not in
fact justified in believing that \emph{p}.

\begin{enumerate}
\def\labelenumi{\arabic{enumi}.}
\tightlist
\item
  \emph{p} entails that Coraline should take the bet, and Coraline knows
  this.
\item
  If \emph{p} entails something, and Coraline knows this, and she
  justifiably believes \emph{p}, she is in a position to justifiably
  believe the thing entailed.
\item
  Coraline is not in a position to justifiably believe that she should
  take the bet.
\item
  So, Coraline does not justifiably believe that \emph{p}
\end{enumerate}

\emph{Reply}: The problem here is that premise 1 is false. What's true
is that \emph{p} entails that Coraline will be better off taking the bet
than declining it. But it doesn't follow that she should take the bet.
Indeed, it isn't actually true that she should take the bet, even though
\emph{p} is actually true. Not just is the entailment claim false, the
world of the example is a counterinstance to it.

It might be controversial to use this very case to reject premise 1. But
the falsity of premise 1 should be clear on independent grounds. What
\emph{p} entails is that Coraline will be best off by taking the bet.
But there are lots of things that will make me better off that I
shouldn't do. Imagine I'm standing by a roulette wheel, and the thing
that will make me best off is betting heavily on the number than will
actually come up. It doesn't follow that I should do that. Indeed, I
should not do it. I shouldn't place any bets at all, since all the bets
have a highly negative expected return.

In short, all \emph{p} entails is that taking the bet will have the best
consequences. Only a very crude kind of consequentialism would identify
what I should do with what will have the best returns, and that crude
consequentialism isn't true. So \emph{p} doesn't entail that Coraline
should take the bet. So premise 1 is false.

\emph{Objection}: Even though \emph{p} doesn't \emph{entail} that
Coraline should take the bet, it does provide inductive support for her
taking the bet. So if she could justifiably believe \emph{p}, she could
justifiably (but non-deductively) infer that she should take the bet.
Since she can't justifiably infer that, she isn't justified in taking
the bet.

\emph{Reply}: The inductive inference here looks weak. One way to make
the inductive inference work would be to deduce from \emph{p} that
taking the bet will have the best outcomes, and infer from that that the
bet should be taken. But the last step doesn't even look like a reliable
ampliative inference. The usual situation is that the best outcome comes
from taking an \emph{ex ante} unjustifiable risk.

It may seem better to use \emph{p} combined with the fact that
conditional on \emph{p}, taking the bet has the highest \emph{expected}
utility. But actually that's still not much of a reason to take the bet.
Think again about cases, completely normal cases, where the action with
the best outcome is an \emph{ex ante} unjustifiable risk. Call that
action \(\varphi\), and let \emph{B}\(\varphi\) be the proposition that
\(\varphi\) has the best outcome. Then \emph{B}\(\varphi\) is true, and
conditional on \emph{B}\(\varphi\), \(\varphi\) has an excellent
expected return. But doing \(\varphi\) is still running a dumb risk.
Since these kinds of cases are normal, it seems it will very often be
the case that this form of inference leads from truth to falsity. So
it's not a reliable inductive inference.

\emph{Objection}: In the example, Coraline isn't just in a position to
justifiably believe \emph{p}, she is in a position to \emph{know} that
she justifiably believes it. And from the fact that she justifiably
believes \emph{p}, and the fact that if \emph{p}, then taking the bet
has the best option, she can infer that she should take the bet.

\emph{Reply}: It's possible at this point that we get to a dialectical
impasse. I think this inference is non-deductive, because I think the
example we're discussing here is one where the premises are true and the
conclusion false. Presumably someone who doesn't like the example will
think that it is a good deductive inference.

Having said that, the more complicated example at the end of Weatherson
(\citeproc{ref-Weatherson2005-WEACWD}{2005}) was designed to raise the
same problem without the consequence that if \emph{p} is true, the bet
is sure to return a positive amount. In that example, conditionalising
on \emph{p} means the bet has a positive expected return, but still
possibly a negative return. But in that case (JJ) still failed. If
accepting there are cases where an agent justifiably believes \emph{p},
and hence justifiably believes taking the bet will return the best
outcome, and knows all this, but still can't rationally bet on \emph{p}
is too much to accept, that more complicated example might be more
persuasive. Otherwise, I concede that someone who believes (JJ) and
thinks rational agents can use it in their reasoning will not think that
a particular case is a counterexample to (JJ).

\emph{Objection}:If Coraline were ideal, then she wouldn't believe
\emph{p}. That's because if she were ideal, she would have a lower
credence in \emph{q}, and if that were the case, her credence in
\emph{p} would have to be much higher (close to 0.999) in order to count
as a belief. So her belief is not justified.

\emph{Reply}: The premise here, that if Coraline were ideal she would
not believe that \emph{p}, is true. The conclusion, that she is not
justified in believing \emph{p}, does not follow. It's always a mistake
to \emph{identify} what should be done with what is done in ideal
circumstances. This is something that has long been known in economics.
The \emph{locus classicus} of the view that this is a mistake is Lipsey
and Lancaster (\citeproc{ref-LipseyLancaster}{1956}). A similar point
has been made in ethics in papers such as Watson
(\citeproc{ref-Watson1977}{1977}) and
(\citeproc{ref-KennettSmith1996b}{Kennett and Smith 1996a},
\citeproc{ref-KennettSmith1996a}{1996b}). And it has been extended to
epistemology by Williamson (\citeproc{ref-Williamson1998-WILCOK}{1998}).

All of these discussions have a common structure. It is first observed
that the ideal is both \emph{F} and \emph{G}. It is then stipulated that
whatever happens, the thing being created (either a social system, an
action, or a cognitive state) will not be \emph{F}. It is then argued
that given the stipulation, the thing being created should not be
\emph{G}. That is not just the claim that we shouldn't \emph{aim} to
make the thing be \emph{G}. It is, rather, that in many cases being
\emph{G} is not the best way to be, given that \emph{F}-ness will not be
achieved. Lipsey and Lancaster argue that (in an admittedly idealised
model) that it is actually quite unusual for \emph{G} to be best given
that the system being created will not be \emph{F}.

It's not too hard to come up with examples that fit this structure.
Following (\citeproc{ref-Williamson2000-WILKAI}{Williamson 2000, 209}),
we might note that I'm justified in believing that there are no ideal
cognitive agents, although were I ideal I would not believe this. Or
imagine a student taking a ten question mathematics exam who has no idea
how to answer the last question. She knows an ideal student would
correctly answer an even number of questions, but that's no reason for
her to throw out her good answer to question nine. In general, once we
have stipulated one departure from the ideal, there's no reason to
assign any positive status to other similarities to the idea. In
particular, given that Coraline has an irrational view towards \emph{q},
she won't perfectly match up with the ideal, so there's no reason it's
good to agree with the ideal in other respects, such as not believing
\emph{p}.

Stepping back a bit, there's a reason the interest-relative theory says
that the ideal and justification come apart right here. On the
interest-relative theory, like on any pragmatic theory of mental states,
the \emph{identification} of mental states is a somewhat holistic
matter. Something is a belief in virtue of its position in a much
broader network. But the \emph{evaluation} of belief is (relatively)
atomistic. That's why Coraline is justified in believing \emph{p},
although if she were wiser she would not believe it. If she were wiser,
i.e., if she had the right attitude towards \emph{q}, the very same
credence in \emph{p} would not count as a belief. Whether her state
counts as a belief, that is, depends on wide-ranging features of her
cognitive system. But whether the state is justified depends on more
local factors, and in local respects she is doing everything right.

\emph{Objection}: If Coraline is justified in believing \emph{p}, then
Coraline can use \emph{p} as a premise in practical reasoning. If
Coraline can use \emph{p} as a premise in practical reasoning, and
\emph{p} is true, and her belief in \emph{p} is not Gettiered, then she
knows \emph{p}. By hypothesis, her belief is true, and her belief is not
Gettiered. So she should know \emph{p}. But she doesn't know \emph{p}.
So by several steps of modus tollens, she isn't justified in believing
\emph{p}.\footnote{Compare the `subtraction argument' on page 99 of
  Fantl and McGrath (\citeproc{ref-FantlMcGrath2009}{2009}).}

\emph{Reply}: This objection this one turns on an equivocation over the
neologism `Gettiered'. Some epistemologists use this to simply mean that
a belief is justified and true without constituting knowledge. By that
standard, the third sentence is false. Or, at least, we haven't been
given any reason to think that it is true. Given everything else that's
said, the third sentence is a raw assertion that Coraline knows that
\emph{p}, and I don't think we should accept that.

The other way epistemologists sometimes use the term is to pick out
justified true beliefs that fail to be knowledge for the reasons that
the beliefs in the original examples from Gettier
(\citeproc{ref-Gettier1963}{1963}) fail to be knowledge. That is, it
picks out a property that beliefs have when they are derived from a
false lemma, or whatever similar property is held to be doing the work
in the original Gettier examples. Now on this reading, Coraline's belief
that \emph{p} is not Gettiered. But it doesn't follow that it is known.
There's no reason, once we've given up on the JTB theory of knowledge,
to think that whatever goes wrong in Gettier's examples is the
\emph{only} way for a justified true belief to fall short of knowledge.
It could be that there's a practical defeater, as in this case. So the
second sentence of the objection is false, and the objection again
fails.

Once we have an expansive theory of defeaters, as I've adopted here, it
becomes problematic to describe the case in the language Fantl and
McGrath use. They focus a lot on whether agents like Coraline have
`knowledge-level justification' for \emph{p}, which is defined as
``justification strong enough so that shortcomings in your strength of
justification stand in the way of your knowing''.
(\citeproc{ref-FantlMcGrath2009}{Fantl and McGrath 2009, 97}). An
important part of their argument is that an agent is justified in
believing \emph{p} iff they have knowledge-level justification for
\emph{p}. I haven't addressed this argument, so I'm not really
addressing the case on their terms.

Well, does Coraline have knowledge-level justification for \emph{p}? I'm
not sure, because I'm not sure I grasp this concept. Compare the agent
in Harman's dead dictator case (\citeproc{ref-Harman1973}{Harman 1973,
75}). Does she have knowledge-level justification that the dictator is
dead? In one sense yes; it is the existence of misleading news sources
that stops her knowing. In another sense no; she doesn't know, but if
she had better evidence (e.g., seeing the death happen) she would know.
I want to say the same thing about Coraline, and that makes it hard to
translate the Coraline case into Fantl and McGrath's terminology.

\subsection*{References}\label{references}
\addcontentsline{toc}{subsection}{References}

\phantomsection\label{refs}
\begin{CSLReferences}{1}{0}
\bibitem[\citeproctext]{ref-MBT2009}
Blome-Tillmann, Michael. 2009. {``Contextualism, Subject-Sensitive
Invariantism, and the Interaction of {`Knowledge'}-Ascriptions with
Modal and Temporal Operators.''} \emph{Philosophy and Phenomenological
Research} 79 (2): 315--31. doi:
\href{https://doi.org/10.1111/j.1933-1592.2009.00280.x}{10.1111/j.1933-1592.2009.00280.x}.

\bibitem[\citeproctext]{ref-DBMJackson2007}
Braddon-Mitchell, David, and Frank Jackson. 2007. \emph{The Philosophy
of Mind and Cognition, {Second Edition}}. Malden, MA: Blackwell.

\bibitem[\citeproctext]{ref-Brown2008-BROKAP}
Brown, Jessica. 2008. {``Knowledge and Practical Reason.''}
\emph{Philosophy Compass} 3 (6): 1135--52. doi:
\href{https://doi.org/10.1111/j.1747-9991.2008.00176.x}{10.1111/j.1747-9991.2008.00176.x}.

\bibitem[\citeproctext]{ref-Cohen1988}
Cohen, Stewart. 1988. {``How to Be a Fallibilist.''} \emph{Philosophical
Perspectives} 2: 91--123. doi:
\href{https://doi.org/10.2307/2214070}{10.2307/2214070}.

\bibitem[\citeproctext]{ref-DeRose1995}
DeRose, Keith. 1995. {``Solving the Skeptical Problem.''}
\emph{Philosophical Review} 104 (1): 1--52. doi:
\href{https://doi.org/10.2307/2186011}{10.2307/2186011}.

\bibitem[\citeproctext]{ref-FantlMcGrath2009}
Fantl, Jeremy, and Matthew McGrath. 2009. \emph{Knowledge in an
Uncertain World}. Oxford: Oxford University Press.

\bibitem[\citeproctext]{ref-FeltzZarpentine2010}
Feltz, Adam, and Chris Zarpentine. 2010. {``Do You Know More When It
Matters Less?''} \emph{Philosophical Psychology} 23 (5): 683--706. doi:
\href{https://doi.org/10.1080/09515089.2010.514572}{10.1080/09515089.2010.514572}.

\bibitem[\citeproctext]{ref-Gettier1963}
Gettier, Edmund L. 1963. {``Is Justified True Belief Knowledge?''}
\emph{Analysis} 23 (6): 121--23. doi:
\href{https://doi.org/10.2307/3326922}{10.2307/3326922}.

\bibitem[\citeproctext]{ref-Hammond1988}
Hammond, Peter J. 1988. {``Consequentialist Foundations for Expected
Utility.''} \emph{Theory and Decision} 25 (1): 25--78. doi:
\href{https://doi.org/10.1007/BF00129168}{10.1007/BF00129168}.

\bibitem[\citeproctext]{ref-Harman1973}
Harman, Gilbert. 1973. \emph{Thought}. Princeton: Princeton University
Press.

\bibitem[\citeproctext]{ref-Hawthorne2004}
Hawthorne, John. 2004. \emph{Knowledge and Lotteries}. Oxford: Oxford
University Press.

\bibitem[\citeproctext]{ref-Hawthorne2008-HAWKAA}
Hawthorne, John, and Jason Stanley. 2008. {``{Knowledge and Action}.''}
\emph{Journal of Philosophy} 105 (10): 571--90. doi:
\href{https://doi.org/10.5840/jphil20081051022}{10.5840/jphil20081051022}.

\bibitem[\citeproctext]{ref-Ichikawa2009}
Ichikawa, Jonathan. 2009. {``Explaining Away Intuitions.''} \emph{Studia
Philosophica Estonica} 22 (2): 94--116. doi:
\href{https://doi.org/10.12697/spe.2009.2.2.06}{10.12697/spe.2009.2.2.06}.

\bibitem[\citeproctext]{ref-Jackson1991}
Jackson, Frank. 1991. {``Decision Theoretic Consequentialism and the
Nearest and Dearest Objection.''} \emph{Ethics} 101 (3): 461--82. doi:
\href{https://doi.org/10.1086/293312}{10.1086/293312}.

\bibitem[\citeproctext]{ref-KennettSmith1996b}
Kennett, Jeanette, and Michael Smith. 1996a. {``Frog and Toad Lose
Control.''} \emph{Analysis} 56 (2): 63--73. doi:
\href{https://doi.org/10.1111/j.0003-2638.1996.00063.x}{10.1111/j.0003-2638.1996.00063.x}.

\bibitem[\citeproctext]{ref-KennettSmith1996a}
---------. 1996b. {``Philosophy and Commonsense: The Case of Weakness of
Will.''} In \emph{The Place of Philosophy in the Study of Mind}, edited
by Michaelis Michael and John O'Leary-Hawthorne, 141--57. Norwell, MA:
Kluwer. doi:
\href{https://doi.org/10.1017/CBO9780511606977.005}{10.1017/CBO9780511606977.005}.

\bibitem[\citeproctext]{ref-Lewis1969a}
Lewis, David. 1969. \emph{Convention: A Philosophical Study}. Cambridge:
Harvard University Press.

\bibitem[\citeproctext]{ref-Lewis1982c}
---------. 1982. {``Logic for Equivocators.''} \emph{No{û}s} 16 (3):
431--41. doi:
\href{https://doi.org/10.1017/cbo9780511625237.009}{10.1017/cbo9780511625237.009}.
Reprinted in his \emph{Papers in Philosophical Logic}, Cambridge:
Cambridge University Press, 1998, 97-110. References to reprint.

\bibitem[\citeproctext]{ref-Lewis1996b}
---------. 1996. {``Elusive Knowledge.''} \emph{Australasian Journal of
Philosophy} 74 (4): 549--67. doi:
\href{https://doi.org/10.1080/00048409612347521}{10.1080/00048409612347521}.
Reprinted in his \emph{Papers in Metaphysics and Epistemology},
Cambridge: Cambridge University Press, 1999, 418-446. References to
reprint.

\bibitem[\citeproctext]{ref-LipseyLancaster}
Lipsey, R. G., and Kelvin Lancaster. 1956. {``The General Theory of
Second Best.''} \emph{Review of Economic Studies} 24 (1): 11--32. doi:
\href{https://doi.org/10.2307/2296233}{10.2307/2296233}.

\bibitem[\citeproctext]{ref-RunyonGuysDolls}
Runyon, Damon. 1992. \emph{Guys \& Dolls: The Stories of {D}amon
{R}unyon}. New York: Penguin.

\bibitem[\citeproctext]{ref-Stalnaker2008}
Stalnaker, Robert. 2008. \emph{Our Knowledge of the Internal World}.
Oxford: Oxford University Press.

\bibitem[\citeproctext]{ref-Stanley2005-STAKAP}
Stanley, Jason. 2005. \emph{{Knowledge and Practical Interests}}. Oxford
University Press.

\bibitem[\citeproctext]{ref-Sturgeon2008-STURAT}
Sturgeon, Scott. 2008. {``Reason and the Grain of Belief.''}
\emph{No{û}s} 42 (1): 139--65. doi:
\href{https://doi.org/10.1111/j.1468-0068.2007.00676.x}{10.1111/j.1468-0068.2007.00676.x}.

\bibitem[\citeproctext]{ref-Watson1977}
Watson, Gary. 1977. {``Skepticism about Weakness of Will.''}
\emph{Philosophical Review} 86 (3): 316--39. doi:
\href{https://doi.org/10.2307/2183785}{10.2307/2183785}.

\bibitem[\citeproctext]{ref-Weatherson2003-WEAWGA}
Weatherson, Brian. 2003. {``{What Good Are Counterexamples?}''}
\emph{Philosophical Studies} 115 (1): 1--31. doi:
\href{https://doi.org/10.1023/A:1024961917413}{10.1023/A:1024961917413}.

\bibitem[\citeproctext]{ref-Weatherson2005-WEACWD}
---------. 2005. {``{Can We Do Without Pragmatic Encroachment?}''}
\emph{Philosophical Perspectives} 19 (1): 417--43. doi:
\href{https://doi.org/10.1111/j.1520-8583.2005.00068.x}{10.1111/j.1520-8583.2005.00068.x}.

\bibitem[\citeproctext]{ref-Weatherson2006-WEAQC}
---------. 2006. {``Questioning Contextualism.''} In \emph{Epistemology
Futures}, edited by Stephen Cade Hetherington, 133--47. Oxford: Oxford
University Press.

\bibitem[\citeproctext]{ref-Williamson1998-WILCOK}
Williamson, Timothy. 1998. {``{Conditionalizing on Knowledge}.''}
\emph{British Journal for the Philosophy of Science} 49 (1): 89--121.
doi: \href{https://doi.org/10.1093/bjps/49.1.89}{10.1093/bjps/49.1.89}.

\bibitem[\citeproctext]{ref-Williamson2000-WILKAI}
---------. 2000. \emph{{Knowledge and its Limits}}. Oxford University
Press.

\end{CSLReferences}



\noindent Published in\emph{
Knowledge Ascriptions}, 2012, pp. 75-103.

\end{document}
