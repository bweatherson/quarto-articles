% Options for packages loaded elsewhere
% Options for packages loaded elsewhere
\PassOptionsToPackage{unicode}{hyperref}
\PassOptionsToPackage{hyphens}{url}
%
\documentclass[
  11pt,
  letterpaper,
  DIV=11,
  numbers=noendperiod,
  twoside]{scrartcl}
\usepackage{xcolor}
\usepackage[left=1.1in, right=1in, top=0.8in, bottom=0.8in,
paperheight=9.5in, paperwidth=7in, includemp=TRUE, marginparwidth=0in,
marginparsep=0in]{geometry}
\usepackage{amsmath,amssymb}
\setcounter{secnumdepth}{3}
\usepackage{iftex}
\ifPDFTeX
  \usepackage[T1]{fontenc}
  \usepackage[utf8]{inputenc}
  \usepackage{textcomp} % provide euro and other symbols
\else % if luatex or xetex
  \usepackage{unicode-math} % this also loads fontspec
  \defaultfontfeatures{Scale=MatchLowercase}
  \defaultfontfeatures[\rmfamily]{Ligatures=TeX,Scale=1}
\fi
\usepackage{lmodern}
\ifPDFTeX\else
  % xetex/luatex font selection
  \setmathfont[]{Garamond-Math}
\fi
% Use upquote if available, for straight quotes in verbatim environments
\IfFileExists{upquote.sty}{\usepackage{upquote}}{}
\IfFileExists{microtype.sty}{% use microtype if available
  \usepackage[]{microtype}
  \UseMicrotypeSet[protrusion]{basicmath} % disable protrusion for tt fonts
}{}
\usepackage{setspace}
% Make \paragraph and \subparagraph free-standing
\makeatletter
\ifx\paragraph\undefined\else
  \let\oldparagraph\paragraph
  \renewcommand{\paragraph}{
    \@ifstar
      \xxxParagraphStar
      \xxxParagraphNoStar
  }
  \newcommand{\xxxParagraphStar}[1]{\oldparagraph*{#1}\mbox{}}
  \newcommand{\xxxParagraphNoStar}[1]{\oldparagraph{#1}\mbox{}}
\fi
\ifx\subparagraph\undefined\else
  \let\oldsubparagraph\subparagraph
  \renewcommand{\subparagraph}{
    \@ifstar
      \xxxSubParagraphStar
      \xxxSubParagraphNoStar
  }
  \newcommand{\xxxSubParagraphStar}[1]{\oldsubparagraph*{#1}\mbox{}}
  \newcommand{\xxxSubParagraphNoStar}[1]{\oldsubparagraph{#1}\mbox{}}
\fi
\makeatother


\usepackage{longtable,booktabs,array}
\usepackage{calc} % for calculating minipage widths
% Correct order of tables after \paragraph or \subparagraph
\usepackage{etoolbox}
\makeatletter
\patchcmd\longtable{\par}{\if@noskipsec\mbox{}\fi\par}{}{}
\makeatother
% Allow footnotes in longtable head/foot
\IfFileExists{footnotehyper.sty}{\usepackage{footnotehyper}}{\usepackage{footnote}}
\makesavenoteenv{longtable}
\usepackage{graphicx}
\makeatletter
\newsavebox\pandoc@box
\newcommand*\pandocbounded[1]{% scales image to fit in text height/width
  \sbox\pandoc@box{#1}%
  \Gscale@div\@tempa{\textheight}{\dimexpr\ht\pandoc@box+\dp\pandoc@box\relax}%
  \Gscale@div\@tempb{\linewidth}{\wd\pandoc@box}%
  \ifdim\@tempb\p@<\@tempa\p@\let\@tempa\@tempb\fi% select the smaller of both
  \ifdim\@tempa\p@<\p@\scalebox{\@tempa}{\usebox\pandoc@box}%
  \else\usebox{\pandoc@box}%
  \fi%
}
% Set default figure placement to htbp
\def\fps@figure{htbp}
\makeatother


% definitions for citeproc citations
\NewDocumentCommand\citeproctext{}{}
\NewDocumentCommand\citeproc{mm}{%
  \begingroup\def\citeproctext{#2}\cite{#1}\endgroup}
\makeatletter
 % allow citations to break across lines
 \let\@cite@ofmt\@firstofone
 % avoid brackets around text for \cite:
 \def\@biblabel#1{}
 \def\@cite#1#2{{#1\if@tempswa , #2\fi}}
\makeatother
\newlength{\cslhangindent}
\setlength{\cslhangindent}{1.5em}
\newlength{\csllabelwidth}
\setlength{\csllabelwidth}{3em}
\newenvironment{CSLReferences}[2] % #1 hanging-indent, #2 entry-spacing
 {\begin{list}{}{%
  \setlength{\itemindent}{0pt}
  \setlength{\leftmargin}{0pt}
  \setlength{\parsep}{0pt}
  % turn on hanging indent if param 1 is 1
  \ifodd #1
   \setlength{\leftmargin}{\cslhangindent}
   \setlength{\itemindent}{-1\cslhangindent}
  \fi
  % set entry spacing
  \setlength{\itemsep}{#2\baselineskip}}}
 {\end{list}}
\usepackage{calc}
\newcommand{\CSLBlock}[1]{\hfill\break\parbox[t]{\linewidth}{\strut\ignorespaces#1\strut}}
\newcommand{\CSLLeftMargin}[1]{\parbox[t]{\csllabelwidth}{\strut#1\strut}}
\newcommand{\CSLRightInline}[1]{\parbox[t]{\linewidth - \csllabelwidth}{\strut#1\strut}}
\newcommand{\CSLIndent}[1]{\hspace{\cslhangindent}#1}



\setlength{\emergencystretch}{3em} % prevent overfull lines

\providecommand{\tightlist}{%
  \setlength{\itemsep}{0pt}\setlength{\parskip}{0pt}}



 


\setlength\heavyrulewidth{0ex}
\setlength\lightrulewidth{0ex}
\usepackage[automark]{scrlayer-scrpage}
\clearpairofpagestyles
\cehead{
  Brian Weatherson
  }
\cohead{
  The Temporal Generality Problem
  }
\ohead{\bfseries \pagemark}
\cfoot{}
\makeatletter
\newcommand*\NoIndentAfterEnv[1]{%
  \AfterEndEnvironment{#1}{\par\@afterindentfalse\@afterheading}}
\makeatother
\NoIndentAfterEnv{itemize}
\NoIndentAfterEnv{enumerate}
\NoIndentAfterEnv{description}
\NoIndentAfterEnv{quote}
\NoIndentAfterEnv{equation}
\NoIndentAfterEnv{longtable}
\NoIndentAfterEnv{abstract}
\renewenvironment{abstract}
 {\vspace{-1.25cm}
 \quotation\small\noindent\emph{Abstract}:}
 {\endquotation}
\newfontfamily\tfont{EB Garamond}
\addtokomafont{disposition}{\rmfamily}
\addtokomafont{title}{\normalfont\itshape}
\let\footnoterule\relax

\makeatletter
\renewcommand{\@maketitle}{%
  \newpage
  \null
  \vskip 2em%
  \begin{center}%
  \let \footnote \thanks
    {\itshape\huge\@title \par}%
    \vskip 0.5em%  % Reduced from default
    {\large
      \lineskip 0.3em%  % Reduced from default 0.5em
      \begin{tabular}[t]{c}%
        \@author
      \end{tabular}\par}%
    \vskip 0.5em%  % Reduced from default
    {\large \@date}%
  \end{center}%
  \par
  }
\makeatother
\RequirePackage{lettrine}

\renewenvironment{abstract}
 {\quotation\small\noindent\emph{Abstract}:}
 {\endquotation\vspace{-0.02cm}}

\setmainfont{EB Garamond Math}[
  BoldFont = {EB Garamond SemiBold},
  ItalicFont = {EB Garamond Italic},
  RawFeature = {+smcp},
]

\newfontfamily\scfont{EB Garamond Regular}[RawFeature=+smcp]
\renewcommand{\textsc}[1]{{\scfont #1}}

\renewcommand{\LettrineTextFont}{\scfont}
\KOMAoption{captions}{tableheading}
\makeatletter
\@ifpackageloaded{caption}{}{\usepackage{caption}}
\AtBeginDocument{%
\ifdefined\contentsname
  \renewcommand*\contentsname{Table of contents}
\else
  \newcommand\contentsname{Table of contents}
\fi
\ifdefined\listfigurename
  \renewcommand*\listfigurename{List of Figures}
\else
  \newcommand\listfigurename{List of Figures}
\fi
\ifdefined\listtablename
  \renewcommand*\listtablename{List of Tables}
\else
  \newcommand\listtablename{List of Tables}
\fi
\ifdefined\figurename
  \renewcommand*\figurename{Figure}
\else
  \newcommand\figurename{Figure}
\fi
\ifdefined\tablename
  \renewcommand*\tablename{Table}
\else
  \newcommand\tablename{Table}
\fi
}
\@ifpackageloaded{float}{}{\usepackage{float}}
\floatstyle{ruled}
\@ifundefined{c@chapter}{\newfloat{codelisting}{h}{lop}}{\newfloat{codelisting}{h}{lop}[chapter]}
\floatname{codelisting}{Listing}
\newcommand*\listoflistings{\listof{codelisting}{List of Listings}}
\makeatother
\makeatletter
\makeatother
\makeatletter
\@ifpackageloaded{caption}{}{\usepackage{caption}}
\@ifpackageloaded{subcaption}{}{\usepackage{subcaption}}
\makeatother
\usepackage{bookmark}
\IfFileExists{xurl.sty}{\usepackage{xurl}}{} % add URL line breaks if available
\urlstyle{same}
\hypersetup{
  pdftitle={The Temporal Generality Problem},
  pdfauthor={Brian Weatherson},
  hidelinks,
  pdfcreator={LaTeX via pandoc}}


\title{The Temporal Generality Problem}
\author{Brian Weatherson}
\date{2012}
\begin{document}
\maketitle
\begin{abstract}
The traditional generality problem for process reliabilism concerns the
difficulty in identifying each belief forming process with a particular
kind of process. That identification is necessary since individual
belief forming processes are typically of many kinds, and those kinds
may vary in reliability. I raise a new kind of generality problem, one
which turns on the difficulty of identifying beliefs with processes by
which they were formed. This problem arises because individual beliefs
may be the culmination of overlapping processes of distinct lengths, and
these processes may differ in reliability. I illustrate the force of
this problem with a discussion of recent work on the bootstrapping
problem.
\end{abstract}


\setstretch{1.1}
\section{Two Kinds of Generality
Problem}\label{two-kinds-of-generality-problem}

The generality problem is a well-known problem for process reliabilist
theories of justification.\footnote{On process reliabilism, see
  (\citeproc{ref-Goldman1979}{Goldman 1979}). On the generality problem,
  see (\citeproc{ref-Feldman1985}{Feldman 1985};
  \citeproc{ref-ConeeFeldman1998}{Conee and Feldman 1998})} Here's how
the problem usually gets started. In the first instance, token processes
of belief formation are not themselves reliable or unreliable. Rather,
it is \emph{types} of processes of belief formation that are reliable or
unreliable. But any token process is an instance of many different
types. And these types may differ in reliability.

For instance, imagine I read in the satirical newspaper \emph{The Onion}
that Barack Obama is the president. On this basis, I come to believe
that Barack Obama as president. The process I have used to form this
belief is an instance of each of these types.

\begin{enumerate}
\def\labelenumi{\arabic{enumi}.}
\tightlist
\item
  Coming to believe that Barack Obama is the president;
\item
  Believing something because it was written in \emph{The Onion}; and
\item
  Believing something because it was written in a newspaper.
\end{enumerate}

The first type of process is very reliable, at least in 2012. The second
is highly unreliable, and the third is very reliable. So should we say
that the token process I used was reliable or unreliable? More
generally, is there a principled way to map token processes to types of
process in a way that lets us systematically say whether a particular
process is reliable or not? Critics of reliabilism argue that there is
not.

As I said, this problem has been around for quite a while, but I don't
think the full force of the problem has been appreciated. Reliabilism is
a theory about whether a belief is justified or unjustified. But to
determine whether the belief is justified, we step back from the belief
itself in two respects. First, we look not to the belief, but to the
token process of belief formation from which it results. Second, we look
not just to that process, but to kinds of processes of which it is an
instant. When carrying this out, we need to make the following two
mappings.

\begin{enumerate}
\def\labelenumi{\arabic{enumi}.}
\setcounter{enumi}{3}
\tightlist
\item
  Belief → Token process of belief formation;
\item
  Token process of belief formation → type of process of belief
  formation
\end{enumerate}

The traditional point of the generality problem is that the second of
these mappings is one-many, not one-one. Each token process is
associated with many, many types of processes. But what hasn't been
sufficiently appreciated is that the first mapping is one-many as well.
And this generates a new, and potentially harder, form of the generality
problem.

That the first mapping is one-many isn't because of any special
properties of beliefs. Typically, an event is the conclusion of more
than one process. Imagine that I travel from Michigan to New York to see
a friend. I conclude this journey by walking to the friend's apartment.
With the last step I take, I conclude several processes. These include:

\begin{enumerate}
\def\labelenumi{\arabic{enumi}.}
\setcounter{enumi}{5}
\tightlist
\item
  Walking from the subway station to the apartment;
\item
  Travelling by public transit from the airport to the apartment; and
\item
  Travelling from Michigan to my friend's apartment.
\end{enumerate}

It is possible that one of these is a quite reliable process, while the
others are not. If I am good at navigating the Manhattan street grid by
foot, but poor at making it to the airport on time, then process one
will be a highly reliable process, while process three will not. So
should we say that my arrival at my friend's apartment was the result of
a reliable process or not? The best reply to that question is to point
out that it is ill formed. Given that I made it to the nearest subway
station, I used a reliable process to traverse the last few blocks. But
the longer process I used was not as reliable.

This raises a conceptual worry for process reliabilist theories. If
there is no such thing as the reliability of a conclusion, but only the
reliability of a process of getting from one or other starting point to
that conclusion, then it seems that in identifying the justifiedness of
a belief with \emph{the} reliability of the process used to generate it,
we commit a kind of category mistake. Note that this problem would
persist even if we had a one-one mapping from token processes to
epistemologically relevant types of processes that would let us solve
the traditional form of the generality problem. We would still need a
way of saying which of the many processes which terminate in a belief is
the epistemologically relevant one. I don't think there's any reason to
think there is a good answer to this question. I call this the
\emph{Temporal} Generality Problem, because the different processes that
culminate in a belief are typically of different durations.

\section{Can the Problems be Solved
Simultaneously?}\label{can-the-problems-be-solved-simultaneously}

I've argued in the previous section that in theory the Temporal
Generality Problem is distinct from the traditional version of the
generality problem. But one might think that in practice a solution to
the latter will solve problems to do with the former. Consider the
following three step process.

\begin{enumerate}
\def\labelenumi{\arabic{enumi}.}
\tightlist
\item
  I hear an astrologer say that Napoleon Bonaparte will win the 2013 US
  Presidential election.
\item
  I form the belief that Napoleon Bonaparte will win the 2013 US
  Presidential election.
\item
  I deduce that there will be a US Presidential election in 2013.
\end{enumerate}

The process by which I got from 2 to 3 is, on the face of it, highly
reliable. Assuming that I'm a mostly sensible person, coming to believe
obvious logical consequences of my prior beliefs is a highly reliable
process. Yet clearly the process that runs from 1 to 3, i.e., the
process of believing obvious logical consequences of the contents of
astrological predictions, is not a reliable process. So, one might ask,
is the resultant belief justified, because it is formed by the reliable
process that runs from 2 to 3, or unjustified, because it is formed by
the unreliable process that runs from 1 to 3?

Clearly, this is a false dilemma. The salient kind of process I'm using
between 2 and 3 is not \emph{believe obvious logical consequences of a
belief}, but \emph{believe obvious logical consequences of a belief
\textbf{formed by an unreliable process}}. Once we identify the kind of
process used at the last stage correctly, we can see that the
unreliability of the whole process causes the process used at the last
stage to be unreliable.

We might even get cases that go the other way. There are plenty of
occasions in science where scientists use mathematical techniques which
cannot be made rigorous, and idealisations that cannot easily be
replaced with approximations, or with any other statement known to be
true.\footnote{On non-rigorous techniques, see
  (\citeproc{ref-Davey2003}{Davey 2003}); on idealisations, see
  (\citeproc{ref-Davey2011}{Davey 2011}).} If we looked at such a step
in isolation, we would possibly think that it is an unreliable step,
even though it is part of a longer, reliable process. But the fact that
it is part of a reliable process matters. In particular, it matters to
the way we identify the step the scientist is using with a larger kind
of inferential processes. That kind won't involve, for instance, all
instances of reasoning from false premises, or of reasoning with
incoherent mathematical models. Rather, it will just include the kind of
reasoning that is licenced by the norms of the science that the
scientist is participating in, and that kind might be a very reliable
kind of process.

But there is one very special case where I think this kind of solution
to the Temporal Generality Problem will not work. It concerns the way in
which a reliabilist will try and solve the bootstrapping problem, as
developed by Stewart Cohen (\citeproc{ref-Cohen2002}{2002}) and Jonathan
Vogel (\citeproc{ref-Vogel2000}{2000}). We'll turn next to that problem.

\section{Generality and
Bootstrapping}\label{generality-and-bootstrapping}

Hilary Kornblith (\citeproc{ref-Kornblith2009}{2009}) has proposed that
looking at processes of longer duration generates a reliabilist solution
to the bootstrapping problem. I'm going to argue that Kornblith's
solution, which I agree is the kind of thing a reliabilist should say,
in fact shows that the Temporal Generality Problem is a distinct kind of
generality problem, and perhaps a much harder problem than the
traditional generality problem.

Let's start with a very abstract version of the problem. Assume device
\emph{D} is highly reliable, and \emph{S} trusts device \emph{D} without
antecedently knowing that it is reliable. Then the following sequence of
events take place.

\begin{itemize}
\tightlist
\item
  At \emph{t}\textsubscript{0}, \emph{S} sees that device \emph{D} says
  that \emph{p}.
\item
  At \emph{t}\textsubscript{1}, \emph{S} forms the belief that \emph{D}
  says at \emph{t}\textsubscript{0} that \emph{p} on the basis of this
  perception.\footnote{On some theories of perception, it might be that
    \emph{t}\textsubscript{0} = \emph{t}\textsubscript{1}, since
    perception involves belief formation. I don't mean to rule those
    theories out; the notation here is meant to be consistent with the
    hypothesis that \emph{t}\textsubscript{0} =
    \emph{t}\textsubscript{1}.}
\item
  At \emph{t}\textsubscript{2}, \emph{S} forms the belief that \emph{p},
  on the basis that the machine says so.
\item
  At \emph{t}\textsubscript{3}, \emph{S} forms the belief that the
  machine is accurate at \emph{t}\textsubscript{0}, on the basis of her
  last two beliefs.
\end{itemize}

What should a reliabilist say about all this? Well, the process that
runs from \emph{t}\textsubscript{0} to \emph{t}\textsubscript{1}, the
process of believing machine readings are as they appear, looks pretty
reliable, so the belief formed at \emph{t}\textsubscript{1} looks pretty
reliable. And the process that runs from \emph{t}\textsubscript{1} to
\emph{t}\textsubscript{2}, i.e., the process of believing that things
are as machine \emph{D} says they are, also looks pretty reliable, so
that belief looks pretty reliable. And the process that runs from
\emph{t}\textsubscript{2} to \emph{t}\textsubscript{3}, i.e., the
process of drawing obvious logical consequences from beliefs formed by
reliable processes, also looks pretty reliable. It's true that at
\emph{t}\textsubscript{2}, \emph{S} doesn't know she's using a reliable
process. And hence at \emph{t}\textsubscript{3}, \emph{S} doesn't know
that this is the kind of process that she's using. But none of this
should matter to an externalist like the reliabilist, since they think
what matters is actual reliability, not known reliability.

But there are two problems lurking in the vicinity. First, many people
think that it is very bizarre that \emph{S} can form a justified belief
that \emph{D} is accurate at \emph{t}\textsubscript{0} on the basis of
simply looking at \emph{D}. That's the intuition behind the
bootstrapping problem. Second, the case looks like an instance of the
Temporal Generality Problem. The two problems are related. Kornblith's
solution to the bootstrapping problem is to insist that the process used
is in fact \emph{unreliable}. What he means to draw our attention to is
that the process which runs from \emph{t}\textsubscript{0} to
\emph{t}\textsubscript{3} is unreliable. And he's right. That looks like
a process of determining whether a machine is accurate by simply looking
at the machine and trusting it. Of course, there are several other ways
we could classify the process used, but Kornblith argues that this is
the best classification, and I think he's right. And if he is right,
then we have part of a solution to the bootstrapping problem.

But if Kornblith is right, then we pretty clearly also have a nasty
instance of the Temporal Generality Problem. Because now it looks like a
chain of three reliable processes, those that run from
\emph{t}\textsubscript{0} to \emph{t}\textsubscript{1}, from
\emph{t}\textsubscript{1} to \emph{t}\textsubscript{2}, and from
\emph{t}\textsubscript{2} to \emph{t}\textsubscript{3}, collectively
form an unreliable process. The belief that is formed at
\emph{t}\textsubscript{3} is the culmination of two processes; a
reliable one that runs from \emph{t}\textsubscript{2} to
\emph{t}\textsubscript{3}, and an unreliable one that runs from
\emph{t}\textsubscript{0} to \emph{t}\textsubscript{3}. If a belief is
justified iff it is the outcome of a reliable process, and unjustified
iff it is the outcome of an unreliable process, then the belief is both
justified and unjustified, which is a contradiction.

How could the reliabilist escape this problem? I can see only two ways
out. One is to say that the process that runs from
\emph{t}\textsubscript{0} to \emph{t}\textsubscript{3} is in fact a
reliable process. But that's to fall back into the bootstrapping
problem. And in any case, it seems absurd, since that process really
does look like a process of determining whether a machine is reliable by
simply looking at it. The other is to say that the process that runs
from \emph{t}\textsubscript{2} to \emph{t}\textsubscript{3} is
unreliable. To do that, we'd need to come up with a natural kind of
process which is unreliable, and which this process instantiates. This
does not look easy. I'm not going to insist this couldn't be done, but
I'll end by noting three challenges that stand in the way of getting it
done, and which seem pretty formidible.

First, if we say the process that runs from \emph{t}\textsubscript{2} to
\emph{t}\textsubscript{3} is unreliable, then we are putting general
restrictions on how we can obtain knowledge by deductive inference. As
John Hawthorne (\citeproc{ref-Hawthorne2005Closure}{2005}) argues, any
such restrictions will be hard to motivate.

Second, the restrictions will have to be fairly sweeping to cover the
range of conclusions that, intuitively, cannot be drawn through this
kind of reasoning. Imagine a variant on the above example where at
\emph{t}\textsubscript{3}, \emph{S} concludes that either \emph{D} is
accurate at \emph{t}\textsubscript{0} or it will snow tomorrow. That's
entailed, obviously, by what she knows at \emph{t}\textsubscript{2}. And
yet the process of getting from \emph{t}\textsubscript{0} to that
conclusion seems unreliable. So we can't simply say that what's ruled
out are cases where the agent draws a conclusion that is simply about
\emph{D}.

Third, the classification of the process that runs from
\emph{t}\textsubscript{2} to \emph{t}\textsubscript{3} must not merely
fail to be ad hoc, it must plausibly be the most natural classification
available. And yet it seems there is one very natural classification
that is not available, namely the classification of the process as an
instance of deduction from known premises, or from premises arrived at
by highly reliable processes.

So the challenge this problem raises for reliabilism is substantial. I
don't mean to say it is a knock-down drawn-out refutation; philosophical
arguments rarely are. But it does add a new dimension to the generality
problem, and as we've seen in the last few paragraphs, put some new
constraints on solutions to the old version of the generality problem.

\subsection*{References}\label{references}
\addcontentsline{toc}{subsection}{References}

\phantomsection\label{refs}
\begin{CSLReferences}{1}{0}
\bibitem[\citeproctext]{ref-Cohen2002}
Cohen, Stewart. 2002. {``Basic Knowledge and the Problem of Easy
Knowledge.''} \emph{Philosophy and Phenomenological Research} 65 (2):
309--29. doi:
\href{https://doi.org/10.1111/j.1933-1592.2002.tb00204.x}{10.1111/j.1933-1592.2002.tb00204.x}.

\bibitem[\citeproctext]{ref-ConeeFeldman1998}
Conee, Earl, and Richard Feldman. 1998. {``The Generality Problem for
Reliabilism.''} \emph{Philosophical Studies} 89 (1): 1--29. doi:
\href{https://doi.org/10.1023/A:1004243308503}{10.1023/A:1004243308503}.

\bibitem[\citeproctext]{ref-Davey2003}
Davey, Kevin. 2003. {``Is Mathematical Rigor Necessary in Physics?''}
\emph{British Journal for the Philosophy of Science} 54 (3): 439--63.
doi:
\href{https://doi.org/10.1093/bjps/54.3.439}{10.1093/bjps/54.3.439}.

\bibitem[\citeproctext]{ref-Davey2011}
---------. 2011. {``Idealizations and Contextualism in Physics.''}
\emph{Philosophy of Science} 78 (1): 16--38. doi:
\href{https://doi.org/10.1086/658093}{10.1086/658093}.

\bibitem[\citeproctext]{ref-Feldman1985}
Feldman, Richard. 1985. {``Reliability and Justification.''}
\emph{Monist} 68 (2): 159--74. doi:
\href{https://doi.org/10.5840/monist198568226}{10.5840/monist198568226}.

\bibitem[\citeproctext]{ref-Goldman1979}
Goldman, Alvin. 1979. {``What Is Justified Belief.''} In
\emph{Justification and Knowledge}, edited by George Pappas, 1--23.
Dordrecht: Reidel.

\bibitem[\citeproctext]{ref-Hawthorne2005Closure}
Hawthorne, John. 2005. {``The Case for Closure.''} In \emph{Contemporary
Debates in Epistemology}, edited by Matthias Steup and Ernest Sosa,
26--43. Malden, MA: Blackwell.

\bibitem[\citeproctext]{ref-Kornblith2009}
Kornblith, Hilary. 2009. {``A Reliabilist Solution to the Problem of
Promiscuous Bootstrapping.''} \emph{Analysis} 69 (2): 263--67. doi:
\href{https://doi.org/10.1093/analys/anp012}{10.1093/analys/anp012}.

\bibitem[\citeproctext]{ref-Vogel2000}
Vogel, Jonathan. 2000. {``Reliabilism Leveled.''} \emph{Journal of
Philosophy} 97 (11): 602--23. doi:
\href{https://doi.org/10.2307/2678454}{10.2307/2678454}.

\end{CSLReferences}



\noindent Published in\emph{
Logos and Episteme}, 2012, pp. 117-122.


\end{document}
