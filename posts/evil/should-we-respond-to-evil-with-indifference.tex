% Options for packages loaded elsewhere
\PassOptionsToPackage{unicode}{hyperref}
\PassOptionsToPackage{hyphens}{url}
%
\documentclass[
  10pt,
  letterpaper,
  DIV=11,
  numbers=noendperiod,
  twoside]{scrartcl}

\usepackage{amsmath,amssymb}
\usepackage{setspace}
\usepackage{iftex}
\ifPDFTeX
  \usepackage[T1]{fontenc}
  \usepackage[utf8]{inputenc}
  \usepackage{textcomp} % provide euro and other symbols
\else % if luatex or xetex
  \usepackage{unicode-math}
  \defaultfontfeatures{Scale=MatchLowercase}
  \defaultfontfeatures[\rmfamily]{Ligatures=TeX,Scale=1}
\fi
\usepackage{lmodern}
\ifPDFTeX\else  
    % xetex/luatex font selection
  \setmainfont[ItalicFont=EB Garamond Italic,BoldFont=EB Garamond
Bold]{EB Garamond Math}
  \setsansfont[]{Europa-Bold}
  \setmathfont[]{Garamond-Math}
\fi
% Use upquote if available, for straight quotes in verbatim environments
\IfFileExists{upquote.sty}{\usepackage{upquote}}{}
\IfFileExists{microtype.sty}{% use microtype if available
  \usepackage[]{microtype}
  \UseMicrotypeSet[protrusion]{basicmath} % disable protrusion for tt fonts
}{}
\usepackage{xcolor}
\usepackage[left=1in, right=1in, top=0.8in, bottom=0.8in,
paperheight=9.5in, paperwidth=6.5in, includemp=TRUE, marginparwidth=0in,
marginparsep=0in]{geometry}
\setlength{\emergencystretch}{3em} % prevent overfull lines
\setcounter{secnumdepth}{3}
% Make \paragraph and \subparagraph free-standing
\ifx\paragraph\undefined\else
  \let\oldparagraph\paragraph
  \renewcommand{\paragraph}[1]{\oldparagraph{#1}\mbox{}}
\fi
\ifx\subparagraph\undefined\else
  \let\oldsubparagraph\subparagraph
  \renewcommand{\subparagraph}[1]{\oldsubparagraph{#1}\mbox{}}
\fi


\providecommand{\tightlist}{%
  \setlength{\itemsep}{0pt}\setlength{\parskip}{0pt}}\usepackage{longtable,booktabs,array}
\usepackage{calc} % for calculating minipage widths
% Correct order of tables after \paragraph or \subparagraph
\usepackage{etoolbox}
\makeatletter
\patchcmd\longtable{\par}{\if@noskipsec\mbox{}\fi\par}{}{}
\makeatother
% Allow footnotes in longtable head/foot
\IfFileExists{footnotehyper.sty}{\usepackage{footnotehyper}}{\usepackage{footnote}}
\makesavenoteenv{longtable}
\usepackage{graphicx}
\makeatletter
\def\maxwidth{\ifdim\Gin@nat@width>\linewidth\linewidth\else\Gin@nat@width\fi}
\def\maxheight{\ifdim\Gin@nat@height>\textheight\textheight\else\Gin@nat@height\fi}
\makeatother
% Scale images if necessary, so that they will not overflow the page
% margins by default, and it is still possible to overwrite the defaults
% using explicit options in \includegraphics[width, height, ...]{}
\setkeys{Gin}{width=\maxwidth,height=\maxheight,keepaspectratio}
% Set default figure placement to htbp
\makeatletter
\def\fps@figure{htbp}
\makeatother
% definitions for citeproc citations
\NewDocumentCommand\citeproctext{}{}
\NewDocumentCommand\citeproc{mm}{%
  \begingroup\def\citeproctext{#2}\cite{#1}\endgroup}
\makeatletter
 % allow citations to break across lines
 \let\@cite@ofmt\@firstofone
 % avoid brackets around text for \cite:
 \def\@biblabel#1{}
 \def\@cite#1#2{{#1\if@tempswa , #2\fi}}
\makeatother
\newlength{\cslhangindent}
\setlength{\cslhangindent}{1.5em}
\newlength{\csllabelwidth}
\setlength{\csllabelwidth}{3em}
\newenvironment{CSLReferences}[2] % #1 hanging-indent, #2 entry-spacing
 {\begin{list}{}{%
  \setlength{\itemindent}{0pt}
  \setlength{\leftmargin}{0pt}
  \setlength{\parsep}{0pt}
  % turn on hanging indent if param 1 is 1
  \ifodd #1
   \setlength{\leftmargin}{\cslhangindent}
   \setlength{\itemindent}{-1\cslhangindent}
  \fi
  % set entry spacing
  \setlength{\itemsep}{#2\baselineskip}}}
 {\end{list}}
\usepackage{calc}
\newcommand{\CSLBlock}[1]{\hfill\break\parbox[t]{\linewidth}{\strut\ignorespaces#1\strut}}
\newcommand{\CSLLeftMargin}[1]{\parbox[t]{\csllabelwidth}{\strut#1\strut}}
\newcommand{\CSLRightInline}[1]{\parbox[t]{\linewidth - \csllabelwidth}{\strut#1\strut}}
\newcommand{\CSLIndent}[1]{\hspace{\cslhangindent}#1}

\setlength\heavyrulewidth{0ex}
\setlength\lightrulewidth{0ex}
\usepackage[automark]{scrlayer-scrpage}
\clearpairofpagestyles
\cehead{
  Brian Weatherson
  }
\cohead{
  Should We Respond to Evil With Indifference?
  }
\ohead{\bfseries \pagemark}
\cfoot{}
\makeatletter
\newcommand*\NoIndentAfterEnv[1]{%
  \AfterEndEnvironment{#1}{\par\@afterindentfalse\@afterheading}}
\makeatother
\NoIndentAfterEnv{itemize}
\NoIndentAfterEnv{enumerate}
\NoIndentAfterEnv{description}
\NoIndentAfterEnv{quote}
\NoIndentAfterEnv{equation}
\NoIndentAfterEnv{longtable}
\NoIndentAfterEnv{abstract}
\renewenvironment{abstract}
 {\vspace{-1.25cm}
 \quotation\small\noindent\rule{\linewidth}{.5pt}\par\smallskip
 \noindent }
 {\par\noindent\rule{\linewidth}{.5pt}\endquotation}
\KOMAoption{captions}{tableheading}
\makeatletter
\@ifpackageloaded{caption}{}{\usepackage{caption}}
\AtBeginDocument{%
\ifdefined\contentsname
  \renewcommand*\contentsname{Table of contents}
\else
  \newcommand\contentsname{Table of contents}
\fi
\ifdefined\listfigurename
  \renewcommand*\listfigurename{List of Figures}
\else
  \newcommand\listfigurename{List of Figures}
\fi
\ifdefined\listtablename
  \renewcommand*\listtablename{List of Tables}
\else
  \newcommand\listtablename{List of Tables}
\fi
\ifdefined\figurename
  \renewcommand*\figurename{Figure}
\else
  \newcommand\figurename{Figure}
\fi
\ifdefined\tablename
  \renewcommand*\tablename{Table}
\else
  \newcommand\tablename{Table}
\fi
}
\@ifpackageloaded{float}{}{\usepackage{float}}
\floatstyle{ruled}
\@ifundefined{c@chapter}{\newfloat{codelisting}{h}{lop}}{\newfloat{codelisting}{h}{lop}[chapter]}
\floatname{codelisting}{Listing}
\newcommand*\listoflistings{\listof{codelisting}{List of Listings}}
\makeatother
\makeatletter
\makeatother
\makeatletter
\@ifpackageloaded{caption}{}{\usepackage{caption}}
\@ifpackageloaded{subcaption}{}{\usepackage{subcaption}}
\makeatother
\ifLuaTeX
  \usepackage{selnolig}  % disable illegal ligatures
\fi
\usepackage{bookmark}

\IfFileExists{xurl.sty}{\usepackage{xurl}}{} % add URL line breaks if available
\urlstyle{same} % disable monospaced font for URLs
\hypersetup{
  pdftitle={Should We Respond to Evil With Indifference?},
  pdfauthor={Brian Weatherson},
  hidelinks,
  pdfcreator={LaTeX via pandoc}}

\title{Should We Respond to Evil With Indifference?}
\author{Brian Weatherson}
\date{2005}

\begin{document}
\maketitle
\begin{abstract}
In a recent article, Adam Elga outlines a strategy for ``Defeating Dr
Evil with Self-Locating Belief''. The strategy relies on an indifference
principle that is not up to the task. In general, there are two things
to dislike about indifference principles: adopting one normally means
confusing risk for uncertainty, and they tend to lead to incoherent
views in some `paradoxical' situations. I argue that both kinds of
objection can be levelled against Elga's indifference principle. There
are also some difficulties with the concept of evidence that Elga uses,
and these create further difficulties for the principle.
\end{abstract}

\setstretch{1.1}
In a recent article, Adam Elga (\citeproc{ref-Elga2004}{2004}) outlines
a strategy for ``Defeating Dr Evil with Self-Locating Belief''. The
strategy relies on an indifference principle that is not up to the task.
In general, there are two things to dislike about indifference
principles: adopting one normally means confusing risk for uncertainty,
and they tend to lead to incoherent views in some `paradoxical'
situations. Each kind of objection can be levelled against Elga's
theory, but because Elga is more careful than anyone has ever been in
choosing the circumstances under which his indifference principle
applies we have to be similarly careful in focussing the objections.
Even with this care the objections I put forward here will be less
compelling than, say, the objections Keynes
(\citeproc{ref-Keynes1921}{1921} Ch. 4) put forward in his criticisms of
earlier indifference principles. But there still may be enough to make
us reject Elga's principle. The structure of this note is as follows. In
and 2 I set out Elga's theory, in and 4 I discuss some initial
objections that I don't think are particularly telling, in I discuss
some paradoxes to which Elga's theory seems to lead (this is reprised in
where I discuss a somewhat different paradoxical case) and in and 8 I
argue that even Elga's careful indifference principle involves a
risk/uncertainty confusion.

\section{From Basel to Princeton}\label{sec-basel}

In (\citeproc{ref-Lewis1979b}{1979}) David Lewis argued that the
contents of contentful mental states were not propositions, but
properties. When I think that I'm a rock star, I don't attribute truth
to the proposition \emph{Brian is a rock star}, but rather attribute the
property of rock stardom to myself. Lewis was led to this position by
considering cases where a believer is mistaken about his own identity.
For example, if I believe that I'm a rock star without believing that
I'm Brian, and in fact while thinking that Brian is an infamous
philosopher, it is odd to attribute to me belief in the proposition
\emph{Brian is a rock star}. But it is perfectly natural to say I
self-attribute rock stardom, and that's just what Lewis says.

If we accept Lewis's position, there are two paths we can take. First,
we can try simply replacing all talk of propositional attitudes with
talk of proprietal attitudes, and trusting and hoping that this won't
make a difference to our subsequent theorising. Alternatively, we can
see if changing the type of entity that is the content of a contentful
state has distinctive consequences, and in particular see if it gives us
the conceptual resources to make progress on some old problems. That's
the approach Adam Elga has taken in a couple of papers, and whatever one
thinks of his conclusions, the early returns certainly suggest that this
Lewisian outlook will prove remarkably fruitful.

On the Lewisian approach, credences are defined over properties, and
properties are sets of possibilia, i.e.~centred worlds. Some properties
are maximally precise, they are satisfied by exactly one possible
object. Elga sometimes calls these maximally specific properties
\emph{predicaments} because they specify exactly what is happening to
the agent that instantiates one. Say predicaments
\emph{F}\textsubscript{1} and \emph{F}\textsubscript{2} are similar iff
the \emph{F}\textsubscript{1} and the \emph{F}\textsubscript{2} are
worldmates and their experiences are indistinguishable. Elga's principle
INDIFFERENCE says that if predicaments \emph{F}\textsubscript{1} and
\emph{F}\textsubscript{2} are similar then any rational agent should
assign equal credence to \emph{F}\textsubscript{1} and
\emph{F}\textsubscript{2}. This becomes most interesting when there are
similar \emph{F}\textsubscript{1} and \emph{F}\textsubscript{2}. So, for
instance, consider poor O'Leary.

\begin{description}
\tightlist
\item[O'LEARY]
O'Leary is locked in the trunk of his car overnight. He knows that he'll
wake up briefly twice during the night (at 1:00 and again at 2:00) and
that the awakenings will be subjectively indistinguishable (because by
2:00 he'll have forgotten the 1:00 awakening). At 1:00 he wakes up.
\end{description}

Elga says that when O'Leary wakes up, he should assign equal credence to
it being 1:00 as to it being 2:00. So, provided O'Leary knows that one
of these two hypotheses is true, INDIFFERENCE says that he should assign
credence 1/2 to it being 1:00 at the wake up.

Elga has an argument for INDIFFERENCE, which we shall get to by , but
for a while I will look at some immediate consequences of the position.
I'll start with two reasons to think that INDIFFERENCE needs to be
strengthened to play the role he wants it to play.

\section{Add it Up}\label{add-it-up}

One difficulty with INDIFFERENCE as stated so far is that it applies
only to very narrow properties, predicaments, and it is not clear how to
generalise to properties in which we are more interested.

\begin{description}
\tightlist
\item[BERNOULLIUM]
Despite months of research, Leslie still doesn't know what the half-life
of Bernoullium, her newly discovered element is. It's between one and
two nanoseconds, but she can't manufacture enough of the stuff to get a
better measurement than that. She does, however, know that she's locked
in the trunk of her car, and that like O'Leary she will have two
indistinguishable nocturnal awakenings. She's having one now in fact,
but naturally she can't tell whether it is the first or the second.
\end{description}

INDIFFERENCE says that Leslie should assign credence 1/2 to it being the
first wake-up, right? Not yet. All that INDIFFERENCE says is that any
two predicaments should receive equal credence. A predicament is
maximally specific, so it specifies, \emph{inter alia}, the half-life of
Bernoullium. But for any \emph{x}, Leslie assigns credence 0 to \emph{x}
being the half-life of Bernoullium, because there are uncountably many
candidates for being the half-life, and none of them look better than
any of the others. So she assigns credence 0 to every predicament, and
so she satisfies INDIFFERENCE no matter what she thinks about what the
time is. Even if, for no reason at all, she is certain it is her second
awakening, she still satisfies INDIFFERENCE as it is written, because
she assigns credence 0 to every predicament, and hence equal credence to
similar predicaments.

Fortunately, we can strengthen INDIFFERENCE to cover this case. To
start, note that the motivations for INDIFFERENCE suggest that if two
predicaments are similar then they should receive equal credence not
just in the agent's actual state, but even when the agent gets more
evidence. Leslie should keep assigning equal credence to it being her
first or second wake up if she somehow learns what the half-life of
Bernoullium is, for example. This suggests the following
principle:\footnote{INDIFFERENCE entails C-INDIFFERENCE given the
  following extra assumptions. First, if INDIFFERENCE is true it is
  indefeasible, so it must remain true whatever one's evidence is.
  Secondly, rational agents should update by conditionalisation.
  Thirdly, it is always possible for an agent to get evidence that tells
  her she is in \emph{F}\textsubscript{1} or \emph{F}\textsubscript{2}
  and no more. The third premise is at best an idealisation, but it is
  hard to see how or why that should tell against C-INDIFFERENCE.}

\begin{description}
\tightlist
\item[C-INDIFFERENCE]
If \emph{F}\textsubscript{1} and \emph{F}\textsubscript{2} are similar,
and an agent does not \emph{know} that she is in neither, then her
conditional credence on being \emph{F}\textsubscript{1}, conditional on
being either \emph{F}\textsubscript{1} or \emph{F}\textsubscript{2},
should be 1/2.
\end{description}

But even this doesn't quite resolve our problem. Simplifying Leslie's
situation somewhat, the live predicaments are all of the following form:
this is the first/second awakening, and the half-life of Bernoullium is
\emph{x}. C-INDIFFERENCE requires that for any \emph{c}, conditional on
the half-life of Bernoullium being \emph{c}, Leslie assign credence 1/2
to it being her first awakening. From this and the fact that Leslie's
credence function is a probability function it \emph{doesn't} follow
that her credence in this being her first awakening is 1/2. So to get
INDIFFERENCE to do the work it is meant to do in Leslie's case (and
presumably O'Leary's case, since in practice there will be some other
propositions about which O'Leary is deeply uncertain) I think we need to
strengthen it to the following.

\begin{description}
\tightlist
\item[P-INDIFFERENCE]
If \emph{G}\textsubscript{1} and \emph{G}\textsubscript{2} are
properties such that:
\end{description}

\begin{enumerate}
\def\labelenumi{\arabic{enumi}.}
\tightlist
\item
  For all worlds \emph{w}, there is at most one
  \emph{G}\textsubscript{1} in \emph{w} and at most one
  \emph{G}\textsubscript{2} in \emph{w};
\item
  For all worlds \emph{w}, there is a \emph{G}\textsubscript{1} in
  \emph{w} iff there is a \emph{G}\textsubscript{2} in \emph{w}; and
\item
  For all worlds \emph{w} where there is a \emph{G}\textsubscript{1} in
  \emph{w}, the \emph{G}\textsubscript{1} and the
  \emph{G}\textsubscript{2} have indistinguishable experiences; then
\end{enumerate}

\emph{G}\textsubscript{1} and \emph{G}\textsubscript{2} deserve equal
credence.

Elga does not endorse either C-INDIFFERENCE or P-INDIFFERENCE, but I
suspect he should given his starting assumptions. It is hard to believe
if O'Leary is certain about everything save what time it is, then
rationality imposes very strong constraints on his beliefs about time,
while rationality imposes no such constraints should he (or Leslie) be
uncertain about the half-life of Bernoullium. Put another way, it is
hard to believe that in her current state Leslie could rationally assign
credence 0.9 to this being her first awakening, but if she decided the
half-life of Bernoullium is 1.415 nanoseconds, then she would be
required to change that credence to 0.5. If we have INDIFFERENCE without
P-INDIFFERENCE, that is possible. So I will assume in what follows that
if C-INDIFFERENCE and P-INDIFFERENCE are false then INDIFFERENCE is
heavily undermined.\footnote{Note also that if P-INDIFFERENCE is false,
  then Dr Evil has an easy way out of the `brain race' that comes up at
  the end of Elga's paper. He just need be told about some new element
  without being told its half-life, and magically he is free to assign
  credence 1 to his being on the spaceship rather than on Earth. This
  would reduce the interest of the puzzle somewhat I fear.}

\section{Out of sight, out of mind}\label{sec-internalism}

Elga's discussion presupposes two kinds of internalism. First, he
assumes that some internalist theory of experience is true. Second, he
assumes that some internalist theory of justification is true. If the
first assumption is false it threatens the applicability of the theory.
If the second assumption is false it threatens the truth of the theory.

An externalist theory of experience says that what kind of experience
\emph{S} is having is determined, inter alia, by what \emph{S} is
experiencing. While setting out such a view, John Campbell
(\citeproc{ref-Campbell2002}{2002, 124--26}) says that two people
sitting in duplicate prison cells looking at duplicate coffee cups will
have different experiences, because one will have an experience of the
coffee cup in her hand, and the other will not have an experience of
that cup. This does not threaten INDIFFERENCE, but it does seem to
render it trivial. On Campbell's view, if two agents are able to make
demonstrative reference to different objects, and there is no reason to
think Elga's agents in allegedly similar but not numerically identical
predicaments cannot, they are having different experiences. Hence the
situations are not really similar after all. Strictly speaking, this is
good news for INDIFFERENCE, since it is hard given this view of
experience to find counterexamples to it. But I doubt that Elga will be
happy with this defence.

The second kind of internalist assumption is more threatening. Many
externalists about justification think whether a particular experience
justifies a belief for an agent depends not just on intrinsic features
of that experience, but on the relationship between experiences of that
kind and the world around the agent. In some versions of this,
especially the version defended by Timothy Williamson
(\citeproc{ref-Williamson1998-WILCOK}{1998}), whether an experience
either constitutes or produces evidence depends on whether it
constitutes or produces knowledge. Since it is not clear that any two
similar agents know the same thing, since it is clear that they do not
have the same \emph{true} \emph{beliefs}, on Williamson's theory it
seems that the agents will not have the same evidence. In particular, it
is possible that part of one agent's evidence is inconsistent with her
being the other agent. If part of her evidence is that she has hands,
then she is not a brain-in-a-vat having experiences like hers, and she
should not assign high credence to the claim that she is one, no matter
what INDIFFERENCE says. So Elga needs to reject this kind of externalism
about evidence. This is not a devastating objection. I am sure that Elga
does reject Campbell's and Williamson's theories, so just raising them
against him without argument would be question-begging. But this does
mean that the target audience for INDIFFERENCE is smaller than for some
philosophical claims, since adherents of Campbell's or Williamson's
views will be antecedently disposed to think INDIFFERENCE is useless or
false.

\section{It's Evidently Intransitive}\label{its-evidently-intransitive}

Dakota is sitting in a bright green room. She is trying to reconstruct
how she got there when Dr Evil informs her just what happened. An
epistemology student, not coincidentally called Dakota, was snatched out
of her study and duplicated 999 times over. The duplicates were then
numbered (though we've lost which number was given to the original) each
put in a coloured cell. The thousand coloured cells rotated slowly
through the colour sphere, starting with cell 0 (the new home of Dakota
number 0) being green, going blueish until cell 250 (for Dakota number
250) is just blue, then reddish until cell 500 is just red, swinging
through the yellows with pure yellow reached at 750, and then back to
the greens, with 999 being practically identical to 1000. For any
\emph{n}, cells number \emph{n} and \emph{n}+1 are indistinguishable.
That means that Dakota number \emph{n} is similar, in Elga's sense, to
Dakota number \emph{n}+1, for their (apparent) experiences before being
in the rooms are identical, and their experiences in the rooms are
indistinguishable. Hence our Dakota, sitting in the bright green room,
should assign equal credence to being Dakota number \emph{n} and Dakota
number \emph{n}+1 for any \emph{n}. But this is absurd. Since she can
see that her walls are green, she should assign high credence to being
Dakota number 0, and credence 0 to being Dakota number 500.

The problem here is that Elga wants to define an equivalence relation on
predicaments, the relation \emph{deserving the same credence as}, out of
an intransitive relation, \emph{being indistinguishable from}. There are
two possible responses, each of them perfectly defensible.

First, Elga could deny the premise that the adjacent cells are
indistinguishable. Although there is some prima facie plausibility to
the claim that some different colours are indistinguishable, Delia Graff
Fara (\citeproc{ref-Fara2001}{2001}) has argued that this is false. It
would mean committing to yet another controversial philosophical
position, but if Elga endorsed Graff's claims, he could easily deal with
Dakota.

Secondly, he could tinker with the definition of similarity. Instead of
saying that possibilia represent similar predicaments iff they are
indistinguishable worldmates, he could say that they represent similar
predicaments iff they are worldmates that are indistinguishable from the
same predicaments. (This kind of strategy for generating an equivalence
relation from an intransitive relation is borrowed from Goodman
(\citeproc{ref-Goodman1951}{1951}).) Even if adjacent cells are
indistinguishable from each other, they will not be indistinguishable
from the same cells. This delivers the plausible result that the
duplicate Dakotas stuck in the cells do not instantiate similar
predicaments. Some might object that this move is ad hoc, but once we
realise the need to make \emph{similar} an equivalence relation, it
seems clear enough that this is the most natural way to do that.

\section{Morgan and Morgan and Morgan and
Morgan}\label{morgan-and-morgan-and-morgan-and-morgan}

I think I outdid myself this time, said Dr Evil. I was just going along
duplicating you, or at least someone like you, and the duplication
process was taking less and less time. So I thought, I wonder what is
the lower bound here? How quick can we make the duplication process? So
I tried a few things to cut down the time it took, and I got a little
better with practice, and, well, it turns out that the time taken can be
made arbitrarily small. Before I knew it, there were infinitely many of
you. Oops.

Morgan was a little shocked. She could cope with having a duplicate or
two around, but having infinitely many duplicates was a little hard to
take. On the other hand, and this was hard to think about, perhaps she
should be grateful. Maybe she was one of the later ones created, and she
wouldn't have existed if not for Evil's irrational exuberance. She
started to ponder how likely that was, but she was worried that it
required knowing more about Evil than any mortal could possibly know.

Well, continued Dr Evil, I did one thing right. As each duplicate was
created I gave it a serial number, 0 for the original Morgan, 1 for the
first duplicate and so on, so the bookkeeping will be easier. Don't go
looking for it, it's written on your left leg in ectoplasmic ink, and
you won't be able to see it.

Now that makes things easier, thought Morgan. By INDIFFERENCE the
probability that my serial number is \emph{x} is 1/\emph{n}, where
\emph{n} is the number of duplicates created. So dividing 1 by infinity,
that's zero. So the probability that my serial number is less than
\emph{x} is the probability that it's zero plus the probability that
it's one plus \ldots{} plus the probability that it's \emph{x}, that's
still zero. So if he had stopped after \emph{x} for any \emph{x}, I
would not exist with probability one. I'm liking Evil more and more,
though something bothers me about that calculation.

Morgan was right to worry. She's just talked herself, with Elga's help,
into a violation of the principle of countable additivity. The
additivity axiom in standard probability theory says that for any two
disjoint propositions, the probability of their disjunction is the sum
of their probabilities. The countable additivity axiom says that for any
countable set of disjoint propositions, the probability that at least
one of them is true is the sum of each of their probabilities. (It
follows from the axioms of probability theory that this sum is always
defined.) Here we have to alter these axioms slightly so they apply to
properties rather than propositions, but still the principle of
countable additivity seems plausible. But Morgan has to violate it. The
probability she assigns to having some serial number or other is not
zero, in fact it is one as long as she takes Evil at his word. But for
each \emph{x}, the probability that her serial number is \emph{x} is
zero. In symbols, we have

\begin{itemize}
\tightlist
\item
  \emph{Pr}(\({\exists}\)\emph{x} (Serial number = \emph{x})) = 1
\item
  \({\Sigma}\)\emph{Pr}(Serial number = \emph{x}) = 0
\end{itemize}

But countable additivity says that these values should be equal.

Orthodoxy endorses countable additivity, but there are notable
dissenters that are particularly relevant here. Bruno Finetti
(\citeproc{ref-DeFinetti1974}{1974}) argued that countable additivity
should be rejected because it rules out the possibility of an even
distribution across the natural numbers. DeFinetti thought, as Morgan
does, that we could rationally be in a position where we know of a
particular random variable only that its value is a non-negative
integer, and for every \emph{x}, we assign equal probability to each
hypothesis that its value is \emph{x}. Since that is inconsistent with
countable additivity, all the worse for countable additivity. This is a
decent argument, though as de Finetti himself noted, it has some
counterintuitive consequences.

I decided, Dr Evil continued, to do something fairly spectacular with
all these people. By some small tinkering with your physiology I found a
way to make you immortal. Unfortunately, a quick scan of your psychology
revealed that you weren't capable of handling eternity. So every fifty
years I will wipe all your memories and return you to the state you were
in when duplicated. I will write, or perhaps I did write, on your right
leg the number of times that your memories have been thus wiped. Don't
look, it's also in ectoplasmic ink. Just to make things fun, I made
enough duplicates of myself so that every fifty years I can tell you
what happened. Each fifty-year segment of each physical duplicate will
be an epistemic duplicate of every other such segment. How cool is
that?\footnote{Evil's plan resembles in many respects a situation
  described by Jamie Dreier (\citeproc{ref-Dreier2001}{2001}) in his
  ``Boundless Good''. The back story is a little different, but the
  situation is closely (and intentionally) modelled on his sphere of
  pain/sphere of pleasure example.}

Morgan was not particularly convinced that it was cool, but an odd
thought crossed her mind once or twice. She had one number \emph{L}
written on her left leg, and another number \emph{R} written on her
right leg. She had no idea what those numbers were, but she thought she
might be in a position to figure out the odds that \emph{L} \({\geq}\)
\emph{R}. So she started reasoning as follows, making repeated appeals
to C-INDIFFERENCE. (She must also appeal to P-INDIFFERENCE at every
stage if there are other propositions about which she is uncertain.
Assume that appeal made.)

Let's say the number on my left leg is 57. Then \emph{L}~\({\geq}\)
\emph{R} iff \emph{R}~\textless{} 58. But since there are 58 ways for
\emph{R}~\textless~58 to be true, and infinitely many ways for
\emph{R}~\textless{} 58 to be false, and by C-INDIFFERENCE each of these
ways deserve the same credence conditional on \emph{L}~=~57, we get
\emph{Pr}(\emph{L}~\({\geq}\)~\emph{R}~~\emph{L}~=~57) = 0. But 57 was
arbitrary in this little argument, so I can conclude
\({\forall}\)\emph{l}:
\emph{Pr}(\emph{L}~\({\geq}\)~\emph{R}~~\emph{L}~=~\emph{l}) = 0. This
seems to imply that \emph{Pr}(\emph{L}~\({\geq}\)~\emph{R})~= 0,
especially since I know \emph{L} takes some value or other, but let's
not be too hasty.

Let's say the number on my right leg is 68. Then
\emph{L}~\({\geq}\)~\emph{R} iff \emph{L}~\({\geq}\) 68. And since there
are 68 ways for \emph{L}~\({\geq}\) 68 to be false, and infinitely many
ways for it to be true, and by C-INDIFFERENCE each of these ways deserve
the same credence conditional on \emph{R}~=~68, we get
\emph{Pr}(\emph{L}~\({\geq}\)~\emph{R}~~\emph{R}~=~68) = 1. But 68 was
arbitrary in this little argument, so I can conclude
\({\forall}\)\emph{r}:
\emph{Pr}(\emph{L}~\({\geq}\)~\emph{R}~~\emph{R}~=~\emph{r}) = 1. This
seems to imply that \emph{Pr}(\emph{L}~\({\geq}\)~\emph{R})~= 1,
especially since I know \emph{R} takes some value or other, but now I'm
just confused.

Morgan is right to be confused. She has not quite been led into
inconsistency, because as she notes the last step, from
\({\forall}\)\emph{l}:~\emph{Pr}(\emph{L}~\({\geq}\)~\emph{R}~~\emph{L}~=~\emph{l})
= 0 to \emph{Pr}(\emph{L}~\({\geq}\)~\emph{R})~= 0 is not \emph{forced}.
In fact, the claim that this is always a valid inferential step is
equivalent to the principle of countable additivity, which we have
already seen a proponent of INDIFFERENCE in all its variations must
reject. But it would be a mistake to conclude from this that we just
have a standoff. What Morgan's case reveals is that accepting the
indifference principles that Elga offers requires giving up on an
intuitively plausible principle of inference. That principle says that
if the probability of \emph{p} conditional on any member of a partition
is \emph{x}, then the probability of \emph{p} is \emph{x}. If we think
that principle of inference is \emph{prima facie} more plausible than
Elga's principle of indifference, as I think we should, that is pretty
good \emph{prima facie} evidence that Elga's principle is wrong.

The next three sections will be devoted to determining whether we can
convert this persuasive argument into a knockdown argument (we cannot)
and whether Elga's arguments in favour of INDIFFERENCE do enough to
overcome this \emph{prima facie} argument that INDIFFERENCE is flawed
(they do not). A concluding section notes how to redo this argument so
it appeals only to potential rather than actual infinities.

\section{Intermission}\label{intermission}

CHARYBDIS: I know how to make that argument stronger. Just get Evil to
offer Morgan a bet on whether \emph{L}~\({\geq}\)~\emph{R}. Ask how much
she'll pay for a bet that pays €1 if \emph{L} \({\geq}\) \emph{R} and
nothing otherwise. If she pays anything for it, tell her the value of
\emph{L}, whatever it is, and ask her if she'd like to sell that bet
back for half what she paid for it. Since she now assigns probability
zero to \emph{L}~\({\geq}\)~\emph{R} she'll happily do that, and then
she'll have lost money. If she won't pay anything for the bet to start
with, offer her the reverse bet. She should pay €1 for that, and now
apply the same tactics except tell her the value of \emph{R} rather than
\emph{L}. Either way the stupid person will lose money.

SCYLLA: Very practical Charybdis, but we're not sure it gets to the
heart of the matter. Not sure. Well, let us say why rather than leaving
it like that. For one thing, Morgan might not like playing dice with
Evil, even if Evil is the source of her life. So she might have a
maximum price of 0 for either bet.

CHARYBDIS: But then surely she'll be turning down a sure win. I mean
between the bets she has a sure gain of at least €1.

SCYLLA: And if she is offered both bets at once we're sure she would
take that gain, but as we heard your story she wasn't.\footnote{Compare
  the objection to Dutch Book arguments in Schick
  (\citeproc{ref-Schick1986}{1986}).}

CHARYBDIS: So does this mean her degree of belief in both
\emph{R}~\({\geq}\)~\emph{L} and \emph{L}~\({\geq}\)~\emph{R} is 0?

SCYLLA: It might mean that, and of course some smart people have argued
that that is coherent, much to the chagrin of your Bayesian friends
we're sure.\footnote{For example, Shafer
  (\citeproc{ref-Shafer1976}{1976}).} But more likely it means that she
just isn't following the patterns of practical reasoning that you
endorse.\footnote{Compare the state-dependent approach to
  decision-making discussed in Chambers and Quiggin
  (\citeproc{ref-ChambersQuiggin2000}{2000}).} Also, we're not so sure
about the overall structure of the argument. We think your reasoning is
as follows. Morgan ends up doing something silly, giving up money.
(Well, we're not sure that's always silly, but let's say it is here.) So
something went wrong. So she has silly beliefs. That last step goes by
fairly fast we think. From her making some mistake or other, we can only
conclude that, well, she made some mistake or other, not that she made
some particular mistake in the composition of her credences.\footnote{This
  point closely resembles an objection to Dutch Book reasoning made in
  Hájek (\citeproc{ref-Hajek2005}{2005}), though Scylla is much more
  sceptical about how much we can learn from these pragmatic arguments
  than Hájek is.}

CHARYBDIS: What other mistake might she have made?

SCYLLA: There are many hidden premises in your chains of reasoning to
conclusions about how Morgan should behave. For instance, she only
values a €1 bet on \emph{L} \({\geq}\) \emph{R} at
\emph{Pr}(\emph{L}~\({\geq}\)~\emph{R}) if she knows she can't buy that
bet more cheaply elsewhere, or sell it for a larger price elsewhere.
Even if those assumptions are \emph{true}, Morgan may unreasonably
believe they are false, and that might be her mistake.\footnote{Scylla's
  reasoning here is based on Milne (\citeproc{ref-Milne1991}{1991}),
  though of course Milne's argument is much less condensed than that.}
But even that isn't our main concern. Our main concern is that you
understate how bad Morgan's position is.

CHARYBDIS:What's worse for a mortal than assured loss of money?

SCYLLA: Morgan is not a mortal any more, you know. And immortals we're
afraid are almost bound to lose money to clever enough tricksters.
Indeed, a so-called Dutch Book can be made against any agent that (a)
has an unbounded utility function and (b) is not overly opinionated, so
there are still infinitely many ways the world could be consistent with
their knowledge.\footnote{This is proven in McGee
  (\citeproc{ref-McGee1999}{1999}).} That includes us, and you dear
Charybdis. And yet we are not as irrational as that Morgan. I don't
think analogising her position to ours really \emph{strengthens} the
case that she is irrational.

CHARYBDIS: Next you might say that making money off her, this
undeserving immortal, is immoral.

SCYLLA: Perish the thoughts.

\section{Risky Business?}\label{risky-business}

There are two kinds of reasons to dislike indifference principles, both
of them developed most extensively in Keynes
(\citeproc{ref-Keynes1921}{1921}). The first, which we have been
exploring a bit so far, is that such principles tend to lead to
incoherence. The second is that such principles promote confusion
between risk and uncertainty.

Often we do not know exactly what the world is like. But not all kinds
of ignorance are alike. Sometimes, our ignorance is like that of a
roulette player facing a fair wheel about to be spun. She knows not what
will happen, but she can provide good reasons for assigning equal
credence to each of the 37 possible outcomes of the spin. Loosely
following Frank Knight (\citeproc{ref-Knight1921}{1921}), we will say
that a proposition like \emph{The ball lands in slot number 18} is
\textbf{risky}. The distinguishing feature of such propositions is that
we do not know whether they are true or false, but we have good reason
to assign a particular probability to their truth. Other propositions,
like say the proposition that there will be a nuclear attack on an
American city this century, are quite unlike this. We do not know
whether they are true, and we aren't really in a position to assign
anything like a precise numerical probability to their truth. Again
following Knight, we will say such propositions are \textbf{uncertain}.
In (\citeproc{ref-Keynes1937}{1937}) Keynes described a number of other
examples that nicely capture the distinction being drawn here.

\begin{quote}
By `uncertain' knowledge, let me explain, I do not mean merely to
distinguish what is known for certain from what is only probable. The
game of roulette is not subject, in this sense, to uncertainty; nor is
the prospect of a Victory bond being drawn. Or, again, the expectation
of life is only slightly uncertain. Even the weather is only moderately
uncertain. The sense in which I am using the term is that in which the
prospect of a European war is uncertain, or the price of copper and the
rate of interest twenty years hence, or the obsolescence of a new
invention, or the position of private wealth owners in the social system
in 1970. About these matters there is no scientific basis on which to
form any calculable probability whatever. We simply do not know.
Nevertheless, the necessity for action and decision compels us as
practical men to do our best to overlook this awkward fact and to behave
exactly as we should if we had behind us a good Benthamite calculation
of a series of prospective advantages and disadvantages, each multiplied
by its appropriate probability, waiting to be summed.
(\citeproc{ref-Keynes1937}{Keynes 1937, 114--15})
\end{quote}

Note that the distinction between risky and uncertain propositions is
not the distinction between propositions whose objective chance we know
and those that we don't. This identification would fail twice over.
First, as Keynes notes, whether a proposition is risky or uncertain is a
matter of degree, but whether we know something is, I presume, not a
matter of degree.\footnote{Though see Hetherington
  (\citeproc{ref-Hetherington2001}{2001}) for an argument to the
  contrary.} Second, there are risky propositions with an unknown
chance. Assume that our roulette player turns away from the table at a
crucial moment, and misses the ball landing in a particular slot. Now
the chance that it lands in slot 18 is 1 (if it did so land) or 0
(otherwise), and she does not know which. Yet typically, the proposition
\emph{The ball lands in slot 18} is still risky for her, for she has no
reason to change her attitude towards the proposition that it did land
in slot 18.

My primary theoretical objection to INDIFFERENCE is that the
propositions it purports to provide guidance on are really uncertain,
but it treats them as risky. Once we acknowledge the risk/uncertainty
distinction, it is natural to think that our default state is
uncertainty. Getting to a position where we can legitimately treat a
proposition as risky is a cognitive achievement. Traditional
indifference principles fail because they trivialise this achievement.
An extreme version of such a principle says we can justify assigning a
particular numerical probability, 0.5, to propositions merely on the
basis of ignorance of any evidence telling for or against it. This might
not be an issue to those who think that ``probability is a measure of
your ignorance.'' (\citeproc{ref-Poole1998}{Poole, Mackworth, and Goebel
1998, 348}) But to those of us who think probability is the very guide
to life, such a position is unacceptable. It seems to violate the
platitude `garbage in, garbage out' since it takes ignorance as input,
and produces a guide to life as output. INDIFFERENCE is more subtle than
these traditional indifference principles, but this theoretical
objection remains. The evidence that O'Leary or Morgan or Leslie has
does not warrant treating propositions about their location or identity
as risky rather than uncertain. When they must make decisions that turn
on their identity or location, this ignorance provides little or no
guidance, not a well-sharpened guide to action.

In this section I argue that treating these propositions as uncertain
lets us avoid the traps that Morgan falls into. In the next section I
argue that the case Elga takes to support INDIFFERENCE says nothing to
the theorist who thinks that the INDIFFERENCE principle conflates risk
and uncertainty. In fact, some features of that case seem to support the
claim that the propositions covered by INDIFFERENCE are uncertain, not
risky.

In (\citeproc{ref-Keynes1921}{1921}), Keynes put forward a theory of
probability that was designed to respect the distinction between risky
propositions and uncertain propositions. He allowed that some
propositions, the risky ones and the ones known to be true or false, had
a numerical probability (relative to a body of evidence) while other
propositions have non-numerical probabilities. Sometimes numerical and
non-numerical probabilities can be compared, sometimes they cannot.
Arithmetic operations are all assumed to be defined over both numerical
and non-numerical probabilities. As Ramsey
(\citeproc{ref-RamseyTruthProb}{1926}) pointed out, in Keynes's system
it is hard to know what \({\alpha}\) +~\({\beta}\) is supposed to mean
when \({\alpha}\) and \({\beta}\) are non-numerical probabilities, and
it is not even clear that `+' still means \emph{addition} in the sense
we are used to.

One popular modern view of probability can help Keynes out here.
Following Ramsey, many people came to the view that the credal states of
a rational agent could be represented by a probability function, that
function being intuitively the function from propositions into the
agent's degree of belief in that proposition. In the last thirty years,
there has been a lot of research on the theory that says we should
represent rational credal states not by a single probability function,
but by a set of such probability functions. Within philosophy, the most
important works on this theory are by Henry Kyburg
(\citeproc{ref-Kyburg1974}{1974}), Isaac (\citeproc{ref-Levi1974}{Levi
1974}, \citeproc{ref-Levi1980}{1980}), Richard Jeffrey
(\citeproc{ref-Jeffrey1983}{1983}) and Bas Fraassen
(\citeproc{ref-vanFraassen1990}{1990}). What is important here about
this theory is that many distinctive features of Keynes's theory are
reflected in it.

Let \emph{S} be the set of probability functions representing the credal
states of a rational agent. Then for each proposition \emph{p} we can
define a set \emph{S}(\emph{p}) =~\{\emph{Pr}(\emph{p}): \emph{Pr}
\({\in}\)~\emph{S}\}. That is, \emph{S}(\emph{p}) is the set of values
that \emph{Pr}(\emph{p}) takes for \emph{Pr} being a probability
function in \emph{S}. We will assume here that \emph{S}(\emph{p}) is an
interval. (See the earlier works cited for the arguments in favour of
this assumption.) When \emph{p} is risky, \emph{S}(\emph{p}) will be a
singleton, the singleton of the number we have compelling reason to say
is the probability of \emph{p}. When \emph{p} is a little uncertain,
\emph{S}(\emph{p}) will be a fairly narrow interval. When it is very
uncertain, \emph{S}(\emph{p}) will be a wide interval, perhaps as wide
as {[}0,~1{]}. We say that \emph{p} is more probable than \emph{q} iff
for all \emph{Pr} in \emph{S}, \emph{Pr}(\emph{p})~\textgreater{}
\emph{Pr}(\emph{q}), and as probable as \emph{q} iff for \emph{Pr} in
\emph{S}, \emph{Pr}(\emph{p})~= \emph{Pr}(\emph{q}). This leaves open
the possibility that Keynes explicitly left open, that for some
uncertain proposition \emph{p} and some risky proposition \emph{q}, it
might be the case that they are not equally probable, but neither is one
more probable than the other. Finally, we assume that when an agent
whose credal states are represented by \emph{S} updates by learning
evidence \emph{e}, her new credal states are updated by conditionalising
each of the probability functions in \emph{S} on \emph{e}. So we can
sensibly talk about \emph{S}(\emph{p}~~\emph{e}), the set
\{\emph{Pr}(\emph{p} ~\emph{e}): \emph{Pr} \({\in}\)~\emph{S}\}, and
this represents her credal states on learning \emph{e}.

(It is an interesting historical question just how much the theory
sketched here agrees with the philosophical motivations of Keynes's
theory. One may think that the agreement is very close. If we take
Keynes's entire book to be a contextual definition of his non-numerical
probabilities, a reading encouraged by Lewis
(\citeproc{ref-Lewis1970c}{1970}), then we should conclude he was
talking about sets like this, with numerical probabilities being
singleton sets.)

This gives us the resources to provide good advice to Morgan. Pick a
monotone increasing function \emph{f} from integers to {[}0,~1{]} such
that as \emph{n} \({\rightarrow}\) \({\infty}\),
\emph{f}(\emph{n})~\({\rightarrow}\)~1. It won't really matter which
function you pick, though different choices of \emph{f} might make the
following story more plausible. Say that
\emph{S}(\emph{L}~\({\geq}\)~\emph{R}~~\emph{L}~=~\emph{l})~=~{[}0,~\emph{f}(\emph{l}){]}.
The rough idea is that if \emph{L} is small, then it is quite improbable
that \emph{L} ~\({\geq}\)~\emph{R}, although this is a little uncertain.
As \emph{l} gets larger, \emph{L}~\({\geq}\)~\emph{R} gets more and more
uncertain. The overall effect is that we simply do not know what
\emph{S}(\emph{L}~\({\geq}\)~\emph{R}) will look like after
conditionalising on the value of \emph{L}, so we cannot apply the kind
of reasoning Morgan uses to now come to some conclusions about the
probability of \emph{L}~\({\geq}\)~\emph{R}.

If we view the situations described by INDIFFERENCE as involving
uncertainty rather than risk, this is exactly what we should expect. And
note that in so doing, we need not undermine the symmetry intuition that
lies behind INDIFFERENCE. Assume that \emph{F} and \emph{G} are similar
predicaments, and I know that I am either \emph{F} or \emph{G}.
INDIFFERENCE says I should assign equal probability to each, so
\emph{S}(I am \emph{F})~=~\emph{S}(I am \emph{G})~= \{0.5\}. But once
we've seen how attractive non-numerical probabilities can be, we should
conclude that all symmetry gives us is that \emph{S}(I am
\emph{F})~=~\emph{S}(I am \emph{G}), which can be satisfied if each is
{[}0.4, 0.6{]}, or {[}0.2, 0.8{]} or even {[}0,~1{]}. (I think that for
O'Leary, for example, \emph{S}(It is 1 o'clock) should be a set somehow
like this.) Since I would \emph{not} be assigning equal credence to
\emph{I am F} and \emph{I am G} if I satisfied symmetry using
non-numerical probabilities, so I will violate INDIFFERENCE without
treating the propositions asymmetrically. Such a symmetric violation of
INDIFFERENCE has much to recommend it. It avoids the incoherence that
INDIFFERENCE leads to in Morgan's case. And it avoids saying that
ignorance about our identity can be a sharp guide to life.\footnote{Bradley
  Monton (\citeproc{ref-Monton2002}{2002}) discusses using sets of
  probability functions to solve another problem proposed by Elga, the
  Sleeping Beauty problem (\citeproc{ref-Elga2000-ELGSBA}{Elga 2000}).
  Monton notes that if Beauty's credence in \emph{The coin landed heads}
  is {[}0,~0.5{]} when she wakes up on Monday, then she doesn't violate
  van Fraassen's General Reflection Principle
  (\citeproc{ref-vanFraassen1995}{Fraassen 1995}). (I assume here
  familiarity with the Sleeping Beauty problem.) Monton has some
  criticisms of this move, in particular the consequences it has for
  updating, that don't seem to carry across to the proposal sketched
  here. But his discussion is noteworthy as a use of this approach to
  uncertainty as a way to solve problems to do with similar
  predicaments.}

A referee noted that the intuitive characterisation here doesn't quite
capture the idea that we should treat similar predicaments alike. The
requirement that if \emph{F} and \emph{G} are similar then \emph{S}(I am
\emph{F})~=~\emph{S}(I am \emph{G}) does not imply that there will be a
symmetric treatment of \emph{F} and \emph{G} within \emph{S} if there
are more than two similar predicaments. What we need is the following
condition. Let \emph{T} be any set of similar predicaments, \emph{g} any
isomorphism from \emph{T} onto itself, and \emph{Pr} any probability
function in \emph{S}. Then there exists a \emph{Pr}′ in \emph{S} such
that for all \emph{A} in \emph{T},
\emph{Pr}(\emph{A})~=~\emph{Pr}′(\emph{g}(\emph{A})). When there are
only two similar predicaments \emph{A} and \emph{B} this is equivalent
to the requirement that \emph{S}(\emph{A})~=~\emph{S}(\emph{B}), but in
the general case it is a much stricter requirement. Still, it is a much
weaker constraint than INDIFFERENCE, and not vulnerable to the
criticisms of INDIFFERENCE set out here.

\section{Boyfriend in a Coma}\label{boyfriend-in-a-coma}

Elga argues for INDIFFERENCE by arguing it holds in a special case, and
then arguing that the special case is effectively arbitrary, so if it
holds there it holds everywhere. The second step is correct, so we must
look seriously at the first step. Elga's conclusions about the special
case, DUPLICATION, eventually rest on treating an uncertain proposition
as risky.

\begin{description}
\tightlist
\item[DUPLICATION]
After Al goes to sleep researchers create a duplicate of him in a
duplicate environment. The next morning, Al and the duplicate awaken in
subjectively indistinguishable states.
\end{description}

Assume (in all these cases) that before Al goes to sleep he knows the
relevant facts of the case. In that case INDIFFERENCE\footnote{As with
  earlier cases, strictly speaking we need C-INDIFFERENCE and
  P-INDIFFERENCE to draw the conclusions suggested unless Al is somehow
  certain about all other propositions. I will ignore that complication
  here, and in .} dictates that when Al wakes up his credence in \emph{I
am Al} should be 0.5. Elga argues this dictate is appropriate by
considering a pair of related cases.

\begin{description}
\tightlist
\item[TOSS-and-DUPLICATION]
After Al goes to sleep, researchers toss a coin that has a 10\% chance
of landing heads. Then (regardless of the toss outcome) they duplicate
Al. The next morning, Al and the duplicate awaken in subjectively
indistinguishable states.
\end{description}

Elga notes, correctly, that the same epistemic norms apply to Al on
waking in DUPLICATION as in TOSS-and-DUPLICATION. So if we can show that
when Al wakes in TOSS-and-DUPLICATION his credence in \emph{I am Al}
should be 0.5, that too will suffice to prove INDIFFERENCE correct in
this case. The argument for that claim has three premises. (I've
slightly relabeled the premises for ease of expression.)

\begin{description}
\tightlist
\item[(1)]
\emph{Pr}(H) = 0.1
\item[(2)]
\emph{Pr}(H (H \({\wedge}\) A) \({\vee}\) (T \({\wedge}\) A)) = 0.1
\item[(3)]
\emph{Pr}(H (H \({\wedge}\) A) \({\vee}\) (T \({\wedge}\) D)) = 0.1
\end{description}

Here \emph{Pr} is the function from \emph{de se} propositions to Al's
degree of belief in them, H = \emph{The coin lands heads}, T = \emph{The
coin lands tails}, A = \emph{I am Al} and D = \emph{I am Al's
duplicate}. From (1), (2) and (3) and the assumption that \emph{Pr} is a
probability function it follows that \emph{Pr}(\emph{A}) = 0.5, as
required. This inference goes through even in the Keynesian theory that
distinguishes risk from uncertainty. Premise (1) is uncontroversial, but
both (2) and (3) look dubious. Since the argument for (3) would, if
successful, support (2), I'll focus, as Elga does, on (3). The argument
for it turns on \emph{another} case.

\begin{description}
\tightlist
\item[COMA]
As in TOSS-and-DUPLICATION, the experimenters toss a coin and duplicate
Al. But the following morning, the experimenters ensure that only one
person wakes up: If the coin lands heads, they allow Al to wake up (and
put the duplicate into a coma); if the coin lands tails, they allow the
duplicate to wake up (and put Al into a coma).
\end{description}

(It's important that no one comes out of this coma, so assume that the
victim gets strangled.)

Elga then argues for the following two claims. If in COMA Al gets lucky
and pulls through, his credence in H should be 0.1, as it was before he
entered the dream world. Al's credence in H in COMA should be the same
as his conditional credence in H should be the same as his conditional
credence in H given (H~\({\wedge}\)~A)~\({\vee}\)~(T~\({\wedge}\)~D) in
TOSS-and-DUPLICATION. The second premise looks right, so the interest is
on what happens in COMA. Elga argues as follows (notation slightly
changed):

\begin{quote}
Before Al was put to sleep, he was sure that the chance of the coin
landing heads was 10\%, and his credence in H should have accorded with
this chance: it too should have been 10\%. When he wakes up, his
epistemic situation with respect to the coin is just the same as it was
before he went to sleep. He has neither gained nor lost information
relevant to the toss outcome. So his degree of belief in H should
continue to accord with the chance of H at the time of the toss. In
other words, his degree of belief in H should continue to be 10\%.
\end{quote}

And this, I think, is entirely mistaken. Al has no evidence that his
evidence is relevant to H, but absence of evidence is not evidence of
absence. Four considerations support this conclusion.

First, Al gets some evidence of some kind or other on waking. Certain
colours are seen, certain pains and sensations are sensed, certain
fleeting thoughts fleet across his mind. Before he sleeps Al doesn't
knows what these shall be. Maybe he thinks of the money supply, maybe of
his girlfriend, maybe of his heroine, maybe of kidneys. He doesn't
\emph{know} that the occurrence of these thoughts is probabilistically
independent of his being Al rather than Dup, so he does not \emph{know}
they are probabilistically independent of H. So perhaps he need not
retain the credence in H he has before he was drugged. Even if this
evidence looks like junk, we can't rule out that it has some force.

Secondly, the kind of internalism about evidence needed to support
Elga's position is remarkably strong. (This is where the concerns raised
in Section~\ref{sec-internalism} become most pressing.) Elga notes that
he sets himself against both an extreme externalist position that says
that Al's memories and/or perceptions \emph{entail} that he is Al and
against an ``intermediate view, according to which Al's beliefs about
the setup only partially undermine his memories of being Al. According
to such a view, when Al wakes up his credence in H ought to be slightly
higher than 10\%.'' But matters are worse than that. Elga must also
reject an even weaker view that says that Al might not know whether
externalism about evidence is true, so he does not know whether his
credence in H should change. My view is more sympathetic to that
position. When Al wakes, he does not know which direction is credences
should move, or indeed whether there is such a direction, so his
credence in H should be a spread of values including 0.1.

Thirdly, Al's position looks like cases where new evidence makes risky
propositions uncertain. Mack's betting strategy for the Gold Cup, a
horse race with six entrants, is fairly simple. He rolls a fair die, and
bets on whatever number comes up. Jane knows this is Mack's strategy,
but does not how the die landed this time. Nor does she know anything
about horses, so the propositions \emph{Horse n wins the Gold Cup} are
uncertain for Jane for each \emph{n}. Call these propositions
\emph{w\textsubscript{n}}, and the proposition that Mack's die landed
\emph{n} \emph{d\textsubscript{n}}. Right now, \emph{d}\textsubscript{2}
is risky, but \emph{h}\textsubscript{2} is uncertain. Jane hears a party
starting next door. Mack's won. Jane has learned, \emph{inter alia},
\emph{d}\textsubscript{2}~\(\leftrightarrow\)~\emph{h}\textsubscript{2}.
Now it seems that \emph{d}\textsubscript{2}, \emph{Mack's die landed 2},
inherits the uncertainty of \emph{h}\textsubscript{2}, \emph{Horse
number 2 won the Gold Cup}. The formal theory of uncertainty I sketched
allows for this possibility. It is possible that there be \emph{p},
\emph{e} such that \emph{S}(\emph{p}) is a singleton, while
\emph{S}(\emph{p}~~\emph{e}) is a wide interval, in theory as wide as
{[}0,~1{]}. This is what happens in Jane's case, and it looks like it
happens in Al's case too. H used to be risky, but when he wakes he comes
to learn H~\({\leftrightarrow}\)~A, just as Jane learned
\emph{d}\textsubscript{2}~\(\leftrightarrow\)~\emph{h}\textsubscript{2}.
In each case, the left-hand clause of the biconditional inherits the
uncertainty of the right-hand clause.

Finally, H being uncertain for Al when he wakes in COMA is consistent
with the intuition that Al has no reason to change his credences in H in
one direction or another when he says goodbye to his duplicate. (Or, for
all he knows, to his source.) Perhaps externalist theories of evidence
provide some reason to raise these credences, as suggested above, but I
do not \emph{rely} on such theories. What I deny is that the absence of
a reason to move one way or the other is a reason to stay put. Al's
credence in H might change in a way that reflects the fact H is now
uncertain, just like A is in COMA, just like A is in
TOSS-and-DUPLICATION, and, importantly, just like A is in DUPLICATION. I
think the rest of Elga's argument is right. DUPLICATION is a perfectly
general case. In any such case, Al should be uncertain, in Keynes's
sense, whether he is the original or the duplicate.

\section{Shooting Dice can be
Dangerous}\label{shooting-dice-can-be-dangerous}

The good news, said Dr Evil, is that you are still mortal. Odysseus was
not as upset as Dr Evil had expected. The bad news is that I'm thinking
of torturing you. I'm going to roll this fair die, and if it lands 6 you
will be tortured. If it does not, you will be (tentatively) released,
and I'll create two duplicates of you as you were when you entered this
room, repeat this story to both them. Depending on another roll of this
fair die, I will either torture them both, or create two duplicates of
each of them, and repeat the process until I get to torture
someone.\footnote{Dr Evil's plans create a situation similar to the well
  known `shooting room' problem. For the best analysis of that problem
  see Bartha and Hitchcock (\citeproc{ref-Bartha1999}{1999}). Dr Evil
  has changed the numbers involved in the puzzle a little bit to make
  the subsequent calculations a little more straightforward. He's not
  very good at arithmetic you see.}

Odysseus thought through this for a bit. So I might be a duplicate
you've just created, he said. I might not be Odysseus.

You might not be, said Dr Evil, although so as to avoid confusion if
you're not him I'll use his name for you.

What happens if the die never lands 6, asked Odysseus. I've seen some
odd runs of chance in my time.

I wouldn't be so sure of that, said Dr Evil. Anyway, that's why I said I
would tentatively release you. I'll make the die rolls and subsequent
duplication quicker and quicker so we'll get through the infinite number
of rolls in a finite amount of time. If we get that far I'll just bring
everyone back and torture you all. Aren't I fair?

Fairness wasn't on Odysseus's mind though. He was trying to figure out
how likely it was that he would be tortured. He was also a little
concerned about how likely it was that he was the original Odysseus, and
if he was not whether Penelope too had been duplicated. As it turns out,
his torturous computations would assist with the second question, though
not the third. Two thoughts crossed his mind.

I will be tortured if that die lands 6, which has a chance of 1 in 6, or
if it never lands 6 again, which has a chance of 0. So the chance of my
being tortured is 1 in 6. I have no inadmissible evidence, so the
probability I should assign to torture is 1 in 6.

Let's think about how many Odysseuses there are in the history of the
world. Either there is 1, in which case I'm him, and I shall be
tortured. Or there are 3, in which case two of them shall be tortured,
so the probability that I shall be tortured is 2 in 3. Or there are 7,
in which case four of them shall be tortured, so the probability that I
shall be tortured is 4 in 7. And so on, it seems like the probability
that I shall be tortured approaches 1 in 2 from above as the number of
Odysseuses approaches infinity. Except, of course, in the case where it
reaches infinity, when it is again certain that I shall be tortured. So
it looks like the probability that I will be tortured is above 1 in 2.
But I just concluded it is 1 in 6. Where did I go wrong?

In his second thought, Odysseus appeals frequently to INDIFFERENCE. He
then appeals to something like the conglomerability principle that
tripped up Morgan. The principle Odysseus uses is a little stronger than
the principle Morgan used. It says that if there is a partition and
conditional on each member of the partition, the probability of \emph{p}
is greater than \emph{x}, then the probability of \emph{p} is greater
than \emph{x}. As we noted, this principle cannot be accepted in its
full generality by one who rejects countable additivity. And one who
accepts INDIFFERENCE must reject countable additivity. So where Odysseus
goes wrong is in appealing to this inference principle after previously
adopting an indifference principle inconsistent with it.

This does not mean the case has no interest. Morgan's case showed that
when we have an actual infinity of duplicates, INDIFFERENCE can lead to
counterintuitive results, and that the best way out might be to say that
Morgan faced a situation of uncertainty, not one of risk. But it might
have been thought that something special about Morgan's case, that she
has infinitely many duplicates, might be responsible for the problems
here. So it may be hoped that INDIFFERENCE can at least be accepted in
more everyday cases. Odysseus shows that hope is in vain. All we need is
the merest possibility of there being infinitely many duplicates, here a
possibility with zero probability, to create a failure of
conglomerability. This suggests that the problems with INDIFFERENCE run
relatively deep.

The details of how Odysseus's case plays out given INDIFFERENCE are also
interesting, especially to those readers not convinced by my refutation
of INDIFFERENCE. For their benefit, I will close with a few observations
about how the case plays out.

As in Morgan's case, we can produce two different partitions of the
possibility space that \emph{seem} to support different conclusions
about Odysseus's prospects. Assume for convenience that Dr Evil makes a
serial number for each Odysseus he makes, the Homeric hero being number
1, the first two duplicates being 2 and 3, and so on. Let \emph{N} stand
for the number of our hero, \emph{M} for the number of Odysseuses that
are made, and \emph{T} for the property of being tortured. Then given
INDIFFERENCE it behoves Odysseus to have his credences governed by the
following \emph{Pr} function.

\begin{description}
\tightlist
\item[(4a)]
\({\forall}\)\emph{k}
\emph{Pr}(\emph{T}~~\emph{M}~=~2\textsuperscript{\emph{k}} - 1) =
2\textsuperscript{\emph{k}-1}/(2\textsuperscript{\emph{k}}~-~1)
\item[(4b)]
\emph{Pr}(\emph{T}~~\emph{M}~=~\({\infty}\)) = 1
\item[(5)]
\({\forall}\)\emph{n}~\emph{Pr}(\emph{T}~~\emph{N}~=~\emph{n}) = 1/6
\end{description}

Between 4a and 4b we cover all possible values for \emph{M}, and in
every case \emph{Pr}(\emph{T}) is greater than 1/2. More interesting are
Odysseus's calculations about whether he is the Homeric hero, i.e.~about
whether \emph{N}~=~1. Consider first a special case of this, what the
value of \emph{Pr}(\emph{N}~=~1 \emph{N}~\textless{} 8) is. At first
glance, it might seem that this should be 1/7, because there are seven
possible values for \emph{N} less than 8. But this is too quick. There
are really eleven possibilities to be considered.

\begin{longtable}[]{@{}
  >{\raggedright\arraybackslash}p{(\columnwidth - 4\tabcolsep) * \real{0.3258}}
  >{\raggedright\arraybackslash}p{(\columnwidth - 4\tabcolsep) * \real{0.3258}}
  >{\raggedright\arraybackslash}p{(\columnwidth - 4\tabcolsep) * \real{0.3483}}@{}}
\toprule\noalign{}
\endhead
\bottomrule\noalign{}
\endlastfoot
\emph{F}\textsubscript{1}: \emph{N}~=~1 and \emph{M} = 1 &
\emph{F}\textsubscript{2}: \emph{N}~= 1 and \emph{M} = 3 &
\emph{F}\textsubscript{5}: \emph{N}~= 1 and \emph{M} \textgreater{} 3 \\
~ & \emph{F}\textsubscript{3}: \emph{N}~= 2 and \emph{M} = 3 &
\emph{F}\textsubscript{6}: \emph{N}~= 2 and \emph{M} \textgreater{} 3 \\
~ & \emph{F}\textsubscript{4}: \emph{N}~= 3 and \emph{M} = 3 &
\emph{F}\textsubscript{7}: \emph{N}~= 3 and \emph{M} \textgreater{} 3 \\
~ & ~ & \emph{F}\textsubscript{8}: \emph{N}~= 4 and \emph{M}
\textgreater{} 3 \\
~ & ~ & \emph{F}\textsubscript{9}: \emph{N}~= 5 and \emph{M}
\textgreater{} 3 \\
~ & ~ & \emph{F}\textsubscript{10}: \emph{N}~= 6 and \emph{M}
\textgreater{} 3 \\
~ & ~ & \emph{F}\textsubscript{11}: \emph{N}~= 7 and \emph{M}
\textgreater{} 3 \\
\end{longtable}

By INDIFFERENCE, each of the properties in each column should be given
equal probability. So we have

\[
\begin{aligned}
x &= Pr(F_1 | N < 8)  \\
y &= Pr(F_2 | N < 8) = Pr(F_3 | N < 8) = Pr(F_4 | N < 8)  \\
z &= Pr(F_5 | N < 8) = \dots = Pr(F_11 | N < 8)  
\end{aligned}
\]

We just have to solve for \emph{x}, \emph{y} and \emph{z}. By the
Principal Principle we get

\begin{enumerate}
\def\labelenumi{\arabic{enumi}.}
\setcounter{enumi}{5}
\item
  \emph{Pr}(\emph{M}~=~1~~\emph{N}~=~1) = 1/6\\
  \({\therefore}\) \emph{x} = (\emph{x}~+~\emph{y}~+~\emph{z}) / 6
\item
  \emph{Pr}(\emph{M}~= 3 \emph{N}~= 1 and \emph{M}~\({\geq}\) 3) = 1/6\\
  \({\therefore}\) \emph{y}~= (\emph{y}~+~\emph{z}) / 6
\end{enumerate}

And since these 11 possibilities are all the possibilities for
\emph{N}~\textless{} 8, we have

\begin{enumerate}
\def\labelenumi{\arabic{enumi}.}
\setcounter{enumi}{7}
\tightlist
\item
  \emph{x} + 3\emph{y} + 7\emph{z} = 1
\end{enumerate}

Solving for all these, we get \emph{x} = 3/98, \emph{y} = 5/196 and
\emph{z} = 25/196, so \emph{Pr}(\emph{N}~=~1~ \emph{N}~\textless~8) =
\emph{x}~+~\emph{y}~+~\emph{z}~= 9/49. More generally, we have the
following (the proof of this is omitted):
\[Pr(N = 1 | N < 2^{k+1}) = \frac{6^k}{\sum_{i=0}^{k}6^i10^{k-i}}\]

Since the RHS \({\rightarrow}\) 0 as \emph{k}~\({\rightarrow}\)
\({\infty}\), \emph{Pr}(\emph{N}~=~1) = 0. Our Odysseus is probably not
the real hero. Similar reasoning shows that
\emph{Pr}(\emph{N}~=~\emph{n}) = 0 for all \emph{n}. So we have another
violation of countable additivity. But we do not have, as in Morgan's
case, a constant distribution across the natural numbers. In a sense,
this distribution is still weighted towards the bottom, since for any
\emph{n}~\textgreater{} 1, \emph{Pr}(\emph{N}~=~1~~\emph{N}~=~1
\({\vee}\)~\emph{N}~=~\emph{n})~\textgreater{} 1/2. Of course, I don't
think INDIFFERENCE is true, so these facts about what Odysseus's
credence function will look like under INDIFFERENCE are of purely
mathematical interest to me. But it might be possible that someone more
enamoured of INDIFFERENCE can use this `unbalanced' distribution to
explain some of the distinctive features of the odd position that
Odysseus is in.\footnote{Thanks to Jamie Dreier, Adam Elga and an
  anonymous referee for helpful discussions about this paper and
  suggestions for improvements.}

\subsection*{References}\label{references}
\addcontentsline{toc}{subsection}{References}

\phantomsection\label{refs}
\begin{CSLReferences}{1}{0}
\bibitem[\citeproctext]{ref-Bartha1999}
Bartha, Paul, and Christopher Hitchcock. 1999. {``The Shooting-Room
Paradox and Conditionalizing on Measurably Challenged Sets.''}
\emph{Synthese} 118 (3): 403--37. doi:
\href{https://doi.org/10.1023/a:1005100407551}{10.1023/a:1005100407551}.

\bibitem[\citeproctext]{ref-Campbell2002}
Campbell, John. 2002. \emph{Reference and Consciousness}. Oxford: Oxford
University Press.

\bibitem[\citeproctext]{ref-ChambersQuiggin2000}
Chambers, Robert, and John Quiggin. 2000. \emph{Uncertainty, Production,
Choice, and Agency: The State-Contingent Approach}. Cambridge: Cambridge
University Press.

\bibitem[\citeproctext]{ref-Dreier2001}
Dreier, James. 2001. {``Boundless Good.''}

\bibitem[\citeproctext]{ref-Elga2000-ELGSBA}
Elga, Adam. 2000. {``Self-Locating Belief and the Sleeping Beauty
Problem.''} \emph{Analysis} 60 (2): 143--47. doi:
\href{https://doi.org/10.1093/analys/60.2.143}{10.1093/analys/60.2.143}.

\bibitem[\citeproctext]{ref-Elga2004}
---------. 2004. {``Defeating Dr. Evil with Self-Locating Belief.''}
\emph{Philosophy and Phenomenological Research} 69 (2): 383--96. doi:
\href{https://doi.org/10.1111/j.1933-1592.2004.tb00400.x}{10.1111/j.1933-1592.2004.tb00400.x}.

\bibitem[\citeproctext]{ref-Fara2001}
Fara, Delia Graff. 2001. {``Phenomenal Continua and the Sorites.''}
\emph{Mind} 110 (440): 905--36. doi:
\href{https://doi.org/10.1093/mind/110.440.905}{10.1093/mind/110.440.905}.
This paper was first published under the name {``Delia Graff.''}

\bibitem[\citeproctext]{ref-DeFinetti1974}
Finetti, Bruno de. 1974. \emph{Theory of Probability}. New York: Wiley.

\bibitem[\citeproctext]{ref-vanFraassen1990}
Fraassen, Bas van. 1990. {``Figures in a Probability Landscape.''} In
\emph{Truth or Consequences}, edited by J. M. Dunn and A. Gupta,
345--56. Amsterdam: Kluwer.

\bibitem[\citeproctext]{ref-vanFraassen1995}
---------. 1995. {``Belief and the Problem of Ulysses and the Sirens.''}
\emph{Philosophical Studies} 77 (1): 7--37. doi:
\href{https://doi.org/10.1007/bf00996309}{10.1007/bf00996309}.

\bibitem[\citeproctext]{ref-Goodman1951}
Goodman, Nelson. 1951. \emph{The Structure of Appearance}. Cambridge,
MA: Harvard University Press.

\bibitem[\citeproctext]{ref-Hajek2005}
Hájek, Alan. 2005. {``Scotching Dutch Books.''} \emph{Philosophical
Perspectives} 19: 139--51. doi:
\href{https://doi.org/10.1111/j.1520-8583.2005.00057.x}{10.1111/j.1520-8583.2005.00057.x}.

\bibitem[\citeproctext]{ref-Hetherington2001}
Hetherington, Stephen. 2001. \emph{Good Knowledge, Bad Knowledge: On Two
Dogmas of Epistemology}. Oxford: Oxford University Press.

\bibitem[\citeproctext]{ref-Jeffrey1983}
Jeffrey, Richard. 1983. {``Bayesianism with a Human Face.''} In
\emph{Testing Scientific Theories}, edited by J. Earman (ed.).
Minneapolis: University of Minnesota Press.

\bibitem[\citeproctext]{ref-Keynes1921}
Keynes, John Maynard. 1921. \emph{Treatise on Probability}. London:
Macmillan.

\bibitem[\citeproctext]{ref-Keynes1937}
---------. 1937. {``The General Theory of Employment.''} \emph{Quarterly
Journal of Economics} 51 (2): 209--23. doi:
\href{https://doi.org/10.2307/1882087}{10.2307/1882087}. Reprinted in
\cite[XIV 109-123]{KeynesCW}, references to reprint.

\bibitem[\citeproctext]{ref-Knight1921}
Knight, Frank. 1921. \emph{Risk, Uncertainty and Profit}. Chicago:
University of Chicago Press.

\bibitem[\citeproctext]{ref-Kyburg1974}
Kyburg, Henry. 1974. \emph{The Logical Foundations of Statistical
Inference}. Dordrecht: Reidel.

\bibitem[\citeproctext]{ref-Levi1974}
Levi, Isaac. 1974. {``On Indeterminate Probabilities.''} \emph{Journal
of Philosophy} 71 (13): 391--418. doi:
\href{https://doi.org/10.2307/2025161}{10.2307/2025161}.

\bibitem[\citeproctext]{ref-Levi1980}
---------. 1980. \emph{The Enterprise of Knowledge}. Cambridge, MA.: MIT
Press.

\bibitem[\citeproctext]{ref-Lewis1970c}
Lewis, David. 1970. {``How to Define Theoretical Terms.''} \emph{Journal
of Philosophy} 67 (13): 427--46. doi:
\href{https://doi.org/10.2307/2023861}{10.2307/2023861}. Reprinted in
his \emph{Philosophical Papers}, Volume 1, Oxford: Oxford University
Press, 1983, 78-95. References to reprint.

\bibitem[\citeproctext]{ref-Lewis1979b}
---------. 1979. {``Attitudes \emph{de Dicto} and \emph{de Se}.''}
\emph{Philosophical Review} 88 (4): 513--43. doi:
\href{https://doi.org/10.2307/2184646}{10.2307/2184646}. Reprinted in
his \emph{Philosophical Papers}, Volume 1, Oxford: Oxford University
Press, 1983, 133-156. References to reprint.

\bibitem[\citeproctext]{ref-McGee1999}
McGee, Vann. 1999. {``An Airtight Dutch Book.''} \emph{Analysis} 59 (4):
257--65. doi:
\href{https://doi.org/10.1093/analys/59.4.257}{10.1093/analys/59.4.257}.

\bibitem[\citeproctext]{ref-Milne1991}
Milne, Peter. 1991. {``Scotching the Dutch Book Argument.''}
\emph{Erkenntnis} 32 (1): 105--26. doi:
\href{https://doi.org/10.1007/bf00209558}{10.1007/bf00209558}.

\bibitem[\citeproctext]{ref-Monton2002}
Monton, Bradley. 2002. {``Sleeping Beauty and the Forgetful Bayesian.''}
\emph{Analysis} 62 (1): 47--53. doi:
\href{https://doi.org/10.1093/analys/62.1.47}{10.1093/analys/62.1.47}.

\bibitem[\citeproctext]{ref-Poole1998}
Poole, David, Alan Mackworth, and Randy Goebel. 1998.
\emph{Computational Intelligence: A Logical Approach}. Oxford: Oxford
University Press.

\bibitem[\citeproctext]{ref-RamseyTruthProb}
Ramsey, Frank. 1926. {``Truth and Probability.''} In \emph{Philosophical
Papers}, edited by D. H. Mellor, 52--94. Cambridge: Cambridge University
Press.

\bibitem[\citeproctext]{ref-Schick1986}
Schick, Frederick. 1986. {``Dutch Bookies and Money Pumps.''}
\emph{Journal of Philosophy} 83 (2): 112--19. doi:
\href{https://doi.org/10.2307/2026054}{10.2307/2026054}.

\bibitem[\citeproctext]{ref-Shafer1976}
Shafer, Glenn. 1976. \emph{A Mathematical Theory of Evidence}.
Princeton: Princeton University Press.

\bibitem[\citeproctext]{ref-Williamson1998-WILCOK}
Williamson, Timothy. 1998. {``{Conditionalizing on Knowledge}.''}
\emph{British Journal for the Philosophy of Science} 49 (1): 89--121.
doi: \href{https://doi.org/10.1093/bjps/49.1.89}{10.1093/bjps/49.1.89}.

\end{CSLReferences}



\noindent Published in\emph{
Philosophy and Phenomenological Research}, 2005, pp. 613-635.

\end{document}
