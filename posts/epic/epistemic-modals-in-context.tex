% Options for packages loaded elsewhere
\PassOptionsToPackage{unicode}{hyperref}
\PassOptionsToPackage{hyphens}{url}
%
\documentclass[
  10pt,
  letterpaper,
  DIV=11,
  numbers=noendperiod,
  twoside]{scrartcl}

\usepackage{amsmath,amssymb}
\usepackage{setspace}
\usepackage{iftex}
\ifPDFTeX
  \usepackage[T1]{fontenc}
  \usepackage[utf8]{inputenc}
  \usepackage{textcomp} % provide euro and other symbols
\else % if luatex or xetex
  \usepackage{unicode-math}
  \defaultfontfeatures{Scale=MatchLowercase}
  \defaultfontfeatures[\rmfamily]{Ligatures=TeX,Scale=1}
\fi
\usepackage{lmodern}
\ifPDFTeX\else  
    % xetex/luatex font selection
  \setmainfont[ItalicFont=EB Garamond Italic,BoldFont=EB Garamond
Bold]{EB Garamond Math}
  \setsansfont[]{Europa-Bold}
  \setmathfont[]{Garamond-Math}
\fi
% Use upquote if available, for straight quotes in verbatim environments
\IfFileExists{upquote.sty}{\usepackage{upquote}}{}
\IfFileExists{microtype.sty}{% use microtype if available
  \usepackage[]{microtype}
  \UseMicrotypeSet[protrusion]{basicmath} % disable protrusion for tt fonts
}{}
\usepackage{xcolor}
\usepackage[left=1in, right=1in, top=0.8in, bottom=0.8in,
paperheight=9.5in, paperwidth=6.5in, includemp=TRUE, marginparwidth=0in,
marginparsep=0in]{geometry}
\setlength{\emergencystretch}{3em} % prevent overfull lines
\setcounter{secnumdepth}{3}
% Make \paragraph and \subparagraph free-standing
\ifx\paragraph\undefined\else
  \let\oldparagraph\paragraph
  \renewcommand{\paragraph}[1]{\oldparagraph{#1}\mbox{}}
\fi
\ifx\subparagraph\undefined\else
  \let\oldsubparagraph\subparagraph
  \renewcommand{\subparagraph}[1]{\oldsubparagraph{#1}\mbox{}}
\fi


\providecommand{\tightlist}{%
  \setlength{\itemsep}{0pt}\setlength{\parskip}{0pt}}\usepackage{longtable,booktabs,array}
\usepackage{calc} % for calculating minipage widths
% Correct order of tables after \paragraph or \subparagraph
\usepackage{etoolbox}
\makeatletter
\patchcmd\longtable{\par}{\if@noskipsec\mbox{}\fi\par}{}{}
\makeatother
% Allow footnotes in longtable head/foot
\IfFileExists{footnotehyper.sty}{\usepackage{footnotehyper}}{\usepackage{footnote}}
\makesavenoteenv{longtable}
\usepackage{graphicx}
\makeatletter
\def\maxwidth{\ifdim\Gin@nat@width>\linewidth\linewidth\else\Gin@nat@width\fi}
\def\maxheight{\ifdim\Gin@nat@height>\textheight\textheight\else\Gin@nat@height\fi}
\makeatother
% Scale images if necessary, so that they will not overflow the page
% margins by default, and it is still possible to overwrite the defaults
% using explicit options in \includegraphics[width, height, ...]{}
\setkeys{Gin}{width=\maxwidth,height=\maxheight,keepaspectratio}
% Set default figure placement to htbp
\makeatletter
\def\fps@figure{htbp}
\makeatother
% definitions for citeproc citations
\NewDocumentCommand\citeproctext{}{}
\NewDocumentCommand\citeproc{mm}{%
  \begingroup\def\citeproctext{#2}\cite{#1}\endgroup}
\makeatletter
 % allow citations to break across lines
 \let\@cite@ofmt\@firstofone
 % avoid brackets around text for \cite:
 \def\@biblabel#1{}
 \def\@cite#1#2{{#1\if@tempswa , #2\fi}}
\makeatother
\newlength{\cslhangindent}
\setlength{\cslhangindent}{1.5em}
\newlength{\csllabelwidth}
\setlength{\csllabelwidth}{3em}
\newenvironment{CSLReferences}[2] % #1 hanging-indent, #2 entry-spacing
 {\begin{list}{}{%
  \setlength{\itemindent}{0pt}
  \setlength{\leftmargin}{0pt}
  \setlength{\parsep}{0pt}
  % turn on hanging indent if param 1 is 1
  \ifodd #1
   \setlength{\leftmargin}{\cslhangindent}
   \setlength{\itemindent}{-1\cslhangindent}
  \fi
  % set entry spacing
  \setlength{\itemsep}{#2\baselineskip}}}
 {\end{list}}
\usepackage{calc}
\newcommand{\CSLBlock}[1]{\hfill\break\parbox[t]{\linewidth}{\strut\ignorespaces#1\strut}}
\newcommand{\CSLLeftMargin}[1]{\parbox[t]{\csllabelwidth}{\strut#1\strut}}
\newcommand{\CSLRightInline}[1]{\parbox[t]{\linewidth - \csllabelwidth}{\strut#1\strut}}
\newcommand{\CSLIndent}[1]{\hspace{\cslhangindent}#1}

\setlength\heavyrulewidth{0ex}
\setlength\lightrulewidth{0ex}
\usepackage[automark]{scrlayer-scrpage}
\clearpairofpagestyles
\cehead{
  Brian Weatherson
  }
\cohead{
  Epistemic Modals in Context
  }
\ohead{\bfseries \pagemark}
\cfoot{}
\makeatletter
\newcommand*\NoIndentAfterEnv[1]{%
  \AfterEndEnvironment{#1}{\par\@afterindentfalse\@afterheading}}
\makeatother
\NoIndentAfterEnv{itemize}
\NoIndentAfterEnv{enumerate}
\NoIndentAfterEnv{description}
\NoIndentAfterEnv{quote}
\NoIndentAfterEnv{equation}
\NoIndentAfterEnv{longtable}
\NoIndentAfterEnv{abstract}
\renewenvironment{abstract}
 {\vspace{-1.25cm}
 \quotation\small\noindent\rule{\linewidth}{.5pt}\par\smallskip
 \noindent }
 {\par\noindent\rule{\linewidth}{.5pt}\endquotation}
\cehead{
  Egan, Hawthorne, and Weatherson
   }
 \usepackage{enumitem}
\KOMAoption{captions}{tableheading}
\makeatletter
\@ifpackageloaded{caption}{}{\usepackage{caption}}
\AtBeginDocument{%
\ifdefined\contentsname
  \renewcommand*\contentsname{Table of contents}
\else
  \newcommand\contentsname{Table of contents}
\fi
\ifdefined\listfigurename
  \renewcommand*\listfigurename{List of Figures}
\else
  \newcommand\listfigurename{List of Figures}
\fi
\ifdefined\listtablename
  \renewcommand*\listtablename{List of Tables}
\else
  \newcommand\listtablename{List of Tables}
\fi
\ifdefined\figurename
  \renewcommand*\figurename{Figure}
\else
  \newcommand\figurename{Figure}
\fi
\ifdefined\tablename
  \renewcommand*\tablename{Table}
\else
  \newcommand\tablename{Table}
\fi
}
\@ifpackageloaded{float}{}{\usepackage{float}}
\floatstyle{ruled}
\@ifundefined{c@chapter}{\newfloat{codelisting}{h}{lop}}{\newfloat{codelisting}{h}{lop}[chapter]}
\floatname{codelisting}{Listing}
\newcommand*\listoflistings{\listof{codelisting}{List of Listings}}
\makeatother
\makeatletter
\makeatother
\makeatletter
\@ifpackageloaded{caption}{}{\usepackage{caption}}
\@ifpackageloaded{subcaption}{}{\usepackage{subcaption}}
\makeatother
\ifLuaTeX
  \usepackage{selnolig}  % disable illegal ligatures
\fi
\usepackage{bookmark}

\IfFileExists{xurl.sty}{\usepackage{xurl}}{} % add URL line breaks if available
\urlstyle{same} % disable monospaced font for URLs
\hypersetup{
  pdftitle={Epistemic Modals in Context},
  pdfauthor={Andy Egan; John Hawthorne; Brian Weatherson},
  hidelinks,
  pdfcreator={LaTeX via pandoc}}

\title{Epistemic Modals in Context}
\author{Andy Egan \and John Hawthorne \and Brian Weatherson}
\date{2005}

\begin{document}
\maketitle
\begin{abstract}
A very simple contextualist treatment of a sentence containing an
epistemic modal, e.g.~\emph{a might be F}, is that it is true iff for
all the contextually salient community knows, \emph{a} is \emph{F}. It
is widely agreed that the simple theory will not work in some cases, but
the counterexamples produced so far seem amenable to a more complicated
contextualist theory. We argue, however, that no contextualist theory
can capture the evaluations speakers naturally make of sentences
containing epistemic modals. If we want to respect these evaluations,
our best option is a \emph{relativist} theory of epistemic modals. On a
relativist theory, an utterance of \emph{a might be F} can be true
relative to one context of evaluation and false relative to another. We
argue that such a theory does better than any rival approach at
capturing all the behaviour of epistemic modals.
\end{abstract}

\setstretch{1.1}
In the 1970s David Lewis argued for a contextualist treatment of modals
(\citeproc{ref-Lewis1976d}{Lewis 1976},
\citeproc{ref-Lewis1979f}{1979}). Although Lewis was primarily
interested in modals connected with freedom and metaphysical
possibility, his arguments for contextualism could easily be taken to
support contextualism about epistemic modals. In the 1990s Keith DeRose
argued for just that position (\citeproc{ref-DeRose1991}{DeRose 1991},
\citeproc{ref-DeRose1998}{1998}).

In all contextualist treatments, the method by which the contextual
variables get their values is not completely specified. For
contextualist treatments of metaphysical modality, the important value
is the class of salient worlds. For contextualist treatments of
epistemic modality, the important value is which epistemic agents are
salient. In this paper, we start by investigating how these values might
be generated, and conclude that it is hard to come up with a plausible
story about how they are generated. There are too many puzzle cases for
a simple contextualist theory to be true, and a complicated
contextualist story is apt to be implausibly ad hoc.

We then look at what happens if we replace contextualism with
relativism. On contextualist theories the truth of an utterance type is
relative to the context in which it is tokened. On relativist theories,
the truth of an utterance token is relative to the context in which it
is evaluated. Many of the puzzles for contextualism turn out to have
natural, even elegant, solutions given relativism. We conclude by
comparing two versions of relativism.

We begin with a puzzle about the role of epistemic modals in speech
reports.

\section{A Puzzle}\label{a-puzzle}

The celebrity reporter looked discomforted, perhaps because there were
so few celebrities in Cleveland.

``Myles'', asked the anchor, ``where are all the celebrities? Where is
Professor Granger?''

``We don't know,'' replied Myles. ``She might be in Prague. She was
planning to travel there, and no one here knows whether she ended up
there or whether she changed her plans at the last minute.''

This amused Professor Granger, who always enjoyed seeing how badly wrong
CNN reporters could be about her location. She wasn't sure exactly where
in the South Pacific she was, but she was certain it wasn't Prague. On
the other hand, it wasn't clear what Myles had gotten wrong. His first
and third sentences surely seemed true: after all, he and the others
certainly \emph{didn't} know where Professor Granger was, and she
\emph{had} been planning to travel to Prague before quietly changing her
destination to Bora Bora.

The sentence causing all the trouble seemed to be the second: ``She
might be in Prague.'' As she wiggled her toes in the warm sand and
listened to the gentle rustling of the palm fronds in the salty breeze,
at least one thing seemed clear: she definitely wasn't in Prague -- so
how could it be true that she might be? But the more she thought about
it, the less certain she became. She mused as follows: when I say
something like \emph{x might be F}, I normally regard myself to be
speaking truly if neither I nor any of my mates know that x is not
\emph{F}. And it's hard to believe that what goes for me does not go for
this CNN reporter. I might be special in many ways, but I'm not
semantically special. So it looks like Myles can truly say that I might
be in Prague just in case neither \emph{he} nor any of \emph{his} mates
knows that I am not. And I'm sure none of them knows that, because I've
taken great pains to make them think that I am, in fact, in Prague --
and reporters always fall for such deceptions.

But something about this reasoning rather confused Professor Granger,
for she was sure Myles had gotten something \emph{wrong}. No matter how
nice that theoretical reasoning looked, the fact was that she definitely
wasn't in Prague, and he said that she might be. Trying to put her
finger on just where the mistake was, she ran through the following
little argument.\footnote{Some of Professor Granger's thoughts sound a
  little odd being in the present tense, but as we shall see, there are
  complications concerning the interaction of tense with epistemic
  modals, so for now it is easier for us to avoid those interactions.}

\begin{description}
\tightlist
\item[(1)]
When he says, ``She might be in Prague'' Myles says that I might be in
Prague.
\item[(2)]
When he says, ``She might be in Prague'' Myles speaks truly iff neither
he nor any of his mates know that I'm not in Prague.
\item[(3)]
Neither Myles nor any of his mates know that I'm not in Prague.
\item[(4)]
If Myles speaks truly when he says that I might be in Prague, then I
might be in Prague.
\item[(5)]
I know I'm not in Prague.
\item[(6)]
It's not the case that I know I'm not in Prague if I might be in Prague.
\end{description}

There must be a problem here somewhere, she thought -- for (1) -- (6)
are jointly inconsistent. (Quick proof: (2) and (3) entail that Myles
speaks truly when he says, ``She might be in Prague''. From that and (1)
it follows he speaks truly when he says Professor Granger might be in
Prague. From that and (4) it follows that Professor Granger might be in
Prague. And that combined with (5) is obviously inconsistent with (6).)
But wherein lies the fault? Unless some fairly radical kind of
scepticism is true, Professor Granger can know by observing her South
Pacific idyll that she's not in Prague -- so (5) looks secure. And it
seems pretty clear that neither Myles nor any of his mates know that
she's not in Prague, since they all have very good reason to think that
she is -- so it looks like (3) is also OK. But the other four premises
are all up for grabs.

Which exactly is the culprit is a difficult matter to settle. While the
semantic theory underlying the reasoning in (1)-(6) is mistaken in its
details, something like it is very plausible. The modal `might' here is,
most theorists agree, an \emph{epistemic} modal. So its truth-value
should depend on what someone knows. But who is this \emph{someone}? If
it is Myles, or the people around him, then the statement ``she might be
in Prague'' is true, and it is unclear where to block the paradox. If it
is Professor Granger, or the people around her, then the statement is
false, but now it is unclear why a competent speaker would ever use this
kind of epistemic modal. Assuming the \emph{someone} is Professor
Granger, and assuming Professor Granger knows where she is, then
``Granger might be in Prague'' will be true iff ``Granger is in Prague''
is true. But this seems to be a mistake. Saying ``Granger \emph{might}
be in Prague'' is a way to weaken one's commitments, which it could not
be if the two sentences have the same truth conditions under plausible
assumptions. So neither option looks particularly promising.

To make the problem even more pressing, consider what happens if a
friend of Professor Granger's who knows she is in the South Pacific
overhears Myles's comment. Call this third party Charles. It is prima
facie very implausible that when Myles says that Professor Granger might
be in Prague he means to rule out that \emph{Charles} knows that she is
not. After all, Charles is not part of the conversation, and Myles need
not even know that he exists. So if Myles knows what he is saying, what
he is saying could be true even if Charles knows Professor Granger is
not in Prague. But if Charles knows this, Charles cannot regard Myles's
statement as true, else he will conclude that Professor Granger might be
in Prague, and he knows she is not. So things are very complicated
indeed.

In reasoning as we have been, we have been assuming that the following
inferences are valid.

\begin{description}
\tightlist
\item[(7)]
A competent English speaker says \emph{It might be that S}; and
\item[(8)]
\emph{S}, on that occasion of use, means that \emph{p}; entail
\item[(9)]
That speaker says that it might be that \emph{p}.
\end{description}

Further, (9) plus

\begin{description}
\tightlist
\item[(10)]
That speaker speaks truly; entail
\item[(11)]
It might be that \emph{p}.
\end{description}

If Charles accepts the validity of both of these inferences, then he is
under considerable pressure to deny that Myles speaks truly. And it
would be quite natural for him to do so -- for instance, by interrupting
Myles to say that ``That's wrong. Granger couldn't be in Prague, since
he left on the midnight flight to Tahiti.'' But it's very hard to find a
plausible semantic theory that backs up this intervention, although such
reactions are extremely common. (To solidify intuitions, here is another
example: I overhear you say that a certain horse might have won a
particular race. I happen to know that the horse is lame. I think: you
are wrong to think that it might have won.)\footnote{Note that it also
  seems implausible to say that this is an instance of
  \emph{metalinguistic} negation, as discussed in Horn
  (\citeproc{ref-Horn1989}{1989}). When Charles interrupts Myles to
  object, the objection isn't that the particular form of words that
  Myles has chosen is inappropriate. The form of words is fine, and
  Myles' utterance would be completely unobjectionable if Charles's
  epistemic state were slightly different. What's wrong is that Myles
  has used a perfectly acceptable form of words to say something that's
  false (at least by Charles' lights---more on this later). We also
  think it's implausible to understand the `might' claims in question
  here as claims of objective chance or objective danger.}

Our solutions to this puzzle consist in proposed semantic theories for
epistemic modals. We start with contextualist solutions, look briefly at
invariantist solutions, and conclude with relativist solutions. Although
we will look primarily at the costs and benefits of these theories with
respect to intuitions about epistemic modals, it is worth remembering
that they differ radically in their presuppositions about what kind of
theory a semantic theory should be. Solving the puzzles to do with
epistemic modals may require settling some of the deepest issues in
philosophy of language

\section{Contextualist Solutions}\label{contextualist-solutions}

In his (\citeproc{ref-DeRose1991}{1991}), Keith DeRose offers the
following proposal:

\begin{quote}
S's assertion ``It is possible that P'' is true if and only if (1) no
member of the relevant community knows that P is false, and (2) there is
no relevant way by which members of the relevant community can come to
know that P is false. (593-4)
\end{quote}

DeRose intends `possible' here to be an epistemic modal, and the
proposal is meant to cover all epistemic modals, including those using
`might'.\footnote{We take the puzzle to be a puzzle about sentences
  containing epistemic modal operators, however they are identified. We
  are sympathetic with DeRose's (\citeproc{ref-DeRose1998}{1998})
  position that many sentences containing `might' and `possible' are
  unambiguously epistemic, but do not wish to argue for that here.
  Rather, we simply take for granted that a class of sentences
  containing epistemic modal operators has been antecedently identified.

  There are two differences between `possible' and `might'. The first
  seems fairly superficial. Sentences where \emph{might} explicitly
  takes a sentence, rather than a predicate, as its argument are awkward
  at best, and may be ungrammatical. \emph{It is possible that Professor
  Granger is in Prague} is much more natural than \emph{It might be the
  case that Professor Granger is in Prague}, but there is no felt
  asymmetry between \emph{Professor Granger is possibly in Prague} and
  \emph{Professor Granger might be in Prague}. We will mostly ignore
  these issues here, and follow philosophical orthodoxy in treating
  epistemic modals as being primarily sentence modifiers rather than
  predicate modifiers. The syntactic features of epistemic modals are
  obviously important, but we're fairly confident that the assumption
  that epistemic modals primarily operate on sentences does not bear any
  theoretical load here, and could be replaced if necessary.

  The other difference will be relevant to some arguments that follow.
  `Might' can interact with tense operators in a way that `possible'
  does not. \emph{It might have rained} could either mean MIGHT (WAS it
  rains) or WAS (MIGHT it rains), while \emph{It possibly rained}
  unambiguously means POSSIBLY (WAS it rains). It is often hard in
  English to tell just which meaning is meant when a sentence contains
  both tense operators and epistemic modals, but in Spanish these are
  expressed differently: Puede haber llovido; Podría haber llovido.} We
will not discuss here the issues that arise under clause (2) of DeRose's
account, since we'll have quite enough to consider just looking at
whether clause (1) or anything like it is correct.\footnote{There are
  three kinds of cases where something like DeRose's clause (2) could be
  relevant.

  First, Jack and Jill are in a conversation, and Jack knows \emph{p}
  while Jill knows \emph{p} \({\rightarrow \neg}\) \emph{Fa}. In this
  case intuitively neither could truly say \emph{a might be F} even
  though neither knows \emph{a} is not \emph{F}.

  Second, there are infinitely many mathematicians discussing Fermat's
  Last Theorem. The first knows just that it has no solutions for
  \emph{n}=3, the second just that it has no solutions for \emph{n}=4,
  and so on. Intuitions are (unsurprisingly) weaker here, but we think
  none of them could say \emph{Fermat's Last Theorem} \emph{might have
  solutions}, because the group's knowledge rules this out.

  Third, if \emph{S} was very recently told that \emph{a} is not
  \emph{F}, but simply forgot this, then intuitively she speaks falsely
  if she says \emph{a might be F}.

  Fourth, if S has the materials for easily coming to know P from her
  current knowledge, but has not performed the relevant inference, then
  we might be inclined (depending on how easy the inferential steps were
  to see and so on) to say that she is wrong to utter `It might be that
  not P'.

  Rather than try and resolve the issues these cases raise, we will
  stick to cases where the only thing that could make \emph{a might be
  F} false is that someone knows that \emph{a} is not \emph{F}.}

In our discussion below, we consider three promising versions of
contextualist theory. What makes the theories contextualist is that they
all say that Myles spoke truly when he said ``She might be in Prague'',
but hold that if Professor Granger had repeated his words she would have
said something false.\footnote{She would also have violated some
  pragmatic principles by knowingly using a third-person pronoun to
  refer to herself, but we take it those principles are defeasible, and
  violation of them does not threaten the truth-aptness of her
  utterance.} And the reason for the variation in truth-value is just
that Myles and Professor Granger are in different contexts, which supply
different relevant communities. Where the three theories differ is in
which constraints they place on how context can supply the community in
question.

The first is the kind of theory that DeRose originally proposed. On this
theory, there is a side constraint that the relevant community always
includes the speaker: whenever S truly utters \emph{a might be F}, S
does not know that \emph{a} is not \emph{F}. We'll call this \emph{the}
\emph{speaker-inclusion constraint}, or sometimes just
\emph{speaker-inclusion}. There is some quite compelling evidence for
speaker-inclusion. Consider, for example, the following sort of case:
Whenever Jack eats pepperoni pizza, he forgets that he has ten fingers,
and thinks ``I might only have eight fingers.'' Jill (who knows full
well that Jack has \emph{ten} fingers) spots Jack sitting all alone
finishing off a pepperoni pizza, and says, ``He might have eight
fingers.'' Jill has said something false. And what she's said is false
because it's not compatible with what \emph{she} knows that Jack has
eight fingers. But if the relevant community could ever exclude the
speaker, one would think it could do so here. After all, Jack is clearly
contextually salient: he's the referent of `he,' the fingers in question
are on his hand, and no one else is around.\footnote{Notice that
  intuitions do not change if we alter the case in such a way that Jack
  has a strange disorder that makes it very hard for him to come to know
  how many fingers he has. Thus clause (2) of Derose's analysis cannot
  do the work of the relevant side constraint.} Now, a single case does
not prove a universal\footnote{And see the case of Tom and Sally in the
  maze below for some countervailing evidence.} -- but the case does
seem to provide good \emph{prima facie} vidence for DeRose's constraint.

One implication of DeRose's theory is that (1) is false, at least when
Professor Granger says it. For when Professor Granger reports that Myles
says ``She might be in Prague,'' she is reporting a claim he makes about
his epistemic community -- that her being in Prague is compatible with
the things that they know. But when she says (in the second clause) that
this means he is saying that she might be in Prague, she speaks falsely.
For in her mouth the phrase ``that I might be in Prague'' denotes the
proposition that it's compatible with the knowledge of an epistemic
community that includes Professor Granger (as the speaker) that
Professor Granger is in Prague. And that is not a proposition that Myles
assented to. So DeRose's theory implies that the very intuitive (1) is
false when uttered by Granger.

\begin{description}
\tightlist
\item[(1)]
When he says, ``She might be in Prague'' Myles says that I might be in
Prague.
\end{description}

It is worth emphasizing how counterintuitive this consequence of
speaker-inclusion is. If the speaker-inclusion constraint holds
universally then in general speech involving epistemic modals cannot be
reported disquotationally. But notice how natural it is, when telling
the story of Jack and Jill, to describe the situation (as we ourselves
did in an earlier draft of this paper) as being one where ``Whenever
Jack eats pepperoni pizza, he forgets that he has ten fingers, and
thinks he might only have eight.'' Indeed, it is an important
generalization about how we use language that speakers usually do not
hesitate to disquote in reporting speeches using epistemic modals. So
much so that exceptions to this general principle are striking -- as
when the tenses of the original speech and the report do not match up,
and the tense difference matters to the plausibility of the attribution.

One might try to explain away the data just presented by maintaining a
laxity for `says that' reports. A chemist might say `The bottle is
empty' meaning it is empty of air, while milkman might utter the same
sentence, meaning in my context that it is empty of milk. Nevertheless,
the milkman might be slightly ambivalent about denying:

\begin{quote}
When the chemist says `The bottle is empty', she says that the bottle is
empty.
\end{quote}

And this is no doubt because the overt `says that' construction
frequently deploys adjectives and verbs in a rather quotational way.
After all, the chemist could get away with the following speech in
ordinary discourse: ``I know the milkman said that the bottle is empty.
But he didn't mean what I meant when I said that the bottle is empty.
When he said that the bottle was empty he meant that it was empty of
milk.''\footnote{Notice that this use prohibits the inference from: The
  speaker said that the bottle was empty, to, The speaker expressed the
  proposition/said something that meant that the bottle was empty.} Thus
the conventions of philosophers for using `say that' involve regimenting
ordinary use in a certain direction.\footnote{We are grateful for
  correspondence with John MacFarlane here.} But the disquotational
facts that we are interested in cannot be explained away simply by
invoking these peculiarities of `says that' constructions, for the same
disquotational ease surrounds the relevant belief reports. In the case
just considered, while we might argue about whether it was acceptable
for the chemist to say, in her conversational context, ``The milkman
said that the bottle was empty'', it is manifestly unacceptable for her
to say ``The milkman believes that the bottle is empty''. This contrasts
with the case of `might': If someone asked Professor Granger where Myles
thought she was, she could quite properly have replied with (12).

\begin{description}
\tightlist
\item[(12)]
He thinks that/believes that I might be in Prague.
\end{description}

Indeed, we in general tend find the following inference pattern -- a
belief-theoretic version of (7) to (9) above -- compelling:

\begin{enumerate}
\def\labelenumi{\arabic{enumi}.}
\tightlist
\item
  A competent English speaker sincerely asserts \emph{It might be that
  S}
\item
  S, in that context of use, means that \emph{p}.; therefore,
\item
  That speaker believes that it might be that \emph{p}
\end{enumerate}

Our puzzle cannot, then, be traced simply to a laxity in the `says that'
construction.\footnote{For what its worth, we also note that `S claimed
  that P' has less laxity (of the sort being discussed) than `S said
  that P'.} Whatever the puzzle comes to, it certainly runs deeper than
that.

Notice that (12) does not suggest that Myles thinks that for all
Professor Granger knows, she is in Prague; it expresses the thought that
Myles thinks that for all he knows, that is where she is. Moreover, this
is hardly a case where Granger's utterance is of doubtful
appropriateness: (12) is one of the ways canonically available for
Granger to express that thought. But if we assume that what is reported
in a belief report of this kind is belief in the proposition the
reporter expresses by \emph{I might be in Prague}, and we assume a
broad-reaching speaker-inclusion constraint, we must concede that the
proposition Granger expresses by uttering (12) is that Myles believes
that for all \emph{Professor Granger} knows, Professor Granger is in
Prague.

If the speaker-inclusion constraint holds universally, then anyone
making such a report is wrong. There are two ways for this to
happen---either they know what the sentences they're using to make the
attributions mean, and they have radically false views about what other
people believe, or they have non-crazy views about what people believe,
but they're wrong about the meanings of the sentences they're using. The
first option is incredibly implausible. So our first contextualist
theory needs to postulate a widespread semantic blindness; in general
speakers making reports are mistaken about the semantics of their own
language. In particular, it requires that such speakers are often blind
to semantic differences between sentence tokens involving epistemic
modals. It is possible that some theories that require semantic
blindness are true, but other things being equal we would prefer
theories that do not assume this.\footnote{Note that the negation of
  semantic blindness concerning some fragment of the language is
  \emph{not} the theory that speakers know all the semantic equivalences
  that hold between terms in that fragment. All we mean by the denial of
  semantic blindness is that speakers not have false beliefs about the
  semantics of their terms.} In general the burden of proof is on those
who think that the folk don't know the meaning of their own words. More
carefully: the burden of proof is on those who think that the folk are
severely handicapped in their ability to discriminate semantic sameness
and difference in their home language.

So the plausibility of (1) counts as evidence against the first
contextualist theory, and provides a suggestion for our second
contextualist theory. The cases that provide the best intuitive support
for the speaker-inclusion constraint and the case we used above,
involved unembedded epistemic modals. Perhaps this constraint is true
for epistemic modals in simple sentences, but not for epistemic modals
in `that' clauses. Perhaps, that is, when S sincerely asserts \emph{X Vs
that a might be F}, she believes that X Vs that for all X (and her
community) knows, \emph{a} is \emph{F}. (This is not meant as an account
of the logical form of \emph{X Vs that a might be F}, just an account of
its truth conditions. We defer consideration of what hypothesis, if any,
about the underlying syntax could generate those truth conditions.) To
motivate this hypothesis, note how we introduced poor Jack, above. We
said that he thinks he might have eight fingers. We certainly didn't
mean by that that Jack thinks something about \emph{our} epistemic
state.

The other problem with the speaker-inclusion constraint is that it does
not seem to hold when epistemic modals are bound by temporal modifiers,
as in the following example. A military instructor is telling his troops
about how to prepare for jungle warfare. He says, ``Before you walk into
an area where there are lots of high trees, if there might be snipers
hiding in the branches, clear away the foliage with flamethrowers.''
Whatever the military and environmental merits of this tactic, the
suggestion is clear. The military instructor is giving generic
conditional advice: in any situation of type \emph{S}, if \emph{C} then
do \emph{A}. The situation \emph{S} is easy to understand, it is when
the troops are advancing into areas where there are high trees. And A,
too, is clear: blaze 'em. But what about \emph{C}? What does it mean to
say that there might be snipers in the high branches? Surely not that
it's compatible with the military instructor's knowledge that there are
snipers in the high branches -- he's sitting happily in West Point,
watching boats sail lazily along the Hudson. What \emph{he} thinks about
where the snipers are is neither here nor there. Intuitively, what he
meant was that the troops should use flamethrowers if \emph{they} don't
know whether there are snipers in the high branches. (Or if they know
that there \emph{are}.) So as well as leading to implausible claims
about speech reports, the speaker-inclusion constraint seems clearly
false when we consider temporal modifiers.

Here is a way to deal with both problems at once. There are constraints
on the application of the speaker-inclusion constraint. It does not
apply when the epistemic modal is in the scope of a temporal modifier
(as the flamethrower example shows) and it does not apply when the
epistemic modal is in a `that' clause.\footnote{~This theory looks like
  one in which propositional attitude operators become monsters, since
  the content of \emph{Jack thinks that Jill might be happy} is
  naturally generated by applying the operator \emph{Jack thinks} to the
  proposition that \emph{that Jill might be happy} denotes when it is
  expressed in Jack's context. But this is not the easiest, or obviously
  the best, way to look at the theory. For one thing, that way of
  looking at things threatens to assign the wrong content to \emph{Jack
  thinks that Jill might have stolen my car}. The content of \emph{Jill
  might have stolen my car} in Jack's context is that for all Jack
  knows, Jill stole Jack's car, which is not what is intended. That is
  to say, thinking of propositional attitude operators as monsters here
  ignores the special status of epistemic modals in the semantics. It is
  better, we think, to hold that on this theory epistemic modals are
  impure indexicals whose value is fixed, inter alia, by their location
  in the sentence as well as their location in the world. But even if
  this theory does not \emph{officially} have monsters, the similarity
  to monstrous theories is worth bearing in mind as one considers the
  pros and cons of the theory.

  Thanks to Ernest Lepore for helpful discussions here.} Our second
contextualist theory then accepts the speaker-inclusion constraint, but
puts constraints on its application.

This kind of theory, with a speaker-inclusion constraint only applying
to relatively simple epistemic modals, allows us to accept (1). The
problematic claim on this theory turns out to be (4):

\begin{description}
\tightlist
\item[(4)]
If Myles speaks truly when he says that I might be in Prague, then I
might be in Prague.
\end{description}

When Myles said that Professor Granger might be in Prague, he was
speaking truly. That utterance expressed a true proposition. So the
antecedent of (4) is true. But the consequent is false: the ``might''
that appears there is not in a that-clause or in the scope of a temporal
modifier; so the speaker-inclusion constraint requires that Professor
Granger be included in the relevant community; and since she knows that
she is not in Prague, it's not true that she might be. We would
similarly have to reject:

\begin{description}
\tightlist
\item[(4\textsuperscript{'})]
If Myles has a true belief that I might be in Prague, then I might be in
Prague.
\end{description}

But there are reasons to be worried about this version of contextualism,
beyond the uneasiness that attaches to denying (4), and, worse still,
(4\textsuperscript{'}). For one, this particular version of the
speaker-inclusion constraint seems a bit \emph{ad hoc}: why should there
be just \emph{these} restrictions on the relevant community? More
importantly, the theory indicts certain inferential patterns that are
intuitively valid. Suppose a bystander in our original example
reasoned\footnote{What follows is a belief theoretic version of Charles'
  reasoning.}:

\begin{description}
\tightlist
\item[(13)]
Myles believes that it might be that Professor Granger is in Prague.
\item[(14)]
Myles's belief is true; therefore,
\item[(15)]
It might be that Professor Granger is in Prague.
\end{description}

But this version of contextualism tells us that while (13) and (14) are
true, (15) is false. In general, there are going to be counter-intuitive
results whenever we reason from cases where the speaker-inclusion
constraint does not apply to cases where it does.

Finally, the theory is unable to deal with certain sorts of puzzle
cases. The first kind of case directly challenges the speaker-inclusion
constraint for simple sentences, although we are a little sceptical
about how much such a case shows.\footnote{~A similar case to the
  following appears in (\citeproc{ref-Hawthorne2004}{Hawthorne 2004,
  27}).} Tom is stuck in a maze. Sally knows the way out, and knows she
knows this, but doesn't want to tell Tom. Tom asks whether the exit is
to the left. Sally says, ``It might be. It might not be.'' Sally might
be being unhelpful here, but it isn't clear that she is \emph{lying}.
Yet if the speaker-inclusion constraint applies to unembedded epistemic
modals, then Sally is clearly saying something that she knows to be
false, for she knows that she knows which way is out.

This case is not altogether convincing, for there is something slightly
awkward about Sally's speech here. For example, if Sally knows the exit
is not to the left, then even if she is prepared to utter, ``It might be
{[}to the left{]},'' she will not normally self-ascribe knowledge that
it might be to the left. And normally speakers don't sincerely assert
things they don't take themselves to know. So it is natural to suppose
that a kind of pretense or projection is going on in Sally's speech that
may well place it beyond the purview of the core semantic theory.

The following case makes more trouble for our second contextualist
theory, though it too has complications. Ann is planning a surprise
party for Bill. Unfortunately, Chris has discovered the surprise and
told Bill all about it. Now Bill and Chris are having fun watching Ann
try to set up the party without being discovered. Currently Ann is
walking past Chris's apartment carrying a large supply of party hats.
She sees a bus on which Bill frequently rides home, so she jumps into
some nearby bushes to avoid being spotted. Bill, watching from Chris's
window, is quite amused, but Chris is puzzled and asks Bill why Ann is
hiding in the bushes. Bill says

\begin{description}
\tightlist
\item[(16)]
I might be on that bus.
\end{description}

It seems Bill has, somehow, conveyed the \emph{correct} explanation for
Ann's dive---he's said something that's both true and explanatory. But
in his mouth, according to either contextualist theory we have
considered, it is not true (and so it can't be explanatory) that he
might have been on the bus. He knows that he is in Chris's apartment,
which is not inside the bus.

Chris's question, like most questions asking for an explanation of an
action, was ambiguous. Chris might have been asking what
\emph{motivated} Ann to hide in the bushes, or he might have been asking
what \emph{justified} her hiding in the bushes. This ambiguity is often
harmless, because the same answer can be given for each. This looks to
be just such a case. Bill seems to provide both a motivation and a
justification for Ann's leap by uttering (16). That point somewhat
undercuts a natural explanation of what's going on in (16). One might
think that what he said was elliptical for \emph{She believed that I
might be on the bus}. And on our second contextualist theory, that will
be \emph{true}. If Bill took himself to be answering a question about
motivation, that might be a natural analysis. (Though there's the
underlying problem that Ann presumably wasn't thinking about her mental
states when she made the leap. She was thinking about the bus, and
whether Bill would be on it.) But that analysis is less natural if we
think that Bill was providing a justification of Ann's
actions.\footnote{Though the theory will allow for the truth of, ``I
  might \emph{have been} on that bus'' (since the epistemic modal clause
  doesn't occur on its own, but in the scope of a temporal operator). So
  if we think that (i) that's enough to do the justificatory and
  explanatory work, and (b) Bill's utterance of ``I might be on that
  bus'' is best understood as a clumsy stab at ``I might have been on
  that bus'', then perhaps we can account for this kind of case using
  our second contextualist theory. Two worries: First, it is a cost of
  the theory that we have to reinterpret Bill's utterance in this way,
  as a clumsy attempt to say something that the theory can accommodate.
  Second, there might be cases where the interpretation is less
  plausible: As a response to, ``Why is Ann getting ready to jump over
  the hedge?'', ``I might have been on that bus'' sounds worse to us
  than ``I might be on that bus''.} And it seems plausible that he could
utter (16) in the course of providing such a justification. This
suggests that (16) simply means that for all \emph{Ann} knew, Bill was
on that bus. Alternatively, we could say that (16) is elliptical for
\emph{Because I might be on that bus}, and that the speaker-inclusion
constraint does not apply to an epistemic modal connected to another
sentence by `because'. This may be right, but by this stage we imagine
some will be thinking that the project of trying to find all the
restrictions on the speaker-inclusion constraint is a degenerating
research program, and a paradigm shift may be in order.

So our final contextualist theory is that DeRose's original semantic
theory, before the addition of any sort of speaker-inclusion constraint,
was correct and complete. So `might' behaves like `local' and `nearby'.
If Susie says ``There are snipers nearby,'' the truth condition for that
might be that there are snipers near Susie, or that there are snipers
near us, or that there are snipers near some other contextually salient
individual or group. Similarly, if she utters ``Professor Granger might
be in Prague'' the truth condition for that might be that for all she
knows Professor Granger is in Prague, or that for all we know Professor
Granger is in Prague, or that for all some other community knows,
Professor Granger is in Prague. There are no universal rules requiring
or preventing the speaker from being included in the class of salient
epistemic agents.

According to the third version of contextualism, if Professor Granger
does not equivocate when working through her paradox, then the problem
lies with (6):

\begin{description}
\item[(6)]
It's not the case that I can know I'm not in Prague if I might be in
Prague.
\end{description}

At the start of her reasoning process, Professor Granger's use of
`might' means (roughly) `is compatible with what Myles and his friends
know'. And if it keeps that meaning to the end, then the antecedent of
(6) is true, because Professor Granger might (in that sense) be in
Prague, even though she knows she is not. Any attempt to show that (1)
through (6) form an inconsistent set will commit a fallacy of
equivocation.\footnote{The same kind of equivocation can be seen in
  other arguments involving contextually variable terms. Assume that
  Nomar lives in Boston, Derek lives in New York, and Nomar, while
  talking about Fenway Park in Boston says, ``I live nearby.'' Derek, at
  home in New York, hears this on television and runs through the
  following argument.

  \begin{enumerate}
  \def\labelenumi{\arabic{enumi}.}
  \tightlist
  \item
    In saying ``I live nearby'' Nomar says that he lives nearby.
    (Plausible disquotational premise about `nearby')
  \item
    Nomar speaks truly when he says ``I live nearby'' (Follows from the
    setup)
  \item
    If Nomar speaks truly when he says ``I live nearby'' and in saying
    ``I live nearby'' he says that he lives nearby, then he lives
    nearby. (I.e. if he speaks truly then what he says is true.)
  \item
    If Nomar lives nearby, then he lives in New York (Since everywhere
    that's nearby to Derek's home is in New York.); therefore
  \item
    Nomar lives in New York.
  \end{enumerate}

  The right thing to say about this argument is that it equivocates.
  Every premise has a true reading. Perhaps every premise is true on its
  most natural reading, but the denotation of `nearby' has to change
  throughout the argument for every premise to be true. The current view
  is that `might' behaves like `nearby', and that Professor Granger's
  argument equivocates, like Derek's.}

But (6) as uttered by Professor Granger sounds extremely plausible. And
there are other, more general problems as well. It is difficult on such
a theory to explain why it is so hard to get the relevant community to
exclude the speaker in present tense cases: Why, for instance, can't
Jill's statement about Jack, ``He might have eight fingers,'' be a
statement about Jack's epistemic state rather than her own? The third
theory offers us no guidance.\footnote{There also seems to be a
  past/future asymmetry about epistemic modals which the third
  contextualist theory will have trouble explaining. Consider this case
  involving past tense epistemic modals. Romeo sees Juliet carrying an
  umbrella home on a sunny afternoon. When he asks her why she is
  carrying an umbrella, she replies ``It might have rained today.''
  There's a scope ambiguity in Juliet's utterance. If the epistemic
  modal takes wide scope with respect to the tense operator, Juliet
  would be claiming that she doesn't know whether it has rained today
  (implicating, oddly, that this is why she now has an umbrella.) Or, as
  Juliet presumably intends, the temporal operator could take wide scope
  with respect to the epistemic modal. In that case Juliet says that it
  was the case at some earlier time (presumably when she left for work
  this morning) that it was compatible with her knowledge that it would
  rain today. And that seems both true and a good explanation of her
  umbrella-carrying.

  It is much harder, if it is even possible, to find cases involving
  future tense operators where the temporal operator takes wide scope
  with respect to the pistemic modal. If S says, ``It might rain
  tomorrow'', that seems to unambiguously mean that it's compatible with
  S's \emph{current} knowledge (and her community's) that it rains
  tomorrow. For a more dramatic case, consider a case where two people,
  Othello and Desdemona, have discovered that a giant earthquake next
  week will destroy humanity. No one else knows this yet, but there's
  nothing that can be done about it. This rather depresses them, so they
  decide to take memory-wiping drugs so that when they wake up tomorrow,
  they won't know about the earthquake. Othello can't say, ``Tomorrow,
  humanity might survive,'' even though it is true that tomorrow, for
  all anyone will know, humanity will survive. If the temporal modifier
  could take wide scope with respect to the epistemic modal, Othello's
  utterance could have a true reading. But it does not. It's possible at
  this point that our policy, announced in footnote 2, of ignoring
  issues relating to DeRose's second clause will come back to haunt us.
  One possibility here is that tomorrow it will still be false that
  humanity might survive because it's not compatible with what people
  tomorrow know and knew that humanity survives. We don't think that's
  what is going on, but it's possible. Here's two quick reasons to think
  that the problem is not so simple. First, if Othello and Desdemona
  commit suicide rather than take the memory-wiping drugs, it will be
  compatible tomorrow with all anyone ever knew that humanity survives.
  But still Othello's speech seems false. Second, it's not obviously
  right that what people ever knew matters for what is epistemically
  possible now. Presumably at one stage Bill Clinton knew what he had
  for lunch on April 20, 1973. (For example, when he was eating lunch on
  April 20, 1973.) But unless he keeps meticulous gastronomical records,
  this bit of knowledge is lost to humanity forever. So there will be
  true sentences of the form \emph{Bill Clinton might have eaten x for
  lunch on April 20, 1973} even though someone once knew he did not. Now
  change the earthquake case so that it will happen in thirty years not
  a week, and no one will then know about it (because Othello and
  Desdemona took the memory-wiping drugs and destroyed the machines that
  could detect it). Still it won't be true if Othello says, ``In thirty
  years, humanity might survive.'' This suggests to us that some kind of
  constraints on epistemic modals will be required. The existence of
  these constraints seems to refute the `no constraints' version of
  contextualism. It also undermines the argument that the second version
  of contextualism is too ad hoc. Once some constraints are in place,
  others may be appropriate.}

We'll close this section with a discussion of the interaction between
syntax and semantics in these contextualist theories. As is well known,
in the last decade many different contextualist theories have been
proposed for various philosophically interesting terms. Jason Stanley
(\citeproc{ref-Stanley2000-STACAL}{2000}) has argued that the following
two constraints should put limits on when we posit contextualist
semantic theories.

\begin{description}
\tightlist
\item[Variable]
Any contextual effect on truth-conditions that is not traceable to an
indexical, pronoun, or demonstrative in the narrow sense must be
traceable to a structural position occupied by a variable.
(\citeproc{ref-Stanley2000-STACAL}{Stanley 2000, 401})\footnote{We
  assume here, following Stanley, a `traditional syntax involving
  variables' (\citeproc{ref-Stanley2000-STACAL}{Stanley 2000} fn. 13).
  At least one of us would prefer a variable-free semantics along the
  lines of Jacobson (\citeproc{ref-Jacobson1999}{1999}) Adopting such a
  semantics would involve, as Stanley says, major revisions to the
  presentation of this argument, but would not \emph{clearly} involve
  serious changes to the argument. Most contextualists happily accept
  the existence of variables so we do not beg any questions against
  them, but see Pagin (\citeproc{ref-Pagin2005}{2005}) for an important
  exception.}
\end{description}

\begin{description}
\tightlist
\item[Syntactic Evidence]
The only good evidence for the existence of a variable in the semantic
structure corresponding to a linguistic string is that the string, or
another that we have reason to believe is syntactically like it, has
interpretations that could only be accounted for by the presence of such
a variable.
\end{description}

If any contextualist theory of epistemic modals is to be justifiably
believed, then \textbf{Variable} and \textbf{Syntactic Evidence}
together entail the existence of sentences where the `relevant
community' is bound by some higher operator. So ideally we would have
sentences like (17) with interpretations like (18).

\begin{description}
\tightlist
\item[(17)]
Everyone might be at the party tonight.
\item[(18)]
For all \emph{x}, it is consistent with all \emph{x} knows that \emph{x}
will be at the party tonight.
\end{description}

Now (17) cannot have this interpretation, which might look like bad news
for the contextualist theory. It's natural to think that if `might'
includes a variable whose value is the relevant community, that variable
could be bound by a quantifier ranging over it. But if such a binding
were possible, it's natural to think that it would be manifested in
(17). So \textbf{Variable} and \textbf{Syntactic Evidence} together
entail that we ought not to endorse contextualism about epistemic
modals.

This argument against contextualism fails in an interesting way, one
that bears on the general question of what should count as evidence for
or against a contextualist theory. The reason that any variable
associated with `might' in (17) cannot be bound by `everyone' is that
`might' takes wider scope than `everyone'. Note that (17) does not mean
(19), but rather means (20).

\begin{description}
\tightlist
\item[(19)]
For all \emph{x}, it is consistent with what we know that \emph{x} will
be at the party tonight.
\item[(20)]
It is consistent with what we know that for all \emph{x}, \emph{x} will
be at the party tonight.
\end{description}

As Kai von Fintel and Sabine Iatridou
(\citeproc{ref-vonFintel2003}{2003}) have shown, in any sentence of the
form \emph{Every F might be G}, the epistemic modal takes wide scope.
For instance, (21) has no true reading if there is at most one winner of
the election, even if there is no candidate that we know is going to
lose.

\begin{description}
\item[(21)]
Every candidate might win.
\end{description}

More generally, epistemic modals take wide scope with respect to a wide
class of quantifiers.\footnote{It is not entirely clear what the
  relevant class of quantifiers is, although von Fintel and Iatridou
  have some intriguing suggestions about what it might be.} This fact is
called the Epistemic Containment Principle by von Fintel and Iatridou.
Even if there is a variable position for the relevant community in the
lexical entry for `might', this might be unbindable because the
epistemic modal always scopes over a quantifier that could bind it. If
that's true then the requirement imposed by \textbf{Syntactic Evidence}
is too strong. If the evidence from binding is genuinely neutral between
the hypothesis that this variable place exists and the hypothesis that
it does not, because there are no instances of epistemic modals that
take narrow scope with respect to quantifiers, it seems reasonable to
conclude that there are these variable places on the basis of other
evidence.

Having said all that, there still may be direct evidence for the
existence of a variable position for relevant communities. Consider
again our example of the military instructor, reprinted here as (22).

\begin{description}
\tightlist
\item[(22)]
Before you walk into an area where there are lots of high trees, if
there might be snipers hiding in the branches use your flamethrowers to
clear away the foliage.
\end{description}

As von Fintel and Iatridou note, it is possible for epistemic modals to
take narrow scope with respect to generic quantifiers. That's exactly
what happens in (22). And it seems that the best interpretation of (22)
requires a variable attached to `might'. Intuitively, (22) means
something like (23).

\begin{description}
\item[(23)]
Generally in situations where you are walking into an area where there
are lots of high trees, if it's consistent with \emph{your party}'s
knowledge that there are snipers hiding in the branches use your
flamethrowers to clear away the foliage.
\end{description}

The italicised \emph{your party} seems to be the semantic contribution
of the unenunciated variable. We are \emph{not} saying that the
existence of sentences like (23) shows that there are such variables in
the logical form of sentences involving epistemic modals.\footnote{As
  previously noted, we are not all convinced that semantics \emph{ever}
  needs to appeal to such variables, let alone that it does to account
  for the behaviour of epistemic modals.} We just want to make two
points here. First, if you are a partisan of \textbf{Syntactic
Evidence}, then (22) should convince you not to object to semantic
accounts of epistemic modals that appeal to variables, as our
contextualist theories do. Second, we note a general concern that
principles like \textbf{Syntactic Evidence} presupposes that a certain
kind of construction, where the contextually variable term is bound at a
level like LF, is always possible. Since there are rinciples like the
Epistemic Containment Principle, we note a mild concern that this
presupposition will not always be satisfied.

\section{Invariantist Solutions}\label{invariantist-solutions}

The most plausible form of invariantism about epistemic modals is that
DeRose's semantics is broadly correct, but the relevant community is not
set by context - it is invariably the world. We will call this position
\emph{universalism}. Of course when we say \emph{a might be F} we don't
normally communicate the proposition that no one in the world knows
whether \emph{a} is \emph{F}. The analogy here is to pragmatic theories
of quantifier domain restriction, according to which when we say
\emph{Everyone is F}, we don't communicate the proposition that everyone
in the world is \emph{F}, even though that is the truth condition for
our utterance.

The universalist position denies (2) in Professor Granger's argument.
Myles did not speak truly when he said ``Professor Granger might be in
Prague'' because someone, namely Professor Granger, knew she was not in
Prague. Although (2) is fairly plausible, it probably has weaker
intuitive support than the other claims, so this is a virtue of the
universalist theory.

The big advantage (besides its simplicity) of the universalist theory is
that it explains some puzzle cases involving eavesdropping. Consider the
following kind of case. Holmes and Watson are using a primitive bug to
listen in on Moriarty's discussions with his underlings as he struggles
to avoid Holmes's plan to trap him. Moriarty says to his assistant,

\begin{description}
\tightlist
\item[(24)]
Holmes might have gone to Paris to search for me.
\end{description}

Holmes and Watson are sitting in Baker Street listening to this. Watson,
rather inexplicably, says ``That's right'' on hearing Moriarty uttering
(24). Holmes is quite perplexed. Surely Watson knows that he is sitting
right here, in Baker Street, which is definitely not in Paris. But
Watson's ignorance is semantic, not geographic. He was reasoning as
follows. For all Moriarty (and his friends) know, Holmes is in Paris
searching for him. If some kind of contextualism is true, then it seems
that (24) is true in Moriarty's mouth. And, thought Watson, if someone
says something true, it's OK to say ``That's right.''

Watson's conclusion is clearly wrong. It's not OK for him to say
``That's right,'' in response to Moriarty saying (24). So his reasoning
must fail somewhere. The universalist says that where the reasoning
fails is in saying the relevant community only contains Moriarty's gang
members. If we include Holmes and Watson, as the universalist requires,
then Moriarty speaks falsely when he says (24).

There are a number of serious (and fairly obvious) problems with the
universalist account. According to universalism, the following three
claims are inconsistent.

\begin{description}
\tightlist
\item[(25)]
\emph{x} might be \emph{F}.
\item[(26)]
\emph{x} might not be \emph{F}.
\item[(27)]
Someone knows whether \emph{x} is \emph{F}.
\end{description}

Since these don't \emph{look} inconsistent, universalism looks to be
false.

The universalist's move here has to be to appeal to the pragmatics. If
(27) is true then one of (25) and (26) is false, although both might be
appropriate to express in some contexts. But if we can appropriately
utter sentences expressing false propositions in some contexts, then
presumably we can inappropriately utter true sentences in other
contexts. (Indeed, the latter possibility seems much more common.) So
one could respond to the universalist's main argument, their analysis of
eavesdropping cases like Watson's, by accepting that Watson can't
\emph{appropriately} say ``That's right'' but he can \emph{truly} say
this. The universalist will have a hard time explaining why such a
theory cannot work, assuming, of course, that she can explain how her
own pragmatic theory can explain all the data.

The major problem here is one common to all appeals to radical
pragmatics in order to defend semantic theories. If universalism is true
then speakers regularly, and properly, express propositions they know to
be false.\footnote{By ``express'' we will always mean ``semantically
  express''. We're not concerned with, and hope not to commit ourselves
  to any views about, for example, what's conveyed via various pragmatic
  processes.} (We assume here that radical scepticism is \emph{not}
true, so sometimes people know some things.) Myles knows full well than
\emph{someone} knows whether Professor Granger is in Prague, namely
Professor Granger. But if he's a normal English speaker, this will
\emph{not} seem like a reason for him to not say, ``Professor Granger
might be in Prague.'' Some might not think this is a deep problem for
the universalist theory, for speakers can be mistaken in their semantic
views in ever so many ways. But many ill regard it as a serious cost of
the universalist claim.

This problem becomes more pressing when we look at what universalism
says about beliefs involving epistemic modals. Myles does not just say
that Professor Granger might be in Prague, he believes it. And he
believes Professor Granger might not be in Prague. If he also believes
that Professor Granger knows where she is, these beliefs are
inconsistent given universalism. Perhaps the universalist can once again
invoke pragmatics. It is not literally true in the story that Myles
believes that Granger might be in Prague. But in escribing the situation
we use ``Myles believes that Granger might be in Prague,'' to
pragmatically communicate truths by a literal falsehood. This appeal to
a pragmatic escape route seems even more strained than the previous
universalist claims.

In general, the universalism under discussion here seems to run up
against a constraint on semantic theorising imposed by Kripke's Weak
Disquotation Principle. The principle says that if a speaker sincerely
accepts a sentence, then she believes its semantic value.\footnote{Note
  that something like this had better be true if what it is to believe
  \emph{p} is to have a sentence that means \emph{p} in one's `belief
  box'.} If we have some independent information about what a speaker
believes, then we can draw certain conclusions about the content of the
sentences she accepts, in particular that she only accepts sentences
whose content she believes. The universalist now has two
options.\footnote{We assume that it is not a serious option to deny that
  we ever accept unnegated epistemic modal sentences.} First, she can
say that Myles here does accept inconsistent propositions. Second, she
can deny the Weak Disquotation Principle, and say that although Myles
sincerely asserts, and accepts, ``Professor Granger might be in Prague''
he doesn't really believe that Professor Granger might be in Prague.
Generally, it's good to have options. But it's bad to have options as
unappealing as these.\footnote{There also a technical problem with
  universalism that mirrors one of the problems Stanley and Szabó
  (\citeproc{ref-Stanley2000-STAOQD}{2000}) raise for pragmatic theories
  of quantifier domain restriction. Normally (28) would be used to
  express a proposition like (29).

  \begin{description}
  \tightlist
  \item[(28)]
  Every professor enjoys every class.
  \item[(29)]
  Every salient professor enjoys every class that s/he teaches.
  \end{description}

  Intuitively, by uttering (28) we express a proposition that contains
  two restricted quantifiers. Let's accept, for the sake of the
  argument, that a pragmatic theory of quantifier domain restriction can
  sometimes explain why the quantifiers in the propositions we express
  are more restricted than the quantifiers in the truth conditions for
  the sentences we use. Stanley and Szabó argue that such an explanation
  will not generalise to cover embedded quantifiers where the quantifier
  domain in the proposition expressed is bound to the outer quantifier.
  One such quantifier is the quantifier over classes in (28). We will
  not repeat their arguments here, but simply note that if they are
  correct, the universalist faces a problem in explaining how we use
  sentences with embedded epistemic modals that are (intuitively)
  defined with respect to a community that is bound by a higher level
  quantifier. As we saw, (22) provides an example of this kind of
  epistemic modal.}

\section{Reporting Epistemic Modals}\label{reporting-epistemic-modals}

Our third class of solutions will be relatively radical, so it's worth
pausing to look at the evidence for it. Consider again the dialogue
between Moriarty, Holmes and Watson. Moriarty, recall, utters (24)

\begin{description}
\item[(24)]
Holmes might have gone to Paris to search for me.
\end{description}

Watson knows that Holmes is in Baker Street, as of course does Holmes.
In the above case we imagined that both Watson and Holmes heard Moriarty
say this. Change the story a little so Holmes does not hear Moriarty
speak, instead when he comes back into the room he asks Watson what
Moriarty thinks. Watson, quite properly, replies with (30).

\begin{description}
\tightlist
\item[(30)]
He thinks that you might have gone to Paris to search for him.
\end{description}

This is clearly not direct quotation because Watson changes the pronouns
in Moriarty's statement. It is not as if Watson said ``He sincerely
said, `Holmes might have gone to Paris to search for me.'\,'' This might
have been appropriate if Holmes suspected Moriarty was speaking in code
so the proposition he expressed was very sensitive to the words he used.

Nor was Watson's quote a `mixed' quote, in the sense of what happens in
(31).\footnote{Earlier we used \emph{speech} reports to illustrate the
  oddities of epistemic modals inside propositional attitude
  ascriptions. There are well-known difficulties with connecting
  appropriate speech reports to the semantic content of what is said, as
  opposed to merely communicated. (For some discussion of these, see
  Soames (\citeproc{ref-Soames2002}{2002}) and Capellen and Lepore
  (\citeproc{ref-Cappelen1997}{1997}).) We don't think those
  difficulties affect the above arguments, where the evidence is fairly
  clear, and fairly overwhelming. But matters get a little more delicate
  in what follows, so we move to belief reports because they are more
  closely tied to the content of what is believed.} The background is
that Arnold always uses the phrase `my little friend' to denote his
Hummer H2, despite that vehicle being neither little nor friendly. No
one else, however, approves of this terminology.\footnote{In this case,
  as with all the belief reports discussed below, the only evidence the
  reporter has for the report is given by the speech immediately
  preceding it. We assume there is good reason from the context to
  assume that the speakers are sincere.}

\begin{enumerate}
\def\labelenumi{(\arabic{enumi})}
\setcounter{enumi}{30}
\tightlist
\item
  Arnold: My little friend could drive up Mt Everest.\\
  Chaz: Arnold believes his little friend could drive up Mt Everest.
\end{enumerate}

We've left off the punctuation here so as to not beg any questions, but
there is a way this could be an acceptable report if the fourth and
fifth word, and those two words only, are part of a quotation. This is
clearly not ordinary direct quotation, for Arnold did not think, in
English or Mentalese, ``His little friend could drive up Mt Everest.''
Nevertheless, this is not ordinary indirect quotation. In ordinary
spoken English Chaz's report will be unacceptable unless `little friend'
is stressed. The stress here seems to be just the same stress as is used
in metalinguistic negation, as described in Horn
(\citeproc{ref-Horn1989}{1989}). Note the length of the pause between
`his' and `little'. With an ordinary pause it sounds as if Chaz is
using, not mentioning, `little friend'. So it is possible \emph{in
principle} to have belief reports, like this one, that are neither
strictly direct nor strictly indirect.\footnote{~There are somewhat
  delicate questions about what a direct belief report \emph{means}, but
  we assume the notion is well enough understood, even if we could not
  formally explicate what is going on in all such reports.}
Nevertheless, it does not seem like (30) need such a case. In
particular, there need be no distinctive metalinguistic stress on
`might' in Watson's utterance of (30), and such stress seems to be
mandatory for this mixed report.

Assuming Moriarty was speaking ordinary English, Watson's report seems
perfectly accurate. This is despite the fact that the relevant community
one would naturally associate with Watson's use of `might' is quite
different to the community we would associate with Moriarty's use. When
reporting speeches involving epistemic modals -- and the beliefs express
by sincere instances of such speeches, speakers can simply
\emph{disquote} the modal terms.

As is reasonably well known, there are many terms for which this kind of
disquoting report is impermissible. In every case, Guildenstern's report
of Ophelia's utterance is inappropriate.

\begin{enumerate}
\def\labelenumi{(\arabic{enumi})}
\setcounter{enumi}{31}
\item
  Ophelia: I love Hamlet.\\
  \ldots{}\\
  Guildenstern: *Ophelia thinks that I love Hamlet.
\item
  Guildenstern: What think you of Lord Hamlet?\\
  Ophelia: He is a jerk.\\
  \ldots{}\\
  Rosencrantz: What does Ophelia think of the King?\\
  Guildenstern: *She thinks that he is a jerk.
\item
  Guildenstern: Are you ready to teach the class on contextualism?\\
  Ophelia: I'm ready.\\
  \ldots{}\\
  Rosencrantz: Does Ophelia think she is ready to defend her
  dissertation?\\
  Guildenstern: *She thinks she is ready.
\item
  (Guildenstern and Ophelia are on the telephone, Guildenstern is in
  Miami, and Ophelia is in San Francisco)\\
  Guildenstern: What do you like best about San Francisco?\\
  Ophelia:There are lots of wineries nearby.\\
  \ldots{}\\
  Rosencrantz: Is it possible to grow wine in south Florida?\\
  Guildenstern: *Ophelia thinks that there are lots of wineries
  nearby.{[}\^{}We do not say that `nearby' in a speech report could
  \emph{never} refer to the area near the location of the original
  speaker. Had Rosencrantz asked a question about San Francisco, and
  Guildenstern given the same response, that is presumably what it would
  have done. We just say that it does not \emph{automatically} refer
  back to that area, and in some cases, like (35), it can refer to a
  quite different area. `Nearby' behaves quite differently in this
  respect to `near here', which always refers to the area near the
  reporter.{]}
\end{enumerate}

Even when the contextualist claim is not \emph{obviously} true, as with
`local' and `enemy', disquotational reports are unacceptable after
context shifts.

\begin{enumerate}
\def\labelenumi{(\arabic{enumi})}
\setcounter{enumi}{35}
\item
  (Brian is calling from Providence, Hud and Andy are in Bellingham)\\
  Brian: When I get all this work done, I'll head off to a local bar for
  some drinks.\\
  Andy: How much work is there?\\
  Brian: Not much. I should get to the bar in a couple of hours.\\
  Hud: Hey, is Brian in town? Where's he going tonight?\\
  Andy: *He thinks he'll be at a local bar in a couple of hours.
\item
  The Enemy, speaking of us: The enemy have the advantage.\\
  One of us: How are we doing?\\
  Another of us: *Someone just informed me that the enemy have the
  advantage.
\item
  (Terrell is an NFL player, and Dennis is his coach.)\\
  Terrell: Why are you cutting me coach?\\
  Dennis: Because you are old and slow.\\
  (After this Terrell returns to academia. Kate and Leopold are students
  in his department.)\\
  Kate: Do you think Terrell would do well on our department ultimate
  frisbee team?\\
  Leopold: ??I'm not sure. Someone thinks he's old and slow.
\end{enumerate}

This data provides us with the penultimate argument against the
contextualist theory of epistemic modals. We have already seen several
such arguments.

First, as seen through the difficulties with each of the options
discussed in section 2, \emph{any} version of contextualism faces
serious problems, though by altering the version of contextualism we are
using, we can alter what problems we have to face.

Second, there is nothing like the speaker-inclusion constraint for terms
like `local' and `enemy' for which contextualism is quite plausible.
This disanalogy tells against the contextualist theory of `might'. With
the right stage setting (and it doesn't usually take very much), we can
get `local' and `enemy' to mean \emph{local to x} and \emph{enemy of x}
for pretty much any x we happen to be interested in talking about. At
least for `bare' (unembedded) epistemic modals, the situation is
markedly different. We can't, just by making Jack salient, make our own
knowledge irrelevant to the truth of our utterance of, for example,
``Jack might have eight fingers.'' The only way we can make our
knowledge irrelevant is if we are using this sentence in an explanation
or justification of Jack's actions.\footnote{And then it would probably
  be more natural to say ``He might have eight fingers,'' but that's
  possibly for unrelated reasons.}

Third, there is a difference in behaviour between embedded and
unembedded occurrences of epistemic modals. When epistemic modals are
embedded in belief contexts, conditionals, etc., they behave
differently---the speaker inclusion constraint seems to be lifted.
(Think about belief reports and that military instructor case.) `Local'
and `enemy' don't seem to show any analogous difference in their
behaviour between their bare and embedded occurrences.

Fourth, `local' and `enemy' don't generate any of the peculiar phenomena
about willingness to agree. If Myles (still in Cleveland), says

\begin{description}
\tightlist
\item[(39)]
Many local bars are full of Browns fans.
\end{description}

Professor Granger (still in the South Pacific), will not hesitate to say
``that's right'' (as long as she knows that many bars in Cleveland
really are, as usual, full of Browns fans). The fact that the relevant
bars aren't local to \emph{her} doesn't interfere with her willingness
to agree with (39) in the way that the fact that \emph{she} knew that
she wasn't in Prague interfered with her willingness to agree with
Myles' claim that she might be in Prague, or in the way that Watson's
knowledge that Holmes was in London (should have) interfered with his
willingness to assent to Moriarty's claim that Holmes might be in Paris.

Fifth, when there is a context shift, we are generally hesitant to
produce belief reports by disquoting sincerely asserted sentences
involving contextually variable terms. This is what the examples (32)
through (36) show. For a wide range of contextually variable terms,
speakers will quite naturally hesitate to make disquotational reports
unless they are in the same context as the original speaker. Such
hesitation is not shown by speakers reporting epistemic modals.

The sixth argument, that there is an alternative theory that does not
have these flaws, will have to wait until the next section. For now,
let's note that there are other words that seem at first to be
contextually variable, but for which disquotational reports seem
acceptable.

\begin{enumerate}
\def\labelenumi{(\arabic{enumi})}
\setcounter{enumi}{39}
\item
  Vinny the Vulture: Rotting flesh tastes great.\\
  John: Vinny thinks that rotting flesh tastes great.
\item
  Ant Z: He's huge (said of 5 foot 3 141 lb NBA player Muggsy Bogues)\\
  Andy: Ant Z thinks that Muggsy's huge.
\item
  Marvin the Martian: These are the same colour (said of two colour
  swatches that look alike to Martians but not to humans.)\\
  Brian: Marvin thinks that these are the same colour.
\end{enumerate}

In all three cases the report is accurate, or at least extremely
natural. And in all three cases it would have been inappropriate for the
reporter to continue ``and he's right''. But crucially, in none of the
three cases is it \emph{clear} that the original speaker made a mistake.
In his context, it seems Vinny utters a truth by uttering, ``Rotting
flesh tastes great'', for rotting flesh does taste great to vultures.
From Ant Z's perspective, Muggsy Bogues is huge. We assume here, a
little controversially, that there is \emph{a} use of comparative
adjectives that is not relativised to a comparison class, but rather to
a perspective. Ant Z does not say that Muggsy is huge for a human, or
for an NBA player, but just relative to him. And he's right. Even Muggsy
is huge relative to an ant. Note the contrast with (36) here. There's
something quite odd about Leopold's statement, which intuitively means
that someone said Terrell is old and slow for a graduate student, when
all that was said was that he is old and slow for an NFL
player.\footnote{Or perhaps something more specific than that, such as
  that he is old and slow for a player at his position.} And, relative
to the Martian's classification of objects into colours, the two
swatches are the same colour. So there's something very odd going on
here.

The following \emph{very} plausible principle looks like it is being
violated.

\begin{description}
\tightlist
\item[Truth in Reporting]
If \emph{X} has a true belief, then \emph{Y}'s report \emph{X believes
that S} accurately reports that belief only if in the context \emph{Y}
is in, \emph{S} expresses a true proposition.\footnote{One might also
  consider a `says that' version of \textbf{Truth in Reporting:} If
  \emph{X} speaks true, then \emph{Y}'s report \emph{X says that S} is
  accurate only if in the context \emph{Y} is in, \emph{S} expresses a
  true proposition. This is more questionable, since it is questionable
  whether `says that' constructions must report what is semantically
  expressed by a speech, as opposed to what is merely communicated. See
  again the papers mentioned in footnote 25.}
\end{description}

Not only do our three reports here seem to constitute counterexamples to
\textbf{Truth in Reporting}, Watson's report in (30) is also such a
counterexample, if Moriarty speaks truly (and sincerely). One response
here would be to give up \textbf{Truth in Reporting}, but that seems
like a desperate measure. And we would still have the puzzle of why we
can't say ``and he's right'' at the end of an accurate report.

Another response to these peculiar phenomena would be to follow the
universalist and conclude that Moriarty, Vinny, Ant Z and Marvin all
believe something false. It should be clear how to formulate this kind
of position: something tastes great iff every creature thinks it tastes
great; something is huge iff it is huge relative to all observers; and
two things are the same colour iff they look alike (in a colour kind of
way) to every observer (in conditions that are normal for them). As we
saw, there are problems for the universalist move for epistemic modals.
And the attractiveness of the other universal seems to dissipate when we
consider the cases from a different perspective.

\begin{enumerate}
\def\labelenumi{(\arabic{enumi})}
\setcounter{enumi}{42}
\item
  Brian: Cognac tastes great.\\
  Vinny: Brian believes that cognac tastes great.
\item
  Andy:He's huge (said of Buggsy Mogues, the shortest ever player in the
  Dinosaur Basketball Association).\\
  Tyrone the T-Rex: Andy believes that Buggsy's huge.
\item
  John: These are the same colour (said of two colour swatches that look
  alike to humans but not to pigeons).\\
  Pete the Pigeon: John believes that these are the same colour.
\end{enumerate}

Again, every report seems acceptable, and in every case it would seem
strange for the reporter to continue ``and he's right.'' The
universalist explanation in every case is that the original utterance is
false. That certainly explains the data about reports, but look at the
cost! All of our utterances about colours and tastes will turn out
false, as will many of our utterances about sizes. It seems we have to
find a way to avoid both contextualism and universalism. Our final
suggestions for how to think about epistemic modals attempt to explain
all this data.

\section{Relativism and Centred
Worlds}\label{relativism-and-centred-worlds}

John MacFarlane (\citeproc{ref-MacFarlane2003-MACFCA-2}{2003}) has
argued that believers in a metaphysically open future should accept that
the truth of an utterance is \emph{relative} to a context of
evaluation.\footnote{We are very grateful in this section to extensive
  conversations with John MacFarlane. His
  (\citeproc{ref-MacFarlane2003-MACFCA-2}{2003}) was one of the main
  inspirations for the relativist theory discussed here. His (ms), which
  he was kind enough to show us a copy of while we were drafting this
  paper, develops the argument for a relativist approach to epistemic
  modals in greater detail than we do here. Mark Richard also has work
  in progress that develops a relativist view on related matters, which
  he has been kind enough to show us, and which has also influenced our
  thinking.} For example, if on Thursday Emily says, ``There will be a
sea battle tomorrow'', the believer in the open future wants to say that
at the time her utterance is neither determinate true nor determinately
false. One quick objection to this kind of theory is that if we look
back at Emily's statement while the sea battle is raging on Friday, we
are inclined to say that she got it right. From Friday's perspective, it
looks like what Emily said is true. The orthodox way to reconcile these
intuitions is that the only sense in which Emily's statement is
\emph{indeterminate} on Thursday is an \emph{epistemic} sense -- we
simply don't know whether there will be a sea battle. MacFarlane argues
instead that we should simply accept the intuitions as they stand. From
Friday's perspective, Emily's statement is determinately true, from
Thursday's it is not. Hence the truth of statements is relative to a
context of evaluation.

There is a natural extension of this theory to the cases described
above. Moriarty's statement is true relative to a context \emph{C} iff
it is compatible with what the people in \emph{C} know that Holmes is in
Paris. So in the context he uttered it, the statement is \emph{true},
because it is consistent with what everyone in his context knows that
Holmes is in Paris. But in the context of Watson's report, it is false,
because Watson and Holmes know that Holmes is not in Paris.

We will call any such theory of epistemic modals a \emph{relativist}
theory, because it says that the truth of an utterance containing an
epistemic modal is \emph{relative} to a context of evaluation. As we
will see, relativist theories do a much better job than contextualist
theories of handling the data that troubled contextualist theories.
Relativist theories are also plausible for the predicates we discussed
at the end of the last section: `huge', `color' and `tastes'. On such a
theory, any utterance that \emph{x tastes F} is true iff \emph{x} tastes
\emph{F} \emph{to us}. Similarly, an utterance \emph{x is huge} that
doesn't have a comparison class, as in (41) or (44), is true iff
\emph{x} is huge relative \emph{to us}. And \emph{Those swatches are the
same color} is true iff they look the same colour \emph{to us}. The
reference to us in the truth conditions of these sentences isn't because
there's a special reference to us in the lexical entry for any of these
worlds. Rather, the truth of any utterance involving these terms is
relative to a context of evaluation, and when that is \emph{our} context
of evaluation, \emph{we} get to determine what is true and what is
false. If the sentences were being evaluated in a different context, it
would be the standards of that context that mattered to their truth.

So far we have not talked about the pragmatics of epistemic modals,
assuming that their assertability conditions are given by their truth
conditions plus some familiar Gricean norms. But it is not obvious how
to apply some of those norms if utterance truth is contextually
relative, because one of the norms is that one should say only what is
true.

One option is to say that utterance appropriateness is, like utterance
truth, relative to a context of evaluation. This is consistent, but it
does not seem to respect the data. Watson might think that Moriarty's
utterance is false, at least relative to his context of
evaluation\footnote{We do not assume here that ordinary speakers, like
  Watson, explicitly make judgments about the truth of utterances
  relative to a context of evaluation, as such. They do make judgments
  about the truth of utterances, and those judgments are made in
  contexts, but they don't explicitly makes judgments of \emph{truth
  relative to context of evaluation}. One of the nice features, however,
  of the relativist account is that it is possible to do an attractive
  rational reconstruction of most of their views in terms of contexts.},
but if he is aware of Moriarty's epistemic state he should think it is
\emph{appropriate}. So if something like truth is a norm of assertion,
it must be truth relative to one or other context. But which one?

We could say that one should only say things that are true relative to
all contexts. But that would mean John's statement about the two
swatches being the same colour would be inappropriate, and that seems
wrong.

We could say that one should only say things that are true relative to
some contexts. But then Brian could have said, ``Rotting carcases taste
great'' and he would have said something appropriate, because that's
true when evaluated by vultures.

The correct norm is that one should only say something that's true when
evaluated in the context you are in. We assume here that contexts can
include more than just the speaker. If Vinny the Vulture is speaking to
a group of humans he arguably cannot say \emph{Rotting flesh tastes
great}. The reason is that rotting flesh does not taste great to the
group of speakers in the conversation, most of whom are humans. This
norm gives us the nice result that Myles's statement is appropriate, as
is Moriarty's, even though in each case their most prominent audience
member knows they speak falsely.\footnote{Can we even say that someone
  speaks falsely here now that truth and falsity is always relative to a
  context of evaluation? It turns out we can, indeed we must, although
  the matter is a little delicate. We return to this point below.}

This helps explain, we think, the somewhat ambivalent attitude we have
towards speakers who express epistemic modals that are false relative to
our context, but true relative to their own. What the speaker said
wasn't true, so we don't want to endorse what they said. Still, there's
still a distinction between such a speaker and someone who says that the
sky is green or that grass is blue. That speaker would violate the
properly relativised version of the \emph{only say true things} rule,
and Myles and Moriarty do not violate that rule.

As MacFarlane notes, relativist theories deny \textbf{Absoluteness of
Utterance Truth}, the claim that if an utterance is true relative to one
context of evaluation it is true relative to all of them. It is
uncontroversial of course that the truth value of an utterance
\emph{type} can be contextually variable, the interesting claim that
relativists make is that the truth value of utterance tokens can also be
different relative to different contexts. So they must deny one or more
premises in any argument for \textbf{Absoluteness of Utterance Truth,}
such as this one.

\begin{enumerate}
\def\labelenumi{\arabic{enumi}.}
\tightlist
\item
  \textbf{Absoluteness of Propositional Content}: If an utterance
  expresses the proposition \emph{p} relative to some context of
  evaluation, then it expresses that proposition relative to all
  contexts of evaluation.
\item
  \textbf{Absoluteness of Propositional Truth Value}: If a proposition
  \emph{p} is true relative to one context in a world it is true
  relative to all contexts in that world; therefore,
\item
  \textbf{Absoluteness of Utterance Truth}
\end{enumerate}

This argument provides a nice way of classifying relativist theories.
One relativist approach is to say that Moriarty (or anyone else who
utters an epistemic modal) says something different relative to each
context of evaluation. Call this approach \emph{content}
\emph{relativism}. Another approach is to say that there is a single
proposition that he expresses with respect to every context, but the
truth value of that proposition is contextually variable. Call this
approach \emph{truth} \emph{relativism}. (So that the meaning of
`proposition' is sufficiently understood here, let us stipulate that we
understand propositions to be the things that are believed and asserted
and thus, relatedly, the semantic values of `that'-clauses.)

It might look like some of our behaviour is directly inconsistent with
\emph{any} sort of relativism. Consider the following dialogue.

\begin{enumerate}
\def\labelenumi{(\arabic{enumi})}
\setcounter{enumi}{45}
\tightlist
\item
  Vinny: Rotting flesh tastes great\\
  Vinny's brother: That's true.\\
  John: That (i.e.~what Vinny's brother said) is not true.
\end{enumerate}

If what Vinny's brother is saying is that Vinny's utterance
\emph{Rotting flesh tastes great} is true in his context, then John is
\emph{wrong} in saying that what Vinny's brother said isn't true. For it
is true, we claim, that \emph{Rotting flesh tastes great} is true in
Vinny's context.\footnote{We assume here the vultures are talking mainly
  to other vultures, and John is talking mainly to other humans.} But
this prediction seems unfortunate, because John's utterance seems
perfectly appropriate in his context.

The solution here is to recognise a \emph{disquotational} concept of
truth, to go alongside the \emph{binary} concept of truth that is at the
heart of the relativist solution.\footnote{We are grateful to John
  Macfarlane for helpful correspondence that influenced what follows.}
The binary concept is a relation between an tterance and a context of
evaluation. Call this \emph{true\textsubscript{B}}. So Vinny's utterance
is true\textsubscript{B} relative to his context, and to his brother's
context, and false\textsubscript{B} relative to John's context. One
crucial feature of the binary concept is that it is not a relativist
concept. If it is true relative to one context that an utterance is
true\textsubscript{B} relative to context C, it is true relative to all
contexts that the utterance is true\textsubscript{B} relative to context
C. The disquotational concept is unary. Call this
\emph{true\textsubscript{T}}. As far as is permitted by the semantic
paradoxes, it claims that sentences of the form \emph{S is
true\textsubscript{T}} \emph{iff S} will be true\textsubscript{B}
relative to any context (note here the primacy of truth\textsubscript{B}
for semantic explanation) True\textsubscript{T} is a relative concept.
An utterance can be true\textsubscript{T} relative to C and not
true\textsubscript{T} relative to C′. When an utterance is given the
honorific \emph{true} in ordinary discourse, it is the unary relative
concept true\textsubscript{T} that is being applied. That explains what
is going on in (46). Vinny's brother says that Vinny's utterance is
true\textsubscript{T}. Relative to his context, that's right, since
Vinny's utterance is true in his context. But relative to John's
context, that's false, because an utterance is true\textsubscript{T}
relative to John's context iff it is true relative to John's context. So
John spoke truly relative to his own context, so he spoke correctly. The
important point is that assignments of truth\textsubscript{T} are
relative rather than contextually rigid, so they might be judged true
relative to some contexts and false relative to others.

Although both truth relativism and content relativism can explain (46)
if they help themselves to the distinction between
truth\textsubscript{B} and truth\textsubscript{T}, there are four major
problems for content relativism that seem to show it is not the correct
theory.

The first problem concerns embeddings of ``might'' clauses in belief
contexts. Suppose Watson says,

\begin{description}
\tightlist
\item[(47)]
Moriarty believes that Holmes might be in Paris.
\end{description}

On the content relativist view, (47) will say, relative to Watson, that
Moriarty believes that, as far as Watson knows, Holmes is in Paris. That
would be a crazy thing for Watson to assert. Suppose Watson is talking
to Holmes. Then, relative to Holmes, Watson will have claimed that
Moriarty believes that, as far as Holmes knows, Holmes is in Paris. That
would \emph{also} be a crazy thing for Watson to assert. But, given what
he's just overheard, it would be perfectly natural---and pretty clearly
\emph{correct}, so long as nothing funny is going on behind the
scenes---for Watson to assert (47). A view that tells us that Watson's
saying something crazy relative to everybody who's likely to be a member
of his audience is in pretty serious conflict with our pretheoretical
judgements about the case. (Enlarging the context to include both Holmes
and Watson obviously doesn't help, either.)

The second problem concerns the social function of assertion. In
particular, it causes difficulties for an attractive part of the
Stalnakerian story about assertion, that the central role of an
assertion is to add the proposition asserted to the stock of
conversational presuppositions (\citeproc{ref-Stalnaker1978}{Stalnaker
1978}). On the content relativist view, it can't be that the essential
effect of assertion is to add \emph{the proposition asserted} to the
stock of common presuppositions, because there's no such thing as
\emph{the} proposition asserted. There will be a different proposition
asserted relative to each audience member. That's not part of an
attractive theory. And it's not terribly clear what the replacement
story about the essential effect of assertion---about the fundamental
role of assertion in communication---is going to be. It may be that
there's a story to be told about \emph{assertability}---about when
Moriarty is entitled to assert, for example, ``it might be that Holmes
is in Paris''---but there's no obvious story about what he's \emph{up
to} when he's making that assertion---about what the assertion is
supposed to accomplish. (And if you think that appropriateness of
assertion's got to be tied up with what your assertion's supposed to
accomplish, then you'll be sceptical about even the first part.)

The third problem concerns epistemic modals in the scope of temporal
modifiers. The content relativist has difficulties explaining what's
going on with sentences like (48).

\begin{description}
\tightlist
\item[(48)]
The Trojans were hesitant in attacking because Achilles might have been
with the Greek army.
\end{description}

On the content relativist view, (48) will be false relative to pretty
much everybody---certainly relative to everybody alive today. It's
certainly false that the Trojans were hesitant because, as far as
\emph{we} know, Achilles was with the Greek army. (Or worse, because, as
far as we knew \emph{then}, Achilles was with the Greek army.) But,
depending on how the Trojan war went, (48) could be \emph{true} relative
to everybody.\footnote{We don't take any stand here on just how the war
  went, if it happened at all. The important point is that whether (48)
  is true when said of a particular battle is a wide-open empirical
  question, not one that can be settled by appeal to the semantics of
  \emph{might}. The content relativist says, falsely, that it can be
  thus settled.}

Finally, content relativism has a problem with commands. Keith's Mom
says:

\begin{description}
\tightlist
\item[(49)]
For all days \emph{d}, you should carry an umbrella on d if and only if
it might rain on \emph{d}.
\end{description}

Suppose on Monday Keith checks the forecast and it says there's a 50\%
chance of rain. So he takes an umbrella. It doesn't rain, and on Tuesday
he wonders whether what he did on Monday was what his Mom said he
should. On the content relativist view, we get the following strange
result: on Monday, it would have been true to say that he was doing what
his Mom said he should, since at the time, the embedded clause expressed
a proposition that was true relative to him. Looking back on Tuesday,
though, it looks like he did what his Mom said he shouldn't, because now
the embedded clause expresses a proposition that's false relative to
him. But that's not right. He just plain did what his Mom told him to
do.

The same thing happens with the soldiers trying to follow the imperative
issued as (22). Assume one of them attempts to follow the command by
burning down some trees that seem to contain snipers. Relative to the
time she is doing the burning, she will be complying with the command.
But later, when it turns out the trees were sniper-free, she will not
have been following the command. If we assume there's an overarching
command to not use flamethrowers unless explicitly instructed to do so,
then it will turn out that, as of \emph{now}, she violated her orders
\emph{then}. But that's not right. She just plain followed her orders.

There's a similar problem with the other terms about which relativism
seems plausible. Consider the following commands:

\begin{description}
\tightlist
\item[(50)]
Don't pick fights with huge opponents.
\item[(51)]
Stack all of the things that are the same color together.
\item[(52)]
If it tastes lousy, spit it out.
\end{description}

It's possible to sensibly issue these commands, even in relevantly mixed
company. And if we're going to get the right compliance conditions, we
don't want \emph{content} relativism about great-tastingness, hugeness,
and same-coloredness here. When we hear a command like (52), we take (a)
the same command to have been issued to everybody, and (b) everybody to
be following it if we all spit out the things that taste lousy to us. On
the content relativist view, we've each gotten different commands, and
the philosopher who spits out the chunk of week-old antelope hasn't
complied with the command that Vinny was given. This seems wrong.

So the content relativist theory has several problems. The truth
relativist theory does much better. Let us begin with the familiar
notion of a function from worlds to truth values. Call any such function
a Modal Profile. On the standard way of looking at things, propositions
-- the objects of belief and assertion, the semantic values of
`that'-clauses -- are, or at least determine a Modal Profile. The truth
relativist denies this. According to the truth-relativist, the relevant
propositions are true or false not relative to \emph{worlds}, but
relative to \emph{positions} within worlds---that is, they're true or
false relative to \emph{centered worlds}. (A centered world is a triple
of a possible world, an individual, and a time.) There's a few ways to
formally spell out this idea. One is to replace talk of Modal Profiles
with Centring Profiles, i.e.~functions from \emph{centred} worlds to
truth values. Another is to say that a centred world and proposition
combine to determine a Modal Profile, so propositions determine
functions from centred worlds to Modal Profiles. Each of these proposals
has some costs and benefits, and we postpone discussion of their
comparative virtues to an appendix. For now we are interested in the
idea, common to these proposals, that propositions only determine truth
values relative to something much more fine-grained than a world. (We
take no stand here on whether propositions should be \emph{identified}
with either Modal Profiles or Centering Profiles or functions from
Centred Worlds to Modal Profiles).

Truth relativism is not threatened by the four problems that undermine
content relativism.

According to truth relativism, Watson and Moriarty express the very same
proposition by the words \emph{Holmes might be in Paris}, so it is no
surprise that Watson can report Moriarty's assertive utterance by using
the very same words. Similarly, it is no surprise that if Moriarty has a
belief that he would express by saying \emph{Holmes might be in Paris},
Watson can report that by (53).

\begin{description}
\tightlist
\item[(53)]
Moriarty believes that Holmes might be in Paris.
\end{description}

Above we noted that it's unlikely that Watson could use this to express
the proposition that for all \emph{Watson} knows Holmes is in Paris. We
used that fact to argue that DeRose's constraint did not apply when an
epistemic modal is inside a propositional attitude report. The truth
relativist theory predicts not only that DeRose's constraint should not
apply, but that a different constraint should apply. When one says that
\emph{a believes that b might be F}, one says that \emph{a} believes the
proposition \emph{b might be F}. And \emph{a} believes that proposition
iff \emph{a} believes it is consistent with what they know that \emph{b}
is \emph{F}. And that prediction seems to be entirely correct. It is
impossible for Watson to use (53) to mean that Moriarty believes that
for all Holmes knows he is in Paris, or that for all Watson knows Holmes
is in Paris. This seems to be an interesting generalisation, and while
it falls out nicely from the truth relativist theory, it needs to be
imposed as a special constraint on contextualist theories.

Since there is a proposition that is common to speakers and hearers when
an epistemic modal is uttered, we can keep Stalnaker's nice idea that
the role of assertion is to add propositions to the conversational
context. Since propositions are no longer identified with sets of
possible worlds we will have to modify other parts of Stalnaker's
theory, but those parts are considerably more controversial.

The truth relativist can also explain how (48) can be true, though the
explanation requires a small detour through the nature of psychological
explanations involving relativist expressions go.

\begin{description}
\tightlist
\item[(48)]
The Trojans were hesitant in attacking because Achilles might have been
with the Greek army.
\end{description}

All of the following could be true, and not because the things in
question are rude, huge or great tasting for us.

\begin{description}
\tightlist
\item[(54)]
Marvin the Martian dropped his pants as the Queen passed by because it
would have been rude not to.
\item[(55)]
Children are scared of adults because they are huge.
\item[(56)]
Vultures eat rotting flesh because it tastes great.
\end{description}

In general it seems that the truth of an explanatory claim of the form,
\emph{X} \(\varphi\)ed because \emph{p} depends only on whether \emph{p}
is true in \emph{X}'s context (plus whether the truth of \emph{p} in
\emph{X's} context bears the right relation to \emph{X}'s
\(\varphi\)ing).. Whether or not \emph{p} is true in \emph{our} context
is neither here nor there. Adults are not huge, rotting flesh does not
taste great, and it is rude to drop one's pants as the Queen passes by,
but (54)-(56) could still be true, and could all count as good
explanations. Similarly, (48) can be true because \emph{Achilles might
have been with the Greek army} could be true relative to the Trojans.

Similarly, what it is to comply with a command is to act in a way that
makes the command true in the context of action. This is not a
particular feature of epistemic modals, but just a general property of
how commands involving propositions with centered-worlds truth
conditions behave. If Don picks a fight with Pedro after Don has shrunk
so much that Pedro is now relatively huge, he violates (50), even if
Pedro was not huge when the command was issued. And he still violates it
from a later perspective when Pedro and Don are the same size. The
general point is that whether the command is violated depends on the
applicability of the salient terms from the perspective of the person to
whom the command applies. Similarly, Keith does not violate his Mom's
command if he takes an umbrella where \emph{It might rain} is true in
the context the action is performed. And this, of course, matches up
perfectly with intuitions about the case.

It's a little tricky to say just which statement in Professor Granger's
original hexalemma gets denied by the truth relativist. It all depends
what we mean by \emph{spoke truly}. If \emph{Myles spoke truly} means
that Myles said something true\textsubscript{T}, then (2) is false
(relative to Granger's context), for its right-hand-side is true but its
left-hand-side is false. If, on the other hand, it means he said
something true\textsubscript{B} relative to his own context, then (4) is
false, for he did speak truly\textsubscript{B} relative to his context,
but it's not the case that Professor Granger might be in Prague. This is
awkward, but we might expect that any good solution to the paradox will
be awkward.

\section{Objections to Truth
Relativism}\label{objections-to-truth-relativism}

It might be thought that the truth relativist has to deny \textbf{Truth
in Reporting,} but in fact this can be retained in its entirety provided
we understand it the right way. The following situation is possible on
the truth relativist theory. \emph{X} has a belief that is true in her
context, and \emph{Y} properly reports this by saying \emph{X believes
that S}, where \emph{S} in \emph{Y}'s mouth expresses a proposition that
is false in \emph{Y}'s mouth in her context. But this is no violation of
\textbf{Truth in Reporting}. What would be a violation is if \emph{X}'s
belief was true in \emph{Y}'s context, and still \emph{Y} could report
it as described here. But there's no case where, intuitively, we
properly report an epistemic modal but violate that constraint. And the
same holds for reports of uses of \emph{huge}, \emph{color} or
\emph{tastes}. Even if Vinny (truly) believes that rotting flesh tastes
great, and the words ``Rotting flesh tastes great'' in John's mouth
express a false proposition, John's report, ``Vinny believes that
rotting flesh tastes great'' would only violate \textbf{Truth in
Reporting} if Vinny's belief is still true in \emph{John's} context. And
it is not.

Given that the relativist has the concept of truth\textsubscript{T}, or
as we might put it truth \emph{simpliciter}, what should be done with
it? The answer seems to be not much. We certainly shouldn't restate the
norms of assertion in terms of it, because that will lead to the
appropriateness of assertion being oddly relativised. Whether it was
appropriate for Vinny to \emph{say} ``Rotting flesh tastes great,'' is
independent of the context of evaluation, even if the truth of what he
uttered is context-relative. (It would not at all be appropriate for him
to have said ``Rotting flesh tastes terrible'' even though we should
think he would have said something true by that remark, and something
false by what he actually said.) And the same thing seems to hold for
generalisations about truth as the end of belief. It is entirely
appropriate for Myles to \emph{believe} that Granger might be in Prague,
because it's true\textsubscript{B} relative to his context. Relatedly,
if knowledge is tied to truth\textsubscript{T} rather than
truth\textsubscript{B}, knowledge can't be the norm of assertion or end
of belief.\footnote{~Arguably, then, one will have to distinguish (and
  posit an ordinary conflation between) knowledge\textsubscript{T} from
  knowledge\textsubscript{B,} the latter being needed to make good on
  the normative importance of knowledge, the former being need to make
  sense of the validity of the inference from knowing that p to p.~Is
  trouble lurking here for the truth relativist, especially given link
  between the truth\textsubscript{B} of `might' claims and facts about
  knowledge? We shall not pursue the matter further here.} On the other
hand, using truth\textsubscript{T} we can say that \textbf{Truth in
Reporting} is true in the truth relativist theory without reinterpreting
it in terms of relative truth concepts. Moreover, we can invoke
truth\textsubscript{T} to explain why we got confused when thinking
about the original puzzle: It is arguable that, even if we should
distinguish truth\textsubscript{T} from truth\textsubscript{B} in our
semantic theorizing, we aren't unreflectively as clear about that
distinction as we might be. No wonder then that we get a little confused
as we think about the Granger case. We want to say Myles doesn't make a
mistake. And we also want to say ``That's wrong'' speaking of the object
of his assertion and belief, and what's more, when we say that, we don't
seem to be making a binary claim about the relation between ourselves
and what is believes. Once we clearly distinguish truth\textsubscript{T}
from truth\textsubscript{B} things become clearly. Using the
disquotational notion, we can say `That is false\textsubscript{T}',
which is a monadic claim, and not a binary one. The binary
truth\textsubscript{B} explains why that claim is assertable (it is
assertable because `That is false\textsubscript{T}' is
truth\textsubscript{B} at my context), but doesn't figure in the
proposition believed. Meanwhile, the relevant notion of mistake -- that
of an agent believing a proposition that is not true\textsubscript{B} at
her context, can only be properly articulated once the distinction
between the more explanatory truth\textsubscript{T} is carefully
distinguished from the (arguably) conceptually more basic
truth\textsubscript{B.}

One final expository point. In general, truth relativism makes for
irresolvable disputes. Let us say that two conversational partners are
in deadlock concerning a claim when the following situation arises:
There is a pair of conversational participants, x and y, and a sentence
S, under dispute, such that each express the same proposition (in the
sense explained) by S but that S is true\textsubscript{B} at each of the
contexts x is in during the conversation, and false\textsubscript{B} at
each of the contexts y is in during the conversation. Neither speaks
past one another in alternately asserting and denying the same sentence,
since each expresses the same proposition by it. And each asserts what
they should be asserting when each says: What I say is
truth\textsubscript{T} and what the other says is
false\textsubscript{T.,} since each makes a speech that is
true\textsubscript{T} at the respective contexts. In general, truth
relativism about a term will lead one to predict deadlock for certain
conversations, traceable to the truth relativity of the term. But in the
case of `might', it is arguable that conversation tends to force a
situation where, even if at the outset, a `might' sentence was true
relative to x and not to y (on account of the truth-relativity of the
`might' sentence), x and y will, in the course of engagement and
dispute, be quickly put into a pair of contexts which do not differ with
respect to truth\textsubscript{B} (unless the `might' sentence contained
other terms that themselves made for deadlock). This is not merely
because the conversational participants will, through testimony, pool
knowledge about the sentence embedded in the `might' claim. It is in any
case arguable that the relevant community whose body of knowledge
determines whether a `might' claim is true\textsubscript{B} at a context
always includes not just that of the person at that context but also
that of his conversational partners. In the special case of `might',
then, Truth Relativism may well generate far less by way of deadlock
than in other cases.

There are two primary objections to the truth relativist theory: it
doesn't quite handle all the cases and that it is too radical.

There are some cases that seem to tell directly against the truth
relativist position. Consider the case again of Tom and Sally stuck in a
maze. Sally knows the way out, but doesn't want to tell Tom. She says,
inter alia, (57), and does not seem to violate any \emph{semantic} norms
in doing so, even though she knows the exit is some other way.

\begin{description}
\tightlist
\item[(57)]
The exit might be that way.
\end{description}

This seems to directly contradict the relativist claim that the norm for
assertion is speaking truly in one's own context. We suspect that what's
going on here is that Sally is projecting herself into Tom's context.
She is, we think, merely trying to verbalise thoughts that are, or
should be, going through \emph{Tom's} head, rather than making a simple
assertion. As some evidence for this, note (as was mentioned above) that
it would be wrong to take (57) as evidence that Sally believes the exit
might be that way, whereas when a speaker asserts that \emph{p} that is
usually strong evidence that she believes that \emph{p}. It is
unfortunate for the relativist to have to appeal to something like
projection, but we think it is the simplest explanation of these cases
that any theorist can provide.

The idea that utterances have their truth value absolutely is
well-entrenched in contemporary semantics, so it should only be
overturned with caution. And it might be worried that once we add
another degree of relativisation, it will be open to relativise in all
sorts of directions. We are sensitive to these concerns, but we think
the virtues of the relativist theory, and the vices of the contextualist
and invariantist theories, provides a decent response to them.
Invariantist theories are simply implausible, and any contextualist
theory will have to include so many ad hoc conditions, conditions that
seem to be natural consequences of relativism, that there are
methodological considerations telling in favour of relativism. (Let us
be clear: we are not recommending a general preference for relativism
over contextualism in semantic theory. As we have been trying to make
clear, for example, the case of `might' is very different from, say, the
case of `ready'.) It is (as always) hard to tell which way the balance
tips when all these methodological considerations are weighed together,
but we think the relativist has a good case.

\section*{Appendix on Types of
Content}\label{appendix-on-types-of-content}
\addcontentsline{toc}{section}{Appendix on Types of Content}

Robert Stalnaker has long promoted the idea that the content of an
assertoric utterance is a set of possible worlds, or a function from
worlds to truth values. This idea has been enormously influential in
formal semantics, although it has come in for detailed criticism by
various philosophers. (See especially Soames
(\citeproc{ref-Soames1987}{1987}) and King
(\citeproc{ref-King1994}{1994}, \citeproc{ref-King1995}{1995},
\citeproc{ref-King1998}{1998}).) But even philosophers who think that
there is more to content than a set of possible worlds would agree that
propositions determine a function from worlds to truth values. Some
would agree that such a function exhausts the `discriminatory role' of a
proposition, although this depends on the (highly contestable)
assumption that the role of propositions is to discriminate amongst
\emph{metaphysical} possibilities. Still, even philosophers who disagree
with what Stalnaker says about the nature of propositions could agree
that if all we wanted from a proposition was to divide up some
metaphysical possibilities, propositions could be functions from worlds
to truth values, but they think some propositions that divide up the
metaphysical possibilities the same way should be distinguished.

We don't want to take sides in that debate, because our truth relativism
means we are in conflict with even the idea that a proposition
determines a function from worlds to truth values. To see this, consider
a sentence whose truth value is relative to a context of evaluation,
such as \emph{Vegemite tastes great}. The truth relativist says that
this sentence should be evaluated as true from a context where people
like the taste of Vegemite (call this the Australian context) and should
be evaluated as false from a context where people dislike this taste
(call that the American context) and both evaluations are correct (from
their own perspective) even though the Australians and Americans agree
about what the content of \emph{Vegemite tastes great} is, and they are
in the same world. So there's just no such thing as \emph{the} truth
value of \emph{Vegemite tastes great} in the actual world, so it does
not determine a function from worlds to truth values. What kind of
function does it determine then?

One option, inspired by Lewis's work on \emph{de se} belief, is to say
that it determines a function from centred worlds to truth values. The
idea is that we can identify a context of evaluation with a centred
world, and then \emph{Vegemite tastes great} will be true relative to a
centred world iff it is properly evaluated as true within that context.
Alternatively, the content of \emph{Vegemite tastes great} will
determine a set of centred worlds, the set of contexts from which that
sentence would be evaluated as true. Just as propositions were
traditionally thought to determine (or be) sets of possible worlds,
properties were traditionally thought to determine (or be) functions
from worlds to sets of individuals.\footnote{~Lewis preferred the theory
  on which properties were sets of individuals, potentially from
  different worlds. This theory has difficulties accounting for
  individuals that exist in more than one world. And since properties
  exist in more than one world, and properties have to be treated as
  individuals in some contexts (e.g.~when they are the subjects of
  predication) this is a serious problem. Treating properties as
  functions from worlds to sets of individuals removes this problem
  without introducing any other costs. (See Egan
  (\citeproc{ref-Egan2004-JACSPA-2}{2004}) for more details.)} Now if we
identify centred worlds with \(\langle\)individual, world\(\rangle\)
pairs, a function from worlds to sets of individuals just is a set of
centred worlds.\footnote{Matters are a little more complicated when we
  introduce times into the story. For purposes of this appendix we
  ignore all matters to do with tense. As you'll see, the story is
  complicated enough as it is, and this omission doesn't seriously
  affect the dialectic to follow.} So the content of \emph{Vegemite
tastes great} could just be a \textbf{property}, very roughly the
property of being in a context where most people are disposed to find
Vegemite great-tasting.

This proposal has three nice features. First, even though the content of
\emph{Vegemite tastes great} is not, and does not even determine, a
proposition as Stalnaker conceived of propositions, it does determine a
property. So the proposal is not as radical as it might at first look.
Second, properties are the kind of thing that divide up possibilities.
The possibilities they divide are individuals, not worlds, but the basic
idea that to represent is to represent yourself as being in one class of
possible states rather than another is retained. The only change is that
instead of representing yourself as being in one class of worlds rather
than another, you represent yourself as being in one class of
\(\langle\)individual, world\(\rangle\) pairs rather than another.
Third, the proposal links up nicely with David Lewis's account of
\emph{de se} belief, and offers some prospects for connecting the
contents of beliefs with the contents of assertions, even when both of
these contents have ceased to be propositions in Stalnaker's
sense.\footnote{It might be that propositions just are whatever things
  are the contents of assertions and beliefs, so we shouldn't say that
  the contents of sentences like \emph{Vegemite tastes great} are not
  propositions. But they will be very different kinds of propositions to
  what we are used to. Thanks here to John MacFarlane.}

But there's a problem for this account. Consider what we want to say
about \emph{Possibly Vegemite tastes great}, where context makes it
clear that the `possibly' is a metaphysical modal. There's a trivial
problem and a potentially deep problem for this account. The trivial
problem is that we know what the meaning of \emph{possibly} is. It's a
function that takes propositions as inputs and delivers as output a
proposition that is true iff the input proposition is true at an
accessible world. If the content of \emph{Vegemite tastes great} is a
property rather than a proposition, then we have a type-mismatch. This
is a trivial problem because it's a fairly routine exercise to convert
the meanings of words like \emph{possibly} so they are the right kind of
things to operate on what we now take the meaning of \emph{Vegemite
tastes great} to be.

The deep problem is that when we go through that routine exercise, we
get the wrong results. We don't want \emph{Possibly Vegemite tastes
great} to be true in virtue of there being an accessible world where the
people \emph{there} like the taste of Vegemite. We want it to be true in
virtue of there being a world where Vegemite's taste is a taste that in
this context we'd properly describe as great. And it's not clear how to
get that on the current story. To see how big a problem this is,
consider (58), where the modal is meant to be metaphysical and have wide
scope.

\begin{description}
\tightlist
\item[(58)]
Possibly everyone hates Vegemite but it tastes great.
\end{description}

That's true, on its most natural reading. But the content of
\emph{Everyone hates Vegemite but it tastes great} will be the empty set
of centred worlds, for there is no centred world on which this is true.
Now it's not clear just what the meaning of \emph{possibly} could be
that delivers the correct result that (58) is true.

So we are tempted to consider an alternative proposal. Start with a very
natural way of thinking about why the relativist has to modify the
Stalnakerian story about content. The problem is that (even given a
context of utterance) \emph{tastes great} does not determine a property.
Rather, relative to any context of evaluation, i.e.~centred world, it
determines a property. That is, its content is (or at least determines)
a function from centred worlds to properties. So given our actual
context, it determines the property of having a taste that people around
here think is great. Now properties combine with individuals to form
Stalnakerian propositions. So \emph{tastes great} is a function from
centred worlds to functions from individuals to sets of worlds. Hence
\emph{Vegemite tastes great} is a function from centred worlds to sets
of worlds, the previous function with the value for the `individual'
being fixed as Vegemite.

Our second option then is that in general that sentences containing
`relative' terms like `tastes' or `huge' or `might' determines a
function from centred worlds to sets of worlds. This makes it quite easy
to understand how (58) could work. \emph{Possibly} type-shifts so that
it is now a function from functions from centred worlds to sets of
worlds to functions from centred worlds to sets of worlds. It's fairly
easy to say what this function is. If the content of \emph{p} is (or
determines) \emph{f}, a function from centred worlds to sets of worlds,
then the content of \(\lozenge p\) is (or determines) \emph{g}, the
function such that for any centred world \emph{c},
\emph{w}~\({\in}\)~\emph{g}(\emph{c}) iff for some \emph{w}′ accessible
from \emph{w}, \emph{w}′ \({\in}\)~\emph{f}(\emph{c}). The core idea is
just that we ignore the role of the centred worlds until the end of our
semantic evaluation, and otherwise just treat \(\lozenge\) as we'd
treated it in traditional semantics. This is a rather nice position in
many ways, but there are two issues to be addressed.

First, it is not clear that functions from centred worlds to sets of
worlds are really kinds of content. They are not things that divide up
intuitive possibilities, in the way that sets of individuals, and sets
of \(\langle\)individual, world\(\rangle\) pairs do. It's no good to say
that relative to a centred world a content is determined. That would be
fine if we were content relativists, and we said the content was meant
to be determined relative to a centred world. But as argued in the text
the content of \emph{Vegemite tastes great} should be the same across
various contexts of evaluation. A better response is to say functions
from centred worlds to sets of worlds do determine a kind of content.
For any such function \emph{f}, we can determine the set of centred
worlds \(\langle\)\emph{i}, \emph{w}\(\rangle\) such that
\emph{w}~\({\in}\)~\emph{f}(\(\langle\)\emph{i}, \emph{w}\(\rangle\)).
These will be the centred worlds that the proposition is true at. It's
not necessarily a problem that the proposition does \emph{more} than
determine this set. (It's not an objection to King's account of
propositions that on his theory propositions do more than determine a
set of possibilities.)

Second, it isn't exactly clear how to fill out these functions when we
get back to our core case: epistemic modals. It's easy to say what it is
for \emph{Vegemite tastes great} to be true in a world relative to our
context of evaluation; indeed we did so above. It's a lot harder to say
what it is for \emph{Granger might be in Prague} to be true in an
arbitrary world \emph{w} relative to an arbitrary context of evaluation
\emph{c}. As a first pass, we might say this is true in \emph{w} iff for
all the people in \emph{c} know, it is true in \emph{w} that Granger is
in Prague. But the problem is that whenever \emph{c} is not a centre in
\emph{w}, it's very hard to say just what the people in \emph{c} know
about \emph{w}. Under different descriptions of \emph{w} they will know
different things about it. If \emph{w} is described as a nearby world in
which Granger is in Cleveland, they will know Granger is not in Prague
in \emph{w}. If it is described as a nearby world in which Myles knows
where Granger is they may not know anything about whether Granger is in
Prague is in \emph{w}, even if those descriptions pick out the same
worlds. Ideally we would cut through this by talking about their
\emph{de re} knowledge about \emph{w}, but most folks have very little
\emph{de} \emph{re} knowledge about other possible worlds. It's not
clear this is a huge problem though. Remember that a sentence containing
an epistemic modal is meant to determine a function from centred worlds
to functions from worlds to truth values. Provided we have a semantics
that allows for semantic indeterminacy, we can just say that the
functions from worlds to truth values are partial functions, and they
simply aren't determined when it's unclear what the people in \emph{c}
know about \emph{w}. Or we can say there's a default semantic rule such
that \emph{w} is not in \emph{f}(\emph{c}) (where \emph{f} is the
function determined by the sentence) whenever this is unclear. Since the
sentences whose meanings are determined by these values of the function,
like \emph{Possibly Granger might be in Prague} are similarly vague, it
is no harm if the function is a little vague.

So we have two options on the table for what kind of functions sentences
might determine if they don't determine functions from world to truth
values. One option is that they determine functions from centred worlds
to truth values, another that they determine functions from centred
worlds to functions from worlds to truth values. Neither is free from
criticism, and the authors aren't in agreement about which is the best
approach, so it isn't entirely clear what the best way to formally
implement truth relativism is. But it does not look like there are no
possible moves here. Moving to truth relativism does not mean that we
will have to totally abandon the fruitful approaches to formal semantics
that are built on ideas like Stalnaker's, although it does mean that
those semantic theories will need to be modified in places.

\subsection*{References}\label{references}
\addcontentsline{toc}{subsection}{References}

\phantomsection\label{refs}
\begin{CSLReferences}{1}{0}
\bibitem[\citeproctext]{ref-Cappelen1997}
Capellen, Herman, and Ernest Lepore. 1997. {``On an Alleged Connection
Between Indirect Quotation and Semantic Theory.''} \emph{Mind and
Language} 12 (3-4): 278--96. doi:
\href{https://doi.org/10.1111/j.1468-0017.1997.tb00075.x}{10.1111/j.1468-0017.1997.tb00075.x}.

\bibitem[\citeproctext]{ref-DeRose1991}
DeRose, Keith. 1991. {``Epistemic Possibilities.''} \emph{Philosophical
Review} 100 (4): 581--605. doi:
\href{https://doi.org/10.2307/2185175}{10.2307/2185175}.

\bibitem[\citeproctext]{ref-DeRose1998}
---------. 1998. {``Simple Might's, Indicative Possibilities, and the
Open Future.''} \emph{The Philosophical Quarterly} 48 (190): 67--82.
doi:
\href{https://doi.org/10.1111/1467-9213.00082}{10.1111/1467-9213.00082}.

\bibitem[\citeproctext]{ref-Egan2004-JACSPA-2}
Egan, Andy. 2004. {``Second-Order Predication and the Metaphysics of
Properties.''} \emph{Australasian Journal of Philosophy} 82 (1): 48--66.
doi: \href{https://doi.org/10.1080/713659803}{10.1080/713659803}.

\bibitem[\citeproctext]{ref-vonFintel2003}
Fintel, Kai von, and Sabine Iatridou. 2003. {``Epistemic Containment.''}
\emph{Linguistic Inquiry} 34 (2): 173--98. doi:
\href{https://doi.org/10.1162/002438903321663370}{10.1162/002438903321663370}.

\bibitem[\citeproctext]{ref-Hawthorne2004}
Hawthorne, John. 2004. \emph{Knowledge and Lotteries}. Oxford: Oxford
University Press.

\bibitem[\citeproctext]{ref-Horn1989}
Horn, Laurence. 1989. \emph{A Natural History of Negation}. Chicago:
University of Chicago Press.

\bibitem[\citeproctext]{ref-Jacobson1999}
Jacobson, Pauline. 1999. {``Towards a Variable Free Semantics.''}
\emph{Linguistics and Philosophy} 22 (2): 117--84. doi:
\href{https://doi.org/10.1023/A:1005464228727}{10.1023/A:1005464228727}.

\bibitem[\citeproctext]{ref-King1994}
King, Jeffrey. 1994. {``Can Propositions Be Naturalistically
Acceptable?''} \emph{Midwest Studies in Philosophy} 19 (1): 53--75. doi:
\href{https://doi.org/10.1111/j.1475-4975.1994.tb00279.x}{10.1111/j.1475-4975.1994.tb00279.x}.

\bibitem[\citeproctext]{ref-King1995}
---------. 1995. {``Structured Propositions and Complex Predicates.''}
\emph{No{û}s} 29 (4): 516--35. doi:
\href{https://doi.org/10.2307/2216285}{10.2307/2216285}.

\bibitem[\citeproctext]{ref-King1998}
---------. 1998. {``What Is a Philosophical Analysis?''}
\emph{Philosophical Studies} 90 (2): 155--79. doi:
\href{https://doi.org/10.1023/A:1004254128428}{10.1023/A:1004254128428}.

\bibitem[\citeproctext]{ref-Lewis1976d}
Lewis, David. 1976. {``The Paradoxes of Time Travel.''} \emph{American
Philosophical Quarterly} 13 (2): 145--52. Reprinted in his
\emph{Philosophical Papers}, Volume 2, Oxford: Oxford University Press,
1986, 67-80. References to reprint.

\bibitem[\citeproctext]{ref-Lewis1979f}
---------. 1979. {``Scorekeeping in a Language Game.''} \emph{Journal of
Philosophical Logic} 8 (1): 339--59. doi:
\href{https://doi.org/10.1007/bf00258436}{10.1007/bf00258436}. Reprinted
in his \emph{Philosophical Papers}, Volume 1, Oxford: Oxford University
Press, 1983, 233-249. References to reprint.

\bibitem[\citeproctext]{ref-MacFarlane2003-MACFCA-2}
MacFarlane, John. 2003. {``{Future Contingents and Relative Truth}.''}
\emph{The Philosophical Quarterly} 53 (212): 321--36. doi:
\href{https://doi.org/10.1111/1467-9213.00315}{10.1111/1467-9213.00315}.

\bibitem[\citeproctext]{ref-Pagin2005}
Pagin, Peter. 2005. {``Compositionality and Context.''} In
\emph{Contextualism in Philosophy: Knowledge, Meaning, and Truth},
edited by Gerhard Preyer and Georg Peter, 303--48. Oxford: Oxford
University Press.

\bibitem[\citeproctext]{ref-Soames1987}
Soames, Scott. 1987. {``Direct Reference, Propositional Attitudes and
Semantic Content.''} \emph{Philosophial Topics} 15 (1): 47--87. doi:
\href{https://doi.org/10.5840/philtopics198715112}{10.5840/philtopics198715112}.

\bibitem[\citeproctext]{ref-Soames2002}
---------. 2002. \emph{Beyond Rigidity}. Oxford: Oxford University
Press.

\bibitem[\citeproctext]{ref-Stalnaker1978}
Stalnaker, Robert. 1978. {``Assertion.''} \emph{Syntax and Semantics} 9:
315--32.

\bibitem[\citeproctext]{ref-Stanley2000-STACAL}
Stanley, Jason. 2000. {``{Context and Logical Form}.''}
\emph{Linguistics and Philosophy} 23 (4): 391--434. doi:
\href{https://doi.org/10.1023/A:1005599312747}{10.1023/A:1005599312747}.

\bibitem[\citeproctext]{ref-Stanley2000-STAOQD}
Stanley, Jason, and Zoltán Gendler Szabó. 2000. {``{On Quantifier Domain
Restriction}.''} \emph{Mind and Language} 15 (2\&3): 219--61. doi:
\href{https://doi.org/10.1111/1468-0017.00130}{10.1111/1468-0017.00130}.

\end{CSLReferences}



\noindent Published in\emph{
Contextualism in Philosophy}, 2005, pp. 131-168.

\end{document}
