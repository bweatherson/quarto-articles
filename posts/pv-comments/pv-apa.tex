% Options for packages loaded elsewhere
% Options for packages loaded elsewhere
\PassOptionsToPackage{unicode}{hyperref}
\PassOptionsToPackage{hyphens}{url}
%
\documentclass[
  11pt,
  letterpaper,
  DIV=11,
  numbers=noendperiod,
  twoside]{scrartcl}
\usepackage{xcolor}
\usepackage[left=1.1in, right=1in, top=0.8in, bottom=0.8in,
paperheight=9.5in, paperwidth=7in, includemp=TRUE, marginparwidth=0in,
marginparsep=0in]{geometry}
\usepackage{amsmath,amssymb}
\setcounter{secnumdepth}{-\maxdimen} % remove section numbering
\usepackage{iftex}
\ifPDFTeX
  \usepackage[T1]{fontenc}
  \usepackage[utf8]{inputenc}
  \usepackage{textcomp} % provide euro and other symbols
\else % if luatex or xetex
  \usepackage{unicode-math} % this also loads fontspec
  \defaultfontfeatures{Scale=MatchLowercase}
  \defaultfontfeatures[\rmfamily]{Ligatures=TeX,Scale=1}
\fi
\usepackage{lmodern}
\ifPDFTeX\else
  % xetex/luatex font selection
  \setmathfont[]{Garamond-Math}
\fi
% Use upquote if available, for straight quotes in verbatim environments
\IfFileExists{upquote.sty}{\usepackage{upquote}}{}
\IfFileExists{microtype.sty}{% use microtype if available
  \usepackage[]{microtype}
  \UseMicrotypeSet[protrusion]{basicmath} % disable protrusion for tt fonts
}{}
\usepackage{setspace}
% Make \paragraph and \subparagraph free-standing
\makeatletter
\ifx\paragraph\undefined\else
  \let\oldparagraph\paragraph
  \renewcommand{\paragraph}{
    \@ifstar
      \xxxParagraphStar
      \xxxParagraphNoStar
  }
  \newcommand{\xxxParagraphStar}[1]{\oldparagraph*{#1}\mbox{}}
  \newcommand{\xxxParagraphNoStar}[1]{\oldparagraph{#1}\mbox{}}
\fi
\ifx\subparagraph\undefined\else
  \let\oldsubparagraph\subparagraph
  \renewcommand{\subparagraph}{
    \@ifstar
      \xxxSubParagraphStar
      \xxxSubParagraphNoStar
  }
  \newcommand{\xxxSubParagraphStar}[1]{\oldsubparagraph*{#1}\mbox{}}
  \newcommand{\xxxSubParagraphNoStar}[1]{\oldsubparagraph{#1}\mbox{}}
\fi
\makeatother


\usepackage{longtable,booktabs,array}
\usepackage{calc} % for calculating minipage widths
% Correct order of tables after \paragraph or \subparagraph
\usepackage{etoolbox}
\makeatletter
\patchcmd\longtable{\par}{\if@noskipsec\mbox{}\fi\par}{}{}
\makeatother
% Allow footnotes in longtable head/foot
\IfFileExists{footnotehyper.sty}{\usepackage{footnotehyper}}{\usepackage{footnote}}
\makesavenoteenv{longtable}
\usepackage{graphicx}
\makeatletter
\newsavebox\pandoc@box
\newcommand*\pandocbounded[1]{% scales image to fit in text height/width
  \sbox\pandoc@box{#1}%
  \Gscale@div\@tempa{\textheight}{\dimexpr\ht\pandoc@box+\dp\pandoc@box\relax}%
  \Gscale@div\@tempb{\linewidth}{\wd\pandoc@box}%
  \ifdim\@tempb\p@<\@tempa\p@\let\@tempa\@tempb\fi% select the smaller of both
  \ifdim\@tempa\p@<\p@\scalebox{\@tempa}{\usebox\pandoc@box}%
  \else\usebox{\pandoc@box}%
  \fi%
}
% Set default figure placement to htbp
\def\fps@figure{htbp}
\makeatother


% definitions for citeproc citations
\NewDocumentCommand\citeproctext{}{}
\NewDocumentCommand\citeproc{mm}{%
  \begingroup\def\citeproctext{#2}\cite{#1}\endgroup}
\makeatletter
 % allow citations to break across lines
 \let\@cite@ofmt\@firstofone
 % avoid brackets around text for \cite:
 \def\@biblabel#1{}
 \def\@cite#1#2{{#1\if@tempswa , #2\fi}}
\makeatother
\newlength{\cslhangindent}
\setlength{\cslhangindent}{1.5em}
\newlength{\csllabelwidth}
\setlength{\csllabelwidth}{3em}
\newenvironment{CSLReferences}[2] % #1 hanging-indent, #2 entry-spacing
 {\begin{list}{}{%
  \setlength{\itemindent}{0pt}
  \setlength{\leftmargin}{0pt}
  \setlength{\parsep}{0pt}
  % turn on hanging indent if param 1 is 1
  \ifodd #1
   \setlength{\leftmargin}{\cslhangindent}
   \setlength{\itemindent}{-1\cslhangindent}
  \fi
  % set entry spacing
  \setlength{\itemsep}{#2\baselineskip}}}
 {\end{list}}
\usepackage{calc}
\newcommand{\CSLBlock}[1]{\hfill\break\parbox[t]{\linewidth}{\strut\ignorespaces#1\strut}}
\newcommand{\CSLLeftMargin}[1]{\parbox[t]{\csllabelwidth}{\strut#1\strut}}
\newcommand{\CSLRightInline}[1]{\parbox[t]{\linewidth - \csllabelwidth}{\strut#1\strut}}
\newcommand{\CSLIndent}[1]{\hspace{\cslhangindent}#1}



\setlength{\emergencystretch}{3em} % prevent overfull lines

\providecommand{\tightlist}{%
  \setlength{\itemsep}{0pt}\setlength{\parskip}{0pt}}



 


\setlength\heavyrulewidth{0ex}
\setlength\lightrulewidth{0ex}
\usepackage[automark]{scrlayer-scrpage}
\clearpairofpagestyles
\cehead{
  Brian Weatherson
  }
\cohead{
  Comments on Property Versatility and Copredication
  }
\ohead{\bfseries \pagemark}
\cfoot{}
\makeatletter
\newcommand*\NoIndentAfterEnv[1]{%
  \AfterEndEnvironment{#1}{\par\@afterindentfalse\@afterheading}}
\makeatother
\NoIndentAfterEnv{itemize}
\NoIndentAfterEnv{enumerate}
\NoIndentAfterEnv{description}
\NoIndentAfterEnv{quote}
\NoIndentAfterEnv{equation}
\NoIndentAfterEnv{longtable}
\NoIndentAfterEnv{abstract}
\renewenvironment{abstract}
 {\vspace{-1.25cm}
 \quotation\small\noindent\emph{Abstract}:}
 {\endquotation}
\newfontfamily\tfont{EB Garamond}
\addtokomafont{disposition}{\rmfamily}
\addtokomafont{title}{\normalfont\itshape}
\let\footnoterule\relax

\makeatletter
\renewcommand{\@maketitle}{%
  \newpage
  \null
  \vskip 2em%
  \begin{center}%
  \let \footnote \thanks
    {\itshape\huge\@title \par}%
    \vskip 0.5em%  % Reduced from default
    {\large
      \lineskip 0.3em%  % Reduced from default 0.5em
      \begin{tabular}[t]{c}%
        \@author
      \end{tabular}\par}%
    \vskip 0.5em%  % Reduced from default
    {\large \@date}%
  \end{center}%
  \par
  }
\makeatother
\RequirePackage{lettrine}

\renewenvironment{abstract}
 {\quotation\small\noindent\emph{Abstract}:}
 {\endquotation\vspace{-0.02cm}}

\setmainfont{EB Garamond Math}[
  BoldFont = {EB Garamond SemiBold},
  ItalicFont = {EB Garamond Italic},
  RawFeature = {+smcp},
]

\newfontfamily\scfont{EB Garamond Regular}[RawFeature=+smcp]
\renewcommand{\textsc}[1]{{\scfont #1}}

\renewcommand{\LettrineTextFont}{\scfont}
\KOMAoption{captions}{tableheading}
\makeatletter
\@ifpackageloaded{caption}{}{\usepackage{caption}}
\AtBeginDocument{%
\ifdefined\contentsname
  \renewcommand*\contentsname{Table of contents}
\else
  \newcommand\contentsname{Table of contents}
\fi
\ifdefined\listfigurename
  \renewcommand*\listfigurename{List of Figures}
\else
  \newcommand\listfigurename{List of Figures}
\fi
\ifdefined\listtablename
  \renewcommand*\listtablename{List of Tables}
\else
  \newcommand\listtablename{List of Tables}
\fi
\ifdefined\figurename
  \renewcommand*\figurename{Figure}
\else
  \newcommand\figurename{Figure}
\fi
\ifdefined\tablename
  \renewcommand*\tablename{Table}
\else
  \newcommand\tablename{Table}
\fi
}
\@ifpackageloaded{float}{}{\usepackage{float}}
\floatstyle{ruled}
\@ifundefined{c@chapter}{\newfloat{codelisting}{h}{lop}}{\newfloat{codelisting}{h}{lop}[chapter]}
\floatname{codelisting}{Listing}
\newcommand*\listoflistings{\listof{codelisting}{List of Listings}}
\makeatother
\makeatletter
\makeatother
\makeatletter
\@ifpackageloaded{caption}{}{\usepackage{caption}}
\@ifpackageloaded{subcaption}{}{\usepackage{subcaption}}
\makeatother
\usepackage{bookmark}
\IfFileExists{xurl.sty}{\usepackage{xurl}}{} % add URL line breaks if available
\urlstyle{same}
\hypersetup{
  pdftitle={Comments on Property Versatility and Copredication},
  hidelinks,
  pdfcreator={LaTeX via pandoc}}


\title{Comments on \emph{Property Versatility and Copredication}}
\usepackage{etoolbox}
\makeatletter
\providecommand{\subtitle}[1]{% add subtitle to \maketitle
  \apptocmd{\@title}{\par {\large #1 \par}}{}{}
}
\makeatother
\subtitle{For symposium at 2026 APA Eastern Division Meeting}
\author{}
\date{2025}
\begin{document}
\maketitle
\begin{abstract}
I'm one of the commentators at an Author Meets Critics on David
Liebesman and Ofra Magidor's book \emph{Property Versatility and
Copredication} at the 2026 Eastern APA. This is a slightly longer
version of what I'm planning to say, including more worked out versions
of some examples that I'll have to cut for time.
\end{abstract}


\setstretch{1.1}
The first thing to say is that this is a great book.\footnote{The book,
  of course, is Liebesman and Magidor (\citeproc{ref-LM2025}{2025}).} I
loved reading it, working through the examples, and thinking about the
puzzles, and I strongly recommend it to people here. It's obviously
relevant to lots of people working in, or teaching, metaphysics and
philosophy of language. But there also turn out to be things relevant to
philosophy of mind, to aesthetics, and, this is what I'll be talking
about, philosophy of social science.

I'm not much of a critic since I'm sympathetic to the overall view. One
background theme to these remarks is that Leiebesman and Magidor
(hereafter, LM) aren't sufficiently radical. There will be two
foreground themes. One is that they need to get more physical; they give
physical objects short shrift at a couple of points. Another is that
once we have abstracta like i-books, plenitude results a la Fairchild
(\citeproc{ref-Fairchild2019}{2019}) are inevitable. And then things get
weird. Not that there's anything wrong with that; the social world is
weirder than it looks. But cataloging the weirdness is
valuable.\footnote{These notes are too long, and there are notes along
  the way about what I'm planning to cut from the presented version. But
  it was useful to me to have them written out.}

\section{Example 1: Box on Knox}\label{example-1-box-on-knox}

One guiding puzzle is explaining (1).

\begin{enumerate}
\def\labelenumi{(\arabic{enumi})}
\tightlist
\item
  There's a famous book on the shelf.
\end{enumerate}

Puzzle: If `book' means p-book (1) is false, since that p-book isn't
famous. But if `book' means i-book (1) is also false, since the i-book
isn't on the shelf. LM argue against that second claim. They say i-books
can be on shelves. But they are more sympathetic to the first. They
resist the idea that p-books can always inherit properties that are
grounded\footnote{I'm being a little cheeky here since I'm not sure
  they'd agree with expressing things in terms of grounding. But it
  seems to fit the picture they have, and it's easy enough to translate
  this into less metaphysically loaded talk.} in facts about i-books.
The evidence for this is that (2) is defective.

\begin{enumerate}
\def\labelenumi{(\arabic{enumi})}
\setcounter{enumi}{1}
\tightlist
\item
  Three wrinkled books on the shelf are bestselling.
\end{enumerate}

That just seems like an awkward locution to me, and not evidence
especially about p-books. (3) is just as defective, even though it's
clearly about i-books.

\begin{enumerate}
\def\labelenumi{(\arabic{enumi})}
\setcounter{enumi}{2}
\tightlist
\item
  Three books featuring Inspector Rebus are bestselling.
\end{enumerate}

We can give an independent argument that p-books can inherit properties
freely. Imagine my wife and I are moving house. There are plenty of
books that we both have in virtue of being cultured people, so our joint
collection has some duplicates. The books go in boxes, and each box has
50 p-books and 40 i-books. Now how should we arrange the boxes? There
are boring options, like sorting them by genre, chronologically,
alphabetically by author, alphabetically by title, etc. But it's more
fun to sort them by sales figures, or by famous, or comedic value.
Depending on what sort order we choose, one of (4) to (6) will be true
of the first box.

\begin{enumerate}
\def\labelenumi{(\arabic{enumi})}
\setcounter{enumi}{3}
\tightlist
\item
  This box has 50 bestselling books.
\item
  This box has 50 famous books.
\item
  This box has 50 funny books.
\end{enumerate}

In those sentences, `book' must be p-book, because there are only 40
i-books per box. But really any feature of i-book could be used in the
penultimate word. So p-books inherit i-book features generally.

\section{Example 2: Lost in
Translations}\label{example-2-lost-in-translations}

Another core example of the book involves duplicate books. My teenager's
Homer shelf has two copies of the \emph{Odyssey} and one of the
\emph{Iliad}. Intuitively both (7) and (8) have true readings.

\begin{enumerate}
\def\labelenumi{(\arabic{enumi})}
\setcounter{enumi}{6}
\tightlist
\item
  Their shelf has exactly two books on it.
\item
  Their shelf has exactly three books on it.
\end{enumerate}

My Homer shelf is a more complicated case. It only has the
\emph{Odyssey}, but it has two copies of the Fagles translation and one
of the Wilson. It seems to me that all of (9) to (11) have true
readings, and I think from what LM say in §9.3 they agree.

\begin{enumerate}
\def\labelenumi{(\arabic{enumi})}
\setcounter{enumi}{8}
\tightlist
\item
  My shelf has exactly one book on it, the \emph{Odyssey}.
\item
  My shelf has exactly two books on it, the Fagles translation and the
  Wilson translation
\item
  My shelf has exactly three books on it, which I can throw at those
  beer bottles.
\end{enumerate}

The solution is to say that the \emph{Odyssey} is an i-book, and so are
the translations of it. That seems like a good step to me, but there are
two questions.

One concerns my wife's Homer shelf, which has one p-book on it, the
Wilson translation of the \emph{Odyssey}. Now (12) is true, but (13) is
false, even though there are two i-books, the \emph{Odyssey} and the
Wilson translation of the \emph{Odyssey} on her shelf.

\begin{enumerate}
\def\labelenumi{(\arabic{enumi})}
\setcounter{enumi}{11}
\tightlist
\item
  Her shelf has exactly one book on it.
\item
  Her shelf has exactly two books on it.
\end{enumerate}

Here's a challenge. It's not immediately an objection, but if it can't
be answered it would be. Given LM's theory of i-books, why don't we get
a true reading of (13)? This is related to an argument Sider
(\citeproc{ref-Sider1996-SIDATW}{1996}) gives against Lewis's theory of
continuants.

The other question concerns how many variants we can get that are like
(9) to (11). (In talk I'll end section here, but the following examples
are possibly interesting.)

Does it apply to editions? My Jeffrey shelf has two copies of the first
edition of \emph{Logic of Decision} and one copy of the second edition.
Can we get a reading of (14) on which it's true?

\begin{enumerate}
\def\labelenumi{(\arabic{enumi})}
\setcounter{enumi}{13}
\tightlist
\item
  My Jeffrey shelf has exactly two books on it.
\end{enumerate}

Here's a slightly harder case. My Keynes shelf at the office has four
p-books on it. It has both volumes of \emph{Treatise on Money} and two
copies of the \emph{General Theory}. It's easy to get readings where we
get two and four books, but can we get a reading where (15) is true?

\begin{enumerate}
\def\labelenumi{(\arabic{enumi})}
\setcounter{enumi}{14}
\tightlist
\item
  My Keynes shelf at the office has exactly three books on it.
\end{enumerate}

I think we can. It has volumes 5 to 7 of the Collected Works, i.e.,
three books. Compare my Beatles shelf. It has seven pieces of vinyl on
it; \emph{Revolver} and three copies of the White Album. Do we get true
readings of each of (16) through (19)?

\begin{enumerate}
\def\labelenumi{(\arabic{enumi})}
\setcounter{enumi}{15}
\tightlist
\item
  My Beatles shelf has two records on it.
\item
  My Beatles shelf has three records on it.
\item
  My Beatles shelf has four records on it.
\item
  My Beatles shelf has seven records on it.
\end{enumerate}

I think so\footnote{For (16) \emph{Revolver} and the White Album; for
  (17), \emph{Revolver} and both discs of the White Album; for (18), the
  four spines; for (19), the seven discs.}, but I'm getting less
certain. I'll end this with two cases where I think we really don't get
the readings, though again I'm not certain.

My home Keynes shelf is cleaner. It has the 30 volume collected works,
and nothing else. I think (20) is false, the collected works is not a
book.

\begin{enumerate}
\def\labelenumi{(\arabic{enumi})}
\setcounter{enumi}{19}
\tightlist
\item
  My Keynes shelf at home has exactly one book on it.
\end{enumerate}

I feel in theory we should be able to get (21), counting the two volumes
of the \emph{Treatise on Money} as one book, or even (22), doing that
and also not counting the index, volume 30, as a book. But personally I
can't hear them as true.

\begin{enumerate}
\def\labelenumi{(\arabic{enumi})}
\setcounter{enumi}{20}
\tightlist
\item
  My Keynes shelf at home has exactly 29 books on it.
\item
  My Keynes shelf at home has exactly 28 books on it.
\end{enumerate}

Or from a different direction, my bible shelf has a modern King James
bible, and nothing else. (So it doesn't have the apocrypha.) So here are
some books on that shelf: Genesis, Exodus, Leviticus, etc. Still, I
can't hear (23) as true.

\begin{enumerate}
\def\labelenumi{(\arabic{enumi})}
\setcounter{enumi}{22}
\tightlist
\item
  My bible shelf has 66 books on it.
\end{enumerate}

But maybe I'm just not sufficiently holy. None of these are objections;
unlike the Sider-style point about overlapping books. I just think it's
worth trying to get clearer on what the data is, and what the theory
would have to be to track it.

\section{Example 3: Schools of
schools}\label{example-3-schools-of-schools}

LM say that `school' is not semantically variable. I think that can't be
right, because of what they say about the variable granularity of
i-books.

The context is that they argue against people who say that sometimes
`school' means `school building'. I agree that can't be right. For one
thing, schools typically, at least in my schooling, have multiple
buildings. For another, the hypothesis that schools are buildings
doesn't capture the data it was meant to capture, that schools can be
vandalized. If one sets fire to the school football field, as I
definitely did not do on April 17th 1986, one has vandalized the school
but not any building.

But the problem is that schools can come in different granularities, or
have different persistence conditions, and that makes `school'
semantically variable. Just in Ann Arbor there are a bunch of cases of
this. Was the school I work out founded in 1817, when something like
it's closest continuant was founded in Detroit, on in 1837, when a
university was first founded in Ann Arbor? Did my current President,
President Grasso, change schools when he `moved' from UM Dearborn to UM
Ann Arbor to become President? I think in both cases both answers are
ok, which is evidence of semantic variability. But there are stronger
cases at the elementary level.

Back in the bad old days of segregation, there were two elementary
schools in southeast Ann Arbor: Bryant and Pattengill. One of these was
predominantly Black, the other predominantly white. (I never learned
which was which.) As a desegregation move, the school district was
reorganised. Bryant became a K-2 school, and Pattengill a 3-5 school;
unlike the typical K-5 organisations. The catchment area was the same
for `each'. This was a kind of bussing, and it basically worked; a
Worthwhile Ann Arbor initiative. There is a level of common organisation
to Bryant and Pattengill, but also some independence. The messiness is
nicely reflected in their websites, where they have their own name, but
also the logo for Bryant-Pattengill.

So question, are these the same school? I think it's vague in a way that
reveals semantic indeterminacy. Imagine two parents whose children just
moved from grade 2 at Bryant to grade 3 at Pattengill are asked if their
child has changed school. One says (24); the other says (25).

\begin{enumerate}
\def\labelenumi{(\arabic{enumi})}
\setcounter{enumi}{23}
\tightlist
\item
  Yes, they moved from Bryant to Pattengill.
\item
  No, though they moved campus from Bryant to Pattengill.
\end{enumerate}

I'm not from that part of Ann Arbor so I'm not sure of this. But I think
in local dialects both answers are acceptable, though they can't be
conjoined. That's good evidence of semantic variability.

In the talk I'll finish there, but I'll note here that there is a case
I'm more familiar with that makes a similar point. Ann Arbor has had for
several decades something it calls an Open school. This doesn't have a
catchment area, and it has an idiosyncratic curriculum. It's basically a
hippie school. From 1986 to 1999 the Open school was at a campus called
Bach. Since then the Open school has been at a campus called Mack. These
are only about half a mile apart, closer than Bryant and Pattengill, but
they are administratively separate. When Open wasn't at those campuses,
they were normal K-5 schools.

Now we have three characters, Betty, Mary and Gauri. Betty and Mary were
elementary school students in the 1990s. Betty went to Bach, i.e., to
Open. Mary went to Mack Elementary. They each, separately, talk to
Gauri, who is a current student at the Open School at Mack. I'm pretty
confident that it's fine in local dialect for each of them to say (26)
to Gauri.

\begin{enumerate}
\def\labelenumi{(\arabic{enumi})}
\setcounter{enumi}{25}
\tightlist
\item
  I went to the same school you're going to.
\end{enumerate}

But it's not fine (again in local dialect) for them to say (27) to each
other.

\begin{enumerate}
\def\labelenumi{(\arabic{enumi})}
\setcounter{enumi}{26}
\tightlist
\item
  We went to the same school.
\end{enumerate}

I think the only way to square those responses is to say that `school'
is semantically variable.

That said, there is one other way to make everything fit together in
this example. We might insist that `school' is univocal, and deny the
transitivity of identity. LM say in places that they have a
particularist approach to metaphysics, staying away from big
generalisations. Is transitivity of identity one of those big
generalisations that they will be happy to give up to get the simplest
explanation of the data? I hope not, but it is a possible way out here.

\section{Example Four: Breakfast at
Sweethearts}\label{example-four-breakfast-at-sweethearts}

On the face of it, restaurants look like they should have a similar
metaphysics to books. LM deny this, but I'm going to argue it's actually
correct. Actually I'll do this for coffee shops, where I'm a little more
confident, but I think the arguments carries over. If they think I'm
right about coffee shops, but the metaphysics of coffee shops is
different to that of restaurants, well that would be a cool outcome.

Both coffee shops and books both have the kinda-abstract/kinda-physical
features that generated the i-books/p-books distinction. If in a
particular terminal there are three Starbucks locations, and no other
places to get coffee, then both of (28) and (29) seem like they have
true readings.

\begin{enumerate}
\def\labelenumi{(\arabic{enumi})}
\setcounter{enumi}{27}
\tightlist
\item
  There is one coffee shop in the terminal, Starbucks.
\item
  There are three coffee shops in the terminal, the Starbuckses by C8,
  C14, and C20.
\end{enumerate}

That looks like we should have an i-shop that makes (28) true, namely
Starbucks, and three p-shops that make (29) true. But LM deny this. They
say the witnesses for (29) are finer grained i-shops, and p-shops aren't
really shops. This seems like a reasonable suggestion, it is what I
think the cases in the previous section suggest is correct about
schools. But I don't think it's correct about coffee shops.

LM argue against the view that `restaurant', or `coffee shop', sometimes
picks out a \emph{building}. And I agree that's a bad view. P-shops, or
P-restaurants, are not buildings. They have one argument for this, here
are four more.

\begin{itemize}
\tightlist
\item
  The storefront by gate C8 is not a building; it is a room in a
  building.
\item
  The cart by gate C20 is even less of a building.
\item
  When there's outdoor dining the restaurant extends into the street,
  but no building extends into the street.
\item
  Outdoor restaurants seem inadvisable rather than incoherent.
\end{itemize}

But p-shops might be \emph{locations} even if they aren't buildings.
Sometimes they will be campuses, as when a restaurant has multiple not
quite connected buildings. But we'll stick with location.

To argue this, I'll work through a simplified, idealised, version of
O'Hare airport. One where the planes run on time, the bags are
transferred, and the food is edible, albeit barely. We'll start with a
real fact about O'Hare: there are 14 Starbucks locations there. And
`location' is the word that is usually used. From here on, we'll depart
somewhat from reality. I'll make the following five other assumptions,
all of which I think are false, though the world where they are true is
closer than the one where O'Hare has figured out bag transfer.

\begin{itemize}
\tightlist
\item
  There are three Dunkin Donuts locations at O'Hare.
\item
  There are no other coffee shops at O'Hare.
\item
  The Starbucks locations are all administratively unified. They order
  goods collectively, they have no individual managers, they share
  staff, and indeed move staff around when one is busy and another is
  quiet.
\item
  Four of the locations have an express lane out front just for simple
  orders, and a regular counter inside the store.
\item
  Starbucks O'Hare has a unified management. They obviously keep track
  of revenue from different parts of the airport. But for revenue
  tracking purposes, the express counters are treated as independent
  entities, just like the carts, or the other locations.
\end{itemize}

Now here are our two natural claims about O'Hare, as described.

\begin{enumerate}
\def\labelenumi{(\arabic{enumi})}
\setcounter{enumi}{29}
\tightlist
\item
  There are two coffee shops at O'Hare: Starbucks and Dunkin.
\item
  There are seventee coffee shops at O'Hare: 14 Starbuckses and 3
  Dunkins.
\end{enumerate}

For most travellers, (31) is the more natural, though they could make
sense of (30). LM say that both of these sentences are about i-shops at
different levels of granularity.

But I just don't think in this story the individual Starbucks locations
are \emph{institutions} in any meaningful sense. They are just physical
locations and nothing more. They don't have staff, or management, or
invoices. Management does keep track of the revenue from most of them,
but by revenue centers we'd say there are 18 Starbuckses, counting the
express lanes as extra. So we still don't get 14.

Even more evidence that they are not institutions: they aren't movable.
Without anyone identified as the staff of the location at C14, \emph{it}
can't move to B9. Starbucks O'Hare could close C14 and open a new one at
B9, but that would be a close-and-open, not a relocation. Now in reality
I'm sure there are enough staff associated with a particular location
that you could make sense of relocating a Starbucks within O'Hare. And
so there's a sense in which the \emph{actual} 14 Starbucks locations are
i-shops.

But I don't think that institutional independence is necessary for (31)
to be true. And I don't think someone who says (31) to their
caffeine-deprived travel companions needs to know about, or be making a
claim about, the internal organisation of the Starbuckses at O'Hare. So,
I conclude, (31) is a claim about p-shops.

\section{Conclusion}\label{conclusion}

This is a really fun book, and as I've been stressing, thinking through
what it says can affect how we think about things that are central to
some of our lives: schools and coffee shops. I'm sure it also has
implications for other, less important things, like municipalities or I
guess minds. So I thoroughly recommend it, and I'm looking forward to
the discussion.

\subsection*{References}\label{references}
\addcontentsline{toc}{subsection}{References}

\phantomsection\label{refs}
\begin{CSLReferences}{1}{0}
\bibitem[\citeproctext]{ref-Fairchild2019}
Fairchild, Maegan. 2019. {``The Barest Flutter of the Smallest Leaf:
Understanding Material Plenitude.''} \emph{The Philosophical Review} 128
(2): 143--78. doi:
\href{https://doi.org/10.1215/00318108-7374932}{10.1215/00318108-7374932}.

\bibitem[\citeproctext]{ref-LM2025}
Liebesman, David, and Ofra Magidor. 2025. \emph{Property Versatility and
Copredication}. Oxford: Oxford University Press.

\bibitem[\citeproctext]{ref-Sider1996-SIDATW}
Sider, Theodore. 1996. {``All the World's a Stage.''} \emph{Australasian
Journal of Philosophy} 74 (3): 433--53. doi:
\href{https://doi.org/10.1080/00048409612347421}{10.1080/00048409612347421}.

\end{CSLReferences}



\noindent Draft of November 24


\end{document}
