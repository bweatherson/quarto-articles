% Options for packages loaded elsewhere
\PassOptionsToPackage{unicode}{hyperref}
\PassOptionsToPackage{hyphens}{url}
\PassOptionsToPackage{dvipsnames,svgnames,x11names}{xcolor}
%
\documentclass[
  11pt,
  letterpaper,
  DIV=11,
  numbers=noendperiod,
  oneside]{scrartcl}

\usepackage{amsmath,amssymb}
\usepackage{iftex}
\ifPDFTeX
  \usepackage[T1]{fontenc}
  \usepackage[utf8]{inputenc}
  \usepackage{textcomp} % provide euro and other symbols
\else % if luatex or xetex
  \ifXeTeX
    \usepackage{mathspec} % this also loads fontspec
  \else
    \usepackage{unicode-math} % this also loads fontspec
  \fi
  \defaultfontfeatures{Scale=MatchLowercase}
  \defaultfontfeatures[\rmfamily]{Ligatures=TeX,Scale=1}
\fi
\usepackage{lmodern}
\ifPDFTeX\else  
    % xetex/luatex font selection
  \setmainfont[Scale = MatchLowercase]{Scala Pro}
  \setsansfont[]{Scala Sans Pro}
  \ifXeTeX
    \setmathfont(Digits,Latin,Greek)[]{Scala Pro}
  \else
    \setmathfont[]{Scala Pro}
  \fi
\fi
% Use upquote if available, for straight quotes in verbatim environments
\IfFileExists{upquote.sty}{\usepackage{upquote}}{}
\IfFileExists{microtype.sty}{% use microtype if available
  \usepackage[]{microtype}
  \UseMicrotypeSet[protrusion]{basicmath} % disable protrusion for tt fonts
}{}
\makeatletter
\@ifundefined{KOMAClassName}{% if non-KOMA class
  \IfFileExists{parskip.sty}{%
    \usepackage{parskip}
  }{% else
    \setlength{\parindent}{0pt}
    \setlength{\parskip}{6pt plus 2pt minus 1pt}}
}{% if KOMA class
  \KOMAoptions{parskip=half}}
\makeatother
\usepackage{xcolor}
\usepackage[left=1in,marginparwidth=2.0666666666667in,textwidth=4.1333333333333in,marginparsep=0.3in]{geometry}
\setlength{\emergencystretch}{3em} % prevent overfull lines
\setcounter{secnumdepth}{3}
% Make \paragraph and \subparagraph free-standing
\ifx\paragraph\undefined\else
  \let\oldparagraph\paragraph
  \renewcommand{\paragraph}[1]{\oldparagraph{#1}\mbox{}}
\fi
\ifx\subparagraph\undefined\else
  \let\oldsubparagraph\subparagraph
  \renewcommand{\subparagraph}[1]{\oldsubparagraph{#1}\mbox{}}
\fi


\providecommand{\tightlist}{%
  \setlength{\itemsep}{0pt}\setlength{\parskip}{0pt}}\usepackage{longtable,booktabs,array}
\usepackage{calc} % for calculating minipage widths
% Correct order of tables after \paragraph or \subparagraph
\usepackage{etoolbox}
\makeatletter
\patchcmd\longtable{\par}{\if@noskipsec\mbox{}\fi\par}{}{}
\makeatother
% Allow footnotes in longtable head/foot
\IfFileExists{footnotehyper.sty}{\usepackage{footnotehyper}}{\usepackage{footnote}}
\makesavenoteenv{longtable}
\usepackage{graphicx}
\makeatletter
\def\maxwidth{\ifdim\Gin@nat@width>\linewidth\linewidth\else\Gin@nat@width\fi}
\def\maxheight{\ifdim\Gin@nat@height>\textheight\textheight\else\Gin@nat@height\fi}
\makeatother
% Scale images if necessary, so that they will not overflow the page
% margins by default, and it is still possible to overwrite the defaults
% using explicit options in \includegraphics[width, height, ...]{}
\setkeys{Gin}{width=\maxwidth,height=\maxheight,keepaspectratio}
% Set default figure placement to htbp
\makeatletter
\def\fps@figure{htbp}
\makeatother

\setlength\heavyrulewidth{0ex}
\setlength\lightrulewidth{0ex}
\makeatletter
\def\@maketitle{%
\newpage
\null
\vskip 2em%
\begin{center}%
\let \footnote \thanks
  {\LARGE \@title \par}%
  \vskip 1.5em%
  {\large
    \lineskip .5em%
    \begin{tabular}[t]{c}%
      \@author
    \end{tabular}\par}%
  %\vskip 1em%
  %{\large \@date}%
\end{center}%
\par
\vskip 1.5em}
\makeatother 
\KOMAoption{captions}{tableheading}
\makeatletter
\@ifpackageloaded{caption}{}{\usepackage{caption}}
\AtBeginDocument{%
\ifdefined\contentsname
  \renewcommand*\contentsname{Table of contents}
\else
  \newcommand\contentsname{Table of contents}
\fi
\ifdefined\listfigurename
  \renewcommand*\listfigurename{List of Figures}
\else
  \newcommand\listfigurename{List of Figures}
\fi
\ifdefined\listtablename
  \renewcommand*\listtablename{List of Tables}
\else
  \newcommand\listtablename{List of Tables}
\fi
\ifdefined\figurename
  \renewcommand*\figurename{Figure}
\else
  \newcommand\figurename{Figure}
\fi
\ifdefined\tablename
  \renewcommand*\tablename{Table}
\else
  \newcommand\tablename{Table}
\fi
}
\@ifpackageloaded{float}{}{\usepackage{float}}
\floatstyle{ruled}
\@ifundefined{c@chapter}{\newfloat{codelisting}{h}{lop}}{\newfloat{codelisting}{h}{lop}[chapter]}
\floatname{codelisting}{Listing}
\newcommand*\listoflistings{\listof{codelisting}{List of Listings}}
\makeatother
\makeatletter
\makeatother
\makeatletter
\@ifpackageloaded{caption}{}{\usepackage{caption}}
\@ifpackageloaded{subcaption}{}{\usepackage{subcaption}}
\makeatother
\makeatletter
\@ifpackageloaded{sidenotes}{}{\usepackage{sidenotes}}
\@ifpackageloaded{marginnote}{}{\usepackage{marginnote}}
\makeatother
\ifLuaTeX
  \usepackage{selnolig}  % disable illegal ligatures
\fi
\IfFileExists{bookmark.sty}{\usepackage{bookmark}}{\usepackage{hyperref}}
\IfFileExists{xurl.sty}{\usepackage{xurl}}{} % add URL line breaks if available
\urlstyle{same} % disable monospaced font for URLs
\hypersetup{
  pdftitle={Review of ``Rethinking Intuition''},
  pdfauthor={Brian Weatherson},
  colorlinks=true,
  linkcolor={black},
  filecolor={Maroon},
  citecolor={Blue},
  urlcolor={Blue},
  pdfcreator={LaTeX via pandoc}}

\title{Review of ``Rethinking Intuition''}
\author{Brian Weatherson}
\date{2002-01-01}

\begin{document}
\maketitle
This collection arose out of a conference on intuitions at the
University of Notre Dame in April 1996. The papers in it mainly address
two related questions: (\emph{a}) How much evidential weight should be
assigned to intuitions? and (\emph{b}) Are concepts governed by
necessary and sufficient conditions, or are they governed by `family
resemblance' conditions, as Wittgenstein suggested? The book includes
four papers by psychologists relating and analyzing some empirical
findings concerning intuitions and eleven papers by philosophers
endorsing various answers to these questions.

\marginnote{\begin{footnotesize}

Published in \emph{Ethics} 112: 361-364.

\end{footnotesize}}

The first section consists of the papers by psychologists. In these
papers, the main target is the traditional philosopher who holds, inter
alia, that the answer to a is ``quite a lot'' and the answer to b is the
former, that there are necessary and sufficient conditions for most
philosophically interesting concepts. If you like these answers, then
you might spend your time Chisholming away at concepts like `justice,'
`knowledge,' and `causation'---proposing snappy analyses and testing
them against intuitions about possible cases. But if you don't like
these answers, you might prefer to make pointed criticisms of the
presuppositions of such a methodology and suggest some more empirically
defensible ways of coming to understand concepts. Indeed, this is just
what the psychologists writing here do.

The papers by the philosophers are, very roughly, divided up according
to their answers to these questions. The second section, titled
``Rethinking Intuition and Philosophical Method,'' consists of papers
disagreeing with traditional philosophy about \emph{a} or \emph{b}.
(This section includes papers by Stephen Stich, Robert Cummins, Hilary
Kornblith, Tamara Horowitz, William Ramsey, and Alvin Goldman and Joel
Pust.) The really radical position, expressed most clearly by Stich, is
that traditional philosophy is wrong on both counts. We need to bring
much more empirical research to bear on explicating crucial concepts in
ethics, epis- temology, and so forth, and the explications we will end
up with will not be short lists of necessary and sufficient conditions.
The third section, titled ``Defending the Philosophical Tradition,''
contains, mostly, defenses of one of the traditional views. (This
section includes papers by George Bealer, Richard Foley, Ernest Sosa,
George Graham and Terry Horgan, and Michael DePaul.) The main aim here
is to defend the value of intuitions as evidence; there is no explicit
defense of the traditional view of concepts. Despite this neat
rationale, the editors' classification breaks down in a few cases. For
example, in Kornblith's paper he indicates substantial agreement with
the paper by Graham and Horgan. So it is a little unclear why these
papers are in these opposing sections. There is one other philosophical
paper: Gary Gutting's historical introduction is printed in a special
`Introduction' section.

Three of the papers have the phrase ``Reflective Equilibrium'' in their
title, so it might be expected that there would be some cutting-edge
discussions about how to balance competing desiderata in achieving
equilibrium. We don't get such a discussion, and perhaps with good
reason. With a nod in the direction of Goodman, Rawls, and Daniels, the
writers mostly agree that if the aim of ethical or epistemological
theory is, primarily, to systematize our intuitions, then reflective
equilibrium (RE) is the way to do it. The papers here are, quite self-
consciously, interested in the more basic question of whether that is
what we want ethics or epistemology to do. I'll conclude by saying a bit
more about the papers which most clearly address this question. For the
radicals, Cummins argues that ``philosophical intuition is
epistemologically useless'' (p.~125). For the traditionals, on the other
hand, Michael DePaul argues that RE provides ``close to a correct
answer'' to the question, ``How should we conduct philo- sophical
inquiry?'' (p.~294).

Cummins compares evidence from intuitions to evidence from other
sources, like telescopes. He notes two related features of telescopes
which, he thinks, makes them more trustworthy sources of evidence than
intuitions. First, telescopes can be calibrated. We can apply telescopes
to cases about which we have reliable independent evidence and see
whether they deliver appropriate answers. For example, we can point a
telescope at a distant mountain and see whether it looks the same
through the telescope as it does up close and personal. If so, we can
trust what it shows about places we have never before seen, such as
heavenly bodies. If not, we not only learn that the telescope is
untrustworthy but also may learn a little about the way in which it
fails. Unlike telescopes, intuitions cannot be independently checked.
They can only be checked against other intuitions. Hence, argues
Cummins, they are untrustworthy. As Sosa notes, the comparison here may
be unfair. Even though we can calibrate telescopes, we cannot calibrate
observation as a whole. We can only calibrate particular kinds of
observations against other kinds of observations and particular kinds of
intuitions against other kinds of intuitions. Intuition, in this
respect, is just like observation, and since we trust observations, we
should trust intuitions.

Cummins's other critique is that what evidence we do have about
intuitions suggests that they are artifacts of the process by which they
are produced rather than reliable guides to their subject matters. The
idea is that the presence of a certain intuition concerning fairness
tells us more about the source of the intuition (usually the person who
has the intuition) than about fairness. If this is right, then
intuitions are obviously not evidential. Cummins's argument is that
there are only five possible sources of intuitions, and examination of
each suggests that intuitions are artifacts of the process by which they
are produced. To prove this, Cummins works through each of the five
possible sources and argues for each that an intuition derived from that
source has no evidential value. Argument by cases in this way, when
there are five possible cases to cover, is never going to be
satisfactory. For example, one of the cases Cummins considers is that
intuitions are evidential because they arise from possession of
concepts. Something like this view is endorsed in the papers by Bealer
and by Goldman and Pust. Cummins thinks this does not work because our
concepts are just sets of beliefs. One's concept of an elevator is just
everything one believes about elevators. If anything like this is right,
then the fact that we intuit that \emph{p} just means that we believe
\emph{p} and that could not be evidence that \emph{p}. But the theory of
concepts he has in mind cannot be right. As Fodor has pointed out, it
seems people can share concepts while having different beliefs involving
those concepts. Indeed, something like this must be right if genuine
disagreement is possible. If possessing a concept just meant having
certain beliefs, then it would be impossible for people with radically
different beliefs about a subject to share concepts relating to that
subject. Since such sharing is possible, concept possession does not
reduce to having certain beliefs. The main point is not that there is an
insurmountable problem for Cummins here---maybe a more detailed
discussion could show that his account of concepts is right and Fodor's
is wrong---but rather that with such a wide terrain to cover, a short
argument is not going to win many converts.

Michael DePaul is much more content with intuitions playing a central
role in philosophy. Indeed, he seems happy to let them do all the work.
His paper imagines a dialogue between himself and a friendly barfly who
wants to be told all about how philosophy works. At some point in the
conversation, DePaul's character decides to present the new friend with
an extended summary of how RE works. The friend is bemused that
philosophers seem to only sit around and compare intuitive judgments. It
does seem, notes the friend, a trifle self-indulgent. DePaul's response
attempts to defend RE by an argument that any alternative method would
be irrational. Any alternative, argues DePaul, would have to (a) abandon
reflection, (b) reflect incompletely, by leaving out certain beliefs,
principles, or whatever enters into reflection, or (c) not allow results
of reflection to influence final theory. As DePaul notes, it would be
irrational to accept any of these options. DePaul acknowledges two
possible criticisms here, criticisms which he admits he is not sure how
to answer. The first is that it is not clear what is wrong with being
irrational, at least in the sense DePaul has in mind. The second is that
even if we have a reason not to be irrational, it is not clear how
strong a reason this is and, hence, whether irrationality might be
justifiable on occasion because it fulfills some greater purpose.

There is a third criticism that more closely reflects the problem raised
by DePaul's interlocutor. When someone says that philosophy should be
about more than systematizing intuitions, they are not advocating
alternatives to RE but, rather, supplements to it. The point of the
criticism was that there must be other sources of evidence for moral or
conceptual claims, other than just intuition. (This, apparently, is
intuitively obvious!) DePaul provides a good response to someone who
wants to say that intuitions have no evidential value at all. But he
does not answer the critic who denies that intuitions provide the only
evidence that might bear on philosophical problems.

This is a very useful collection to have published. A study of the role
of intuition should be at the heart of any investigation into
philosophical methodology. And such an investigation will have to take
into account both the empirical findings about how intuition works and
the philosophical considerations about how much importance should be
attached to intuitions. The papers here do not look like the last word
on any of these questions, but they are a helpful, and perhaps overdue,
first word.



\end{document}
