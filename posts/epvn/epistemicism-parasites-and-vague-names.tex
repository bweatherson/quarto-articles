% Options for packages loaded elsewhere
\PassOptionsToPackage{unicode}{hyperref}
\PassOptionsToPackage{hyphens}{url}
%
\documentclass[
  10pt,
  letterpaper,
  DIV=11,
  numbers=noendperiod,
  twoside]{scrartcl}

\usepackage{amsmath,amssymb}
\usepackage{setspace}
\usepackage{iftex}
\ifPDFTeX
  \usepackage[T1]{fontenc}
  \usepackage[utf8]{inputenc}
  \usepackage{textcomp} % provide euro and other symbols
\else % if luatex or xetex
  \usepackage{unicode-math}
  \defaultfontfeatures{Scale=MatchLowercase}
  \defaultfontfeatures[\rmfamily]{Ligatures=TeX,Scale=1}
\fi
\usepackage{lmodern}
\ifPDFTeX\else  
    % xetex/luatex font selection
  \setmainfont[ItalicFont=EB Garamond Italic,BoldFont=EB Garamond
Bold]{EB Garamond Math}
  \setsansfont[]{Europa-Bold}
  \setmathfont[]{Garamond-Math}
\fi
% Use upquote if available, for straight quotes in verbatim environments
\IfFileExists{upquote.sty}{\usepackage{upquote}}{}
\IfFileExists{microtype.sty}{% use microtype if available
  \usepackage[]{microtype}
  \UseMicrotypeSet[protrusion]{basicmath} % disable protrusion for tt fonts
}{}
\usepackage{xcolor}
\usepackage[left=1in, right=1in, top=0.8in, bottom=0.8in,
paperheight=9.5in, paperwidth=6.5in, includemp=TRUE, marginparwidth=0in,
marginparsep=0in]{geometry}
\setlength{\emergencystretch}{3em} % prevent overfull lines
\setcounter{secnumdepth}{3}
% Make \paragraph and \subparagraph free-standing
\ifx\paragraph\undefined\else
  \let\oldparagraph\paragraph
  \renewcommand{\paragraph}[1]{\oldparagraph{#1}\mbox{}}
\fi
\ifx\subparagraph\undefined\else
  \let\oldsubparagraph\subparagraph
  \renewcommand{\subparagraph}[1]{\oldsubparagraph{#1}\mbox{}}
\fi


\providecommand{\tightlist}{%
  \setlength{\itemsep}{0pt}\setlength{\parskip}{0pt}}\usepackage{longtable,booktabs,array}
\usepackage{calc} % for calculating minipage widths
% Correct order of tables after \paragraph or \subparagraph
\usepackage{etoolbox}
\makeatletter
\patchcmd\longtable{\par}{\if@noskipsec\mbox{}\fi\par}{}{}
\makeatother
% Allow footnotes in longtable head/foot
\IfFileExists{footnotehyper.sty}{\usepackage{footnotehyper}}{\usepackage{footnote}}
\makesavenoteenv{longtable}
\usepackage{graphicx}
\makeatletter
\def\maxwidth{\ifdim\Gin@nat@width>\linewidth\linewidth\else\Gin@nat@width\fi}
\def\maxheight{\ifdim\Gin@nat@height>\textheight\textheight\else\Gin@nat@height\fi}
\makeatother
% Scale images if necessary, so that they will not overflow the page
% margins by default, and it is still possible to overwrite the defaults
% using explicit options in \includegraphics[width, height, ...]{}
\setkeys{Gin}{width=\maxwidth,height=\maxheight,keepaspectratio}
% Set default figure placement to htbp
\makeatletter
\def\fps@figure{htbp}
\makeatother
% definitions for citeproc citations
\NewDocumentCommand\citeproctext{}{}
\NewDocumentCommand\citeproc{mm}{%
  \begingroup\def\citeproctext{#2}\cite{#1}\endgroup}
\makeatletter
 % allow citations to break across lines
 \let\@cite@ofmt\@firstofone
 % avoid brackets around text for \cite:
 \def\@biblabel#1{}
 \def\@cite#1#2{{#1\if@tempswa , #2\fi}}
\makeatother
\newlength{\cslhangindent}
\setlength{\cslhangindent}{1.5em}
\newlength{\csllabelwidth}
\setlength{\csllabelwidth}{3em}
\newenvironment{CSLReferences}[2] % #1 hanging-indent, #2 entry-spacing
 {\begin{list}{}{%
  \setlength{\itemindent}{0pt}
  \setlength{\leftmargin}{0pt}
  \setlength{\parsep}{0pt}
  % turn on hanging indent if param 1 is 1
  \ifodd #1
   \setlength{\leftmargin}{\cslhangindent}
   \setlength{\itemindent}{-1\cslhangindent}
  \fi
  % set entry spacing
  \setlength{\itemsep}{#2\baselineskip}}}
 {\end{list}}
\usepackage{calc}
\newcommand{\CSLBlock}[1]{\hfill\break\parbox[t]{\linewidth}{\strut\ignorespaces#1\strut}}
\newcommand{\CSLLeftMargin}[1]{\parbox[t]{\csllabelwidth}{\strut#1\strut}}
\newcommand{\CSLRightInline}[1]{\parbox[t]{\linewidth - \csllabelwidth}{\strut#1\strut}}
\newcommand{\CSLIndent}[1]{\hspace{\cslhangindent}#1}

\setlength\heavyrulewidth{0ex}
\setlength\lightrulewidth{0ex}
\usepackage[automark]{scrlayer-scrpage}
\clearpairofpagestyles
\cehead{
  Brian Weatherson
  }
\cohead{
  Epistemicism, Parasites, and Vague Names
  }
\ohead{\bfseries \pagemark}
\cfoot{}
\makeatletter
\newcommand*\NoIndentAfterEnv[1]{%
  \AfterEndEnvironment{#1}{\par\@afterindentfalse\@afterheading}}
\makeatother
\NoIndentAfterEnv{itemize}
\NoIndentAfterEnv{enumerate}
\NoIndentAfterEnv{description}
\NoIndentAfterEnv{quote}
\NoIndentAfterEnv{equation}
\NoIndentAfterEnv{longtable}
\NoIndentAfterEnv{abstract}
\renewenvironment{abstract}
 {\vspace{-1.25cm}
 \quotation\small\noindent\rule{\linewidth}{.5pt}\par\smallskip
 \noindent }
 {\par\noindent\rule{\linewidth}{.5pt}\endquotation}
\KOMAoption{captions}{tableheading}
\makeatletter
\@ifpackageloaded{caption}{}{\usepackage{caption}}
\AtBeginDocument{%
\ifdefined\contentsname
  \renewcommand*\contentsname{Table of contents}
\else
  \newcommand\contentsname{Table of contents}
\fi
\ifdefined\listfigurename
  \renewcommand*\listfigurename{List of Figures}
\else
  \newcommand\listfigurename{List of Figures}
\fi
\ifdefined\listtablename
  \renewcommand*\listtablename{List of Tables}
\else
  \newcommand\listtablename{List of Tables}
\fi
\ifdefined\figurename
  \renewcommand*\figurename{Figure}
\else
  \newcommand\figurename{Figure}
\fi
\ifdefined\tablename
  \renewcommand*\tablename{Table}
\else
  \newcommand\tablename{Table}
\fi
}
\@ifpackageloaded{float}{}{\usepackage{float}}
\floatstyle{ruled}
\@ifundefined{c@chapter}{\newfloat{codelisting}{h}{lop}}{\newfloat{codelisting}{h}{lop}[chapter]}
\floatname{codelisting}{Listing}
\newcommand*\listoflistings{\listof{codelisting}{List of Listings}}
\makeatother
\makeatletter
\makeatother
\makeatletter
\@ifpackageloaded{caption}{}{\usepackage{caption}}
\@ifpackageloaded{subcaption}{}{\usepackage{subcaption}}
\makeatother
\ifLuaTeX
  \usepackage{selnolig}  % disable illegal ligatures
\fi
\usepackage{bookmark}

\IfFileExists{xurl.sty}{\usepackage{xurl}}{} % add URL line breaks if available
\urlstyle{same} % disable monospaced font for URLs
\hypersetup{
  pdftitle={Epistemicism, Parasites, and Vague Names},
  pdfauthor={Brian Weatherson},
  hidelinks,
  pdfcreator={LaTeX via pandoc}}

\title{Epistemicism, Parasites, and Vague Names}
\author{Brian Weatherson}
\date{2003}

\begin{document}
\maketitle
\begin{abstract}
John Burgess has recently argued that Timothy Williamson's attempts to
avoid the objection that his theory of vagueness is based on an
untenable metaphysics of content are unsuccessful. Burgess's arguments
are important, and largely correct, but there is a mistake in the
discussion of one of the key examples. In this note I provide some
alternative examples and use them to repair the mistaken section of the
argument.
\end{abstract}

\setstretch{1.1}
Why is it so implausible that there is a sharp boundary between the rich
and the non-rich? Perhaps we find it implausible merely because we
(implicitly) believe that if there were such a boundary we would be able
to discover where it is. If this is so we should revise our judgements.
As Timothy Williamson (\citeproc{ref-Williamson1994-WILV}{1994},
\citeproc{ref-Williamson2000-WILKAI}{2000}) has shown, if there were
such a boundary we would \emph{not} know where it is. Still, this is not
the only reason for being sceptical about the existence of such a
boundary. In ``Vagueness, Epistemicism and Response-Dependence'' John
Burgess (\citeproc{ref-Burgess2001}{2001}) outlines an impressive
objection to the existence of such boundaries, and in particular to
epistemicist theories that posit their existence. Burgess's objection is
based not on principles about the \emph{epistemology} of content, as the
bad objection just stated is, but rather on principles about the
\emph{metaphysics} of content.

If a word \emph{t} has content \emph{c}, this must be the case in virtue
of some more primitive fact obtaining. Facts about content, such as
this, are not among the fundamental constituents of reality. Roughly,
facts about linguistic content must obtain in virtue of facts about use.
But there are simply not enough facts about use to determine a precise
meaning for paradigmatically vague terms like `rich'. Any theory that
holds that `rich' does have a precise meaning must meet this objection.
As Burgess argues, Williamson's attempts to do this have not been
entirely successful. Burgess argues, persuasively, that epistemicists
owe us a theory of how terms like `rich' get to have the precise meaning
they apparently have given that the facts about use do not seem to
generate a precise meaning. He also argues, less persuasively, that
Williamson's `parasitic' strategy for meeting this obligation is
unsuccessful. Indeed, the argument here rests at one point on a premiss
that is clearly false. I will suggest a way to patch the argument and
reinstate the objection to epistemicism.

The obligation to provide a theory that generates content in terms of
use does not just fall on the epistemicists. We indeterminists about
content must also discharge it. Assume that we have done so, and we have
a theory of content that divides sentences into (at least) the true, the
false and the indeterminate. Williamson
(\citeproc{ref-Williamson1994-WILV}{1994, 207--8}) argues that the only
reason we believe that any sentences fall into this third category is
that we are respecting a mythical symmetry between truth and falsity. We
are falling into the trap of thinking that if a sentence is not somehow
made false, it is not false. The true story is that if an assertoric
sentence has content, and it is not made true, it is\emph{false}. This
provides the basis for Williamson's `parasitic' strategy: wait for the
indeterminist to offer a theory of when sentences are true, accept that
part of the indeterminist theory, and say all other sentences that
express propositions are false. If the strategy works, then there is no
way the indeterminist can meet the obligation to provide a theory of
content without the epistemicist also being able to do so, so there is
no argument for indeterminism here. (There are complications, to put it
mildly, with this strategy when the indeterminist allows the border
between the true and the indeterminate to be vague. Burgess lets these
potential problems slide, and so shall I.)

The strategy rests on the purported asymmetry between truth and falsity.
Burgess claims that positing such an asymmetry makes epistemicism
inconsistent. Consider a colour patch that is around the border between
red and orange. Burgess claims, correctly, that an indeterminist theory
of content may say that (1) and (2) are indeterminate, and hence
Williamson might be committed to the position that (1) and (2) are
false, and hence so is (3).

\begin{description}
\tightlist
\item[(1)]
That patch is red.
\item[(2)]
That patch is orange.
\item[(3)]
That patch is either red or orange.
\end{description}

But this is hopeless because ``on the epistemicist view, there is a
sharp boundary in the series between red and orange; every patch is
either one or the other.'' (\citeproc{ref-Burgess2001}{Burgess 2001,
519}) This last claim is false. According to epistemicism, there is a
sharp boundary between red and not red, so the patch is either red or
not red. But the epistemicist need not hold that if the patch is not
red, then it is orange. It is consistent with epistemicism that there
are colours strictly between red and orange, just as it is consistent
with epistemicism that there are colours strictly between red and
yellow, and just as it is consistent with epistemicism that there are
colours strictly between red and blue. Hence it is possible that the
colour of this patch is strictly between red and orange, and thus is
neither red nor orange. So this line of reasoning does not work. Perhaps
the argument can be easily fixed. According to the indeterminist, both
(1) and (4) are indeterminate. Hence according to Williamson's
`parasitic' theory of content, both (1) and (4) are false, so (5) is
false.

\begin{description}
\tightlist
\item[(1)]
That patch is red.
\item[(4)]
That patch is not red.
\item[(5)]
That patch is either red or not red.
\end{description}

This is more like a problem, because Williamson certainly is committed
to the truth of (5). However, it is easy to see how Williamson should
respond. The theory of content sketched above (or more precisely, the
strategy for converting indeterminist theories to determinist ones) was
only meant to apply to simple sentences. A \emph{simple} sentence is
true iff the indeterminist says it is true. The truth value of
\emph{compound} sentences, like (4) and (5), is given by a standard
Davidsonian theory of truth. Hence (1) is false and (4) is true, as
required.

The best way to resurrect Burgess's argument is to shift our attention
from vague predicates to vague names. Consider any mountain, say
Kilimanjaro. It is vague just where the mountain starts, so it will be
vague just which atoms constitute Kilimanjaro. Kilimanjaro is some
fusion of atoms or other, but it is indeterminate just which one it is.
Some of these fusions have different masses, and some have different
shapes, so no sentence of the form of (6) or of (7) will be true
according to the indeterminist.

\begin{description}
\tightlist
\item[(6)]
Kilimanjaro has shape \emph{s}.
\item[(7)]
Kilimanjaro has mass \emph{m}.
\end{description}

Hence according to the Williamson's asymmetric theory of truth, any
sentence of either of these forms is false. Note that this holds even if
we restrict the application of his theory to simple sentences. Now let
\emph{K} be a set of fusions of atoms \{\emph{f}\textsubscript{1},
\emph{f}\textsubscript{2}, \ldots, \emph{f\textsubscript{n}}\} such that
it is determinate that Kilimanjaro is one of these fusions. (Because of
higher-order vagueness it may be impossible to find such a set that does
not contain any fusion that is determinately not Kilimanjaro. That will
not matter; all that we require is that Kilimanjaro is one of these
fusions.) Let \emph{s\textsubscript{i}} be the shape of
\emph{f\textsubscript{i}} and \emph{m\textsubscript{i}} its mass. Then
for all \emph{i}, (6.\emph{i}) and (7.\emph{i}) are false, as we just
argued.

\begin{description}
\tightlist
\item[(6.\emph{i})]
Kilimanjaro has shape \emph{s\textsubscript{i}}.
\item[(7.\emph{i})]
Kilimanjaro has mass \emph{m\textsubscript{i}}.
\end{description}

Hence both (8) and (9) are false.

\begin{description}
\tightlist
\item[(8)]
Kilimanjaro has shape \emph{s}\textsubscript{1} or Kilimanjaro has shape
\emph{s}\textsubscript{2} or \ldots{} or Kilimanjaro has shape
\emph{s\textsubscript{n}}.
\item[(9)]
Kilimanjaro has mass \emph{m}\textsubscript{1} or Kilimanjaro has mass
\emph{m}\textsubscript{2} or \ldots{} or Kilimanjaro has mass
\emph{m\textsubscript{n}}.
\end{description}

And the epistemicist is committed to (8) and (9) being true. We may not
be able to discover which disjunct is true, but that is no reason to
think that the disjunction is not true. Burgess's argument was that if
we adopt Williamson's advice for constructing a theory of content, we
will misclassify sentences that express penumbral connections. He was
basically right, but we need to use a different example to prove it.

I assumed above that Kilimanjaro is a fusion of atoms. Some may object
to this on the grounds that Kilimanjaro has different temporal and modal
properties to any fusion of atoms. I doubt such objections ultimately
work, but for present purposes the important thing to note is that the
argument can go through without such an assumption. Even if Kilimanjaro
is not identical to any fusion in \emph{K}, it is clear that Kilimanjaro
(actually, now) exactly overlaps some member of \emph{K}. And since
Kilimanjaro has the same (actual, present) shape and mass as any fusion
of atoms it exactly overlaps, it still follows that (8) and (9) are
true.

If we do assume that Kilimanjaro is one of the fusions, then we can
generate another case where Williamson's theory generates false
predictions. Since at most one of the fusions is a mountain, it follows
that (10.\emph{i}) is indeterminate for all \emph{i} on an indeterminist
theory of content, and hence false according to Williamson.

\begin{description}
\tightlist
\item[(10.\emph{i})]
\emph{f\textsubscript{i}} is a mountain.
\end{description}

Hence his theory mistakenly predicts that (11) is false, when it is by
hypothesis true.

\begin{description}
\tightlist
\item[(11)]
\emph{f}\textsubscript{1} is a mountain or \emph{f}\textsubscript{2} is
a mountain or \ldots{} or \emph{f\textsubscript{n}} is a mountain.
\end{description}

This argument does rest on a contentious bit of metaphysics, but it
still seems basically sound.

I did not assume at any point that Kilimanjaro is a vague object. I did
assume that `Kilimanjaro' is a vague name, but it is consistent with the
argument I have presented that there are no vague objects, and the
vagueness in `Kilimanjaro' consists in it being indeterminate which
precise object it denotes.

As Burgess demonstrates, it is fair to require that the epistemicist
provide a theory of how terms get the precise content they do.
Williamson attempted to show he was in just as good a position to
discharge this obligation as the indeterminist by providing an algorithm
for converting any indeterminist theory of content into one acceptable
to the epistemicist. Burgess argued that the algorithm produced
unacceptable results when we applied it to vague sentences such as (1)
and (2). This particular argument is no good; the algorithm does not
seem to produce implausible results in that case. We can make this form
of argument work, however, especially if we focus on vague names.
Applying the algorithm to any plausible indeterminist theory produces
the result that every disjunct in (8) and (9) are false, and hence that
these disjunctions are false. Since the epistemicist is (correctly)
committed to these sentences being true, Burgess was correct to conclude
that ``this particular attempt to implement the parasite strategy is
doomed to failure.''

\section*{References}\label{references}
\addcontentsline{toc}{section}{References}

\phantomsection\label{refs}
\begin{CSLReferences}{1}{0}
\bibitem[\citeproctext]{ref-Burgess2001}
Burgess, John. 2001. {``Vagueness, Epistemicism and
Response-Dependence.''} \emph{Australasian Journal of Philosophy} 79
(4): 507--24. doi:
\href{https://doi.org/10.1080/713659306}{10.1080/713659306}.

\bibitem[\citeproctext]{ref-Williamson1994-WILV}
Williamson, Timothy. 1994. \emph{{Vagueness}}. Routledge.

\bibitem[\citeproctext]{ref-Williamson2000-WILKAI}
---------. 2000. \emph{{Knowledge and its Limits}}. Oxford University
Press.

\end{CSLReferences}



\noindent Published in\emph{
Australasian Journal of Philosophy}, 2003, pp. 276-279.

\end{document}
