% Options for packages loaded elsewhere
\PassOptionsToPackage{unicode}{hyperref}
\PassOptionsToPackage{hyphens}{url}
%
\documentclass[
  11pt,
  letterpaper,
  DIV=11,
  numbers=noendperiod,
  twoside]{scrartcl}

\usepackage{amsmath,amssymb}
\usepackage{setspace}
\usepackage{iftex}
\ifPDFTeX
  \usepackage[T1]{fontenc}
  \usepackage[utf8]{inputenc}
  \usepackage{textcomp} % provide euro and other symbols
\else % if luatex or xetex
  \usepackage{unicode-math}
  \defaultfontfeatures{Scale=MatchLowercase}
  \defaultfontfeatures[\rmfamily]{Ligatures=TeX,Scale=1}
\fi
\usepackage{lmodern}
\ifPDFTeX\else  
    % xetex/luatex font selection
    \setmainfont[ItalicFont=EB Garamond Italic,BoldFont=EB Garamond
Bold]{EB Garamond Math}
    \setsansfont[]{EB Garamond}
  \setmathfont[]{Garamond-Math}
\fi
% Use upquote if available, for straight quotes in verbatim environments
\IfFileExists{upquote.sty}{\usepackage{upquote}}{}
\IfFileExists{microtype.sty}{% use microtype if available
  \usepackage[]{microtype}
  \UseMicrotypeSet[protrusion]{basicmath} % disable protrusion for tt fonts
}{}
\usepackage{xcolor}
\usepackage[left=1.1in, right=1in, top=0.8in, bottom=0.8in,
paperheight=9.5in, paperwidth=7in, includemp=TRUE, marginparwidth=0in,
marginparsep=0in]{geometry}
\setlength{\emergencystretch}{3em} % prevent overfull lines
\setcounter{secnumdepth}{3}
% Make \paragraph and \subparagraph free-standing
\makeatletter
\ifx\paragraph\undefined\else
  \let\oldparagraph\paragraph
  \renewcommand{\paragraph}{
    \@ifstar
      \xxxParagraphStar
      \xxxParagraphNoStar
  }
  \newcommand{\xxxParagraphStar}[1]{\oldparagraph*{#1}\mbox{}}
  \newcommand{\xxxParagraphNoStar}[1]{\oldparagraph{#1}\mbox{}}
\fi
\ifx\subparagraph\undefined\else
  \let\oldsubparagraph\subparagraph
  \renewcommand{\subparagraph}{
    \@ifstar
      \xxxSubParagraphStar
      \xxxSubParagraphNoStar
  }
  \newcommand{\xxxSubParagraphStar}[1]{\oldsubparagraph*{#1}\mbox{}}
  \newcommand{\xxxSubParagraphNoStar}[1]{\oldsubparagraph{#1}\mbox{}}
\fi
\makeatother


\providecommand{\tightlist}{%
  \setlength{\itemsep}{0pt}\setlength{\parskip}{0pt}}\usepackage{longtable,booktabs,array}
\usepackage{calc} % for calculating minipage widths
% Correct order of tables after \paragraph or \subparagraph
\usepackage{etoolbox}
\makeatletter
\patchcmd\longtable{\par}{\if@noskipsec\mbox{}\fi\par}{}{}
\makeatother
% Allow footnotes in longtable head/foot
\IfFileExists{footnotehyper.sty}{\usepackage{footnotehyper}}{\usepackage{footnote}}
\makesavenoteenv{longtable}
\usepackage{graphicx}
\makeatletter
\newsavebox\pandoc@box
\newcommand*\pandocbounded[1]{% scales image to fit in text height/width
  \sbox\pandoc@box{#1}%
  \Gscale@div\@tempa{\textheight}{\dimexpr\ht\pandoc@box+\dp\pandoc@box\relax}%
  \Gscale@div\@tempb{\linewidth}{\wd\pandoc@box}%
  \ifdim\@tempb\p@<\@tempa\p@\let\@tempa\@tempb\fi% select the smaller of both
  \ifdim\@tempa\p@<\p@\scalebox{\@tempa}{\usebox\pandoc@box}%
  \else\usebox{\pandoc@box}%
  \fi%
}
% Set default figure placement to htbp
\def\fps@figure{htbp}
\makeatother
% definitions for citeproc citations
\NewDocumentCommand\citeproctext{}{}
\NewDocumentCommand\citeproc{mm}{%
  \begingroup\def\citeproctext{#2}\cite{#1}\endgroup}
\makeatletter
 % allow citations to break across lines
 \let\@cite@ofmt\@firstofone
 % avoid brackets around text for \cite:
 \def\@biblabel#1{}
 \def\@cite#1#2{{#1\if@tempswa , #2\fi}}
\makeatother
\newlength{\cslhangindent}
\setlength{\cslhangindent}{1.5em}
\newlength{\csllabelwidth}
\setlength{\csllabelwidth}{3em}
\newenvironment{CSLReferences}[2] % #1 hanging-indent, #2 entry-spacing
 {\begin{list}{}{%
  \setlength{\itemindent}{0pt}
  \setlength{\leftmargin}{0pt}
  \setlength{\parsep}{0pt}
  % turn on hanging indent if param 1 is 1
  \ifodd #1
   \setlength{\leftmargin}{\cslhangindent}
   \setlength{\itemindent}{-1\cslhangindent}
  \fi
  % set entry spacing
  \setlength{\itemsep}{#2\baselineskip}}}
 {\end{list}}
\usepackage{calc}
\newcommand{\CSLBlock}[1]{\hfill\break\parbox[t]{\linewidth}{\strut\ignorespaces#1\strut}}
\newcommand{\CSLLeftMargin}[1]{\parbox[t]{\csllabelwidth}{\strut#1\strut}}
\newcommand{\CSLRightInline}[1]{\parbox[t]{\linewidth - \csllabelwidth}{\strut#1\strut}}
\newcommand{\CSLIndent}[1]{\hspace{\cslhangindent}#1}

\setlength\heavyrulewidth{0ex}
\setlength\lightrulewidth{0ex}
\usepackage[automark]{scrlayer-scrpage}
\clearpairofpagestyles
\cehead{
  Brian Weatherson
  }
\cohead{
  Margins and Errors
  }
\ohead{\bfseries \pagemark}
\cfoot{}
\makeatletter
\newcommand*\NoIndentAfterEnv[1]{%
  \AfterEndEnvironment{#1}{\par\@afterindentfalse\@afterheading}}
\makeatother
\NoIndentAfterEnv{itemize}
\NoIndentAfterEnv{enumerate}
\NoIndentAfterEnv{description}
\NoIndentAfterEnv{quote}
\NoIndentAfterEnv{equation}
\NoIndentAfterEnv{longtable}
\NoIndentAfterEnv{abstract}
\renewenvironment{abstract}
 {\vspace{-1.25cm}
 \quotation\small\noindent\emph{Abstract}:}
 {\endquotation}
\newfontfamily\tfont{EB Garamond}
\addtokomafont{disposition}{\rmfamily}
\addtokomafont{title}{\normalfont\itshape}
\let\footnoterule\relax
\KOMAoption{captions}{tableheading}
\makeatletter
\@ifpackageloaded{caption}{}{\usepackage{caption}}
\AtBeginDocument{%
\ifdefined\contentsname
  \renewcommand*\contentsname{Table of contents}
\else
  \newcommand\contentsname{Table of contents}
\fi
\ifdefined\listfigurename
  \renewcommand*\listfigurename{List of Figures}
\else
  \newcommand\listfigurename{List of Figures}
\fi
\ifdefined\listtablename
  \renewcommand*\listtablename{List of Tables}
\else
  \newcommand\listtablename{List of Tables}
\fi
\ifdefined\figurename
  \renewcommand*\figurename{Figure}
\else
  \newcommand\figurename{Figure}
\fi
\ifdefined\tablename
  \renewcommand*\tablename{Table}
\else
  \newcommand\tablename{Table}
\fi
}
\@ifpackageloaded{float}{}{\usepackage{float}}
\floatstyle{ruled}
\@ifundefined{c@chapter}{\newfloat{codelisting}{h}{lop}}{\newfloat{codelisting}{h}{lop}[chapter]}
\floatname{codelisting}{Listing}
\newcommand*\listoflistings{\listof{codelisting}{List of Listings}}
\makeatother
\makeatletter
\makeatother
\makeatletter
\@ifpackageloaded{caption}{}{\usepackage{caption}}
\@ifpackageloaded{subcaption}{}{\usepackage{subcaption}}
\makeatother

\usepackage{bookmark}

\IfFileExists{xurl.sty}{\usepackage{xurl}}{} % add URL line breaks if available
\urlstyle{same} % disable monospaced font for URLs
\hypersetup{
  pdftitle={Margins and Errors},
  pdfauthor={Brian Weatherson},
  hidelinks,
  pdfcreator={LaTeX via pandoc}}


\title{Margins and Errors}
\author{Brian Weatherson}
\date{2013}

\begin{document}
\maketitle
\begin{abstract}
Timothy Williamson has argued that cases involving fallible measurement
show that knowledge comes apart from justified true belief in ways quite
distinct from the familiar `double luck' cases. I start by describing
some assumptions that are necessary to generate Williamson's conclusion,
and arguing that these assumptions are well justified. I then argue that
the existence of these cases poses problems for theorists who suppose
that knowledge comes apart from justified true belief only in a well
defined class of cases. I end with some general discussion of what we
can know on the basis of imperfect measuring devices.
\end{abstract}


\setstretch{1.1}
Recently, Timothy Williamson (\citeproc{ref-WilliamsonLofoten}{2013})
has argued that considerations about margins of errors can generate a
new class of cases where agents have justified true beliefs without
knowledge. I think this is a great argument, and it has a number of
interesting philosophical conclusions. In this note I'm going to go over
the assumptions of Williamson's argument. I'm going to argue that the
assumptions which generate the justification without knowledge are true.
I'm then going to go over some of the recent arguments in epistemology
that are refuted by Williamson's work. And I'm going to end with an
admittedly inconclusive discussion of what we can know when using an
imperfect measuring device.

\section{Measurement, Justification and
Knowledge}\label{measurement-justification-and-knowledge}

Williamson's core example involves detecting the angle of a pointer on a
wheel by eyesight. For various reasons, I find it easier to think about
a slightly different example: measuring a quantity using a digital
measurement device. This change has some costs relative to Williamson's
version -- for one thing, if we are measuring a quantity it might seem
that the margin of error is related to the quantity measured. If I
eyeball how many stories tall a building is, my margin of error is 0 if
the building is 1-2 stories tall, and over 10 if the building is as tall
as the World Trade Center. But this problem is not as pressing for
digital devices, which are often very \emph{unreliable} for small
quantities. And, at least relative to my preferences, the familiarity of
quantities makes up for the loss of symmetry properties involved in
angular measurement.

To make things explicit, I'll imagine the agent \emph{S} is using a
digital scale. The scale has a \textbf{margin of error} \emph{m}. That
means that if the reading, i.e., the \textbf{apparent mass} is \emph{a},
then the agent is justified in believing that the mass is in
{[}\emph{a}-\emph{m}, \emph{a}+\emph{m}{]}. We will assume that \emph{a}
and \emph{m} are luminous; i.e., the agent knows their values, and knows
she knows them, and so on. This is a relatively harmless idealisation
for \emph{a}; it is pretty clear what a digital scale reads.\footnote{This
  isn't always true. If a scale flickers between reading 832g and 833g,
  it takes a bit of skill to determine what \emph{the reading} is. But
  we'll assume it is clear in this case. On an analogue scale, the
  luminosity assumption is rather implausible, since it is possible to
  eyeball with less than perfect accuracy how far between one marker and
  the next the pointer is.} It is a somewhat less plausible assumption
for \emph{m}. But we'll assume that \emph{S} has been very diligent
about calibrating her scale, and that the calibration has been recently
and skilfully carried out, so in practice \emph{m} can be assessed very
accurately.

We'll make three further assumptions about \emph{m} that strike me as
plausible, but which may I guess be challenged. I need to be a bit
careful with terminology to set out the first one. I'll use \emph{V} and
\emph{v} as variables that both pick out the \textbf{true value} of the
mass. The difference is that \emph{v} picks it out rigidly, while
\emph{V} picks out the value of the mass in any world under
consideration. Think of \emph{V} as shorthand for \emph{the mass of the
object} and \emph{v} as shorthand for \emph{the actual mass of the
object}. (More carefully, \emph{V} is a \emph{random} variable, while
\emph{v} is a standard, rigid, variable.) Our first assumption then is
that \emph{m} is also related to what the agent can know. In particular,
we'll assume that if the reading \emph{a} equals \emph{v}, then the
agent can know that \emph{V} ∈ {[}\emph{a}-\emph{m},
\emph{a}+\emph{m}{]}, and can't know anything stronger than that. That
is, the margin of error for justification equals, in the best case, the
margin of error for knowledge. The second is that the scale has a
readout that is finer than \emph{m}. This is usually the case; the last
digit on a digital scale is often not significant. The final assumption
is that it is metaphysically possible that the scale has an error on an
occasion that is greater than \emph{m}. This is a kind of fallibilism
assumption -- saying that the margin of error is \emph{m} does not mean
there is anything incoherent about talking about cases where the error
on an occasion is greater than \emph{m}.

This error term will do a lot of work in what follows, so I'll use
\emph{e} to be the \textbf{error} of the measurement, i.e.,
\textbar{}\emph{a}-\emph{v}\textbar. For ease of exposition, I'll assume
that \emph{a}~⩾~\emph{v}, i.e., that any error is on the high side. But
this is entirely dispensible, and just lets me drop some disjunctions
later on.

Now we are in a position to state Williamson's argument. Assume that on
a particular occasion, 0~\textless~\emph{e}~\textless~\emph{m}. Perhaps
\emph{v}~=~830, m~=~10 and \emph{a}~=~832, so \emph{e}~=~2. Williamson
appears to make the following two assumptions.\footnote{I'm not actually
  sure whether Williamson \emph{makes} the first, or thinks it is the
  kind of thing anyone who thinks justification is prior to knowledge
  should make.}

\begin{enumerate}
\def\labelenumi{\arabic{enumi}.}
\tightlist
\item
  The agent is justified in believing what they would know if
  appearances matched reality, i.e., if \emph{V} equalled \emph{a}.
\item
  The agent cannot come to know something about \emph{V} on the basis of
  a suboptimal measurement that they could not also know on the basis of
  an optimal measurement.
\end{enumerate}

I'm assuming here that the optimal measurement displays the correct
mass. I don't assume the actual measurement is \emph{wrong}. That would
require saying something implausible about the semantic content of the
display. It's not obvious that the display has a content that could be
true or false, and if it does have such a content it might be true. (For
instance, the content might be that the object on the scale has a mass
near to \emph{a}, or that with a high probability it has a mass near to
\emph{a}, and both of those things are true.) But the optimal
measurement would be to have \emph{a}~=~\emph{v}, and in this sense the
measurement is suboptimal.

The argument then is pretty quick. From the first assumption, we get
that the agent is justified in believing that
\emph{V}~∈~{[}\emph{a}-\emph{m}, \emph{a}+\emph{m}{]}. Assume then that
the agent forms this justified belief. This belief is incompatible with
\emph{V}~∈~{[}\emph{v}-\emph{m}, \emph{a}-\emph{m}). But if \emph{a}
equalled \emph{v}, then the agent wouldn't be in a position to rule out
that \emph{V}~∈~{[}\emph{v}-\emph{m}, \emph{a}-\emph{m}). So by premise
2 she can't knowledgeably rule it out on the basis of a mismeasurement.
So her belief that \emph{V}~⩾~\emph{a}-\emph{m} cannot be knowledge. So
this justified true belief is not knowledge.

If you prefer doing this with numbers, here's the way the example works
using the numbers above. The mass of the object is 830. So if the
reading was correct, the agent would know just that the mass is between
820 and 840. The reading is 832. So she's justified in believing, and
we'll assume she does believe, that the mass is between 822 and 842.
That belief is incompatible with the mass being 821. But by premise 2
she can't know the mass is greater than 821. So the belief doesn't
amount to knowledge, despite being justified and, crucially, true. After
all, 830 is between 822 and 842, so her belief that the mass is in this
range is true. So simple reflections on the workings on measuring
devices let us generate cases of justified true beliefs that are not
knowledge.

I'll end this section with a couple of objections and replies.

\emph{Objection}: The argument that the agent can't know that
\emph{V}~∈~{[}\emph{a}-\emph{m}, \emph{a}+\emph{m}{]} is also an
argument that the argument can't justifiably believe that
\emph{V}~∈~{[}\emph{a}-\emph{m}, \emph{a}+\emph{m}{]}. After all, why
should it be possible to get justification from a suboptimal measurement
when it isn't possible to get the same justification from an optimal
measurement?

\emph{Reply}: It is possible to have justification to believe an
outright falsehood. It is widely believed that you can have
justification even when none of your evidential sources are even
approximately accurate (\citeproc{ref-Cohen1984}{Cohen 1984}). And even
most reliabilists will say that you can have false justified beliefs if
you use a belief forming method that is normally reliable, but which
badly misfires on this occasion. In such cases we clearly get
justification to believe something from a mismeasurement that we
wouldn't get from a correct measurement. So the objection is based on a
mistaken view of justification.

\emph{Objection}: Premise 2 fails in cases using random sampling. Here's
an illustration. An experimenter wants to know what percentage of
\emph{F}s are \emph{G}. She designs a survey to ask people whether they
are \emph{G}. The survey is well designed; everyone gives the correct
answer about themselves. And she designs a process for randomly sampling
the \emph{F}s to get a good random selection of 500. It's an excellent
process; every \emph{F} had an equal chance of being selected, and the
sample fairly represents the different demographically significant
subgroups of the \emph{F}s. But by the normal processes of random
variation, her group contains slightly more \emph{G}s than the average.
In her survey, 28\% of people said (truly!) that they were \emph{G},
while only 26\% of \emph{F}s are \emph{G}s. Assuming a margin of error
in such a study of 4\%, it seems plausible to say she knows that between
25 and 32\% of \emph{F}s are \emph{G}s. But that's not something she
could have known the survey had come back correctly reporting that 26\%
of \emph{F}s are \emph{G}s.

\emph{Reply}: I think the core problem with this argument comes in the
last sentence. A random survey isn't, in the first instance, a
measurement of a population. It's a measurement of those surveyed, from
which we draw extrapolations about the population. In that sense, the
only \emph{measurement} in the imagined example was as good as it could
be; 28\% of surveyed people are in fact \emph{G}. So the survey was
correct, and it is fine to conclude that we can in fact know that
between 24 and 32 percent of \emph{F}s are \emph{G}s.

There are independent reasons for thinking this is the right way to talk
about the case. If a genuine measuring device, like a scale, is off by a
small amount, we regard that as a reason for tinkering with the device,
and trying to make it more accurate. That's one respect in which the
measurement is suboptimal, even if it is correct within the margin of
error. This reason to tinker with the scale is a reason that often will
be outweighed. Perhaps it is technologically infeasible to make the
machine more accurate. More commonly, the only way to guarantee greater
accuracy would be more cost and hassle than it is worth. But it remains
a reason. The fact that this experiment came out with a deviation
between the sample and the population is \emph{not} a reason to think
that it could have been run in a better way, or that there is some
reason to improve the survey. That's just how random sampling goes. If
it were a genuine measurement of the population, the deviation between
the `measurement' and what is being measured would be a reason to do
things differently. There isn't any such reason, so the sample is not
truly a measurement.

So I don't think this objection works, and I think the general principle
that you can't get extra knowledge from a suboptimal measurement is
right. But note also that we don't need this general principle to
suggest that there will be cases of justified true belief without
knowledge in the cases of measurement. Consider a special case where
\emph{e} is just less than \emph{m}. For concreteness, say
\emph{a}~=~\emph{v}+0.95\emph{m}, so \emph{e}~=~0.95\emph{m}. Now assume
that whatever is justifiedly truly believed in this case is known, so
\emph{S} knows that \emph{V}~∈~{[}\emph{a}-\emph{m},
\emph{a}+\emph{m}{]}. That is, \emph{S} knows that
\emph{V}~∈~{[}v-0.05\emph{m}, \emph{a}+\emph{m}{]}.

We don't need any principles about measurement to show this is false;
safety considerations will suffice. Williamson
(\citeproc{ref-Williamson2000-WILKAI}{2000}) says that a belief that
\emph{p} is safe only if \emph{p} is true in all nearby worlds. But
given how close \emph{v} is to the edge of the range
{[}\emph{v}-0.05\emph{m},~\emph{a}+\emph{m}{]}. Rival conceptions of
safety don't help much more than this. The most prominent of these,
suggested by Sainsbury (\citeproc{ref-Sainsbury1996}{1995}), says that a
belief is safe only if the method that produced it doesn't produce a
false belief in any nearby world. But if the scale was off by
0.95\emph{m}, it could have been off by 1.05\emph{m}, so that condition
fails too.

I don't want the last two paragraphs to leave too concessive an
impression. I think the objection fails because it relies on a
misconception of the notion of measurement. But I think that even if the
objection works, we can get a safety based argument that some
measurement cases will produce justified true beliefs without knowledge.
And that will matter for the argument of the next two sections.

\section{The Class of Gettier Cases is
Disjunctive}\label{the-class-of-gettier-cases-is-disjunctive}

There's an unfortunate terminological confusion surrounding gaps between
knowledge and justification. Some philosophers use the phrase `Gettier
case' to describe any case of a justified true belief that isn't
knowledge. Others use it to describe just cases that look like the cases
in Gettier (\citeproc{ref-Gettier1963}{1963}), i.e., cases of true
belief derived from justified false belief. I don't particularly have
strong views on whether either of these uses is \emph{better}, but I do
think it is important to keep them apart.

I'll illustrate the importance of this by discussing a recent argument
due to Jeremy Fantl and Matthew McGrath
(\citeproc{ref-FantlMcGrath2009}{Fantl and McGrath 2009} Ch. 4). I've
previously discussed this argument
(\citeproc{ref-Weatherson2011-WEAKBI}{Weatherson 2011}), but I don't
think I quite got to the heart of why I don't like the kind of reasoning
they are using.

The argument concerns an agent, call her \emph{T}, who has the following
unfortunate combination of features. She is very confident that
\emph{p}. And with good reason; her evidence strongly supports \emph{p}.
For normal reasoning, she takes \emph{p} for granted. That is, she
doesn't distinguish between ϕ is best given \emph{p}, and that ϕ is
simply best. And that's right too, given the strong evidence that
\emph{p}. But she's not crazy. Were she to think that she was facing a
bet on extreme odds concerning \emph{p}, she would cease taking \emph{p}
for granted, and revert to trying to maximise expected value given the
high probability that \emph{p}. But she doesn't think any such bet is
salient, so her disposition to retreat from \emph{p} to \emph{Probably
p} has not been triggered. So far, all is going well. I'm inclined to
say that this is enough to say that \emph{T} justifiedly believes that
\emph{p}. She believes that \emph{p} in virtue of the fact that she
takes \emph{p} for granted in actual reasoning.\footnote{There are some
  circumlocutions here because I'm being careful to be sensitive to the
  points raised in Ross and Schroeder
  (\citeproc{ref-SchroederRoss2012}{2014}) about the relationship
  between belief and reasoning. I think there's less distance between
  the view they put forward and the view I defended in Weatherson
  (\citeproc{ref-Weatherson2005-WEACWD}{2005}) than they do, but this is
  a subtle matter, and for this paper's purposes I want to go along with
  Ross and Schroeder's picture of belief.} She's disposed to stop doing
so in some circumstances, but until that disposition is triggered, she
has the belief. And this is the right way to act given her evidence, so
her belief is justified. So far, so good.

Unfortunately, \emph{T} really does face a bet on long odds about
\emph{p}. She knows she has to choose between ϕ and ψ. And she knows
that ϕ will produce the better outcome iff \emph{p}. But she thinks the
amount she'll gain by choosing ψ if ¬\emph{p} is roughly the same as the
amount she'll gain by choosing ϕ if \emph{p}. That's wrong, and her
evidence clearly shows it is wrong. If \emph{p} is false, then ϕ will be
\emph{much} worse than ψ. In fact, the potential loss here is so great
that ψ has the greater expected value given the correct evidential
probability of \emph{p}. I think that means she doesn't know that
\emph{p}. Someone who knows that \emph{p} can ignore ¬\emph{p}
possibilities in practical reasoning. And someone who could ignore
¬\emph{p} possibilities in practical reasoning would choose ϕ over ψ,
since it is better if \emph{p}. But \emph{T} isn't in a position to make
that choice, so she doesn't know that \emph{p}.

(I've said here that \emph{T} is wrong about the costs of choosing ϕ if
\emph{p}, and her evidence shows she is wrong. In fact I think she
doesn't know \emph{p} if either of those conditions obtain. But here I
only want to use the weaker claim that she doesn't know \emph{p} if both
conditions obtain.)

Fantl and McGrath agree about the knowledge claim, but disagree about
the justified belief claim. They argue as follows (this is my version of
the `Subtraction Argument' from page 97 of their book).

\begin{enumerate}
\def\labelenumi{\arabic{enumi}.}
\tightlist
\item
  \emph{T} is justfied in choosing ϕ iff she knows that \emph{p}.
\item
  Whether \emph{T}'s belief that \emph{p} is true is irrelevant to
  whether she is justified in choosing ϕ.
\item
  Whether \emph{T}'s belief that \emph{p} is `Gettiered' is irrelevant
  to whether she is justified in choosing ϕ.
\item
  Knowledge is true, justified, UnGettiered belief.
\item
  So \emph{T} is justfied in choosing ϕ iff she is justified in
  believing that \emph{p}.
\item
  \emph{T} is not justified in choosing ϕ.
\item
  So \emph{T} is not justified in believing that \emph{p}.
\end{enumerate}

I think this argument is only plausible if we equivocate on what it is
for a belief to be `Gettiered'.

Assume first that `Gettiered' means `derived from a false intermediate
step'. Then premise 4 is false, as Williamson's example shows. \emph{S}
has a justified true belief that is neither knowledge nor derived from a
false premise.

Assume then that `Gettiered' simply means that the true belief is
justified without being known. In that case we have no reason to accept
premise 3. After all, the class of true justified beliefs that are not
knowledge is pretty open ended. Before reading Williamson, we may not
have thought that this class included the beliefs of agents using
measuring devices that were functioning properly but imperfectly. But it
does. Prior to the end of epistemology, we simply don't know what other
kind of beliefs might be in this class. There's no way to survey all the
ways for justification to be insufficient for knowledge, and see if all
of them are irrelevant to the justification for action. I think one way
a justified belief can fall short of knowledge is if it is tied up with
false beliefs about the stakes of bets. It's hard to say that that is
irrelevant to the justification of action.

It is by now reasonably well known that logical subtraction is a very
messy and complicated business. See, for instance, Humberstone
(\citeproc{ref-Humberstone2000}{2000}) for a clear discussion of the
complications. In general, unless it is analytic that \emph{F}s are
\emph{G}s and \emph{H}s, for some antecedently understood \emph{G} and
\emph{H}, there's nothing interesting to say about the class of things
that are \emph{G} but not \emph{F}. It will just be a disjunctive
shambles. The same is true for knowledge and justification. The class of
true beliefs that are justified but not known is messy and disjunctive.
We shouldn't expect to have any neat way of overviewing it. That in part
means we can't say much interesting about it as a class, contra premise
3 in the above argument. It also means the prospects for `solving the
Gettier problem' are weak. We'll turn to that issue next.

\section{There is No Solution to the Gettier
Problem}\label{there-is-no-solution-to-the-gettier-problem}

The kind of example that Edmund Gettier
(\citeproc{ref-Gettier1963}{1963}) gives to refute the justified true
belief theory of knowledge has what Linda Zagzebski
(\citeproc{ref-Zagzebski2009}{2009, 117}) aptly calls a ``double luck''
structure. In Gettier's original cases, there's some bad luck that leads
to a justified belief being false. But then there's some good luck that
leads to an inference from that being true. As was quickly realised in
the literature, the good and bad luck doesn't need to apply to separate
inferential steps. It might be that the one belief that would have been
false due to bad luck also ends up being true due to good luck.

This has led to a little industry, especially in the virtue epistemology
section of the market, of attempts to ``solve the Gettier problem'' by
adding an anti-luck condition to justification, truth and belief and
hoping that the result is something like an analysis of knowledge. As
Zagzebski (\citeproc{ref-Zagzebski1994}{1994}) showed, this can't be an
\emph{independent} condition on knowledge. If it doesn't entail truth,
then we will be able to recreate the Gettier cases. But maybe a `fourth'
condition that entails truth (and perhaps belief) will suffice. Let's
quickly review some of these proposals.

So Zagzebski (\citeproc{ref-Zagzebski1996}{1996}) suggested that the
condition is that the belief be true \emph{because} justified. John
Greco (\citeproc{ref-Greco2010}{2010}) says that the extra condition is
that the beliefs be ``intellectually creditable''. That is, the primary
that the subject ended up with a true belief is that it was the result
of her reliable cognitive faculties. Ernest Sosa
(\citeproc{ref-Sosa2007}{2007}) said that knowledge is belief that is
true because it manifests intellectual competence. John Turri
(\citeproc{ref-Turri2011}{2011}) says that knowledge is belief the truth
of which is a manifestation of the agent's intellectual competence.

It should be pretty clear that no such proposal can work if what I've
said in earlier sections is remotely right. Assume again that
\emph{v}~~=~830, a~=~832 and \emph{m}~=~10. The agent believes that
\emph{V}~∈~{[}822, 842{]}. This belief is, we've said, justified and
true. Does it satisfy these extra conditions?

My short answer is that it does. My longer answer is that it does if any
belief derived from the use of a measuring device does, and since some
beliefs derived from the use of measuring devices amount to knowledge,
the epistemologists are committed to the belief satisfying the extra
condition. Let's go through those arguments in turn.

In our story, \emph{S} demonstrates a range of intellectual
competencies. She uses a well-functioning measuring device. It is the
right kind of device for the purpose she is using. By hypothesis, she
has had the machine carefully checked, and knows exactly the accuracy of
the machine. She doesn't form any belief that is too precise to be
justified by the machine. And she ends up with a true belief precisely
because she has so many competencies.

Note that if we change the story so \emph{a} is closer to
\emph{v}+\emph{m}, the case that the belief is true in virtue of
\emph{S} being so competent becomes even stronger. Change the case so
that \emph{a}~=~839, and she forms the true belief that
\emph{V}~∈~{[}829, 849{]}. Now if \emph{S} had not been so competent,
she may have formed a belief with a tighter range, since she could
easily have guessed that the margin of error of the machine is smaller.
So in this case the truth of the belief is very clearly due to her
competence. But as we noted at the end of section 1, in the cases where
\emph{a} is near \emph{v}+\emph{m}, the argument that we have justified
true belief without knowledge is particularly strong. Just when the gap
between justification and knowledge gets most pronounced, the competence
based approach to knowledge starts to issue the strongest verdicts
\emph{in favour} of knowledge.

But maybe this is all a mistake. After all, the object doesn't have the
mass it has because of \emph{S}'s intellectual competence. The truth of
any claim about its mass is not because of \emph{S}'s competence, or a
manifestation of that competence. So maybe these epistemologists get the
correct verdict that \emph{S} does not know that
\emph{V}~∈~{[}\emph{a}-\emph{m}, \emph{a}+\emph{m}{]}?

Not so quick. Even had \emph{a} equalled \emph{v}, all these claims
would have been true. And in that case, \emph{S} would have known that
\emph{V} was within \emph{m} of the measurement. What is needed for
these epistemological theories to be right is that there can be a sense
that a belief that \emph{p} can be true in virtue of some cause \emph{C}
without \emph{C} being a cause of \emph{p}. I'm inclined to agree with
the virtue epistemologists that such a sense can be given. (I think it
helps to give up on content essentialism for this project, as suggested
by David (\citeproc{ref-David2002}{2002}) and endorsed in Weatherson
(\citeproc{ref-Weatherson2004-WEALMT}{2004}).) But I don't think it will
help. There's no real way in which a belief is true because of
competencies, or in which the truth of a belief manifests competence, in
the good case where \emph{a}~=~\emph{v}, but not in the bad cases, where
\emph{a} is in (0,~\emph{m}). These proposals might help with `double
luck' cases, but there is more to the space between justification and
knowledge than those cases. Of course, I think the space in question
includes some cases involving false beliefs about the practical
significance of \emph{p}, but I don't expect everyone to agree with
that. Happily, the Williamsonian cases should be less controversial.

\section{What Can We Learn from Fallible
Machines?}\label{what-can-we-learn-from-fallible-machines}

My presentation of Williamson's argument in section 1 abstracted away
from several features of his presentation. In particular, I didn't make
any positive assumption about what the agent can know when they find out
that the machine reads \emph{a}. Williamson makes a suggestion, though
he offers it more as the most internalist friendly suggestion than the
most likely correct hypothesis.

The suggestion, which I'll call the \textbf{Circular Reading Centred}
hypothesis, is that the most the agent can know is that
\emph{V}~∈~{[}a-(\emph{e}+\emph{m}), a+(\emph{e}+\emph{m}){]}. That is,
the agent can know that \emph{V} is in a region centred on \emph{a}, the
`radius' of which is the margin of error \emph{m}, plus the error on
this occasion \emph{e}. This is actually a quite attractive suggestion,
though not the only suggestion we could make. Let's look through some
other options and see how well they work.

We said above that the agent can't know more from a mismeasurement than
they can know from an accurate measurement. And we said that given an
accurate measurement, the most they can know is that
\emph{V}~∈~{[}\emph{v}-\emph{m}, \emph{v}+\emph{m}{]}. So here's one
very restrictive suggestion: if \emph{a}~∈~{[}\emph{v}-\emph{m},
\emph{v}+\emph{m}{]}, then the agent can know that
\emph{V}~∈~{[}\emph{v}-\emph{m},~\emph{v}+\emph{m}{]}. But we can easily
rule that out on the basis of considerations about justification. The
strongest proposition the agent is justified in believing is that
\emph{V}~∈~{[}\emph{a}-\emph{m}, \emph{a}+\emph{m}{]}. If the agent
could know that \emph{V}~∈~{[}\emph{v}-\emph{m}, \emph{v}+\emph{m}{]},
then she could know that \emph{V}~∉ (\emph{v}+\emph{m},
\emph{a}+\emph{m}{]}, even though she isn't justified in believing this.
This is absurd, so that proposal is wrong.

We now have two principles on the table: \emph{S} can't know anything by
a mismeasurement that she knows on the basis of a correct measurement,
and that she can only know things she's justified in believing. The
first principle implies that for all \emph{x}~∈~{[}\emph{v}-\emph{m},
\emph{v}+\emph{m}{]}, that \emph{V}~~=~\emph{x} is epistemically
possible. The second implies that for all
\emph{x}~∈~{[}\emph{a}-\emph{m}, \emph{a}+\emph{m}{]}, that
\emph{V}~~=~\emph{x} is epistemically possible. Our next proposal is
that the epistemic possibilities, given a reading of \emph{a}, are just
that \emph{V}~∈~{[}\emph{v}-\emph{m}, \emph{v}+\emph{m}{]} ∪
{[}\emph{a}-\emph{m}, \emph{a}+\emph{m}{]}.

But this is fairly clearly absurd too. Assume that
\emph{a}~\textgreater~\emph{v}+2\emph{m}. This is unlikely, but as we
said above not impossible. Now consider the hypothesis that
\emph{V}~∈~(\emph{v}+\emph{m}, \emph{a}-\emph{m}). On the current
hypothesis, this would be ruled out. That is, she would know it doesn't
obtain. But this seems bizarre. There are epistemic possibilities all
around it, but somehow she's ruled out this little gap, and done so on
the basis of a horrifically bad measurement.

This suggests two other approaches that are consistent with the two
principles, and which do not have such an odd result. I'll list them
alongside the proposal we mentioned earlier.

\begin{description}
\tightlist
\item[Circular Appearance Centred]
The strongest proposition the agent can know is that
\emph{V}~∈~{[}a-(\emph{e}+\emph{m}), a+(\emph{e}+\emph{m}){]}.
\item[Circular Reality Centred]
The strongest proposition the agent can know is that
\emph{V}~∈~{[}\emph{v}-(\emph{e}+\emph{m}), v+(\emph{e}+\emph{m}){]}.
\item[Elliptical]
The strongest proposition the agent can know is that
\emph{V}~∈~{[}\emph{v}-\emph{m}, \emph{a}+\emph{m}{]}.
\end{description}

The last proposal is called \textbf{Elliptical} because it in effect
says that there are two foci for the range of epistemic possibilities.
The agent can't rule out anything within \emph{m} of the true value, or
anything within \emph{m} of the apparent value, or anything between
those.

Actually we can motivate the name even more by considering a slight
generalisation of the puzzle that we started with. Assume that \emph{R}
is trying to determine the location of an object in a two-dimensional
array. As before, she has a digital measuring device, perhaps a GPS
locator trained on the object in question. And she knows that margin of
error of the device is \emph{m}. The object is actually located at
⟨\emph{x\textsubscript{v}}, \emph{y\textsubscript{v}}⟩, and the device
says it is at ⟨\emph{x\textsubscript{a}}, \emph{y\textsubscript{a}}⟩. So
the epistemic possibilities, by the reasoning given above, should
include the circles with radius \emph{m} centred on
⟨\emph{x\textsubscript{v}}, \emph{y\textsubscript{v}}⟩ and
⟨\emph{x\textsubscript{a}}, \emph{y\textsubscript{a}}⟩. Call these
circles \emph{C\textsubscript{v}} and \emph{C\textsubscript{a}}. Unless
⟨\emph{x\textsubscript{v}}, \emph{y\textsubscript{v}}⟩ =
⟨\emph{x\textsubscript{a}}, \emph{y\textsubscript{a}}⟩, the union of
these circles will not be convex. If the distance between
⟨\emph{x\textsubscript{v}}, \emph{y\textsubscript{v}}⟩ and
⟨\emph{x\textsubscript{a}}, \emph{y\textsubscript{a}}⟩ is greater than
2\emph{m}, the union won't even be connected. So just as we `filled in'
the gap in the one-dimensional case, the natural thing to say is that
any point in the convex hull of \emph{C\textsubscript{v}} and
\emph{C\textsubscript{a}} is an epistemic possibility.

But now see what happens if we say those are all of the epistemic
possibilities, i.e., that the agent knows that the true value lies in
the convex hull of the two circles. Here's what it might look like.

Now consider the line from ⟨\emph{x\textsubscript{v}},
\emph{y\textsubscript{v}}⟩ to ⟨\emph{x\textsubscript{a}},
\emph{y\textsubscript{a}}⟩. No matter how bad the measurement is, the
convex hull of the two circles \emph{C\textsubscript{v}} and
\emph{C\textsubscript{a}} will include no points more than distance
\emph{m} from the line between ⟨\emph{x\textsubscript{v}},
\emph{y\textsubscript{v}}⟩ to ⟨\emph{x\textsubscript{a}},
\emph{y\textsubscript{a}}⟩. That is, the agent can know something
surprisingly precise about how close \emph{V} is to a particular line,
even on the basis of a catastrophically bad measurement.

There are some circumstances where this wouldn't be counterintuitive.
Assume that \emph{x\textsubscript{v}}~=~\emph{x\textsubscript{a}}, while
\emph{y\textsubscript{v}} and \emph{y\textsubscript{a}} are very very
different. And assume further that ⟨\emph{x\textsubscript{a}},
\emph{y\textsubscript{a}}⟩ is calculated by using two very different
procedures for the \emph{x} and \emph{y} coordinates. (Much as sailors
used to use very different procedures to calculate longitude and
latitude.) Then the fact that one process failed badly doesn't, I think,
show that we can't get fairly precise knowledge from the other process.

But that's not the general case. If the machine determines
⟨\emph{x\textsubscript{a}}, \emph{y\textsubscript{a}}⟩ by a more
holistic process, then a failure on one dimension should imply that we
get less knowledge on other dimensions, since it makes it considerably
flukier that we got even one dimension right. So I think the space of
epistemic possibilities, in a case involving this kind of errant
measurement, must be greater than the convex hull of
\emph{C\textsubscript{v}} and \emph{C\textsubscript{a}}.

Fortunately, there are a couple of natural generalisations of the
elliptical proposal that avoid this complication. One of them says that
the space of epistemic possibilities forms an ellipse. In particular, it
is the set of all points such that the sum of the distance from that
point to ⟨\emph{x\textsubscript{v}}, \emph{y\textsubscript{v}}⟩ and the
distance from that point to ⟨\emph{x\textsubscript{a}},
\emph{y\textsubscript{a}}⟩ is less than or equal to 2\emph{m}+\emph{e},
where \emph{e} again is the distance between the measured and actual
value. As you can quickly verify, that includes all points on the line
from ⟨\emph{x\textsubscript{v}}, \emph{y\textsubscript{v}}⟩ to
⟨\emph{x\textsubscript{a}}, \emph{y\textsubscript{a}}⟩, plus an
extension of length \emph{m} beyond in each direction. But it doesn't
just contain the straight path between \emph{C\textsubscript{v}} and
\emph{C\textsubscript{a}}; it `bulges' in the middle. And the
considerations above suggest that is what should happen.

The other alternative is to drop the idea that the space of
possibilities should be elliptical, and have another circular proposal.
In particular, we say that the space of possibilities is the circle
whose centre is halfway between ⟨\emph{x\textsubscript{v}},
\emph{y\textsubscript{v}}⟩ and ⟨\emph{x\textsubscript{a}},
\emph{y\textsubscript{a}}⟩, and whose radius is \emph{m}+\emph{e}/2.
Again, that will include all points on the line from
⟨\emph{x\textsubscript{v}}, \emph{y\textsubscript{v}}⟩ to
⟨\emph{x\textsubscript{a}}, \emph{y\textsubscript{a}}⟩, plus an
extension of length \emph{m} beyond in each direction. But it will
include a much larger space in the middle.

I think both of these are somewhat plausible proposals, though the
second suffers from a slightly weaker version of the objection I'm about
to mount to the Circular Reality Centred proposal. But they do share one
weakness that I think counts somewhat against them. It's easy enough to
see what the weakness is in the one-dimensional case, so let's return to
it for the time being, and remember we're assuming that
\emph{a}~\textgreater~\emph{v}.

Consider a case where \emph{e} is rather large, much larger than
\emph{m}. This affects how far below \emph{v} we have to go in order to
reach possibilities that are ruled out by the measurement. But it
doesn't affect how far above \emph{v} we have to go in order to reach
such possibilities. Indeed, no matter how bad \emph{e} is, we can be
absolutely certain that we know \emph{V}~\textless~\emph{a}+2\emph{m},
or that we know that \emph{V}~\textgreater~\emph{a}-2\emph{m}. That
seems a little odd; if the measurement is so badly mistaken, it seems
wrong that it can give us such a fine verdict, at least in one
direction.

I don't think that's a conclusive objection. Well, I don't think many of
the considerations I've listed here are \emph{conclusive}, but this
seems even weaker. But it is a reason to look away from the elliptical
proposal and back towards the circular proposals that we started with.

If we just look at first order knowledge claims, it is hard to feel much
of an intuitive pull towards one or other of the alternatives. Perhaps
safety based considerations favour the Reality Centred over the
Appearance Centred version, but I don't think the salient safety
consideration is that strong.

If we look at iterated knowledge claims, however, there is a big problem
with the Reality Centred approach. The intuition here is clearer if we
use numerical examples, so I'll work through a case with numbers first,
then do the general version next.

Assume, as above, that \emph{v}~~=~830, \emph{a}~=~834 and
\emph{m}~=~10. So we have a pretty decent measurement here. On the
Reality Centred proposal, the strongest thing that \emph{S} can know is
that \emph{V}~∈~{[}816, 844{]}. So it is an epistemic possibility that
\emph{V}~~=~816. Assume that that's the actual possibility. Then the
measurement is rather bad; the new value for \emph{e} is 18. Were
\emph{V} to equal 816, while \emph{a} equalled 834, then on the Reality
Centred approach, the epistemic possibilities would be a circle of
radius \emph{e}+\emph{m}, i.e., 28, around the actual value, i.e., 816.
So the strongest thing the agent could know is that \emph{V}~∈~{[}788,
844{]}. On the other hand, if \emph{V} were 844, the strongest thing the
agent could know is that \emph{V}~∈~{[}824, 864{]}. Putting those
together, the strongest thing the agent can know that she knows is that
\emph{V}~∈~{[}788, 864{]}. That's a very large range already. Similar
calculations show that the strongest thing the agent can know that she
knows that she knows is that \emph{V}~∈~{[}732, 904{]}.

Now I'll grant that intuitions about second and third order knowledge
are not always maximally sharp. But I think it is very implausible that
a relatively accurate measurement like this could lead to such radical
ignorance in the second and third orders of knowledge. So I think the
Reality Centred approach can't be right.

The general form the case is as follows. The strongest thing the agent
can know is that \emph{V}~∈~{[}\emph{v}-(\emph{e}+\emph{m}),
\emph{a}+\emph{m}{]}. The strongest thing she can know that she knows is
that \emph{V}~∈~{[}\emph{v}-3(\emph{e}+\emph{m}), a+3m{]}. And the
strongest thing she can know that she knows that she knows is that
\emph{V}~∈~{[}\emph{v}-7(\emph{e}+\emph{m}), \emph{a}+7\emph{m}{]}. In
general, we have \emph{exponential} growth of the possibilities as we
add one extra order of knowledge. That seems absurd to me, so the
Reality Centred approach is wrong.

Note that this isn't a problem with the Appearance Centred approach. The
first-order epistemic possibilities are that
\emph{V}~∈~{[}a-(\emph{e}+\emph{m}), a+\emph{e}+\emph{m}{]}. If \emph{V}
is at the extremes of this range, then \emph{e} will be rather large.
For example, if \emph{V} were equal to \emph{a}+\emph{e}+\emph{m}, then
the new error would be \emph{e}+\emph{m}, since the measured value is
still \emph{a}. So the range of possibilities would be that
\emph{V}~∈~{[}a-((\emph{e}+\emph{m})+m), a+((\emph{e}+\emph{m})+m){]}.
Somewhat surprisingly, those would also be the possibilities if \emph{V}
were equal to \emph{a}-(\emph{e}+\emph{m}), since the only feature of
\emph{V} that affects the epistemic possibilities for \emph{V} is its
distance from \emph{a}. So for all \emph{S} knows that she knows,
\emph{V} could be anything in {[}a-(e+2m), a+(e+2m){]}. Similar
reasoning shows that for all \emph{V} knows that she knows that she
knows, \emph{V} could be anything in {[}a-(e+3m), a+(e+3m){]}. In
general, \emph{V} has \emph{n}'th order knowledge that \emph{V} is in
{[}\emph{a}-(\emph{e}+\emph{nm}), \emph{a}+(\emph{e}+\emph{nm}){]}. This
linear growth in the size of the range of epistemic possibilities is
more plausible than the exponential growth on the Reality Centred
approach.

So all things considered, I think the Circular Appearance Centred
approach is the right one, as Williamson suggests. Any simple
alternative seems to have rather counterintuitive consequences.

\subsection*{References}\label{references}
\addcontentsline{toc}{subsection}{References}

\phantomsection\label{refs}
\begin{CSLReferences}{1}{0}
\bibitem[\citeproctext]{ref-Cohen1984}
Cohen, Stewart. 1984. {``Justification and Truth.''} \emph{Philosophical
Studies} 46 (3): 279--95. doi:
\href{https://doi.org/10.1007/bf00372907}{10.1007/bf00372907}.

\bibitem[\citeproctext]{ref-David2002}
David, Marian. 2002. {``Content Essentialism.''} \emph{Acta Analytica}
17: 103--14. doi:
\href{https://doi.org/10.1007/bf03177510}{10.1007/bf03177510}.

\bibitem[\citeproctext]{ref-FantlMcGrath2009}
Fantl, Jeremy, and Matthew McGrath. 2009. \emph{Knowledge in an
Uncertain World}. Oxford: Oxford University Press.

\bibitem[\citeproctext]{ref-Gettier1963}
Gettier, Edmund L. 1963. {``Is Justified True Belief Knowledge?''}
\emph{Analysis} 23 (6): 121--23. doi:
\href{https://doi.org/10.2307/3326922}{10.2307/3326922}.

\bibitem[\citeproctext]{ref-Greco2010}
Greco, John. 2010. \emph{Achieving Knowledge}. Cambridge: Cambridge
University Press.

\bibitem[\citeproctext]{ref-Humberstone2000}
Humberstone, Lloyd. 2000. {``Parts and Partitions.''} \emph{Theoria} 66
(1): 41--82. doi:
\href{https://doi.org/10.1111/j.1755-2567.2000.tb01144.x}{10.1111/j.1755-2567.2000.tb01144.x}.

\bibitem[\citeproctext]{ref-SchroederRoss2012}
Ross, Jacob, and Mark Schroeder. 2014. {``Belief, Credence, and
Pragmatic Encroachment.''} \emph{Philosophy and Phenomenological
Research} 88 (2): 259--88. doi:
\href{https://doi.org/10.1111/j.1933-1592.2011.00552.x}{10.1111/j.1933-1592.2011.00552.x}.

\bibitem[\citeproctext]{ref-Sainsbury1996}
Sainsbury, Mark. 1995. {``Vagueness, Ignorance and Margin for Error.''}
\emph{British Journal for the Philosophy of Science} 46: 589--601. doi:
\href{https://doi.org/10.1093/bjps/46.4.589}{10.1093/bjps/46.4.589}.

\bibitem[\citeproctext]{ref-Sosa2007}
Sosa, Ernest. 2007. \emph{A Virtue Epistemology: Apt Belief and
Reflective Knowledge}. Oxford: Oxford University Press.

\bibitem[\citeproctext]{ref-Turri2011}
Turri, John. 2011. {``Manifest Failure: The Gettier Problem Solved.''}
\emph{Philosophers' Imprint} 11 (8): 1--11.
\url{http://hdl.handle.net/2027/spo.3521354.0011.008}.

\bibitem[\citeproctext]{ref-Weatherson2004-WEALMT}
Weatherson, Brian. 2004. {``Luminous Margins.''} \emph{Australasian
Journal of Philosophy} 82 (3): 373--83. doi:
\href{https://doi.org/10.1080/713659874}{10.1080/713659874}.

\bibitem[\citeproctext]{ref-Weatherson2005-WEACWD}
---------. 2005. {``{Can We Do Without Pragmatic Encroachment?}''}
\emph{Philosophical Perspectives} 19 (1): 417--43. doi:
\href{https://doi.org/10.1111/j.1520-8583.2005.00068.x}{10.1111/j.1520-8583.2005.00068.x}.

\bibitem[\citeproctext]{ref-Weatherson2011-WEAKBI}
---------. 2011. {``Knowledge, Bets and Interests.''} In
\emph{Forthcoming Volume on Knowledge Ascriptions}, edited by Jessica
Brown and Mikkel Gerken, 75--103. Oxford: Oxford University Press.

\bibitem[\citeproctext]{ref-Williamson2000-WILKAI}
Williamson, Timothy. 2000. \emph{{Knowledge and its Limits}}. Oxford
University Press.

\bibitem[\citeproctext]{ref-WilliamsonLofoten}
---------. 2013. {``Gettier Cases in Epistemic Logic.''} \emph{Inquiry}
56 (1): 1--14. doi:
\href{https://doi.org/10.1080/0020174X.2013.775010}{10.1080/0020174X.2013.775010}.

\bibitem[\citeproctext]{ref-Zagzebski1994}
Zagzebski, Linda. 1994. {``The Inescapability of Gettier Problems.''}
\emph{The Philosophical Quarterly} 44 (174): 65--73. doi:
\href{https://doi.org/10.2307/2220147}{10.2307/2220147}.

\bibitem[\citeproctext]{ref-Zagzebski1996}
---------. 1996. \emph{Virtues of the Mind: An Inquiry into the Nature
of Virtue and the Ethical Foundations of Knowledge}. Cambridge:
Cambridge University Press.

\bibitem[\citeproctext]{ref-Zagzebski2009}
---------. 2009. \emph{On Epistemology}. Belmont, CA.: Wadsworth.

\end{CSLReferences}



\noindent Published in\emph{
Inquiry}, 2013, pp. 63-76.


\end{document}
