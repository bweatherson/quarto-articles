% Options for packages loaded elsewhere
\PassOptionsToPackage{unicode}{hyperref}
\PassOptionsToPackage{hyphens}{url}
%
\documentclass[
  11pt,
  letterpaper,
  DIV=11,
  numbers=noendperiod,
  twoside]{scrartcl}

\usepackage{amsmath,amssymb}
\usepackage{setspace}
\usepackage{iftex}
\ifPDFTeX
  \usepackage[T1]{fontenc}
  \usepackage[utf8]{inputenc}
  \usepackage{textcomp} % provide euro and other symbols
\else % if luatex or xetex
  \usepackage{unicode-math}
  \defaultfontfeatures{Scale=MatchLowercase}
  \defaultfontfeatures[\rmfamily]{Ligatures=TeX,Scale=1}
\fi
\usepackage{lmodern}
\ifPDFTeX\else  
    % xetex/luatex font selection
    \setmainfont[ItalicFont=EB Garamond Italic,BoldFont=EB Garamond
Bold]{EB Garamond Math}
    \setsansfont[]{EB Garamond}
  \setmathfont[]{Garamond-Math}
\fi
% Use upquote if available, for straight quotes in verbatim environments
\IfFileExists{upquote.sty}{\usepackage{upquote}}{}
\IfFileExists{microtype.sty}{% use microtype if available
  \usepackage[]{microtype}
  \UseMicrotypeSet[protrusion]{basicmath} % disable protrusion for tt fonts
}{}
\usepackage{xcolor}
\usepackage[left=1.1in, right=1in, top=0.8in, bottom=0.8in,
paperheight=9.5in, paperwidth=7in, includemp=TRUE, marginparwidth=0in,
marginparsep=0in]{geometry}
\setlength{\emergencystretch}{3em} % prevent overfull lines
\setcounter{secnumdepth}{3}
% Make \paragraph and \subparagraph free-standing
\makeatletter
\ifx\paragraph\undefined\else
  \let\oldparagraph\paragraph
  \renewcommand{\paragraph}{
    \@ifstar
      \xxxParagraphStar
      \xxxParagraphNoStar
  }
  \newcommand{\xxxParagraphStar}[1]{\oldparagraph*{#1}\mbox{}}
  \newcommand{\xxxParagraphNoStar}[1]{\oldparagraph{#1}\mbox{}}
\fi
\ifx\subparagraph\undefined\else
  \let\oldsubparagraph\subparagraph
  \renewcommand{\subparagraph}{
    \@ifstar
      \xxxSubParagraphStar
      \xxxSubParagraphNoStar
  }
  \newcommand{\xxxSubParagraphStar}[1]{\oldsubparagraph*{#1}\mbox{}}
  \newcommand{\xxxSubParagraphNoStar}[1]{\oldsubparagraph{#1}\mbox{}}
\fi
\makeatother


\providecommand{\tightlist}{%
  \setlength{\itemsep}{0pt}\setlength{\parskip}{0pt}}\usepackage{longtable,booktabs,array}
\usepackage{calc} % for calculating minipage widths
% Correct order of tables after \paragraph or \subparagraph
\usepackage{etoolbox}
\makeatletter
\patchcmd\longtable{\par}{\if@noskipsec\mbox{}\fi\par}{}{}
\makeatother
% Allow footnotes in longtable head/foot
\IfFileExists{footnotehyper.sty}{\usepackage{footnotehyper}}{\usepackage{footnote}}
\makesavenoteenv{longtable}
\usepackage{graphicx}
\makeatletter
\newsavebox\pandoc@box
\newcommand*\pandocbounded[1]{% scales image to fit in text height/width
  \sbox\pandoc@box{#1}%
  \Gscale@div\@tempa{\textheight}{\dimexpr\ht\pandoc@box+\dp\pandoc@box\relax}%
  \Gscale@div\@tempb{\linewidth}{\wd\pandoc@box}%
  \ifdim\@tempb\p@<\@tempa\p@\let\@tempa\@tempb\fi% select the smaller of both
  \ifdim\@tempa\p@<\p@\scalebox{\@tempa}{\usebox\pandoc@box}%
  \else\usebox{\pandoc@box}%
  \fi%
}
% Set default figure placement to htbp
\def\fps@figure{htbp}
\makeatother
% definitions for citeproc citations
\NewDocumentCommand\citeproctext{}{}
\NewDocumentCommand\citeproc{mm}{%
  \begingroup\def\citeproctext{#2}\cite{#1}\endgroup}
\makeatletter
 % allow citations to break across lines
 \let\@cite@ofmt\@firstofone
 % avoid brackets around text for \cite:
 \def\@biblabel#1{}
 \def\@cite#1#2{{#1\if@tempswa , #2\fi}}
\makeatother
\newlength{\cslhangindent}
\setlength{\cslhangindent}{1.5em}
\newlength{\csllabelwidth}
\setlength{\csllabelwidth}{3em}
\newenvironment{CSLReferences}[2] % #1 hanging-indent, #2 entry-spacing
 {\begin{list}{}{%
  \setlength{\itemindent}{0pt}
  \setlength{\leftmargin}{0pt}
  \setlength{\parsep}{0pt}
  % turn on hanging indent if param 1 is 1
  \ifodd #1
   \setlength{\leftmargin}{\cslhangindent}
   \setlength{\itemindent}{-1\cslhangindent}
  \fi
  % set entry spacing
  \setlength{\itemsep}{#2\baselineskip}}}
 {\end{list}}
\usepackage{calc}
\newcommand{\CSLBlock}[1]{\hfill\break\parbox[t]{\linewidth}{\strut\ignorespaces#1\strut}}
\newcommand{\CSLLeftMargin}[1]{\parbox[t]{\csllabelwidth}{\strut#1\strut}}
\newcommand{\CSLRightInline}[1]{\parbox[t]{\linewidth - \csllabelwidth}{\strut#1\strut}}
\newcommand{\CSLIndent}[1]{\hspace{\cslhangindent}#1}

\setlength\heavyrulewidth{0ex}
\setlength\lightrulewidth{0ex}
\usepackage[automark]{scrlayer-scrpage}
\clearpairofpagestyles
\cehead{
  Brian Weatherson
  }
\cohead{
  Prankster’s Ethics
  }
\ohead{\bfseries \pagemark}
\cfoot{}
\makeatletter
\newcommand*\NoIndentAfterEnv[1]{%
  \AfterEndEnvironment{#1}{\par\@afterindentfalse\@afterheading}}
\makeatother
\NoIndentAfterEnv{itemize}
\NoIndentAfterEnv{enumerate}
\NoIndentAfterEnv{description}
\NoIndentAfterEnv{quote}
\NoIndentAfterEnv{equation}
\NoIndentAfterEnv{longtable}
\NoIndentAfterEnv{abstract}
\renewenvironment{abstract}
 {\vspace{-1.25cm}
 \quotation\small\noindent\emph{Abstract}:}
 {\endquotation}
\newfontfamily\tfont{EB Garamond}
\addtokomafont{disposition}{\rmfamily}
\addtokomafont{title}{\normalfont\itshape}
\let\footnoterule\relax
\cehead{
       Andy Egan and Brian Weatherson
        }
\KOMAoption{captions}{tableheading}
\makeatletter
\@ifpackageloaded{caption}{}{\usepackage{caption}}
\AtBeginDocument{%
\ifdefined\contentsname
  \renewcommand*\contentsname{Table of contents}
\else
  \newcommand\contentsname{Table of contents}
\fi
\ifdefined\listfigurename
  \renewcommand*\listfigurename{List of Figures}
\else
  \newcommand\listfigurename{List of Figures}
\fi
\ifdefined\listtablename
  \renewcommand*\listtablename{List of Tables}
\else
  \newcommand\listtablename{List of Tables}
\fi
\ifdefined\figurename
  \renewcommand*\figurename{Figure}
\else
  \newcommand\figurename{Figure}
\fi
\ifdefined\tablename
  \renewcommand*\tablename{Table}
\else
  \newcommand\tablename{Table}
\fi
}
\@ifpackageloaded{float}{}{\usepackage{float}}
\floatstyle{ruled}
\@ifundefined{c@chapter}{\newfloat{codelisting}{h}{lop}}{\newfloat{codelisting}{h}{lop}[chapter]}
\floatname{codelisting}{Listing}
\newcommand*\listoflistings{\listof{codelisting}{List of Listings}}
\makeatother
\makeatletter
\makeatother
\makeatletter
\@ifpackageloaded{caption}{}{\usepackage{caption}}
\@ifpackageloaded{subcaption}{}{\usepackage{subcaption}}
\makeatother

\usepackage{bookmark}

\IfFileExists{xurl.sty}{\usepackage{xurl}}{} % add URL line breaks if available
\urlstyle{same} % disable monospaced font for URLs
\hypersetup{
  pdftitle={Prankster's Ethics},
  pdfauthor={Andy Egan; Brian Weatherson},
  hidelinks,
  pdfcreator={LaTeX via pandoc}}


\title{Prankster's Ethics}
\author{Andy Egan \and Brian Weatherson}
\date{2004}

\begin{document}
\maketitle
\begin{abstract}
We raise an objection to a very weak form of consequentialism. A world
with only moral saints would be improved by adding a few mostly harmless
pranksters. This result is not dependent on a particular way of thinking
about value; it is resilient across a lot of measures of the value of
worlds. But these pranksters would be doing things that are morally
wrong. So we cannot identify rightness with making the world a better
place.
\end{abstract}


\setstretch{1.1}
\section{A Quick Argument for
Boorishness}\label{a-quick-argument-for-boorishness}

Diversity is a good thing. Some of its value is instrumental. Having
people around with diverse beliefs, or customs, or tastes, can expand
our horizons and potentially raise to salience some potential true
beliefs, useful customs or apt tastes. Even diversity of error can be
useful. Seeing other people fall away from the true and the useful in
distinctive ways can immunise us against similar errors. And there are a
variety of pleasant interactions, not least philosophical exchange, that
wouldn't be possible unless some kinds of diversity existed. Diversity
may also have intrinsic value. It may be that a society with diverse
views, customs and tastes is simply thereby a better society. But we
will mostly focus on diversity's instrumental value here.

We think that what is true of these common types of diversity is also
true of moral diversity. By moral diversity we mean not only diversity
of moral views, though that is no doubt valuable, but diversity of moral
behaviour. In a morally diverse society, at least some people will not
conform as tightly to moral norms as others. In short, there will be
some wrongdoers. To be sure, moral diversity has some costs, and too
much of it is undoubtedly a bad thing. Having rapists and murderers adds
to moral diversity (assuming, as we do, that most people are basically
moral) but not in a way that is particularly valuable. Still, smaller
amounts of moral diversity may be valuable, all things considered. It
seems particularly clear that moral diversity within a subgroup has
value, but sometimes society as a whole is better off for being morally
diverse. Let us consider some examples.

Many violations of etiquette are not moral transgressions. Eating
asparagus spears with one's fork is not sinful, just poor form. But more
extreme violations may be sinful. Hurtful use of racial epithets, for
example, is clearly immoral as well as a breach of etiquette. Even use
of language that causes not hurt, but strong discomfort, may be morally
wrong. Someone who uses an offensive term in polite company, say at a
dinner party or in a professional philosophical forum, may be doing the
wrong thing. But having the wrongdoer around may have valuable
consequences. For example, they generate stories that can be told, to
great amusement, at subsequent dinner parties. They also prompt us to
reconsider the basis for the standards we ourselves adopt in such
matters. The reconsideration may cause us to abandon useless practices,
and it may reinforce useful practices. These benefits seem to outweigh
the disutility of the discomfort felt by those in attendance when the
fateful word drops from the speaker's lips. These side benefits do not
make the original action morally permissible. Indeed, it is precisely
because the action is not morally permissible that the benefits accrue.

While we think that case is one of valuable moral diversity, some may
question the \emph{immorality} of the act in question. So let us try a
more clearly immoral case: the mostly harmless prankster. Sam is a
pie-thrower. Sam doesn't just throw pies at the rich and infamous. No,
Sam's pies land on common folk like you and I, often for no reason
beyond Sam's amusement. Causing gratuitous harm for one's own amusement
is immoral. And a pie in the face, while better than a poke in the eye
with a burnt stick, is harmful. But it may, in some circumstances, have
side benefits. There will be the (guilty) pleasure occasioned in the
unharmed bystanders, though it would be wrong to put too much weight on
that. Other more significant benefits may accrue if Sam's society is
otherwise saintly. Sam's existence will prompt people to take some
simple, and worthwhile, precautions against perpetrators of such
attacks. Even if society currently contains no malfeasants, such
precautions will be useful against future wrongdoers. This benefit will
increase if Sam graduates from pie-throwing to more varied pranks. (As
may the entertainment value of Sam's pranks.) Many computer hackers
perform just this function in the present world. Malicious hackers on
the whole cause more harm than good. But other hackers, who hack without
gratuitously harming, provide a protective benefit by informing us of
our weaknesses. These are the pie-throwers of the virtual world. Sam's
actions have other benefits. If Sam's pranks are harmless enough, some
will mistakenly think that they are morally acceptable, and we can have
enjoyable, valuable, philosophical discussions with them. (Note that
this benefit also increases if Sam varies the pranks.) The upshot is
that Sam's pranks can make the world a better place, all things
considered, despite being immoral. Indeed, in some ways they make the
world a better place \emph{because} they are immoral.

The philosophical point, or points, here may be familiar. One point
certainly \emph{is} familiar: we have here an example of a Moorean
organic unity. The goodness of the whole is no simple function of the
goodness of the parts. It might be thought that this follows simply from
the familiar counterexamples to utilitarianism, and that our examples
have no more philosophical interest than those old counterexamples. Both
of these thoughts would be mistaken.

The familiar counterexamples we have in mind include, for example, the
case of the doctor who kills a healthy patient to harvest her organs, or
the judge who executes an innocent man to prevent a riot. Importantly,
those examples do not refute consequentialism in general, but only a
version of consequentialism that adopts a particular kind of reductive
analysis of the good. The details of the analysis won't matter here, but
it may be an analysis of goodness in terms on happiness, or preference
satisfaction. If we give up the reductive analysis of goodness, we can
say that the doctor and the judge do not make for a better society. A
familiar heuristic supports that claim. (We take no stand here on
whether this heuristic can be turned into an analysis.) Behind the
Rawlsian veil of ignorance, we would prefer that there not be such
doctors or judges in society. We think that most of us would agree, even
in full appreciation of the possibility that we will be saved by the
doctor, or possibly the judge. On the other hand, we think we'd prefer a
society with the occasional boorish dinner guest, or a rare pie-thrower,
to a society of moral saints. We say this in full appreciation of the
possibility that we may get a pie in the face for our troubles. Possibly
if we \emph{knew} we would be the pie-throwee we would change our minds,
but fortunately pies cannot penetrate the veil of ignorance.

Although it isn't much discussed in the literature, we think this form
of consequentialism is interesting for several reasons beyond its
capacity to avoid counterexamples. For one thing, it is not easy to say
whether this counts as an agent-neutral ethical theory. On the one hand,
we can say what everyone should do in neutral terms: for each person it
is better if they do things that create a better world from the
perspective of those behind the veil of ignorance. On the other hand
this rule leads to obligations on agents that do not seem at all
neutral. From behind the veil of ignorance we'd prefer that parents love
their children and hence privilege their interests, and that they love
them because they are their children not because this creates a better
world, so parents end up with a special obligation to their children.
Having this much (or more importantly this \emph{little}) neutrality in
a moral theory sounds quite plausible to us, and although we won't
develop the point here there is possibly an attractive answer to the
`nearest and dearest' objection to consequentialism
(\citeproc{ref-Jackson1991}{Jackson 1991}). More generally, because we
have preferences from behind the veil of ignorance about \emph{why}
people act and not just about \emph{how} they act -- we prefer for
instance that people visit sick friends in hospital because they are
friends not because of an abstract sense of duty -- this form of
consequentialism is not particularly vulnerable to objections that claim
consequentialists pay too little attention to motives.

So we think a consequentialist can avoid the standard objections to
utilitarianism by being less ambitious and not trying to provide a
reductive analysis of goodness. The most natural retreat is to behind
the veil of ignorance, but our examples can reach even there. This is
far from the only interesting consequence of the examples.

\section{The Good, the Right, and the
Saintly}\label{the-good-the-right-and-the-saintly}

We think that the cases of the curser and the pie-thrower are examples
of situations in which (a) an agent ought not to \(\varphi\), and (b)
it's best that the agent \emph{does} \(\varphi\). Our judgements about
the cases are not based on any theoretical analysis of the right and the
good. They're simply intuitions about cases---it just seems to us that
the right thing to say about the pie thrower is that she ought not to do
what she does, but that it's still best if she does it. To the extent
that these intuitions are puzzling or theoretically problematic (and we
think that they are at least a little bit puzzling, and at least
potentially problematic), it's open to us to reject one or the other
intuition about the cases, and either deny that the curser and the pie
thrower ought not to curse or throw pies, or deny that it's best that
they \emph{do} curse and throw pies. This is an option, but we think
it's not a very attractive one. Suppose that instead we take the
intuitions at face value, and accept our judgements about the cases.
What follows?

Our analysis of the examples is incompatible with two attractive views
about the connection between \emph{goodness} (that is, the property of
things---in particular worlds---in virtue of which some of them stand in
the \emph{better than} relation to others) and \emph{rightness}, and
between goodness and good character:\footnote{Proposition (2) is quite a
  natural position to hold if one is trying to capture the insights of
  virtue ethics in a consequentialist framework, as in Driver
  (\citeproc{ref-Driver2001}{2001}) or Hurka
  (\citeproc{ref-Hurka2001}{2001}). But if we take `better' in a more
  neutral way, so (2) does not mean that there are better consequences
  if everyone has good character, but simply that the world is a better
  place if this is so, even if this has few consequences, or even
  negative consequences, then it will be a position common to most
  virtue ethicists.}

\begin{description}
\tightlist
\item[(1)]
It's better if everyone does what's right.
\item[(2)]
It's better if everyone has good character.
\end{description}

Now, neither of these will do as a philosophical thesis. But it's
probably not worth spending the time and effort on patching them up,
since even the patched-up versions will be false.

If the pie-thrower ought not to throw her pies, but it's nonetheless
best that she does, no patched-up version of (1) that captures the
intuition behind it can be right. Any patched-up version of (1) will
still be claiming that there's a very tight connection between what it
would be \emph{right} for us to do (what we \emph{ought} to do) and what
it would be \emph{best} for us to do. Any plausible elaboration on (1)
will include a commitment to the thesis that, if we ought not to do
something, then it's best if we don't do it. But if our analyses of the
cases of the curser and the pie-thrower are right, then these are
counterexamples.

What about (2)? Well, it's \emph{not} better if the cursing dinner guest
has good character. What happens if we suppose that the curser
\emph{does} have good character? One of two things: (i) He'll no longer
curse at dinner parties, and we'll lose the benefits that come from his
cursing. This would be bad. (ii) He'll still curse at dinner parties,
but he'll be cursing in a \emph{studied} way. He'll be cursing because
he's seen that things will be better if \emph{somebody} uses foul
language in inappropriate circumstances, and he's taken it upon himself
to fill the unfilled functional role. This would also be bad. This sort
of \emph{studied} bad conduct doesn't have the same value as bad conduct
that springs from bad character. Here is some evidence for this: We
value the curser's breaching of societal norms, even though he ought not
to do it. Were we to \emph{find out} that every expletive had been
\emph{studied}, produced either to produce these important social goods,
or to create a familiar bad-boy image, we would \emph{stop} valuing his
breachings of the moral order. They would, instead, become merely
tiresome and annoying. Since we value spontaneous cursings which are
products of less-than-optimal character, but we do \emph{not} value
studied cursings which are products of exemplary character, it's very
plausible to conclude (though admittedly not quite mandatory) that the
spontaneous curses are much more valuable than the studied ones. We're
inclined to say, in fact, that while having a few \emph{spontaneous}
cursers around makes things better, having \emph{studied} cursers around
makes things worse. Since you have to have less-than-perfect character
in order to be a spontaneous curser, it follows that you can't get the
benefits of having cursers around without having some people with
less-than-perfect character around. And since it's better to have the
cursers than not, it's better to have some people with less-than-perfect
character around than not. This will be incompatible with almost any
plausible way of cashing out (2).\footnote{Specifically, it will be
  incompatible with any \emph{maximizing} version of (2). There are
  `threshold' versions of (2) that don't fall afoul of this kind of
  problem because they don't claim it would be best for everyone to have
  \emph{perfect} character, but only that it would be best for everyone
  to have \emph{pretty good} character, or at least for nobody to have
  \emph{really bad} character.}

\section{A Problem about Quantifier
Scope?}\label{a-problem-about-quantifier-scope}

But isn't there a sense in which (for example) the pie-thrower
\emph{ought} to throw his pies? After all, if nobody was throwing pies,
we might think to ourselves, ``gosh, it would be better if there were a
few---not many, but a few---pie throwers around''. Then it would be
natural to conclude, ``somebody ought to start throwing pies at
strangers''. And then it would be natural to infer that at least the
first person to start throwing pies at strangers would be doing what
they ought. It would be natural, but it would be wrong. The plausible
reading of ``someone ought to start throwing pies at strangers'' is,
``it ought to be that somebody starts throwing pies at strangers'', not,
``there's somebody out there such that they ought to start throwing pies
at strangers''. So we haven't gotten anybody a moral license to throw
pies yet. And in fact it's very plausible that we ought to understand
assertions that \emph{it ought to be that P} as claiming that it would
be \emph{better} if it were the case that P; that is, as making claims
about what would be \emph{good}, not about what would be \emph{right}.

There's a puzzle about what to make of cases where we're inclined to say
that it ought to be that somebody \(\varphi\)s---that is, that somebody
ought to \(\varphi\); but also that there's nobody such that \emph{they}
ought to \(\varphi\)---in fact, that everybody is such that they ought
\emph{not} to \(\varphi\).\footnote{It's actually the second part that
  makes it puzzling. Compare the familiar and unproblematic situation in
  which we ought to give you a horse, but there's no horse such that we
  ought to give you \emph{that} one, and the more troubling situation in
  which we ought to give you a horse, but every horse is such that we
  ought \emph{not} to give you that one.} Maybe the fact that our
intuitions about the examples give rise to these kinds of puzzling cases
is evidence that one or the other of our intuitions ought to be
rejected. The move we suggested above is that the reason this seems so
puzzling is that we've been punning on ``ought''. The ``ought'' in
``somebody ought to start throwing pies'' doesn't have anything much to
do with what moral obligations anybody has---doesn't have anything much
to do with what's \emph{right}---but has a great deal to do with what's
\emph{good}. And if that's the case, then all we have is more evidence
against the tight connection between the right and the good: it would be
better if somebody started throwing pies, but everybody has a moral
obligation not to. So it would be better if somebody did what they
oughtn't.

\section{Value, Desire and Advice}\label{value-desire-and-advice}

Although the ``ought'' in ``somebody ought to throw pies'' has little to
do with what's \emph{right}, it might have a lot to do with what we find
desirable. And this will cause problems for some familiar meta-ethical
theories. Quite naturally, Jack does not desire to throw pies at
strangers for amusement in the actual world. Jack's a very civic minded
fellow in that respect. In fact, his concern for others goes deeper than
that. He'd be quite prepared to risk his body for the sake of his fellow
citizens. As it turns out, he's been a volunteer fire fighter for years
now. And Jack likes to think that if need be, he would be prepared, to
use an old fashioned phrase, to risk his soul for the community. He
hopes he would be morally depraved if what the society needed was
depravity. Jack agrees with the discussion of character in section 2, so
he hopes that when society needs a pie-thrower, he will step up with the
plate, and do so directly because he wants to throw pies at innocent
bystanders. Letting C stand for the circumstances described above, where
it would be good for there to be more wrongdoing, Jack's position can be
summarised by saying that he desires that in C he desires that \emph{he}
throws pies at innocents.

Does this all mean Jack values his throwing pies at innocents in C? Not
necessarily. Does it mean that if we were all like Jack, and we are
subjectivists about what is right, it would be right to throw pies at
innocents in C? Definitely not. David Lewis
(\citeproc{ref-Lewis1989b}{1989}) equates what we value with what we
desire to desire.\footnote{More precisely, with what we desire to desire
  in circumstances of appropriate imaginative acquaintance. We can
  suppose that Jack, and everyone else under discussion in this
  paragraph, is suitably imaginatively acquainted with the salient
  situations. Jack knows full well what it is like to get a pie in the
  face.} And he equates what is valuable with what we value. The text is
not transparent, but it seems Lewis wants \emph{valuable} to subsume
both what we call the `right' and the `good'. And this he cannot have.
Assume that everyone in Jack's community desires to (\emph{de se})
desire that (s)he throw pies at innocents in C. That does not make it
right that pies are thrown at innocents. We take no stand here on
whether the flaw is in the equation of personal value with second-order
desire, or in the reduction of both rightness and goodness to personal
value, but there is a problem for Lewis's dispositional theory of
value.\footnote{Someone might think it obvious that Lewisian value can't
  be used in an analysis of both rightness and goodness, since it is
  \emph{one} concept and we are analysing \emph{two} concepts. But
  Lewisian value bifurcates in a way that one might think it is suitable
  for analysing both rightness and goodness. Since there are both
  \emph{de dicto} and \emph{de se} desires, one can easily draw out both
  \emph{de dicto} and \emph{de se} values. And it is \emph{prima facie}
  plausible that the \emph{de dicto} values correspond to what is good,
  and the \emph{de se} values to what is right. Indeed, given a weak
  version of consequentialism where these two can be guaranteed to not
  directly conflict, this correspondence may well hold. But we think the
  pie-thrower threatens even those consequentialists. The net
  philosophical conclusion is that the pie-thrower is a problem for
  Lewis's meta-ethics, but only because (a) she is a problem for Lewis's
  consequentialism, and, surprisingly, (b) Lewis's meta-ethics depends
  on his consequentialism being at least roughly right.}

This point generalises to cause difficulties for several dispositional
theories of value. For example, Michael Smith
(\citeproc{ref-Smith1994}{1994}) holds that right actions are what our
perfectly rational selves would advise us to do. This assumes that when
the good and the right come apart, our perfectly rational selves would
choose the right over the good. And it's far from clear that Smith has
the resources to argue for this assumption. Smith's argument that our
perfectly rational selves will advise us to do what is right relies on
his earlier argument that anyone who does not do what she judges to be
\emph{right} is practically irrational, unlike presumably our perfectly
rational selves. And the main argument for that principle is that it is
the best explanation of why actually good people are motivated to do
what they judge to be right, even when they change their judgements
about what is right. But now we should be able to see that there's an
alternative explanation available. Actually good people might be
motivated to do what they judge to be \emph{good} rather than
\emph{right}. We have seen no reason to believe that the right and the
good actually come \emph{radically} apart, so this is just as good an
explanation of the behaviour actual moral agents as Smith's explanation.
So for all Smith has argued, one might judge \(\varphi\)ing to be right,
also judge it not to be good, hence be not motivated to \(\varphi\), and
not be practically irrational. Indeed, our perfectly rational self might
be just like this.\footnote{We have glossed over a technical point here
  that is irrelevant to the current discussion. What matters is not
  whether our perfectly rational selves are motivated to \(\varphi\), it
  matters whether they desire that we \(\varphi\), and hence whether
  they are motivated to advise us to \(\varphi\). Keeping this point
  clear matters for all sorts of purposes, but not we think the present
  one.} Hence we cannot rely on our perfectly rational self to be a
barometer of what is right, as opposed to what is good.

\subsection*{References}\label{references}
\addcontentsline{toc}{subsection}{References}

\phantomsection\label{refs}
\begin{CSLReferences}{1}{0}
\bibitem[\citeproctext]{ref-Driver2001}
Driver, Julia. 2001. \emph{Uneasy Virtues}. Cambridge: Cambridge
University Press.

\bibitem[\citeproctext]{ref-Hurka2001}
Hurka, Thomas. 2001. {``Vices as Higher-Level Evils.''} \emph{Utilitas}
13 (2): 195--212. doi:
\href{https://doi.org/10.1017/s0953820800003137}{10.1017/s0953820800003137}.

\bibitem[\citeproctext]{ref-Jackson1991}
Jackson, Frank. 1991. {``Decision Theoretic Consequentialism and the
Nearest and Dearest Objection.''} \emph{Ethics} 101 (3): 461--82. doi:
\href{https://doi.org/10.1086/293312}{10.1086/293312}.

\bibitem[\citeproctext]{ref-Lewis1989b}
Lewis, David. 1989. {``Dispositional Theories of Value.''}
\emph{Aristotelian Society Supplementary Volume} 63 (1): 113--37. doi:
\href{https://doi.org/10.1093/aristoteliansupp/63.1.89}{10.1093/aristoteliansupp/63.1.89}.
Reprinted in his \emph{Papers in Ethics and Social Philosophy},
Cambridge: Cambridge University Press, 2000, 68-94. References to
reprint.

\bibitem[\citeproctext]{ref-Smith1994}
Smith, Michael. 1994. \emph{The Moral Problem}. Oxford: Blackwell.

\end{CSLReferences}



\noindent Published in\emph{
Philosophical Perspectives}, 2004, pp. 45-52.


\end{document}
