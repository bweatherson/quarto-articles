% Options for packages loaded elsewhere
\PassOptionsToPackage{unicode}{hyperref}
\PassOptionsToPackage{hyphens}{url}
%
\documentclass[
  10pt,
  letterpaper,
  DIV=11,
  numbers=noendperiod,
  twoside]{scrartcl}

\usepackage{amsmath,amssymb}
\usepackage{setspace}
\usepackage{iftex}
\ifPDFTeX
  \usepackage[T1]{fontenc}
  \usepackage[utf8]{inputenc}
  \usepackage{textcomp} % provide euro and other symbols
\else % if luatex or xetex
  \usepackage{unicode-math}
  \defaultfontfeatures{Scale=MatchLowercase}
  \defaultfontfeatures[\rmfamily]{Ligatures=TeX,Scale=1}
\fi
\usepackage{lmodern}
\ifPDFTeX\else  
    % xetex/luatex font selection
  \setmainfont[ItalicFont=EB Garamond Italic,BoldFont=EB Garamond
Bold]{EB Garamond Math}
  \setsansfont[]{Europa-Bold}
  \setmathfont[]{Garamond-Math}
\fi
% Use upquote if available, for straight quotes in verbatim environments
\IfFileExists{upquote.sty}{\usepackage{upquote}}{}
\IfFileExists{microtype.sty}{% use microtype if available
  \usepackage[]{microtype}
  \UseMicrotypeSet[protrusion]{basicmath} % disable protrusion for tt fonts
}{}
\usepackage{xcolor}
\usepackage[left=1in, right=1in, top=0.8in, bottom=0.8in,
paperheight=9.5in, paperwidth=6.5in, includemp=TRUE, marginparwidth=0in,
marginparsep=0in]{geometry}
\setlength{\emergencystretch}{3em} % prevent overfull lines
\setcounter{secnumdepth}{3}
% Make \paragraph and \subparagraph free-standing
\ifx\paragraph\undefined\else
  \let\oldparagraph\paragraph
  \renewcommand{\paragraph}[1]{\oldparagraph{#1}\mbox{}}
\fi
\ifx\subparagraph\undefined\else
  \let\oldsubparagraph\subparagraph
  \renewcommand{\subparagraph}[1]{\oldsubparagraph{#1}\mbox{}}
\fi


\providecommand{\tightlist}{%
  \setlength{\itemsep}{0pt}\setlength{\parskip}{0pt}}\usepackage{longtable,booktabs,array}
\usepackage{calc} % for calculating minipage widths
% Correct order of tables after \paragraph or \subparagraph
\usepackage{etoolbox}
\makeatletter
\patchcmd\longtable{\par}{\if@noskipsec\mbox{}\fi\par}{}{}
\makeatother
% Allow footnotes in longtable head/foot
\IfFileExists{footnotehyper.sty}{\usepackage{footnotehyper}}{\usepackage{footnote}}
\makesavenoteenv{longtable}
\usepackage{graphicx}
\makeatletter
\def\maxwidth{\ifdim\Gin@nat@width>\linewidth\linewidth\else\Gin@nat@width\fi}
\def\maxheight{\ifdim\Gin@nat@height>\textheight\textheight\else\Gin@nat@height\fi}
\makeatother
% Scale images if necessary, so that they will not overflow the page
% margins by default, and it is still possible to overwrite the defaults
% using explicit options in \includegraphics[width, height, ...]{}
\setkeys{Gin}{width=\maxwidth,height=\maxheight,keepaspectratio}
% Set default figure placement to htbp
\makeatletter
\def\fps@figure{htbp}
\makeatother
% definitions for citeproc citations
\NewDocumentCommand\citeproctext{}{}
\NewDocumentCommand\citeproc{mm}{%
  \begingroup\def\citeproctext{#2}\cite{#1}\endgroup}
\makeatletter
 % allow citations to break across lines
 \let\@cite@ofmt\@firstofone
 % avoid brackets around text for \cite:
 \def\@biblabel#1{}
 \def\@cite#1#2{{#1\if@tempswa , #2\fi}}
\makeatother
\newlength{\cslhangindent}
\setlength{\cslhangindent}{1.5em}
\newlength{\csllabelwidth}
\setlength{\csllabelwidth}{3em}
\newenvironment{CSLReferences}[2] % #1 hanging-indent, #2 entry-spacing
 {\begin{list}{}{%
  \setlength{\itemindent}{0pt}
  \setlength{\leftmargin}{0pt}
  \setlength{\parsep}{0pt}
  % turn on hanging indent if param 1 is 1
  \ifodd #1
   \setlength{\leftmargin}{\cslhangindent}
   \setlength{\itemindent}{-1\cslhangindent}
  \fi
  % set entry spacing
  \setlength{\itemsep}{#2\baselineskip}}}
 {\end{list}}
\usepackage{calc}
\newcommand{\CSLBlock}[1]{\hfill\break\parbox[t]{\linewidth}{\strut\ignorespaces#1\strut}}
\newcommand{\CSLLeftMargin}[1]{\parbox[t]{\csllabelwidth}{\strut#1\strut}}
\newcommand{\CSLRightInline}[1]{\parbox[t]{\linewidth - \csllabelwidth}{\strut#1\strut}}
\newcommand{\CSLIndent}[1]{\hspace{\cslhangindent}#1}

\setlength\heavyrulewidth{0ex}
\setlength\lightrulewidth{0ex}
\usepackage[automark]{scrlayer-scrpage}
\clearpairofpagestyles
\cehead{
  Brian Weatherson
  }
\cohead{
  Games, Beliefs and Credences
  }
\ohead{\bfseries \pagemark}
\cfoot{}
\makeatletter
\newcommand*\NoIndentAfterEnv[1]{%
  \AfterEndEnvironment{#1}{\par\@afterindentfalse\@afterheading}}
\makeatother
\NoIndentAfterEnv{itemize}
\NoIndentAfterEnv{enumerate}
\NoIndentAfterEnv{description}
\NoIndentAfterEnv{quote}
\NoIndentAfterEnv{equation}
\NoIndentAfterEnv{longtable}
\NoIndentAfterEnv{abstract}
\renewenvironment{abstract}
 {\vspace{-1.25cm}
 \quotation\small\noindent\rule{\linewidth}{.5pt}\par\smallskip
 \noindent }
 {\par\noindent\rule{\linewidth}{.5pt}\endquotation}
\KOMAoption{captions}{tableheading}
\makeatletter
\@ifpackageloaded{caption}{}{\usepackage{caption}}
\AtBeginDocument{%
\ifdefined\contentsname
  \renewcommand*\contentsname{Table of contents}
\else
  \newcommand\contentsname{Table of contents}
\fi
\ifdefined\listfigurename
  \renewcommand*\listfigurename{List of Figures}
\else
  \newcommand\listfigurename{List of Figures}
\fi
\ifdefined\listtablename
  \renewcommand*\listtablename{List of Tables}
\else
  \newcommand\listtablename{List of Tables}
\fi
\ifdefined\figurename
  \renewcommand*\figurename{Figure}
\else
  \newcommand\figurename{Figure}
\fi
\ifdefined\tablename
  \renewcommand*\tablename{Table}
\else
  \newcommand\tablename{Table}
\fi
}
\@ifpackageloaded{float}{}{\usepackage{float}}
\floatstyle{ruled}
\@ifundefined{c@chapter}{\newfloat{codelisting}{h}{lop}}{\newfloat{codelisting}{h}{lop}[chapter]}
\floatname{codelisting}{Listing}
\newcommand*\listoflistings{\listof{codelisting}{List of Listings}}
\makeatother
\makeatletter
\makeatother
\makeatletter
\@ifpackageloaded{caption}{}{\usepackage{caption}}
\@ifpackageloaded{subcaption}{}{\usepackage{subcaption}}
\makeatother
\ifLuaTeX
  \usepackage{selnolig}  % disable illegal ligatures
\fi
\usepackage{bookmark}

\IfFileExists{xurl.sty}{\usepackage{xurl}}{} % add URL line breaks if available
\urlstyle{same} % disable monospaced font for URLs
\hypersetup{
  pdftitle={Games, Beliefs and Credences},
  pdfauthor={Brian Weatherson},
  hidelinks,
  pdfcreator={LaTeX via pandoc}}

\title{Games, Beliefs and Credences}
\author{Brian Weatherson}
\date{2016}

\begin{document}
\maketitle
\begin{abstract}
In previous work I've defended an interest-relative theory of belief.
This paper continues the defence. I have four aims. First, to offer a
new kind of reason for being unsatisfied with the simple Lockean
reduction of belief to credence. Second, to defend the legitimacy of
appealing to credences in a theory of belief. Third, to illustrate the
importance of theoretical, as well as practical, interests in an
interest-relative account of belief. And finally, to have another try at
extending my basic account of belief to cover propositions that are
practically and theoretically irrelevant to the agent.
\end{abstract}

\setstretch{1.1}
In previous work (\citeproc{ref-Weatherson2005-WEACWD}{Weatherson 2005},
\citeproc{ref-Weatherson2011-WEADIR}{2011},
\citeproc{ref-Weatherson2012-WEAKBI}{2012b}) I've defended an
interest-relative theory of belief. This paper continues the defence. I
have four aims.

\begin{enumerate}
\def\labelenumi{\arabic{enumi}.}
\tightlist
\item
  To offer a new kind of reason for being unsatisfied with the simple
  Lockean reduction of belief to credence.
\item
  To defend the legitimacy of appealing to credences in a theory of
  belief.
\item
  To illustrate the importance of theoretical, as well as practical,
  interests in an interest-relative account of belief.
\item
  To have another try at extending my basic account of belief to cover
  propositions that are practically and theoretically irrelevant to the
  agent.
\end{enumerate}

You're probably familiar with the following dialectic. We want there to
be some systematic connection between credences and beliefs. At first
blush, saying that a person believes \emph{p} and has a very low
credence in \emph{p} isn't just an accusation of irrationality, it is
literally incoherent. The simplest such connection would be a reduction
of beliefs to credences. But the simplest reductions don't work.

If we identify beliefs with credence 1, and take credences to support
betting dispositions, then a rational agent will have very few beliefs.
There are lots of things that an agent, we would normally say, believes
even though she wouldn't bet on them at absurd odds. Note that this
argument doesn't rely on \emph{reducing} credences to betting
dispositions; as long as credences support the betting dispositions, the
argument goes through.

A simple retreat is to the so-called \textbf{Lockean thesis}, which
holds that to believe that \emph{p} is to have credence in \emph{p}
greater than some threshold \emph{t}, where \(t < 1\). Just how the
threshold is determined could be a matter of some discretion. Perhaps it
is a function of the agent's situation, or of the person ascribing
beliefs to the agent, or to the person evaluating that ascription. Never
mind these complexities; assuming all such things are held fixed, the
Lockean thesis says that there is a threshold \emph{t} such that
everything with credence above \emph{t} is believed.

There's a simple objection to the Lockean thesis. Given some very weak
assumptions about the world, it implies that there are plenty of
quadruples \(\langle S, A, B, A \wedge B \rangle\) such that

\begin{itemize}
\tightlist
\item
  \(S\) is a rational agent.
\item
  \(A, B\) and \(A \wedge B\) are propositions.
\item
  \(S\) believes \(A\) and believes \(B\).
\item
  \(S\) does not believe \(A \wedge B\).
\item
  \(S\) knows that she has all these states, and consciously
  reflectively endorses them.
\end{itemize}

Now one might think, indeed I do think, that such quadruples do not
exist at all. But set that objection aside. If the Lockean is correct,
these quadruples should be everywhere. That's because for any
\(t \in (0, 1)\) you care to pick, quadruples of the form
\(\langle S, C, D, C \wedge D \rangle\) are very very common.

\begin{itemize}
\tightlist
\item
  \(S\) is a rational agent.
\item
  \(C, D\) and \(C \wedge D\) are propositions.
\item
  \(S\)'s credence in \(C\) is greater than \emph{t}, and her credence
  in \(D\) is greater than \emph{t}.
\item
  \(S\)'s credence in \(C \wedge D\) is less than \emph{t}.
\item
  \(S\) knows that she has all these states, and reflectively endorses
  them.
\end{itemize}

The best arguments for the existence of quadruples
\(\langle S, A, B, A \wedge B \rangle\) are non-constructive existence
proofs. David Christensen (\citeproc{ref-Christensen2005}{2005}) for
instance, argues from the existence of the preface paradox to the
existence of these quadruples. I've expressed some reservations about
that argument in the past
(\citeproc{ref-Weatherson2005-WEACWD}{Weatherson 2005}). But what I want
to stress here is that even if these existence proofs work, they don't
really prove what the Lockean needs. They don't show that quadruples
satisfying the constraints we associated with
\(\langle S, A, B, A \wedge B \rangle\) are just as common as quadruples
satisfying the constraints we associated with
\(\langle S, C, D, C \wedge D \rangle\), for any \emph{t}. But if the
Lockean were correct, they should be exactly as common.

This kind of consideration pushes some of us, well me in any case,
towards an interest-relative account of belief. But I'm going to set
that move aside to start by investigating a different objection. This
objection holds that the Lockean thesis could not be true, because
credence 1 is not \emph{sufficient} for belief. That is, the Lockean is
committed to the thesis known as \emph{regularity}; that everything left
open by belief gets a positive credence. I think regularity is false.
That's hardly news, there are plenty of good arguments against it,
though most of these involve cases with some idealisations. Timothy
Williamson (\citeproc{ref-Williamson2007-WILHPI}{2007a}) has a
compelling argument against regularity turning on reflections about a
case involving infinite coin flips.\footnote{If there's any gap in
  Williamson's argument, it is I think at the point where he concludes
  that any two infinite sequences of coin flips have the same
  probability of landing all heads. I think that the defender of
  non-numerical, comparative approaches to probability can deny that
  with some plausibility. Perhaps the two sequences of coin flips have
  \emph{incomparable} probabilities of landing all heads. But this leads
  us into complications that are irrlevant to this paper, especially
  since I think it turns out there is a sound Williamsonian argument
  against the Lockean who lets different sequences have incomparable
  probabilities. For a more pessimistic take on Williamson's argument,
  see Weintraub (\citeproc{ref-Weintraub2008}{2008}).} I'm going to
offer a `finite' argument against regularity, which I hope is of
independent interest, and from that conclude the Lockean is mistaken.
There is a worry that my argument against the Lockean also undermines my
preferred positive view, and I'll suggest an independently motivated
patch. I'll then turn to Richard Holton's attack on the very notion of
credence, which obviously would have repercussions for attempts to
understand beliefs in terms of credences were it to succeed. I think it
doesn't succeed, but it does show there are important and
underappreciated constraints on a theory of belief. I'll conclude with a
comparison between my preferred interest-relative account of belief, and
a recent account suggested by Jacob Ross and Mark Schroeder. The short
version of the comparison is that I think there's less difference
between the views than Ross and Schroeder think, though naturally I
think what differences there are favour my view.

\section{Playing Games with a
Lockean}\label{playing-games-with-a-lockean}

I'm going to raise problems for Lockeans, and for defenders of
regularity in general, by discussing a simple game. The game itself is a
nice illustration of how a number of distinct solution concepts in game
theory come apart. (Indeed, the use I'll make of it isn't a million
miles from the use that Kohlberg and Mertens
(\citeproc{ref-KohlbergMertens1986}{1986}) make of it.) To set the
problem up, I need to say a few words about how I think of game theory.
This won't be at all original - most of what I say is taken from
important works by Robert Stalnaker (\citeproc{ref-Stalnaker1994}{1994},
\citeproc{ref-Stalnaker1996}{1996}, \citeproc{ref-Stalnaker1998}{1998},
\citeproc{ref-Stalnaker1999}{1999}). But it is different to what I used
to think, and perhaps to what some other people think too, so I'll set
it out slowly.\footnote{I'm grateful to the participants in a game
  theory seminar at Arché in 2011, especially Josh Dever and Levi
  Spectre, for very helpful discussions that helped me see through my
  previous confusions.}

Start with a simple decision problem, where the agent has a choice
between two acts \(A_1\) and \(A_2\), and there are two possible states
of the world, \(S_1\) and \(S_2\), and the agent knows the payouts for
each act-state pair are given by the following able.

\begin{longtable}[]{@{}lcc@{}}
\toprule\noalign{}
\endhead
\bottomrule\noalign{}
\endlastfoot
& \(S_1\) & \(S_2\) \\
\(A_1\) & 4 & 0 \\
\(A_2\) & 1 & 1 \\
\end{longtable}

What to do? I hope you share the intuition that it is radically
underdetermined by the information I've given you so far. If \(S_2\) is
much more probable than \(S_1\), then \(A_2\) should be chosen;
otherwise \(A_1\) should be chosen. But I haven't said anything about
the relative probability of those two states. Now compare that to a
simple game. Row has two choices, which I'll call \(A_1\) and \(A_2\).
Column also has two choices, which I'll call \(S_1\) and \(S_2\). It is
common knowledge that each player is rational, and that the payouts for
the pairs of choices are given in the following table. (As always, Row's
payouts are given first.)

\begin{longtable}[]{@{}lcc@{}}
\toprule\noalign{}
\endhead
\bottomrule\noalign{}
\endlastfoot
& \(S_1\) & \(S_2\) \\
\(A_1\) & 4, 0 & 0, 1 \\
\(A_2\) & 1, 0 & 1, 1 \\
\end{longtable}

What should Row do? This one is easy. Column gets 1 for sure if she
plays \(S_2\), and 0 for sure if she plays \(S_1\). So she'll play
\(S_2\). And given that she's playing \(S_2\), it is best for Row to
play \(A_2\).

You probably noticed that the game is just a version of the decision
problem that we discussed a couple of paragraphs ago. The relevant
states of the world are choices of Column. But that's fine; we didn't
say in setting out the decision problem what constituted the states
\(S_1\) and \(S_2\). And note that we solved the problem without
explicitly saying anything about probabilities. What we added was some
information about Column's payouts, and the fact that Column is
rational. From there we deduced something about Column's play, namely
that she would play \(S_2\). And from that we concluded what Row should
do.

There's something quite general about this example. What's distinctive
about game theory isn't that it involves any special kinds of decision
making. Once we get the probabilities of each move by the other player,
what's left is (mostly) expected utility maximisation. (We'll come back
to whether the `mostly' qualification is needed below.) The distinctive
thing about game theory is that the probabilities aren't specified in
the setup of the game; rather, they are solved for. Apart from special
cases, such as where one option strictly dominates another, we can't say
much about a decision problem with unspecified probabilities. But we can
and do say a lot about games where the setup of the game doesn't specify
the probabilities, because we can solve for them given the other
information we have.

This way of thinking about games makes the description of game theory as
`interactive epistemology' (\citeproc{ref-Aumann1999}{Aumann 1999})
rather apt. The theorist's work is to solve for what a rational agent
should think other rational agents in the game should do. From this
perspective, it isn't surprising that game theory will make heavy use of
equilibrium concepts. In solving a game, we must deploy a theory of
rationality, and attribute that theory to rational actors in the game
itself. In effect, we are treating rationality as something of an
unknown, but one that occurs in every equation we have to work with. Not
surprisingly, there are going to be multiple solutions to the puzzles we
face.

This way of thinking lends itself to an epistemological interpretation
of one of the most puzzling concepts in game theory, the mixed strategy.
Arguably the core solution concept in game theory is the Nash
equilibrium. As you probably know, a set of moves is a Nash equilibrium
if no player can improve their outcome by deviating from the
equilibrium, conditional on no other player deviating. In many simple
games, the only Nash equilibria involve mixed strategies. Here's one
simple example.

\newpage

\begin{longtable}[]{@{}lcc@{}}
\toprule\noalign{}
\endhead
\bottomrule\noalign{}
\endlastfoot
& \(S_1\) & \(S_2\) \\
\(A_1\) & 0, 1 & 10, 0 \\
\(A_2\) & 9, 0 & -1, 1 \\
\end{longtable}

This game is reminiscent of some puzzles that have been much discussed
in the decision theory literature, namely asymmetric Death in Damascus
puzzles. Here Column wants herself and Row to make the `same' choice,
i.e., \(A_1\) and \(S_1\) or \(A_2\) and \(S_2\). She gets 1 if they do,
0 otherwise. And Row wants them to make different choices, and gets 10
if they do. Row also dislikes playing \(A_2\), and this costs her 1
whatever else happens. It isn't too hard to prove that the only Nash
equilibrium for this game is that Row plays a mixed strategy playing
both \(A_1\) and \(A_2\) with probability , while Column plays the mixed
strategy that gives \(S_1\) probability , and \(S_2\) with probability .

Now what is a mixed strategy? It is easy enough to take away form the
standard game theory textbooks a \textbf{metaphysical} interpretation of
what a mixed strategy is. Here, for instance, is the paragraph
introducing mixed strategies in Dixit and Skeath's \emph{Games of
Strategy}.

\begin{quote}
When players choose to act unsystematically, they pick from among their
pure strategies in some random way \ldots We call a random mixture
between these two pure strategies a mixed strategy.
(\citeproc{ref-DixitSkeath2004}{Dixit and Skeath 2004, 186})
\end{quote}

Dixit and Skeath are saying that it is definitive of a mixed strategy
that players use some kind of randomisation device to pick their plays
on any particular run of a game. That is, the probabilities in a mixed
strategy must be in the world; they must go into the players' choice of
play. That's one way, the paradigm way really, that we can think of
mixed strategies metaphysically.

But the understanding of game theory as interactive epistemology
naturally suggests an \textbf{epistemological} interpretation of mixed
strategies.

\begin{quote}
One could easily \ldots{[}model players{]} \ldots turning the choice
over to a randomizing device, but while it might be harmless to permit
this, players satisfying the cognitive idealizations that game theory
and decision theory make could have no motive for playing a mixed
strategy. So how are we to understand Nash equilibrium in model
theoretic terms as a solution concept? We should follow the suggestion
of Bayesian game theorists, interpreting mixed strategy profiles as
representations, not of players' choices, but of their beliefs.
(\citeproc{ref-Stalnaker1994}{Stalnaker 1994, 57--58})
\end{quote}

One nice advantage of the epistemological interpretation, as noted by
Binmore (\citeproc{ref-Binmore2007}{2007, 185}) is that we don't require
players to have \(n\)-sided dice in their satchels, for every \(n\),
every time they play a game.\footnote{Actually, I guess it is worse than
  if some games have the only equilibria involving mixed strategies with
  irrational probabilities. And it might be noted that Binmore's
  introduction of mixed strategies, on page 44 of his
  (\citeproc{ref-Binmore2007}{2007}), sounds much more like the
  metaphysical interpretation. But I think the later discussion is meant
  to indicate that this is just a heuristic introduction; the
  epistemological interpretation is the correct one.} But another
advantage is that it lets us make sense of the difference between
playing a pure strategy and playing a mixed strategy where one of the
`parts' of the mixture is played with probability one.

With that in mind, consider the below game, which I'll call Red-Green.
I've said something different about this game in earlier work
(\citeproc{ref-Weatherson2012-WEAGAT}{Weatherson 2012a}). But I now
think that to understand what's going on, we need to think about mixed
strategies where one element of the mixture has probability one.

Informally, in this game \(A\) and \(B\) must each play either a green
or red card. I will capitalise \(A\)'s moves, i.e., \(A\) can play GREEN
or RED, and italicise \(B\)'s moves, i.e., \(B\) can play \emph{green}
or \emph{red}. If two green cards, or one green card and one red card
are played, each player gets \$1. If two red cards are played, each gets
nothing. Each cares just about their own wealth, so getting \$1 is worth
1 util. All of this is common knowledge. More formally, here is the game
table, with \(A\) on the row and \(B\) on the column.

\begin{longtable}[]{@{}lcc@{}}
\toprule\noalign{}
\endhead
\bottomrule\noalign{}
\endlastfoot
& \emph{green} & \emph{red} \\
GREEN & 1, 1 & 1, 1 \\
RED & 1, 1 & 0, 0 \\
\end{longtable}

When I write game tables like this, and I think this is the usual way
game tables are to be interpreted
(\citeproc{ref-Weatherson2012-WEAKBI}{Weatherson 2012b}), I mean that
the players know that these are the payouts, that the players know the
other players to be rational, and these pieces of knowledge are common
knowledge to at least as many iterations as needed to solve the game.
With that in mind, let's think about how the agents should approach this
game.

I'm going to make one big simplifying assumption at first. We'll relax
this later, but it will help the discussion to start with this
assumption. This assumption is that the doctrine of \textbf{Uniqueness}
applies here; there is precisely one rational credence to have in any
salient proposition about how the game will play. Some philosophers
think that Uniqueness always holds (\citeproc{ref-White2005-WHIEP}{White
2005}). I join with those such as North (\citeproc{ref-North2010}{2010})
and Schoenfield (\citeproc{ref-Schoenfield2013}{2013}) who don't. But it
does seem like Uniqueness might \emph{often} hold; there might often be
a right answer to a particular problem. Anyway, I'm going to start by
assuming that it does hold here.

The first thing to note about the game is that it is symmetric. So the
probability of \(A\) playing GREEN should be the same as the probability
of \(B\) playing \emph{green}, since \(A\) and \(B\) face exactly the
same problem. Call this common probability \emph{x}. If \(x < 1\), we
get a quick contradiction. The expected value, to Row, of GREEN, is 1.
Indeed, the known value of GREEN is 1. If the probability of
\emph{green} is \emph{x}, then the expected value of RED is \emph{x}. So
if \(x < 1\), and Row is rational, she'll definitely play GREEN. But
that's inconsistent with the claim that \(x < 1\), since that means that
it isn't definite that Row will play GREEN.

So we can conclude that \(x = 1\). Does that mean we can know that Row
will play GREEN? No.~Assume we could conclude that. Whatever reason we
would have for concluding that would be a reason for any rational person
to conclude that Column will play \emph{green}. Since any rational
person can conclude this, Row can conclude it. So Row knows that she'll
get 1 whether she plays GREEN or RED. But then she should be indifferent
between playing GREEN and RED. And if we know she's indifferent between
playing GREEN and RED, and our only evidence for what she'll play is
that she's a rational player who'll maximise her returns, then we can't
be in a position to know she'll play GREEN.

I think the arguments of the last two paragraphs are sound. We'll turn
to an objection presently, but let's note how bizarre is the conclusion
we've reached. One argument has shown that it could not be more probable
that Row will play GREEN. A second argument has shown that we can't know
that Row will play GREEN. It reminds me of examples involving blindspots
(\citeproc{ref-Sorensen1988}{Sorensen 1988}). Consider this case:

\begin{description}
\tightlist
\item[(B)]
Brian does not know (B).
\end{description}

That's true, right? Assume it's false, so I do know (B). Knowledge is
factive, so (B) is true. But that contradicts the assumption that it's
false. So it's true. But I obviously don't know that it's true; that's
what this very true proposition says.\footnote{It's received wisdom in
  philosophy that one can never properly say something of the form
  \emph{p, but I don't know that p}. This is used as a data point in
  views as far removed from each other as those defended in Heal
  (\citeproc{ref-Heal1994}{1994}) and Williamson
  (\citeproc{ref-Williamson1996-WILKAA}{1996}). But I don't feel the
  force of this alleged datum at all, and (B) is just one reason. For a
  different kind of case that makes the same point, see Maitra and
  Weatherson (\citeproc{ref-MaitraWeatherson}{2010}).}

Now I'm not going to rest anything on this case, because there are so
many tricky things one can say about blindspots, and about the paradoxes
generally. It does suggest that there are other finite cases where one
can properly have maximal credence in a true proposition without
knowledge.\footnote{As an aside, the existence of these cases is why I
  get so irritated when epistemologists try to theorise about `Gettier
  Cases' as a class. What does (B) have in common with inferences from a
  justified false belief, or with otherwise sound reasoning that is ever
  so close to issuing in a false conclusion due to relatively bad luck?
  As far as I can tell, the class of justified true beliefs that aren't
  knowledge is a disjunctive mess, and this should matter for thinking
  about the nature of knowledge. For further examples, see Williamson
  (\citeproc{ref-WilliamsonLofoten}{2013}) and Nagel
  (\citeproc{ref-Nagel2013-Williamson}{2013}).} And, assuming that we
shouldn't believe things we know we don't know, that means we can have
maximal credence in things we don't believe. All I want to point out is
that this phenomena of maximal credence without knowledge, and
presumably without full belief, isn't a quirky feature of
self-reference, or of games, or of puzzles about infinity; it comes up
in a wide range of cases.

For the rest of this section I want to reply to one objection, and
weaken an assumption I made earlier. The objection is that I'm wrong to
assume that agents will only maximise expected utility. They may have
tie-breaker rules, and those rules might undermine the arguments I gave
above. The assumption is that there's a uniquely rational credence to
have in any given situation.

I argued that if we knew that \(A\) would play GREEN, we could show that
\(A\) had no reason to play GREEN. But actually what we showed was that
the expected utility of playing GREEN would be the same as playing RED.
Perhaps \(A\) has a reason to play GREEN, namely that GREEN weakly
dominates RED. After all, there's one possibility on the table where
GREEN does better than RED, and none where RED does better. And perhaps
that's a reason, even if it isn't a reason that expected utility
considerations are sensitive to.

Now I don't want to insist on expected utility maximisation as the only
rule for rational decision making. Sometimes, I think some kind of
tie-breaker procedure is part of rationality. In the papers by Stalnaker
I mentioned above, he often appeals to this kind of weak dominance
reasoning to resolve various hard cases. But I don't think weak
dominance provides a reason to play GREEN in this particular case. When
Stalnaker says that agents should use weak dominance reasoning, it is
always in the context of games where the agents' attitude towards the
game matrix is different to their attitude towards each other. One case
that Stalnaker discusses in detail is where the game table is common
knowledge, but there is merely common (justified, true) belief in common
rationality. Given such a difference in attitudes, it does seem there's
a good sense in which the most salient departure from equilibrium will
be one in which the players end up somewhere else on the table. And
given that, weak dominance reasoning seems appropriate.

But that's not what we've got here. Assuming that rationality requires
playing GREEN/\emph{green}, the players know we'll end up in the top
left corner of the table. There's no chance that we'll end up elsewhere.
Or, perhaps better, there is just as much chance we'll end up `off the
table', as that we'll end up in a non-equilibrium point on the table. To
make this more vivid, consider the `possibility' that \(B\) will play
\emph{blue}, and if \(B\) plays \emph{blue}, \(A\) will receive 2 if she
plays RED, and -1 if she plays GREEN. Well hold on, you might think,
didn't I say that \emph{green} and \emph{red} were the only options, and
this was common knowledge? Well, yes, I did, but if the exercise is to
consider what would happen if something the agent knows to be true
doesn't obtain, then the possibility that one agent will play blue
certainly seems like one worth considering. It is, after all, a
metaphysical possibility. And if we take it seriously, then it isn't
true that under \emph{any} possible play of the game, GREEN does better
than RED.

We can put this as a dilemma. Assume, for \emph{reductio}, that
GREEN/\emph{green} is the only rational play. Then if we restrict our
attention to possibilities that are epistemically open to \(A\), then
GREEN does just as well as RED; they both get 1 in every possibility. If
we allow possibilities that are epistemically closed to \(A\), then the
possibility where \(B\) plays \emph{blue} is just as relevant as the
possibility that \(B\) is irrational. After all, we stipulated that this
is a case where rationality is common knowledge. In neither case does
the weak dominance reasoning get any purchase.

With that in mind, we can see why we don't need the assumption of
Uniqueness. Let's play through how a failure of Uniqueness could
undermine the argument. Assume, again for \textbf{reductio}, that we
have credence \(\varepsilon > 0\) that \(A\) will play RED. Since \(A\)
maximises expected utility, that means \(A\) must have credence 1 that
\(B\) will play \emph{green}. But this is already odd. Even if you think
people can have different reactions to the same evidence, it is odd to
think that one rational agent could regard a possibility as infinitely
less likely than another, given isomorphic evidence. And that's not all
of the problems. Even if \(A\) has credence 1 that \(B\) will play
\emph{green}, it isn't obvious that playing RED is rational. After all,
relative to the space of epistemic possibilities, GREEN weakly dominates
RED. Remember that we're no longer assuming that it can be known what
\(A\) or \(B\) will play. So even without Uniqueness, there are two
reasons to think that it is wrong to have credence \(\varepsilon > 0\)
that \(A\) will play RED. So we've still shown that credence 1 doesn't
imply knowledge, and since the proof is known to us, and full belief is
incompatible with knowing that you can't know, this is a case where
credence 1 doesn't imply full belief. So whether \(A\) plays GREEN, like
whether the coin will ever land tails, is a case the Lockean cannot get
right. No matter where they set the threshold for belief our credence is
above that threshold, but we don't believe.

So I think this case is a real problem for a Lockean view about the
relationship between credence and belief. If \emph{A} is rational, she
can have credence 1 that \emph{B} will play \emph{green}, but won't
believe that \emph{B} will play \emph{green}. But now you might worry
that my own account of the relationship between belief and credence is
in just as much trouble. After all, I said that to believe \emph{p} is,
roughly, to have the same attitudes towards all salient questions as you
have conditional on \emph{p}. And it's hard to identify a question that
rational \emph{A} would answer differently upon conditionalising on the
proposition that \emph{B} plays \emph{green}.

I think what went wrong in my earlier view was that I'd too quickly
equated updating with conditionalisation. The two can come apart. Here's
an example from Gillies (\citeproc{ref-Gillies2010}{2010}) that makes
the point well.\footnote{A similar example is in Kratzer
  (\citeproc{ref-Kratzer2012}{2012, 94}).}

\begin{quote}
I have lost my marbles. I know that just one of them -- Red or Yellow --
is in the box. But I don't know which. I find myself saying things like
\ldots{}``If Yellow isn't in the box, the Red must be.'' (4:13)
\end{quote}

As Gillies goes on to point out, this isn't really a problem for the
Ramsey test view of conditionals.

\begin{quote}
The Ramsey test -- the schoolyard version, anyway -- is a test for when
an indicative conditional is acceptable given your beliefs. It says that
(if \emph{p})(\emph{q}) is acceptable in belief state \emph{B} iff
\emph{q} is acceptable in the derived or subordinate state
\emph{B}-plus-the-information-that-\emph{p}. (4:27)
\end{quote}

And he notes that this can explain what goes on with the marbles
conditional. Add the information that Yellow isn't in the box, and it
isn't just true, but must be true, that Red is in the box.

Note though that while we can explain this conditional using the Ramsey
test, we can't explain it using any version of the idea that
probabilities of conditionals are conditional probabilities. The
probability that Red must be in the box is 0. The probability that
Yellow isn't in the box is not 0. So conditional on Yellow not being in
the box, the probability that Red must be in the box is still 0. Yet the
conditional is perfectly assertable.

There is, and this is Gillies's key point, something about the behaviour
of modals in the consequents of conditionals that we can't capture using
conditional probabilities, or indeed many other standard tools. And what
goes for consequents of conditionals goes for updated beliefs too. Learn
that Yellow isn't in the box, and you'll conclude that Red must be. But
that learning can't go via conditionalisation; just conditionalise on
the new information and the probability that Red must be in the box goes
from 0 to 0.

Now it's a hard problem to say exactly how this alternative to updating
by conditionalisation should work. But very roughly, the idea is that at
least some of the time, we update by eliminating worlds from the space
of possibilities. This affects dramatically the probability of
propositions whose truth is sensitive to which worlds are in the space
of possibiilties.

For example, in the game I've been discussing, we should believe that
rational \emph{B} might play \emph{red}. Indeed, the probability of that
is, I think, 1. And whether or not \emph{B} might play red is highly
salient; it matters to the probability of whether \emph{A} will play
GREEN or RED. Conditionalising on something that has probability 1, such
as that \emph{B} will play \emph{green}, can hardly change that
probability. But updating on the proposition that \emph{B} will play
\emph{green} can make a difference. We can see that by simply noting
that the conditional \emph{If B plays green, she might play red} is
incoherent.

So I conclude that a theory of belief like mine can handle the puzzle
this game poses, as long as it distinguishes between conditionalising
and updating, in just the way Gillies suggests. To believe that \emph{p}
is to be disposed to not change any attitude towards a salient question
on updating that \emph{p}. (Plus some bells and whistles to deal with
propositions that are not relevant to salient questions. We'll return to
them below.) Updating often goes by conditionalisation, so we can often
say that belief means having attitudes that match unconditionally and
conditionally on \emph{p}. But not all updating works that way, and the
theory of belief needs to acknowledge this.

\section{Holton on Credence}\label{holton-on-credence}

While I don't agree with the Lockeans, I do endorse a lot of similar
theses to them about the relationship between belief and credence. These
theses include that both beliefs and credences exist and that the two
are constitutively (as opposed to merely causally) connected. I differ
from the Lockeans in holding that both belief and credence have
important explanatory roles, and that the connection between the two
goes via the interests of the agent. As with most work in this area, my
views start off from considerations of cases much like DeRose's famous
bank cases.\footnote{The idea of using allergies to illustrate the kind
  of case we're interested in is due to Ross and Schroeder
  (\citeproc{ref-SchroederRoss2012}{2014}), and I'm grateful for the
  idea. It makes the intuitions much more vivid. The kind of cases we're
  considering play a big role in, \emph{inter alia},
  (\citeproc{ref-DeRose1992}{DeRose 1992};
  \citeproc{ref-Cohen1999}{Cohen 1999}) and Fantl and McGrath
  (\citeproc{ref-Fantl2002}{2002}).} Here's another contribution to the
genre. I know it's an overcrowded field, but I wanted a case that (a) is
pretty realistic, and (b) doesn't involve the attribution (either to
oneself or others) of a propositional attitude. In the example, X and Y
are parents of a child, Z.

\begin{quote}
Y: This salad you bought is very good. Does it have nuts in it?\\
X: No.~The nuttiness you're tasting is probably from the beans.\\
Y: Oh, so we could pack it for Z's lunch tomorrow.\\
X: Hang on, I better check about the nuts. Z's pre-school is very fussy
about nuts. One of the children there might have an allergy, and it
would be awful to get in trouble over her lunch.
\end{quote}

Here's what I think is going on in that exchange.\footnote{What I say
  here obviously has some similarities to a view put forward by Jennifer
  Nagel (\citeproc{ref-Nagel2008}{2008}), but I ultimately end up
  drawing rather different conclusions to the ones she draws.} At
\(t_2\) (I'll use \(t_i\) for the time of the \(i\)'th utterance in the
exchange), X believes that the salad has no nuts in it. Indeed, the one
word sentence ``No'' expresses that belief. But by \(t_4\), X has lost
that belief. It would be fine to pack the salad for lunch if it has no
nuts, but X isn't willing to do this for the simple reason that X no
longer believes that it has no nuts. Moreover, this change of belief
was, or at least could have been for all we've said so far, rational on
X's part.

There's something a little puzzling about that. Jacob Ross and Mark
Schroeder (\citeproc{ref-SchroederRoss2012}{2014}) voice a common
intuition when they say that beliefs should only change when new
evidence comes in. Indeed, they use this intuition as a key argument
against my view of belief. But X doesn't get any evidence that bears on
the nuttiness of the salad. Yet X rationally changes beliefs. So I just
conclude that sometimes we can change beliefs without new evidence
coming in; sometimes our interests, broadly construed, change, and that
is enough to change beliefs.

We'll come back to Ross and Schroeder's arguments in the next section,
because first I want to concede something to the view that only evidence
changes beliefs. That view is false, but there might be a true view in
the area. And that's the view that only change in evidence can change
\emph{credences}. But that view only makes sense if there are such
things as credences, and that's something that Richard Holton
(\citeproc{ref-Holton2013}{2014}) has recently launched an intriguing
argument against.

Holton's broader project is a much more sweeping attack on the Lockean
thesis than I have proposed. Actually, it is a pair of more sweeping
attacks. One of the pair is that the Lockeans identify something that
exists, namely belief, with something that doesn't, namely high
credence. I would not, could not, sign up for that critique. But I am
much more sympathetic to the other attack in the pair, namely that
credences and beliefs have very different dynamics.

Credences are, by their nature, exceedingly unstable. Whether an agent's
credence that \emph{p} is above or below any number \emph{x} is liable
to change according to any number of possible changes in evidence. But,
at least if the agent is rational, beliefs are not so susceptible to
change. Holton thinks that rational agents, or at least rational humans,
frequently instantiate the following pattern. They form a belief that
\emph{p}, on excellent grounds. They later get some evidence that
\(\neg p\). The evidence is strong enough that, had they had it to begin
with, they would have remained uncertain about \emph{p}. But they do not
decide to reopen the investigation into whether \emph{p}. They hold on
to their belief that \emph{p}, the matter having been previously
decided.

Such an attitude might look like unprincipled dogmatism. But it need not
be, I think, as long as four criteria are met. (I think Holton agrees
with these criteria.) One is that the agent's willingness to reopen the
question of whether \emph{p} must increase. She must be more willing, in
the light of yet more evidence against \emph{p}, to consider whether
\emph{p} is really true. A second is that, should the agent
(irrationally) reopen the question of whether \emph{p}, she should not
use the fact that she previously closed that question as evidence. Once
the genie is out of the box, only reasoning about \emph{p} can get it
back in. A third is that the costs of the inquiry must be high enough to
warrant putting it off. If simply turning one's head fifteen degrees to
the left will lead to acquiring evidence that definitively settles
whether \emph{p}, it is a little dogmatic to refuse to do so in the face
of evidence against one's previously formed opinion that \emph{p}. And
finally, the costs of being wrong about \emph{p} must not be too high.
X, in our little dialogue above, would be terribly dogmatic if they
didn't reopen the question of whether the salad had nuts in it, on being
informed that this information was being used in a high stakes inquiry.

So beliefs should have a kind of resilience. Credences, if they exist,
should not have this kind of resilience. So this suggests that a simple
reduction of belief to credence, as the Lockeans suggest, cannot be
right. You might worry that things are worse, that no reduction of
belief to credence can be compatible with the difference in resilience
between belief and credence. We'll return to that point, because first I
want to look at Holton's stronger claim: that there are no such things
as credences.

Holton acknowledges, as of course he must, that we have probabilistic
truth-directed attitudes. We can imagine a person, call her Paula, who
thinks it's likely that Richard III murdered his nephews, for instance.
But Holton offers several reasons for thinking that in these
probabilistic truth-directed attitudes, the probability goes in the
content, not in the attitude. That is, we should interpret Paula as
believing the probabilistic claim, Richard III probably murdered his
nephews, and not as having some graded attitude towards the simple
proposition \emph{Richard III murdered his nephews}. More precisely,
Holton thinks we should understand Paula's \emph{explicit} attitudes
that way, and that independent of having reason to think that agents
explicitly have probabilistic attitudes, there's no good way to make
sense of the claim that they implicitly have probabilistic attitudes. So
there's no such thing as credences, as usually understood. Or, at least,
there's no good sense to be made of the claim that there are credences.

In response, I want to make six points.

\begin{enumerate}
\def\labelenumi{\arabic{enumi}.}
\tightlist
\item
  Holton is right about cases like Paula's, and the possibility of
  iterating terms like \emph{probably} provides independent support for
  this view.
\item
  Beliefs like the one Paula has are odd; they seem to have very strange
  truth conditions.
\item
  Our theory of mind needs some mechanism for explaining the
  relationship between confidence and action.
\item
  The `explanatory gap' here could be filled by positing a binary
  attitude \emph{is more confident that}.
\item
  This binary attitude can do all the work that graded attitudes were
  posited to do, and in a (historically sensitive) way saves the
  credences story.
\item
  Credences (or at least confidences) can have a key role within a
  Holton-like story about graded belief. They can both explain why
  agents reconsider some beliefs, and provide a standard of correctness
  for decisions to reconsider.
\end{enumerate}

Let's take those in order.

I'm not going to rehearse Holton's argument for the `content view': that
in cases like Paula's the content of her attitude, and not the attitude
itself, is probabilistic. But I do want to offer one extra consideration
in its favour. (I'm indebted here to work in progress by my colleague
Sarah Moss (\citeproc{ref-MossPragmaticsEpistemicModals}{2015}), though
I'm not sure she'd approve of this conclusion.) As well as Paula, we can
imagine a person Pip who isn't sure that Paula is right, but thinks
she's probably right. That is, Pip thinks that Richard III probably
probably murdered his nephews. It's easy to make sense of Pip on the
content view. Modalities in propositions iterate smoothly; that's what
they are designed to do. But it's much harder to iterate attitudes. The
possibility of cases like Pip suggests Holton must be right about
Paula's case.

But Paula's case is odd. Beliefs have truth conditions. What are the
truth conditions for Paula's belief? On the one hand, it seems they must
be sensitive to her evidence. If she later acquires conclusive evidence
that Richard III was framed, she won't think her earlier self had a
false belief. But if we put the evidence into the content of the belief,
we get the strange result that her belief can't be preserved by uttering
the same words to herself over again. That is, if the content of Paula's
belief is \emph{Given the evidence I have now, Richard III likely
murdered his nephews}, she can't have the very same belief tomorrow by
retaining the thought \emph{Richard III likely murdered his nephews}.
And she can't have a belief with the same content as anyone else by the
two of them both thinking \emph{Richard III likely murdered his
nephews}. Those seem like unhappy conclusions, especially in the midst
of a project that wants to emphasise the resiliency of belief. So
perhaps we should say, following Stephenson
(\citeproc{ref-Stephenson2007}{2007}) or MacFarlane
(\citeproc{ref-MacFarlane2011}{2011}), that the truth conditions of the
belief are agent-relative. Or, if we're unhappy with the MacFarlane
story, we might be pushed towards a kind of expressivism (perhaps a la
Yalcin (\citeproc{ref-Yalcin2011}{2011})), which isn't quite like either
the content view or the attitude view that Holton discusses. I'm
personally partial to the relativist view, but I don't want to argue for
that here, just note that the content view raises some interesting
problems, and that natural solutions to them could in a way blur the
boundaries between the content and attitude views.

As Holton notes in his discussion of Brutus, when our confidence in a
proposition changes, our actions will change. Paula gets a little
evidence that Richard III was framed, and her actions may change. Of
course, not much of what we do in everyday life is sensitive to facts
about English royal history, but there may be some effects. Maybe she'll
be less inclined to speak up if the topic of the princes' murder comes
up, or she'll take a slightly more jaundiced view of Shakespeare's play
(compare Friend (\citeproc{ref-Friend2003}{2003}).) Holton says that
these falling confidences need not have all the precise structure of
credences. In particular, they may not have the topology of the interval
\([0, 1]\). But lots of credence lovers think that's too demanding.
There's a long tradition of thinking that credences need not all be
comparable.\footnote{Notable members of the tradition include Levi
  (\citeproc{ref-Levi1974}{1974}), Jeffrey
  (\citeproc{ref-Jeffrey1983}{1983}) and Fraassen
  (\citeproc{ref-vanFraassen1989}{1989}).} What's important is that the
relative confidences exist, and that they have a robust relationship to
action.

There's an old fashioned way of doing this. The idea is implicit in
Ramsey (\citeproc{ref-RamseyTruthProb}{1926}), and made central in
DeFinetti (\citeproc{ref-DeFinetti1964}{1964}). Take the binary attitude
\emph{is more confident that p than q} as primitive. As Holton notes,
surface structure of our attitude reports suggest that this attitude,
unlike the graded attitude of credence, is part of folk psychology. Lay
down some constraints on this attitude. To get enough constraints that
the binary relation determines a unique probability function, the
constraints will have to be very tight. In particular, you'll need some
kind of Archimedean principle, and a principle of universal
comparability. Those aren't very plausible, especially the second. But
even weaker constraints will get you something interesting. In
particular, it isn't hard to lay down enough constraints that there is a
unique set \(S\) of probability functions such that the agent is more
confident that \emph{p} than \(q\) just in case \(\Pr( p) > \Pr(q)\) for
all \(\Pr \in S\). (For much more detail, see for instance Walley
(\citeproc{ref-Walley1991}{1991}).)

In that way, we can derive credences from the relative confidences of a
reasonably coherent agent. But we can do with even less coherence than
that I think. A throwaway remark from Ramsey
(\citeproc{ref-Ramsey1929}{1929/1990}) provides a key clue. What is it
to have credence \(\frac{2}{3}\) in \emph{p}? Don't say it's a betting
disposition; mental states and behavioural dispositions aren't that
tightly linked. Here's Ramsey's idea. To have credence \(\frac{2}{3}\)
in \emph{p} is to be exactly as confident in \emph{p} as in
\(q \vee r\), where \(q, r\) and \emph{s} are taken to be exclusive and
exhaustive, and one has equal confidence in all three. It's easy to see
how to extend that to a definition of credence \(\frac{m}{n}\) for any
integer \(m, n\). It's a little trickier to say precisely what, say,
credence \(\frac{1}{\pi}\) is, but rational credences are probably
credences enough to explain action. And just like that, we have a way of
talking about credences, i.e., graded attitudes, without positing
anything more than a binary attitude \emph{more confident than}.

Perhaps Holton could argue that we only have unary attitudes, not binary
attitudes like \emph{more confident than}. If Maury is more confident
that Oswald shot JFK than that Richard III murdered his nephews, that
means he really believes the proposition \emph{It is more likely that
Oswald shot JFK than that Richard III murdered his nephews}. But such a
view seems forced at best, and isn't motivated by Holton's other
arguments for the `content view'. This attitude of \emph{more confident
than} isn't iterable. It isn't subject to the particular kind of
reasoning errors that Holton takes to be evidence for the content view
in the probabilistic case. It is an attitude we ordinarily report as a
binary attitude in normal speech. In short, it looks like a genuine
binary attitude.

Given that the binary attitude exists, and that we can define numerical
(at least rational) credences in terms of it, I'd say that's enough to
say that credences exist. In a sense, credences will be epiphenomenal.
What does the explanatory work is the binary relation \emph{more
confident that}. Maury might stay away from a showing of \emph{Richard
III} because he is less confident that it is historically accurate than
he used to be. We can work out from Maury's other relative confidences
what his credence in Richard III's guilt is and was. Or, at least, we
can work out bounds on these. But those numbers aren't in a fundamental
sense explanatory, and neither are the complicated sets of relative
confidences that constitute the numbers. What's \emph{really}
explanatory are relative confidences. But it's a harmless enough mode of
speech to talk as if credences are explanatory; they are easier to talk
about than the underlying relative confidences.

\section{The Power of Theoretical
Interests}\label{the-power-of-theoretical-interests}

So I think we should accept that credences exist. And we can just about
reduce beliefs to credences. In previous work I argued that we could do
such a reduction. I'm not altogether sure whether the amendments to that
view I'm proposing here means it no longer should count as a reductive
view; we'll come back to that question in the conclusion.

The view I defended in previous work is that the reduction comes through
the relationship between conditional and unconditional attitudes. Very
roughly, to believe that \emph{p} is simply to have the same attitudes,
towards all salient questions, unconditionally as you have conditional
on \emph{p}. In a syrupy slogan, belief means never having to say you've
conditionalised. For reasons I mentioned in section 1, I now think that
was inaccurate; I should have said that belief means never having to say
you've updated, or at least that you've updated your view on any salient
question.

The restriction to salient questions is important. Consider any \emph{p}
that I normally take for granted, but such that I wouldn't bet on it at
insane odds. I prefer declining such a bet to taking it. But conditional
on \emph{p}, I prefer taking the bet. So that means I don't believe any
such \emph{p}. But just about any \emph{p} satisfies that description,
for at least some `insane' odds. So I believe almost nothing. That would
be a \emph{reductio} of the position. I respond by saying that the
choice of whether to take an insane bet is not normally salient.

But now there's a worry that I've let in too much. For many \emph{p},
there is no salient decision that they even bear on. What I would do
conditional on \emph{p}, conditional on \(\neg p\), and unconditionally
is exactly the same, over the space of salient choices. (And this isn't
a case where updating and conditionalising come apart; I'll leave this
proviso mostly implicit from now on.) So with the restriction in place,
I believe \emph{p} and \(\neg p\). That seems like a \emph{reductio} of
the view too. I probably do have inconsistent beliefs, but not in virtue
of \emph{p} being irrelevant to me right now. I've changed my mind a
little about what the right way to avoid this problem is, in part
because of some arguments by Jacob Ross and Mark Schroeder.

They have what looks like, on the surface, a rather different view to
mine. They say that to believe \emph{p} is to have a \textbf{default
reasoning disposition} to use \emph{p} in reasoning. Here's how they
describe their own view.

\begin{quote}
What we should expect, therefore, is that for some propositions we would
have a \emph{defeasible} or \emph{default} disposition to treat them as
true in our reasoning--a disposition that can be overridden under
circumstances where the cost of mistakenly acting as if these
propositions are true is particularly salient. And this expectation is
confirmed by our experience. We do indeed seem to treat some uncertain
propositions as true in our reasoning; we do indeed seem to treat them
as true automatically, without first weighing the costs and benefits of
so treating them; and yet in contexts such as High where the costs of
mistakenly treating them as true is salient, our natural tendency to
treat these propositions as true often seems to be overridden, and
instead we treat them as merely probable.

But if we concede that we have such defeasible dispositions to treat
particular propositions as true in our reasoning, then a hypothesis
naturally arises, namely, that beliefs consist in or involve such
dispositions. More precisely, at least part of the functional role of
belief is that believing that \emph{p} defeasibly disposes the believer
to treat \emph{p} as true in her reasoning. Let us call this hypothesis
the \emph{reasoning disposition account} of belief.
(\citeproc{ref-SchroederRoss2012}{Ross and Schroeder 2014, 9--10})
\end{quote}

There are, relative to what I'm interested in, three striking
characteristics of Ross and Schroeder's view.

\begin{enumerate}
\def\labelenumi{\arabic{enumi}.}
\tightlist
\item
  Whether you believe \emph{p} is sensitive to how you reason; that is,
  your theoretical interests matter.
\item
  How you would reason about some questions that are not live is
  relevant to whether you believe \emph{p}.
\item
  Dispositions can be masked, so you can believe \emph{p} even though
  you don't actually use \emph{p} in reasoning now.
\end{enumerate}

I think they take all three of these points to be reasons to favour
their view over mine. As I see it, we agree on point 1 (and I always had
the resources to agree with them), I can accommodate point 2 with a
modification to my theory, and point 3 is a cost of their theory, not a
benefit. Let's take those points in order.

There are lots of reasons to dislike what Ross and Schroeder call
\emph{Pragmatic Credal Reductionism} (PCR). This is, more or less, the
view that the salient questions, in the sense relevant above, are just
those which are practically relevant to the agent. So to believe
\emph{p} just is to have the same attitude towards all practically
relevant questions unconditionally as conditional on \emph{p}. There are
at least three reasons to resist this view.

One reason comes from the discussions of Ned Block's example Blockhead
~(\citeproc{ref-Block1978}{Block 1978}). As Braddon-Mitchell and Jackson
point out, the lesson to take from that example is that beliefs are
constituted in part by their relations to other mental states
~(\citeproc{ref-DBMJackson2007}{Braddon-Mitchell and Jackson 2007,
114ff}). There's a quick attempted refutation of PCR via the Blockhead
case which doesn't quite work. We might worry that if all that matters
to belief given PCR is how it relates to action, PCR will have the
implausible consequence that Blockhead has a rich set of beliefs. That
isn't right; PCR is compatible with the view that Blockhead doesn't have
credences, and hence doesn't have credences that constitute beliefs. But
the Blockhead example's value isn't exhausted by its use in quick
refutations.\footnote{The point I'm making here is relevant I think to
  recent debates about the proper way to formalise counterexamples in
  philosophy, as in Williamson
  (\citeproc{ref-Williamson2007-WILTPO-17}{2007b}), Ichikawa and Jarvis
  (\citeproc{ref-IchikawaJarvis2009}{2009}), and Malmgren
  (\citeproc{ref-Malmgren2011}{2011}). I worry that too much of that
  debate is focussed on the role that examples play in one-step
  refutations. But there's more, much more, to a good example than that.}
The lesson is that beliefs are, by their nature, interactive. It seems
to me that PCR doesn't really appreciate that lesson.

Another reason comes from recent work by Jessica Brown
(\citeproc{ref-Brown2013}{2014}). Compare these two situations.

\begin{enumerate}
\def\labelenumi{\arabic{enumi}.}
\tightlist
\item
  \emph{S} is in circumstances \emph{C}, and has to decide whether to do
  \emph{X}.
\item
  \emph{S} is in completely different circumstances to \emph{C}, but is
  seriously engaged in planning for future contingencies. She's
  currently trying to decide whether in circumstances \emph{C} to do
  \emph{X}.
\end{enumerate}

Intuitively, \emph{S} can bring exactly the same evidence, knowledge and
beliefs to bear on the two problems. If \emph{C} is a particularly high
stakes situation, say it is a situation where one has to decide what to
feed someone with a severe peanut allergy, then a lot of things that can
ordinarily be taken for granted cannot, in this case, be taken for
granted. And that's true whether \emph{S} is actually in \emph{C}, or
she is just planning for the possibility that she finds herself in
\emph{C}.

So I conclude that both practical and theoretical interests matter for
what we can take for granted in inquiry. The things we can take for
granted into a theoretical inquiry into what to do in high stakes
contexts as restricted, just as they are when we are in a high stakes
context, and must make a practical decision. Since the latter
restriction on what we can take for granted is explained by (and
possibly constituted by) a restriction on what we actually believe in
those contexts, we should similarly conclude that agents simply believe
less when they are reasoning about high stakes contexts, whatever their
actual context.

A third reason to dislike PCR comes from the `Renzi' example in Ross and
Schroeder's paper. I'll run through a somewhat more abstract version of
the case, because I don't think the details are particularly important.
Start with a standard decision problem. The agent knows that X is better
to do if \emph{p}, and Y is better to do if \(\neg p\). The agent should
then go through calculating the relative gains to doing X or Y in the
situations they are better, and the probability of \emph{p}. But the
agent imagined doesn't do that. Rather, the agent divides the
possibility space in four, taking the salient possibilities to be
\(p \wedge q, p \wedge \neg q, \neg p \wedge q\) and
\(\neg p \wedge \neg q\), and then calculates the expected utility of X
and Y accordingly. This is a bad bit of reasoning on the agent's part.
In the cases we are interested in, \emph{q} is exceedingly likely.
Moreover, the expected utility of each act doesn't change a lot
depending on \emph{q}'s truth value. So it is fairly obvious that we'll
end up making the same decision whether we take the `small worlds' in
our decision model to be just the world where \emph{p}, and the world
where \(\neg p\), or the four worlds this agent uses. But the agent does
use these four, and the question is what to say about them.

Ross and Schroeder say that such an agent should not be counted as
believing that \emph{q}. If they are consciously calculating the
probability that \emph{q}, and taking \(\neg q\) possibilities into
account when calculating expected utilities, they regard \emph{q} as an
open question. And regarding \emph{q} as open in this way is
incompatible with believing it. I agree with all this.

They also think that PCR implies that the agent \emph{does} believe
\emph{q}. The reason is that conditionalising on \emph{q} doesn't change
the agent's beliefs about any practical question. I think that's right
too, at least on a natural understanding of what `practical' is.

My response to all these worries is to say that whether someone believes
that \emph{p} depends not just on how conditionalising (or more
generally updating) on \emph{p} would affect someone's action, but on
how it would affect their reasoning. That is, just as we learned from
the Blockhead example, to believe that \emph{p} requires having a mental
state that is connected to the rest of one's cognitive life in roughly
the way a belief that \emph{p} should be connected. Let's go through
both the last two cases to see how this works on my theory.

One of the things that happens when the stakes go up is that
conditionalising on very probable things can change the outcome of
interesting decisions. Make the probability that some nice food is
peanut-free be high, but short of one. Conditional on it being
peanut-free, it's a good thing to give to a peanut-allergic guest. But
unconditionally, it's a bad thing to give to such a guest, because the
niceness of the food doesn't outweigh the risk of killing them. And
that's true whether the guest is actually there, or you're just thinking
about what to do should such a guest arrive in the future. In general,
the same questions will be relevant whether you're in \emph{C} trying to
decide whether to do \emph{X}, or simply trying to decide whether to
\emph{X} in \emph{C}. In one case they will be practically relevant
questions, in the other they will be theoretically relevant questions.
But this feels a lot like a distinction without a difference, since the
agent should have similar beliefs in the two cases.

The same response works for Ross and Schroeder's case. The agent was
trying to work out the expected utility of X and Y by working out the
utility of each action in each of four `small worlds', then working out
the probability of each of these. Conditional on \emph{q}, the
probability of two of them (\(p \wedge \neg q, \neg p \wedge \neg q\)),
will be 0. Unconditionally, this probability won't be 0. So the agent
has a different view on some question they have taken an interest in
unconditionally to their view conditional on \emph{q}. So they don't
believe \emph{q}. The agent shouldn't care about that question, and
conditional on each question they should care about, they have the same
attitude unconditionally and conditional on \emph{q}. But they do care
about these probabilistic questions, so they don't believe \emph{q}.

So I think that Ross and Schroeder and I agree on point 1; something
beyond practical interests is relevant to belief.

They have another case that I think does suggest a needed revision to my
theory. I'm going to modify their case a little to change the focus a
little, and to avoid puzzles about vagueness. (What follows is a version
of their example about Dalı́'s moustache, purged of any worries about
vagueness, and without the focus on consistency. I don't think the
problem they true to press on me, that my theory allows excessive
inconsistency of belief among rational agents, really sticks. Everyone
will have to make qualifications to consistency to deal with the preface
paradox, and for reasons I went over in ~Weatherson
(\citeproc{ref-Weatherson2005-WEACWD}{2005}), I think the qualifications
I make are the best ones to make.)

Let \emph{D} be the proposition that the number of games the Detroit
Tigers won in 1976 (in the MLB regular season) is not a multiple of 3.
At most times, \emph{D} is completely irrelevant to anything I care
about, either practically or theoretically. My attitudes towards any
relevant question are the same unconditionally as conditional on
\emph{D}. So there's a worry that I'll count as believing \emph{D}, and
believing \(\neg D\), by default.

In earlier work, I added a clause meant to help with cases like this. I
said that for determining whether an agent believes that \emph{p}, we
should treat the question of whether \emph{p}'s probability is above or
below 0.5 as salient, even if the agent doesn't care about it. Obviously
this won't help with this particular case. The probability of \emph{D}
is around , and is certainly above 0.5. My `fix' avoids the consequence
that I implausibly count as believing \(\neg D\). But I still count,
almost as implausibly, as believing \emph{D}. This needs to be fixed.

Here's my proposed change. For an agent to count as believing \emph{p},
it must be possible for \emph{p} to do some work for them in reasoning.
Here's what I mean by work. Consider a very abstract set up of a
decision problem, as follows.

\begin{longtable}[]{@{}lcc@{}}
\toprule\noalign{}
\endhead
\bottomrule\noalign{}
\endlastfoot
& \emph{p} & \emph{q} \\
X & 4 & 1 \\
Y & 3 & 2 \\
\end{longtable}

That table encodes a lot of information. It encodes that \(p \vee q\) is
true; otherwise there are some columns missing. It encodes that the only
live choices are X or Y; otherwise there are rows missing. It encodes
that doing X is better than doing Y if \emph{p}, and worse if \emph{q}.

For any agent, and any decision problem, there is a table like this that
they would be disposed to use to resolve that problem. Or, perhaps,
there are a series of tables and there is no fact about which of them
they would be most disposed to use.

Given all that terminology, here's my extra constraint on belief. To
believe that \emph{p}, there must be some decision problem such that
some table the agent would be disposed to use to solve it encodes that
\emph{p}. If there is no such problem, the agent does not believe that
\emph{p}. For anything that I intuitively believe, this is an easy
condition to satisfy. Let the problem be whether to take a bet that pays
1 if \emph{p}, and loses 1 otherwise. Here's the table I'd be disposed
to use to solve the problem.

\begin{longtable}[]{@{}lc@{}}
\toprule\noalign{}
\endhead
\bottomrule\noalign{}
\endlastfoot
& \emph{p} \\
Take bet & 1 \\
Decline bet & 0 \\
\end{longtable}

This table encodes that \emph{p}, so it is sufficient to count as
believing that \emph{p}. And it doesn't matter that this bet isn't on
the table. I'm disposed to use this table, so that's all that matters.

But might there be problems in the other direction. What about an agent
who, if offered such a bet on \emph{D}, would use such a simple table? I
simply say that they believe that \emph{D}. I would not use any such
table. I'd use this table.

\begin{longtable}[]{@{}lcc@{}}
\toprule\noalign{}
\endhead
\bottomrule\noalign{}
\endlastfoot
& \emph{D} & \(\neg D\) \\
Take bet & 1 & --1 \\
Decline bet & 0 & 0 \\
\end{longtable}

Now given the probability of \emph{D}, I'd still end up taking the bet;
it has an expected return of . (Well, actually I'd probably decline the
bet because being offered the bet would change the probability of
\emph{D} for reasons made clear in ~Runyon
(\citeproc{ref-RunyonGuysDolls}{1992, 14--15}). But that hardly
undermines the point I'm making.) But this isn't some analytic fact
about me, or even I think some respect in which I'm obeying the dictates
of rationality. It's simply a fact that I wouldn't take \emph{D} for
granted in any inquiry. And that's what my non-belief that \emph{D}
consists in.

This way of responding to the Tigers example helps respond to a nice
observation that Ross and Schroeder make about correctness. A belief
that \emph{p} is, in some sense, \emph{incorrect} if \(\neg p\). It
isn't altogether clear how to capture this sense given a simple
reduction of beliefs to credences. I propose to capture it using this
idea that decision tables encode propositions. A table is incorrect if
it encodes something that's false. To believe something is, \emph{inter
alia}, to be disposed to use a table that encodes it. So if it is false,
it involves a disposition to do something incorrect.

It also helps capture Holton's observation that beliefs should be
resilient. If someone is disposed to use decision tables that encode
that \emph{p}, that disposition should be fairly resilient. And to the
extent that it is resilient, they will satisfy all the other clauses in
my preferred account of belief. So anyone who believes \emph{p} should
have a resilient belief that \emph{p}.

The last point is where I think my biggest disagreement with Ross and
Schroeder lies. They think it is very important that a theory of belief
vindicate a principle they call \textbf{Stability}.

\begin{quote}
\textbf{Stability}: A fully rational agent does not change her beliefs
purely in virtue of an evidentially irrelevant change in her credences
or preferences. (20)
\end{quote}

Here's the kind of case that is meant to motivate Stability, and show
that views like mine are in tension with it.

\begin{quote}
Suppose Stella is extremely confident that steel is stronger than
Styrofoam, but she's not so confident that she'd bet her life on this
proposition for the prospect of winning a penny. PCR implies,
implausibly, that if Stella were offered such a bet, she'd cease to
believe that steel is stronger than Styrofoam, since her credence would
cease to rationalize acting as if this proposition is true. (22)
\end{quote}

Ross and Schroeder's own view is that if Stella has a defeasible
disposition to treat as true the proposition that steel is stronger than
Styrofoam, that's enough for her to believe it. And that can be true if
the disposition is not only defeasible, but actually defeated in the
circumstances Stella is in. This all strikes me as just as implausible
as the failure of Stability. Let's go over its costs.

The following propositions are clearly not mutually consistent, so one
of them must be given up. We're assuming that Stella is facing, and
knows she is facing, a bet that pays a penny if steel is stronger than
Styrofoam, and costs her life if steel is not stronger than Styrofoam.

\begin{enumerate}
\def\labelenumi{\arabic{enumi}.}
\tightlist
\item
  Stella believes that steel is stronger than Styrofoam.
\item
  Stella believes that if steel is stronger than Styrofoam, she'll win a
  penny and lose nothing by taking the bet.
\item
  If 1 and 2 are true, and Stella considers the question of whether
  she'll win a penny and lose nothing by taking the bet, she'll believe
  that she'll win a penny and lose nothing by taking the bet.
\item
  Stella prefers winning a penny and losing nothing to getting nothing.
\item
  If Stella believes that she'll win a penny and lose nothing by taking
  the bet, and prefers winning a penny and losing nothing to getting
  nothing, she'll take the bet.
\item
  Stella won't take the bet.
\end{enumerate}

It's part of the setup of the problem that 2 and 4 are true. And it's
common ground that 6 is true, at least assuming that Stella is rational.
So we're left with 1, 3 and 5 as the possible candidates for falsehood.

Ross and Schroeder say that it's implausible to reject 1. After all,
Stella believed it a few minutes ago, and hasn't received any evidence
to the contrary. And I guess rejecting 1 isn't the most intuitive
philosophical conclusion I've ever drawn. But compare the alternatives!

If we reject 3, we must say that Stella will simply refuse to infer
\emph{r} from \emph{p}, \emph{q} and \((p \wedge q) \rightarrow r\). Now
it is notoriously hard to come up with a general principle for closure
of beliefs. But it is hard to see why this particular instance would
fail. And in any case, it's hard to see why Stella wouldn't have a
general, defeasible, disposition to conclude \emph{r} in this case, so
by Ross and Schroeder's own lights, it seems 3 should be acceptable.

That leaves 5. It seems on Ross and Schroeder's view, Stella simply must
violate a very basic principle of means-end reasoning. She desires
something, she believes that taking the bet will get that thing, and
come with no added costs. Yet, she refuses to take the bet. And she's
rational to do so! At this stage, I think I've lost what's meant to be
belief-like about their notion of belief. I certainly think attributing
this kind of practical incoherence to Stella is much less plausible than
attributing a failure of Stability to her.

Put another way, I don't think presenting Stability on its own as a
desideratum of a theory is exactly playing fair. The salient question
isn't whether we should accept or reject Stability. The salient question
is whether giving up Stability is a fair price to pay for saving basic
tenets of means-end rationality. And I think that it is. Perhaps there
will be some way of understanding cases like Stella's so that we don't
have to choose between theories of belief that violate Stability
constraints, and theories of belief that violate coherence constraints.
But I don't see one on offer, and I'm not sure what such a theory could
look like.

I have one more argument against Stability, but it does rest on somewhat
contentious premises. There's often a difference between the best
\emph{methodology} in an area, and the correct \emph{epistemology} of
that area. When that happens, it's possible that there is a good
methodological rule saying that if such-and-such happens, re-open a
certain inquiry. But that rule need not be epistemologically
significant. That is, it need not be the case that the happening of
such-and-such provides evidence against the conclusion of the inquiry.
It just provides a reason that a good researcher will re-open the
inquiry. And, as we've stated above, an open inquiry is incompatible
with belief.

Here's one way that might happen. Like other non-conciliationists about
disagreement, e.g., ~Kelly (\citeproc{ref-Kelly2010-KELPDA}{2010}), I
hold that disagreement by peers with the same evidence as you doesn't
provide \emph{evidence} that you are wrong. But it might provide an
excellent reason to re-open an inquiry. We shouldn't draw conclusions
about the methodological significance of disagreement from the
epistemology of disagreement. So learning that your peers all disagree
with a conclusion might be a reason to re-open inquiry into that
conclusion, and hence lose belief in the conclusion, without providing
evidence that the conclusion is false. This example rests on a very
contentious claim about the epistemology of disagreement. But any gap
that opens up between methodology and epistemology will allow such an
example to be constructed, and hence provide an independent reason to
reject Stability.

\section{Conclusion}\label{conclusion}

You might well worry that the view here is too \emph{complex} to really
be a theory of belief. Belief is a simple state; why all the epicycles?
This is a good question, and I'm not sure I have a sufficiently good
answer to it.

At heart, the theory I've offered here is simple. To believe \emph{p} is
to take \emph{p} for granted, to take it as given, to take it as a
settled question. But one doesn't take a question as settled in a
vacuum. I will take some questions as settled in some circumstances and
not others. It's here that the complexities enter in.

To believe \emph{p}, it isn't necessary that we take it as settled in
all contexts. That would mean that anything one believes one would bet
on at any odds. But it isn't sufficient to take it as settled in some
context or other. If I'm facing a tricky bet on \emph{p}, the fact that
I'd take \emph{p} as settled in some other context doesn't mean that I
believe \emph{p}. After all, I might even decline the bet, although I
desire the reward for winning the bet, and believe that if \emph{p} I
will win. And we can't just focus on the actual circumstances. Five
minutes ago, I neither took it as settled or as open that the Cubs
haven't won the World Series for quite a while. I simply wasn't thinking
about that proposition, and didn't really take it to be one thing or
another.

This is why things get so complex. To believe \emph{p} is to hold a
fairly simple attitude towards \emph{p} in some relevant circumstances.
But which circumstances? That's what's hard to say, and it's why the
theory is so messy. And I think we have an argument that it must be a
little hard to say, namely an argument by exhaustion of all the possible
simple things to say. The previous paragraph starts such an argument.

I'd be a little surprised if the account here is the best or last word
on the matter though. It does feel a little disjunctive, as if there is
a simpler reduction to be had. But I think it's better than what came
before, so I'm putting it forward.

The previous version of the theory I put forward was clearly reductive;
beliefs were reduced to credences and preferences. This version is not
quite as clearly reductive. Which decision tables the agent is disposed
to use, and which propositions those tables encode, are not obviously
facts about credences and preferences. So it feels like I've given up on
the reductive project.

I'm not altogether happy about this; reduction is a good aim to have.
But if reduction of belief to other states fails, I'd think this kind of
reason is why it is going to fail. Facts about how an agent
conceptualises a problem, how she sets up the decision table, are
distinct from facts about which values she writes into the table. This
is the deepest reason why the Lockean theory is false. Belief is not the
difference between one column in the decision table getting probability
0.98 rather than 0.97; it is the difference between one column being
excluded rather than included. If that difference can't be accounted for
in terms of actual credences and preferences, the reductionist project
will fail.

\subsection*{References}\label{references}
\addcontentsline{toc}{subsection}{References}

\phantomsection\label{refs}
\begin{CSLReferences}{1}{0}
\bibitem[\citeproctext]{ref-Aumann1999}
Aumann, Robert J. 1999. {``Interactive Epistemology i: Knowledge.''}
\emph{International Journal of Game Theory} 28 (3): 263--300. doi:
\href{https://doi.org/10.1007/s001820050111}{10.1007/s001820050111}.

\bibitem[\citeproctext]{ref-Binmore2007}
Binmore, Ken. 2007. \emph{Playing for Real: A Text on Game Theory}.
Oxford: Oxford University Press.

\bibitem[\citeproctext]{ref-Block1978}
Block, Ned. 1978. {``Troubles with Functionalism.''} \emph{Minnesota
Studies in the Philosophy of Science} 9: 261--325.

\bibitem[\citeproctext]{ref-DBMJackson2007}
Braddon-Mitchell, David, and Frank Jackson. 2007. \emph{The Philosophy
of Mind and Cognition, {Second Edition}}. Malden, MA: Blackwell.

\bibitem[\citeproctext]{ref-Brown2013}
Brown, Jessica. 2014. {``Impurism, Practical Reasoning and the Threshold
Problem.''} \emph{No{û}s} 48 (1): 179--92. doi:
\href{https://doi.org/10.1111/nous.12008}{10.1111/nous.12008}.

\bibitem[\citeproctext]{ref-Christensen2005}
Christensen, David. 2005. \emph{Putting Logic in Its Place}. Oxford:
Oxford University Press.

\bibitem[\citeproctext]{ref-Cohen1999}
Cohen, Stewart. 1999. {``Contextualism, Skepticism, and the Structure of
Reasons.''} \emph{Philosophical Perspectives} 13: 57--89. doi:
\href{https://doi.org/10.1111/0029-4624.33.s13.3}{10.1111/0029-4624.33.s13.3}.

\bibitem[\citeproctext]{ref-DeFinetti1964}
DeFinetti, Bruno. 1964. {``Foresight: Its Logical Laws, Its Subjective
Sources.''} In \emph{Studies in Subjective Probability}, edited by Henry
E. Kyburg and Howard E. Smokler, 93--156. New York: Wiley. First
published as {``La pr{é}vision : ses lois logiques, ses sources
subjectives,''} in \emph{Annals de l'institut Henri Poincar{é}}, vol 7,
number 1, 1937, pp 1-68.

\bibitem[\citeproctext]{ref-DeRose1992}
DeRose, Keith. 1992. {``Contextualism and Knowledge Attributions.''}
\emph{Philosophy and Phenomenological Research} 52 (4): 913--29. doi:
\href{https://doi.org/10.2307/2107917}{10.2307/2107917}.

\bibitem[\citeproctext]{ref-DixitSkeath2004}
Dixit, Avinash K., and Susan Skeath. 2004. \emph{Games of Strategy}.
Second. New York: W. W. Norton \& Company.

\bibitem[\citeproctext]{ref-Fantl2002}
Fantl, Jeremy, and Matthew McGrath. 2002. {``Evidence, Pragmatics, and
Justification.''} \emph{Philosophical Review} 111: 67--94. doi:
\href{https://doi.org/10.2307/3182570}{10.2307/3182570}.

\bibitem[\citeproctext]{ref-vanFraassen1989}
Fraassen, Bas van. 1989. \emph{Laws and Symmetry}. Oxford: Clarendon
Press.

\bibitem[\citeproctext]{ref-Friend2003}
Friend, Stacie. 2003. {``How i Really Feel about \emph{JFK}.''} In
\emph{Imagination, Philosophy and the Arts}, edited by Matthew Kieran
and Dominic McIver Lopes, 35--53. London. Routledge.

\bibitem[\citeproctext]{ref-Gillies2010}
Gillies, Anthony S. 2010. {``Iffiness.''} \emph{Semantics and
Pragmatics} 3 (4): 1--42. doi:
\href{https://doi.org/10.3765/sp.3.4}{10.3765/sp.3.4}.

\bibitem[\citeproctext]{ref-Heal1994}
Heal, Jane. 1994. {``Moore's Paradox: A Wittgensteinian Approach.''}
\emph{Mind} 103 (409): 5--24. doi:
\href{https://doi.org/10.1093/mind/103.409.5}{10.1093/mind/103.409.5}.

\bibitem[\citeproctext]{ref-Holton2013}
Holton, Richard. 2014. {``Intention as a Model for Belief.''} In
\emph{Rational and Social Agency: Essays on the Philosophy of Michael
Bratman}, edited by Manuel Vargas and Gideon Yaffe, 12--37. Oxford:
Oxford University Press.

\bibitem[\citeproctext]{ref-IchikawaJarvis2009}
Ichikawa, Jonathan, and Benjamin Jarvis. 2009. {``Thought-Experiment
Intuitions and Truth in Fiction.''} \emph{Philosophical Studies} 142
(2): 221--46. doi:
\href{https://doi.org/10.1007/s11098-007-9184-y}{10.1007/s11098-007-9184-y}.

\bibitem[\citeproctext]{ref-Jeffrey1983}
Jeffrey, Richard. 1983. {``Bayesianism with a Human Face.''} In
\emph{Testing Scientific Theories}, edited by J. Earman (ed.).
Minneapolis: University of Minnesota Press.

\bibitem[\citeproctext]{ref-Kelly2010-KELPDA}
Kelly, Thomas. 2010. {``Peer Disagreement and Higher Order Evidence.''}
In \emph{Disagreement}, edited by Ted Warfield and Richard Feldman,
111--74. Oxford: Oxford University Press.

\bibitem[\citeproctext]{ref-KohlbergMertens1986}
Kohlberg, Elon, and Jean-Francois Mertens. 1986. {``On the Strategic
Stability of Equilibria.''} \emph{Econometrica} 54 (5): 1003--37. doi:
\href{https://doi.org/10.2307/1912320}{10.2307/1912320}.

\bibitem[\citeproctext]{ref-Kratzer2012}
Kratzer, Angelika. 2012. \emph{Modals and Conditionals}. Oxford: Oxford
University Press.

\bibitem[\citeproctext]{ref-Levi1974}
Levi, Isaac. 1974. {``On Indeterminate Probabilities.''} \emph{Journal
of Philosophy} 71 (13): 391--418. doi:
\href{https://doi.org/10.2307/2025161}{10.2307/2025161}.

\bibitem[\citeproctext]{ref-MacFarlane2011}
MacFarlane, John. 2011. {``Epistemic Modals Are Assessment-Sensitive.''}
In \emph{Epistemic Modality}, edited by Andy Egan and Brian Weatherson,
144--78. Oxford: Oxford University Press.

\bibitem[\citeproctext]{ref-MaitraWeatherson}
Maitra, Ishani, and Brian Weatherson. 2010. {``Assertion, Knowledge and
Action.''} \emph{Philosophical Studies} 149 (1): 99--118. doi:
\href{https://doi.org/10.1007/s11098-010-9542-z}{10.1007/s11098-010-9542-z}.

\bibitem[\citeproctext]{ref-Malmgren2011}
Malmgren, Anna-Sara. 2011. {``Rationalism and the Content of Intuitive
Judgements.''} \emph{Mind} 120 (478): 263--327. doi:
\href{https://doi.org/10.1093/mind/fzr039}{10.1093/mind/fzr039}.

\bibitem[\citeproctext]{ref-MossPragmaticsEpistemicModals}
Moss, Sarah. 2015. {``On the Semantics and Pragmatics of Epistemic
Vocabulary.''} \emph{Semantics and Pragmatics} 8: 1--81. doi:
\href{https://doi.org/10.3765/sp.8.5}{10.3765/sp.8.5}.

\bibitem[\citeproctext]{ref-Nagel2008}
Nagel, Jennifer. 2008. {``Knowledge Ascriptions and the Psychological
Consequences of Changing Stakes.''} \emph{Australasian Journal of
Philosophy} 86 (2): 279--94. doi:
\href{https://doi.org/10.1080/00048400801886397}{10.1080/00048400801886397}.

\bibitem[\citeproctext]{ref-Nagel2013-Williamson}
---------. 2013. {``Motivating Williamson's Model Gettier Cases.''}
\emph{Inquiry} 56 (1): 54--62. doi:
\href{https://doi.org/10.1080/0020174X.2013.775014}{10.1080/0020174X.2013.775014}.

\bibitem[\citeproctext]{ref-North2010}
North, Jill. 2010. {``An Empirical Approach to Symmetry and
Probability.''} \emph{Studies In History and Philosophy of Science Part
B: Studies In History and Philosophy of Modern Physics} 41 (1): 27--40.
doi:
\href{https://doi.org/10.1016/j.shpsb.2009.08.008}{10.1016/j.shpsb.2009.08.008}.

\bibitem[\citeproctext]{ref-Ramsey1929}
Ramsey, Frank. 1929/1990. {``Probability and Partial Belief.''} In
\emph{Philosophical Papers}, edited by D. H. Mellor, 95--96. Cambridge
University Press.

\bibitem[\citeproctext]{ref-RamseyTruthProb}
---------. 1926. {``Truth and Probability.''} In \emph{Philosophical
Papers}, edited by D. H. Mellor, 52--94. Cambridge: Cambridge University
Press.

\bibitem[\citeproctext]{ref-SchroederRoss2012}
Ross, Jacob, and Mark Schroeder. 2014. {``Belief, Credence, and
Pragmatic Encroachment.''} \emph{Philosophy and Phenomenological
Research} 88 (2): 259--88. doi:
\href{https://doi.org/10.1111/j.1933-1592.2011.00552.x}{10.1111/j.1933-1592.2011.00552.x}.

\bibitem[\citeproctext]{ref-RunyonGuysDolls}
Runyon, Damon. 1992. \emph{Guys \& Dolls: The Stories of {D}amon
{R}unyon}. New York: Penguin.

\bibitem[\citeproctext]{ref-Schoenfield2013}
Schoenfield, Miriam. 2013. {``Permission to Believe: Why Permissivism Is
True and What It Tells Us about Irrelevant Influences on Belief.''}
\emph{No{û}s} 47 (1): 193--218. doi:
\href{https://doi.org/10.1111/nous.12006}{10.1111/nous.12006}.

\bibitem[\citeproctext]{ref-Sorensen1988}
Sorensen, Roy A. 1988. \emph{Blindspots}. Oxford: Clarendon Press.

\bibitem[\citeproctext]{ref-Stalnaker1994}
Stalnaker, Robert. 1994. {``On the Evaluation of Solution Concepts.''}
\emph{Theory and Decision} 37 (1): 49--73. doi:
\href{https://doi.org/10.1007/BF01079205}{10.1007/BF01079205}.

\bibitem[\citeproctext]{ref-Stalnaker1996}
---------. 1996. {``Knowledge, Belief and Counterfactual Reasoning in
Games.''} \emph{Economics and Philosophy} 12: 133--63. doi:
\href{https://doi.org/10.1017/S0266267100004132}{10.1017/S0266267100004132}.

\bibitem[\citeproctext]{ref-Stalnaker1998}
---------. 1998. {``Belief Revision in Games: Forward and Backward
Induction.''} \emph{Mathematical Social Sciences} 36 (1): 31--56. doi:
\href{https://doi.org/10.1016/S0165-4896(98)00007-9}{10.1016/S0165-4896(98)00007-9}.

\bibitem[\citeproctext]{ref-Stalnaker1999}
---------. 1999. {``Extensive and Strategic Forms: Games and Models for
Games.''} \emph{Research in Economics} 53 (3): 293--319. doi:
\href{https://doi.org/10.1006/reec.1999.0200}{10.1006/reec.1999.0200}.

\bibitem[\citeproctext]{ref-Stephenson2007}
Stephenson, Tamina. 2007. {``Judge Dependence, Epistemic Modals, and
Predicates of Personal Taste.''} \emph{Linguistics and Philosophy} 30
(4): 487--525. doi:
\href{https://doi.org/10.1007/s10988-008-9023-4}{10.1007/s10988-008-9023-4}.

\bibitem[\citeproctext]{ref-Walley1991}
Walley, Peter. 1991. \emph{Statisical Reasoning with Imprecise
Probabilities}. London: Chapman \& Hall.

\bibitem[\citeproctext]{ref-Weatherson2005-WEACWD}
Weatherson, Brian. 2005. {``{Can We Do Without Pragmatic
Encroachment?}''} \emph{Philosophical Perspectives} 19 (1): 417--43.
doi:
\href{https://doi.org/10.1111/j.1520-8583.2005.00068.x}{10.1111/j.1520-8583.2005.00068.x}.

\bibitem[\citeproctext]{ref-Weatherson2011-WEADIR}
---------. 2011. {``Defending Interest-Relative Invariantism.''}
\emph{Logos \& Episteme} 2 (4): 591--609. doi:
\href{https://doi.org/10.5840/logos-episteme2011248}{10.5840/logos-episteme2011248}.

\bibitem[\citeproctext]{ref-Weatherson2012-WEAGAT}
---------. 2012a. {``Games and the Reason-Knowledge Principle.''}
\emph{The Reasoner} 6 (1): 6--8.

\bibitem[\citeproctext]{ref-Weatherson2012-WEAKBI}
---------. 2012b. {``Knowledge, Bets and Interests.''} In
\emph{Knowledge Ascriptions}, edited by Jessica Brown and Mikkel Gerken,
75--103. Oxford: Oxford University Press.

\bibitem[\citeproctext]{ref-Weintraub2008}
Weintraub, Ruth. 2008. {``How Probable Is an Infinite Sequence of Heads?
A Reply to Williamson.''} \emph{Analysis} 68 (3): 247--50. doi:
\href{https://doi.org/10.1093/analys/68.3.247}{10.1093/analys/68.3.247}.

\bibitem[\citeproctext]{ref-White2005-WHIEP}
White, Roger. 2005. {``Epistemic Permissiveness.''} \emph{Philosophical
Perspectives} 19: 445--59. doi:
\href{https://doi.org/10.1111/j.1520-8583.2005.00069.x}{10.1111/j.1520-8583.2005.00069.x}.

\bibitem[\citeproctext]{ref-Williamson1996-WILKAA}
Williamson, Timothy. 1996. {``{Knowing and Asserting}.''}
\emph{Philosophical Review} 105 (4): 489--523. doi:
\href{https://doi.org/10.2307/2998423}{10.2307/2998423}.

\bibitem[\citeproctext]{ref-Williamson2007-WILHPI}
---------. 2007a. {``How Probable Is an Infinite Sequence of Heads?''}
\emph{Analysis} 67 (295): 173--80. doi:
\href{https://doi.org/10.1111/j.1467-8284.2007.00671.x}{10.1111/j.1467-8284.2007.00671.x}.

\bibitem[\citeproctext]{ref-Williamson2007-WILTPO-17}
---------. 2007b. \emph{{The Philosophy of Philosophy}}. Blackwell.

\bibitem[\citeproctext]{ref-WilliamsonLofoten}
---------. 2013. {``Gettier Cases in Epistemic Logic.''} \emph{Inquiry}
56 (1): 1--14. doi:
\href{https://doi.org/10.1080/0020174X.2013.775010}{10.1080/0020174X.2013.775010}.

\bibitem[\citeproctext]{ref-Yalcin2011}
Yalcin, Seth. 2011. {``Nonfactualism about Epistemic Modality.''} In
\emph{Epistemic Modality}, edited by Andy Egan and Brian Weatherson,
295--332. Oxford: Oxford University Press.

\end{CSLReferences}



\noindent Published in\emph{
Philosophy and Phenomenological Research}, 2016, pp. 209-236.

\end{document}
