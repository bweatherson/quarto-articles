% Options for packages loaded elsewhere
\PassOptionsToPackage{unicode}{hyperref}
\PassOptionsToPackage{hyphens}{url}
%
\documentclass[
  10pt,
  letterpaper,
  DIV=11,
  numbers=noendperiod,
  twoside]{scrartcl}

\usepackage{amsmath,amssymb}
\usepackage{setspace}
\usepackage{iftex}
\ifPDFTeX
  \usepackage[T1]{fontenc}
  \usepackage[utf8]{inputenc}
  \usepackage{textcomp} % provide euro and other symbols
\else % if luatex or xetex
  \usepackage{unicode-math}
  \defaultfontfeatures{Scale=MatchLowercase}
  \defaultfontfeatures[\rmfamily]{Ligatures=TeX,Scale=1}
\fi
\usepackage{lmodern}
\ifPDFTeX\else  
    % xetex/luatex font selection
    \setmainfont[ItalicFont=EB Garamond Italic,BoldFont=EB Garamond
Bold]{EB Garamond Math}
    \setsansfont[]{Europa-Bold}
  \setmathfont[]{Garamond-Math}
\fi
% Use upquote if available, for straight quotes in verbatim environments
\IfFileExists{upquote.sty}{\usepackage{upquote}}{}
\IfFileExists{microtype.sty}{% use microtype if available
  \usepackage[]{microtype}
  \UseMicrotypeSet[protrusion]{basicmath} % disable protrusion for tt fonts
}{}
\usepackage{xcolor}
\usepackage[left=1in, right=1in, top=0.8in, bottom=0.8in,
paperheight=9.5in, paperwidth=6.5in, includemp=TRUE, marginparwidth=0in,
marginparsep=0in]{geometry}
\setlength{\emergencystretch}{3em} % prevent overfull lines
\setcounter{secnumdepth}{3}
% Make \paragraph and \subparagraph free-standing
\makeatletter
\ifx\paragraph\undefined\else
  \let\oldparagraph\paragraph
  \renewcommand{\paragraph}{
    \@ifstar
      \xxxParagraphStar
      \xxxParagraphNoStar
  }
  \newcommand{\xxxParagraphStar}[1]{\oldparagraph*{#1}\mbox{}}
  \newcommand{\xxxParagraphNoStar}[1]{\oldparagraph{#1}\mbox{}}
\fi
\ifx\subparagraph\undefined\else
  \let\oldsubparagraph\subparagraph
  \renewcommand{\subparagraph}{
    \@ifstar
      \xxxSubParagraphStar
      \xxxSubParagraphNoStar
  }
  \newcommand{\xxxSubParagraphStar}[1]{\oldsubparagraph*{#1}\mbox{}}
  \newcommand{\xxxSubParagraphNoStar}[1]{\oldsubparagraph{#1}\mbox{}}
\fi
\makeatother


\providecommand{\tightlist}{%
  \setlength{\itemsep}{0pt}\setlength{\parskip}{0pt}}\usepackage{longtable,booktabs,array}
\usepackage{calc} % for calculating minipage widths
% Correct order of tables after \paragraph or \subparagraph
\usepackage{etoolbox}
\makeatletter
\patchcmd\longtable{\par}{\if@noskipsec\mbox{}\fi\par}{}{}
\makeatother
% Allow footnotes in longtable head/foot
\IfFileExists{footnotehyper.sty}{\usepackage{footnotehyper}}{\usepackage{footnote}}
\makesavenoteenv{longtable}
\usepackage{graphicx}
\makeatletter
\def\maxwidth{\ifdim\Gin@nat@width>\linewidth\linewidth\else\Gin@nat@width\fi}
\def\maxheight{\ifdim\Gin@nat@height>\textheight\textheight\else\Gin@nat@height\fi}
\makeatother
% Scale images if necessary, so that they will not overflow the page
% margins by default, and it is still possible to overwrite the defaults
% using explicit options in \includegraphics[width, height, ...]{}
\setkeys{Gin}{width=\maxwidth,height=\maxheight,keepaspectratio}
% Set default figure placement to htbp
\makeatletter
\def\fps@figure{htbp}
\makeatother
% definitions for citeproc citations
\NewDocumentCommand\citeproctext{}{}
\NewDocumentCommand\citeproc{mm}{%
  \begingroup\def\citeproctext{#2}\cite{#1}\endgroup}
\makeatletter
 % allow citations to break across lines
 \let\@cite@ofmt\@firstofone
 % avoid brackets around text for \cite:
 \def\@biblabel#1{}
 \def\@cite#1#2{{#1\if@tempswa , #2\fi}}
\makeatother
\newlength{\cslhangindent}
\setlength{\cslhangindent}{1.5em}
\newlength{\csllabelwidth}
\setlength{\csllabelwidth}{3em}
\newenvironment{CSLReferences}[2] % #1 hanging-indent, #2 entry-spacing
 {\begin{list}{}{%
  \setlength{\itemindent}{0pt}
  \setlength{\leftmargin}{0pt}
  \setlength{\parsep}{0pt}
  % turn on hanging indent if param 1 is 1
  \ifodd #1
   \setlength{\leftmargin}{\cslhangindent}
   \setlength{\itemindent}{-1\cslhangindent}
  \fi
  % set entry spacing
  \setlength{\itemsep}{#2\baselineskip}}}
 {\end{list}}
\usepackage{calc}
\newcommand{\CSLBlock}[1]{\hfill\break\parbox[t]{\linewidth}{\strut\ignorespaces#1\strut}}
\newcommand{\CSLLeftMargin}[1]{\parbox[t]{\csllabelwidth}{\strut#1\strut}}
\newcommand{\CSLRightInline}[1]{\parbox[t]{\linewidth - \csllabelwidth}{\strut#1\strut}}
\newcommand{\CSLIndent}[1]{\hspace{\cslhangindent}#1}

\setlength\heavyrulewidth{0ex}
\setlength\lightrulewidth{0ex}
\usepackage[automark]{scrlayer-scrpage}
\clearpairofpagestyles
\cehead{
  Brian Weatherson
  }
\cohead{
  Trends in Philosophical Studies
  }
\ohead{\bfseries \pagemark}
\cfoot{}
\makeatletter
\newcommand*\NoIndentAfterEnv[1]{%
  \AfterEndEnvironment{#1}{\par\@afterindentfalse\@afterheading}}
\makeatother
\NoIndentAfterEnv{itemize}
\NoIndentAfterEnv{enumerate}
\NoIndentAfterEnv{description}
\NoIndentAfterEnv{quote}
\NoIndentAfterEnv{equation}
\NoIndentAfterEnv{longtable}
\NoIndentAfterEnv{abstract}
\renewenvironment{abstract}
 {\vspace{-1.25cm}
 \quotation\small\noindent\rule{\linewidth}{.5pt}\par\smallskip
 \noindent }
 {\par\noindent\rule{\linewidth}{.5pt}\endquotation}
\setkomafont{descriptionlabel}{\normalfont\scshape\bfseries}
\KOMAoption{captions}{tableheading}
\makeatletter
\@ifpackageloaded{caption}{}{\usepackage{caption}}
\AtBeginDocument{%
\ifdefined\contentsname
  \renewcommand*\contentsname{Table of contents}
\else
  \newcommand\contentsname{Table of contents}
\fi
\ifdefined\listfigurename
  \renewcommand*\listfigurename{List of Figures}
\else
  \newcommand\listfigurename{List of Figures}
\fi
\ifdefined\listtablename
  \renewcommand*\listtablename{List of Tables}
\else
  \newcommand\listtablename{List of Tables}
\fi
\ifdefined\figurename
  \renewcommand*\figurename{Figure}
\else
  \newcommand\figurename{Figure}
\fi
\ifdefined\tablename
  \renewcommand*\tablename{Table}
\else
  \newcommand\tablename{Table}
\fi
}
\@ifpackageloaded{float}{}{\usepackage{float}}
\floatstyle{ruled}
\@ifundefined{c@chapter}{\newfloat{codelisting}{h}{lop}}{\newfloat{codelisting}{h}{lop}[chapter]}
\floatname{codelisting}{Listing}
\newcommand*\listoflistings{\listof{codelisting}{List of Listings}}
\makeatother
\makeatletter
\makeatother
\makeatletter
\@ifpackageloaded{caption}{}{\usepackage{caption}}
\@ifpackageloaded{subcaption}{}{\usepackage{subcaption}}
\makeatother

\ifLuaTeX
  \usepackage{selnolig}  % disable illegal ligatures
\fi
\usepackage{bookmark}

\IfFileExists{xurl.sty}{\usepackage{xurl}}{} % add URL line breaks if available
\urlstyle{same} % disable monospaced font for URLs
\hypersetup{
  pdftitle={Trends in Philosophical Studies},
  pdfauthor={Brian Weatherson},
  hidelinks,
  pdfcreator={LaTeX via pandoc}}


\title{Trends in \emph{Philosophical Studies}}
\author{Brian Weatherson}
\date{2024}

\begin{document}
\maketitle
\begin{abstract}
\emph{Philosophical Studies} has become one of the most important
journals for work in several large topics in philosophy. This paper uses
data from the word distributions in those papers, and the citations of
the papers, to look at how it has changed over time, and how it became
so central.
\end{abstract}


\setstretch{1.1}
\section{Intro}\label{intro}

TKTK

\section{Sources}\label{sec-sources}

This article is primarily based on data-driven analysis of articles
published in philosophy journals, and in particular in
\emph{Philosophical Studies}, from 1980 to 2019. The studies here are
primarily based on two sources: citation data from Web of Science, and
word lists from JSTOR.

Through {[}University X{]} I downloaded the Web of Science (hereafter,
WoS) Core Collection in XML format. Within it, I selected 100 prominent
philosophy journals that WoS indexes. The journals I selected are, like
\emph{Philosophical Studies} primarily English-language, analytic
philosophy journals. I filtered the citations for just citations from
and to those 100 journals. So what we're working with is citations of
philosophy journals in philosophy journals.

This is obviously a small subset of all citations. It excludes citations
in academic journals in other fields, in books and edited volumes, and
in many other places that Google Scholar indexes, such as dissertations,
lecture notes, slides, and draft manuscripts. Losing that information is
a cost, but there are three large upsides. First, these citations are
much more accurate; I haven't found any false positives when using this
filtered set, and only a handful of false negatives. Second, we can be
more confident that our data set is relatively complete; finding a full
list of philosophy journals is easier than finding a full list of edited
volumes in philosophy. Third, by looking at citations internal to
philosophy, we can get a sense of philosophy's self-image, and how it
changes over time.

There are a number of delicate methodological issues I've elided in that
last paragraph; I'll go over these in more detail at the end in
\textbf{?@sec-methodology}.

The downloadable citation data is not particularly up to date. I am
including citations beyond 2019, because it's helpful to get a sense of
how some of these articles have been received in more recent years. But
the data I have only goes through mid-2022. I'll often simply say 2022;
but note that even that year is incomplete.

The other source I used is JSTOR, and in particular the Data for
Research (DfR) program that they provide through their Constellate
project. This lets you download lists of the words used in various
journal articles, along with a count of how often each word is
used.\footnote{It also provides bigrams and trigrams, which I've
  occasionally used.} It also provides word counts for the articles,
which I have used in Section~\ref{sec-overview}. The words an author
uses are a pretty good guide to what they are talking about; if the word
`denotation' is used frequently, it's probably a philosophy of language
article.

\section{Overview}\label{sec-overview}

Articles in \emph{Philosophical Studies} get cited a lot. Our first
study is simply a count of citations in the 100 journals to articles
published in the 100 journals between 1980 and 2019.
Table~\ref{tbl-all-cites} shows the five journals with the largest
number of citations.

\emph{Philosophical Studies} is in first place on that list in part
because it publishes so much. Table~\ref{tbl-all-articles} lists the top
five journals by the number of articles they have published.

Just publishing a lot of articles is no guarantee of having a lot of
citations. TKTK. This makes sense. Citations tend to follow something
like a log-normal distribution
(\citeproc{ref-Brzezinski2015}{\textbf{Brzezinski2015?}}). The bulk of
the citations come from a handful of highly cited articles. Publishing
more articles

\section{Articles}\label{articles}

Notes - including things that JSTOR counts as book reviews Not including
things that WOS counts as discussions This is, to put it mildly, a bit
random; some precis are included, some aren't; some replies are
included, some aren't. In general if the word `comment' appears in the
title excluded, but it's a bit random. But going with them seems best.
Alternative is to work through 100 other journals if we wanted an
apples-to-apples comparison. Missing the supplement in 2013, which I
don't understand, and was coded differently in the two databases

\section{Studies}\label{studies}

Number of articles Length of articles Number of articles over 10K, 15K,
20K Average citations over the 100 journals - boring because can't
compare Average citations compared to top 25 - more interesting Maybe
redo that graph as a ratio

\section{Comparisons}\label{comparisons}

Articles with highest ratio of PS cites/general cites Articles with
lowest ratio of PS cites/general cites Previously I just did that with
articles published in PS; that seems wrong

Hmm, this could be complicated. Study: For each decade find 20 most
cited, and look at how often they are cited in PS.

For 1980s, the low is 0 for Rawls (1980) and Dennett (1971). But Rawls
is cited in Post (1984), under a variant name, and Dennett is cited in
Lormond (1985), but to the reprint. Maybe just live with the messiness
High five in 1980s are all language, or language-adjacent: Lewis (1979)
(both de se and time's arrow), two Perry papers, and Donnellan

For 1990s, the low is 0 for McDowell's Virtue and Reason. Is in
Brighouse (1990), but the citation is incomplete and uses the wrong
year. Oddly the third lowest is for Discrimination and Perceptual
Knowledge, only 2/38. And this looks right. The highs in 1990s are still
language based, though with metaphysics seeping in (Kim on
supervenience, Lewis on Universals)

For 2000s, lows include Cummins (1975) (which is hard to track, such a
common name), Anderson (1999), and Mechanisms paper (2000). The highs
are very epistemology based - Goldman is among them, and the two big
contextualism papers by DeRose and Lewis. Obviously the editor at the
time was also a prominent contextualist, but his papers are not cited in
the journal at a particularly out of the ordinary rate. (There are many
cases in history where editors are cited a lot, but this isn't one of
them.) Just outside top 20, but Jackson and Chalmers has nearly 1/3 its
cites in PS. Tracing back to Davies and Humberstone's original paper.
Remarkably, Elusive Knowledge is most cited in the decade; normally this
kind of calculation pushes towards less cited, high variance cases.

For 2010s, still mechanisms, Anderson (1999), also Kripke (1975) are the
low ones Top 5 is more varied, both in topic and time: Kolodny (2005),
Pryor (2000), Lewis (1986), Frankfurt (1969), Jackson (1982). Could be
result of publishing more papers, but striking that it's pulling these
older papers in.

\section{The LDA}\label{the-lda}

Build the model Note the five categories Graph the trends Flag the
methodology

\section{Language}\label{language}

Find the 40 most cited in each topic over the 40 years (i.e., most cited
per year) Look how often they are cited in PS Look how often they are
cited across the 100 See if the trends in PS track wider trends Find
other journals that have similar trends (Analysis, PQ, AJP)

\section{Metaphysics}\label{metaphysics}

Note the two Schaffer papers, and differences in citing Schaffer
(\citeproc{ref-WOS000368189400004}{2016}) Is that because little
grounding There is some - see graph of words Maybe just count how
`modal' vs `postmodal' the 2019 papers are Maybe do a small LDA of the
metaphysics papers

\section{Ethics}\label{ethics}

Is there anything to say here? Does it go more political? Still not
citing Anderson, but something?

\phantomsection\label{refs}
\begin{CSLReferences}{1}{0}
\bibitem[\citeproctext]{ref-WOS000368189400004}
Schaffer, Jonathan. 2016. {``Grounding in the Image of Causation.''}
\emph{Philosophical Studies} 173 (1): 49--100.

\end{CSLReferences}



\noindent Published online in October 2024.


\end{document}
