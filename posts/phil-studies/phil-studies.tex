% Options for packages loaded elsewhere
\PassOptionsToPackage{unicode}{hyperref}
\PassOptionsToPackage{hyphens}{url}
%
\documentclass[
  10pt,
  letterpaper,
  DIV=11,
  numbers=noendperiod,
  twoside]{scrartcl}

\usepackage{amsmath,amssymb}
\usepackage{setspace}
\usepackage{iftex}
\ifPDFTeX
  \usepackage[T1]{fontenc}
  \usepackage[utf8]{inputenc}
  \usepackage{textcomp} % provide euro and other symbols
\else % if luatex or xetex
  \usepackage{unicode-math}
  \defaultfontfeatures{Scale=MatchLowercase}
  \defaultfontfeatures[\rmfamily]{Ligatures=TeX,Scale=1}
\fi
\usepackage{lmodern}
\ifPDFTeX\else  
    % xetex/luatex font selection
    \setmainfont[ItalicFont=EB Garamond Italic,BoldFont=EB Garamond
Bold]{EB Garamond Math}
    \setsansfont[]{Europa-Bold}
  \setmathfont[]{Garamond-Math}
\fi
% Use upquote if available, for straight quotes in verbatim environments
\IfFileExists{upquote.sty}{\usepackage{upquote}}{}
\IfFileExists{microtype.sty}{% use microtype if available
  \usepackage[]{microtype}
  \UseMicrotypeSet[protrusion]{basicmath} % disable protrusion for tt fonts
}{}
\usepackage{xcolor}
\usepackage[left=1in, right=1in, top=0.8in, bottom=0.8in,
paperheight=9.5in, paperwidth=6.5in, includemp=TRUE, marginparwidth=0in,
marginparsep=0in]{geometry}
\setlength{\emergencystretch}{3em} % prevent overfull lines
\setcounter{secnumdepth}{3}
% Make \paragraph and \subparagraph free-standing
\makeatletter
\ifx\paragraph\undefined\else
  \let\oldparagraph\paragraph
  \renewcommand{\paragraph}{
    \@ifstar
      \xxxParagraphStar
      \xxxParagraphNoStar
  }
  \newcommand{\xxxParagraphStar}[1]{\oldparagraph*{#1}\mbox{}}
  \newcommand{\xxxParagraphNoStar}[1]{\oldparagraph{#1}\mbox{}}
\fi
\ifx\subparagraph\undefined\else
  \let\oldsubparagraph\subparagraph
  \renewcommand{\subparagraph}{
    \@ifstar
      \xxxSubParagraphStar
      \xxxSubParagraphNoStar
  }
  \newcommand{\xxxSubParagraphStar}[1]{\oldsubparagraph*{#1}\mbox{}}
  \newcommand{\xxxSubParagraphNoStar}[1]{\oldsubparagraph{#1}\mbox{}}
\fi
\makeatother


\providecommand{\tightlist}{%
  \setlength{\itemsep}{0pt}\setlength{\parskip}{0pt}}\usepackage{longtable,booktabs,array}
\usepackage{calc} % for calculating minipage widths
% Correct order of tables after \paragraph or \subparagraph
\usepackage{etoolbox}
\makeatletter
\patchcmd\longtable{\par}{\if@noskipsec\mbox{}\fi\par}{}{}
\makeatother
% Allow footnotes in longtable head/foot
\IfFileExists{footnotehyper.sty}{\usepackage{footnotehyper}}{\usepackage{footnote}}
\makesavenoteenv{longtable}
\usepackage{graphicx}
\makeatletter
\def\maxwidth{\ifdim\Gin@nat@width>\linewidth\linewidth\else\Gin@nat@width\fi}
\def\maxheight{\ifdim\Gin@nat@height>\textheight\textheight\else\Gin@nat@height\fi}
\makeatother
% Scale images if necessary, so that they will not overflow the page
% margins by default, and it is still possible to overwrite the defaults
% using explicit options in \includegraphics[width, height, ...]{}
\setkeys{Gin}{width=\maxwidth,height=\maxheight,keepaspectratio}
% Set default figure placement to htbp
\makeatletter
\def\fps@figure{htbp}
\makeatother
% definitions for citeproc citations
\NewDocumentCommand\citeproctext{}{}
\NewDocumentCommand\citeproc{mm}{%
  \begingroup\def\citeproctext{#2}\cite{#1}\endgroup}
\makeatletter
 % allow citations to break across lines
 \let\@cite@ofmt\@firstofone
 % avoid brackets around text for \cite:
 \def\@biblabel#1{}
 \def\@cite#1#2{{#1\if@tempswa , #2\fi}}
\makeatother
\newlength{\cslhangindent}
\setlength{\cslhangindent}{1.5em}
\newlength{\csllabelwidth}
\setlength{\csllabelwidth}{3em}
\newenvironment{CSLReferences}[2] % #1 hanging-indent, #2 entry-spacing
 {\begin{list}{}{%
  \setlength{\itemindent}{0pt}
  \setlength{\leftmargin}{0pt}
  \setlength{\parsep}{0pt}
  % turn on hanging indent if param 1 is 1
  \ifodd #1
   \setlength{\leftmargin}{\cslhangindent}
   \setlength{\itemindent}{-1\cslhangindent}
  \fi
  % set entry spacing
  \setlength{\itemsep}{#2\baselineskip}}}
 {\end{list}}
\usepackage{calc}
\newcommand{\CSLBlock}[1]{\hfill\break\parbox[t]{\linewidth}{\strut\ignorespaces#1\strut}}
\newcommand{\CSLLeftMargin}[1]{\parbox[t]{\csllabelwidth}{\strut#1\strut}}
\newcommand{\CSLRightInline}[1]{\parbox[t]{\linewidth - \csllabelwidth}{\strut#1\strut}}
\newcommand{\CSLIndent}[1]{\hspace{\cslhangindent}#1}

\setlength\heavyrulewidth{0ex}
\setlength\lightrulewidth{0ex}
\usepackage[automark]{scrlayer-scrpage}
\clearpairofpagestyles
\cehead{
  Brian Weatherson
  }
\cohead{
  Trends in Philosophical Studies
  }
\ohead{\bfseries \pagemark}
\cfoot{}
\makeatletter
\newcommand*\NoIndentAfterEnv[1]{%
  \AfterEndEnvironment{#1}{\par\@afterindentfalse\@afterheading}}
\makeatother
\NoIndentAfterEnv{itemize}
\NoIndentAfterEnv{enumerate}
\NoIndentAfterEnv{description}
\NoIndentAfterEnv{quote}
\NoIndentAfterEnv{equation}
\NoIndentAfterEnv{longtable}
\NoIndentAfterEnv{abstract}
\renewenvironment{abstract}
 {\vspace{-1.25cm}
 \quotation\small\noindent\rule{\linewidth}{.5pt}\par\smallskip
 \noindent }
 {\par\noindent\rule{\linewidth}{.5pt}\endquotation}
\setkomafont{descriptionlabel}{\normalfont\scshape\bfseries}
\KOMAoption{captions}{tableheading}
\makeatletter
\@ifpackageloaded{caption}{}{\usepackage{caption}}
\AtBeginDocument{%
\ifdefined\contentsname
  \renewcommand*\contentsname{Table of contents}
\else
  \newcommand\contentsname{Table of contents}
\fi
\ifdefined\listfigurename
  \renewcommand*\listfigurename{List of Figures}
\else
  \newcommand\listfigurename{List of Figures}
\fi
\ifdefined\listtablename
  \renewcommand*\listtablename{List of Tables}
\else
  \newcommand\listtablename{List of Tables}
\fi
\ifdefined\figurename
  \renewcommand*\figurename{Figure}
\else
  \newcommand\figurename{Figure}
\fi
\ifdefined\tablename
  \renewcommand*\tablename{Table}
\else
  \newcommand\tablename{Table}
\fi
}
\@ifpackageloaded{float}{}{\usepackage{float}}
\floatstyle{ruled}
\@ifundefined{c@chapter}{\newfloat{codelisting}{h}{lop}}{\newfloat{codelisting}{h}{lop}[chapter]}
\floatname{codelisting}{Listing}
\newcommand*\listoflistings{\listof{codelisting}{List of Listings}}
\makeatother
\makeatletter
\makeatother
\makeatletter
\@ifpackageloaded{caption}{}{\usepackage{caption}}
\@ifpackageloaded{subcaption}{}{\usepackage{subcaption}}
\makeatother

\ifLuaTeX
  \usepackage{selnolig}  % disable illegal ligatures
\fi
\usepackage{bookmark}

\IfFileExists{xurl.sty}{\usepackage{xurl}}{} % add URL line breaks if available
\urlstyle{same} % disable monospaced font for URLs
\hypersetup{
  pdftitle={Trends in Philosophical Studies},
  pdfauthor={Brian Weatherson},
  hidelinks,
  pdfcreator={LaTeX via pandoc}}


\title{Trends in \emph{Philosophical Studies}}
\author{Brian Weatherson}
\date{2024}

\begin{document}
\maketitle
\begin{abstract}
\emph{Philosophical Studies} has become one of the most important
journals for work in several large topics in philosophy. This paper uses
data from the word distributions in those papers, and the citations of
the papers, to look at how it has changed over time, and how it became
so central.
\end{abstract}


\setstretch{1.1}
\section{Introduction}\label{sec-introduction}

\emph{Philosophical Studies} is the most cited journal in other
philosophy journals in recent times, so it is worthy of study both in
its own right, and as a mirror of broader trends in philosophy. This
paper mixes two techniques, topic modeling and citation analysis, to
look at the nature of \emph{Philosophical Studies}, and how it has
changed over time, from 1980 to 2019.

In the first half of the paper, I look at the growth the citation rate
(i.e., number of citations per article) in \emph{Philosophical Studies}.
If we take its peers to be the twenty philosophy journals with the
highest citation rates over the last forty years, it has gone from
having a citation rate 10-20 percent below the average, to a rate 10-20
percent above the average. And it's done this despite increasing its
publication volume, which one would normally expect to decrease citation
rate. This is explained in part by the move to publishing more special
issues, with the special issues from the Bellingham Summer Philosophy
Conference being particularly important, and in part by the topic mix in
\emph{Philosophical Studies} changing in ways that prefigured changes in
the broader philosophical community.

This leads to the second half of the paper, on just how the topic mix of
the journal changed over the years. The short version of those changes
is shown in \textbf{?@fig-five-topics}.

In the late twentieth century, \emph{Philosophical Studies} was focussed
on philosophy of language, and in particular on core analytic questions
about reference and description. These were central to philosophy in the
1970s after \emph{Naming and Necessity}, and stayed central to graduate
education at top schools well into the twenty-first century. But the
focus of other journals moved away from these questions possibly a bit
earlier than \emph{Philosophical Studies} did.

After the turn of the century, the journal moved into two topics that
did not hurt its citation rates. First, it was part of the enormous
discussion on epistemology, and in particular on contextualism in
epistemology, in the 2000s. Second, in the 2010s, it, along with much of
the rest of the discipline, moved more strongly into social philosophy.

One notable way in which \emph{Philosophical Studies} bucked a trend in
philosophy is that it did not move heavily into work on grounding. As
you can see in \textbf{?@fig-five-topics}, this wasn't because it moved
away from metaphysics more broadly. But is has, more than most journals,
resisted the turn to what Ted
(\citeproc{ref-Sider2021}{\textbf{Sider2021?}}) calls `postmodal
metaphysics'.

\section{Overview}\label{sec-overview}

\subsection{Editorial History}\label{editorial-history}

\emph{Philosophical Studies} was founded by Herbert Feigl and Wilfrid
Sellars, both then at the University of Minnesota, in 1950, as the
``first American journal expressly devoted to analytic philosophy''
(\citeproc{ref-DeVries2005}{DeVries 2005, 1--2}). They stayed as editors
until Feigl's retirement in 1971, though after 1954 Sellars was listed
as the first editor. At Feigl's retirement the journal moved from the
University of Minnesota Press to Reidel, where it has stayed ever
since.\footnote{Springer is the current continuant of Reidel after
  several mergers and takeovers.} Sellars edited the journal alone until
Keith Lehrer was brought on as associate editor in 1974, starting an
association with the journal that would last nearly half a century.

In 1975 Lehrer, who had just moved from Rochester to Arizona, became
editor. He stayed in that role until 1982, having been joined by John
Pollock (also at Arizona) in 1979. From 1982, Pollock was
Editor-in-Chief, and Lehrer went back to being Associate Editor.

In 1992 the journal moved 100 miles up I-10, as Stewart Cohen, then at
Arizona State, took over as Editor.\footnote{While the journal was at
  Arizona, at grad student assistant was recognised on the title page as
  an editorial assistant. Many prominent philosophers had this role over
  the years, including, in 1983, Stewart Cohen.} Cohen stayed as editor
through the rest of the time covered in this study, eventually being an
editor of the journal for longer than even Wilfrid Sellars.

Thomas Blackson joined as book symposium editor in 2003. In 2010, Cohen
moved to the University of Arizona, and so the journal was edited out of
Tuscon for a second time. Jennifer Lackey and Wayne Davis, who would
eventually take over from Cohen, joined as Associate Editors in 2014. In
2016 Cohen was made Editor-in-Chief, while Davis and Lackey became
Editors, and that was the arrangement that persisted through 2019, the
end of the focus of this paper.

There are three things about this editorial history that will become
important in what follows. One is that the names here include several of
the most important epistemologists of the last half century. A second is
that the journal has had a very stable editorial history; a summary like
this for most other peer journals would take twice as long to write. And
the third is that the biggest single change, the transition from John
Pollock to Stewart Cohen as Editor in 1992, does not seem to have had an
immediate impact on the journal. You see dramatic changes straight away
at \emph{Mind} when G.~E.~Moore takes over in 1921, and again when
Gilbert Ryle takes over in 1947.\footnote{See Lewy
  (\citeproc{ref-Lewy1976}{1976}) on Moore, and \emph{include citation
  from LDA} on Ryle.} The effects of the switch from Pollock to Cohen
are much more delayed.

\subsection{Articles}\label{articles}

\emph{Philosophical Studies} increased its output considerably when it
moved to Reidel, and then increased it again between 1980 and 2019.
Figure~\ref{fig-article-count-by-year} shows how many articles
\emph{Philosophical Studies} has published each year;
Figure~\ref{fig-word-count-by-year} shows the total word count for the
journal each year; Figure~\ref{fig-word-max-by-year} shows the count of
the longest article each year; and Figure~\ref{fig-word-quartiles} shows
the 25th, 50th, and 75th percentile article by word count for each year.

\begin{figure}

\centering{

\includegraphics{phil-studies_files/figure-pdf/fig-article-count-by-year-1.pdf}

}

\caption{\label{fig-article-count-by-year}Number of articles published
each year}

\end{figure}%

\begin{figure}

\centering{

\includegraphics{phil-studies_files/figure-pdf/fig-word-count-by-year-1.pdf}

}

\caption{\label{fig-word-count-by-year}Number of words published each
year}

\end{figure}%

\begin{figure}

\centering{

\includegraphics{phil-studies_files/figure-pdf/fig-word-max-by-year-1.pdf}

}

\caption{\label{fig-word-max-by-year}Longest article published each
year}

\end{figure}%

\begin{figure}

\centering{

\includegraphics{phil-studies_files/figure-pdf/fig-word-quartiles-1.pdf}

}

\caption{\label{fig-word-quartiles}25th, 50th, and 75th percentile word
lengths each year.}

\end{figure}%

Some of these changes reflect wider disciplinary changes, but others do
not. Most journals have a much more stable publication rate, though
there is more variation in commercially published journals. Articles
have been getting longer all across philosophy. What's striking here is
the 25th percentile rising to over 8000 words by the end of the 2010s.

It used to be common for philosophy colloquia to include papers that
were read out by the author. This was a bad practice, and it's been
mostly gone for several years. But starting in the mid-2010s it became
an impractical practice. No one was writing papers that even could be
read out in the length of a typical colloquium slot.

Everything I've said so far is about \emph{articles} in
\emph{Philosophical Studies}. As I'll go over in
Section~\ref{sec-what-is-an-article}, this is a less clear category than
we might like. To get to why that's so, I'll first go over the sources
I'm using.

\subsection{Special Issues}\label{sec-special-issues}

\emph{Philosophical Studies} has had many special issues, especially
since the mid-1990s. These fall into four main categories. The first
three are papers from three long-running conferences that
\emph{Philosophical Studies} published selected papers from:

\begin{enumerate}
\def\labelenumi{\arabic{enumi}.}
\tightlist
\item
  The APA Pacific Divsion
\item
  The Oberlin Colloquium
\item
  The Bellingham Summer Philosophy Conference (BSPC)
\end{enumerate}

The fourth is a less unified set of one-off issues, either on a special
topic, or, in two cases, conferences that \emph{Philosophical Studies}
published once but did not continue with. I'll call all of these
\textbf{One-off} issues.

Often these were double, or occasionally triple, issues. I'm counting
these as 2, or 3, issues, because provides a better sense of what
proportion of the papers in a year are from special issues. As
Table~\ref{tbl-issues-by-decade} shows, the special issues become a big
part of what \emph{Philosophical Studies} does in the 1990s.

\begin{longtable}[]{@{}lrrrr@{}}

\caption{\label{tbl-issues-by-decade}How many of each type of special
issue were published each decade.}

\tabularnewline

\toprule\noalign{}
Type & 1980s & 1990s & 2000s & 2010s \\
\midrule\noalign{}
\endhead
\bottomrule\noalign{}
\endlastfoot
Normal & 62 & 79 & 111 & 107 \\
One-off & 4 & 14 & 13 & 10 \\
APA Pacific & 0 & 18 & 12 & 10 \\
Oberlin & 0 & 6 & 5 & 4 \\
BSPC & 0 & 0 & 8 & 5 \\

\end{longtable}

The special issues differ from the normal issues in some striking ways,
so it will be helpful to keep their presence in mind.

\section{Methods}\label{sec-methods}

\subsection{Sources}\label{sec-sources}

The studies here are primarily based on two sources: citation data from
Web of Science, and word lists from JSTOR.

Through {[}University X{]} I downloaded the Web of Science (hereafter,
WoS) Core Collection in XML format. Within it, I selected 100 prominent
philosophy journals that WoS indexes. The journals I selected are, like
\emph{Philosophical Studies} primarily English-language, analytic
philosophy journals. I filtered the citations for just citations from
and to those 100 journals. WoS has a special way of recording citations
in indexed articles to other articles that it has indexed. These records
are easy to extract, and are considerably more reliable than citation
records in general.\footnote{That's not to say they are perfect. They
  certainly have false negatives, especially when there are any errors
  in the original citation. As
  (\citeproc{ref-Petrovich2023}{\textbf{Petrovich2023?}}) notes, they
  are more reliable when the citations are in a bibliography than when
  they are in footnotes. They also do badly with supplements. So for
  this study I've excluded all the supplemenmts to \emph{Noûs}, i.e.,
  those issues of \emph{Philosophical Perspectives} and
  \emph{Philosophical Issues} which were listed as supplements to
  \emph{Noûs}. I did include the supplemental issue \emph{Philosophical
  Studies} issued in 2013, because the data there looked reliable
  enough.} What follows uses just those citations. So it is citations
from indexed journals to indexed journals, where WoS recognised that
both the cited and citing journal were in its database.

This is obviously a small subset of all citations. It excludes citations
in academic journals in other fields, in books and edited volumes, and
in many other places that Google Scholar indexes, such as dissertations,
lecture notes, slides, and draft manuscripts. Losing that information is
a cost, but there are three large upsides. First, these citations are
much more accurate. Second, we can be more confident that our data set
is relatively complete; finding a full list of philosophy journals is
easier than finding a full list of edited volumes in philosophy. Third,
by looking at citations internal to philosophy, we can get a sense of
philosophy's self-image, and how it changes over time.

The downloadable citation data is not particularly up to date. I am
including citations beyond 2019, because it's helpful to get a sense of
how some of these articles have been received in more recent years. But
the data I have only goes through mid-2022. I'll often simply say 2022;
but note that even that year is incomplete.

The other source I used is JSTOR, and in particular the Data for
Research (DfR) program that they provide through their Constellate
project. This lets you download lists of the words used in various
journal articles, along with a count of how often each word is
used.\footnote{It also provides bigrams and trigrams, which I've
  occasionally used.} It also provides word counts for the articles,
which I have used in Section~\ref{sec-overview}. The words an author
uses are a pretty good guide to what they are talking about; if the word
`denotation' is used frequently, it's probably a philosophy of language
article.

\subsection{Articles?}\label{sec-what-is-an-article}

I said I'm talking about articles here, but what exactly is an
\emph{article}? A more helpful question is, which things that philosophy
journals publish are not articles?

Some things are easy. The table of contents is not an article. Nor is a
correction, or a report on editorial change. Book reviews are not
articles. If they were, \emph{Philosophical Review} would have the
lowest rate of citations per article, not the highest. Both WoS and
JSTOR also distinguish articles from discussion notes, especially if the
journal has a designated discussions section. Without this distinction,
\emph{Mind} would have a much lower rate of citations per article.

Both of these last two categories are relevant to \emph{Philosophical
Studies}, even though it does not run book reviews or have designated
discussion sections. They are relevant because it has many book
symposia. The classifiers, WoS and JSTOR, struggle with how to classify
articles in these symposia. They disagree with each other, and with
their own past practice.

I have some sympathy for the classifiers; these are really borderline
cases. Mostly what they settled on was that the précis and replies by
the book author are not articles, and that the contributions by
commentators are. But they did not stick precisely to this.

For the most part, I've gone with WoS's classifications. It would be
practical, just barely, to go through \emph{Philosophical Studies} issue
by issue and reclassify the book symposium entries so all and only the
commentaries are articles. But it would not be practical to do this for
all one hundred journals. And for this paper, we're mostly interested in
comparing articles in \emph{Philosophical Studies} with articles in
other journals, so it's best to not modify only one journal.

There is one place we're I've overridden WoS's classifications. It has
categories of Discussion, Note, and Review, each of which is about
0.75\% of the articles across the 100 journals. The three categories
include similar enough articles that I'll treat them as a unified
category. Mostly these are discussion notes, or longer book reviews,
that we want to exclude. But, especially in \emph{Philosophical Review},
they occasionally put ordinary articles here. So important articles by
Stanley Cavell (\citeproc{ref-WOSA1962CGX0500005}{1962}), Jonathan
Schaffer (\citeproc{ref-WOS000272855000002}{2010}), and Harvey Lederman
(\citeproc{ref-WOS000810220800002}{2022}) all got listed as not being
articles.\footnote{The Cavell article was in the discussion section of
  the January 1962 issue of the Review, so this classification is
  understandable. The other two are not.} I've counted any piece in
these three categories 20 pages or longer as an article.

There is one last tricky category to flag. The special issues on the
Oberlin Colloquium sometimes include commentaries on the main articles.
These are mostly not counted as articles, and I think rightly so.
Occasionally, as when Andy Egan (\citeproc{ref-Egan2011}{2011}) was the
commentator on an important paper by Tamar Szabò Gendler
(\citeproc{ref-WOS000295087100003}{2011}), the commentary gets a
reasonable number of citations. But mostly these commentaries are rarely
if ever cited, and I think they aren't really what most people think of
as journal articles. So I've been happy to exclude them.\footnote{If
  these aren't articles, what are they? WoS calls Egan's paper
  `Editorial-Matter'. That's wrong, but I'm not sure what I would say in
  their position.}

\section{Outbound Citations}\label{sec-overview}

\subsection{Overview of Citations}\label{sec-citations-overview}

Articles in \emph{Philosophical Studies} get cited a lot.
Table~\ref{tbl-all-cites} shows the five journals with the largest
number of citations of articles published between 1980 and 2019 in the
100 journals we're looking at.

\begin{longtable}[]{@{}lr@{}}

\caption{\label{tbl-all-cites}Leading journals by total number of
citations (Articles published 1980-2019).}

\tabularnewline

\toprule\noalign{}
Journal & Citations \\
\midrule\noalign{}
\endhead
\bottomrule\noalign{}
\endlastfoot
Philosophical Studies & 30424 \\
Synthese & 23280 \\
Journal Of Philosophy & 20023 \\
Noûs & 19716 \\
Philosophy Of Science & 18810 \\

\end{longtable}

\emph{Philosophical Studies} is in first place on that list in part, but
only in part, because it publishes so much. Table~\ref{tbl-all-articles}
lists the top five journals by the number of articles they have
published.

\begin{longtable}[]{@{}lr@{}}

\caption{\label{tbl-all-articles}Leading journals by total number of
articles (Articles published 1980-2019).}

\tabularnewline

\toprule\noalign{}
Journal & Articles \\
\midrule\noalign{}
\endhead
\bottomrule\noalign{}
\endlastfoot
Synthese & 4525 \\
Philosophical Studies & 3776 \\
Journal Of Medical Ethics & 3518 \\
Journal Of Symbolic Logic & 3283 \\
Analysis & 2236 \\

\end{longtable}

\emph{Synthese} has 20\% more articles, but 25\% fewer citations. The
other three journals on Table~\ref{tbl-all-articles} are somewhat
special cases. Two of them get a lot of citations outside of philosophy,
and this is only a study of citations in philosophy journals.
\emph{Analysis} only publishes short papers, and so while they get a lot
of citations per page, they don't get as many citations per article as
other journals.\footnote{Recall that in
  Section~\ref{sec-what-is-an-article} I noted that some \emph{Analysis}
  style papers in other journals are not even counted because they are
  classified as Discussions or Notes, not Articles. These excluded
  articles are cited at much lower rates than the typical
  \emph{Analysis} article.}

Still, we'd expect on general principles that raw volume of publication
wouldn't make a big difference. Citations tend to follow something like
a log-normal distribution (\citeproc{ref-Brzezinski2015}{Brzezinski
2015}). The bulk of the citations come from a handful of highly cited
articles. Publishing more articles helps, but is no guarantee.

If we look not at total citations, but at citations per article as in
Table~\ref{tbl-citation-rate}, we get a list that looks a bit more like
a familiar ranking of philosophy journals by prestige.

\begin{longtable}[]{@{}
  >{\raggedleft\arraybackslash}p{(\columnwidth - 8\tabcolsep) * \real{0.0633}}
  >{\raggedright\arraybackslash}p{(\columnwidth - 8\tabcolsep) * \real{0.5190}}
  >{\raggedleft\arraybackslash}p{(\columnwidth - 8\tabcolsep) * \real{0.1139}}
  >{\raggedleft\arraybackslash}p{(\columnwidth - 8\tabcolsep) * \real{0.1266}}
  >{\raggedleft\arraybackslash}p{(\columnwidth - 8\tabcolsep) * \real{0.1772}}@{}}

\caption{\label{tbl-citation-rate}Leading journals by citation rate
(Articles published 1980-2019).}

\tabularnewline

\toprule\noalign{}
\begin{minipage}[b]{\linewidth}\raggedleft
Rank
\end{minipage} & \begin{minipage}[b]{\linewidth}\raggedright
Journal
\end{minipage} & \begin{minipage}[b]{\linewidth}\raggedleft
Articles
\end{minipage} & \begin{minipage}[b]{\linewidth}\raggedleft
Citations
\end{minipage} & \begin{minipage}[b]{\linewidth}\raggedleft
Citation Rate
\end{minipage} \\
\midrule\noalign{}
\endhead
\bottomrule\noalign{}
\endlastfoot
1 & Philosophical Review & 510 & 14706 & 28.84 \\
2 & Journal Of Philosophy & 1221 & 20023 & 16.40 \\
3 & Philosophy \& Public Affairs & 521 & 8277 & 15.89 \\
4 & Mind & 1071 & 14391 & 13.44 \\
5 & Ethics & 1045 & 13040 & 12.48 \\
& \ldots{} & & & \\
13 & Philosophy And Phenomenological Research & 2165 & 17737 & 8.19 \\
14 & Philosophical Studies & 3776 & 30424 & 8.06 \\

\end{longtable}

I've included \emph{Philosophy and Phenomenological Research} there
because, like \emph{Philosophical Studies}, it publishes many book
symposia. And, like \emph{Philosophical Studies}, the articles in these
symposia are typically not cited very much.

\subsection{Large Trend}\label{sec-large-trend}

There is a striking step-change in citations to \emph{Philosophical
Studies} that occurred in the mid-2000s. Cross-temporal comparisons of
citations are hard because changes in the number of journals, the number
of articles in those journals, and citation norms, make most comparisons
tricky. To try to screen off some of that noise, I'll mostly compare
citations to articles published in \emph{Philosophical Studies} to
citations to other articles published at the same time.

In particular, in this section I'll compare \emph{Philosophical Studies}
to a list of nineteen other prominent philosophy journals. From the one
hundred journals that I'm primarily looking at, I selected the twenty
that have the highest rate of citations per published article, and which
Web of Science has indexed every year since 1980.\footnote{The last
  constraint notably rules out \emph{Philosophers' Imprint} and
  \emph{Mind and Language}.} That list includes \emph{Philosophical
Studies}, and the other nineteen journals are the comparison class.

Figure~\ref{fig-compare-cites-dots} shows, for each year from 1980 to
2019, the average number of citations for articles published in
\emph{Philosophical Studies} (in blue), and in the other nineteen (in
red). The figure is fairly noisy, but some trends are clear. Before
2000, the red dots, for the other journals, are mostly above the blue
dots, for \emph{Philosophical Studies}. After 2000, and especially from
2003 onwards, that is mostly reversed. Despite having less time to
accrue citations, articles from the 2000s are cited more, on average,
than articles published earlier. But articles published in the 2010s,
especially the late 2010s, have many fewer cites largely because they
haven't been around as long.

\begin{figure}

\centering{

\includegraphics{phil-studies_files/figure-pdf/fig-compare-cites-dots-1.pdf}

}

\caption{\label{fig-compare-cites-dots}Average citation rates for
\emph{Philosophical Studies} and peer journals.}

\end{figure}%

Figure~\ref{fig-compare-cites-rolling} smooths out some of the noise in
Figure~\ref{fig-compare-cites-dots} in two ways. First, instead of
measuring average citations per year, I measure average citations over a
five-year rolling window. This doesn't make a huge difference to the
measure for the other nineteen, which is already fairly smooth, but it
is useful for smoothing the values for just one journal. Second, instead
of showing the red and blue dots separately, I've just displayed the
ratio between them.

\begin{figure}

\centering{

\includegraphics{phil-studies_files/figure-pdf/fig-compare-cites-rolling-1.pdf}

}

\caption{\label{fig-compare-cites-rolling}Ratio of citations to
Philosophical Studies to citations to other journals, for five year
rolling windows.}

\end{figure}%

The difference in Figure~\ref{fig-compare-cites-rolling} between the
earlier and recent years is striking. By this one measure, citations per
article, \emph{Philosophical Studies} was doing ok before 2003, but was
towards the lower end of the top 20 journals. After 2003, it is doing
better than the average journal \emph{in the top 20}.

My very anecdotal impression is that \emph{Philosophical Studies} is
viewed as being more prestigious by younger philosophers than by older
philosophers. A toy model of prestige, where it is heavily anchored to
how often a journal was cited when one was in graduate school, would
explain that difference. That said, I have not done (and am not going to
do) a careful study of comparative prestige judgments to know if there
is even an effect here to find, or whether my informal sample was not
reflective of the wider population.

\subsection{Means, Medians and Quartiles}\label{sec-means-and-ways}

In some ways looking at the averages here understates the impact of
\emph{Philosophical Studies} in the 2000s. Average numbers of citations
can be very misleading. Table~\ref{tbl-quartiles-two-journals} compares
five year periods from two journals: the \emph{Australasian Journal of
Philosophy} (AJP) from 1980-1984, and \emph{Philosophical Studies} (PS)
from 2003-2007. Q1, Q2 and Q3 are the 25th, 50th, and 75th percentile
citations, and Mean is the average number of citations.

\begin{longtable}[]{@{}lrr@{}}

\caption{\label{tbl-quartiles-two-journals}Statistics for AJP 1980-1984,
and PS 2003-2007.}

\tabularnewline

\toprule\noalign{}
Statistic & AJP 1980-1984 & PS 2003-2007 \\
\midrule\noalign{}
\endhead
\bottomrule\noalign{}
\endlastfoot
Q1 & 0.0 & 2.0 \\
Q2 & 2.0 & 5.0 \\
Q3 & 7.0 & 14.0 \\
Mean & 16.7 & 13.5 \\

\end{longtable}

The mean number of citations per article is higher for AJP 1980-1984
than for \emph{Philosophical Studies} 2003-2007. But that's because of
the 1892 citations to articles from the AJP in those years, 1269 of them
are to five articles by David Lewis. The citations to
\emph{Philosophical Studies} are much more widely spread around.
Comparing the 75th percentiles is more indicative of what's being
published in the journals.

Part of the difference between AJP 1980-1984 and \emph{Philosophical
Studies} 2003-2007 is a general trend in the discipline to spread around
the citations more. But it's not just that. \emph{Philosophical Studies}
stands out for how widely spread around its citations are.
Figure~\ref{fig-quartiles-ps-vs-19} shows this. I've taken the same 19
journals that I used in Figure~\ref{fig-compare-cites-rolling}, and
calculated the same quartile statistics for \emph{Philosophical
Studies}, and for those 19 journals collectively, for five-year moving
windows each year from 1980 to 2019. The results are shown in
Figure~\ref{fig-quartiles-ps-vs-19}. Note that the dot for each year is
really the five-year window centrered on that year.

\begin{figure}

\centering{

\includegraphics{phil-studies_files/figure-pdf/fig-quartiles-ps-vs-19-1.pdf}

}

\caption{\label{fig-quartiles-ps-vs-19}Ratio of 75 percentile of
\emph{Philosophical Studies} citations to 75th percentile of citations
for other leading journals, five year rolling window.}

\end{figure}%

The pattern is rather similar to Figure~\ref{fig-compare-cites-rolling}.
Before 2000, the ratio was below one for 11 out of 20 years. After 2000,
the only time it is below one is right at the end, where the data is
very noisy. (The citation counts are very low for such recently
published articles, so it's hard to read too much into the numbers.)
There is a particular peak in the mid-2000s, just as in
Figure~\ref{fig-compare-cites-rolling}. Perhaps the biggest difference
is that in the 1980s, the ratio of the means was below one, but the
ratio of the 75th percentiles is sometimes above one. The explanation
for that is that there were very few \emph{Philosophical Studies}
articles from those years with huge citation counts, even though a
quarter of the articles were being fairly frequently cited.

\subsection{Citations of Special
Issues}\label{sec-citations-of-special-issues}

Part of the explanation of the pattern in
Figure~\ref{fig-compare-cites-rolling} and
Figure~\ref{fig-quartiles-ps-vs-19} is that the special issues that
\emph{Philosophical Studies} published in the 2000s were very heavily
cited. Table~\ref{tbl-citation-by-type} shows three summary statistics,
mean, median, and 75th percentile (Q3), for the normal
\emph{Philosophical Studies} issues, and for the four classes of special
issues. Table~\ref{tbl-citation-by-type-decade} shows the means for the
five classes over each of the four decades from 1980-2019. (The blank
cells mean that there aren't any special issues of that type that
decade.)

\begin{longtable}[]{@{}lrrr@{}}

\caption{\label{tbl-citation-by-type}Summary citation statistics for the
five types of \emph{Philosophical Studies} issues.}

\tabularnewline

\toprule\noalign{}
Type & Median & Q3 & Mean \\
\midrule\noalign{}
\endhead
\bottomrule\noalign{}
\endlastfoot
BSPC & 12 & 39 & 22.0 \\
Oberlin & 4 & 12 & 15.6 \\
One-off & 4 & 12 & 11.0 \\
APA Pacific & 2 & 7 & 7.4 \\
Normal & 3 & 8 & 7.1 \\

\end{longtable}

\begin{longtable}[]{@{}lrrrr@{}}

\caption{\label{tbl-citation-by-type-decade}Mean citations by decade for
the five types of \emph{Philosophical Studies} issues.}

\tabularnewline

\toprule\noalign{}
Type & 1980s & 1990s & 2000s & 2010s \\
\midrule\noalign{}
\endhead
\bottomrule\noalign{}
\endlastfoot
Normal & 6.2 & 7.1 & 9.3 & 6.5 \\
One-off & 8.7 & 12.9 & 14.2 & 8.6 \\
APA Pacific & & 4.7 & 14.6 & 5.3 \\
Oberlin & & 16.5 & 19.2 & 13.0 \\
BSPC & & & 27.7 & 16.7 \\

\end{longtable}

The numbers for BSPC are particularly striking. In the full set of
articles I'm working from, i.e., all the indexed articles from 100
journals, only 1.1\% of articles have 39 or more citations. But one
quarter of the articles that \emph{Philosophical Studies} published from
BSPC are in that 1.1\%. Surprisingly, given the numbers here, none of
the 10 most cited articles in \emph{Philosophical Studies}, and only one
of the 30 most cited articles, was from a BSPC special issue. The high
average isn't being caused by outliers, but many BSPC articles being
cited very frequently. By contrast, four of the ten most cited articles
were from the Oberlin colloquium. This is why the Q3 numbers for BSPC
and Oberlin are so different, even though the means aren't that far
apart.

The fact that the normal issues have such lower citation statistics than
the special issues might make us think that the explanation of
Figure~\ref{fig-compare-cites-rolling} and
Figure~\ref{fig-quartiles-ps-vs-19} can be found entirely in the special
issues. Indeed, if we just compare the normal issues of
\emph{Philosophical Studies} to the other 19 journals, the effect we
were seeing basically vanishes. This can be seen in
Figure~\ref{fig-quartiles-normal-only}.

\begin{figure}

\centering{

\includegraphics{phil-studies_files/figure-pdf/fig-quartiles-normal-only-1.pdf}

}

\caption{\label{fig-quartiles-normal-only}Ratio of 75 percentile of
normal issues of \emph{Philosophical Studies} citations to 75th
percentile of citations for other leading journals, five year rolling
window.}

\end{figure}%

This is surely part of the explanation of what's going on. But
Figure~\ref{fig-quartiles-normal-only} is not an apples-to-apples
comparison; we've removed the special issues from \emph{Philosophical
Studies}, but not from the other 19. This could be a significant issue.
I looked at the most cited article since 2005 in each of the 20 journals
that go into Figure~\ref{fig-quartiles-normal-only}, and in 6 of them
(including \emph{Philosophical Studies}), the most cited article was in
a special issue. Or, in one case, a special unit in a regular issue.
It's really hard to tell exactly which issues are special issues from
the archival record, and even harder to tell what the refereeing process
was for those special issues.\footnote{BSPC selected almost all its
  papers via anonymous review, and then the papers were refereed again
  before going into \emph{Philosophical Studies}. I have no idea how
  common that was for the other special issues.} The best we can say is
that special issues are sufficiently widespread, and sufficiently widely
cited, that Figure~\ref{fig-quartiles-normal-only} might be a misleading
comparison. The success of special issues is part of why
\emph{Philosophical Studies} was so widely cited in the 2000s, but maybe
not all of it.

Rather than go further down this rabbit hole, let's turn to what was in
\emph{Philosophical Studies} over the years, starting with its inbound
citations.

\section{Inbound Citatations}\label{sec-inbound-citations}

This section looks at the papers cited in \emph{Philosophical Studies}
in each of the last four decades. For each decade, I find the 20 papers
most cited in all philosophy journals over that decade. (They need not
be published in the decade, indeed some of them are published nearly a
half-century earlier.) I'll then look at which of them have the highest
and lowest proportion of their citations in \emph{Philosophical
Studies}, and comment on what that tells us about the journal, and the
discipline as a whole.

\subsection{1980s}\label{s}

%quarto-define-uuid: 85b77c8a-261c-4f58-9b04-f21c67e0a758-1-85b77c8a-261c-4f58-9b04-f21c67e0a758
Lewis (\citeproc{ref-WOSA1979JB14500003}{1979b})
%quarto-end-define-uuid
%quarto-define-uuid: 85b77c8a-261c-4f58-9b04-f21c67e0a758-2-85b77c8a-261c-4f58-9b04-f21c67e0a758
Perry (\citeproc{ref-WOSA1979HE39600001}{1979})
%quarto-end-define-uuid
%quarto-define-uuid: 85b77c8a-261c-4f58-9b04-f21c67e0a758-3-85b77c8a-261c-4f58-9b04-f21c67e0a758
Lewis (\citeproc{ref-WOSA1979JC64200001}{1979a})
%quarto-end-define-uuid
%quarto-define-uuid: 85b77c8a-261c-4f58-9b04-f21c67e0a758-4-85b77c8a-261c-4f58-9b04-f21c67e0a758
Perry (\citeproc{ref-WOSA1977EA01800002}{1977})
%quarto-end-define-uuid
%quarto-define-uuid: 85b77c8a-261c-4f58-9b04-f21c67e0a758-5-85b77c8a-261c-4f58-9b04-f21c67e0a758
Donnellan (\citeproc{ref-WOSA1966ZC83800001}{1966})
%quarto-end-define-uuid
%quarto-define-uuid: 85b77c8a-261c-4f58-9b04-f21c67e0a758-6-85b77c8a-261c-4f58-9b04-f21c67e0a758
Grice (\citeproc{ref-WOSA1957CGZ6000005}{1957})
%quarto-end-define-uuid
%quarto-define-uuid: 85b77c8a-261c-4f58-9b04-f21c67e0a758-7-85b77c8a-261c-4f58-9b04-f21c67e0a758
Dennett (\citeproc{ref-10.2307_2025382}{1971})
%quarto-end-define-uuid
%quarto-define-uuid: 85b77c8a-261c-4f58-9b04-f21c67e0a758-8-85b77c8a-261c-4f58-9b04-f21c67e0a758
Rawls (\citeproc{ref-WOSA1980KH88100001}{1980})
%quarto-end-define-uuid

\begin{longtable}[]{@{}rlrrl@{}}

\caption{\label{tbl-1980s-inbound}Highly cited articles in the 1980s.}

\tabularnewline

\toprule\noalign{}
Rank & Citation & PS Citations & All Citations & Percent \\
\midrule\noalign{}
\endhead
\bottomrule\noalign{}
\endlastfoot
1 &
85b77c8a-261c-4f58-9b04-f21c67e0a758-1-85b77c8a-261c-4f58-9b04-f21c67e0a758
& 8 & 36 & 22.22\% \\
2 &
85b77c8a-261c-4f58-9b04-f21c67e0a758-2-85b77c8a-261c-4f58-9b04-f21c67e0a758
& 13 & 59 & 22.03\% \\
3 &
85b77c8a-261c-4f58-9b04-f21c67e0a758-3-85b77c8a-261c-4f58-9b04-f21c67e0a758
& 7 & 34 & 20.59\% \\
4 &
85b77c8a-261c-4f58-9b04-f21c67e0a758-4-85b77c8a-261c-4f58-9b04-f21c67e0a758
& 6 & 33 & 18.18\% \\
5 &
85b77c8a-261c-4f58-9b04-f21c67e0a758-5-85b77c8a-261c-4f58-9b04-f21c67e0a758
& 9 & 55 & 16.36\% \\
& \ldots{} & & & \\
18 &
85b77c8a-261c-4f58-9b04-f21c67e0a758-6-85b77c8a-261c-4f58-9b04-f21c67e0a758
& 1 & 36 & 2.78\% \\
19 &
85b77c8a-261c-4f58-9b04-f21c67e0a758-7-85b77c8a-261c-4f58-9b04-f21c67e0a758
& 0 & 30 & 0.00\% \\
20 &
85b77c8a-261c-4f58-9b04-f21c67e0a758-8-85b77c8a-261c-4f58-9b04-f21c67e0a758
& 0 & 49 & 0.00\% \\

\end{longtable}

\section{Comparisons}\label{comparisons}

Articles with highest ratio of PS cites/general cites Articles with
lowest ratio of PS cites/general cites Previously I just did that with
articles published in PS; that seems wrong

Hmm, this could be complicated. Study: For each decade find 20 most
cited, and look at how often they are cited in PS.

For 1980s, the low is 0 for Rawls (1980) and Dennett (1971). But Rawls
is cited in Post (1984), under a variant name, and Dennett is cited in
Lormond (1985), but to the reprint. Maybe just live with the messiness
High five in 1980s are all language, or language-adjacent: Lewis (1979)
(both de se and time's arrow), two Perry papers, and Donnellan

For 1990s, the low is 0 for McDowell's Virtue and Reason. Is in
Brighouse (1990), but the citation is incomplete and uses the wrong
year. Oddly the third lowest is for Discrimination and Perceptual
Knowledge, only 2/38. And this looks right. The highs in 1990s are still
language based, though with metaphysics seeping in (Kim on
supervenience, Lewis on Universals)

For 2000s, lows include Cummins (1975) (which is hard to track, such a
common name), Anderson (1999), and Mechanisms paper (2000). The highs
are very epistemology based - Goldman is among them, and the two big
contextualism papers by DeRose and Lewis. Obviously the editor at the
time was also a prominent contextualist, but his papers are not cited in
the journal at a particularly out of the ordinary rate. (There are many
cases in history where editors are cited a lot, but this isn't one of
them.) Just outside top 20, but Jackson and Chalmers has nearly 1/3 its
cites in PS. Tracing back to Davies and Humberstone's original paper.
Remarkably, Elusive Knowledge is most cited in the decade; normally this
kind of calculation pushes towards less cited, high variance cases.

For 2010s, still mechanisms, Anderson (1999), also Kripke (1975) are the
low ones Top 5 is more varied, both in topic and time: Kolodny (2005),
Pryor (2000), Lewis (1986), Frankfurt (1969), Jackson (1982). Could be
result of publishing more papers, but striking that it's pulling these
older papers in.

\section{The LDA}\label{the-lda}

Build the model Note the five categories Graph the trends Flag the
methodology

\section{Language}\label{language}

Find the 40 most cited in each topic over the 40 years (i.e., most cited
per year) Look how often they are cited in PS Look how often they are
cited across the 100 See if the trends in PS track wider trends Find
other journals that have similar trends (Analysis, PQ, AJP)

\section{Metaphysics}\label{metaphysics}

Note the two Schaffer papers, and differences in citing Schaffer
(\citeproc{ref-WOS000368189400004}{2016}) Is that because little
grounding There is some - see graph of words Maybe just count how
`modal' vs `postmodal' the 2019 papers are Maybe do a small LDA of the
metaphysics papers

\section{Ethics}\label{ethics}

Is there anything to say here? Does it go more political? Still not
citing Anderson, but something?

Delicate. Even happening.

\phantomsection\label{refs}
\begin{CSLReferences}{1}{0}
\bibitem[\citeproctext]{ref-Brzezinski2015}
Brzezinski, Michal. 2015. {``Power Laws in Citation Distributions:
Evidence from Scopus.''} \emph{Scientometrics} 103 (1): 213--28. doi:
\href{https://doi.org/10.1007/s11192-014-1524-z}{10.1007/s11192-014-1524-z}.

\bibitem[\citeproctext]{ref-WOSA1962CGX0500005}
Cavell, S. 1962. {``The Availability of Wittgenstein's Later
Philosophy.''} \emph{Philosophical Review} 71 (1): 67--93. doi:
\href{https://doi.org/10.2307/2183682}{10.2307/2183682}.

\bibitem[\citeproctext]{ref-10.2307_2025382}
Dennett, D. C. 1971. {``Intentional Systems.''} \emph{Journal Of
Philosophy} 68 (4): 87--106.

\bibitem[\citeproctext]{ref-DeVries2005}
DeVries, Willem A. 2005. \emph{Wilfrid Sellars}. Abingdon: Routledge.

\bibitem[\citeproctext]{ref-WOSA1966ZC83800001}
Donnellan, Keith S. 1966. {``Reference and Definite Descriptions.''}
\emph{Philosophical Review} 75 (3): 281--304. doi:
\href{https://doi.org/10.2307/2183143}{10.2307/2183143}.

\bibitem[\citeproctext]{ref-Egan2011}
Egan, Andy. 2011. {``Comments on Gendler's, 'the Epistemic Costs of
Implicit Bias'.''} \emph{Philosophical Studies} 156 (1): 65--79. doi:
\href{https://doi.org/10.1007/s11098-011-9803-5}{10.1007/s11098-011-9803-5}.

\bibitem[\citeproctext]{ref-WOS000295087100003}
Gendler, Tamar Szabo. 2011. {``On the Epistemic Costs of Implicit
Bias.''} \emph{Philosophical Studies} 156 (1): 33--63. doi:
\href{https://doi.org/10.1007/s11098-011-9801-7}{10.1007/s11098-011-9801-7}.

\bibitem[\citeproctext]{ref-WOSA1957CGZ6000005}
Grice, HP. 1957. {``Meaning.''} \emph{Philosophical Review} 66 (3):
377--88. doi: \href{https://doi.org/10.2307/2182440}{10.2307/2182440}.

\bibitem[\citeproctext]{ref-WOS000810220800002}
Lederman, Harvey. 2022. {``The Introspective Model of Genuine Knowledge
in Wang Yangming.''} \emph{Philosophical Review} 131 (2): 169--213. doi:
\href{https://doi.org/10.1215/00318108-9554691}{10.1215/00318108-9554691}.

\bibitem[\citeproctext]{ref-WOSA1979JC64200001}
Lewis, David. 1979a. {``Attitudes de Dicto and de Se.''}
\emph{Philosophical Review} 88 (4): 513--43. doi:
\href{https://doi.org/10.2307/2184843}{10.2307/2184843}.

\bibitem[\citeproctext]{ref-WOSA1979JB14500003}
---------. 1979b. {``Counterfactual Dependence and Time's Arrow.''}
\emph{No{û}s} 13 (4): 455--76. doi:
\href{https://doi.org/10.2307/2215339}{10.2307/2215339}.

\bibitem[\citeproctext]{ref-Lewy1976}
Lewy, C. 1976. {``Mind Under {G. E. Moore} (1921-1947).''} \emph{Mind}
85 (337): 37--46. doi:
\href{https://doi.org/10.1093/mind/LXXXV.337.37}{10.1093/mind/LXXXV.337.37}.

\bibitem[\citeproctext]{ref-WOSA1977EA01800002}
Perry, John. 1977. {``Frege on Demonstratives.''} \emph{Philosophical
Review} 86 (4): 474--97. doi:
\href{https://doi.org/10.2307/2184564}{10.2307/2184564}.

\bibitem[\citeproctext]{ref-WOSA1979HE39600001}
---------. 1979. {``The Problem of the Essential Indexical.''}
\emph{No{û}s} 13 (1): 3--21. doi:
\href{https://doi.org/10.2307/2214792}{10.2307/2214792}.

\bibitem[\citeproctext]{ref-WOSA1980KH88100001}
Rawls, John. 1980. {``Kantian Constructivism in Moral Theory.''}
\emph{Journal Of Philosophy} 77 (9): 515--35. doi:
\href{https://doi.org/10.2307/2025790}{10.2307/2025790}.

\bibitem[\citeproctext]{ref-WOS000272855000002}
Schaffer, Jonathan. 2010. {``Monism: The Priority of the Whole.''}
\emph{Philosophical Review} 119 (1): 31--76. doi:
\href{https://doi.org/10.1215/00318108-2009-025}{10.1215/00318108-2009-025}.

\bibitem[\citeproctext]{ref-WOS000368189400004}
---------. 2016. {``Grounding in the Image of Causation.''}
\emph{Philosophical Studies} 173 (1): 49--100. doi:
\href{https://doi.org/10.1007/s11098-014-0438-1}{10.1007/s11098-014-0438-1}.

\end{CSLReferences}



\noindent Published online in November 2024.


\end{document}
