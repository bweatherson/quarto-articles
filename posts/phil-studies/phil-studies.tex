% Options for packages loaded elsewhere
\PassOptionsToPackage{unicode}{hyperref}
\PassOptionsToPackage{hyphens}{url}
%
\documentclass[
  10pt,
  letterpaper,
  DIV=11,
  numbers=noendperiod,
  twoside]{scrartcl}

\usepackage{amsmath,amssymb}
\usepackage{setspace}
\usepackage{iftex}
\ifPDFTeX
  \usepackage[T1]{fontenc}
  \usepackage[utf8]{inputenc}
  \usepackage{textcomp} % provide euro and other symbols
\else % if luatex or xetex
  \usepackage{unicode-math}
  \defaultfontfeatures{Scale=MatchLowercase}
  \defaultfontfeatures[\rmfamily]{Ligatures=TeX,Scale=1}
\fi
\usepackage{lmodern}
\ifPDFTeX\else  
    % xetex/luatex font selection
    \setmainfont[ItalicFont=EB Garamond Italic,BoldFont=EB Garamond
Bold]{EB Garamond Math}
    \setsansfont[]{Europa-Bold}
  \setmathfont[]{Garamond-Math}
\fi
% Use upquote if available, for straight quotes in verbatim environments
\IfFileExists{upquote.sty}{\usepackage{upquote}}{}
\IfFileExists{microtype.sty}{% use microtype if available
  \usepackage[]{microtype}
  \UseMicrotypeSet[protrusion]{basicmath} % disable protrusion for tt fonts
}{}
\usepackage{xcolor}
\usepackage[left=1in, right=1in, top=0.8in, bottom=0.8in,
paperheight=9.5in, paperwidth=6.5in, includemp=TRUE, marginparwidth=0in,
marginparsep=0in]{geometry}
\setlength{\emergencystretch}{3em} % prevent overfull lines
\setcounter{secnumdepth}{3}
% Make \paragraph and \subparagraph free-standing
\makeatletter
\ifx\paragraph\undefined\else
  \let\oldparagraph\paragraph
  \renewcommand{\paragraph}{
    \@ifstar
      \xxxParagraphStar
      \xxxParagraphNoStar
  }
  \newcommand{\xxxParagraphStar}[1]{\oldparagraph*{#1}\mbox{}}
  \newcommand{\xxxParagraphNoStar}[1]{\oldparagraph{#1}\mbox{}}
\fi
\ifx\subparagraph\undefined\else
  \let\oldsubparagraph\subparagraph
  \renewcommand{\subparagraph}{
    \@ifstar
      \xxxSubParagraphStar
      \xxxSubParagraphNoStar
  }
  \newcommand{\xxxSubParagraphStar}[1]{\oldsubparagraph*{#1}\mbox{}}
  \newcommand{\xxxSubParagraphNoStar}[1]{\oldsubparagraph{#1}\mbox{}}
\fi
\makeatother


\providecommand{\tightlist}{%
  \setlength{\itemsep}{0pt}\setlength{\parskip}{0pt}}\usepackage{longtable,booktabs,array}
\usepackage{calc} % for calculating minipage widths
% Correct order of tables after \paragraph or \subparagraph
\usepackage{etoolbox}
\makeatletter
\patchcmd\longtable{\par}{\if@noskipsec\mbox{}\fi\par}{}{}
\makeatother
% Allow footnotes in longtable head/foot
\IfFileExists{footnotehyper.sty}{\usepackage{footnotehyper}}{\usepackage{footnote}}
\makesavenoteenv{longtable}
\usepackage{graphicx}
\makeatletter
\newsavebox\pandoc@box
\newcommand*\pandocbounded[1]{% scales image to fit in text height/width
  \sbox\pandoc@box{#1}%
  \Gscale@div\@tempa{\textheight}{\dimexpr\ht\pandoc@box+\dp\pandoc@box\relax}%
  \Gscale@div\@tempb{\linewidth}{\wd\pandoc@box}%
  \ifdim\@tempb\p@<\@tempa\p@\let\@tempa\@tempb\fi% select the smaller of both
  \ifdim\@tempa\p@<\p@\scalebox{\@tempa}{\usebox\pandoc@box}%
  \else\usebox{\pandoc@box}%
  \fi%
}
% Set default figure placement to htbp
\def\fps@figure{htbp}
\makeatother
% definitions for citeproc citations
\NewDocumentCommand\citeproctext{}{}
\NewDocumentCommand\citeproc{mm}{%
  \begingroup\def\citeproctext{#2}\cite{#1}\endgroup}
\makeatletter
 % allow citations to break across lines
 \let\@cite@ofmt\@firstofone
 % avoid brackets around text for \cite:
 \def\@biblabel#1{}
 \def\@cite#1#2{{#1\if@tempswa , #2\fi}}
\makeatother
\newlength{\cslhangindent}
\setlength{\cslhangindent}{1.5em}
\newlength{\csllabelwidth}
\setlength{\csllabelwidth}{3em}
\newenvironment{CSLReferences}[2] % #1 hanging-indent, #2 entry-spacing
 {\begin{list}{}{%
  \setlength{\itemindent}{0pt}
  \setlength{\leftmargin}{0pt}
  \setlength{\parsep}{0pt}
  % turn on hanging indent if param 1 is 1
  \ifodd #1
   \setlength{\leftmargin}{\cslhangindent}
   \setlength{\itemindent}{-1\cslhangindent}
  \fi
  % set entry spacing
  \setlength{\itemsep}{#2\baselineskip}}}
 {\end{list}}
\usepackage{calc}
\newcommand{\CSLBlock}[1]{\hfill\break\parbox[t]{\linewidth}{\strut\ignorespaces#1\strut}}
\newcommand{\CSLLeftMargin}[1]{\parbox[t]{\csllabelwidth}{\strut#1\strut}}
\newcommand{\CSLRightInline}[1]{\parbox[t]{\linewidth - \csllabelwidth}{\strut#1\strut}}
\newcommand{\CSLIndent}[1]{\hspace{\cslhangindent}#1}

\setlength\heavyrulewidth{0ex}
\setlength\lightrulewidth{0ex}
\usepackage[automark]{scrlayer-scrpage}
\clearpairofpagestyles
\cehead{
  Brian Weatherson
  }
\cohead{
  Trends in Philosophical Studies, 1980-2019
  }
\ohead{\bfseries \pagemark}
\cfoot{}
\makeatletter
\newcommand*\NoIndentAfterEnv[1]{%
  \AfterEndEnvironment{#1}{\par\@afterindentfalse\@afterheading}}
\makeatother
\NoIndentAfterEnv{itemize}
\NoIndentAfterEnv{enumerate}
\NoIndentAfterEnv{description}
\NoIndentAfterEnv{quote}
\NoIndentAfterEnv{equation}
\NoIndentAfterEnv{longtable}
\NoIndentAfterEnv{abstract}
\renewenvironment{abstract}
 {\vspace{-1.25cm}
 \quotation\small\noindent\rule{\linewidth}{.5pt}\par\smallskip
 \noindent }
 {\par\noindent\rule{\linewidth}{.5pt}\endquotation}
\setkomafont{descriptionlabel}{\normalfont\scshape\bfseries}
\KOMAoption{captions}{tableheading}
\makeatletter
\@ifpackageloaded{caption}{}{\usepackage{caption}}
\AtBeginDocument{%
\ifdefined\contentsname
  \renewcommand*\contentsname{Table of contents}
\else
  \newcommand\contentsname{Table of contents}
\fi
\ifdefined\listfigurename
  \renewcommand*\listfigurename{List of Figures}
\else
  \newcommand\listfigurename{List of Figures}
\fi
\ifdefined\listtablename
  \renewcommand*\listtablename{List of Tables}
\else
  \newcommand\listtablename{List of Tables}
\fi
\ifdefined\figurename
  \renewcommand*\figurename{Figure}
\else
  \newcommand\figurename{Figure}
\fi
\ifdefined\tablename
  \renewcommand*\tablename{Table}
\else
  \newcommand\tablename{Table}
\fi
}
\@ifpackageloaded{float}{}{\usepackage{float}}
\floatstyle{ruled}
\@ifundefined{c@chapter}{\newfloat{codelisting}{h}{lop}}{\newfloat{codelisting}{h}{lop}[chapter]}
\floatname{codelisting}{Listing}
\newcommand*\listoflistings{\listof{codelisting}{List of Listings}}
\makeatother
\makeatletter
\makeatother
\makeatletter
\@ifpackageloaded{caption}{}{\usepackage{caption}}
\@ifpackageloaded{subcaption}{}{\usepackage{subcaption}}
\makeatother

\usepackage{bookmark}

\IfFileExists{xurl.sty}{\usepackage{xurl}}{} % add URL line breaks if available
\urlstyle{same} % disable monospaced font for URLs
\hypersetup{
  pdftitle={Trends in Philosophical Studies, 1980-2019},
  pdfauthor={Brian Weatherson},
  hidelinks,
  pdfcreator={LaTeX via pandoc}}


\title{Trends in \emph{Philosophical Studies}, 1980-2019}
\author{Brian Weatherson}
\date{2024}

\begin{document}
\maketitle
\begin{abstract}
\emph{Philosophical Studies} is the most cited journal in
English-language philosophy. This essay looks at how it became so
central, and how the changes in its subject matter over time reflect
changes in English language philosophy. Special attention is paid to the
role of special issues of the journal, notably the publication of
conference proceedings, and to the rise in papers in social philosophy
in the second half of the 2010s.
\end{abstract}


\setstretch{1.1}
\section{Introduction}\label{sec-introduction}

\emph{Philosophical Studies} is the most cited journal in philosophy
journals, so it is worthy of study both in its own right, and as a
mirror of broader trends in philosophy. This paper uses a variety of
tools from data analysis to look at the nature of \emph{Philosophical
Studies}, and how it has changed over time, from 1980 to 2019.

In the first half of the paper, I look at the growth the citation rate
(i.e., number of citations per article) in \emph{Philosophical Studies}.
If we take its peers to be the twenty philosophy journals with the
highest citation rates over the last forty years, it has gone from
having a citation rate 10-20 percent below the average, to a rate 10-20
percent above the average. And it's done this despite increasing its
publication volume, which one would normally expect to decrease citation
rate. This is explained in part by the move to publishing more special
issues, with the special issues from the Bellingham Summer Philosophy
Conference being particularly important, and in part by the topic mix in
\emph{Philosophical Studies} changing in ways that tracked changes in
the broader philosophical community.

This leads to the second half of the paper, on just how the topic mix of
the journal changed over the years. The short version of those changes
is shown in Figure~\ref{fig-five-topics}.

\begin{figure}

\centering{

\pandocbounded{\includegraphics[keepaspectratio]{phil-studies_files/figure-pdf/fig-five-topics-1.pdf}}

}

\caption{\label{fig-five-topics}Proportion of \emph{Philosophical
Studies} articles that are in different topics, 1980-2019.}

\end{figure}%

In the late twentieth century, \emph{Philosophical Studies} was focussed
on philosophy of language, and in particular on core analytic questions
about reference and description. These were central to philosophy in the
1970s after \emph{Naming and Necessity}, and stayed central to graduate
education at top schools well into the twenty-first century. But the
focus of other journals moved away from these questions earlier than
\emph{Philosophical Studies} did.

After the turn of the century, the journal moved into two topics that
did not hurt its citation rates. First, it was part of the enormous
discussion on epistemology, and in particular on contextualism in
epistemology, in the 2000s. Second, in the 2010s, it, along with much of
the rest of the discipline, moved more strongly into social philosophy.

\section{Overview}\label{sec-overview}

\subsection{Editorial History}\label{editorial-history}

\emph{Philosophical Studies} was founded by Herbert Feigl and Wilfrid
Sellars, both then at the University of Minnesota, in 1950, as the
``first American journal expressly devoted to analytic philosophy''
(\citeproc{ref-DeVries2005}{DeVries 2005, 1--2}). They stayed as editors
until Feigl's retirement in 1971, though after 1954 Sellars was listed
as the first editor. At Feigl's retirement the journal moved from the
University of Minnesota Press to Reidel, where it has stayed ever
since.\footnote{Springer is the current continuant of Reidel after
  several mergers and takeovers.} Sellars edited the journal alone until
Keith Lehrer was brought on as associate editor in 1974, starting an
association with the journal that would last nearly half a century.

In 1975 Lehrer, who had just moved from Rochester to Arizona, became
editor. He stayed in that role until 1982, having been joined by John
Pollock (also at Arizona) in 1979. From 1982, Pollock was
Editor-in-Chief, and Lehrer went back to being Associate Editor.

In 1992 the journal moved 100 miles up I-10, as Stewart Cohen, then at
Arizona State, took over as Editor.\footnote{While the journal was at
  Arizona, each year a grad student assistant was recognised on the
  title page as an editorial assistant. Many prominent philosophers had
  this role over the years, including, in 1983, Stewart Cohen.} Cohen
stayed as editor through the rest of the time covered in this study,
eventually being an editor of the journal for longer than even Wilfrid
Sellars.

Thomas Blackson joined as book symposium editor in 2003. In 2010, Cohen
moved to the University of Arizona, and so the journal was edited out of
Tuscon for a second time. Jennifer Lackey and Wayne Davis, who would
eventually take over from Cohen, joined as Associate Editors in 2014. In
2016 Cohen was made Editor-in-Chief, while Davis and Lackey became
Editors, and that was the arrangement that persisted through 2019, the
end of the focus of this paper.

There are three things about this editorial history that will become
important in what follows. One is that the names here include several of
the most important epistemologists of the last half century. A second is
that the journal has had a very stable editorial history; a summary like
this for most other leading journals would take twice as long to write.
And the third is that the biggest single change, the transition from
John Pollock to Stewart Cohen as Editor in 1992, does not seem to have
had an immediate impact on the journal. You see dramatic changes
straight away at \emph{Mind} when G.~E.~Moore takes over in 1921, and
again when Gilbert Ryle takes over in 1947.\footnote{See Lewy
  (\citeproc{ref-Lewy1976}{1976}) on Moore, and Warnock
  (\citeproc{ref-Warnock1976}{1976}) on Ryle.} The effects of the switch
from Pollock to Cohen are much more delayed.

\subsection{Articles}\label{articles}

\emph{Philosophical Studies} increased its output considerably when it
moved to Reidel, and then increased it again between 1980 and 2019.
Figure~\ref{fig-article-count-by-year} shows how many articles
\emph{Philosophical Studies} has published each year;
Figure~\ref{fig-word-count-by-year} shows the total word count for the
journal each year; Figure~\ref{fig-word-max-by-year} shows the count of
the longest article each year; and Figure~\ref{fig-word-quartiles} shows
the 25th, 50th, and 75th percentile article by word count for each year.

\begin{figure}

\centering{

\pandocbounded{\includegraphics[keepaspectratio]{phil-studies_files/figure-pdf/fig-article-count-by-year-1.pdf}}

}

\caption{\label{fig-article-count-by-year}Number of articles published
each year}

\end{figure}%

\begin{figure}

\centering{

\pandocbounded{\includegraphics[keepaspectratio]{phil-studies_files/figure-pdf/fig-word-count-by-year-1.pdf}}

}

\caption{\label{fig-word-count-by-year}Number of words published each
year}

\end{figure}%

\begin{figure}

\centering{

\pandocbounded{\includegraphics[keepaspectratio]{phil-studies_files/figure-pdf/fig-word-max-by-year-1.pdf}}

}

\caption{\label{fig-word-max-by-year}Longest article published each
year}

\end{figure}%

\begin{figure}

\centering{

\pandocbounded{\includegraphics[keepaspectratio]{phil-studies_files/figure-pdf/fig-word-quartiles-1.pdf}}

}

\caption{\label{fig-word-quartiles}25th, 50th, and 75th percentile word
lengths each year.}

\end{figure}%

Some of these changes reflect wider disciplinary changes, but others do
not. Most journals have a much more stable publication rate. Journals
that are commercially published, like \emph{Philosophical Studies}, have
tended to increase their production in recent years, but this growth is
still unusual.

Articles have been getting longer all across philosophy. What's striking
in Figure~\ref{fig-word-quartiles} is the 25th percentile rising to over
8000 words by the end of the 2010s. It used to be common for philosophy
colloquia to include papers that were read out by the author. This was a
bad practice, and it's been mostly gone for several years. But starting
in the mid-2010s it became an impractical practice. No one was writing
papers that even could be read out in the length of a typical colloquium
slot.

Everything I've said so far is about \emph{articles} in
\emph{Philosophical Studies}. As I'll go over in
Section~\ref{sec-what-is-an-article}, this is a less clear category than
we might like, in part because of the variety of kinds of issues
\emph{Philosophical Studies} puts out.

\subsection{Special Issues}\label{sec-special-issues}

\emph{Philosophical Studies} has had many special issues, especially
since the mid-1990s. These fall into four main categories. The first
three are papers from three long-running conferences that
\emph{Philosophical Studies} published selected papers from:

\begin{enumerate}
\def\labelenumi{\arabic{enumi}.}
\tightlist
\item
  The APA Pacific Divsion
\item
  The Oberlin Colloquium
\item
  The Bellingham Summer Philosophy Conference (BSPC)
\end{enumerate}

The fourth category consists of one-off issues, either on a special
topic, or, in two cases, conferences that \emph{Philosophical Studies}
published once but did not continue with. I'll call all of these
\textbf{One-off} issues.

Often these were double, or occasionally triple, issues. I'm counting
these as 2, or 3, issues, because provides a better sense of what
proportion of the papers in a year are from special issues. As
Table~\ref{tbl-issues-by-decade} shows, the special issues become a big
part of what \emph{Philosophical Studies} does in the 1990s.

\begin{longtable}[]{@{}lrrrr@{}}

\caption{\label{tbl-issues-by-decade}How many of each type of special
issue were published each decade.}

\tabularnewline

\toprule\noalign{}
Type & 1980s & 1990s & 2000s & 2010s \\
\midrule\noalign{}
\endhead
\bottomrule\noalign{}
\endlastfoot
Normal & 62 & 79 & 111 & 107 \\
One-off & 4 & 14 & 13 & 10 \\
APA Pacific & 0 & 18 & 12 & 10 \\
Oberlin & 0 & 6 & 5 & 4 \\
BSPC & 0 & 0 & 8 & 5 \\

\end{longtable}

The special issues differ from the normal issues in some striking ways,
so it will be helpful to keep their presence in mind.

\section{Methods}\label{sec-methods}

\subsection{Sources}\label{sec-sources}

The studies here are primarily based on two sources: citation data from
Web of Science, and word lists from JSTOR.

Through the University of Michigan I downloaded the Web of Science
(hereafter, WoS) Core Collection in XML format. Within it, I selected
100 prominent philosophy journals that WoS indexes. The journals I
selected are, like \emph{Philosophical Studies} primarily
English-language, analytic philosophy journals. I filtered the citations
for just citations from and to those 100 journals.

WoS has a special way of recording citations in indexed articles to
other articles that it has indexed. These records are easy to extract,
and are considerably more reliable than citation records in general.
That's not to say they are perfect. They certainly have false negatives,
especially when there are any errors in the original citation. As
Eugenio Petrovich (\citeproc{ref-Petrovich2024}{2024, 77n10}) notes,
they are more reliable when the citations are in a bibliography than
when they are in footnotes. They also do badly with supplements. So for
this study I've excluded all the supplemenmts to \emph{Noûs}, i.e.,
those issues of \emph{Philosophical Perspectives} and
\emph{Philosophical Issues} which were listed as supplements to
\emph{Noûs}. I did include the supplemental issue \emph{Philosophical
Studies} issued in 2013, because the data there looked reliable enough.
What follows uses just those citations. So it is citations from indexed
journals to indexed journals, where WoS recognised that both the cited
and citing journal were in its database.

This is obviously a small subset of all citations. It excludes citations
in academic journals in other fields, in books and edited volumes, and
in many other places that Google Scholar indexes, such as dissertations,
lecture notes, slides, and draft manuscripts. Losing that information is
a cost, but there are three large upsides. First, these citations are
much more accurate. Second, we can be more confident that our data set
is relatively complete; finding a full list of philosophy journals is
easier than finding a full list of edited volumes, or manuscripts on
websites, in philosophy. Third, by looking at citations internal to
philosophy, we can get a sense of philosophy's self-image, and how it
changes over time.

The downloadable citation data is not particularly up to date. I am
including citations beyond 2019, because it's helpful to get a sense of
how some of these articles have been received in more recent years. But
the data I have only goes through mid-2022.

The other source I used is JSTOR, and in particular the Data for
Research (DfR) program that they provide through their Constellate
project. This lets you download lists of the words used in various
journal articles, along with a count of how often each word is
used.\footnote{It also provides bigrams and trigrams, which I've looked
  at in preparing this paper, but didn't end up using.} It also provides
word counts for the articles, which I have used in
Section~\ref{sec-overview}. The words an author uses are a pretty good
guide to what they are talking about; if the word `denotation' is used
frequently, it's probably a philosophy of language article.

\subsection{Articles?}\label{sec-what-is-an-article}

I said I'm talking about articles here, but what exactly is an
\emph{article}? A more helpful question is, which things that philosophy
journals publish are not articles?

Some things are easy. The table of contents is not an article. Nor is a
correction, or a report on editorial change. Book reviews are not
articles. If they were, \emph{Philosophical Review} would have the
lowest rate of citations per article, not the highest.\footnote{I'll
  report on citations per article in Table~\ref{tbl-citation-rate}.}
Both WoS and JSTOR also distinguish articles from discussion notes,
especially if the journal has a designated discussions section. Without
this distinction, \emph{Mind} would have a much lower rate of citations
per article.

Both of these last two categories are relevant to \emph{Philosophical
Studies}, even though it does not run book reviews or have designated
discussion sections. Although it has neither of those things, it does
have many book symposia. The classifiers, WoS and JSTOR, struggle with
how to classify articles in these symposia. They disagree with each
other, and occasionally with their own past practice.

I have some sympathy for the classifiers; these are really borderline
cases. Mostly what they settled on was that the précis and replies by
the book author are not articles, and that the contributions by
commentators are. But they did not stick precisely to this.

For the most part, I've gone with WoS's classifications. It would be
practical, just barely, to go through \emph{Philosophical Studies} issue
by issue and reclassify the book symposium entries so all and only the
commentaries are articles. But it would not be practical to do this for
all one hundred journals. And for this paper, we're mostly interested in
comparing articles in \emph{Philosophical Studies} with articles in
other journals, so it's best to not modify only one journal.

There is one place we're I've overridden WoS's classifications. It has
categories of Discussion, Note, and Review, each of which make up about
0.75\% of the articles across the 100 journals. The three categories
include similar enough pieces that I'll treat them as a unified
category. Mostly these are discussion notes, or longer book reviews,
that we want to exclude. But, especially in \emph{Philosophical Review},
they occasionally put ordinary articles here. So important articles by
Stanley Cavell (\citeproc{ref-WOSA1962CGX0500005}{1962}), Jonathan
Schaffer (\citeproc{ref-WOS000272855000002}{2010}), and Harvey Lederman
(\citeproc{ref-WOS000810220800002}{2022}) all got listed as not being
articles.\footnote{The Cavell article was in the discussion section of
  the January 1962 issue of the Review, so this classification is
  understandable. The other two are not.} I've counted any piece in
these three categories twenty pages or longer as an article.

There is one last tricky category to flag. The special issues on the
Oberlin Colloquium sometimes include commentaries on the main articles.
These are mostly not counted as articles, and I think rightly so.
Occasionally, as when Andy Egan (\citeproc{ref-Egan2011}{2011}) was the
commentator on an important paper by Tamar Szabò Gendler
(\citeproc{ref-WOS000295087100003}{2011}), the commentary gets a
reasonable number of citations. But mostly these commentaries are rarely
if ever cited, and I think they aren't really what most people think of
as journal articles. So I've been happy to exclude them.\footnote{If
  these aren't articles, what are they? WoS classifies Egan's paper as
  `Editorial-Matter'. That's wrong, but I'm not sure what I would say in
  their position.}

\section{Outbound Citations}\label{sec-outbound}

\subsection{Overview of Citations}\label{sec-citations-overview}

Articles in \emph{Philosophical Studies} get cited a lot.
Table~\ref{tbl-all-cites} shows the five journals with the largest
number of citations of articles published between 1980 and 2019 in the
100 journals we're looking at.

\begin{longtable}[]{@{}lr@{}}

\caption{\label{tbl-all-cites}Leading journals by total number of
citations (Articles published 1980-2019).}

\tabularnewline

\toprule\noalign{}
Journal & Citations \\
\midrule\noalign{}
\endhead
\bottomrule\noalign{}
\endlastfoot
Philosophical Studies & 30613 \\
Synthese & 23583 \\
Journal Of Philosophy & 20199 \\
Philosophy Of Science & 18857 \\
Philosophy And Phenomenological Research & 17754 \\

\end{longtable}

\emph{Philosophical Studies} is in first place on that list in part, but
only in part, because it publishes so much. Table~\ref{tbl-all-articles}
lists the top five journals by the number of articles they have
published.

\begin{longtable}[]{@{}lr@{}}

\caption{\label{tbl-all-articles}Leading journals by total number of
articles (Articles published 1980-2019).}

\tabularnewline

\toprule\noalign{}
Journal & Articles \\
\midrule\noalign{}
\endhead
\bottomrule\noalign{}
\endlastfoot
Synthese & 4525 \\
Philosophical Studies & 3776 \\
Journal Of Medical Ethics & 3518 \\
Journal Of Symbolic Logic & 3283 \\
Analysis & 2236 \\

\end{longtable}

\emph{Synthese} has 20\% more articles, but 25\% fewer citations. The
other three journals on Table~\ref{tbl-all-articles} are somewhat
special cases. Two of them get a lot of citations outside of philosophy,
and this is only a study of citations in philosophy journals.
\emph{Analysis} only publishes short papers, and so while they get a lot
of citations per page, they don't get as many citations per article as
other journals.

Still, we'd expect on general principles that raw volume of publication
wouldn't make a big difference. Citations tend to follow something like
a log-normal distribution (\citeproc{ref-Brzezinski2015}{Brzezinski
2015}). The bulk of the citations come from a handful of highly cited
articles. Publishing more articles helps, but is no guarantee.

If we look not at total citations, but at citations per article as in
Table~\ref{tbl-citation-rate}, we get a list that looks a bit more like
a familiar ranking of philosophy journals by prestige.

\begin{longtable}[]{@{}
  >{\raggedleft\arraybackslash}p{(\linewidth - 8\tabcolsep) * \real{0.0633}}
  >{\raggedright\arraybackslash}p{(\linewidth - 8\tabcolsep) * \real{0.5190}}
  >{\raggedleft\arraybackslash}p{(\linewidth - 8\tabcolsep) * \real{0.1139}}
  >{\raggedleft\arraybackslash}p{(\linewidth - 8\tabcolsep) * \real{0.1266}}
  >{\raggedleft\arraybackslash}p{(\linewidth - 8\tabcolsep) * \real{0.1772}}@{}}

\caption{\label{tbl-citation-rate}Leading journals by citation rate
(Articles published 1980-2019).}

\tabularnewline

\toprule\noalign{}
\begin{minipage}[b]{\linewidth}\raggedleft
Rank
\end{minipage} & \begin{minipage}[b]{\linewidth}\raggedright
Journal
\end{minipage} & \begin{minipage}[b]{\linewidth}\raggedleft
Articles
\end{minipage} & \begin{minipage}[b]{\linewidth}\raggedleft
Citations
\end{minipage} & \begin{minipage}[b]{\linewidth}\raggedleft
Citation Rate
\end{minipage} \\
\midrule\noalign{}
\endhead
\bottomrule\noalign{}
\endlastfoot
1 & Philosophical Review & 510 & 14749 & 28.92 \\
2 & Journal Of Philosophy & 1221 & 20199 & 16.54 \\
3 & Philosophy \& Public Affairs & 521 & 8292 & 15.92 \\
4 & Noûs & 1170 & 15854 & 13.55 \\
5 & Mind & 1071 & 14435 & 13.48 \\
& \ldots{} & & & \\
13 & Philosophy And Phenomenological Research & 2165 & 17754 & 8.20 \\
14 & Philosophical Studies & 3776 & 30613 & 8.11 \\

\end{longtable}

I've included \emph{Philosophy and Phenomenological Research} there
because, like \emph{Philosophical Studies}, it publishes many book
symposia. And, like \emph{Philosophical Studies}, the articles in these
symposia are typically not cited very much.

\subsection{Large Trend}\label{sec-large-trend}

There is a striking step-change in citations to \emph{Philosophical
Studies} that occurred in the mid-2000s. To bring this out, I'll compare
\emph{Philosophical Studies} to nineteen other prominent philosophy
journals. From the one hundred journals that I'm primarily looking at, I
selected the twenty (including \emph{Philosophical Studies}) that have
the highest rate of citations per published article, and which Web of
Science has indexed every year since 1980.\footnote{The last constraint
  notably rules out \emph{Philosophers' Imprint} and \emph{Mind and
  Language}.}

Figure~\ref{fig-compare-cites-dots} shows, for each year from 1980 to
2019, the average number of citations for articles published in
\emph{Philosophical Studies} (in green), and in the other nineteen (in
red). The figure is fairly noisy, but some trends are clear. Before
2000, the red dots, for the other journals, are mostly above the green
dots, for \emph{Philosophical Studies}. After 2000, and especially from
2003 onwards, that is mostly reversed.

\begin{figure}

\centering{

\pandocbounded{\includegraphics[keepaspectratio]{phil-studies_files/figure-pdf/fig-compare-cites-dots-1.pdf}}

}

\caption{\label{fig-compare-cites-dots}Average citation rates for
\emph{Philosophical Studies} and peer journals.}

\end{figure}%

Figure~\ref{fig-compare-cites-rolling} smooths out some of the noise in
Figure~\ref{fig-compare-cites-dots} in two ways. First, instead of
measuring average citations per year, I measure average citations over a
five-year rolling window. Second, instead of showing the red and green
dots separately, I've just displayed the ratio between them.

\begin{figure}

\centering{

\pandocbounded{\includegraphics[keepaspectratio]{phil-studies_files/figure-pdf/fig-compare-cites-rolling-1.pdf}}

}

\caption{\label{fig-compare-cites-rolling}Ratio of citations to
Philosophical Studies to citations to other journals, for five year
rolling windows.}

\end{figure}%

The difference in Figure~\ref{fig-compare-cites-rolling} between the
earlier and later years is striking. By this one measure, citations per
article, \emph{Philosophical Studies} was doing ok before 2003, but was
below the average of the top 20 journals. After 2003, it is consistently
doing better than the average journal \emph{in the top 20}.

My very anecdotal impression is that \emph{Philosophical Studies} is
viewed as being more prestigious by younger philosophers than by older
philosophers. A toy model of prestige, where it is heavily anchored to
how often a journal was cited when one was in graduate school, would
explain that difference. That said, I have not done (and am not going to
do) a careful study of comparative prestige judgments to know if there
is even an effect here to find, or whether my informal sample was not
reflective of the wider population.

I won't repeat the analysis here, but if build
Figure~\ref{fig-compare-cites-rolling} using median citations, or 75th
percentile citations, rather than means, the shape of the graph doesn't
change much. On any of these measures, \emph{Philosophical Studies} was
below the middle of this group of twenty before 2000, and above it
afterwards.

\subsection{Citations of Special
Issues}\label{sec-citations-of-special-issues}

Part of the explanation of the pattern in
Figure~\ref{fig-compare-cites-rolling} is that the special issues that
\emph{Philosophical Studies} published in the 2000s were very heavily
cited. Table~\ref{tbl-citation-by-type} shows three summary statistics,
mean, median, and 75th percentile (Q3), for the normal
\emph{Philosophical Studies} issues, and for the four classes of special
issues. Table~\ref{tbl-citation-by-type-decade} shows the means for the
five classes over each of the four decades from 1980-2019. (The blank
cells mean that there aren't any special issues of that type that
decade.)

\begin{longtable}[]{@{}lrrr@{}}

\caption{\label{tbl-citation-by-type}Summary citation statistics for the
five types of \emph{Philosophical Studies} issues.}

\tabularnewline

\toprule\noalign{}
Type & Mean & Median & Q3 \\
\midrule\noalign{}
\endhead
\bottomrule\noalign{}
\endlastfoot
BSPC & 22.5 & 13 & 39 \\
Oberlin & 15.4 & 4 & 12 \\
One-off & 11.1 & 4 & 12 \\
APA Pacific & 7.4 & 2 & 7 \\
Normal & 7.2 & 3 & 8 \\

\end{longtable}

\begin{longtable}[]{@{}lrrrr@{}}

\caption{\label{tbl-citation-by-type-decade}Mean citations by decade for
the five types of \emph{Philosophical Studies} issues.}

\tabularnewline

\toprule\noalign{}
Type & 1980s & 1990s & 2000s & 2010s \\
\midrule\noalign{}
\endhead
\bottomrule\noalign{}
\endlastfoot
Normal & 6.2 & 7.1 & 9.3 & 6.6 \\
One-off & 8.6 & 12.8 & 14.1 & 8.8 \\
APA Pacific & & 4.6 & 14.4 & 5.4 \\
Oberlin & & 16.5 & 18.3 & 13.1 \\
BSPC & & & 28.3 & 17.1 \\

\end{longtable}

The numbers for BSPC are particularly striking. In the full set of
articles I'm working from, i.e., all the indexed articles from 100
journals, only 1.1\% of articles have 39 or more citations. But one
quarter of the articles that \emph{Philosophical Studies} published from
BSPC are in that 1.2\%. Surprisingly, only one of the 25 most cited
articles in \emph{Philosophical Studies} was from a BSPC special
issue.\footnote{``Epistemic Modals, Relativism and Assertion''
  (\citeproc{ref-WOS000245280800001}{Egan 2007}) is the equal 12th most
  cited paper in \emph{Philosophical Studies}.} The high average isn't
being caused by outliers, but many BSPC articles being cited very
frequently. By contrast, four of the ten most cited articles were from
the Oberlin colloquium. This is why the Q3 numbers for BSPC and Oberlin
are so different, even though the means aren't that far apart.

The fact that the normal issues have such lower citation statistics than
the special issues might make us think that the explanation of
Figure~\ref{fig-compare-cites-rolling} can be found entirely in the
special issues. Indeed, if we just compare the normal issues of
\emph{Philosophical Studies} to the other 19 journals, the effect we
were seeing basically vanishes. This can be seen in
Figure~\ref{fig-means-normal-only}.

\begin{figure}

\centering{

\pandocbounded{\includegraphics[keepaspectratio]{phil-studies_files/figure-pdf/fig-means-normal-only-1.pdf}}

}

\caption{\label{fig-means-normal-only}Ratio of citations of normal
issues of \emph{Philosophical Studies} citations to citations of other
leading journals, five year rolling window.}

\end{figure}%

When we just focus on the normal issues, there are on average more
citations to the other 19 journals than to \emph{Philosophical Studies}
until the mid-2010s. So part of the story behind
Figure~\ref{fig-compare-cites-rolling} is that the special issues of
\emph{Philosophical Studies} in the 2000s were well cited. But it's
potentially misleading to compare normal issues of one journal to all
issues of other journals. I looked at the most cited article since 2005
in each of the 20 journals that go into
Figure~\ref{fig-means-normal-only}, and in 6 of them (including
\emph{Philosophical Studies}), the most cited article was in a special
issue. (Or, in one case, a special unit in a regular issue.) Special
issues, including refereed special issues, have been very widely cited
in recent times.

Rather than go further down this rabbit hole, let's turn to what was in
\emph{Philosophical Studies} over the years, starting with its inbound
citations.

\section{Inbound Citatations}\label{sec-inbound-citations}

This section looks at the eighty journal articles that are most cited
across the hundred journals between 1980 and 2019, and looks at what
proportion of their citations are in \emph{Philosophical Studies}. On
average, those articles have about 10\% of their citations in
\emph{Philosophical Studies}, but the proportion varies greatly by
articles. Table~\ref{tbl-mainly-in-ps} lists the ten articles whose
citations are most concentrated in \emph{Philosophical
Studies}.\footnote{The ranking here is not by proportion of cites in
  \emph{Philosophical Studies}, but instead by the probability that an
  article would have this many cites in \emph{Philosophical Studies},
  given how many overall citations it has and that highly cited articles
  have about 10\% of their citations in \emph{Philosophical Studies}. I
  think this is a slightly better way to separate the signal from the
  noise.}


\begin{longtable}[]{@{}lrr@{}}

\caption{\label{tbl-mainly-in-ps}Articles with a high concentration of
ciations in \emph{Philosophical Studies}.}

\tabularnewline

\toprule\noalign{}
Article & PS Citations & All Citations \\
\midrule\noalign{}
\endhead
\bottomrule\noalign{}
\endlastfoot
Pryor (\citeproc{ref-WOS000165361800002}{2000})
& 54 & 270 \\
Kolodny (\citeproc{ref-WOS000231037900002}{2005})
& 37 & 173 \\
Schaffer (\citeproc{ref-WOS000272855000002}{2010})
& 38 & 181 \\
Lewis (\citeproc{ref-WOSA1996VY21200001}{1996})
& 65 & 391 \\
Johnston (\citeproc{ref-WOSA1992KC39800002}{1992})
& 40 & 217 \\
Lewis (\citeproc{ref-WOSA1979JB14500003}{1979})
& 50 & 298 \\
Lewis (\citeproc{ref-WOSA1984TQ70900001}{1984})
& 32 & 171 \\
Lewis (\citeproc{ref-WOSA1997WP33800001}{1997})
& 31 & 171 \\
Frankfurt (\citeproc{ref-WOSA1969Y444700002}{1969})
& 60 & 412 \\
Broome (\citeproc{ref-WOS000084073700005}{1999})
& 24 & 142 \\
Lewis (\citeproc{ref-WOSA1983RR51600001}{1983})
& 70 & 518 \\
Dretske (\citeproc{ref-WOSA1970ZE33800001}{1970})
& 41 & 279 \\

\end{longtable}

There are articles here in the five main topics \emph{Philosophical
Studies} covers: language, ethics, metaphysics, mind, and epistemology
(hereafter, LEMME). The other end of the scale looks quite different, as
shown in Table~\ref{tbl-mainly-out-ps}. There we mainly see articles in
political philosophy, philosophy of science, and logic, with Clark and
Chalmers (\citeproc{ref-WOS000073222300002}{1998}) being a notable (and
surprising) exception.


\begin{longtable}[]{@{}lrr@{}}

\caption{\label{tbl-mainly-out-ps}Articles with a low concentration of
ciations in \emph{Philosophical Studies}.}

\tabularnewline

\toprule\noalign{}
Article & PS Citations & All Citations \\
\midrule\noalign{}
\endhead
\bottomrule\noalign{}
\endlastfoot
Machamer, Darden, and Craver (\citeproc{ref-WOS000087305900001}{2000})
& 11 & 372 \\
Alchourrón, Gärdenfors, and Makinson(\citeproc{ref-WOSA1985AKA2200025}{1985})
& 3 & 218 \\
Laudan (\citeproc{ref-WOSA1981LY92900002}{1981})
& 4 & 226 \\
Anderson (\citeproc{ref-WOS000078432400003}{1999})
& 7 & 264 \\
Clark and Chalmers (\citeproc{ref-WOS000073222300002}{1998})
& 11 & 307 \\
Cummins (\citeproc{ref-WOSA1975BF60100001}{1975})
& 9 & 275 \\
Singer (\citeproc{ref-WOSA1972Z066400001}{1972})
& 10 & 286 \\
Rawls (\citeproc{ref-WOSA1980KH88100001}{1980})
& 4 & 182 \\
Cohen (\citeproc{ref-WOSA1989AE70300010}{1989})
& 7 & 230 \\
Kitcher (\citeproc{ref-WOSA1981NA08400001}{1981})
& 6 & 197 \\

\end{longtable}

These tables tell us that the LEMME topics are at the heart of what
\emph{Philosophical Studies} has published over these four decades. I'll
use that fact in the next analysis.

\section{\texorpdfstring{A Topic Model for \emph{Philosophical
Studies}}{A Topic Model for Philosophical Studies}}\label{sec-topic-model}

\subsection{The Model}\label{the-model}

Following the groundbreaking work of Christophe Malaterre and
colleagues, I analysed the content of \emph{Philosophical Studies}
articles by building a five-topic topic model.\footnote{This technique
  is used to analyse \emph{Philosophy of Science} by Malaterre,
  Chartier, and Pulizzotto (\citeproc{ref-Malaterre2019}{2019}), to
  analyse \emph{Biology and Philosophy} by Malaterre, Pulizzotto, and
  Lareau (\citeproc{ref-Malaterre2019b}{2019}), and to analyse a family
  of eight philosophy of science journals by Malaterre et al.
  (\citeproc{ref-Malaterre2020}{2020}). In Malaterre and Lareau
  (\citeproc{ref-Malaterre2022}{2022}) the technique is extended to
  journals that don't publish exclusively in one language, but that's
  not relevant to \emph{Philosophical Studies}.} Without going too deep
into the details of what topic modeling is, I'll note four features of
it that are relevant.

First, a topic model divides some texts into \emph{k} topics, where
\emph{k} is given by the model builder. On the one hand, the model
itself doesn't tell you how many topics to use. On the other hand, the
model builder doesn't tell the model what those topics should be. I
thought a five topic model should work because \emph{Philosophical
Studies} concentrates on the LEMME topics. After asking the model to
divide the articles into five topics, it more or less found those five
groups. (Though with some caveats that will become clearer in what
follows.) The model doesn't give names to those topics; that's something
I did. But it does create the five groupings.

Second, the model takes as input the words in an article, but not the
order those words appear in. For this reason, it is poorly suited to
determining which side of a debate an article is on. Articles on
utilitarianism use different words to articles on mental content;
articles for and against utilitarianism tend to use similar words. So
the model isn't much help at telling us whether articles in
\emph{Philosophical Studies} are for or against the topic being
discussed; for that we have to read the articles.

Third, the model doesn't simply say that an article is in one or other
topic. For each of the five topics, it gives a probability to the
article being in that topic. This is helpful for classifying articles
that don't slot cleanly into one or other group. \emph{Philosophical
Studies} has published a lot of papers on physicalism. For many of
these, it would be arbitrary to call them mind articles or metaphysics
articles. The model doesn't require that we choose. It gives a
probability to the article being in metaphysics, and a probability to it
being in mind. In some cases, such as ``Chalmers on the Addition of
Consciousness To the Physical World''
(\citeproc{ref-WOS000086712500005}{Latham 2000}), the model gives almost
equal probability to the two topics.

Often when topic models are used, articles are classified by their
maximal probability. I prefer to use the actual probabilities. So the
graph in Figure~\ref{fig-five-topics} shows the average probability that
an article is in each topic by year. I think it shouldn't matter much
whether Latham's article is counted as 51\% mind and 49\% metaphysics or
vice versa, and using the probabilities allows for that.

Finally, the models are rather random. Somewhat metaphorically, they
work by picking a random starting point, and scanning for a local
maximum. The scanning procedure is deterministic, but the starting point
is not. This can make a difference to individual articles. Every model I
built was unsure how to classify ``What Do Philosophers Believe?''
(\citeproc{ref-WOS000340619100006}{Bourget and Chalmers 2014}). The one
I'm using said it was 50\% metaphysics, and 50\% spread between the
other four topics, but other models said it was 60\% mind, or largely
ethics. All this makes sense; the article cuts across all the fields.

But this randomness raises a problem: how do we choose which random
starting point to use? We could just use the first one we run, but that
might be a very idiosyncratic output. We could hand code the starting
point, but that would introduce our own biases. We could build many
models and just which one looks most like familiar topics, but that
would just introduce different biases.

Here I got around this problem by using citation data. I had the
computer build (or at least start building) several thousand models. I
told it to focus on the ones where articles were classified in a similar
way to articles that they cited. In general, the greater the similarity
between the classification of articles and classification of citations
of those articles, the more the model resembled familiar topics. So
rather than using my own judgment about the best model\footnote{If I had
  used my own judgment, I would have picked a model similar to, but not
  quite identical to, this one. But I'd rather trust citation data than
  my idiosyncratic judgments.}, I used the one that best conformed to
the citation data.\footnote{There was one real oddity in this model. It
  put articles on phenomenal properties in metaphysics, and articles on
  dispositions in mind. Most other models did not do this, and I would
  have preferred this one didn't either. But overall, this model did
  best at co-classifying articles and articles they cite, and every
  model is odd in some respect, so I stuck with this model.}

\subsection{The Five Topics}\label{sec-five-topics}

I picked five topics for the model hoping, correctly, that it would land
on something like the LEMME topics. And that's more or less what
happened.

I labeled the topics, and mostly the labels are straightforward. The
only one of those labels that might be misleading is `ethics'. In the
model, that topic includes all papers on norms broadly construed. So it
includes papers on decision theory\footnote{I think this is the right
  way to classify decision theory; it could sensibly be called `formal
  ethics'. But I'm not relying on my views here, just noting that the
  model was remarkably confident that decision theory papers go in
  ethics.}, on political philosophy (though there aren't a lot of
those), and even some papers on epistemic norms. So Kelly's ``The
Rationality of Belief and Some Other Propositional Attitudes''
(\citeproc{ref-WOS000178572700004}{Kelly 2002}) is classified as 50\%
ethics and 50\% epistemology.\footnote{The model I'm using actually had
  it at 51\% ethics. Most models I looked at had it around 51/49 the
  other direction, but we shouldn't treat this difference as meaningful.}
Other articles which seem similar to me are classified as 99\% or more
epistemology. This model (and all the others I looked at) put any
article with explicit discussion of norms at least partially in ethics.

What happens to articles that are not from the LEMME topics? The simple
answer is that they are mostly not published in \emph{Philosophical
Studies}. That's part of what we learn from
Table~\ref{tbl-mainly-out-ps}. If there were more political philosophy
papers, for example, we would see more citations to Rawls or Anderson.

Philosophy of science raises a complication though. There is a good
sense in which \emph{Philosophical Studies} publishes a fair bit of
philosophy of science; it publishes many papers that could easily appear
in \emph{Philosophy of Science}. Despite that, the five way
classification used here makes sense. The important thing is that the
philosophy of science articles that \emph{Philosophical Studies} publish
also fall into one of those five topics. So there are a lot of articles
on Bayesian epistemology, e.g., ``Belief and the Problem of Ulysses and
the Sirens'' (\citeproc{ref-WOSA1995QE80400001}{van Fraassen 1995}), but
it makes sense to count them in epistemology. There are also a lot of
articles on the metaphysics of science, e.g., ``Realism,
Anti-foundationalism and the Enthusiasm for Natural Kinds''
(\citeproc{ref-WOSA1991FC38500010}{Boyd 1991}), but we don't lose much
if we count these as metaphysics. What \emph{Philosophical Studies}
doesn't have (or, more precisely, doesn't have much of) are articles
like ``Thinking about Mechanisms''
(\citeproc{ref-WOS000087305900001}{Machamer, Darden, and Craver 2000}),
that are clearly philosophy of science articles but not in one of these
five topics. So we aren't losing much by not having a separate topic for
those articles. And we can see that by looking at the fact that
``Thinking about Mechanisms'', and several other papers like it, are
cited rather rarely in \emph{Philosophical Studies}.

The main use I'll put this model to is looking for changes in
publication rates, and Section~\ref{sec-words-and-trends} will go over
those changes in more detail. I'll end this section by looking at how
these five topics relate to the special issues, and the citation rates
of articles in those five topics.

\subsection{Topics and Special Issues}\label{topics-and-special-issues}

Table~\ref{tbl-topics-special} shows the distribution of the five topics
over the types of issues that \emph{Philosophical Studies} publishes.

\begin{longtable}[]{@{}llllll@{}}

\caption{\label{tbl-topics-special}Distribution of philosophical topics
over types of issue.}

\tabularnewline

\toprule\noalign{}
Type & Epistemology & Mind & Ethics & Language & Metaphysics \\
\midrule\noalign{}
\endhead
\bottomrule\noalign{}
\endlastfoot
APA Pacific & 17.6\% & 21.5\% & 32.6\% & 11.2\% & 17.1\% \\
BSPC & 20.7\% & 6.1\% & 30.5\% & 12.4\% & 30.3\% \\
Normal & 17.5\% & 15.2\% & 23.3\% & 22.3\% & 21.6\% \\
Oberlin & 20.5\% & 35.5\% & 8.5\% & 12\% & 23.5\% \\
One-off & 28.6\% & 22.1\% & 13\% & 19.6\% & 16.8\% \\

\end{longtable}

The differences between Oberlin and BSPC are striking, even if not
surprising. BSPC was primarily an ethics and metaphysics conference,
while Oberlin has focused more on philosophy of mind. The higher numbers
for language in the normal issues is in part a factor of there just
being more normal issues before the mid-1990s, when language was more
central to the journal and perhaps the discipline.

I've noted a few times that you see less epistemology in the journal
than you might expect given the prominence of the editors in
epistemology. Here we do see an effect. The one-off issues, where the
editors have more say over the topic, lean more towards epistemology
than the rest of the journal.

\subsection{Topics and Citations}\label{topics-and-citations}

Table~\ref{tbl-topics-cites} shows how often, on average, articles in
the different topics were cited.

\begin{longtable}[]{@{}lrrr@{}}

\caption{\label{tbl-topics-cites}Average citation rates for the five
main topics.}

\tabularnewline

\toprule\noalign{}
Topic & Articles & Citations & Rate \\
\midrule\noalign{}
\endhead
\bottomrule\noalign{}
\endlastfoot
Epistemology & 722.8 & 7033.0 & 9.73 \\
Ethics & 899.0 & 6291.5 & 7.00 \\
Language & 746.0 & 4985.5 & 6.68 \\
Metaphysics & 775.5 & 7179.4 & 9.26 \\
Mind & 628.6 & 5101.5 & 8.12 \\

\end{longtable}

The differences are not as stark as in Table~\ref{tbl-citation-by-type},
but they are still notable. Epistemology articles are cited, on average,
about 50\% more often than language articles. Now a simple explanation
for that might be that epistemology articles appear on average later,
and articles that appear later get cited more often. As
Table~\ref{tbl-topics-cites-decades} shows, that is partially right. But
breaking up the citations by decades reveals something suprising.

\begin{longtable}[]{@{}lrrrr@{}}

\caption{\label{tbl-topics-cites-decades}Average citation rates for the
five main topics by decade.}

\tabularnewline

\toprule\noalign{}
Topic & 1980s & 1990s & 2000s & 2010s \\
\midrule\noalign{}
\endhead
\bottomrule\noalign{}
\endlastfoot
Epistemology & 7.94 & 8.89 & 13.90 & 8.32 \\
Ethics & 6.03 & 6.61 & 10.40 & 5.89 \\
Language & 5.41 & 5.53 & 10.73 & 5.45 \\
Metaphysics & 6.94 & 9.25 & 12.29 & 8.61 \\
Mind & 6.01 & 9.52 & 9.97 & 7.16 \\

\end{longtable}

The difference between the 2000s and every other decade in
Table~\ref{tbl-topics-cites-decades} is stark. But the really striking
thing is the lack of citations to language articles before the 2000s.
I'll come back to this point in Section~\ref{sec-1980s-articles}, as
part of a broader look at how the topic distribution in
\emph{Philosophical Studies} changed over time.

\section{Words and Trends}\label{sec-words-and-trends}

Four of the five topics retain a stable stable share of papers in
\emph{Philosophical Studies} over time, while philosophy of language
falls away dramatically. Figure~\ref{fig-language} isolates the language
panel from Figure~\ref{fig-five-topics}, and adds a trendline, to make
this more visible.

\begin{figure}

\centering{

\pandocbounded{\includegraphics[keepaspectratio]{phil-studies_files/figure-pdf/fig-language-1.pdf}}

}

\caption{\label{fig-language}Proportion of \emph{Philosophical Studies}
articles in philosophy of language.}

\end{figure}%

If you don't trust black box methods like topic modeling, you can see a
similar trend just in the word usage. Figure~\ref{fig-frege-russell}
shows the frequency of the words ``description'' and ``reference'' over
time.

\begin{figure}

\centering{

\pandocbounded{\includegraphics[keepaspectratio]{phil-studies_files/figure-pdf/fig-frege-russell-1.pdf}}

}

\caption{\label{fig-frege-russell}Frequency of the words `description'
and `reference' in \emph{Philosophical Studies}.}

\end{figure}%

These trends in word usage are a more fine-grained guide to what is
happening in \emph{Philosophical Studies} than the five topic model. I
did use topic modeling to build many more fine-grained models for the
journal, but there was too much variation between the models to use any
one model as the basis for this section. Instead I'll look at word
frequency, and how it relates to citation patterns.

The main study I'll talk about looks at the relative frequency of the
roughly 6000 most common word types over the four decades from
1980-2019. For each word, and each decade, I'll look at how improbable
it is that the word is used that often in that decade, given how many
words the journal printed that decade, and how often the word is used
overall.

This is easiest to work through with an example. The word
``contextualism'' is used 2282 times between 1980 and 2019, with 984 of
those uses in the 2000s. Overall, 26.48\% of the words published are in
issues from the 2000s. A random process with 2282 trials (one for each
usage), and a `success' probability 0f 26.48\%, has a probability of
roughly 10\textsuperscript{-65} of succeeding 984 or more times. That
is, if the uses of `contextualism' were distributed at random over the
journal, the probability at least that many would appear in the 2000s is
10\textsuperscript{-65}. Not surprisingly, we get the result papers
using ``contextualism'' are concentrated in the 2000s. I'll use these
probabilities as measures of concentration in what follows. When the
decade is clear, I'll simply write the probability after the word, e.g.,
contextualism (10\textsuperscript{-65}), to indicate how concentrated it
is.

\subsection{Three Initial
Observations}\label{three-initial-observations}

Before getting into words that tell us about changes in the subject
matter, I want to briefly note three other things we can see in the word
count.

The JSTOR set includes all the words in the article, including headers,
footnotes, and bibliography. \emph{Philosophical Studies} gradually
moved from footnote citations to author-text plus bibliography. So `op'
and `cit' appear primarily in the 1980s dataset. More strikingly,
``Oxford'' is concentrated in the 2010s, with 17159 of its 24239
appearances being in the 2010s. The probability of that happening by
chance is less than 10\textsuperscript{-300}. This is in part a result
of longer bibliographies in the 2010s, in part a consequence of the
proliferation of Oxford Handbooks and Oxford Studies publications, but
largely a sign of the dominance of philosophy book publishing in the
2010s by Oxford University Press.

Another word that appears a lot in the 2010s is ``thanks''
(10\textsuperscript{-60}). As Eugenio Petrovich
(\citeproc{ref-Petrovich2024}{2024}) notes, one pattern in the 2010s is
that acknowledgements footnotes became more frequent, and more informal.
So papers will cite ``Theodore Sider'' and thank ``Ted Sider''. And they
will not start with ``I am grateful to\ldots{}'' but simply ``Thanks
to\ldots{}''.

Finally, the trend in gender terms is remarkable. The word most
concentrated in the 1980s (by the probability measure I'm using) is
`which'; I think this just reflects stylistic changes. The second most
concentrated is `he'. Figure~\ref{fig-gender-words} shows the frequency
over time of male (he/him/his) and female (she/her/hers) pronouns over
the years.

\begin{figure}

\centering{

\pandocbounded{\includegraphics[keepaspectratio]{phil-studies_files/figure-pdf/fig-gender-words-1.pdf}}

}

\caption{\label{fig-gender-words}Frequency of gendered words in
\emph{Philosophical Studies}.}

\end{figure}%

I'll leave it for someone else to figure out what proportion of the
change in Figure~\ref{fig-gender-words} is due to there being more
references to particular women, and what proportion is due to
philosophers using female pronouns for arbitrary or fictional
characters.

\subsection{1980s}\label{sec-1980s-articles}

From the 1990s to at least the mid 2010s, the bulk of ethics papers in
\emph{Philosophical Studies} were on fairly theoretical questions, with
special focus on questions moral realism, moral motivation, and (a topic
I'll return to) moral responsibility. But in the 1980s the ethics papers
in the journal had a different focus.

There were several papers on utilitarianism. So there was a lot of
discussion of ``utility'' (10\textsuperscript{-15}), ``utilitarianism''
(10\textsuperscript{-41}) and ``utilitarian'' (10\textsuperscript{-45}).
One can use almost any philosophical term in a paper attacking the
theory it names; indeed the 1980s paper with the most uses of
``Kantian'' (\citeproc{ref-WOSA1989CE37600004}{Terzis 1989}) is an
anti-Kantian paper. But most of these papers were supporting
utilitarianism. (Some of the uses of ``utility'', both in the 1980s and
afterwards, were in decision theory papers.)

There were also many papers on abortion.\footnote{One of these,
  (\citeproc{ref-WOSA1985ANT6600005}{Gensler 1985},
  \citeproc{ref-WOSA1986AZA8100007}{1986}), got printed twice because
  the pages were out of order the first time.} These were largely
\emph{anti}-abortion arguments. They did not get much uptake in
philosophy journals. The nine\footnote{Or ten, if you count both
  printings of the Gensler piece.} papers from the 1980s that use
``abortion'' five times or more average less than one citation per
article in philosophy journals. Google Scholar shows that some of them
have more uptake elsewhere, especially in specialist bioethics journals.
After the 1980s, \emph{Philosophical Studies} largely lost interest in
this topic. Over the last forty years, ``A Defense of Abortion''
(\citeproc{ref-WOSA1971Y116900003}{Thomson 1971}) is cited less often in
\emph{Philosophical Studies} than you'd expect given its citation rate
elsewhere.

The big story of the 1980s is the centrality of philosophy of language
to the journal, and how different this is to what the journal became in
later years. The kind of philosophy of language \emph{Philosophical
Studies} published in the 1980s (and 1990s) was centered around issues
about reference and description that became central to philosophy over
the 1960s and 1970s.

Part of the evidence for that consists of which words were central to
the journal in the 1980s. There was a lot of discussion of Donnellan's
distinction between referential and attributive uses of definite
descriptions. The words ``Donnellan'' (10\textsuperscript{-72}),
``Donnellan's'' (10\textsuperscript{-58}), ``referential''
(10\textsuperscript{-34}) and ``attributive'' (10\textsuperscript{-23})
appear much more often than they do in later decades. The 21 articles
that use ``Donnellan's'' five times or more are cited on average 15
times, which is a relatively high number given their publication year.
Strikingly, over 25\% of those citations are in \emph{Philosophical
Studies}. That shows discussion of this topic was playing a bigger role
in \emph{Philosophical Studies} than it was in the wider literature.

More generally, the story here is that while \emph{Philosophical
Studies} was publishing a lot of philosophy of language, it was not
publishing the kind of philosophy of language that would become central
to the next couple of decades. Ultimately, this is why the language
articles it published in the 1980s and 1990s have relatively few
citations. I'll quickly go over three kinds of articles that rarely
appear in \emph{Philosophical Studies} : articles on conditionals;
articles on speech acts; and interdisciplinary articles.

Despite the prominence of philosophy of language before 2000, the words
``conditional'', ``conditionals'' and ``counterfactuals'' appear less
frequently these decades than in later decades. There are no
\emph{Philosophical Studies} articles cited in Dorothy Edgington's
influential and comprehensive ``On Conditionals''
(\citeproc{ref-WOSA1995QX94800001}{Edgington 1995}).

Similarly, there was very little on speech acts, or on context
sensitivity. The words ``assertion'' and ``Austin'' appear at lower than
average frequency before 2000, as do all the cognates of ``context''.

Perhaps most significantly, despite the centrality of philosophy of
language, there was very little interaction with linguistics, and
especially not with linguists working on formal semantics. Here it helps
to look at the 1980s and 1990s together, because this trend continues
through the turn of the century. Before 2000, only one paper mentioned
Irene Heim (\citeproc{ref-WOSA1988P180000006}{King 1988}), only two
mentioned Angelika Kratzer (\citeproc{ref-WOSA1986D182100003}{Cross
1986}; \citeproc{ref-WOSA1984SK63200001}{Tichý 1984}), and only three
mentioned James McCawley (\citeproc{ref-WOSA1985ABV7900005}{Asher and
Bonevac 1985}; \citeproc{ref-WOSA1988M482100005}{Blackburn 1988};
\citeproc{ref-WOS000083809100001}{Ostertag 1999}). As you can see from
those dates, if anything there was even less engagement with linguists
in the 1990s than there had been in the 1980s. There were more mentions
of Chomsky, though they were mostly in passing, and as often in
philosophy of mind papers as philosophy of language. Similarly, there
were a few mentions of George Lakoff, but again not always in philosophy
of language. The closest thing I see to interdisciplinary work are some
applications of Montague grammar, e.g., by
(\citeproc{ref-WOSA1984TJ24700012}{Hazen 1984}). But there is a much
sharper division between philosophers of language and linguists working
on semantics in \emph{Philosophical Studies} before 2000 than you see in
more recent work in philosophy of language.

In Figure~\ref{fig-1980s-words-big} and
Figure~\ref{fig-1980s-words-small} I show the frequency of some of the
words mentioned in this subsection over time. I've put this in two
graphs so the movement in the less frequently used words is easier to
see. And as I will do in later sections, I've shaded the decade under
discussion.

\begin{figure}

\centering{

\pandocbounded{\includegraphics[keepaspectratio]{phil-studies_files/figure-pdf/fig-1980s-words-big-1.pdf}}

}

\caption{\label{fig-1980s-words-big}Words that appear more often in the
1980s.}

\end{figure}%

\begin{figure}

\centering{

\pandocbounded{\includegraphics[keepaspectratio]{phil-studies_files/figure-pdf/fig-1980s-words-small-1.pdf}}

}

\caption{\label{fig-1980s-words-small}Words that are concentrated in the
1980s.}

\end{figure}%

\subsection{1990s}\label{sec-1990s-articles}

There are two notable trends in the 1990s. First, as well as many
articles on the nature of linguistic content, there are many articles on
the nature of mental content. Second, there is a major focus, across
many different parts of philosophy, on issues about causation.

As Figure~\ref{fig-frege-russell} shows, there is still a lot of talk
about `description' and `reference' through the 1990s. What's new is the
focus on mental content. So as well as continued discussion of Hesperus
and Phosphorus, there is now a lot of discussion of arthritis, mostly in
response to Burge (\citeproc{ref-WOSA1986AYX3200001}{1986}). There is
more discussion of Fodor and mentalese than before or after.

Looking back, a lot of this discussion feels stale by the mid-1990s.
There is a sense that the important things to say have been said. This
shows up in the citation data. The \emph{Philosophical Studies} papers
using these terms have a disproportionately high rate of citation in
\emph{Philosophical Studies} itself. The rest of the field was moving
on, but \emph{Philosophical Studies} was a step behind.

Even the giants of the field weren't immune to this change in
philosophical fashion. Fodor published \emph{The Elm and the Expert} in
1995. It has around 500 citations on Google Scholar. Most of us would be
happy with that kind of recognition, but by Fodor's lofty standards it
was a flop. \emph{Holism} (\citeproc{ref-FodorLepore1992}{Fodor and
Lepore 1992}) has over 1500 Google Scholar citations; \emph{Concepts}
(\citeproc{ref-Fodor1998}{Fodor 1998}) has nearly 3000. Content
externalism was not as central to philosophy in the 1990s as it had been
in earlier decades.

The other big trend in \emph{Philosophical Studies} this decade is
causation. The word `causal' (10\textsuperscript{-68}) is heavily
concentrated in this decade. This is for three reasons. First, there
were lots of papers simply on causation. Second, there were more uses of
causal notions in analysing other concepts. This was hardly unique to
the 1990s, but it was more frequent then.

But the biggest factor is the discussion of mental causation in general,
and Jaegwon Kim's causal exclusion argument in particular. Dozens of
papers in \emph{Philosophical Studies} either centered on, or at least
touched on, these issues. Unlike the topics I've discussed so far, here
the topic rose to prominence in \emph{Philosophical Studies} at the same
time it was becoming central in the discipline. The articles using words
like `Kim', `exclusion' and `causal' are being cited widely outside of
\emph{Philosophical Studies}.

Kim himself published two articles that discussed causation at length in
\emph{Philosophical Studies} in the 1990s.\footnote{These were both in
  special issues.} ``Making Sense of Emergence''
(\citeproc{ref-WOS000082592000002}{Kim 1999b}) surveys classical
emergentist theories, arguing that they get into troubles with
causation. And ``Hempel, Explanation, Metaphysics''
(\citeproc{ref-WOS000081337700001}{Kim 1999a}) is a lovely appreciation
of Kim's graduate supervisor, Carl Hempel, also stressing the importance
of causation for thinking about explanation.

The causal exclusion argument has a strange history after the 1990s. By
the late 2000s it had largely faded from philosophy of mind. This was in
part because it was an in-house dispute between physicalists, and
Chalmers (\citeproc{ref-Chalmers1996}{1996}) had made the prior question
of whether physicalism was true more pressing. And it was in part
because of various objections, Bennett
(\citeproc{ref-WOS000184542900004}{2003}) is particularly important
here, to the stronger forms of Kim's arguments. But while discussions of
whether mental causation was possible (on various theories) faded away,
there was a lot of interest elsewhere in the question of when mental
states are really causes. This was important for thinking about the
broader case of causation throughout the special sciences, and of
mechanistic causation/explanation in particular. In particular, ``Mental
Causation'' (\citeproc{ref-WOSA1992JA62400001}{Yablo 1992}) continues to
be widely cited, though now more often in philosophy of science for its
discussion of proportionality, rather than in discussions directly about
the metaphysics of mind. Since \emph{Philosophical Studies} publishes
little on philosophy of special sciences, or on mechanistic explanation,
this won't be part of our story going forward.

\begin{figure}

\centering{

\pandocbounded{\includegraphics[keepaspectratio]{phil-studies_files/figure-pdf/fig-1990s-words-big-1.pdf}}

}

\caption{\label{fig-1990s-words-big}Words that appear more often in the
1990s.}

\end{figure}%

\begin{figure}

\centering{

\pandocbounded{\includegraphics[keepaspectratio]{phil-studies_files/figure-pdf/fig-1990s-words-small-1.pdf}}

}

\caption{\label{fig-1990s-words-small}Words that are concentrated in the
1990s.}

\end{figure}%

\subsection{2000s}\label{sec-2000s-articles}

Philosophers have long been fond of named ideologies, and labels for
proponents of these ideologies. The earliest years of philosophy
journals are full of disputes between idealists, realists, pragmatists,
and positivists. These labels can obscure as much as they reveal.
Indeed, another theme of those early years are long disputes about just
what idealism, realism, pragmaticism, and positivism, really are.

This trend intensified in the 2000s, with a twist. Now the `isms' most
used tended to be words not in wide circulation just a few years
earlier. In some cases that's because an old view got a new name. In
some cases, it's because a new view appeared on center stage.
Figure~\ref{fig-2000s-isms} shows the usage of three such words:
``contextualism'', ``fictionalism'', and ``physicalism'', as well as the
nouns/adjectives for proponents of those views.\footnote{A similar graph
  for the 2010s would highlight ``expressivism'', ``monism'', and
  ``presentism''.}

\begin{figure}

\centering{

\pandocbounded{\includegraphics[keepaspectratio]{phil-studies_files/figure-pdf/fig-2000s-isms-1.pdf}}

}

\caption{\label{fig-2000s-isms}Frequently discussed ideologies in the
2000s.}

\end{figure}%

The discussions of fictionalism and fictionalists are much more
concentrated: all these usages are in just 14 articles. But there are 70
articles that talk about contextualism, and 72 articles that talk about
physicalism. In both of the last two cases, that means the term is used
in about 8\% of all articles that decade. In both cases, that's a
dramatic increase on previous usage. The term `contextualism' appears
more often in ``Contextualism, Externalism and Epistemic Standards''
(\citeproc{ref-WOS000168853300001}{Williams 2001}) than it had appeared
in the history of the journal to that point.

As well as the novelty, there are two things worth noting about this
rise. One is that these were, like with mental causation in the 1990s,
debates where \emph{Philosophical Studies} was publishing pieces that
became central to the literature. This was especially true for debates
about contextualism, where papers like ``Contextualism and Warranted
Assertibility Manoeuvres'' (\citeproc{ref-WOS000241143100001}{Brown
2006}), ``From Contextualism To Contrastivism''
(\citeproc{ref-WOS000222384400005}{Schaffer 2004}) and ``Knowledge
Claims and Context: Loose Use'' (\citeproc{ref-WOS000245280900001}{Davis
2007}) quickly became widely cited.

This leads to the big story of the 2000s, the rise of epistemology.
According to the topic model, epistemology went from being 15\% of the
journal over the previous two decades, to being 20\% in the 2000s, and
then rising again in the 2010s. In the 2000s, this was mostly because of
discussions about knowledge. (In the 2010s, the focus would shift more
to justification.) The center of that were debates about contextualism.
``Elusive Knowledge'' (\citeproc{ref-WOSA1996VY21200001}{Lewis 1996})
was cited in 3.3\% of all articles that decade, and ``Solving the
Sceptical Problem'' in 2.9\% of all articles. That's a higher rate than
any other article was cited over a decade. The next highest rate is that
``The Skeptic and the Dogmatist''
(\citeproc{ref-WOS000165361800002}{Pryor 2000}) is cited in 2.8\% of
articles in the 2010s, as the focus moved over to justification.

Figure~\ref{fig-2000s-epistemology-big} shows the increase in word usage
for some words primarily used in epistemology articles. The notable
thing about these graphs is the small number of dots at the lower end of
the scale during the 2000s. What drives the increase here is not a
handful of articles using the term repeatedly, but that in the 2000s,
unlike other decades, the words get used year after year.

\begin{figure}

\centering{

\pandocbounded{\includegraphics[keepaspectratio]{phil-studies_files/figure-pdf/fig-2000s-epistemology-big-1.pdf}}

}

\caption{\label{fig-2000s-epistemology-big}Terms from epistemology that
appear more often in the 2000s.}

\end{figure}%

\emph{Philosophical Studies} isn't the only journal that sees a shift to
epistemology at this time. Petrovich
(\citeproc{ref-Petrovich2024}{2024}) looked at citation data from
\emph{Mind}, \emph{Noûs}, \emph{Philosophical Review}, \emph{Journal of
Philosophy} and \emph{Philosophy and Phenomenological Research}, and
concluded that there was a marked uptick in epistemology articles being
written, and cited, in the 2000s. His study wasn't restricted to
citations of journal articles, like this study is. That's relevant not
because there are so many epistemology books that get cited in that
time, but rather because his study lets us see when books like
\emph{Word and Object} and \emph{Naming and Necessity} start getting
cited less, and articles like ``Elusive Knowledge'' start getting cited
more. As with \emph{Philosophical Studies}, that seems to happen in the
2000s.

\subsection{2010s}\label{sec-2010s-articles}

The big story of the 2010s is the rise of social philosophy. This isn't
specific to \emph{Philosophical Studies}. Indeed, none of the stories
I'll tell here are particularly unique to \emph{Philosophical Studies};
everything that happens in the journal reflects wider trends in the
discipline.\footnote{The reverse isn't always true, especially outside
  the LEMME topics there are changes that aren't reflected in
  \emph{Philosophical Studies}.} I'll start with some more superficial
results, and end with social philosophy.

I've already mentioned that get longer, and make more references to
``Oxford'' over this decade. Figure~\ref{fig-citations-by-year} shows
how many more citations there are to articles in philosophy journals.

\begin{figure}

\centering{

\pandocbounded{\includegraphics[keepaspectratio]{phil-studies_files/figure-pdf/fig-citations-by-year-1.pdf}}

}

\caption{\label{fig-citations-by-year}Citations per article to other
philosophy journals in \emph{Philosophical Studies} 1980-2019.}

\end{figure}%

The graph isn't just going up, it is \emph{accelerating}. The data I
have from the 2020s suggests this trend does not stop.

\emph{Philosophical Studies} was founded to be a journal of analytic
philosophy. By the 2010s, what it meant to be doing analytic philosophy
had changed. It was less about analysis, and more about continuity with
a tradition. That continuity is still there; there are no radical breaks
in the citation patterns. But as Figure~\ref{fig-analytic-words} shows,
some keywords associated with analytic philosophy are used less and
less.

\begin{figure}

\centering{

\pandocbounded{\includegraphics[keepaspectratio]{phil-studies_files/figure-pdf/fig-analytic-words-1.pdf}}

}

\caption{\label{fig-analytic-words}Keywords of analytic philosophy}

\end{figure}%

There is an uptick in ``logic'' in 2019, but the trend in
Figure~\ref{fig-analytic-words} looks clear. Some of this could be
fashion. I don't think Figure~\ref{fig-theory-words} shows that
\emph{Philosophical Studies} authors are less interested in theories,
but rather than the words they use for theories are changing.

\begin{figure}

\centering{

\pandocbounded{\includegraphics[keepaspectratio]{phil-studies_files/figure-pdf/fig-theory-words-1.pdf}}

}

\caption{\label{fig-theory-words}How to refer to theories}

\end{figure}%

The pattern in Figure~\ref{fig-theory-words} extends to negative terms
as well. I had associated robust challenges more with football than
philosophy, but Figure~\ref{fig-objection-words} suggests this is an
outdated view.

\begin{figure}

\centering{

\pandocbounded{\includegraphics[keepaspectratio]{phil-studies_files/figure-pdf/fig-objection-words-1.pdf}}

}

\caption{\label{fig-objection-words}How to critcise theories}

\end{figure}%

This last graph perhaps reveals a deeper point than the previous ones.
One reason for greater use of ``robust'' is that \emph{Philosophical
Studies} increased the amount of empirical, inter-disciplinary work that
it published, especially in philosophy of mind.
Figure~\ref{fig-empirical-words} shows some other words that reveal this
pattern.

\begin{figure}

\centering{

\pandocbounded{\includegraphics[keepaspectratio]{phil-studies_files/figure-pdf/fig-empirical-words-1.pdf}}

}

\caption{\label{fig-empirical-words}Words associated with empirical
work}

\end{figure}%

This empirical turn comes with a more dramatic than usual turnover in
which philosophers are frequently mentioned in the journal. I'm just
going to use a crude method here of counting tokens, and not try to
distinguish philosophers with the same last name. I'll mostly stick to
names which, in \emph{Philosophical Studies} at least, almost
exclusively pick out one philosopher. That said, I've included
``Schroeder'' because both Mark and Tim are important to the story of
the decade. For names it's distracting to have dots for each year, the
data is too noisy, so I've included Table~\ref{tbl-names-early-late}
showing how often the name is used before and after 2010. (There are
roughly 90\% as many words published from 1980-2009 as 2010-2019, so
this is almost a half-way point.)

\begin{longtable}[]{@{}lrr@{}}

\caption{\label{tbl-names-early-late}Uses of some names before and after
2010.}

\tabularnewline

\toprule\noalign{}
Name & 1980-2009 & 2010-2019 \\
\midrule\noalign{}
\endhead
\bottomrule\noalign{}
\endlastfoot
Sellars & 1348 & 161 \\
Chisholm & 1128 & 272 \\
Putnam & 1675 & 469 \\
Davidson & 2080 & 703 \\
Quine & 2146 & 831 \\
Wittgenstein & 765 & 302 \\
Aristotle & 1291 & 567 \\
Plantinga & 860 & 406 \\
Searle & 926 & 449 \\
Fodor & 1713 & 850 \\
Sider & 654 & 1303 \\
Hawthorne & 412 & 1188 \\
Huemer & 117 & 683 \\
Schaffer & 239 & 1477 \\
Siegel & 51 & 591 \\
Schroeder & 120 & 1472 \\
Cappelen & 31 & 522 \\
Enoch & 19 & 470 \\
Schwitzgebel & 7 & 501 \\
Schellenberg & 3 & 485 \\

\end{longtable}

I don't know a good way to measure turnover in who is being talked
about, but this feels like a more dramatic change than one sees most
decades.

The biggest story in metaphysics this decade was the end of what Sider
(\citeproc{ref-Sider2020}{2020}) calls the `modal era' in metaphysics.
From the late 1960s to the late 2000s, the most active discussions in
metaphysics were about ontology and modality. The most notable book of
this period, \emph{On the Plurality of Worlds}
(\citeproc{ref-Lewis1986a}{Lewis 1986}) was about both. From 2010
onwards, the focus shifted to discussions of fundamentality, priority,
and, most of all, grounding. As Figure~\ref{fig-grounding} shows,
\emph{Philosophical Studies} was definitely part of this shift.

\begin{figure}

\centering{

\pandocbounded{\includegraphics[keepaspectratio]{phil-studies_files/figure-pdf/fig-grounding-1.pdf}}

}

\caption{\label{fig-grounding}Uses of grounding in \emph{Philosophical
Studies}.}

\end{figure}%

Although there was less discussion of grounding before 2010, there were
several important, and widely cited, papers on grounding published in
\emph{Philosophical Studies} in the first part of the decade
(\citeproc{ref-WOS000341924300014}{Nolan 2014};
\citeproc{ref-WOS000325717500012}{Bliss 2013};
\citeproc{ref-WOS000289572300004}{Bennett 2011}). This discussion was
mainly confined to metaphysics. At the start of the modal era, modal
notions became commonplace across philosophy. That's mostly not happened
with grounding. That said, \emph{Philosophical Studies} has published
papers by Beddor (\citeproc{ref-WOS000356355000009}{2015}) and Wodak
(\citeproc{ref-WOS000468369800007}{2019}) that apply grounding ideas to
debates in epistemology.

Over the 2010s, the space between ethics and epistemology continued to
shrink. This wasn't an entirely new phenomenon; ``Why Be Rational?''
(\citeproc{ref-WOS000231037900002}{Kolodny 2005}) was arguably the most
cited ethics paper of the 2000s. But the focus on questions common to
ethics and epistemology, particularly about the nature of reasons, meant
the two fields were more entwined. Figure~\ref{fig-normative-reasons}
shows how dramatic this was over the 2010s.

\begin{figure}

\centering{

\pandocbounded{\includegraphics[keepaspectratio]{phil-studies_files/figure-pdf/fig-normative-reasons-1.pdf}}

}

\caption{\label{fig-normative-reasons}Uses of normative and reasons in
\emph{Philosophical Studies}.}

\end{figure}%

The scale of Figure~\ref{fig-normative-reasons} is larger than any of
the previous graphs. That's not surprising for `reasons'; it's a word
you can use when not writing a paper on reasons themselves. It's more
surprising for `normative'.

The biggest story of the 2010s though, is the social turn across
philosophy. The decade saw a growing emphasis on social epistemology,
social metaphysics and social philosophy of language.
Figure~\ref{fig-social} displays the simplest way to measure this
growth: how often the word `social' is used.

\begin{figure}

\centering{

\pandocbounded{\includegraphics[keepaspectratio]{phil-studies_files/figure-pdf/fig-social-1.pdf}}

}

\caption{\label{fig-social}Uses of social in \emph{Philosophical
Studies}.}

\end{figure}%

Figure~\ref{fig-social} might be misleading unless you know how
philosophers use `social'.\footnote{I'm indebted here to Ishani Maitra.}
Almost anything involving multiple people will be classed as social.
There isn't a distinction here between interpersonal, or what Darwall
(\citeproc{ref-Darwall2006}{2006}) calls second-personal, phenomena, and
phenomena that only emerge at the level of large societies. So the
increased discussion of reactive attitudes, especially blame, is part of
this social turn. I'm not sure this is optimal; society-level phenomena
are rather different from phenomena that arise in groups of two or
three. But that's how the terminology has developed, and I'll follow
common practice here.

The social turn can be seen in the word usage graphs for many different
kinds of words. Figure~\ref{fig-social-general},
Figure~\ref{fig-social-ethics}, and Figure~\ref{fig-social-epistemology}
display some notable examples.

\begin{figure}

\centering{

\pandocbounded{\includegraphics[keepaspectratio]{phil-studies_files/figure-pdf/fig-social-general-1.pdf}}

}

\caption{\label{fig-social-general}Uses of terms for groups.}

\end{figure}%

\begin{figure}

\centering{

\pandocbounded{\includegraphics[keepaspectratio]{phil-studies_files/figure-pdf/fig-social-ethics-1.pdf}}

}

\caption{\label{fig-social-ethics}Uses of terms associated with
interpersonal ethics.}

\end{figure}%

\begin{figure}

\centering{

\pandocbounded{\includegraphics[keepaspectratio]{phil-studies_files/figure-pdf/fig-social-epistemology-1.pdf}}

}

\caption{\label{fig-social-epistemology}Uses of terms from social
epistemology and social philosophy of language}

\end{figure}%

There are other terms where graphs like this don't work because the term
was so rare before 2010. Between 1980 and 2009, the terms `slur',
`slurs' and `slurring' appeared a total of 10 times. Between 2010 and
2019 they appeared 266 times.

So we have some evidence for a social turn in \emph{Philosophical
Studies}. I'll end with some reasons to think this isn't particularly
distinctive to \emph{Philosophical Studies}.

One reason comes from where articles are cited. On average, articles in
\emph{Philosophical Studies} have about 13.7\% of their citations in
\emph{Philosophical Studies}. But this proportion varies in systematic
ways. For words associated with topics the journal does not concentrate
on, the proportion is much lower. Articles that use the word
``institutions'' five or more times have 11.2\% of their citations in
\emph{Philosophical Studies}. For ``simulation'', the proportion is
10.2\%. But for words associated with topics the journal does publish on
frequently, the proportion is much higher. Articles using the word
``analyticity'' five or more times have 13.7\% of their citations in
\emph{Philosophical Studies}. When a word appears particularly
frequently in one kind of article, this measure is helpful for
distinguishing trends relatively specific to \emph{Philosophical
Studies} from trends in the wider journal ecosystem.

For some topics that are prominent in the 2000s and 2010s, this
proportion is rather high. We see that with `grounding' (14.3\%),
`context-sensitive' (14.9\%), and `incompatibilism' (19.5\%). But for
many of the words in the graphs above, the percentage is lower than
average. Examples of this include `data' (11.9\%), `participants'
(12.2\%), `groups' (11.6\%), `interpersonal' (9.3\%), `racial' (9.9\%),
`gender' (11.1\%) and `deference' (11.2\%). This isn't conclusive, but
suggests that while \emph{Philosophical Studies} is more empirical, and
more social, than it was in previous decades, other philosophy journals
have been changing even more rapidly.

And there are some parts of the social turn that are not particularly
visible in \emph{Philosophical Studies} through 2019. Kristie Dotson's
paper on epistemic violence (\citeproc{ref-WOS000289948200002}{Dotson
2011}) is one of the most cited papers of the 2010s, but it doesn't get
cited in \emph{Philosophical Studies} until 2019
(\citeproc{ref-WOS000460037300009}{Matthes 2019}). It has been cited
more often in \emph{Philosophical Studies} since 2019, but this adds to
the evidence in the previous paragraph that \emph{Philosophical Studies}
didn't move into social philosophy as rapidly as some of its peers.

\section{Conclusions and Further
Questions}\label{conclusions-and-further-questions}

One theme of this essay has been the change in the status of
\emph{Philosophical Studies}. In 1980, it primarily published papers
that responded to ideas introduced in books or higher prestige journals
some time earlier, often several years earlier. A few decades later, it
is publishing papers that set the research agenda others to which other
journals respond. There are two interestingly different stories for how
this could have happened.

Philosophy is notoriously hierarchical. If you're reading this journal
you probably don't need much evidence for this. If you want such
evidence, look back to Table~\ref{tbl-citation-rate}. If
\emph{Philosophical Studies} is playing more of an agenda setting role,
this could be because it has moved up in the hierarchy, or because the
hierarchy is getting flatter. (Or both, these are not exclusive.) When
\emph{Noûs} underwent a similar change a decade or two before
\emph{Philosophical Studies}, this was not a sign of the hierarchy
flattening; \emph{Noûs} just got more prestigious. In this case it's
possible we are seeing a journal ecosystem that is getting more
receptive to ideas from places it had traditionally treated as lower
prestige. But it's also possible that all we're seeing is one journal
move up while others move down. This can't be resolved by just looking
at one journal, and it's an interesting question for future research.

\emph{Philosophical Studies} is often referred to as a generalist
journal. I've presented some reasons here for questioning that label. It
doesn't publish much logic, or history of philosophy, or political
philosophy, or philosophy of various academic subjects, e.g.,
mathematics, biology, economics, history, or music. One question is
whether that should matter. Must one publish articles on philosophy of
history to be a generalist journal? If so, how many such articles must
one publish? If this is a reason for not counting \emph{Philosophical
Studies} as truly generalist, are there \emph{any} generalist journals
in philosophy. If so, what are they, and how would we tell?

But the biggest open questions are ones we cannot answer until we see
how philosophy itself develops. The very short version of the story of
\emph{Philosophical Studies} over these four decades is that it moved
from being centered on core analytic questions about language, to first
covering more epistemology, and then doing more empirical and social
philosophy. These moves tracked broader changes in the discipline; if
anything, the changes in \emph{Philosophical Studies} were less dramatic
than the changes in peer journals. The biggest question, at least by my
lights, is which of the trends of the 2010s will continue in the 2020s
and beyond.

We can make some of these questions more precise. Will metaphysics
continue to be post-modal, or will we see a revival of modal
metaphysics? Will philosophy of mind continue its trend to becoming more
empirical, or will there be a return to armchair approaches? Will
epistemology and ethics continue to move closer together? Will
philosophy of language keep moving closer to linguistics? And, most of
all, will the social turn across so many different parts of philosophy
continue? If it does, will it turn more towards questions about society
as a whole, or will it continue to be based in interpersonal, or
second-personal, phenomena?

One could use the evidence from these four decades of
\emph{Philosophical Studies} to give two contradictory answers to these
questions. On the one hand, the trends, especially since 2014, seem so
dramatic it's hard to imagine them suddenly reversing. (And the
preliminary data we have from the early 2020s backs up this intuition.)
On the other hand, previous trends have ended rather abruptly, without
much obvious warning, and perhaps these trends will as well.

\subsection*{References}\label{references}
\addcontentsline{toc}{subsection}{References}

\phantomsection\label{refs}
\begin{CSLReferences}{1}{0}
\bibitem[\citeproctext]{ref-WOSA1985AKA2200025}
Alchourrón, Carlos E., Peter Gärdenfors, and David Makinson. 1985. {``On
the Logic of Theory Change: Partial Meet Contraction and Revision
Functions.''} \emph{Journal Of Symbolic Logic} 50 (2): 510--30. doi:
\href{https://doi.org/10.2307/2274239}{10.2307/2274239}.

\bibitem[\citeproctext]{ref-WOS000078432400003}
Anderson, Elizabeth S. 1999. {``What Is the Point of Equality?''}
\emph{Ethics} 109 (2): 287--337. doi:
\href{https://doi.org/10.1086/233897}{10.1086/233897}.

\bibitem[\citeproctext]{ref-WOSA1985ABV7900005}
Asher, Nicholas, and Daniel Bonevac. 1985. {``Situations and Events.''}
\emph{Philosophical Studies} 47 (1): 57--77. doi:
\href{https://doi.org/10.1007/BF00355087}{10.1007/BF00355087}.

\bibitem[\citeproctext]{ref-WOS000356355000009}
Beddor, Bob. 2015. {``Evidentialism, Circularity, and Grounding.''}
\emph{Philosophical Studies} 172 (7): 1847--68. doi:
\href{https://doi.org/10.1007/s11098-014-0375-z}{10.1007/s11098-014-0375-z}.

\bibitem[\citeproctext]{ref-WOS000184542900004}
Bennett, Karen. 2003. {``Why the Exclusion Problem Seems Intractable,
and How, Just Maybe To Tract It.''} \emph{Noûs} 37 (3): 471--97. doi:
\href{https://doi.org/10.1111/1468-0068.00447}{10.1111/1468-0068.00447}.

\bibitem[\citeproctext]{ref-WOS000289572300004}
---------. 2011. {``Construction Area (No Hard Hat Required).''}
\emph{Philosophical Studies} 154 (1): 79--104. doi:
\href{https://doi.org/10.1007/s11098-011-9703-8}{10.1007/s11098-011-9703-8}.

\bibitem[\citeproctext]{ref-WOSA1988M482100005}
Blackburn, William K. 1988. {``Wettstein on Definite Descriptions.''}
\emph{Philosophical Studies} 53 (2): 263--78. doi:
\href{https://doi.org/10.1007/BF00354644}{10.1007/BF00354644}.

\bibitem[\citeproctext]{ref-WOS000325717500012}
Bliss, Ricki Leigh. 2013. {``Viciousness and the Structure of
Reality.''} \emph{Philosophical Studies} 166 (2): 399--418. doi:
\href{https://doi.org/10.1007/s11098-012-0043-0}{10.1007/s11098-012-0043-0}.

\bibitem[\citeproctext]{ref-WOS000340619100006}
Bourget, David, and David J. Chalmers. 2014. {``What Do Philosophers
Believe?''} \emph{Philosophical Studies} 170 (3): 465--500. doi:
\href{https://doi.org/10.1007/s11098-013-0259-7}{10.1007/s11098-013-0259-7}.

\bibitem[\citeproctext]{ref-WOSA1991FC38500010}
Boyd, Richard. 1991. {``Realism, Anti-Foundationalism and the Enthusiasm
for Natural Kinds.''} \emph{Philosophical Studies} 61 (1-2): 127--48.
doi: \href{https://doi.org/10.1007/BF00385837}{10.1007/BF00385837}.

\bibitem[\citeproctext]{ref-WOS000084073700005}
Broome, John. 1999. {``Normative Requirements.''} \emph{Ratio} 12 (4):
398--419. doi:
\href{https://doi.org/10.1111/1467-9329.00101}{10.1111/1467-9329.00101}.

\bibitem[\citeproctext]{ref-WOS000241143100001}
Brown, J. 2006. {``Contextualism and Warranted Assertibility
Manoeuvres.''} \emph{Philosophical Studies} 130 (3): 407--35. doi:
\href{https://doi.org/10.1007/s11098-004-5747-3}{10.1007/s11098-004-5747-3}.

\bibitem[\citeproctext]{ref-Brzezinski2015}
Brzezinski, Michal. 2015. {``Power Laws in Citation Distributions:
Evidence from Scopus.''} \emph{Scientometrics} 103 (1): 213--28. doi:
\href{https://doi.org/10.1007/s11192-014-1524-z}{10.1007/s11192-014-1524-z}.

\bibitem[\citeproctext]{ref-WOSA1986AYX3200001}
Burge, Tyler. 1986. {``Individualism and Psychology.''}
\emph{Philosophical Review} 95 (1): 3--45. doi:
\href{https://doi.org/10.2307/2185131}{10.2307/2185131}.

\bibitem[\citeproctext]{ref-WOSA1962CGX0500005}
Cavell, S. 1962. {``The Availability of Wittgenstein's Later
Philosophy.''} \emph{Philosophical Review} 71 (1): 67--93. doi:
\href{https://doi.org/10.2307/2183682}{10.2307/2183682}.

\bibitem[\citeproctext]{ref-Chalmers1996}
Chalmers, David J. 1996. \emph{The Conscious Mind}. Oxford: Oxford
University Press.

\bibitem[\citeproctext]{ref-WOS000073222300002}
Clark, Andy, and David J. Chalmers. 1998. {``The Extended Mind.''}
\emph{Analysis} 58 (1): 7--19. doi:
\href{https://doi.org/10.1111/1467-8284.00096}{10.1111/1467-8284.00096}.

\bibitem[\citeproctext]{ref-WOSA1989AE70300010}
Cohen, G. A. 1989. {``On the Currency of Egalitarian Justice.''}
\emph{Ethics} 99 (4): 906--44. doi:
\href{https://doi.org/10.1086/293126}{10.1086/293126}.

\bibitem[\citeproctext]{ref-WOSA1986D182100003}
Cross, Charles B. 1986. {``Can and the Logic of Ability.''}
\emph{Philosophical Studies} 50 (1): 53--64. doi:
\href{https://doi.org/10.1007/BF00355160}{10.1007/BF00355160}.

\bibitem[\citeproctext]{ref-WOSA1975BF60100001}
Cummins, Robert. 1975. {``Functional Analysis.''} \emph{Journal Of
Philosophy} 72 (20): 741--65. doi:
\href{https://doi.org/10.2307/2024640}{10.2307/2024640}.

\bibitem[\citeproctext]{ref-Darwall2006}
Darwall, Stephen. 2006. \emph{The Second Person Standpoint: Morality,
Respect, and Accountability,}. Cambridge, MA: Harvard University Press.

\bibitem[\citeproctext]{ref-WOS000245280900001}
Davis, Wayne A. 2007. {``Knowledge Claims and Context: Loose Use.''}
\emph{Philosophical Studies} 132 (3): 395--438. doi:
\href{https://doi.org/10.1007/s11098-006-9035-2}{10.1007/s11098-006-9035-2}.

\bibitem[\citeproctext]{ref-DeVries2005}
DeVries, Willem A. 2005. \emph{Wilfrid Sellars}. Abingdon: Routledge.

\bibitem[\citeproctext]{ref-WOS000289948200002}
Dotson, Kristie. 2011. {``Tracking Epistemic Violence, Tracking
Practices of Silencing.''} \emph{Hypatia} 26 (2): 236--57. doi:
\href{https://doi.org/10.1111/j.1527-2001.2011.01177.x}{10.1111/j.1527-2001.2011.01177.x}.

\bibitem[\citeproctext]{ref-WOSA1970ZE33800001}
Dretske, Fred. 1970. {``Epistemic Operators.''} \emph{Journal Of
Philosophy} 67 (24): 1007--23. doi:
\href{https://doi.org/10.2307/2024710}{10.2307/2024710}.

\bibitem[\citeproctext]{ref-WOSA1995QX94800001}
Edgington, Dorothy. 1995. {``On Conditionals.''} \emph{Mind} 104 (414):
235--329. doi:
\href{https://doi.org/10.1093/mind/104.414.235}{10.1093/mind/104.414.235}.

\bibitem[\citeproctext]{ref-WOS000245280800001}
Egan, Andy. 2007. {``Epistemic Modals, Relativism and Assertion.''}
\emph{Philosophical Studies} 133 (1): 1--22. doi:
\href{https://doi.org/10.1007/s11098-006-9003-x}{10.1007/s11098-006-9003-x}.

\bibitem[\citeproctext]{ref-Egan2011}
---------. 2011. {``Comments on Gendler's, 'the Epistemic Costs of
Implicit Bias'.''} \emph{Philosophical Studies} 156 (1): 65--79. doi:
\href{https://doi.org/10.1007/s11098-011-9803-5}{10.1007/s11098-011-9803-5}.

\bibitem[\citeproctext]{ref-Fodor1998}
Fodor, Jerry A. 1998. \emph{Concepts: Where Cognitive Science Went
Wrong}. Oxford: Oxford University Press.

\bibitem[\citeproctext]{ref-FodorLepore1992}
Fodor, Jerry A., and Ernest Lepore. 1992. \emph{Holism: A Shopper's
Guide}. Cambridge: Blackwell.

\bibitem[\citeproctext]{ref-WOSA1995QE80400001}
van Fraassen, Bas C. 1995. {``Belief and the Problem of Ulysses and the
Sirens.''} \emph{Philosophical Studies} 77 (1): 7--37. doi:
\href{https://doi.org/10.1007/BF00996309}{10.1007/BF00996309}.

\bibitem[\citeproctext]{ref-WOSA1969Y444700002}
Frankfurt, Harry G. 1969. {``Alternate Possibilities and Moral
Responsibility.''} \emph{Journal Of Philosophy} 66 (23): 829--39. doi:
\href{https://doi.org/10.2307/2023833}{10.2307/2023833}.

\bibitem[\citeproctext]{ref-WOS000295087100003}
Gendler, Tamar Szabò. 2011. {``On the Epistemic Costs of Implicit
Bias.''} \emph{Philosophical Studies} 156 (1): 33--63. doi:
\href{https://doi.org/10.1007/s11098-011-9801-7}{10.1007/s11098-011-9801-7}.

\bibitem[\citeproctext]{ref-WOSA1985ANT6600005}
Gensler, Harry J. 1985. {``A Kantian Argument Against Abortion.''}
\emph{Philosophical Studies} 48 (1): 57--72. doi:
\href{https://doi.org/10.1007/BF00372407}{10.1007/BF00372407}.

\bibitem[\citeproctext]{ref-WOSA1986AZA8100007}
---------. 1986. {``A Kantian Argument Against Abortion.''}
\emph{Philosophical Studies} 49 (1): 83--98. doi:
\href{https://doi.org/10.1007/BF00372885}{10.1007/BF00372885}.

\bibitem[\citeproctext]{ref-WOSA1984TJ24700012}
Hazen, Allen. 1984. {``Modality as Many Metalinguistic Predicates.''}
\emph{Philosophical Studies} 46 (2): 271--77. doi:
\href{https://doi.org/10.1007/BF00373111}{10.1007/BF00373111}.

\bibitem[\citeproctext]{ref-WOSA1992KC39800002}
Johnston, Mark. 1992. {``How To Speak of the Colors.''}
\emph{Philosophical Studies} 68 (3): 221--63. doi:
\href{https://doi.org/10.1007/BF00694847}{10.1007/BF00694847}.

\bibitem[\citeproctext]{ref-WOS000178572700004}
Kelly, Thomas. 2002. {``The Rationality of Belief and Some Other
Propositional Attitudes.''} \emph{Philosophical Studies} 110 (2):
163--96. doi:
\href{https://doi.org/10.1023/A:1020212716425}{10.1023/A:1020212716425}.

\bibitem[\citeproctext]{ref-WOS000081337700001}
Kim, Jaegwon. 1999a. {``Hempel, Explanation, Metaphysics.''}
\emph{Philosophical Studies} 94 (1-2): 1--20. doi:
\href{https://doi.org/10.1023/A:1004420102896}{10.1023/A:1004420102896}.

\bibitem[\citeproctext]{ref-WOS000082592000002}
---------. 1999b. {``Making Sense of Emergence.''} \emph{Philosophical
Studies} 95 (1-2): 3--36. doi:
\href{https://doi.org/10.1023/A:1004563122154}{10.1023/A:1004563122154}.

\bibitem[\citeproctext]{ref-WOSA1988P180000006}
King, Jeffrey C. 1988. {``Are Indefinite Descriptions Ambiguous.''}
\emph{Philosophical Studies} 53 (3): 417--40. doi:
\href{https://doi.org/10.1007/BF00353515}{10.1007/BF00353515}.

\bibitem[\citeproctext]{ref-WOSA1981NA08400001}
Kitcher, Philip. 1981. {``Explanatory Unification.''} \emph{Philosophy
Of Science} 48 (4): 507--31. doi:
\href{https://doi.org/10.1086/289019}{10.1086/289019}.

\bibitem[\citeproctext]{ref-WOS000231037900002}
Kolodny, Niko. 2005. {``Why Be Rational?''} \emph{Mind} 114 (455):
509--63. doi:
\href{https://doi.org/10.1093/mind/fzi509}{10.1093/mind/fzi509}.

\bibitem[\citeproctext]{ref-WOS000086712500005}
Latham, Noa. 2000. {``Chalmers on the Addition of Consciousness To the
Physical World.''} \emph{Philosophical Studies} 98 (1): 71--97.

\bibitem[\citeproctext]{ref-WOSA1981LY92900002}
Laudan, Larry. 1981. {``A Confutation of Convergent Realism.''}
\emph{Philosophy Of Science} 48 (1): 19--49. doi:
\href{https://doi.org/10.1086/288975}{10.1086/288975}.

\bibitem[\citeproctext]{ref-WOS000810220800002}
Lederman, Harvey. 2022. {``The Introspective Model of Genuine Knowledge
in Wang Yangming.''} \emph{Philosophical Review} 131 (2): 169--213. doi:
\href{https://doi.org/10.1215/00318108-9554691}{10.1215/00318108-9554691}.

\bibitem[\citeproctext]{ref-WOSA1979JB14500003}
Lewis, David. 1979. {``Counterfactual Dependence and Time's Arrow.''}
\emph{Noûs} 13 (4): 455--76. doi:
\href{https://doi.org/10.2307/2215339}{10.2307/2215339}.

\bibitem[\citeproctext]{ref-WOSA1983RR51600001}
---------. 1983. {``New Work for a Theory of Universals.''}
\emph{Australasian Journal Of Philosophy} 61 (4): 343--77. doi:
\href{https://doi.org/10.1080/00048408312341131}{10.1080/00048408312341131}.

\bibitem[\citeproctext]{ref-WOSA1984TQ70900001}
---------. 1984. {``Putnam's Paradox.''} \emph{Australasian Journal Of
Philosophy} 62 (3): 221--36. doi:
\href{https://doi.org/10.1080/00048408412340013}{10.1080/00048408412340013}.

\bibitem[\citeproctext]{ref-Lewis1986a}
---------. 1986. \emph{On the Plurality of Worlds}. Oxford: Blackwell
Publishers.

\bibitem[\citeproctext]{ref-WOSA1996VY21200001}
---------. 1996. {``Elusive Knowledge.''} \emph{Australasian Journal Of
Philosophy} 74 (4): 549--67. doi:
\href{https://doi.org/10.1080/00048409612347521}{10.1080/00048409612347521}.

\bibitem[\citeproctext]{ref-WOSA1997WP33800001}
---------. 1997. {``Finkish Dispositions.''} \emph{Philosophical
Quarterly} 47 (187): 143--58. doi:
\href{https://doi.org/10.1111/1467-9213.00052}{10.1111/1467-9213.00052}.

\bibitem[\citeproctext]{ref-Lewy1976}
Lewy, C. 1976. {``Mind Under {G. E. Moore} (1921-1947).''} \emph{Mind}
85 (337): 37--46. doi:
\href{https://doi.org/10.1093/mind/LXXXV.337.37}{10.1093/mind/LXXXV.337.37}.

\bibitem[\citeproctext]{ref-WOS000087305900001}
Machamer, Peter, Lindley Darden, and Carl F. Craver. 2000. {``Thinking
About Mechanisms.''} \emph{Philosophy Of Science} 67 (1): 1--25. doi:
\href{https://doi.org/10.1086/392759}{10.1086/392759}.

\bibitem[\citeproctext]{ref-Malaterre2019}
Malaterre, Christophe, Jean-François Chartier, and Davide Pulizzotto.
2019. {``What Is This Thing Called Philosophy of Science? A
Computational Topic-Modeling Perspective, 1934?2015.''} \emph{Hopos: The
Journal of the International Society for the History of Philosophy of
Science} 9 (2): 215--49. doi:
\href{https://doi.org/10.1086/704372}{10.1086/704372}.

\bibitem[\citeproctext]{ref-Malaterre2022}
Malaterre, Christophe, and Francis Lareau. 2022. {``The Early Days of
Contemporary Philosophy of Science: Novel Insights from Machine
Translation and Topic-Modeling of Non-Parallel Multilingual Corpora.''}
\emph{Synthese} 200 (3): 1--33. doi:
\href{https://doi.org/10.1007/s11229-022-03722-x}{10.1007/s11229-022-03722-x}.

\bibitem[\citeproctext]{ref-Malaterre2020}
Malaterre, Christophe, Francis Lareau, Davide Pulizzotto, and Jonathan
St-Onge. 2020. {``Eight Journals over Eight Decades: A Computational
Topic-Modeling Approach to Contemporary Philosophy of Science.''}
\emph{Synthese} 199 (1-2): 2883--923. doi:
\href{https://doi.org/10.1007/s11229-020-02915-6}{10.1007/s11229-020-02915-6}.

\bibitem[\citeproctext]{ref-Malaterre2019b}
Malaterre, Christophe, Davide Pulizzotto, and Francis Lareau. 2019.
{``Revisiting Three Decades of Biology and Philosophy: A Computational
Topic-Modeling Perspective.''} \emph{Biology and Philosophy} 35 (5):
1--25. doi:
\href{https://doi.org/10.1007/s10539-019-9729-4}{10.1007/s10539-019-9729-4}.

\bibitem[\citeproctext]{ref-WOS000460037300009}
Matthes, Erich Hatala. 2019. {``Cultural Appropriation and
Oppression.''} \emph{Philosophical Studies} 176 (4): 1003--13. doi:
\href{https://doi.org/10.1007/s11098-018-1224-2}{10.1007/s11098-018-1224-2}.

\bibitem[\citeproctext]{ref-WOS000341924300014}
Nolan, Daniel. 2014. {``Hyperintensional Metaphysics.''}
\emph{Philosophical Studies} 171 (1): 149--60. doi:
\href{https://doi.org/10.1007/s11098-013-0251-2}{10.1007/s11098-013-0251-2}.

\bibitem[\citeproctext]{ref-WOS000083809100001}
Ostertag, Gary. 1999. {``A Scorekeeping Error.''} \emph{Philosophical
Studies} 96 (2): 123--46. doi:
\href{https://doi.org/10.1023/A:1004293605139}{10.1023/A:1004293605139}.

\bibitem[\citeproctext]{ref-Petrovich2024}
Petrovich, Eugenio. 2024. \emph{A Quantitative Portrait of Analytic
Philosophy: Looking Through the Margins}. Cham: Springer.

\bibitem[\citeproctext]{ref-WOS000165361800002}
Pryor, James. 2000. {``The Skeptic and the Dogmatist.''} \emph{Noûs} 34
(4): 517--49. doi:
\href{https://doi.org/10.1111/0029-4624.00277}{10.1111/0029-4624.00277}.

\bibitem[\citeproctext]{ref-WOSA1980KH88100001}
Rawls, John. 1980. {``Kantian Constructivism in Moral Theory.''}
\emph{Journal Of Philosophy} 77 (9): 515--35. doi:
\href{https://doi.org/10.2307/2025790}{10.2307/2025790}.

\bibitem[\citeproctext]{ref-WOS000222384400005}
Schaffer, Jonathan. 2004. {``From Contextualism To Contrastivism.''}
\emph{Philosophical Studies} 119 (1-2): 73--103. doi:
\href{https://doi.org/10.1023/B:PHIL.0000029351.56460.8c}{10.1023/B:PHIL.0000029351.56460.8c}.

\bibitem[\citeproctext]{ref-WOS000272855000002}
---------. 2010. {``Monism: The Priority of the Whole.''}
\emph{Philosophical Review} 119 (1): 31--76. doi:
\href{https://doi.org/10.1215/00318108-2009-025}{10.1215/00318108-2009-025}.

\bibitem[\citeproctext]{ref-Sider2020}
Sider, Theodore. 2020. \emph{The Tools of Metaphysics and the
Metaphysics of Science}. Oxford: Oxford University Press.

\bibitem[\citeproctext]{ref-WOSA1972Z066400001}
Singer, Peter. 1972. {``Famine, Affluence, and Morality.''}
\emph{Philosophy \& Public Affairs} 1 (3): 229--43.

\bibitem[\citeproctext]{ref-WOSA1989CE37600004}
Terzis, George N. 1989. {``An Objection To Kantian Ethical
Rationalism.''} \emph{Philosophical Studies} 57 (3): 299--313. doi:
\href{https://doi.org/10.1007/BF00372699}{10.1007/BF00372699}.

\bibitem[\citeproctext]{ref-WOSA1971Y116900003}
Thomson, Judith Jarvis. 1971. {``A Defense of Abortion.''}
\emph{Philosophy \& Public Affairs} 1 (1): 47--66.

\bibitem[\citeproctext]{ref-WOSA1984SK63200001}
Tichý, Pavel. 1984. {``Subjunctive Conditionals: Two Parameters
Vs.~Three.''} \emph{Philosophical Studies} 45 (2): 147--79. doi:
\href{https://doi.org/10.1007/BF00372476}{10.1007/BF00372476}.

\bibitem[\citeproctext]{ref-Warnock1976}
Warnock, G. J. 1976. {``Gilbert Ryle's Editorship.''} \emph{Mind} 85
(337): 47--56. doi:
\href{https://doi.org/10.1093/mind/LXXXV.337.47}{10.1093/mind/LXXXV.337.47}.

\bibitem[\citeproctext]{ref-WOS000168853300001}
Williams, Michael. 2001. {``Contextualism, Externalism and Epistemic
Standards.''} \emph{Philosophical Studies} 103 (1): 1--23. doi:
\href{https://doi.org/10.1023/A:1010349100272}{10.1023/A:1010349100272}.

\bibitem[\citeproctext]{ref-WOS000468369800007}
Wodak, Daniel. 2019. {``Moral Perception, Inference, and Intuition.''}
\emph{Philosophical Studies} 176 (6): 1495--1512. doi:
\href{https://doi.org/10.1007/s11098-019-01250-y}{10.1007/s11098-019-01250-y}.

\bibitem[\citeproctext]{ref-WOSA1992JA62400001}
Yablo, Stephen. 1992. {``Mental Causation.''} \emph{Philosophical
Review} 101 (2): 245--80. doi:
\href{https://doi.org/10.2307/2185535}{10.2307/2185535}.

\end{CSLReferences}



\noindent Published online in November 2024.


\end{document}
