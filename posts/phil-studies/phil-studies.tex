% Options for packages loaded elsewhere
\PassOptionsToPackage{unicode}{hyperref}
\PassOptionsToPackage{hyphens}{url}
%
\documentclass[
  10pt,
  letterpaper,
  DIV=11,
  numbers=noendperiod,
  twoside]{scrartcl}

\usepackage{amsmath,amssymb}
\usepackage{setspace}
\usepackage{iftex}
\ifPDFTeX
  \usepackage[T1]{fontenc}
  \usepackage[utf8]{inputenc}
  \usepackage{textcomp} % provide euro and other symbols
\else % if luatex or xetex
  \usepackage{unicode-math}
  \defaultfontfeatures{Scale=MatchLowercase}
  \defaultfontfeatures[\rmfamily]{Ligatures=TeX,Scale=1}
\fi
\usepackage{lmodern}
\ifPDFTeX\else  
    % xetex/luatex font selection
  \setmainfont[ItalicFont=EB Garamond Italic,BoldFont=EB Garamond
Bold]{EB Garamond Math}
  \setsansfont[]{Europa-Bold}
  \setmathfont[]{Garamond-Math}
\fi
% Use upquote if available, for straight quotes in verbatim environments
\IfFileExists{upquote.sty}{\usepackage{upquote}}{}
\IfFileExists{microtype.sty}{% use microtype if available
  \usepackage[]{microtype}
  \UseMicrotypeSet[protrusion]{basicmath} % disable protrusion for tt fonts
}{}
\usepackage{xcolor}
\usepackage[left=1in, right=1in, top=0.8in, bottom=0.8in,
paperheight=9.5in, paperwidth=6.5in, includemp=TRUE, marginparwidth=0in,
marginparsep=0in]{geometry}
\setlength{\emergencystretch}{3em} % prevent overfull lines
\setcounter{secnumdepth}{3}
% Make \paragraph and \subparagraph free-standing
\ifx\paragraph\undefined\else
  \let\oldparagraph\paragraph
  \renewcommand{\paragraph}[1]{\oldparagraph{#1}\mbox{}}
\fi
\ifx\subparagraph\undefined\else
  \let\oldsubparagraph\subparagraph
  \renewcommand{\subparagraph}[1]{\oldsubparagraph{#1}\mbox{}}
\fi


\providecommand{\tightlist}{%
  \setlength{\itemsep}{0pt}\setlength{\parskip}{0pt}}\usepackage{longtable,booktabs,array}
\usepackage{calc} % for calculating minipage widths
% Correct order of tables after \paragraph or \subparagraph
\usepackage{etoolbox}
\makeatletter
\patchcmd\longtable{\par}{\if@noskipsec\mbox{}\fi\par}{}{}
\makeatother
% Allow footnotes in longtable head/foot
\IfFileExists{footnotehyper.sty}{\usepackage{footnotehyper}}{\usepackage{footnote}}
\makesavenoteenv{longtable}
\usepackage{graphicx}
\makeatletter
\def\maxwidth{\ifdim\Gin@nat@width>\linewidth\linewidth\else\Gin@nat@width\fi}
\def\maxheight{\ifdim\Gin@nat@height>\textheight\textheight\else\Gin@nat@height\fi}
\makeatother
% Scale images if necessary, so that they will not overflow the page
% margins by default, and it is still possible to overwrite the defaults
% using explicit options in \includegraphics[width, height, ...]{}
\setkeys{Gin}{width=\maxwidth,height=\maxheight,keepaspectratio}
% Set default figure placement to htbp
\makeatletter
\def\fps@figure{htbp}
\makeatother
% definitions for citeproc citations
\NewDocumentCommand\citeproctext{}{}
\NewDocumentCommand\citeproc{mm}{%
  \begingroup\def\citeproctext{#2}\cite{#1}\endgroup}
\makeatletter
 % allow citations to break across lines
 \let\@cite@ofmt\@firstofone
 % avoid brackets around text for \cite:
 \def\@biblabel#1{}
 \def\@cite#1#2{{#1\if@tempswa , #2\fi}}
\makeatother
\newlength{\cslhangindent}
\setlength{\cslhangindent}{1.5em}
\newlength{\csllabelwidth}
\setlength{\csllabelwidth}{3em}
\newenvironment{CSLReferences}[2] % #1 hanging-indent, #2 entry-spacing
 {\begin{list}{}{%
  \setlength{\itemindent}{0pt}
  \setlength{\leftmargin}{0pt}
  \setlength{\parsep}{0pt}
  % turn on hanging indent if param 1 is 1
  \ifodd #1
   \setlength{\leftmargin}{\cslhangindent}
   \setlength{\itemindent}{-1\cslhangindent}
  \fi
  % set entry spacing
  \setlength{\itemsep}{#2\baselineskip}}}
 {\end{list}}
\usepackage{calc}
\newcommand{\CSLBlock}[1]{\hfill\break\parbox[t]{\linewidth}{\strut\ignorespaces#1\strut}}
\newcommand{\CSLLeftMargin}[1]{\parbox[t]{\csllabelwidth}{\strut#1\strut}}
\newcommand{\CSLRightInline}[1]{\parbox[t]{\linewidth - \csllabelwidth}{\strut#1\strut}}
\newcommand{\CSLIndent}[1]{\hspace{\cslhangindent}#1}

\setlength\heavyrulewidth{0ex}
\setlength\lightrulewidth{0ex}
\usepackage[automark]{scrlayer-scrpage}
\clearpairofpagestyles
\cehead{
  Brian Weatherson
  }
\cohead{
  Trends in Philosophical Studies
  }
\ohead{\bfseries \pagemark}
\cfoot{}
\makeatletter
\newcommand*\NoIndentAfterEnv[1]{%
  \AfterEndEnvironment{#1}{\par\@afterindentfalse\@afterheading}}
\makeatother
\NoIndentAfterEnv{itemize}
\NoIndentAfterEnv{enumerate}
\NoIndentAfterEnv{description}
\NoIndentAfterEnv{quote}
\NoIndentAfterEnv{equation}
\NoIndentAfterEnv{longtable}
\NoIndentAfterEnv{abstract}
\renewenvironment{abstract}
 {\vspace{-1.25cm}
 \quotation\small\noindent\rule{\linewidth}{.5pt}\par\smallskip
 \noindent }
 {\par\noindent\rule{\linewidth}{.5pt}\endquotation}
\setkomafont{descriptionlabel}{\normalfont\scshape\bfseries}
\KOMAoption{captions}{tableheading}
\makeatletter
\@ifpackageloaded{caption}{}{\usepackage{caption}}
\AtBeginDocument{%
\ifdefined\contentsname
  \renewcommand*\contentsname{Table of contents}
\else
  \newcommand\contentsname{Table of contents}
\fi
\ifdefined\listfigurename
  \renewcommand*\listfigurename{List of Figures}
\else
  \newcommand\listfigurename{List of Figures}
\fi
\ifdefined\listtablename
  \renewcommand*\listtablename{List of Tables}
\else
  \newcommand\listtablename{List of Tables}
\fi
\ifdefined\figurename
  \renewcommand*\figurename{Figure}
\else
  \newcommand\figurename{Figure}
\fi
\ifdefined\tablename
  \renewcommand*\tablename{Table}
\else
  \newcommand\tablename{Table}
\fi
}
\@ifpackageloaded{float}{}{\usepackage{float}}
\floatstyle{ruled}
\@ifundefined{c@chapter}{\newfloat{codelisting}{h}{lop}}{\newfloat{codelisting}{h}{lop}[chapter]}
\floatname{codelisting}{Listing}
\newcommand*\listoflistings{\listof{codelisting}{List of Listings}}
\makeatother
\makeatletter
\makeatother
\makeatletter
\@ifpackageloaded{caption}{}{\usepackage{caption}}
\@ifpackageloaded{subcaption}{}{\usepackage{subcaption}}
\makeatother
\ifLuaTeX
  \usepackage{selnolig}  % disable illegal ligatures
\fi
\IfFileExists{bookmark.sty}{\usepackage{bookmark}}{\usepackage{hyperref}}
\IfFileExists{xurl.sty}{\usepackage{xurl}}{} % add URL line breaks if available
\urlstyle{same} % disable monospaced font for URLs
\hypersetup{
  pdftitle={Trends in Philosophical Studies},
  pdfauthor={Brian Weatherson},
  hidelinks,
  pdfcreator={LaTeX via pandoc}}

\title{Trends in \emph{Philosophical Studies}}
\author{Brian Weatherson}
\date{2024}

\begin{document}
\maketitle
\begin{abstract}
\emph{Philosophical Studies} has become one of the most important
journals for work in several large topics in philosophy. This paper uses
data from the word distributions in those papers, and the citations of
the papers, to look at how it has changed over time, and how it became
so central.
\end{abstract}

\setstretch{1.1}
\section{Intro}\label{intro}

TKTK Weatherson (\citeproc{ref-Weatherson2003}{2003})

\section{Sources}\label{sec-sources}

This article is primarily based on data-driven analysis of articles
published in philosophy journals, and in particular in
\emph{Philosophical Studies}, from 1980 to 2019. The studies here are
primarily based on two sources: citation data from Web of Science, and
word lists from JSTOR.

Through {[}University X{]} I downloaded the Web of Science (hereafter,
WoS) Core Collection in XML format. Within it, I selected 100 prominent
philosophy journals that WoS indexes. The journals I selected are, like
\emph{Philosophical Studies} primarily English-language, analytic
philosophy journals. I filtered the citations for just citations from
and to those 100 journals. So what we're working with is citations of
philosophy journals in philosophy journals.

This is obviously a small subset of all citations. It excludes citations
in academic journals in other fields, in books and edited volumes, and
in many other places that Google Scholar indexes, such as dissertations,
lecture notes, slides, and draft manuscripts. Losing that information is
a cost, but there are three large upsides. First, these citations are
much more accurate; I haven't found any false positives when using this
filtered set, and only a handful of false negatives. Second, we can be
more confident that our data set is relatively complete; finding a full
list of philosophy journals is easier than finding a full list of edited
volumes in philosophy. Third, by looking at citations internal to
philosophy, we can get a sense of philosophy's self-image, and how it
changes over time.

The downloadable citation data is not particularly up to date. I am
including citations beyond 2019, because it's helpful to get a sense of
how some of these articles have been received in more recent years. But
the data I have only goes through mid-2022. I'll often simply say 2022;
but note that even that year is incomplete.

The other source I used is JSTOR, and in particular the Data for
Research (DfR) program that they provide through their Constellate
project. This lets you download lists of the words used in various
journal articles, along with a count of how often each word is
used.\footnote{It also provides bigrams and trigrams, which I've
  occasionally used.} It also provides word counts for the articles,
which I have used in Section~\ref{sec-overview}. The words an author
uses are a pretty good guide to what they are talking about; if the word
`denotation' is used frequently, it's probably a philosophy of language
article.

\section{Inbound Citations}\label{sec-overview}

\subsection{Overview of Citations}\label{sec-citations-overview}

Articles in \emph{Philosophical Studies} get cited a lot. Our first
study is simply a count of citations in the 100 journals to articles
published in the 100 journals between 1980 and 2019.
Table~\ref{tbl-all-cites} shows the five journals with the largest
number of citations.

\begin{longtable}[]{@{}lr@{}}

\caption{\label{tbl-all-cites}Leading journals by total number of
citations (Articles published 1980-2019).}

\tabularnewline

\toprule\noalign{}
Journal & Citations \\
\midrule\noalign{}
\endhead
\bottomrule\noalign{}
\endlastfoot
Philosophical Studies & 30424 \\
Synthese & 23280 \\
Journal Of Philosophy & 20023 \\
Noûs & 19716 \\
Philosophy Of Science & 18810 \\

\end{longtable}

\emph{Philosophical Studies} is in first place on that list in part, but
only in part, because it publishes so much. Table~\ref{tbl-all-articles}
lists the top five journals by the number of articles they have
published.

\begin{longtable}[]{@{}lr@{}}

\caption{\label{tbl-all-articles}Leading journals by total number of
articles (Articles published 1980-2019).}

\tabularnewline

\toprule\noalign{}
Journal & Articles \\
\midrule\noalign{}
\endhead
\bottomrule\noalign{}
\endlastfoot
Synthese & 4525 \\
Philosophical Studies & 3776 \\
Journal Of Medical Ethics & 3518 \\
Journal Of Symbolic Logic & 3283 \\
Analysis & 2236 \\

\end{longtable}

\emph{Synthese} has 20\% more articles, but 25\% fewer citations. The
other three journals on that list are somewhat special cases. Two of
them get a lot of citations outside of philosophy, and this is only a
study of citations in philosophy journals, and \emph{Analysis} only
publishes short papers, and so while they get a lot of citations per
page, they don't get as many citations per article as other journals.

Still, we'd expect on general principles that raw volume of publication
wouldn't make a big difference. Citations tend to follow something like
a log-normal distribution
(\citeproc{ref-Brzezinski2015}{\textbf{Brzezinski2015?}}). The bulk of
the citations come from a handful of highly cited articles. Publishing
more articles helps, but is no guarantee.

If we look not at total citations, but at citations per article as in
Table~\ref{tbl-citation-rate}, we get a list that looks a bit more like
a familiar ranking of philosophy journals by prestige.

\begin{longtable}[]{@{}
  >{\raggedleft\arraybackslash}p{(\columnwidth - 8\tabcolsep) * \real{0.0633}}
  >{\raggedright\arraybackslash}p{(\columnwidth - 8\tabcolsep) * \real{0.5190}}
  >{\raggedleft\arraybackslash}p{(\columnwidth - 8\tabcolsep) * \real{0.1139}}
  >{\raggedleft\arraybackslash}p{(\columnwidth - 8\tabcolsep) * \real{0.1266}}
  >{\raggedleft\arraybackslash}p{(\columnwidth - 8\tabcolsep) * \real{0.1772}}@{}}

\caption{\label{tbl-citation-rate}Leading journals by citation rate
(Articles published 1980-2019).}

\tabularnewline

\toprule\noalign{}
\begin{minipage}[b]{\linewidth}\raggedleft
Rank
\end{minipage} & \begin{minipage}[b]{\linewidth}\raggedright
Journal
\end{minipage} & \begin{minipage}[b]{\linewidth}\raggedleft
Articles
\end{minipage} & \begin{minipage}[b]{\linewidth}\raggedleft
Citations
\end{minipage} & \begin{minipage}[b]{\linewidth}\raggedleft
Citation Rate
\end{minipage} \\
\midrule\noalign{}
\endhead
\bottomrule\noalign{}
\endlastfoot
1 & Philosophical Review & 510 & 14706 & 28.84 \\
2 & Journal Of Philosophy & 1221 & 20023 & 16.40 \\
3 & Philosophy \& Public Affairs & 521 & 8277 & 15.89 \\
4 & Mind & 1071 & 14391 & 13.44 \\
5 & Ethics & 1045 & 13040 & 12.48 \\
& \ldots{} & & & \\
13 & Philosophy And Phenomenological Research & 2165 & 17737 & 8.19 \\
14 & Philosophical Studies & 3776 & 30424 & 8.06 \\

\end{longtable}

I've included \emph{Philosophy and Phenomenological Research} there
because, like \emph{Philosophical Studies}, it publishes many book
symposia. And, like \emph{Philosophical Studies}, the articles in these
symposia are typically not cited very much.

\subsection{Articles?}\label{sec-what-is-article}

This raises a tricky question of what exactly counts as an
\emph{article}. We all have a sense of what a paradigm article, like
``Two Notions of Necessity'' (\citeproc{ref-WOSA1980KA40400001}{Davies
and Humberstone 1980}) is an article; the table of contents is not, and
nor are corrections. Nor are book reviews; if they were then
\emph{Philosophical Review} would have ten times as many articles
listed. But there are trickier cases. Which parts of a book symposium
are articles? A symposium typically has a précis, some commentary
articles, and some replies. One could make a case for all of these being
articles, or none of them. They are all somewhat borderline cases.
\emph{Philosophical Studies} also publishes various conference
proceedings. The main papers at these conferences are clearly articles;
in fact, they include some of the most highly cited articles the journal
has published. But in some cases they also publish commentaries on the
papers, and one could go either way on classifying those as articles or
something else.\footnote{One of these conference proceedings, in 2011,
  included Tamar Szabò Gendler's influential ``The Epistemic Costs of
  Implicit Bias'' (\citeproc{ref-WOS000295087100003}{Gendler 2011}),
  which is the first article in the journals I'm looking at with the
  phrase `implicit bias' in the title. It also included a widely cited
  commentary on that article by Andy Egan
  (\citeproc{ref-Egan2011}{2011}). Web of Science classified the
  commentary as ``Editorial Matter''. While that's clearly wrong, I'm
  not sure what the right classification should be. Unlike Egan's
  commentary, most of these commentaries get very few citations, so
  excluding them doesn't make a huge difference to the totals, and
  slightly increases the average.}

Web of Science has five main classifications of entries other than
corrections, front matter, etc: Articles, which is what we're primarily
using, Book-Review, which we're wholly excluding, and then Discussion,
Review, and Note. These sometimes get used for things that either are
clearly not articles, e.g., lists or work published on a topic recently,
and sometimes for short pieces that are probably best excluded from
citation analysis. The short pieces listed as discussion notes that
journals like \emph{Mind}, \emph{Australasian Journal of Philosophy},
and \emph{British Journal for the Philosophy of Science} frequently
print often fall into this category.

Sometimes, however, more substantive pieces get listed under one of
these categories. This seemed to be a particular problem with
\emph{Philosophical Review}, I suspect simply because of the word
`Review' in the name. Major works by Stanley Cavell
(\citeproc{ref-WOSA1962CGX0500005}{1962}), Jonathan Schaffer
(\citeproc{ref-WOS000272855000002}{2010}), and Harvey Lederman
(\citeproc{ref-WOS000810220800002}{2022}) all got listed as not being
articles.\footnote{The Cavell article was in the discussion section of
  the January 1962 issue of the Review, so this classification is
  understandable. The other two are not.} For purposes of this paper,
I've decided to treat anything twenty pages or longer as an article,
even if it was listed as a Discussion, Review, or Note. One could argue
that this includes too much. Should the review essay that Sophie Grace
Chappell (\citeproc{ref-WOS000540751100005}{2020}) wrote on Cora
Diamond's recent book count as an article? It was originally listed as a
review, but because it's twenty pages, I counted it. You could make a
case either way. This doesn't much affect \emph{Philosophical Studies},
but I'm bringing it up to note the complexities here.

Another way to see the complexities comes from comparing the two data
sources that I'm using for this piece: Web of Science and JSTOR. The two
do not agree on which articles in book symposia are capital-A Articles,
as well as on some conference papers. Unfortunately, this is not because
they have different principles for how to classify these pieces. Rather,
they are both a bit haphazard in their classifications, but in slightly
different ways. Both of them will sometimes count symposia entries as
book reviews (or occasionally as discussions, notes, or reviews), and
sometimes not, and if there's a pattern in either case, I haven't found
it. My view is that these are all borderline cases, and so none of the
classifications is determinately wrong, but it would be nice to have a
more consistent principle.

Because it would be impractical to reclassify all 200,000 entries from
the journals I'm looking at, I've decided to mostly go with Web of
Science's classifications, subject to the proviso above that I've added
back in things twenty pages or longer from three small categories.
Different ways of classifying papers, at least as plausible as the one
I'm using from Web of Science, could move any of these citation numbers
by ten percent or more, so take all the numbers I'm listing as having
large error bars. But hopefully they are at least directionally useful.

\subsection{Large Trend}\label{large-trend}

Even with those possible errors in mind, it is striking to look at the
step-change in citations to \emph{Philosophical Studies} that occurred
in the mid-2000s. Cross-temporal comparisons of citations are hard
because changes in the number of journals, the number of articles in
those journals, and citation norms, make most comparisons tricky. To try
to screen off some of that noise, I'll mostly compare citations to
articles published in \emph{Philosophical Studies} to citations to other
articles published at the same time.

In particular, in this section I'll compare \emph{Philosophical Studies}
to a list of nineteen other prominent philosophy journals. From the one
hundred journals that I'm primarily looking at, I selected the twenty
that have the highest rate of citations per published article, and which
Web of Science has indexed every year since 1980.\footnote{The last
  constraint notably rules out \emph{Philosophers' Imprint} and
  \emph{Mind and Language}.} That list includes \emph{Philosophical
Studies}, and the other nineteen journals are the comparison class.

Figure~\ref{fig-compare-cites-dots} shows, for each year from 1980 to
2019, the average number of citations for articles published in
\emph{Philosophical Studies} (in blue), and in the other nineteen (in
red). The figure is fairly noisy, but some trends are clear. Before
2000, the red dots, for the other journals, are mostly above the blue
dots, for \emph{Philosophical Studies}. After 2000, and especially from
2003 onwards, that is mostly reversed. Despite having less time to
accrue citations, articles from the 2000s are cited more, on average,
than articles published earlier. But articles published in the 2010s,
especially the late 2010s, have many fewer cites largely because they
haven't been around as long.

\begin{figure}

\centering{

\includegraphics{phil-studies_files/figure-pdf/fig-compare-cites-dots-1.pdf}

}

\caption{\label{fig-compare-cites-dots}Average citation rates for
\emph{Philosophical Studies} and peer journals.}

\end{figure}%

Figure~\ref{fig-compare-cites-rolling} smooths out some of the noise in
Figure~\ref{fig-compare-cites-dots} in two ways. First, instead of
measuring average citations per year, I measure average citations over a
five-year rolling window. This doesn't make a huge difference to the
measure for the other nineteen, which is already fairly smooth, but it
is useful for smoothing the values for just one journal. Second, instead
of showing the red and blue dots separately, I've just displayed the
ratio between them.

\begin{figure}

\centering{

\includegraphics{phil-studies_files/figure-pdf/fig-compare-cites-rolling-1.pdf}

}

\caption{\label{fig-compare-cites-rolling}Ratio of citations to
Philosophical Studies to citations to other journals, for five year
rolling windows}

\end{figure}%

The difference in Figure~\ref{fig-compare-cites-rolling} between the
earlier and recent years is striking. By this one measure, citations per
article, \emph{Philosophical Studies} was doing ok before 2003, but was
towards the lower end of the top 20 journals. After 2003, it is doing
better than the average journal \emph{in the top 20}.

My very anecdotal impression is that \emph{Philosophical Studies} is
viewed as being more prestigious by younger philosophers than by older
philosophers. A toy model of prestige, where it is heavily anchored to
how often a journal was cited when one was in graduate school, would
explain that difference. That said, I have not done (and am not going to
do) a careful study of comparative prestige judgments to know if there
is even an effect here to find, or whether my informal sample was not
reflective of the wider population.

What's more useful is to look at how things changed in the mid-2000s.

\section{Articles}\label{articles}

Notes - including things that JSTOR counts as book reviews Not including
things that WOS counts as discussions This is, to put it mildly, a bit
random; some precis are included, some aren't; some replies are
included, some aren't. In general if the word `comment' appears in the
title excluded, but it's a bit random. But going with them seems best.
Alternative is to work through 100 other journals if we wanted an
apples-to-apples comparison. Missing the supplement in 2013, which I
don't understand, and was coded differently in the two databases

\section{Studies}\label{studies}

Number of articles Length of articles Number of articles over 10K, 15K,
20K Average citations over the 100 journals - boring because can't
compare Average citations compared to top 25 - more interesting Maybe
redo that graph as a ratio

\section{Comparisons}\label{comparisons}

Articles with highest ratio of PS cites/general cites Articles with
lowest ratio of PS cites/general cites Previously I just did that with
articles published in PS; that seems wrong

Hmm, this could be complicated. Study: For each decade find 20 most
cited, and look at how often they are cited in PS.

For 1980s, the low is 0 for Rawls (1980) and Dennett (1971). But Rawls
is cited in Post (1984), under a variant name, and Dennett is cited in
Lormond (1985), but to the reprint. Maybe just live with the messiness
High five in 1980s are all language, or language-adjacent: Lewis (1979)
(both de se and time's arrow), two Perry papers, and Donnellan

For 1990s, the low is 0 for McDowell's Virtue and Reason. Is in
Brighouse (1990), but the citation is incomplete and uses the wrong
year. Oddly the third lowest is for Discrimination and Perceptual
Knowledge, only 2/38. And this looks right. The highs in 1990s are still
language based, though with metaphysics seeping in (Kim on
supervenience, Lewis on Universals)

For 2000s, lows include Cummins (1975) (which is hard to track, such a
common name), Anderson (1999), and Mechanisms paper (2000). The highs
are very epistemology based - Goldman is among them, and the two big
contextualism papers by DeRose and Lewis. Obviously the editor at the
time was also a prominent contextualist, but his papers are not cited in
the journal at a particularly out of the ordinary rate. (There are many
cases in history where editors are cited a lot, but this isn't one of
them.) Just outside top 20, but Jackson and Chalmers has nearly 1/3 its
cites in PS. Tracing back to Davies and Humberstone's original paper.
Remarkably, Elusive Knowledge is most cited in the decade; normally this
kind of calculation pushes towards less cited, high variance cases.

For 2010s, still mechanisms, Anderson (1999), also Kripke (1975) are the
low ones Top 5 is more varied, both in topic and time: Kolodny (2005),
Pryor (2000), Lewis (1986), Frankfurt (1969), Jackson (1982). Could be
result of publishing more papers, but striking that it's pulling these
older papers in.

\section{The LDA}\label{the-lda}

Build the model Note the five categories Graph the trends Flag the
methodology

\section{Language}\label{language}

Find the 40 most cited in each topic over the 40 years (i.e., most cited
per year) Look how often they are cited in PS Look how often they are
cited across the 100 See if the trends in PS track wider trends Find
other journals that have similar trends (Analysis, PQ, AJP)

\section{Metaphysics}\label{metaphysics}

Note the two Schaffer papers, and differences in citing Schaffer
(\citeproc{ref-WOS000368189400004}{2016}) Is that because little
grounding There is some - see graph of words Maybe just count how
`modal' vs `postmodal' the 2019 papers are Maybe do a small LDA of the
metaphysics papers

\section{Ethics}\label{ethics}

Is there anything to say here? Does it go more political? Still not
citing Anderson, but something?

Delicate. Even happening.

\phantomsection\label{refs}
\begin{CSLReferences}{1}{0}
\bibitem[\citeproctext]{ref-WOSA1962CGX0500005}
Cavell, S. 1962. {``The Availability of Wittgenstein's Later
Philosophy.''} \emph{Philosophical Review} 71 (1): 67--93. doi:
\href{https://doi.org/10.2307/2183682}{10.2307/2183682}.

\bibitem[\citeproctext]{ref-WOS000540751100005}
Chappell, Sophie Grace. 2020. {``Cora Diamond: Reading Wittgenstein with
Anscombe, Going on to Ethics.''} \emph{Ethics} 130 (4): 588--608. doi:
\href{https://doi.org/10.1086/708537}{10.1086/708537}.

\bibitem[\citeproctext]{ref-WOSA1980KA40400001}
Davies, M, and L Humberstone. 1980. {``Two Notions of Necessity.''}
\emph{Philosophical Studies} 38 (1): 1--30. doi:
\href{https://doi.org/10.1007/BF00354523}{10.1007/BF00354523}.

\bibitem[\citeproctext]{ref-Egan2011}
Egan, Andy. 2011. {``Comments on Gendler's, 'the Epistemic Costs of
Implicit Bias'.''} \emph{Philosophical Studies} 156 (1): 65--79. doi:
\href{https://doi.org/10.1007/s11098-011-9803-5}{10.1007/s11098-011-9803-5}.

\bibitem[\citeproctext]{ref-WOS000295087100003}
Gendler, Tamar Szabo. 2011. {``On the Epistemic Costs of Implicit
Bias.''} \emph{Philosophical Studies} 156 (1): 33--63. doi:
\href{https://doi.org/10.1007/s11098-011-9801-7}{10.1007/s11098-011-9801-7}.

\bibitem[\citeproctext]{ref-WOS000810220800002}
Lederman, Harvey. 2022. {``The Introspective Model of Genuine Knowledge
in Wang Yangming.''} \emph{Philosophical Review} 131 (2): 169--213. doi:
\href{https://doi.org/10.1215/00318108-9554691}{10.1215/00318108-9554691}.

\bibitem[\citeproctext]{ref-WOS000272855000002}
Schaffer, Jonathan. 2010. {``Monism: The Priority of the Whole.''}
\emph{Philosophical Review} 119 (1): 31--76. doi:
\href{https://doi.org/10.1215/00318108-2009-025}{10.1215/00318108-2009-025}.

\bibitem[\citeproctext]{ref-WOS000368189400004}
---------. 2016. {``Grounding in the Image of Causation.''}
\emph{Philosophical Studies} 173 (1): 49--100. doi:
\href{https://doi.org/10.1007/s11098-014-0438-1}{10.1007/s11098-014-0438-1}.

\bibitem[\citeproctext]{ref-Weatherson2003}
Weatherson, Brian. 2003. {``From Classical to Intuitionistic
Probability.''} \emph{Notre Dame Journal of Formal Logic} 44 (2):
111--23. doi:
\href{https://doi.org/10.1305/ndjfl/1082637807}{10.1305/ndjfl/1082637807}.

\end{CSLReferences}



\noindent Published online in October 2024.

\end{document}
