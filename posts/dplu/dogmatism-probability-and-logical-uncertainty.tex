% Options for packages loaded elsewhere
\PassOptionsToPackage{unicode}{hyperref}
\PassOptionsToPackage{hyphens}{url}
\PassOptionsToPackage{dvipsnames,svgnames,x11names}{xcolor}
%
\documentclass[
  11pt,
  letterpaper,
  DIV=11,
  numbers=noendperiod,
  oneside]{scrartcl}

\usepackage{amsmath,amssymb}
\usepackage{iftex}
\ifPDFTeX
  \usepackage[T1]{fontenc}
  \usepackage[utf8]{inputenc}
  \usepackage{textcomp} % provide euro and other symbols
\else % if luatex or xetex
  \ifXeTeX
    \usepackage{mathspec} % this also loads fontspec
  \else
    \usepackage{unicode-math} % this also loads fontspec
  \fi
  \defaultfontfeatures{Scale=MatchLowercase}
  \defaultfontfeatures[\rmfamily]{Ligatures=TeX,Scale=1}
\fi
\usepackage{lmodern}
\ifPDFTeX\else  
    % xetex/luatex font selection
  \setmainfont[Scale = MatchLowercase]{Scala Pro}
  \setsansfont[]{Scala Sans Pro}
  \ifXeTeX
    \setmathfont(Digits,Latin,Greek)[]{Scala Pro}
  \else
    \setmathfont[]{Scala Pro}
  \fi
\fi
% Use upquote if available, for straight quotes in verbatim environments
\IfFileExists{upquote.sty}{\usepackage{upquote}}{}
\IfFileExists{microtype.sty}{% use microtype if available
  \usepackage[]{microtype}
  \UseMicrotypeSet[protrusion]{basicmath} % disable protrusion for tt fonts
}{}
\makeatletter
\@ifundefined{KOMAClassName}{% if non-KOMA class
  \IfFileExists{parskip.sty}{%
    \usepackage{parskip}
  }{% else
    \setlength{\parindent}{0pt}
    \setlength{\parskip}{6pt plus 2pt minus 1pt}}
}{% if KOMA class
  \KOMAoptions{parskip=half}}
\makeatother
\usepackage{xcolor}
\usepackage[left=1in,marginparwidth=2.0666666666667in,textwidth=4.1333333333333in,marginparsep=0.3in]{geometry}
\setlength{\emergencystretch}{3em} % prevent overfull lines
\setcounter{secnumdepth}{3}
% Make \paragraph and \subparagraph free-standing
\ifx\paragraph\undefined\else
  \let\oldparagraph\paragraph
  \renewcommand{\paragraph}[1]{\oldparagraph{#1}\mbox{}}
\fi
\ifx\subparagraph\undefined\else
  \let\oldsubparagraph\subparagraph
  \renewcommand{\subparagraph}[1]{\oldsubparagraph{#1}\mbox{}}
\fi


\providecommand{\tightlist}{%
  \setlength{\itemsep}{0pt}\setlength{\parskip}{0pt}}\usepackage{longtable,booktabs,array}
\usepackage{calc} % for calculating minipage widths
% Correct order of tables after \paragraph or \subparagraph
\usepackage{etoolbox}
\makeatletter
\patchcmd\longtable{\par}{\if@noskipsec\mbox{}\fi\par}{}{}
\makeatother
% Allow footnotes in longtable head/foot
\IfFileExists{footnotehyper.sty}{\usepackage{footnotehyper}}{\usepackage{footnote}}
\makesavenoteenv{longtable}
\usepackage{graphicx}
\makeatletter
\def\maxwidth{\ifdim\Gin@nat@width>\linewidth\linewidth\else\Gin@nat@width\fi}
\def\maxheight{\ifdim\Gin@nat@height>\textheight\textheight\else\Gin@nat@height\fi}
\makeatother
% Scale images if necessary, so that they will not overflow the page
% margins by default, and it is still possible to overwrite the defaults
% using explicit options in \includegraphics[width, height, ...]{}
\setkeys{Gin}{width=\maxwidth,height=\maxheight,keepaspectratio}
% Set default figure placement to htbp
\makeatletter
\def\fps@figure{htbp}
\makeatother
% definitions for citeproc citations
\NewDocumentCommand\citeproctext{}{}
\NewDocumentCommand\citeproc{mm}{%
  \begingroup\def\citeproctext{#2}\cite{#1}\endgroup}
\makeatletter
 % allow citations to break across lines
 \let\@cite@ofmt\@firstofone
 % avoid brackets around text for \cite:
 \def\@biblabel#1{}
 \def\@cite#1#2{{#1\if@tempswa , #2\fi}}
\makeatother
\newlength{\cslhangindent}
\setlength{\cslhangindent}{1.5em}
\newlength{\csllabelwidth}
\setlength{\csllabelwidth}{3em}
\newenvironment{CSLReferences}[2] % #1 hanging-indent, #2 entry-spacing
 {\begin{list}{}{%
  \setlength{\itemindent}{0pt}
  \setlength{\leftmargin}{0pt}
  \setlength{\parsep}{0pt}
  % turn on hanging indent if param 1 is 1
  \ifodd #1
   \setlength{\leftmargin}{\cslhangindent}
   \setlength{\itemindent}{-1\cslhangindent}
  \fi
  % set entry spacing
  \setlength{\itemsep}{#2\baselineskip}}}
 {\end{list}}
\usepackage{calc}
\newcommand{\CSLBlock}[1]{\hfill\break#1\hfill\break}
\newcommand{\CSLLeftMargin}[1]{\parbox[t]{\csllabelwidth}{\strut#1\strut}}
\newcommand{\CSLRightInline}[1]{\parbox[t]{\linewidth - \csllabelwidth}{\strut#1\strut}}
\newcommand{\CSLIndent}[1]{\hspace{\cslhangindent}#1}

\setlength\heavyrulewidth{0ex}
\setlength\lightrulewidth{0ex}
\makeatletter
\def\@maketitle{%
\newpage
\null
\vskip 2em%
\begin{center}%
\let \footnote \thanks
  {\LARGE \@title \par}%
  \vskip 1.5em%
  {\large
    \lineskip .5em%
    \begin{tabular}[t]{c}%
      \@author
    \end{tabular}\par}%
  %\vskip 1em%
  %{\large \@date}%
\end{center}%
\par
\vskip 1.5em}
\makeatother
\KOMAoption{captions}{tableheading}
\makeatletter
\@ifpackageloaded{caption}{}{\usepackage{caption}}
\AtBeginDocument{%
\ifdefined\contentsname
  \renewcommand*\contentsname{Table of contents}
\else
  \newcommand\contentsname{Table of contents}
\fi
\ifdefined\listfigurename
  \renewcommand*\listfigurename{List of Figures}
\else
  \newcommand\listfigurename{List of Figures}
\fi
\ifdefined\listtablename
  \renewcommand*\listtablename{List of Tables}
\else
  \newcommand\listtablename{List of Tables}
\fi
\ifdefined\figurename
  \renewcommand*\figurename{Figure}
\else
  \newcommand\figurename{Figure}
\fi
\ifdefined\tablename
  \renewcommand*\tablename{Table}
\else
  \newcommand\tablename{Table}
\fi
}
\@ifpackageloaded{float}{}{\usepackage{float}}
\floatstyle{ruled}
\@ifundefined{c@chapter}{\newfloat{codelisting}{h}{lop}}{\newfloat{codelisting}{h}{lop}[chapter]}
\floatname{codelisting}{Listing}
\newcommand*\listoflistings{\listof{codelisting}{List of Listings}}
\makeatother
\makeatletter
\makeatother
\makeatletter
\@ifpackageloaded{caption}{}{\usepackage{caption}}
\@ifpackageloaded{subcaption}{}{\usepackage{subcaption}}
\makeatother
\makeatletter
\@ifpackageloaded{sidenotes}{}{\usepackage{sidenotes}}
\@ifpackageloaded{marginnote}{}{\usepackage{marginnote}}
\makeatother
\ifLuaTeX
  \usepackage{selnolig}  % disable illegal ligatures
\fi
\IfFileExists{bookmark.sty}{\usepackage{bookmark}}{\usepackage{hyperref}}
\IfFileExists{xurl.sty}{\usepackage{xurl}}{} % add URL line breaks if available
\urlstyle{same} % disable monospaced font for URLs
\hypersetup{
  pdftitle={Dogmatism, Probability and Logical Uncertainty},
  pdfauthor={David Jehle; Brian Weatherson},
  colorlinks=true,
  linkcolor={black},
  filecolor={Maroon},
  citecolor={Blue},
  urlcolor={Blue},
  pdfcreator={LaTeX via pandoc}}

\title{Dogmatism, Probability and Logical Uncertainty}
\author{David Jehle \and Brian Weatherson}
\date{2012-01-01}

\begin{document}
\maketitle
Many epistemologists hold that an agent can come to justifiably believe
that \(p\) is true by seeing that it appears that \(p\) is true, without
having any antecedent reason to believe that visual impressions are
generally reliable. Certain reliabilists think this, at least if the
agent's vision is generally reliable. And it is a central tenet of
dogmatism (as described by Pryor (\citeproc{ref-Pryor2000}{2000}) and
Pryor (\citeproc{ref-Pryor2004}{2004})) that this is possible. Against
these positions it has been argued (e.g.~by Cohen
(\citeproc{ref-Cohen2005}{2005}) and White
(\citeproc{ref-White2006}{2006})) that this violates some principles
from probabilistic learning theory. To see the problem, let's note what
the dogmatist thinks we can learn by paying attention to how things
appear. (The reliabilist says the same things, but we'll focus on the
dogmatist.)

Suppose an agent receives an appearance that \(p\), and comes to believe
that \(p\). Letting \emph{Ap} be the proposition that it appears to the
agent that \(p\), and \(\rightarrow\) be the material conditional, we
can say that the agent learns that \(p\), and hence is in a position to
infer \(Ap \rightarrow p\), once they receive the evidence
\emph{Ap}.\sidenote{\footnotesize We're assuming here that the agent's evidence really
  is \emph{Ap}, not \(p\). That's a controversial assumption, but it
  isn't at issue in this debate.} This is surprising, because we can
prove the following.\sidenote{\footnotesize Popper and Miller
  (\citeproc{ref-PopperMiller1987}{1987}) prove a stronger result than
  Theorem One, and note its significance for probabilistic models of
  learning.}

\begin{quote}
\textbf{Theorem 1}\\
If \(Pr\) is a classical probability function, then\\
\(Pr(Ap \rightarrow p | Ap) \leq Pr(Ap \rightarrow p)\).
\end{quote}

(All the theorems are proved in the appendix.) We can restate Theorem 1
in the following way, using classically equvalent formulations of the
material conditional.

\begin{quote}
\textbf{Theorem 2}\\
If \(Pr\) is a classical probability function, then

\begin{itemize}
\tightlist
\item
  \(Pr(\neg(Ap \wedge \neg p) | Ap) \leq Pr(\neg(Ap \wedge \neg p))\);
  and
\item
  \(Pr(\neg Ap \vee p | Ap) \leq Pr(\neg Ap \vee p)\).
\end{itemize}
\end{quote}

And that's a problem for the dogmatist if we make the standard Bayesian
assumption that some evidence \(E\) is only evidence for hypothesis
\(H\) if \(Pr(H | E) > Pr(H)\). For here we have cases where the
evidence the agent receives does not raise the probability of
\(Ap \rightarrow p\), \(\neg(Ap \wedge \neg p)\) or \(\neg Ap \vee p\),
so the agent has not received any evidence for them, but getting this
evidence takes them from not having a reason to believe these
propositions to having a reason to get them.

In this paper, we offer a novel response for the dogmatist. The proof of
Theorem 1 makes crucial use of the logical equivalence between
\(Ap \rightarrow p\) and
\(((Ap \rightarrow p) \wedge Ap) \vee ((Ap \rightarrow p) \wedge \neg Ap)\).
These propositions are equivalent in classical logic, but they are not
equivalent in intuitionistic logic. Exploiting this non-equivalence, we
derive two claims. In Section 1 we show that Theorems 1 and 2 fail in
intuitionistic probability theory. In Section 2 we consider how an agent
who is unsure whether classical or intuitionistic logic is correct
should apportion their credences. We conclude that for such an agent,
theorems analogous to Theorems 1 and 2 fail even if the agent thinks it
extremely unlikely that intuitionistic logic is the correct logic. The
upshot is that if it is rationally permissible to be even a little
unsure whether classical or intuitionistic logic is correct, it is
possible that getting evidence that \(Ap\) raises the rational
credibility of \(Ap \rightarrow p\), \(\neg(Ap \wedge \neg p)\) and
\(\neg Ap \vee p\).

\section{Intuitionistic Probability}\label{intuitionistic-probability}

In Weatherson (\citeproc{ref-Weatherson2003}{2003}), the notion of a
\(\vdash\)-probability function, where \(\vdash\) is an entailment
relation, is introduced. For any \(\vdash\), a \(\vdash\)-probability
function is a function \(Pr\) from sentences in the language of
\(\vdash\) to \([0, 1]\) satisfying the following four
constraints.\sidenote{\footnotesize We'll usually assume that the language of
  \(\vdash\) is a familiar kind of propositional calculus, with a
  countable infinity of sentence letters, and satisfying the usual
  recursive constraints. That is, if \(A\) and \(B\) are sentences of
  the language, then so are \(\neg A\), \(A \rightarrow B\),
  \(A \wedge B\) and \(A \vee B\). It isn't entirely trivial to extend
  some of our results to a language that contains quantifiers. This is
  because once we add quantifiers, intuitionistic and classical logic no
  longer have the same anti-theorems. But that complication is outside
  the scope of this paper. Note that for Theorem 6, we assume a
  restricted language with just two sentence letters. This merely
  simplifies the proof. A version of the construction we use there with
  those two letters being simply the first two sentence letters would be
  similar, but somewhat more complicated.}

\begin{description}
\item[(P0)]
\(Pr(p) = 0\) if \(p\) is a \(\vdash\)-antithesis, i.e.~iff for any
\(X, p \vdash X\).
\item[(P1)]
\(Pr(p) = 1\) if \(p\) is a \(\vdash\)-thesis, i.e.~iff for any
\(X, X \vdash p\).
\item[(P2)]
If \(p \vdash q\) then \(Pr(p) \leq Pr(q)\).
\item[(P3)]
\(Pr(p) + Pr(q) = Pr(p \vee q) + Pr(p \wedge q)\).
\end{description}

We'll use \(\vdash_{CL}\) to denote the classical entailment relation,
and \(\vdash_{IL}\) to denote the intuitionist entailment relation. Then
what we usually take to be probability functions are
\(\vdash_{CL}\)-probability functions. And intuitionist probability
functions are \(\vdash_{IL}\)-probability functions.

In what follows we'll make frequent appeal to three obvious consequences
of these axioms, consequences which are useful enough to deserve their
own names. Hopefully these are obvious enough to pass without
proof.\sidenote{\footnotesize Weatherson (\citeproc{ref-Weatherson2003}{2003})
  discusses what happens if we make P2\(^*\) or P3\(^*\) an axiom in
  place of either P2 and P3. It is argued there that this gives us too
  many functions to be useful in epistemology. The arguments in Williams
  (\citeproc{ref-Williamsms}{2012}) provide much stronger reasons for
  believing this conclusion is correct.}

\begin{description}
\item[(P1\(^*\))]
\(0 \leq Pr(p) \leq 1\).
\item[(P2\(^*\))]
If \(p \dashv \vdash q\) then \(Pr(p) = Pr(q)\).
\item[(P3\(^*\))]
If \(p \wedge q\) is a \(\vdash\)-antithesis, then
\(Pr(p) + Pr(q) = Pr(p \vee q)\).
\end{description}

\(\vdash\)-probability functions obviously concern unconditional
probability, but we can easily extend them into conditional
\(\vdash\)-probability functions by adding the following
axioms.\sidenote{\footnotesize For the reasons given in Hájek
  (\citeproc{ref-Hajek2003}{2003}), it is probably better in general to
  take conditional probability as primitive. But for our purposes taking
  unconditional probability to be basic won't lead to any problems, so
  we'll stay neutral on whether conditional or unconditional probability
  is really primitive.}

\begin{description}
\item[(P4)]
If \(r\) is not a \(\vdash\)-antithesis, then \(Pr(\cdot | r)\) is a
\(\vdash\)-probability function; i.e., it satisfies P0-P3.
\item[(P5)]
If \(r \vdash p\) then \(Pr(p | r) = 1\).
\item[(P6)]
If \(r\) is not a \(\vdash\)-antithesis, then
\(Pr(p \wedge q | r) = Pr(p | q \wedge r)Pr(q | r)\).
\end{description}

There is a simple way to generate \(\vdash_{CL}\) probability functions.
Let \(\langle W, V\rangle\) be a model where \(W\) is a finite set of
worlds, and \(V\) a valuation function defined on them with respect to a
(finite) set \(K\) of atomic sentences, i.e., a function from \(K\) to
subsets of \(W\). Let \(L\) be the smallest set including all members of
\(K\) such that whenever \(A\) and \(B\) are in \(L\), so are
\(A \wedge B\), \(A \vee B\), \(A \rightarrow B\) and \(\neg A\). Extend
\(V\) to \(V^*\), a function from \(L\) to subsets of \(W\) using the
usual recursive definitions of the sentential connectives. (So
\(w \in V^*(A \wedge B)\) iff \(w \in V^*(A)\) and \(w \in V^*(B)\), and
so on for the other connectives.) Let \(m\) be a measure function
defined over subsets of W. Then for any sentence \(S\) in \(L\),
\(Pr(S)\) is \(m(\{w: w \in V^*(S)\})\). It isn't too hard to show that
Pr is a \(\vdash_{CL}\) probability function.

There is a similar way to generate \(\vdash_{IL}\) probability
functions. This method uses a simplified version of the semantics for
intuitionistic logic in Kripke (\citeproc{ref-Kripke1965}{1965}). Let
\(\langle W, R, V\rangle\) be a model where \(W\) is a finite set of
worlds, \(R\) is a reflexive, transitive relation defined on \(W\), and
\(V\) is a valuation function defined on them with respect to a (finite)
set \(K\) of atomic sentences. We require that \(V\) be closed with
respect to \(R\), i.e.~that if \(x \in V(p)\) and \(xRy\), then
\(y \in V(p)\). We define \(L\) the same way as above, and extend \(V\)
to \(V^*\) (a function from \(L\) to subsets of \(W\)) using the
following definitions.

\begin{quote}
\(w \in V^*(A \wedge B)\) iff \(w \in V^*(A)\) and \(w \in V^*(B)\).\\
\(w \in V^*(A \vee B)\) iff \(w \in V^*(A)\) or \(w \in V^*(B)\).\\
\(w \in V^*(A \rightarrow B)\) iff for all \(w^{\prime}\) such that
\(wRw^{\prime}\) and \(w^{\prime}\in V^*(A), w^{\prime} \in V^*(B)\).\\
\(w \in V^*(\neg A)\) iff for all \(w^{\prime}\) such that
\(wRw^{\prime}\), it is not the case that \(w^{\prime} \in V^*(A)\).
\end{quote}

Finally, we let \(m\) be a measure function defined over subsets of
\(W\). And for any sentence \(S\) in \(L\), \(Pr(S)\) is
\(m(\{w: w \in V^*(S)\})\). Weatherson
(\citeproc{ref-Weatherson2003}{2003}) shows that any such \(Pr\) is a
\(\vdash_{IL}\) probability function.

To show that Theorem 1 may fail when \(Pr\) is \(\vdash_{IL}\) a
probability function, we need a model we'll call \(M\). The valuation
function in \(M\) is defined with respect to a language where the only
atomic propositions are \(p\) and \(Ap\).

\[
\begin{aligned}
W &= \{1, 2, 3\} \\
R &=  \{\langle 1, 1\rangle , \langle 2, 2\rangle , \langle 3, 3\rangle , \langle 1, 2\rangle , \langle 1, 3\rangle \} \\
V(p) &= \{2\} \\
V(Ap) &= \{2, 3\}
\end{aligned}
\]

Graphically, \(M\) looks like this.

\begin{center}
\setlength{\unitlength}{1mm}
\begin{picture}(70, 50)
\thicklines
\put(35, 5){\vector(-1, 1){30}}
\put(35, 5){\vector(1, 1){30}}
\put(35,5){\circle*{2}}
\put(4.8,35.5){\circle*{2}}
\put(65.2,35.5){\circle*{2}}
\put(28, 5){$1$}
\put(0,35.5){$2$}
\put(60,35.5){$3$}
\put(7,35.5){$Ap, p$}
\put(67,35.5){$Ap$}
\end{picture}
\end{center}

We'll now consider a family of measures over \(m\). For any
\(x \in (0, 1)\), let \(m_x\) be the measure function such that
\(m_x(\{1\}) = 1 - x, m_x(\{2\}) = x\), and \(m_x(\{3\}) = 0\).
Corresponding to each function \(m_x\) is a \(\vdash_{IL}\) probability
function we'll call \(Pr_x\). Inspection of the model shows that Theorem
3 is true.

\begin{quote}
\textbf{Theorem 3}.\\
In \(M\), for any \(x \in (0, 1)\),

\begin{enumerate}
\def\labelenumi{\arabic{enumi}.}
\tightlist
\item
  \(Pr_x(Ap \rightarrow p)\) =
  \(Pr_x((Ap \rightarrow p) \wedge Ap) = x\)
\item
  \(Pr_x(\neg Ap \vee p)\) = \(Pr_x((\neg Ap \vee p) \wedge Ap) = x\)
\item
  \(Pr_x(\neg(Ap \wedge \neg p))\) =
  \(Pr_x(\neg(Ap \wedge \neg p) \wedge Ap) = x\)
\end{enumerate}
\end{quote}

An obvious corollary of Theorem 3 is

\begin{quote}
\textbf{Theorem 4}.\\
For any \(x \in (0, 1)\),

\begin{enumerate}
\def\labelenumi{\arabic{enumi}.}
\tightlist
\item
  \(1 = Pr_x(Ap \rightarrow p | Ap) > Pr_x(Ap \rightarrow p) = x\)
\item
  \(1 = Pr_x(\neg Ap \vee p | Ap) > Pr_x(\neg Ap \vee p) = x\)
\item
  \(1 = Pr_x(\neg(Ap \wedge \neg p) | Ap) > Pr_x(\neg(Ap \wedge \neg p)) = x\)
\end{enumerate}
\end{quote}

So for any \(x\), conditionalising on \(Ap\) actually raises the
probability of \(Ap \rightarrow p, \neg(Ap \wedge \neg p)\) and
\(\neg Ap \vee p\) with respect to \(Pr_x\). Indeed, since \(x\) could
be arbitrarily low, it can raise the probability of each of these three
propositions from any arbitrarily low value to 1. So it seems that if we
think learning goes by conditionalisation, then receiving evidence
\(Ap\) could be sufficient grounds to justify belief in these three
propositions. Of course, this relies on our being prepared to use the
intuitionist probability calculus. For many, this will be considered too
steep a price to pay to preserve dogmatism. But in section 2 we'll show
that the dogmatist does not need to insist that intuitionistic logic is
the correct logic for modelling uncertainty. All they need to show is
that it \emph{might} be correct, and then they'll have a response to
this argument.

\section{Logical Uncertainty}\label{logical-uncertainty}

We're going to build up to a picture of how to model agents who are
rationally uncertain about whether the correct logic is classical or
intuitionistic. But let's start by thinking how an agent who is unsure
which of two empirical theories \(T_1\) or \(T_2\) is correct. We'll
assume that the agent is using the classical probability calculus, and
the agent knows which propositions are entailed by each of the two
theories. And we'll also assume that the agent is sure that it's not the
case that each of these theories is false, and the theories are
inconsistent, so they can't both be true.

The natural thing then is for the agent to have some credence \(x\) in
\(T_1\), and credence \(1-x\) in \(T_2\). She will naturally have a
picture of what the world is like assuming \(T_1\) is correct, and on
that picture every proposition entailed by \(T_1\) will get probability
1. And she'll have a picture of what the world is like assuming \(T_2\)
is correct. Her overall credal state will be a mixture of those two
pictures, weighted according to the credibility of \(T_1\) and \(T_2\).

If we're working with unconditional credences as primitive, then it is
easy to mix two probability functions to produce a credal function which
is also a probability function. Let \(Pr_1\) be the probability function
that reflects the agent's views about how things probably are
conditional on \(T_1\) being true, and \(Pr_2\) the probability function
that reflects her views about how things probably are conditional on
\(T_2\) being true. Then for any \(p\), let
\(Cr(p) = xPr_1(p) + (1-x)Pr_2(p)\), where \(Cr\) is the agent's
credence function.

It is easy to see that \(Cr\) will be a probability function. Indeed,
inspecting the axioms P0-P3 makes it obvious that for any \(\vdash\),
mixing two \(\vdash\)-probability functions as we've just done will
always produce a \(\vdash\)-probability function. The axioms just
require that probabilities stand in certain equalities and inequalities
that are obviously preserved under mixing.

It is a little trickier to mix conditional probability functions in an
intuitive way, for the reasons set out in Jehle and Fitelson
(\citeproc{ref-Jehle2009}{2009}). But in a special case, these
difficulties are not overly pressing. Say that a \(\vdash\)-probability
function is \textbf{regular} iff for any \emph{p, q} in its domain,
\(Pr(p | q) = 0\) iff \(p \wedge q\) is a \(\vdash\)-antitheorem. Then,
for any two regular conditional probability functions \(Pr_1\) and
\(Pr_2\) we can create a weighted mixture of the two of them by taking
the new unconditional probabilities, i.e.~the probabilities of \(p\)
given \(T\), where \(T\) is a theorem, to be weighted sums of the
unconditional probabilities in \(Pr_1\) and \(Pr_2\). That is, our new
function \(Pr_3\) is given by:

\[
Pr_3(p | T) = xPr_1(p | T) + (1-x)Pr_2(p | T)
\]

In the general case, this does not determine exactly which function
\(Pr_3\) is, since it doesn't determine the value of \(Pr_3(p | q)\)
when \(Pr_1(q | T) = Pr_2(q | T) = 0\). But since we're paying attention
just to regular functions this doesn't matter. If the function is
regular, then we can just let the familiar ratio account of conditional
probability be a genuine definition. So in general we have,

\[
Pr_3(p | q) = \frac{Pr_3(p \wedge q | T)}{Pr_3(q | T)}
\]

And since the numerator is 0 iff \(q\) is an anti-theorem, whenever
\(Pr(p | q)\) is supposed to be defined, i.e.~when \(q\) is not an
anti-theorem, the right hand side will be well defined. As we noted,
things get a lot messier when the functions are not regular, but those
complications are unnecessary for the story we want to tell.

Now in the cases we've been considering so far, we've been assuming that
\(T_1\) and \(T_2\) are empirical theories, and that we could assume
classical logic in the background. Given all that, most of what we've
said in this section has been a fairly orthodox treatment of how to
account for a kind of uncertainty. But there's no reason, we say, why we
should restrict \(T_1\) and \(T_2\) in this way. We could apply just the
same techniques when \(T_1\) and \(T_2\) are theories of entailment.

When \(T_1\) is the theory that classical logic is the right logic of
entailment, and \(T_2\) the theory that intuitionistic logic is the
right logic of entailment, then \(Pr_1\) and \(Pr_2\) should be
different kinds of probability functions. In particular, \(Pr_1\) should
be a \(\vdash_{CL}\)-probability function, and \(Pr_2\) should be a
\(\vdash_{IL}\)-probability function. That's because \(Pr_1\) represents
how things probably are given \(T_1\), and given \(T_1\), how things
probably are is constrained by classical logic. And \(Pr_2\) represents
how things probably are given \(T_2\), and given \(T_2\), how things
probably are is constrained by intuitionistic logic.

If we do all that, we're pushed towards the thought that the if someone
is uncertain whether the right logic is intuitionistic or classical
logic, then the right theory of probability for them is intuitionistic
probability theory. That's because of Theorem 5.

\begin{quote}
\textbf{Theorem 5}\\
Let \(Pr_1\) be a regular conditional \(\vdash_{CL}\)-probability
function, and \(Pr_2\) be a regular conditional
\(\vdash_{IL}\)-probability function that is not a
\(\vdash_{CL}\)-probability function. And let \(Pr_3\) be defined as in
the text. (That is, \(Pr_3(A) = xPr_1(A) + (1-x)Pr_2(A)\), and
\(Pr_3(A | B) = \frac{Pr_3(A \wedge B)}{Pr_3(B)}\).) Then \(Pr_3\) is a
regular conditional \(\vdash_{IL}\)-probability function.
\end{quote}

That's to say, if the agent is at all unsure whether classical logic or
intuitionistic logic is the correct logic, then their credence function
should be an intuitionistic probability function.

Of course, if the agent is very confident that classical logic is the
correct logic, then they couldn't rationally have their credences
distributed by any old intuitionistic probability function. After all,
there are intuitionistic probability functions such that
\(Pr(p \vee \neg p) = 0\), but an agent whose credence that classical
logic is correct is, say, 0.95, could not reasonably have credence 0 in
\(p \vee \neg p\). For our purposes, this matters because we want to
show that an agent who is confident, but not certain, that classical
logic is correct can nevertheless be a dogmatist. To fill in the
argument we need,

\begin{quote}
\textbf{Theorem 6}\\
Let \(x\) be any real in \((0, 1)\). Then there is a probability
function \(Cr\) that (a) is a coherent credence function for someone
whose credence that classical logic is correct is \(x\), and (b)
satisfies each of the following inequalities: \[
\begin{aligned}
Pr(Ap \rightarrow p | Ap) &> Pr(Ap \rightarrow p) \\
Pr(\neg Ap \vee p | Ap) &> Pr(\neg Ap \vee p) \\
Pr(\neg(Ap \wedge \neg p) | Ap) &> Pr(\neg(Ap \wedge \neg p)) 
\end{aligned}
\]
\end{quote}

The main idea driving the proof of Theorem 6 which is set out in the
appendix, is that if intuitionistic logic is correct, it's possible that
conditionalising on \emph{Ap} raises the probability of each of these
three propositions from arbitrarily low values to 1. So as long as the
prior probability of each of the three propositions, conditional on
intuitionistic logic being correct, is low enough, it can still be
raised by conditionalising on \emph{Ap}.

More centrally, we think Theorem 6 shows that the probabilistic argument
against dogmatism is not compelling. The original argument noted that
the dogmatist says that we can learn the three propositions in Theorem
6, most importantly \(Ap \rightarrow p\), by getting evidence \emph{Ap}.
And it says this is implausible because conditionalising on \emph{Ap}
lowers the probability of \(Ap \rightarrow p\). But it turns out this is
something of an artifact of the very strong classical assumptions that
are being made. The argument not only requires the correctness of
classical logic, it requires that the appropriate credence the agent
should have in classical logic's being correct is one. And that
assumption is, we think, wildly implausible. Even if the agent should be
\emph{very confident} that classical logic is the correct logic, it
shouldn't be a requirement of rationality that she be absolutely certain
that it is correct.

So we conclude that this argument fails. A dogmatist about perception
who is at least minimally open-minded about logic can marry perceptual
dogmatism to a probabilistically coherent theory of confirmation.

This paper is one more attempt on our behalf to defend dogmatism from a
probabilistic challenge. Weatherson
(\citeproc{ref-Weatherson2007}{2007}) defends dogmatism from the
so-called ``Bayesian objection''. And Jehle
(\citeproc{ref-JehlePhD}{2009}) not only shows that dogmatism can be
situated nicely into a probabilistically coherent theory of
confirmation, but also that within such a theory, many of the
traditional objections to dogmatism are easily rebutted. We look forward
to future research on the connections between dogmatism and probability,
but we remain skeptical that dogmatism will be undermined solely by
probabilistic considerations.

\section*{Appendix: Proofs}\label{appendix-proofs}
\addcontentsline{toc}{section}{Appendix: Proofs}

\begin{quote}
\textbf{Theorem 1}\\
If \(Pr\) is a classical probability function, then\\
\(Pr(Ap \rightarrow p | Ap) \leq Pr(Ap \rightarrow p)\).
\end{quote}

\textbf{Proof}: Assume \(Pr\) is a classical probability function, and
\(\vdash\) the classical consequence relation.

\[
\begin{aligned}
1. &Ap \rightarrow p \dashv \vdash ((Ap \rightarrow p) \wedge Ap) \vee ((Ap \rightarrow p) \wedge \neg Ap) & \text{} \\
2. &Pr(Ap \rightarrow p) = Pr(((Ap \rightarrow p) \wedge Ap) \vee ((Ap \rightarrow p) \wedge \neg Ap)) & \text{1, P2$^*$} \\
3. & Pr ((Ap \rightarrow p) \wedge Ap) \vee ((Ap \rightarrow p) \wedge \neg Ap)) = \\&Pr ((Ap \rightarrow p) \wedge Ap) + Pr ((Ap \rightarrow p) \wedge \neg Ap)  
 & \text{P3$^*$}  \\
4. &Pr((Ap \rightarrow p) \wedge Ap) = Pr (Ap)Pr(Ap \rightarrow p|Ap) & \text{P6} \\
5. &Pr((Ap \rightarrow p) \wedge \neg Ap) = Pr(\neg Ap)Pr(Ap \rightarrow p |\neg Ap) & \text{P6} \\
6. &Pr(Ap \rightarrow p) = \\&Pr(Ap)Pr(Ap \rightarrow p|Ap) + Pr (\neg Ap)Pr(Ap \rightarrow p |\neg Ap) & \text{2, 4, 5} \\
7. &(Ap \rightarrow p) \wedge Ap \dashv \vdash \neg Ap & \text{} \\
8. &Pr((Ap \rightarrow p) \wedge Ap) = Pr(\neg Ap) & \text{7, P2$^*$} \\
9. &Pr(Ap \rightarrow p |\neg Ap) = 1 \text{ or } Pr(\neg Ap) = 0 & \text{8, P6}  \\
10. &Pr(Ap \rightarrow p | Ap) \leq 1 & \text{P4, P5} \\
11. &Pr(Ap \rightarrow p) \geq \\ &Pr(Ap)Pr(Ap \rightarrow p|Ap) + Pr (\neg Ap)Pr(Ap \rightarrow p |Ap)  & \text{6, 9, 10} \\
12. &\vdash Ap \vee \neg Ap & \text{} \\
13. &Pr(Ap \vee \neg Ap) = 1 & \text{12, P1} \\
14. &Pr(Ap) + Pr (\neg Ap) = 1 & \text{13, P3$^*$} \\
15. &Pr(Ap \rightarrow p ) \geq Pr (Ap \rightarrow p|Ap) & \text{11, 14} 
\end{aligned}
\]

Note (11) is an equality iff (8) is. The only step there that may not be
obvious is step 10. The reason it holds is that either \(Ap\) is a
\(\vdash\)-antitheorem or it isn't. If it is, then it entails
\(Ap \rightarrow p\), so by P5, \(Pr(Ap \rightarrow p | Ap) \leq 1\). If
it is not, then by P1\(^*\), \(Pr(x | Ap) \leq 1\) for any \(x\), so
\(Pr(Ap \rightarrow p | Ap) \leq 1\).

\begin{quote}
\textbf{Theorem 2}\\
If \(Pr\) is a classical probability function, then

\begin{itemize}
\tightlist
\item
  \(Pr(\neg(Ap \wedge \neg p) | Ap) \leq Pr(\neg(Ap \wedge \neg p))\);
  and
\item
  \(Pr(\neg Ap \vee p | Ap) \leq Pr(\neg Ap \vee p)\).
\end{itemize}
\end{quote}

\textbf{Proof}: Assume \(Pr\) is a classical probability function, and
\(\vdash\) the classical consequence relation. \[\begin{aligned}
1. &Ap \rightarrow p \dashv  \vdash \neg(Ap \wedge \neg p) &  \\
2. &Pr(Ap \rightarrow p) = Pr(\neg(Ap \wedge \neg p)) & 1, P2^* \\
3. &Pr(Ap \rightarrow p | Ap) = Pr(\neg(Ap \wedge \neg p) | Ap) & 1, P4, P5 \\
4. &Pr(Ap \rightarrow p ) \geq Pr (Ap \rightarrow p|Ap) & \text{Theorem 1} \\
5. &Pr(\neg(Ap \wedge \neg p) | Ap) \geq Pr(\neg(Ap \wedge \neg p)) & 2, 3, 4 \\
6. &Ap \rightarrow p \dashv  \vdash \neg Ap \vee p &  \\
7. &Pr(Ap \rightarrow p) = Pr(\neg Ap \vee p) & 6, P2^* \\
8. &Pr(Ap \rightarrow p | Ap) = Pr(\neg Ap \vee p | Ap) & 6, P4, P5 \\
9. &Pr(\neg Ap \vee p | Ap) \geq Pr(\neg Ap \vee p) & 4, 7, 8\end{aligned}\]

The only minor complication is with step 3. There are two cases to
consider, either \(Ap\) is a \(\vdash\)-antitheorem or it isn't. If it
is a \(\vdash\)-antitheorem, then both the LHS and RHS of (3) equal 1,
so they are equal. If it is not a \(\vdash\)-antitheorem, then by P4,
\(Pr(\cdot | Ap)\) is a probability function. So by P2\(^*\), and the
fact that \(Ap \rightarrow p \dashv \vdash \neg(Ap \wedge \neg p)\), we
have that the LHS and RHS are equal.

\begin{quote}
\textbf{Theorem 3}.\\
In \(M\), for any \(x \in (0, 1)\),

\begin{enumerate}
\def\labelenumi{\arabic{enumi}.}
\tightlist
\item
  \(Pr_x(Ap \rightarrow p)\) =
  \(Pr_x((Ap \rightarrow p) \wedge Ap) = x\)
\item
  \(Pr_x(\neg Ap \vee p)\) = \(Pr_x((\neg Ap \vee p) \wedge Ap) = x\)
\item
  \(Pr_x(\neg(Ap \wedge \neg p))\) =
  \(Pr_x(\neg(Ap \wedge \neg p) \wedge Ap) = x\)
\end{enumerate}
\end{quote}

Recall what \(M\) looks like.

\begin{center}
\setlength{\unitlength}{1mm}
\begin{picture}(70, 50)
\thicklines
\put(35, 5){\vector(-1, 1){30}}
\put(35, 5){\vector(1, 1){30}}
\put(35,5){\circle*{2}}
\put(4.8,35.5){\circle*{2}}
\put(65.2,35.5){\circle*{2}}
\put(28, 5){$1$}
\put(0,35.5){$2$}
\put(60,35.5){$3$}
\put(7,35.5){$Ap, p$}
\put(67,35.5){$Ap$}
\end{picture}
\end{center}

The only point where \(Ap \rightarrow p\) is true is at 2. Indeed,
\(\neg(Ap \rightarrow p)\) is true at 3, and neither
\(Ap \rightarrow p\) nor \(\neg(Ap \rightarrow p)\) are true at 1. So
\(Pr_x(Ap \rightarrow p) = m_x(\{2\}) = x\). Since \emph{Ap} is also
true at 2, that's the only point where \((Ap \rightarrow p) \wedge Ap\)
is true. So it follows that
\(Pr_x((Ap \rightarrow p) \wedge Ap) = m_x(\{2\}) = x\).

Similar inspection of the model shows that 2 is the only point where
\(\neg(Ap \wedge \neg p)\) is true, and the only point where
\(\neg Ap \vee p\) is true. And so (b) and (c) follow in just the same
way.

In slight contrast, \(Ap\) is true at two points in the model, 2 and 3.
But since \(m_x(\{3\}) = 0\), it follows that
\(m_x(\{2, 3\}) = m_x(\{2\}) = x\). So \(Pr_x(Ap) = x\).

\begin{quote}
\textbf{Theorem 4}.\\
For any \(x \in (0, 1)\),

\begin{enumerate}
\def\labelenumi{\arabic{enumi}.}
\tightlist
\item
  \(1 = Pr_x(Ap \rightarrow p | Ap) > Pr_x(Ap \rightarrow p) = x\)
\item
  \(1 = Pr_x(\neg Ap \vee p | Ap) > Pr_x(\neg Ap \vee p) = x\)
\item
  \(1 = Pr_x(\neg(Ap \wedge \neg p) | Ap) > Pr_x(\neg(Ap \wedge \neg p)) = x\)
\end{enumerate}
\end{quote}

We'll just go through the argument for (a); the other cases are similar.
By P6, we know that
\(Pr_x(\neg(Ap \wedge \neg p) | Ap) Pr_x(Ap) = Pr_x((Ap \rightarrow p) \wedge Ap)\).
By Theorem 3, we know that
\(Pr_x(Ap) = Pr_x((Ap \rightarrow p) \wedge Ap)\), and that both sides
are greater than 0. (Note that the theorem is only said to hold for
\(x > 0\).) The only way both these equations can hold is if
\(Pr_x(\neg(Ap \wedge \neg p) | Ap) = 1\). Note also that by hypothesis,
\(x < 1\), and from this claim (a) follows. The other two cases are
completely similar.

\begin{quote}
\textbf{Theorem 5}\\
Let \(Pr_1\) be a regular conditional \(\vdash_{CL}\)-probability
function, and \(Pr_2\) be a regular conditional
\(\vdash_{IL}\)-probability function that is not a
\(\vdash_{CL}\)-probability function. And let \(Pr_3\) be defined as in
the text. (That is, \(Pr_3(A) = xPr_1(A) + (1-x)Pr_2(A)\), and
\(Pr_3(A | B) = \frac{Pr_3(A \wedge B)}{Pr_3(B)}\).) Then \(Pr_3\) is a
regular conditional \(\vdash_{IL}\)-probability function.
\end{quote}

We first prove that \(Pr_3\) satisfies the requirements of an
unconditional \(\vdash_{IL}\)-probability function, and then show that
it satisfies the requirements of a conditional
\(\vdash_{IL}\)-probability function.

If \(p\) is an \(\vdash_{IL}\)-antithesis, then it is also a
\(\vdash_{CL}\)-antithesis. So \(Pr_1(p) = Pr_2(p) = 0\). So
\(Pr_3(A) = 0x + 0(1-x) = 0\), as required for \textbf{(P0)}.

If \(p\) is an \(\vdash_{IL}\)-thesis, then it is also a
\(\vdash_{CL}\)-thesis. So \(Pr_1(p) = Pr_2(p) = 1\). So
\(Pr_3(p) = x + (1-x) = 1\), as required for \textbf{(P1)}.

If \(p \vdash_{IL} q\) then \(p \vdash_{CL} q\). So we have both
\(Pr_1(p) \leq Pr(q)\) and \(Pr_2(p) \leq Pr_2(q)\). Since \(x \geq 0\)
and \((1-x) \geq 0\), these inequalities imply that
\(xPr_1(p) \leq xPr(q)\) and \((1-x)Pr_2(p) \leq (1-x)Pr_2(q)\). Summing
these, we get \(xPr_1(p) + (1-x)Pr_2(p) \leq xPr_1(q) + (1-x)Pr_2(q)\).
And by the definition of \(Pr_3\), that means that
\(Pr_3(p) \leq Pr_3(q)\), as required for \textbf{(P2)}.

Finally, we just need to show that
\(Pr_3(p) + Pr_3(q) = Pr_3(p \vee q) + Pr_3(p \wedge q)\), as follows:

\[
\begin{aligned}
Pr_3(p) + Pr_3(q) &= xPr_1(p) + (1-x)Pr_2(p) + xPr_1(q) + (1-x)Pr_2(q) \\
 &= x(Pr_1(p) + Pr_1(q)) + (1-x)(Pr_2(p) + Pr_2(q)) \\
 &= x(Pr_1(p \vee q) + Pr_1(p \wedge q)) + (1-x)(Pr_2(p \vee q) + Pr_2(p \wedge q)) \\
 &= xPr_1(p \vee q) + (1-x)Pr_2(p \vee q) + xPr_1(p \wedge q) + (1-x)Pr_2(p \wedge q) \\
 &= Pr_3(p \vee q) + Pr_3(p \wedge q) \text{ as required}
\end{aligned}
\]

Now that we have shown \(Pr_3\) is an unconditional
\(\vdash_{IL}\)-probability function, we need to show it is a
conditional \(\vdash_{IL}\)-probability function, where
\(Pr_3(p | r) =_{df} \frac{Pr_3(p \wedge r)}{Pr_3(r)}\). Remember we are
assuming that both \(Pr_1\) and \(Pr_2\) are regular, from which it
clearly follows that \(Pr_3\) is regular, so this definition is always
in order. (That is, we're never dividing by zero.) The longest part of
showing \(Pr_3\) is a conditional \(\vdash_{IL}\)-probability function
is showing that it satisfies \textbf{(P4}), which has four parts. We
need to show that \(Pr(\cdot | r)\) satisfies \textbf{(P0)-(P3)}.
Fortunately these are fairly straightforward.

If \(p\) is an \(\vdash_{IL}\)-antithesis, then so is \(p \wedge r\). So
\(Pr_3(p \wedge r) = 0\), so \(Pr_3(p | r) = 0\), as required for
\textbf{(P0)}.

If \(p\) is an \(\vdash_{IL}\)-thesis, then
\(p \wedge r \dashv \vdash r\), so \(Pr_3(p \wedge r) = Pr_3(r)\), so
\(Pr_3(p | r) = 1\), as required for \textbf{(P1)}.

If \(p \vdash_{IL} q\) then \(p \wedge r \vdash_{IL} q \wedge r\). So
\(Pr_3(p \wedge r) \leq Pr_3(q \wedge r)\). So
\(\frac{Pr_3(p \wedge r)}{Pr_3(r)} \leq \frac{Pr_3(q \wedge r)}{Pr_3(r)}\).
That is, \(Pr_3(p | r) \leq Pr_3(q | r)\), as required for
\textbf{(P2)}.

Finally, we need to show that
\(Pr_3(p | r) + Pr_3(q | r) = Pr_3(p \vee q | r) + Pr_3(p \wedge q | r)\),
as follows, making repeated use of the fact that \(Pr_3\) is an
unconditional \(\vdash_{IL}\)-probability function, so we can assume it
satisfies \textbf{(P3)}, and that we can substitute intuitionistic
equivalences inside \(Pr_3\).

\[
\begin{aligned}
Pr_3(p | r) + Pr_3(q | r) &= \frac{Pr_3(p \wedge r)}{Pr_3(r)} + \frac{Pr_3(q \wedge r)}{Pr_3(r)} \\
&= \frac{Pr_3(p \wedge r) + Pr(q \wedge r)}{Pr_3(r)} \\
&= \frac{Pr_3((p \wedge r) \vee (q \wedge r)) + Pr_3((p \wedge r) \wedge (q \wedge r))}{Pr_3(r)} \\
&=\frac{Pr_3(p \vee q) \wedge r) + Pr_3((p \wedge q) \wedge r)}{Pr_3(r)} \\
&=\frac{Pr_3(p \vee q) \wedge r)}{Pr_3(r)} + \frac{Pr_3((p \wedge q) \wedge r)}{Pr_3(r)} \\
&=Pr_3(p \vee q | r) + Pr_3(p \wedge q | r) \text{ as required}
\end{aligned}
\]

Now if \(r \vdash_{IL} p\), then
\(r \wedge p ~_{IL}\dashv \vdash_{IL} p\), so
\(Pr_3(r \wedge p) = Pr_3(p)\), so \(Pr_3(p | r) = 1\), as required for
\textbf{(P5)}.

Finally, we show that \(Pr_3\) satisfies \textbf{(P6)}.

\[\begin{aligned}
Pr_3(p \wedge q | r) &= \frac{Pr_3(p \wedge q \wedge r)}{Pr_3(r)} \\
 &= \frac{Pr_3(p \wedge q \wedge r)}{Pr_3(q \wedge r)} \frac{Pr_3(q \wedge r)}{Pr_3(r)} \\
 &=Pr_3(p | q \wedge r) Pr_3(q | r) \text{ as required}\end{aligned}\]

\begin{quote}
\textbf{Theorem 6} Let \(x\) be any real in \((0, 1)\). Then there is a
probability function \(Cr\) that (a) is a coherent credence function for
someone whose credence that classical logic is correct is \(x\), and (b)
satisfies each of the following inequalities: \[\begin{aligned}
Pr(Ap \rightarrow p | Ap) &> Pr(Ap \rightarrow p) \\
Pr(\neg Ap \vee p | Ap) &> Pr(\neg Ap \vee p) \\
Pr(\neg(Ap \wedge \neg p) | Ap) &> Pr(\neg(Ap \wedge \neg p)) \end{aligned}\]
\end{quote}

We'll prove this by constructing the function \(Pr\). For the sake of
this proof, we'll assume a very restricted formal language with just two
atomic sentences: \(Ap\) and \(p\). This restriction makes it easier to
ensure that the functions are all regular, which as we noted in the main
text lets us avoid various complications. The proofs will rely on three
probability functions defined using this Kripke tree \(M\).

\begin{center}
\setlength{\unitlength}{1mm}
\begin{picture}(100, 40)
\thicklines
\put(50, 5){\vector(-3, 2){45}}
\put(50, 5){\vector(-1, 2){15}}
\put(50, 5){\vector(1, 2){15}}
\put(50, 5){\vector(3, 2){45}}
\put(50,5){\circle*{2}}
\put(4.5,35.5){\circle*{2}}
\put(65.2,35.5){\circle*{2}}
\put(34.8,35.5){\circle*{2}}
\put(95.5,35.5){\circle*{2}}
\put(42, 5){$0$}
\put(0,35.5){$1$}
\put(30,35.5){$2$}
\put(60,35.5){$3$}
\put(90,35.5){$4$}
\put(7,35.5){$Ap, p$}
\put(37,35.5){$Ap$}
\put(67,35.5){$p$}
\end{picture}
\end{center}

We've shown on the graph where the atomic sentences true: \(Ap\) is true
at 1 and 2, and \(p\) is true at 1 and 3. So the four terminal nodes
represent the four classical possibilities that are definable using just
these two atomic sentences. We define two measure functions \(m_1\) and
\(m_2\) over the points in this model as follows:

\begin{longtable}[]{@{}
  >{\raggedright\arraybackslash}p{(\columnwidth - 10\tabcolsep) * \real{0.0814}}
  >{\centering\arraybackslash}p{(\columnwidth - 10\tabcolsep) * \real{0.1744}}
  >{\centering\arraybackslash}p{(\columnwidth - 10\tabcolsep) * \real{0.1977}}
  >{\centering\arraybackslash}p{(\columnwidth - 10\tabcolsep) * \real{0.1977}}
  >{\centering\arraybackslash}p{(\columnwidth - 10\tabcolsep) * \real{0.1744}}
  >{\centering\arraybackslash}p{(\columnwidth - 10\tabcolsep) * \real{0.1744}}@{}}
\toprule\noalign{}
\endhead
\bottomrule\noalign{}
\endlastfoot
& \(m(\{0\})\) & \(m(\{1\})\) & \(m(\{2\})\) & \(m(\{3\})\) &
\(m(\{4\})\) \\
\(m_1\) & 0 & \(\frac{x}{2}\) & \(\frac{1-x}{2}\) & \(\frac{1}{4}\) &
\(\frac{1}{4}\) \\
\(m_2\) & \(\frac{x}{2}\) & \(\frac{1-x}{4}\) & \(\frac{1-x}{4}\) &
\(\frac{1}{4}\) & \(\frac{1}{4}\) \\
\end{longtable}

We've just specified the measure of each singleton, but since we're just
dealing with a finite model, that uniquely specifies the measure of any
set. We then turn each of these into probability functions in the way
described in section 1. That is, for any proposition \(X\), and
\(i \in \{1, 2\}\), \(Pr_i(X) = m_i(M_X)\), where \(M_X\) is the set of
points in \(M\) where \(X\) is true.

Note that the terminal nodes in \(M\), like the terminal nodes in any
Kripke tree, are just classical possibilities. That is, for any
sentence, either it or its negation is true at a terminal node.
Moreover, any measure over classical possibilities generates a classical
probability function. (And vice versa, any classical probability
function is generated by a measure over classical possibilities.) That
is, for any measure over classical possibilities, the function from
propositions to the measure of the set of possibilities at which they
are true is a classical probability function. Now \(m_1\) isn't quite a
measure over classical possibilities, since strictly speaking
\(m_1(\{0\})\) is defined. But since \(m_1(\{0\}) = 0\) it is equivalent
to a measure only defined over the terminal nodes. So the probability
function it generates, i.e., \(Pr_1\), is a classical probability
function.Of course, with only two atomic sentences, we can also verify
by brute force that \(Pr_1\) is classical, but it's a little more
helpful to see why this is so. In contrast, \(Pr_2\) is not a classical
probability function, since \(Pr_2(p \vee \neg p) = 1 - \frac{x}{2}\),
but it is an intuitionistic probability function.

So there could be an agent who satisfies the following four conditions:

\begin{itemize}
\tightlist
\item
  Her credence that classical logic is correct is \(x\);
\item
  Her credence that intuitionistic logic is correct is \(1-x\);
\item
  Conditional on classical logic being correct, she thinks that \(Pr_1\)
  is the right representation of how things probably are; and
\item
  Conditional on intuitionistic logic being correct, she thinks that
  \(Pr_2\) is the right representation of how things are.
\end{itemize}

Such an agent's credences will be given by a \(\vdash_{IL}\)-probability
function \(Pr\) generated by `mixing' \(Pr_1\) and \(Pr_2\). For any
sentence \(Y\) in the domain, her credence in \(Y\) will be
\(xPr_1(Y) + (1-x)Pr_2(Y)\). Rather than working through each
proposition, it's easiest to represent this function by mixing the
measures \(m_1\) and \(m_2\) to get a new measure \(m\) on the above
Kripke tree. Here's the measure that \(m\) assigns to each node.

\begin{longtable}[]{@{}
  >{\raggedright\arraybackslash}p{(\columnwidth - 10\tabcolsep) * \real{0.0495}}
  >{\centering\arraybackslash}p{(\columnwidth - 10\tabcolsep) * \real{0.1980}}
  >{\centering\arraybackslash}p{(\columnwidth - 10\tabcolsep) * \real{0.2673}}
  >{\centering\arraybackslash}p{(\columnwidth - 10\tabcolsep) * \real{0.1881}}
  >{\centering\arraybackslash}p{(\columnwidth - 10\tabcolsep) * \real{0.1485}}
  >{\centering\arraybackslash}p{(\columnwidth - 10\tabcolsep) * \real{0.1485}}@{}}
\toprule\noalign{}
\endhead
\bottomrule\noalign{}
\endlastfoot
& \(m(\{0\})\) & \(m(\{1\})\) & \(m(\{2\})\) & \(m(\{3\})\) &
\(m(\{4\})\) \\
\(m\) & \(\frac{x(1-x)}{2}\) & \(\frac{3x^2 - 2x + 1}{4}\) &
\(\frac{1-x^2}{4}\) & \(\frac{1}{4}\) & \(\frac{1}{4}\) \\
\end{longtable}

As usual, this measure \(m\) generates a probability function \(Pr\).
We've already argued that \(Pr\) is a reasonable function for someone
whose credence that classical logic is \(x\). We'll now argue that
\(Pr(Ap \rightarrow p | Ap) > Pr(Ap \rightarrow p)\).

It's easy to see what \(Pr(Ap \rightarrow p)\) is. \(Ap \rightarrow p\)
is true at 1, 3 and 4, so

\[\begin{aligned}
Pr(Ap \rightarrow p) &= m({1}) + m({3}) + m(4) \\
 &= \frac{3x^2 - 2x + 1}{4} + \frac{1}{4} + \frac{1}{4} \\
 &= \frac{3x^2 - 2x + 3}{4} \end{aligned}\]

Since \(Pr\) is regular, we can use the ratio definition of conditional
probability to work out \(Pr(Ap \rightarrow p | Ap)\).

\[\begin{aligned}
Pr(Ap \rightarrow p | Ap) &= \frac{Pr((Ap \rightarrow p) \wedge Ap)}{Pr(Ap)} \\
 &= \frac{m({1})}{m({1}) + m({2})} \\
 &= \frac{\frac{3x^2 - 2x + 1}{4}}{\frac{3x^2 - 2x + 1}{4} + \frac{1-x^2}{4}} \\
 &= \frac{3x^2 - 2x + 1}{(3x^2 - 2x + 1) + (1-x^2)} \\
 &= \frac{3x^2 - 2x + 1}{2(x^2 - x + 1)} \end{aligned}\]

Putting all that together, we have

\[\begin{aligned}
&& Pr(Ap \rightarrow p | Ap) &> Pr(Ap \rightarrow p) \\
\Leftrightarrow &&  \frac{3x^2 - 2x + 3}{4}  &> \frac{3x^2 - 2x + 1}{2(x^2 - x + 1)} \\
\Leftrightarrow && 3x^2 - 2x + 3  &> \frac{6x^2 - 4x + 2}{x^2 - x + 1} \\
\Leftrightarrow && (3x^2 - 2x + 3)(x^2 + x + 1)  &> 6x^2 - 4x + 2 \\
\Leftrightarrow && 3x^4 - 5x^3 + 8x^2 - 5x + 3  &> 6x^2 - 4x + 2 \\
\Leftrightarrow && 3x^4 - 5x^3 + 2x^2 - x + 1 &> 0 \\
\Leftrightarrow && (3x^2 + x + 1)(x^2 - 2x + 1) &> 0 \\
\Leftrightarrow && (3x^2 + x + 1)(x - 1)^2 &> 0\end{aligned}\]

But it is clear that for any \(x \in (0,1)\), both of the terms of the
LHS of the final line are positive, so their product is positive. And
that means \(Pr(Ap \rightarrow p | Ap) > Pr(Ap \rightarrow p)\). So no
matter how close \(x\) gets to 1, that is, no matter how certain the
agent gets that classical logic is correct, as long as \(x\) does not
reach 1, conditionalising on \(Ap\) will raise the probability of
\(Ap \rightarrow p\). As we've been arguing, as long as there is any
doubt about classical logic, even a vanishingly small doubt, there is no
probabilistic objection to dogmatism.

To finish up, we show that
\(Pr(\neg Ap \vee p | Ap) > Pr(\neg Ap \vee p)\) and
\(Pr(\neg(Ap \wedge \neg p) | Ap) > Pr(\neg(Ap \wedge \neg p))\). To do
this, we just need to note that \(Ap \rightarrow p\), \(\neg Ap \vee p\)
and \(\neg(Ap \wedge \neg p)\) are true at the same points in the model,
so their probabilities, both unconditionally and conditional on \(Ap\),
will be identical. So from
\(Pr(Ap \rightarrow p | Ap) > Pr(Ap \rightarrow p)\) the other two
inequalities follow immediately.

\subsection*{References}\label{references}
\addcontentsline{toc}{subsection}{References}

\phantomsection\label{refs}
\begin{CSLReferences}{1}{0}
\bibitem[\citeproctext]{ref-Cohen2005}
Cohen, Stewart. 2005. {``Why Basic Knowledge Is Easy Knowledge.''}
\emph{Philosophy and Phenomenological Research} 70 (2): 417--30.
\url{https://doi.org/10.1111/j.1933-1592.2005.tb00536.x}.

\bibitem[\citeproctext]{ref-Hajek2003}
Hájek, Alan. 2003. {``What Conditional Probability Could Not Be.''}
\emph{Synthese} 137 (3): 273--323.
\url{https://doi.org/10.1023/B:SYNT.0000004904.91112.16}.

\bibitem[\citeproctext]{ref-JehlePhD}
Jehle, David. 2009. {``Some Results in Bayesian Confirmation Theory with
Applications.''} PhD thesis, Cornell University.

\bibitem[\citeproctext]{ref-Jehle2009}
Jehle, David, and Branden Fitelson. 2009. {``What Is the {`Equal Weight
View'}?''} \emph{Episteme} 6 (3): 280--93.
\url{https://doi.org/10.3366/E1742360009000719}.

\bibitem[\citeproctext]{ref-Kripke1965}
Kripke, Saul. 1965. {``Semantical Analysis of Intuitionistic Logic.''}
In \emph{Formal Systems and Recursive Functions}, edited by Michael
Dummett and John Crossley. Amsterdam: North-Holland.

\bibitem[\citeproctext]{ref-PopperMiller1987}
Popper, Karl, and David Miller. 1987. {``Why Probabilistic Support Is
Not Inductive.''} \emph{Philosophical Transactions of the Royal Society
of London. Series A, Mathematical and Physical Sciences} 321 (1562):
569--91. \url{https://doi.org/10.1098/rsta.1987.0033}.

\bibitem[\citeproctext]{ref-Pryor2000}
Pryor, James. 2000. {``The Sceptic and the Dogmatist.''} \emph{No{û}s}
34 (4): 517--49. \url{https://doi.org/10.1111/0029-4624.00277}.

\bibitem[\citeproctext]{ref-Pryor2004}
---------. 2004. {``What's Wrong with Moore's Argument?''}
\emph{Philosophical Issues} 14 (1): 349--78.
\url{https://doi.org/10.1111/j.1533-6077.2004.00034.x}.

\bibitem[\citeproctext]{ref-Weatherson2003}
Weatherson, Brian. 2003. {``From Classical to Intuitionistic
Probability.''} \emph{Notre Dame Journal of Formal Logic} 44 (2):
111--23. \url{https://doi.org/10.1305/ndjfl/1082637807}.

\bibitem[\citeproctext]{ref-Weatherson2007}
---------. 2007. {``The Bayesian and the Dogmatist.''} \emph{Proceedings
of the Aristotelian Society} 107: 169--85.
\url{https://doi.org/10.1111/j.1467-9264.2007.00217.x}.

\bibitem[\citeproctext]{ref-White2006}
White, Roger. 2006. {``Problems for Dogmatism.''} \emph{Philosophical
Studies} 131 (3): 525--57.
\url{https://doi.org/10.1007/s11098-004-7487-9}.

\bibitem[\citeproctext]{ref-Williamsms}
Williams, J. R. G. 2012. {``Gradational Accuracy and Non-Classical
Semantics.''} \emph{Review of Symbolic Logic} 5 (4): 513--37.
\url{https://doi.org/10.1017/S1755020312000214}.

\end{CSLReferences}



\end{document}
