% Options for packages loaded elsewhere
% Options for packages loaded elsewhere
\PassOptionsToPackage{unicode}{hyperref}
\PassOptionsToPackage{hyphens}{url}
%
\documentclass[
  11pt,
  letterpaper,
  DIV=11,
  numbers=noendperiod,
  twoside]{scrartcl}
\usepackage{xcolor}
\usepackage[left=1.1in, right=1in, top=0.8in, bottom=0.8in,
paperheight=9.5in, paperwidth=7in, includemp=TRUE, marginparwidth=0in,
marginparsep=0in]{geometry}
\usepackage{amsmath,amssymb}
\setcounter{secnumdepth}{3}
\usepackage{iftex}
\ifPDFTeX
  \usepackage[T1]{fontenc}
  \usepackage[utf8]{inputenc}
  \usepackage{textcomp} % provide euro and other symbols
\else % if luatex or xetex
  \usepackage{unicode-math} % this also loads fontspec
  \defaultfontfeatures{Scale=MatchLowercase}
  \defaultfontfeatures[\rmfamily]{Ligatures=TeX,Scale=1}
\fi
\usepackage{lmodern}
\ifPDFTeX\else
  % xetex/luatex font selection
  \setmainfont[ItalicFont=EB Garamond Italic,BoldFont=EB Garamond
SemiBold]{EB Garamond Math}
  \setsansfont[]{EB Garamond}
  \setmathfont[]{Garamond-Math}
\fi
% Use upquote if available, for straight quotes in verbatim environments
\IfFileExists{upquote.sty}{\usepackage{upquote}}{}
\IfFileExists{microtype.sty}{% use microtype if available
  \usepackage[]{microtype}
  \UseMicrotypeSet[protrusion]{basicmath} % disable protrusion for tt fonts
}{}
\usepackage{setspace}
% Make \paragraph and \subparagraph free-standing
\makeatletter
\ifx\paragraph\undefined\else
  \let\oldparagraph\paragraph
  \renewcommand{\paragraph}{
    \@ifstar
      \xxxParagraphStar
      \xxxParagraphNoStar
  }
  \newcommand{\xxxParagraphStar}[1]{\oldparagraph*{#1}\mbox{}}
  \newcommand{\xxxParagraphNoStar}[1]{\oldparagraph{#1}\mbox{}}
\fi
\ifx\subparagraph\undefined\else
  \let\oldsubparagraph\subparagraph
  \renewcommand{\subparagraph}{
    \@ifstar
      \xxxSubParagraphStar
      \xxxSubParagraphNoStar
  }
  \newcommand{\xxxSubParagraphStar}[1]{\oldsubparagraph*{#1}\mbox{}}
  \newcommand{\xxxSubParagraphNoStar}[1]{\oldsubparagraph{#1}\mbox{}}
\fi
\makeatother


\usepackage{longtable,booktabs,array}
\usepackage{calc} % for calculating minipage widths
% Correct order of tables after \paragraph or \subparagraph
\usepackage{etoolbox}
\makeatletter
\patchcmd\longtable{\par}{\if@noskipsec\mbox{}\fi\par}{}{}
\makeatother
% Allow footnotes in longtable head/foot
\IfFileExists{footnotehyper.sty}{\usepackage{footnotehyper}}{\usepackage{footnote}}
\makesavenoteenv{longtable}
\usepackage{graphicx}
\makeatletter
\newsavebox\pandoc@box
\newcommand*\pandocbounded[1]{% scales image to fit in text height/width
  \sbox\pandoc@box{#1}%
  \Gscale@div\@tempa{\textheight}{\dimexpr\ht\pandoc@box+\dp\pandoc@box\relax}%
  \Gscale@div\@tempb{\linewidth}{\wd\pandoc@box}%
  \ifdim\@tempb\p@<\@tempa\p@\let\@tempa\@tempb\fi% select the smaller of both
  \ifdim\@tempa\p@<\p@\scalebox{\@tempa}{\usebox\pandoc@box}%
  \else\usebox{\pandoc@box}%
  \fi%
}
% Set default figure placement to htbp
\def\fps@figure{htbp}
\makeatother


% definitions for citeproc citations
\NewDocumentCommand\citeproctext{}{}
\NewDocumentCommand\citeproc{mm}{%
  \begingroup\def\citeproctext{#2}\cite{#1}\endgroup}
\makeatletter
 % allow citations to break across lines
 \let\@cite@ofmt\@firstofone
 % avoid brackets around text for \cite:
 \def\@biblabel#1{}
 \def\@cite#1#2{{#1\if@tempswa , #2\fi}}
\makeatother
\newlength{\cslhangindent}
\setlength{\cslhangindent}{1.5em}
\newlength{\csllabelwidth}
\setlength{\csllabelwidth}{3em}
\newenvironment{CSLReferences}[2] % #1 hanging-indent, #2 entry-spacing
 {\begin{list}{}{%
  \setlength{\itemindent}{0pt}
  \setlength{\leftmargin}{0pt}
  \setlength{\parsep}{0pt}
  % turn on hanging indent if param 1 is 1
  \ifodd #1
   \setlength{\leftmargin}{\cslhangindent}
   \setlength{\itemindent}{-1\cslhangindent}
  \fi
  % set entry spacing
  \setlength{\itemsep}{#2\baselineskip}}}
 {\end{list}}
\usepackage{calc}
\newcommand{\CSLBlock}[1]{\hfill\break\parbox[t]{\linewidth}{\strut\ignorespaces#1\strut}}
\newcommand{\CSLLeftMargin}[1]{\parbox[t]{\csllabelwidth}{\strut#1\strut}}
\newcommand{\CSLRightInline}[1]{\parbox[t]{\linewidth - \csllabelwidth}{\strut#1\strut}}
\newcommand{\CSLIndent}[1]{\hspace{\cslhangindent}#1}



\setlength{\emergencystretch}{3em} % prevent overfull lines

\providecommand{\tightlist}{%
  \setlength{\itemsep}{0pt}\setlength{\parskip}{0pt}}



 


\setlength\heavyrulewidth{0ex}
\setlength\lightrulewidth{0ex}
\usepackage[automark]{scrlayer-scrpage}
\clearpairofpagestyles
\cehead{
  Brian Weatherson
  }
\cohead{
  Indicative and Subjunctive Conditionals
  }
\ohead{\bfseries \pagemark}
\cfoot{}
\makeatletter
\newcommand*\NoIndentAfterEnv[1]{%
  \AfterEndEnvironment{#1}{\par\@afterindentfalse\@afterheading}}
\makeatother
\NoIndentAfterEnv{itemize}
\NoIndentAfterEnv{enumerate}
\NoIndentAfterEnv{description}
\NoIndentAfterEnv{quote}
\NoIndentAfterEnv{equation}
\NoIndentAfterEnv{longtable}
\NoIndentAfterEnv{abstract}
\renewenvironment{abstract}
 {\vspace{-1.25cm}
 \quotation\small\noindent\emph{Abstract}:}
 {\endquotation}
\newfontfamily\tfont{EB Garamond}
\addtokomafont{disposition}{\rmfamily}
\addtokomafont{title}{\normalfont\itshape}
\let\footnoterule\relax

\makeatletter
\renewcommand{\@maketitle}{%
  \newpage
  \null
  \vskip 2em%
  \begin{center}%
  \let \footnote \thanks
    {\itshape\huge\@title \par}%
    \vskip 0.5em%  % Reduced from default
    {\large
      \lineskip 0.3em%  % Reduced from default 0.5em
      \begin{tabular}[t]{c}%
        \@author
      \end{tabular}\par}%
    \vskip 0.5em%  % Reduced from default
    {\large \@date}%
  \end{center}%
  \par
  }
\makeatother
\RequirePackage{lettrine}

\renewenvironment{abstract}
 {\quotation\small\noindent\emph{Abstract}:}
 {\endquotation\vspace{-0.02cm}}
\KOMAoption{captions}{tableheading}
\makeatletter
\@ifpackageloaded{caption}{}{\usepackage{caption}}
\AtBeginDocument{%
\ifdefined\contentsname
  \renewcommand*\contentsname{Table of contents}
\else
  \newcommand\contentsname{Table of contents}
\fi
\ifdefined\listfigurename
  \renewcommand*\listfigurename{List of Figures}
\else
  \newcommand\listfigurename{List of Figures}
\fi
\ifdefined\listtablename
  \renewcommand*\listtablename{List of Tables}
\else
  \newcommand\listtablename{List of Tables}
\fi
\ifdefined\figurename
  \renewcommand*\figurename{Figure}
\else
  \newcommand\figurename{Figure}
\fi
\ifdefined\tablename
  \renewcommand*\tablename{Table}
\else
  \newcommand\tablename{Table}
\fi
}
\@ifpackageloaded{float}{}{\usepackage{float}}
\floatstyle{ruled}
\@ifundefined{c@chapter}{\newfloat{codelisting}{h}{lop}}{\newfloat{codelisting}{h}{lop}[chapter]}
\floatname{codelisting}{Listing}
\newcommand*\listoflistings{\listof{codelisting}{List of Listings}}
\makeatother
\makeatletter
\makeatother
\makeatletter
\@ifpackageloaded{caption}{}{\usepackage{caption}}
\@ifpackageloaded{subcaption}{}{\usepackage{subcaption}}
\makeatother
\usepackage{bookmark}
\IfFileExists{xurl.sty}{\usepackage{xurl}}{} % add URL line breaks if available
\urlstyle{same}
\hypersetup{
  pdftitle={Indicative and Subjunctive Conditionals},
  pdfauthor={Brian Weatherson},
  hidelinks,
  pdfcreator={LaTeX via pandoc}}


\title{Indicative and Subjunctive Conditionals}
\author{Brian Weatherson}
\date{2001}
\begin{document}
\maketitle
\begin{abstract}
In any plausible semantics for conditionals, the semantics for
indicatives and subjunctives will resemble each other closely. This
means that if we are to keep the possible‐worlds semantics for
subjunctives suggested by Lewis, we need to find a possible‐worlds
semantics for indicatives. One reason for thinking that this will be
impossible is the behaviour of rigid designators in indicatives. An
indicative like `If the stuff in the rivers, lakes and oceans really is
H\textsubscript{3}O, then water is H\textsubscript{3}O' is non‐vacuously
true, even though its consequent is true in no possible worlds, and
hence not in the nearest possible world where the antecedent is true. I
solve this difficulty by providing a semantics for conditionals within
the framework of two‐dimensional modal logic. In doing so, I show that
we can have a reasonably unified semantics for indicative and
subjunctive conditionals.
\end{abstract}


\setstretch{1.1}
This paper presents a new theory of the truth conditions for indicative
conditionals. The theory allows us to give a fairly unified account of
the semantics for indicative and subjunctive conditionals, though there
remains a distinction between the two classes. Put simply, the idea
behind the theory is that the distinction between the indicative and the
subjunctive parallels the distinction between the necessary and the
\emph{a priori}. Since that distinction is best understood formally
using the resources of two-dimensional modal logic, those resources will
be brought to bear on the logic of conditionals.

\section{A Grand Unified Theory?}\label{a-grand-unified-theory}

Our primary focus is the indicative conditional `If \emph{A}, \emph{B}',
written as \emph{A}~→~\emph{B}. Most theorists fail to distinguish
between this conditional and `If \emph{A}, then \emph{B}', and for the
most part I will follow this tradition. The most notable philosophical
exception is Grice, who suggested that only the latter says that
\emph{B} follows from \emph{A} in some relevant way
(\citeproc{ref-Grice1989}{1989, 63}). Theorists do distinguish between
this conditional and the subjunctive `If it were the case that \emph{A},
it would be the case that \emph{B}', written as \emph{A}~□→~\emph{B}.
There is some debate about precisely where to draw the line between
these two classes, which I'll discuss in section three, but for now I'll
focus on cases far from the borderline. One important tradition in work
on conditionals holds that the semantics of indicatives differs
radically from the semantics of subjunctives. According to David Lewis
(\citeproc{ref-Lewis1973a}{1973}, \citeproc{ref-Lewis1976b}{1976}) and
Frank Jackson (\citeproc{ref-Jackson1987}{1987}) for example,
indicatives are truth-functional, but subjunctives are not. This makes a
mystery of some of the data. For example, as Jackson himself writes:

\begin{quote}
Before the last presidential election commentators said `If Reagan
loses, the opinion polls will be totally discredited', afterwards they
said `If Reagan had lost, the opinion polls would have been totally
discredited', and this switch from indicative to subjunctive
counterfactual did \emph{not} count as a change of mind
(\citeproc{ref-Jackson1987}{Jackson 1987, 66}).
\end{quote}

The point can be pushed further. To communicate the commentators'
pre-election opinions using indirect speech we would say something like
(1).

\begin{description}
\tightlist
\item[(1)]
Commentators have said that if Reagan were to lose the opinion polls
would be totally discredited.
\end{description}

Yet it is possible on Jackson's view that what the commentators said was
\emph{true}, since Reagan won, yet the words after `that' in (1) form a
\emph{false} sentence. So we can accurately report someone speaking
truly by using a false sentence. Jackson's response plays on the
connections between \emph{A}~→~\emph{B} and the disjunction
`Not-\emph{A} or \emph{B}'. That disjunction has undeniably different
truth conditions to \emph{A}~□→~\emph{B}. Pushing the truth conditions
of \emph{A}~→~\emph{B} closer to those of \emph{A}~□→~\emph{B} will move
them away from `Not-\emph{A}~or~\emph{B}'. One gain in similarity and
theoretical simplicity is bought at the cost of another. Jackson's
account, by making \emph{A}~→~\emph{B} have similar truth conditions to
`Not-\emph{A}~or \emph{B}' but similar assertibility conditions to
\emph{A}~□→~\emph{B}, tries to have the best of both worlds. How great
the similarity between indicative conditionals and disjunctions really
is, and hence how great the cost of linking indicatives and
subjunctives, might well be questioned. After all, we don't report an
utterance of an indicative using a disjunction.

Two types of cases seem to threaten the success of a unified theory.
First, rigidifying expressions like `actually' behave differently in
indicatives and subjunctives. Secondly, some conditionals differ in
intuitive truth value when we transpose them from the indicative to the
subjunctive. The most famous examples of this phenomenon involve various
presidential assassinations. The effects of rigidity on conditionals are
less explored, so we will first look at that. Consider the following
example, from page 55 of \emph{Naming and Necessity}.

\begin{description}
\tightlist
\item[(2)]
If heat had been applied to this stick \emph{S} at
\emph{t}\textsubscript{0}, then at \emph{t}\textsubscript{0} stick
\emph{S} would not have been one meter long.
\end{description}

The background is that we have stipulated that a metre is the length of
stick \emph{S} at time \emph{t}\textsubscript{0}. (2) contrasts with
(3), which seems false.

\begin{description}
\tightlist
\item[(3)]
If heat was applied to this stick \emph{S} at \emph{t}\textsubscript{0},
then at \emph{t}\textsubscript{0} stick \emph{S} was not one meter long.
\end{description}

If we have stipulated that to be a meter long is to be the length of
\emph{S} at \emph{t}\textsubscript{0}, then whatever conditions \emph{S}
was under at \emph{t}\textsubscript{0}, it was one meter long. As
Jackson points out, we can get the same effect with explicit rigidifiers
like `actually'. We could, somewhat wistfully, say (4). It may even be
true. But (5) seems barely coherent, and certainly not something we
could ever say.

\begin{description}
\tightlist
\item[(4)]
If Hillary Clinton were to become the next U.S. President, things would
be different from the way they actually will be.
\item[(5)]
If Hillary Clinton becomes the next U.S. President, things will be
different from the way they actually will be.
\end{description}

It looks like any theory of conditionals will have to account for a
difference between the behaviour of rigid designators in indicatives and
subjunctives. We may avoid the conclusion by showing that the difference
only appears in certain types of conditionals, and we already have an
explanation for those cases. For example, it is well known that usually
one cannot say \emph{A}~→~\emph{B} if it is known that not-\emph{A}. As
Dudman (\citeproc{ref-Dudman1994}{1994}) points out, (6) is clearly
infelicitous on its most obvious reading.

\begin{description}
\tightlist
\item[(6)]
*Granny won, but if she lost she was furious.
\end{description}

To complete the diagnosis, note that the most striking examples of the
different behaviour of rigid designators in different types of
conditionals comes up in cases where the antecedent is almost certainly
false. The effect is that the subjunctive can be asserted, but not the
indicative. So this phenomenon may be explainable by some other part of
the theory of conditionals.\footnote{An anonymous reviewer for
  \emph{Philosophical Quarterly} suggested this point.} These are the
most striking exemplars of the difference I am highlighting, but not the
only examples. Hence, this point cannot explain all the data, though it
may explain why pairs like (2)/(3) and (4)/(5) are striking. For
instance, in the following pairs, the indicative seems appropriate and
intuitively true, and the subjunctive seems inappropriate and
intuitively false.

\begin{description}
\tightlist
\item[(7)]
If C-fibres firing is what causes pain sensations, then C-fibres firing
is what actually causes pain sensations.
\item[(8)]
If C-fibres firing were what caused pain sensations, then C-fibres
firing would be what actually causes pain sensations.
\item[(9)]
If the stuff that plays the gold role has atomic number 42, then gold
has atomic number 42.
\item[(10)]
If the stuff that played the gold role had atomic number 42, gold would
have atomic number 42.
\end{description}

In (9) and (10) I assume that to play the gold role one must play it
throughout a large part of the world, and not just on a small stage.
Something may play the gold role in a small part of the world without
being gold. Since there are pairs of conditionals like these where the
indicative is appropriate, but the subjunctive is not, the explanation
of the behaviour of rigid terms cannot rely on the fact that the
antecedents of indicatives must be not known to be false. We will also
need a more traditional example of the differences between indicatives
and subjunctives, as in (11) and (12).

\begin{description}
\tightlist
\item[(11)]
If Hinckley didn't shoot Reagan, someone else did.
\item[(12)]
If Hinckley hadn't shot Reagan, someone else would have.
\end{description}

I have concentrated on the examples involving rigidity because they seem
to pose a deeper problem for unifying the theory of conditionals than
the presidential examples. As Jackson (\citeproc{ref-Jackson1987}{1987,
75}) points out, one can presumably explain (11) and (12) on a possible
worlds account by varying the similarity metric between indicatives and
subjunctives, or on a probabilistic account by varying the background
evidence. It is unclear, however, how this will help with the rigidity
examples. Assume, for example, that C-fibres firing is not what causes
pain sensations. Still, (7) seems true, but its consequent is false in
all possible worlds. Therefore, the nearest world in which its
antecedent is true is a world in which its consequent is false, and on a
simple possible worlds theory it should turn out false. On a simple
probabilistic account, the probability that C-fibres firing actually
cause pain sensations given that they do is 1, whatever the background
evidence, so (8) should turn out true, contrary to our intuitions. So
while the details deal with the presidential examples, the structure of
the theory must deal with the rigidity examples.

I will follow that strategy here. In section two I set out the framework
of a unified possible worlds account of indicatives and subjunctives. In
section three I present my preferred way of filling out the details of
that framework. The framework deals with the differing behaviour of
rigid designators in indicatives and subjunctives; the details deal with
examples like (11) and (12). One reason for dividing the presentation in
this way is to highlight the option of accepting the framework and
filling in the details in different ways.

\section{The New Theory}\label{the-new-theory}

\subsection{Actually}\label{actually}

As Kripke (\citeproc{ref-Kripke1980}{{[}1972{]} 1980}) showed, the
reference for some terms is fixed by what plays a particular role in the
actual world. Even if it were the case that XYZ fills the ocean, falls
from the sky, is drinkable and transparent and so on, for short is
\emph{watery}, it would still be the case that water is
H\textsubscript{2}O, not XYZ. For it would still be that
H\textsubscript{2}O \emph{actually} is watery. Whatever were the case,
this world would be actual.

Yet, we want to have a way to talk about what would have happened had
some other world been actual. In particular, had the actual world been
one in which XYZ is watery, it would be true, indeed necessarily true,
that water is XYZ. Throughout the 1970s a number of methods for doing
this were produced. The following presentation is indebted to Davies and
Humberstone (\citeproc{ref-Davies1980}{1980}), but other approaches
might have been used. The notation \(\vDash_y^x\)\emph{A} is interpreted
as `\emph{A} is true in world \emph{y} from the perspective of world
\emph{x} as actual'. So, letting @ be the actual world and \emph{W} be a
world in which only XYZ is watery, we can represent what was said
informally above as follows.

\begin{quote}
\(\vDash_@^@\) H\textsubscript{2}O is watery and H\textsubscript{2}O is
water.

\(\vDash_w^@\) XYZ is watery and H\textsubscript{2}O is water.

\(\vDash_@^w\) H\textsubscript{2}O is watery and XYZ is water.

\(\vDash_w^w\)XYZ is watery and XYZ is water.
\end{quote}

Now as Kripke noted, it is necessary but \emph{a posteriori} that water
is H\textsubscript{2}O. Conversely, it is \emph{a priori} but contingent
that water is watery. This is \emph{a priori} because we knew before we
determined what water really is that it would be whatever plays the
watery role in this world, the actual world. In general \emph{A} is
necessary iff, given this is the actual world, it is true in all worlds.
And \emph{A} is \emph{a priori} iff, whatever the actual world turns out
to be like, it makes \emph{A} true. So we get the following definitions.

\begin{quote}
\emph{A} is \emph{a priori} iff for all worlds \emph{W}, \(\vDash_w^w\)
\emph{A}.

\emph{A} is necessary iff for all worlds \emph{W}, \(\vDash_w^@\)
\emph{A}.
\end{quote}

The connection between actuality and the \emph{a priori} is important.
It is \emph{a priori} that we are in the actual world. Something is
\emph{a priori} iff it is true whenever the two indices are the same. If
we regard possible worlds as sets of sentences, we can think of the sets
\{\emph{A}:~\(\vDash_x^x\)~\emph{A}\} for each possible world \emph{x}
as the epistemically possible worlds. Note that I don't make the set of
epistemically possible worlds relative to an evidence set, as others
commonly do. Rather they are just the sets of sentences consistent with
what we know \emph{a priori}. More accurately, identify a world pair
⟨\emph{x},~\emph{y}⟩ with the set of
\{\emph{A}:~\(\vDash_y^x\)~\emph{A}\}. Then ⟨\emph{x},~\emph{y}⟩ is an
epistemically possible world pair iff \emph{x}~=~\emph{y}.

To finish this formal excursion, we note the definition of `Actually
\emph{A}'. Given what has been said so far, this needs no explanation.

\begin{quote}
\(\vDash_y^x\)Actually \emph{A} iff \(\vDash_x^x\) \emph{A}.
\end{quote}

\subsection{The Analysis of
Indicatives}\label{the-analysis-of-indicatives}

Now we have the resources for my theory of the truth conditions for
indicatives. I also give the parallel truth condition for subjunctives
to show the similarities.

\begin{quote}
\(\vDash_@^@\)\emph{A}~→~\emph{B} iff the nearest possible world
\emph{x} that \(\vDash_x^x\)\emph{A} is such that
\(\vDash_x^x\)\emph{B}.

\(\vDash_@^@\)\emph{A}~□→~\emph{B} iff the nearest possible world
\emph{x} that \(\vDash_x^@\)\emph{A} is such that
\(\vDash_x^@\)\emph{B}.
\end{quote}

These only cover the special case of what is true here from the
perspective of this world as actual. We can partially generalise the
analysis of indicatives in one dimension as follows.

\begin{quote}
\(\vDash_w^w\)\emph{A}~→~\emph{B} iff the nearest possible world
\emph{x} to \emph{W} such that \(\vDash_x^x\)\emph{A} is such that
\(\vDash_x^x\)\emph{B}.
\end{quote}

I will make some comments below about how we might fully generalise the
analysis, but for now, I want to focus on these simpler cases. Note that
straight away this makes \emph{A}~→~Actually \emph{A} come out true, by
the definition of `Actually'. If we allow ourselves quantification over
propositions, we can give an analysis of `things are different from the
way they actually are', as follows:

\begin{quote}
(\(\vDash_y^x\) Things are different from the way they actually are)
iff\\
(∃\emph{p}: \(\vDash_y^x\) \emph{p} and not \(\vDash_x^x\) \emph{p})
\end{quote}

Since nothing both is and is not the case in \emph{x} from the
perspective of \emph{x} as actual, this can never be true when \emph{y}
is \emph{x}. This explains why it can never serve as the consequent of
an indicative conditional.

\subsection{Motivations}\label{motivations}

The theory outlined here is reasonably unified, and accounts for the
rigidity phenomena, but without any further justification, the resort to
two-dimensional modal logic is \emph{ad hoc}. This subsection responds
to that problem with some independent motivations for the theory. In
particular I argue that this theory best captures the well-known
epistemic feel of the indicative conditional.

Ever since Ramsey (\citeproc{ref-Ramsey1929}{1929/1990}) most theorists
have held that there is an epistemic element to indicatives. Here is
Ramsey's sketch of an analysis of indicatives.

\begin{quote}
If two people are arguing `If \emph{p} will \emph{q}?' and are both in
doubt as to \emph{p}, they are adding \emph{p} hypothetically to their
stock of knowledge and arguing on that basis about \emph{q}; so that in
a sense `If \emph{p}, \emph{q}' and `If \emph{p}, ¬\emph{q}' are
contradictories (\citeproc{ref-Ramsey1929}{Ramsey 1929/1990, 247n}).
\end{quote}

Nothing of the sort could be true about subjunctives. What is in our
`stock of knowledge', or the contextually relevant knowledge, makes at
most an indirect contribution to the truth- value of a subjunctive. It
makes an indirect contribution because the common knowledge might affect
the context, which in turn determines the similarity measure. But given
a context, a subjunctive makes a broadly metaphysical claim, an
indicative a broadly epistemic claim. Hence, the relationship between
the indicative and subjunctive should parallel the relationship between
the necessary and the \emph{a priori}. As should be clear, this is
exactly what happens on this theory.

The close similarity between the indicative/subjunctive distinction and
the \emph{a priori}/necessary distinction can be demonstrated in other
ways. For example, corresponding to the contingent \emph{a priori} (13)
the indicative (14) is true, but the subjunctive (15) is false. And
corresponding to the necessary \emph{a posteriori} (16) the subjunctive
(17) is true but the indicative (18) is false. (I am assuming that it is
part of the definitions of the water role and the fire role that nothing
can play both roles.)

\begin{description}
\tightlist
\item[(13)]
Water is what plays the water role.
\item[(14)]
If XYZ plays the water role, XYZ is water.
\item[(15)]
If XYZ played the water role, it would be water.
\item[(16)]
Water is H\textsubscript{2}O.
\item[(17)]
If all H\textsubscript{2}O played the fire role, all water would be
fire.
\item[(18)]
If all H\textsubscript{2}O plays the fire role, all water is fire.
\end{description}

This suggests the analysis sketched here is not \emph{ad hoc} at all,
but follows naturally from considerations about the necessary and
\emph{a priori}. These sketchy considerations might not provide much
positive support for my theory. The main evidence for the theory,
however, is the way it manages the hard cases, particularly cases
involving rigid designation. What these considerations show is that the
correct theory of indicatives may invoke the resources of
two-dimensional modal logic without automatically renouncing any claim
to systematicity.

\section{The Details}\label{the-details}

In this section, I want to look at four questions. First, what can we
say about the similarity measure at the core of this account? Secondly,
how should we generalise the theory to cover cases where the definite
description in the analysis appears to denote nothing? Thirdly, how
should we generalise the theory to cover cases where the two indices
differ? Finally, how should we draw the line between indicatives and
subjunctives? If what I said in the previous section is correct, there
should be something to say about each of these questions, and what is
said should be motivated. While it is not important that what I say here
is precisely true, I do hope that it is.

\subsection{Nearness}\label{nearness}

Ideally, we could use exactly the same similarity metric for both
indicatives and subjunctives. The existence of pairs like (11) and (12)
suggests this is impossible. So we must come up with a pair of measures
on the worlds satisfying three constraints. First, the measure for
subjunctives must deliver plausible verdicts for most subjunctive
conditionals. Secondly, the measure for indicatives must deliver
plausible verdicts for most indicative conditionals. Thirdly, the
measures must be similar enough that we can explain the close
relationship between indicatives and subjunctives set out in section
one. The theory of section two requires that these objectives be jointly
satisfiable. I will attempt to demonstrate that they are by outlining a
pair of measures satisfying all three.

Lewis (\citeproc{ref-Lewis1979c}{1979a}) provides the measure for
subjunctives. He suggests the following four rules for locating the
nearest possible world in which \emph{A} is true.

\begin{enumerate}
\def\labelenumi{\arabic{enumi}.}
\tightlist
\item
  It is of the first importance to avoid big, widespread, diverse
  violations of law.
\item
  It is of the second importance to maximise the spatio-temporal region
  throughout which perfect match of particular fact prevails.
\item
  It is of the third importance to avoid even small, localized, simple
  violations of law.
\item
  It is of little or no importance to secure approximate similarity of
  particular fact, even in matters that concern us greatly.
  (\citeproc{ref-Lewis1979c}{Lewis 1979a, 47--48})
\end{enumerate}

The right measure for indicatives is somewhat simpler. Notice that
whenever we know that \emph{A}~⊃~\emph{B} and don't know whether
\emph{A}, \emph{A}~→~\emph{B} seems true. More generally, if I know some
sentence \emph{S} such that \emph{A} and \emph{S} together entail
\emph{B}, and I would continue to know \emph{S} even were I to come to
doubt \emph{B}, then \emph{A}~→~\emph{B} will seem true to me. No matter
how good a card cheat I know Sly Pete to be, if I know that he has the
worse hand, and that whenever someone with the worse hand calls they
lose, it will seem true to me that \emph{If Sly Pete calls, he will
lose}. Further, if someone else knows these background facts and tells
me that \emph{If Sly Pete calls, he will lose}, she speaks truthfully.

This data suggests that whenever there is a true \emph{S} such that
\emph{A} and \emph{S} entail \emph{B}, \emph{A}~→~\emph{B} is true. But
this would mean \emph{A}~→~\emph{B} is true whenever \emph{A}~⊃~\emph{B}
is true, which seems incredible. On this theory it is true that \emph{If
there is a nuclear war tomorrow, life will go on as normal}. There are
some very subtle attempts to make this palatable. The `Supplemented
Equivalence Theory' in Jackson (\citeproc{ref-Jackson1987}{1987}) may
even be successful. But two problems remain for all theories saying
\emph{A}~→~\emph{B} has the truth value of \emph{A}~⊃~\emph{B}. First,
they make some apparently true negated conditionals turn out false, such
as \emph{It is not true that if there is a nuclear war tomorrow, life
will go on as normal}. It is hard to see how an appeal to Gricean
pragmatics will avoid this problem. Secondly, such theories fail the
third task we set ourselves at the start of the section: explaining the
close connections between indicatives and subjunctives.

So we might be tempted to try a different path. Let's take the data at
face value and say that \emph{A}~→~\emph{B} is true in a context if
there is some \emph{S} such that some person in that context knows
\emph{S}, and \emph{A} and \emph{S} together entail \emph{B}. We can
formalise this claim as follows. Let \emph{d}(\emph{x}, \emph{y}) be the
`distance' from \emph{x} to \emph{y}. This function will satisfy few of
the formal properties of a distance relationship, so remember this is
just an analogy. Let \textbf{K} be the set of all propositions \emph{S}
known by someone in the context, \emph{W} the set of all possible
worlds, and \emph{i} the impossible world, where everything is true.
Then \emph{d}:~\emph{W}~⨉~\emph{W}~∪~\emph{i}~→~R is as follows:

\begin{quote}
If \emph{y} = \emph{x} then \emph{d}(\emph{x}, \emph{y}) = 0

If \emph{y} ∈ \emph{W}, \emph{y} ≠ \emph{x} and
∀\emph{S}:~\emph{S}~∈~\textbf{K}~⊃~\(\vDash_y^y\) ∀\emph{S}: \emph{S} ∈
\textbf{K} ⊃ \(\vDash_y^y\) \emph{S}, then \emph{d}(\emph{x}, \emph{y})
= 1

If \emph{y}~= \emph{i} then \emph{d}(\emph{x}, \emph{y}) = 2

Otherwise, \emph{d}(\emph{x}, \emph{y}) = 3
\end{quote}

Less formally, the nearest world to a world is itself. The next closest
worlds are any compatible with everything known in the context, then the
impossible world, then the possible worlds incompatible with something
known in the context. It may seem odd to have the impossible world
closer than some possible worlds, but there are two reasons for doing
this. First, in the impossible world everything known to any
conversational participant is true. Secondly, putting the impossible
world at this position accounts for some examples. This is a variant on
a well known case; see for example Gibbard
(\citeproc{ref-Gibbard1981}{1981}) and Barker
(\citeproc{ref-Barker1997}{1997}).

Jack and Jill are trying to find out how their local representative Kim,
a Democrat from Texas, voted on a resolution at a particular committee
meeting. So far, they have not even found out whether Kim was at the
meeting. Jack finds out that all Democrats at the meeting voted against
the resolution; Jill finds out that all Texans at the meeting voted for
it. When they return to compare notes, Jack can truly say \emph{If Kim
was at the meeting, she voted against the resolution}, and Jill can
truly say \emph{If Kim was at the meeting, she voted for the
resolution}. If \emph{i} is further from the actual world than some
possible world where Kim attended the meeting, these statements cannot
both be true.

It may be thought the distance function needs to be more fine-grained to
account for the following phenomena\footnote{Lewis
  (\citeproc{ref-Lewis1973a}{1973}) makes this objection to a similar
  proposal for subjunctives; the objection has just as much force here
  as it does in the original case.}. It seems possible that in each of
the following pairs, the first sentence is true and the second false.

\begin{description}
\tightlist
\item[(19)]
\hfill
\begin{enumerate}
\def\labelenumi{\alph{enumi}.}
\tightlist
\item
  If Anne goes to the party, so will Billy.
\item
  If Anne goes to the party, Billy will not go.
\end{enumerate}
\item[(20)]
\hfill
\begin{enumerate}
\def\labelenumi{\alph{enumi}.}
\tightlist
\item
  If Anne and Carly go to the party, Billy will not go.
\item
  If Anne and Carly go to the party, so will Billy.
\end{enumerate}
\item[(21)]
\hfill
\begin{enumerate}
\def\labelenumi{\alph{enumi}.}
\tightlist
\item
  If Anne, Carly and Donna go to the party, so will Billy.
\item
  If Anne, Carly and Donna go to the party, Billy will not.
\end{enumerate}
\end{description}

Assume, as seems plausible, it is necessary and sufficient for
\emph{A}~→~\emph{B} to be true that the nearest \emph{A}~∧~\emph{B}
world is closer than the nearest \emph{A}~∧~¬\emph{B} world. (This does
not immediately follow from the analysis in section 2, but is obviously
compatible with it.) Given this, there is no context in which the first
conditional in each pair is true, and the second false. McCawley
(\citeproc{ref-McCawley1996}{1996}) points out a way to accommodate
these intuitions. Every time a conditional is uttered, or considered in
a private context, the context shifts so as to accommodate the
possibility that its antecedent is true. So at first we don't consider
worlds where Carly or Donna turn up, and agree that (19a) is true and
(19b) false because in those worlds Billy loyally follows Anne to the
party. When (20a) or (20b) is uttered, or considered, we have to allow
some worlds where Carly goes to the party into the context set. In some
of these worlds Anne goes to the party and Billy doesn't, the worlds
where Carly goes to party. A similar story explains how (21a) can be
true despite (20b) being false.\footnote{There is an obvious similarity
  between this argument and some of the uses of contextual dependence in
  Lewis's theory of knowledge (\citeproc{ref-Lewis1996b}{Lewis 1996}).
  Indeed, McCawley credits Lewis (\citeproc{ref-Lewis1979f}{1979b}) as
  an inspiration for his ideas.}

This move does seem to save the theory from potentially troubling data,
but without further support it may seems rather desperate. There are two
independent motivations for it. First, it explains the inappropriateness
of (6).

\begin{description}
\tightlist
\item[(6)]
*Grannie won, but if she lost she was furious.
\end{description}

If assertion narrows the contextually relevant worlds to those where the
assertion is true, as Stalnaker (\citeproc{ref-Stalnaker1978}{1978})
suggests, and uttering a conditional requires expanding the context to
include worlds where the antecedent is true, it follows that utterances
like (6) will be defective. The speech acts performed by uttering each
clause give the hearer opposite instructions regarding how to amend the
context set. Secondly, McCawley's assumption explains why we generally
have little use for indicative conditionals whose antecedents we know
are false. To interpret an indicative we first have to expand the
context set to include a world where the antecedent is true, but if we
know the antecedent is false we usually have little reason to want to do
that. If there is a dispute over the size of the context set, we may
want to expand it so as to avoid miscommunication, which explains why we
will sometimes assert conditionals with antecedents we know to be false
when trying to convince someone else that the antecedent really is
false.

So we have a pair of measures that give plausible answers on a wide
range of cases. Such a pair should also validate the close connection
between indicatives and subjunctives we saw earlier. The data set out in
section one suggests that this connection may be close to synonymy, as
in (1), but in some cases, as in (11) and (12), the connection is much
looser. The differing behaviour of rigid designators in indicatives and
subjunctives reveals a further difference, but the two-dimensional
nature of the analysis, not the particulars of the similarity metric,
accounts for that. I propose to explain the data by looking at which
facts we hold fixed when trying to determine the nearest possible world.
The facts we hold fixed in evaluating indicatives and subjunctives,
according to the two metrics outlined above, are the same in just the
cases we feel that the indicatives and subjunctives say the same thing.

When evaluating an indicative we hold fixed all the facts known by any
member of the conversation. When evaluating a subjunctive we hold fixed
(a) all facts about the world up to some salient time \emph{t} and (b)
the holding of the laws of nature at all times after \emph{t}. The time
\emph{t} is the latest time such that some worlds fitting this
description make \emph{A} true and contain no large miracles. The two
sets of facts held fixed match when we know all the salient facts about
times before \emph{t}, and know no particular facts about what happens
after \emph{t}.

In the opinion poll case, when evaluating the original indicative our
knowledge at the earlier time was held fixed. We knew that the polls
predicted a Reagan landslide, that when one makes spectacularly false
predictions one is discredited, and so on. When we turn to evaluating
the subjunctive, we hold fixed the facts about the world before the
election (presumably the relevant time \emph{t}) and some laws.
Therefore, we hold fixed the polls predictions, and the law that when
one makes spectacularly false predictions one is discredited. So the
same facts are held fixed. And in general, this will happen whenever all
we know is all the specific facts up to the relevant time, and some laws
that allow us to extrapolate from those facts.

In the case where indicatives and subjunctives come apart, as in (11)
and (12), the relevant knowledge differs from the first case. By
hypothesis, we do not know who pulled the trigger, but we do know that a
trigger was pulled. Our knowledge of the relevant facts does not consist
in knowledge of all the details up to a salient time, and knowledge that
the world will continue in a law-governed way after this. Therefore, we
would predict that the indicatives and subjunctives would come apart,
because what is held fixed when evaluating the two conditionals differs.
We find exactly that. So the pair of measures can explain the close
connection between indicatives and subjunctives when it exists, and
explain why the two come apart when they do come apart.

\subsection{No Nearest Possible World}\label{no-nearest-possible-world}

Generally, there are three kinds of problems under this heading. First,
there may be no \emph{A}-worlds, and so no nearest \emph{A}-world.
Secondly, there may be an infinite sequence of ever-nearer
\emph{A}-worlds without a nearest \emph{A}-world. Thirdly, there may be
several worlds in a tie for nearest \emph{A}-world. If the measure
suggested in the previous section is correct, the first two problems do
not arise here. The third problem, however, arises almost all the time,
so we need to say something about it.

The approach I favour is set out in Stalnaker
(\citeproc{ref-Stalnaker1981}{1981}). The comparative similarity measure
is a partial order on the possible worlds. Stalnaker recommends we
assess conditionals using supervaluations, taking the precisifications
to be the complete extensions of this partial order. In particular, if
several possible worlds tie for being the closest
\emph{A}-worlds\footnote{Of course in this context \emph{x} is an
  \emph{A}- world iff \(\vDash_x^x\)\emph{A}.}, then \emph{A}~→~\emph{B}
will be true if they are all \emph{B}-worlds, false if they are all
¬\emph{B}-worlds, and not truth-valued otherwise. For consistent
\emph{A}, this makes ¬(\emph{A}~→~\emph{B}) equivalent to
\emph{A}~→~¬\emph{B}. Since we generally deny \emph{A}~→~\emph{B} just
when we would be prepared to assert \emph{A}~→~¬\emph{B}, this seems
like a good outcome.\footnote{Edgington
  (\citeproc{ref-Edgington1996}{1996}) furnishes some nice examples
  against the view that \emph{A}~□→~\emph{B} should be false when there
  are several equally close \emph{A}-worlds in a tie for closest and
  some are \emph{B}-worlds but some are ¬\emph{B}-worlds.} Further, this
account makes \emph{A}~→~\emph{B} generally come out gappy when \emph{A}
is false. Many theorists hold that indicative conditionals, especially
those with false antecedents, lack truth values.\footnote{See Edgington
  (\citeproc{ref-Edgington1995}{1995}) for an endorsement of this
  position and discussion of others who have held it.} This can't be
right in general, since it is a platitude that \emph{A}~→~A is true for
every \emph{A}, but the position has some attraction. Happily, our
theory respects the motivations behind such positions without violating
the platitude.

In any case, these details are not important to the overall analysis. If
someone favours a resolution of ties along the lines Lewis suggested
this could easily be appended onto the basic theory.

\subsection{The General Theory}\label{the-general-theory}

So far, I have just defined what it is for \emph{A}~→~\emph{B} to be
true in this world from the perspective of this world as actual. To have
a fully general theory I need to say when \emph{A}~→~\emph{B} is true in
an arbitrary world from the perspective of another (possibly different)
world as actual. And that general theory must yield the theory above as
a special case when applied to our world. As with the special theory
above, the general theory will mostly be derived from Twin Earth
considerations.

In general, \(\vDash_y^x\) \emph{A}~→~\emph{B} iff the nearest world
pair ⟨\emph{z},~\emph{v}⟩ such that \(\vDash_v^z\) \emph{A}~is such that
\(\vDash_v^z\)\emph{B}. Nearness is again defined epistemically, but
what \emph{we} know about \emph{x} and \emph{y} matters. In particular
if \(\vDash_v^z\)\emph{C} for all sentences \emph{C} such that someone
in the context knows that \(\vDash_y^x\)\emph{C} , but not
\(\vDash_w^u\) \emph{C} for some such \emph{C} , then
⟨\emph{z},~\emph{v}⟩ is closer to ⟨\emph{x},~\emph{y}⟩ than is
⟨\emph{u},~\emph{w}⟩. As should be clear from this, nearness is
context-dependent, and the context it depends on is the actual speaker's
context. For conditionals as for quantified sentences, the same words
will express different propositions in different contexts.

Let's draw out some consequences of this definition. First, for any
\emph{x} we know that \(\vDash_x^x\)\emph{C} for all \emph{a priori}
propositions \emph{C}. In particular, we know that
\(\vDash_x^x\)\emph{d}~↔~(Actually \emph{d}) for any proposition
\emph{d}, where `↔' represents the material biconditional. So the
nearest world pair ⟨\emph{z},~\emph{v}⟩ to ⟨\emph{x}, \emph{x}⟩ must be
one in which \emph{z}~=~v\emph{, even if that means }z* is the
impossible world \emph{i}. Hence the general theory of indicatives
reduces to the special theory set out above when applied to
epistemically possible worlds: when assessing the truth value of an
indicative in an epistemically possible world pair we need only look at
other epistemically possible world pairs.

Secondly, when evaluating conditionals with respect to epistemically
impossible world pairs ⟨\emph{x},~\emph{y}⟩, we need to use other
epistemically impossible world pairs. For example, imagine some
explorers are wandering around Twin Australia, a dry continent to the
south of Twin Earth. As explorers of such lands are wont to do, they are
dying of thirst, so they are seeking some watery stuff to save
themselves. Without knowing whether they succeed, we know (22) is false.

\begin{description}
\tightlist
\item[(22)]
If the explorers find some watery stuff, they will find some water.
\end{description}

This theory can explain the falsity of (22). We know, from the way Twin
Earth is stipulated, that all the watery stuff of the explorers'
acquaintance is not water. So we know any watery stuff they find will
not be water. And we know that water is scarce on Twin Earth, even
scarcer than watery stuff in Twin Australia, so it is unlikely they will
find some watery stuff and simultaneously stumble across some water.

This theory also explains occurrences of indicatives embedded in
subjunctives. These are very odd, as should be expected if indicatives
are about epistemic connections and subjunctives about metaphysical
connections, but we can just make sense of them some of the time. For
example, it seems possible to make sense of (23) and that it is true.

\begin{description}
\tightlist
\item[(23)]
If the bullet that actually killed JFK had instead killed Jackie
Kennedy, then it would be true that if Oswald didn't kill Jackie
Kennedy, someone else did.
\end{description}

On our theory, to evaluate this we first find the nearest world pair
⟨@,~w⟩ such that \(\vDash_w^@\) The bullet that actually killed JFK
instead killed Jackie Kennedy, and then evaluate the indicative relative
to it. Now one thing we know about this world pair is that in it,
someone killed Jackie Kennedy. So this must hold in all nearby world
pairs. Hence in any such world pair that Oswald did not kill Jackie
Kennedy, someone else did, so (23) turns out true.

It might be thought that such embeddings do not make particularly good
sense. I have some sympathy for such a view. If one adopts the `special
theory' developed in the previous section, and rejects the general
theory developed in this subsection, one may have an explanation for the
impossibility of such embeddings. However, even if we cannot make sense
of such embeddings, we still need to account for the truth conditions of
indicatives relative to epistemically impossible world pairs to make
sense of claims such as \emph{Necessarily} (\emph{A} →~A).\footnote{I am
  indebted to Lloyd Humberstone for pointing this out to me.}

\subsection{Classifying Conditionals}\label{classifying-conditionals}

In recent years, there has been extensive debate over where the line
between indicatives and subjunctives falls. This debate focuses on
whether `future indicatives' like (24) are properly classified with
indicatives or subjunctives.

\begin{description}
\tightlist
\item[(24)]
If Booth doesn't shoot Lincoln, someone else will.
\end{description}

Jackson (\citeproc{ref-Jackson1990}{1990}) and Bennett
(\citeproc{ref-Bennett1995}{1995}) argue that this should go with
ordinary indicatives. Dudman (\citeproc{ref-Dudman1994}{1994}) and
Bennett (\citeproc{ref-Bennett1988}{1988}) argue that it should go with
ordinary subjunctives, though this is not how Dudman would put it. This
theory of indicatives appears to favour Jackson and (the later) Bennett,
because of the apparent triviality of conditionals like (25).

\begin{description}
\tightlist
\item[(25)]
If it will rain then it will actually rain.
\end{description}

\section{Conclusion}\label{conclusion}

Despite its lack of attention in the literature, data about the role of
rigid designators in indicatives deserve close attention. Any plausible
theory of indicatives must be able to deal with it, and it isn't clear
how existing possible worlds theories could do so. The easiest way to
build a semantics for indicatives is to say that ``If \emph{A} then
\emph{C}'' is true just in case the nearest world in which \emph{A} is
true is a world where \emph{C} is true. Even before the hard questions
about the meaning of `nearest' here start to be asked, we know a theory
of this form is wrong because it makes mistaken predictions about the
role of rigid designators. A conditional like ``If the stuff in the
rivers, lakes and oceans really is XYZ, then water is XYZ'' is
\emph{true}, even though the consequent is true in no possible worlds.
The simplest way to solve this difficulty is to revisit the idea of
`true in a world'. Rather than looking for a nearby world in which
\emph{A} is true, and asking whether \emph{C} is true in it, we look for
a nearby world \emph{W} such that \emph{A} is true under the supposition
that \emph{W} is actual, and ask whether \emph{C} is true under the
supposition that \emph{W} is actual. In the terminology of Jackson
(\citeproc{ref-Jackson1998}{1998}), we look at worlds considered as
actual, rather than worlds considered as counterfactual. This simple
change makes an important difference to the way rigid designators
behave. There is no world in which water is XYZ. However, under the
supposition that the stuff in the rivers, lakes and oceans really is
XYZ, and the H\textsubscript{2}O theory is just a giant mistake, that
is, under the supposition that we are in the world known as Twin Earth,
water is XYZ. In short, ``water is XYZ'' is true in Twin Earth
\emph{considered as actual}, even though it is false in Twin Earth
considered as counterfactual. So the data about behaviour of rigid
designators in indicatives, data like the truth of ``If the stuff in the
rivers, lakes and oceans really is XYZ, then water is XYZ'', does not
refute the hypothesis that ``If \emph{A} then \emph{C}'' is true iff the
nearest world such that \emph{A} is true in that world considered as
actual is a world where \emph{C} is true in that world considered as
actual.

In section two we looked at how the formal structure of a theory built
around that hypothesis might look. In section three we looked at how
some of the details may be filled in. The most pressing task is to
provide a similarity metric so we can have some idea about which worlds
will count as being nearby. The theory I defended has three important
features. First, it is \emph{epistemic}. Which worlds are nearby depends
on what is known by conversational participants. Secondly, it is
\emph{contextualist} in two respects. The first respect is that it is
the knowledge of the audience that matters, not just the knowledge of
the speaker and the intended audience. The second respect is that it
allows that what is known by the audience may be affected by the
utterance of the conditional. In particular, if the utterance of ``If
\emph{A}, \emph{B}'' causes the audience to consider \emph{A} to be
possible, and hence cease to know that ¬\emph{A}, then \emph{A} is not
part of what is known for purposes of determining which worlds are
nearby. (I assume here a broadly contextualist account of knowledge, as
in Lewis (\citeproc{ref-Lewis1996b}{1996}), but this is inessential. If
you do not like Lewis's theory, replace all references to knowledge
here, and in section 3.1, with references to epistemic certainty. I
presume that what is epistemically certain really is contextually
variable in the way Lewis suggests.) Thirdly, it is \emph{coarse-
grained}: whether a world is nearby depends only on whether it is
consistent with what is known, not `how much' it agrees with what is
known. The resultant theory seems to capture all the data, to explain
the generally close connection between indicatives and subjunctives, and
to explain the few differences which do arise between indicatives and
subjunctives.

The other detail to be filled in concerns embeddings of indicatives
inside subjunctives. The formalism here requires that we use the full
resources of two- dimensional modal logic, but the basic idea is very
simple. Consider a sentence of the form ``If it were the case that
\emph{A}, it would be the case that if \emph{B}, \emph{C} .'' Roughly,
this will be true iff the metaphysically nearest world in which \emph{A}
is true, call it \emph{w\textsubscript{A}}, is a world where
\emph{B}~→~\emph{C} is true. And that will be true iff the epistemically
nearest world to \emph{w\textsubscript{A}} is which \emph{B} is true is
a world where \emph{C} is true. Less roughly, we have to quantify not
over worlds, but over pairs of worlds, where the first element of the
pair determines the reference for rigid designators, and the second
element determines the truth of sentences given those references. But
this only adds to the formal complexity; the underlying idea is still
the same. The important philosophical point to note is that when we are
trying to find the epistemically nearest world to
\emph{w\textsubscript{A}} (or, more strictly, the nearest world pair to
⟨ @,~\emph{w\textsubscript{A}}⟩) the facts that have to be held fixed
are the facts that we know about \emph{w\textsubscript{A}}, not what our
counterparts in \emph{w\textsubscript{A}}, or indeed what any inhabitant
of \emph{w\textsubscript{A}} knows about their world. These embeddings
may be rare in everyday speech, but since they are our best guide to the
truth values of indicatives in other possible worlds, they are
theoretically very important.

\subsection*{References}\label{references}
\addcontentsline{toc}{subsection}{References}

\phantomsection\label{refs}
\begin{CSLReferences}{1}{0}
\bibitem[\citeproctext]{ref-Barker1997}
Barker, Stephen. 1997. {``Material Implication and General Indicative
Conditionals.''} \emph{The Philosophical Quarterly} 47 (187): 195--211.
doi:
\href{https://doi.org/10.1111/1467-9213.00055}{10.1111/1467-9213.00055}.

\bibitem[\citeproctext]{ref-Bennett1988}
Bennett, Jonathan. 1988. {``Farewell to the Phlogiston Theory of
Conditionals.''} \emph{Mind} 97 (388): 509--27. doi:
\href{https://doi.org/10.1093/mind/xcvii.388.509}{10.1093/mind/xcvii.388.509}.

\bibitem[\citeproctext]{ref-Bennett1995}
---------. 1995. {``Classifying Conditionals: The Traditional Way Is
Right.''} \emph{Mind} 104 (414): 331--54. doi:
\href{https://doi.org/10.1093/mind/104.414.331}{10.1093/mind/104.414.331}.

\bibitem[\citeproctext]{ref-Davies1980}
Davies, Martin, and I. L. Humberstone. 1980. {``Two Notions of
Necessity.''} \emph{Philosophical Studies} 38 (1): 1--30. doi:
\href{https://doi.org/10.1007/bf00354523}{10.1007/bf00354523}.

\bibitem[\citeproctext]{ref-Dudman1994}
Dudman, V. H. 1994. {``Against the Indicative.''} \emph{Australasian
Journal of Philosophy} 72 (1): 17--26. doi:
\href{https://doi.org/10.1080/00048409412345851}{10.1080/00048409412345851}.

\bibitem[\citeproctext]{ref-Edgington1995}
Edgington, Dorothy. 1995. {``On Conditionals.''} \emph{Mind} 104 (414):
235--327. doi:
\href{https://doi.org/10.1093/mind/104.414.235}{10.1093/mind/104.414.235}.

\bibitem[\citeproctext]{ref-Edgington1996}
---------. 1996. {``Lowe on Conditional Probability.''} \emph{Mind} 105
(420): 617--30. doi:
\href{https://doi.org/10.1093/mind/105.420.617}{10.1093/mind/105.420.617}.

\bibitem[\citeproctext]{ref-Gibbard1981}
Gibbard, Allan. 1981. {``Two Recent Theories of Conditionals.''} In
\emph{Ifs}, edited by William Harper, Robert C. Stalnaker, and Glenn
Pearce, 211--47. Dordrecht: Reidel.

\bibitem[\citeproctext]{ref-Grice1989}
Grice, H. Paul. 1989. \emph{Studies in the Way of Words}. Cambridge,
MA.: Harvard University Press.

\bibitem[\citeproctext]{ref-Jackson1987}
Jackson, Frank. 1987. \emph{Conditionals}. Blackwell: Oxford.

\bibitem[\citeproctext]{ref-Jackson1990}
---------. 1990. {``Classifying Conditionals.''} \emph{Analysis} 50 (2):
134--47. doi:
\href{https://doi.org/10.1093/analys/50.2.134}{10.1093/analys/50.2.134}.

\bibitem[\citeproctext]{ref-Jackson1998}
---------. 1998. \emph{From Metaphysics to Ethics: A Defence of
Conceptual Analysis}. Clarendon Press: Oxford.

\bibitem[\citeproctext]{ref-Kripke1980}
Kripke, Saul. (1972) 1980. \emph{Naming and Necessity}. Cambridge:
Harvard University Press.

\bibitem[\citeproctext]{ref-Lewis1973a}
Lewis, David. 1973. \emph{Counterfactuals}. Oxford: Blackwell
Publishers.

\bibitem[\citeproctext]{ref-Lewis1976b}
---------. 1976. {``Probabilities of Conditionals and Conditional
Probabilities.''} \emph{Philosophical Review} 85 (3): 297--315. doi:
\href{https://doi.org/10.2307/2184045}{10.2307/2184045}. Reprinted in
his \emph{Philosophical Papers}, Volume 2, Oxford: Oxford University
Press, 1986, 133-152. References to reprint.

\bibitem[\citeproctext]{ref-Lewis1979c}
---------. 1979a. {``Counterfactual Dependence and Time's Arrow.''}
\emph{No{û}s} 13 (4): 455--76. doi:
\href{https://doi.org/10.2307/2215339}{10.2307/2215339}. Reprinted in
his \emph{Philosophical Papers}, Volume 2, Oxford: Oxford University
Press, 1986, 32-52. References to reprint.

\bibitem[\citeproctext]{ref-Lewis1979f}
---------. 1979b. {``Scorekeeping in a Language Game.''} \emph{Journal
of Philosophical Logic} 8 (1): 339--59. doi:
\href{https://doi.org/10.1007/bf00258436}{10.1007/bf00258436}. Reprinted
in his \emph{Philosophical Papers}, Volume 1, Oxford: Oxford University
Press, 1983, 233-249. References to reprint.

\bibitem[\citeproctext]{ref-Lewis1996b}
---------. 1996. {``Elusive Knowledge.''} \emph{Australasian Journal of
Philosophy} 74 (4): 549--67. doi:
\href{https://doi.org/10.1080/00048409612347521}{10.1080/00048409612347521}.
Reprinted in his \emph{Papers in Metaphysics and Epistemology},
Cambridge: Cambridge University Press, 1999, 418-446. References to
reprint.

\bibitem[\citeproctext]{ref-McCawley1996}
McCawley, James. 1996. {``Conversational Scorekeeping and the
Interpretation of Conditional Sentences.''} In \emph{Grammatical
Constructions}, edited by Masayoshi Shibatani and Sandra Thompson,
77--101. Oxford: Clarendon Press.

\bibitem[\citeproctext]{ref-Ramsey1929}
Ramsey, Frank. 1929/1990. {``Probability and Partial Belief.''} In
\emph{Philosophical Papers}, edited by D. H. Mellor, 95--96. Cambridge
University Press.

\bibitem[\citeproctext]{ref-Stalnaker1978}
Stalnaker, Robert. 1978. {``Assertion.''} \emph{Syntax and Semantics} 9:
315--32.

\bibitem[\citeproctext]{ref-Stalnaker1981}
---------. 1981. {``A Defence of Conditional Excluded Middle.''} In
\emph{Ifs}, edited by William Harper, Robert C. Stalnaker, and Glenn
Pearce, 87--104. Dordrecht: Reidel.

\end{CSLReferences}



\noindent Published in\emph{
Philosophical Quarterly}, 2001, pp. 200-216.


\end{document}
