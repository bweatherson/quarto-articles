% Options for packages loaded elsewhere
% Options for packages loaded elsewhere
\PassOptionsToPackage{unicode}{hyperref}
\PassOptionsToPackage{hyphens}{url}
%
\documentclass[
  11pt,
  letterpaper,
  DIV=11,
  numbers=noendperiod,
  twoside]{scrartcl}
\usepackage{xcolor}
\usepackage[left=1.1in, right=1in, top=0.8in, bottom=0.8in,
paperheight=9.5in, paperwidth=7in, includemp=TRUE, marginparwidth=0in,
marginparsep=0in]{geometry}
\usepackage{amsmath,amssymb}
\setcounter{secnumdepth}{3}
\usepackage{iftex}
\ifPDFTeX
  \usepackage[T1]{fontenc}
  \usepackage[utf8]{inputenc}
  \usepackage{textcomp} % provide euro and other symbols
\else % if luatex or xetex
  \usepackage{unicode-math} % this also loads fontspec
  \defaultfontfeatures{Scale=MatchLowercase}
  \defaultfontfeatures[\rmfamily]{Ligatures=TeX,Scale=1}
\fi
\usepackage{lmodern}
\ifPDFTeX\else
  % xetex/luatex font selection
  \setmainfont[ItalicFont=EB Garamond Italic,BoldFont=EB Garamond
SemiBold]{EB Garamond Math}
  \setsansfont[]{EB Garamond}
  \setmathfont[]{Garamond-Math}
\fi
% Use upquote if available, for straight quotes in verbatim environments
\IfFileExists{upquote.sty}{\usepackage{upquote}}{}
\IfFileExists{microtype.sty}{% use microtype if available
  \usepackage[]{microtype}
  \UseMicrotypeSet[protrusion]{basicmath} % disable protrusion for tt fonts
}{}
\usepackage{setspace}
% Make \paragraph and \subparagraph free-standing
\makeatletter
\ifx\paragraph\undefined\else
  \let\oldparagraph\paragraph
  \renewcommand{\paragraph}{
    \@ifstar
      \xxxParagraphStar
      \xxxParagraphNoStar
  }
  \newcommand{\xxxParagraphStar}[1]{\oldparagraph*{#1}\mbox{}}
  \newcommand{\xxxParagraphNoStar}[1]{\oldparagraph{#1}\mbox{}}
\fi
\ifx\subparagraph\undefined\else
  \let\oldsubparagraph\subparagraph
  \renewcommand{\subparagraph}{
    \@ifstar
      \xxxSubParagraphStar
      \xxxSubParagraphNoStar
  }
  \newcommand{\xxxSubParagraphStar}[1]{\oldsubparagraph*{#1}\mbox{}}
  \newcommand{\xxxSubParagraphNoStar}[1]{\oldsubparagraph{#1}\mbox{}}
\fi
\makeatother


\usepackage{longtable,booktabs,array}
\usepackage{calc} % for calculating minipage widths
% Correct order of tables after \paragraph or \subparagraph
\usepackage{etoolbox}
\makeatletter
\patchcmd\longtable{\par}{\if@noskipsec\mbox{}\fi\par}{}{}
\makeatother
% Allow footnotes in longtable head/foot
\IfFileExists{footnotehyper.sty}{\usepackage{footnotehyper}}{\usepackage{footnote}}
\makesavenoteenv{longtable}
\usepackage{graphicx}
\makeatletter
\newsavebox\pandoc@box
\newcommand*\pandocbounded[1]{% scales image to fit in text height/width
  \sbox\pandoc@box{#1}%
  \Gscale@div\@tempa{\textheight}{\dimexpr\ht\pandoc@box+\dp\pandoc@box\relax}%
  \Gscale@div\@tempb{\linewidth}{\wd\pandoc@box}%
  \ifdim\@tempb\p@<\@tempa\p@\let\@tempa\@tempb\fi% select the smaller of both
  \ifdim\@tempa\p@<\p@\scalebox{\@tempa}{\usebox\pandoc@box}%
  \else\usebox{\pandoc@box}%
  \fi%
}
% Set default figure placement to htbp
\def\fps@figure{htbp}
\makeatother


% definitions for citeproc citations
\NewDocumentCommand\citeproctext{}{}
\NewDocumentCommand\citeproc{mm}{%
  \begingroup\def\citeproctext{#2}\cite{#1}\endgroup}
\makeatletter
 % allow citations to break across lines
 \let\@cite@ofmt\@firstofone
 % avoid brackets around text for \cite:
 \def\@biblabel#1{}
 \def\@cite#1#2{{#1\if@tempswa , #2\fi}}
\makeatother
\newlength{\cslhangindent}
\setlength{\cslhangindent}{1.5em}
\newlength{\csllabelwidth}
\setlength{\csllabelwidth}{3em}
\newenvironment{CSLReferences}[2] % #1 hanging-indent, #2 entry-spacing
 {\begin{list}{}{%
  \setlength{\itemindent}{0pt}
  \setlength{\leftmargin}{0pt}
  \setlength{\parsep}{0pt}
  % turn on hanging indent if param 1 is 1
  \ifodd #1
   \setlength{\leftmargin}{\cslhangindent}
   \setlength{\itemindent}{-1\cslhangindent}
  \fi
  % set entry spacing
  \setlength{\itemsep}{#2\baselineskip}}}
 {\end{list}}
\usepackage{calc}
\newcommand{\CSLBlock}[1]{\hfill\break\parbox[t]{\linewidth}{\strut\ignorespaces#1\strut}}
\newcommand{\CSLLeftMargin}[1]{\parbox[t]{\csllabelwidth}{\strut#1\strut}}
\newcommand{\CSLRightInline}[1]{\parbox[t]{\linewidth - \csllabelwidth}{\strut#1\strut}}
\newcommand{\CSLIndent}[1]{\hspace{\cslhangindent}#1}



\setlength{\emergencystretch}{3em} % prevent overfull lines

\providecommand{\tightlist}{%
  \setlength{\itemsep}{0pt}\setlength{\parskip}{0pt}}



 


\setlength\heavyrulewidth{0ex}
\setlength\lightrulewidth{0ex}
\usepackage[automark]{scrlayer-scrpage}
\clearpairofpagestyles
\cehead{
  Brian Weatherson
  }
\cohead{
  Intrinsic Knowledge
  }
\ohead{\bfseries \pagemark}
\cfoot{}
\makeatletter
\newcommand*\NoIndentAfterEnv[1]{%
  \AfterEndEnvironment{#1}{\par\@afterindentfalse\@afterheading}}
\makeatother
\NoIndentAfterEnv{itemize}
\NoIndentAfterEnv{enumerate}
\NoIndentAfterEnv{description}
\NoIndentAfterEnv{quote}
\NoIndentAfterEnv{equation}
\NoIndentAfterEnv{longtable}
\NoIndentAfterEnv{abstract}
\renewenvironment{abstract}
 {\vspace{-1.25cm}
 \quotation\small\noindent\emph{Abstract}:}
 {\endquotation}
\newfontfamily\tfont{EB Garamond}
\addtokomafont{disposition}{\rmfamily}
\addtokomafont{title}{\normalfont\itshape}
\let\footnoterule\relax

\makeatletter
\renewcommand{\@maketitle}{%
  \newpage
  \null
  \vskip 2em%
  \begin{center}%
  \let \footnote \thanks
    {\itshape\huge\@title \par}%
    \vskip 0.5em%  % Reduced from default
    {\large
      \lineskip 0.3em%  % Reduced from default 0.5em
      \begin{tabular}[t]{c}%
        \@author
      \end{tabular}\par}%
    \vskip 0.5em%  % Reduced from default
    {\large \@date}%
  \end{center}%
  \par
  }
\makeatother
\RequirePackage{lettrine}

\renewenvironment{abstract}
 {\quotation\small\noindent\emph{Abstract}:}
 {\endquotation\vspace{-0.02cm}}
\KOMAoption{captions}{tableheading}
\makeatletter
\@ifpackageloaded{caption}{}{\usepackage{caption}}
\AtBeginDocument{%
\ifdefined\contentsname
  \renewcommand*\contentsname{Table of contents}
\else
  \newcommand\contentsname{Table of contents}
\fi
\ifdefined\listfigurename
  \renewcommand*\listfigurename{List of Figures}
\else
  \newcommand\listfigurename{List of Figures}
\fi
\ifdefined\listtablename
  \renewcommand*\listtablename{List of Tables}
\else
  \newcommand\listtablename{List of Tables}
\fi
\ifdefined\figurename
  \renewcommand*\figurename{Figure}
\else
  \newcommand\figurename{Figure}
\fi
\ifdefined\tablename
  \renewcommand*\tablename{Table}
\else
  \newcommand\tablename{Table}
\fi
}
\@ifpackageloaded{float}{}{\usepackage{float}}
\floatstyle{ruled}
\@ifundefined{c@chapter}{\newfloat{codelisting}{h}{lop}}{\newfloat{codelisting}{h}{lop}[chapter]}
\floatname{codelisting}{Listing}
\newcommand*\listoflistings{\listof{codelisting}{List of Listings}}
\makeatother
\makeatletter
\makeatother
\makeatletter
\@ifpackageloaded{caption}{}{\usepackage{caption}}
\@ifpackageloaded{subcaption}{}{\usepackage{subcaption}}
\makeatother
\usepackage{bookmark}
\IfFileExists{xurl.sty}{\usepackage{xurl}}{} % add URL line breaks if available
\urlstyle{same}
\hypersetup{
  pdftitle={Intrinsic Knowledge},
  pdfauthor={Brian Weatherson},
  hidelinks,
  pdfcreator={LaTeX via pandoc}}


\title{Intrinsic Knowledge}
\author{Brian Weatherson}
\date{2024}
\begin{document}
\maketitle
\begin{abstract}
This is a commentary on Jonathan Stoltz's paper ``Inferential Knowledge
and What that Determines'' for a volume called \emph{Learning from
Buddhist Logic} edited by Koji Tanaka.

Jonathon Stoltz's ``Inferential Knowledge and What that Determines''
discusses the idea that knowledge is (at least sometimes)
\emph{intrinsically determined}. In this note, I discuss how that idea
relates to contemporary work on the nature of intrinsicness, the nature
of knowledge, and the nature of essences.
\end{abstract}


\setstretch{1.1}
Jonathan Stoltz's paper is really insightful; I learned a lot from it
and I basically agree with the conclusion. In this note I wanted to make
some observations about how these debates connect to some contemporary
literature. In particular, I want to use the older debates that Stoltz
talks about as a way into resolving some unclarity in a few related
contemporary debates.

\section{Intrinsic Determination}\label{sec-intrinsic-determination}

I want to start from sort of the opposite end of the question about
intrinsic determination to Stoltz. Let's think about how intrinsic
determination could possibly fail to obtain. We're told that what it is
for intrinsic determination to hold is (INT*) is true.

\begin{description}
\tightlist
\item[(INT*)]
K(\emph{p}) can produce D(K(\emph{p}) is a cognition in which \emph{p}
is true).
\end{description}

Here's a very quick argument for (INT*). Since K is factive, the fact
that K(\emph{p}) determines that p.~And since K(\emph{p}) has \emph{p}
as its content, what factivity means in this case is that \emph{p} is
true. So K(\emph{p}) determines, i.e., produces a determination, that
\emph{p} is true, and hence that K(\emph{p}) is a cognition in which
\emph{p} is true.

Now something must be wrong with that `quick argument'. Too many careful
philosophers reject the thesis of universal intrinsic determination for
it to be that simple to argue for it. And the obvious thing that's wrong
with it is that it leaves out any role for \emph{intrinsicness}. We
don't want to just ask whether K(\emph{p}) can produce this
determination, but whether it can produce the determination in virtue of
its intrinsic features.

Let's give a name to the state K(\emph{p}) which doesn't quite so
strongly presuppose that it has certain features. Let's call it `Fred'.
Then we can reformulate the `very quick argument' as follows.

\begin{quote}
\textbf{Master Argument for Intrinsic Determination}

\begin{enumerate}
\def\labelenumi{\arabic{enumi}.}
\tightlist
\item
  Fred is intrinsically some kind of knowledge state.
\item
  Fred is intrinsically a state with content \emph{p}.
\item
  By 1, it is intrinsic to Fred that it's content is true.
\item
  By 2 and 3, it is intrinsic to Fred that \emph{p}.
\item
  By 4 and (INT*), intrinsic determination holds for Fred.
\end{enumerate}
\end{quote}

Since (INT*) is a definition, the real issues are about 1 and 2. What I
want to do for the rest of this note is go through how different views
about the nature of intrisnicness, and different views about knowledge
and essence, can motivate the various different views that Stoltz finds
in the historical literature.

\section{Intrinsicness}\label{sec-intrinsicness}

The modern literature on intrinsicness is very confusing, because there
are three different questions being discussed.

The first question concerns what kind of thing intrinsicness is. In
Marshall and Weatherson (\citeproc{ref-sep-intrinsic-extrinsic}{2023,
sec. 2}), there are four broad kinds identified. For simplicity, I'll
just focus on the last two: intrinsicness is a matter of a feature being
internal, and intrinsicness is a matter of a feature being shared by all
duplicates. These two most notably come apart with respect to identity
properties, like \emph{being Fred}. That's plausibly an internal
property, and so intrinsic in that sense, but it's not shared by
duplicates. Fred could have an identical twin who is not numerically
identical with them.

The second question is how to implement each of the various kinds. Even
once one has settled on particular kind of thing intrinsicness might be,
there are a lot of questions of detail to fill in. My contributions to
this literature (\citeproc{ref-Weatherson2001-WEAIPA}{Weatherson 2001},
\citeproc{ref-Weatherson2006-WEATAM}{2006}) presupposed that the
relevant kind is being shared by duplicates, and spent some time working
out precisely what it might be for a property to be shared by all
duplicates. At the time I thought this was the right way to think about
intrinsicness, but I really didn't argue for that; I was mostly just
working out details.

Then the third question is what kind of dispute the first question
really is. Is there one true notion of intrinsicness, and the proponents
of the different kinds of theory are debating what it is? Or are there
different notions, and the real question is which of them is most
useful, or perhaps which is most useful for different purposes? I used
to hold the `one true notion' view, but I've come around to thinking the
more pragmatic approach is the right one. Different kinds of
intrinsicness play different philosophical roles, and we need multiple
notions around here. Indeed, we'll see roles for the different kinds
just in this note.

\section{Knowledge as Extrinsic}\label{knowledge-as-extrinsic}

If we do understand intrinsicness as being shared by duplicates, then we
quickly get to Śaṅkaranandana's view that all knowledge is extrinsic.
Any knower has a duplicate in what Williams
(\citeproc{ref-Williams2016}{2016}) calls a `bubble world'; a world that
is a duplicate of the actual world immediately around the knower, but
where things go strange outside the bubble. Since the knower is
duplicated, presumably the knowledge, i.e., Fred, is also duplicated.
But since \emph{p} is false, it follows that either 1 or 2 must be
false. Either Fred's counterpart has different content, falsifying 2, or
Fred's counterpart is not factive, falsifying 1.

One might worry that's a little quick. That argument shows that all
knowledge of the external world is extrinsic, but it doesn't show that
knowledge of the internal world is extrinsic. If \emph{p} is the result
of performing the \emph{cogito} inference, then wouldn't it still be a
piece of knowledge when Fred's host's counterpart performs it?

To get around this problem, we'll draw on what Sider
(\citeproc{ref-Sider2001}{2001}) says about `maximal' properties. He
notes that most properties of ordinary macroscopic objects, like
knowers, are \emph{maximal}. That is, part of what it is to have that
property is that one is not part of a larger thing that also has it. For
example, I'm a thinker, but the mereological difference between me and
one of my hairs is not. I'm thinking about epistemology, and I know a
bit about it, and that mereological difference does not. But the
mereological difference has duplicates who are maximal; intuitively they
are worlds just like this one except that hair was plucked out. So the
property of knowing stuff about epistemology isn't shared by duplicates.
Similarly if both I and my counterpart with one fewer hair perform the
cogito, we both end up with knowledge. But my counterpart's duplicate in
this world, i.e., the mereological difference between me and that hair,
does not get knowledge that way. So even the knowledge one gets from the
cogito is not intrinsic.

This shows that if we understand intrinsicness as being shared by
duplicates, then Śaṅkaranandana is right and all knowledge is extrinsic.
But that's not the most helpful way to think about knowledge and
intrinsic determination. The same kind of argument shows that if we
understand intrinsic values to be values that all duplicates share, then
virtually nothing has intrinsic value. That doesn't seem like a
particularly useful way of thinking.\footnote{There is a tension here in
  the approach to intrinsic value in Moore
  (\citeproc{ref-Moore1903}{1903}). The `isolation test' for intrinsic
  value that Moore uses only really makes sense on the shared by
  duplicates approach to intriniscness. But that's not a helpful way to
  think about intrinsic value. For much more on this, see Zimmerman and
  Bradley (\citeproc{ref-sep-value-intrinsic-extrinsic}{2019, sec. 2}).}
A better approach is to say that some other notion of intrinsicness is
relevant to intrinsic value, and, relatedly, to intrinsic determination
of knowledge.

\section{Knowledge as Intrinsic}\label{sec-as-intrinsic}

To get the view that all knowledge is intrinsically determined, the
simplest way is to adopt these three premises.

\begin{itemize}
\tightlist
\item
  Intrinsicness is a matter of interiority, and that implies that all
  essential properties of a thing are intrinsic to it.
\item
  As Williamson (\citeproc{ref-Williamson2000}{2000}) argues, knowledge
  is a mental state. What I take him to mean by that is that knowledge
  states are essentially pieces of knowledge. It is part of their
  identity that they are pieces of knowledge.
\item
  The content of a mental state is essential to it.
\end{itemize}

Given the first and second premise, we can infer that it is intrinsic to
Fred that it's a piece of knowledge. Given the first and third, we can
infer that it is intrinsic to Fred that it has \emph{p} as its content.
And then as in the Master Argument, we can infer that it is intrinsic to
Fred that \emph{p}. Indeed, we can infer that it is intrinsic to Fred
that it meets all the pre-requisites for being knowledge that \emph{p}.

The first and third premises here are, by philosophy standards, not that
controversial. The second is much more controversial.

Before Williamson, the completely standard view was that beliefs were
mental states, and to say that a belief amounted to knowledge was
basically to positively evaluate it in a certain way. But being
knowledge was no more part of the essence of the belief than being good
was part of the essence of someone who was, as it turns out, mostly
good. Williamson argued, convincingly to many, that this was the wrong
way to look at things. He argued that we should reconceptualise
knowledge in much the same way that we had, over the preceding decades,
reconceptualised vision. There was an old view that states of visual
awareness, seeings, were essentially states of visual appearance, and
accidentally states where the appearances matched reality for the right
reasons.\footnote{See Lewis (\citeproc{ref-Lewis1980den}{1980}) for some
  of the complications involved in trying to say more precisely what
  those accidental features are.} The better view was to say that
seeings are a particular kind of mental state, and hallucinations and
the like are not worse versions of the same state, but different states
altogether. Similarly, knowledge states are not just better versions of
belief states, they are different states.

My sense is that this view fits better with Buddhist and other South
Asian traditions than the more traditional analytic view that knowledge
states were essentially beliefs and accidentally knowledge. So I'll
simply assume it in what follows. Then given the other two assumptions,
which as I said are also widely accepted in recent analytic philosophy,
we get that all knowledge is intrinsically determined.

So far I've shown how to get a view where knowledge is always
extrinsically determined, and a view where it is always intrinsically
determined. There is an important remaining question. What kind of view
could get the view that knowledge is (a) often intrinsically determined,
(b) always intrinsically determined when it is inferential knowledge,
but (c) sometimes extrinsically determined? To get this we need to
question the third assumption above, that cognitive states have their
contents essentially.

\section{The Genetic Conception}\label{sec-genetic}

Let's think a bit about the relationship between the genetic conception
of knowledge and what I was calling the third premise: content
essentialism. As I said earlier, that premise is very widely accepted in
recent analytic philosophy. One of the only discussions I know that
rejects it is Marian David's paper ``Content Essentialism''
(\citeproc{ref-David2002}{David 2002}), which is also where I took the
name for the premise.

In ``Luminous Margins'' (\citeproc{ref-Weatherson2004-WEALMT}{Weatherson
2004}) I used David's idea in a way that's relevant to some issues
raised in Stoltz's discussion. In particular, I wanted to make sense of
the idea that a correct mathematical guess could easily have been
mistaken. For example, if someone guesses that 193 plus 245 is 438, they
don't thereby know that 193 plus 245 is 438, and they don't know it
because that belief could easily have been wrong. But how could that be?
If the belief has its content essentially, and mathematical facts don't
change, then that belief is true in all nearby worlds. What I suggested,
following David, was that the belief could easily have had a different
content, and that different content could have been false.

In retrospect, it would have been useful to articulate the genetic
conception of cognition to back up this thought. What's essential to the
guess that 193 plus 245 is 438 is that it's a \emph{guess}. That's how
it came about, and on the genetic conception that's what is essential to
it. That very guess could easily have been the guess that 193 plus 245
is 448. Because of that it could easily have been false. I was
interested in how this related to ideas of safety, but let's set that
aside for now and focus on the idea of essence.

There is one quick way to argue against content essentialism that
doesn't go through, but which fails in an interesting way. Given content
externalism, the content of a cognitive state is not intrinsic to it.
Given that essential properties are intrinsic, it follows that the
content is not essential, as required. This argument fails because it
turns on an equivocation on `intrinsic'. Content externalism says that
the content of a cognitive state is not shared \emph{with its
duplicates}. It says that content is not intrinsic on the understanding
of intrinsicness as being shared by duplicates. That notion of
intrinsicness is useful, especially for characterising theses like
content externalism, but it isn't the one we're primarily working with
here. On the understanding of intrinsicness as something more like
internality, content might be essential even if it isn't shared by
duplicates, just like identity can be essential without being shared by
duplicates.

The genetic conception of cognition might remind the reader of the
thesis, associated with Kripke (\citeproc{ref-Kripke1980}{{[}1972{]}
1980}), of origin essentialism. Origin essentialism is the thesis that
things in general have their origins essentially. But there are two
important distinctions between the genetic conception and origin
essentialism that are a bit important.

Origin essentialism is a very general thesis about the essences of
things in general. The genetic conception is just about cognitive
states.

And in the literature following Kripke, origin essentialism is rarely
distinguished from a nearby thesis: the necessity of origins. The
necessity of origins thesis is that necessarily, if a thing exists, it
has the origins it actually has. It isn't unusual to see that thesis
simply equated with origin essentialism.\footnote{The clearest version
  I've found of that is in the opening of Graham Forbes's ``Origins and
  Identities'' (\citeproc{ref-Forbes2001}{2002}), where he starts with a
  definition of origin essentialism, and the definition is a statement,
  in quantified modal logic, of the necessity of origins thesis.}
Intuitively, origin essentialism and the necessity of origins could be
distinct.

It isn't true in general that a thing has a property essentially iff it
necessarily has the propertty. For instance, this coffee cup is
necessarily such that 193 plus 245 is 438, but I don't think that's part
of the essence of the coffee cup. Following Fine
(\citeproc{ref-Fine1994b}{1994}), we might think that Socrates is
necessarily a member of the singleton set containing Socrates, but he
isn't essentially a member of that set. (Conversely, the singleton does
essentially have Socrates as a member.) I think the criticisms that
Fraser MacBride and Frederique Janssen-Lauret
(\citeproc{ref-MacBrideJL2022}{2022}) make of the assumptions about the
metaphysics of sets behind that example are fairly convincing. Still,
there are enough cases where essential properties and necessary
properties come intuitively apart that we should keep the two theses,
origin essentialism and the necessity of origins, separate.

That last sentence probably won't be too controversial in very recent
philosophy. Fine's argument convinced a lot of philosophers to
distinguish modal claims and essential claims. But the history of how it
did so was surprising. Fine's paper had relatively little uptake when it
first came out. Google Scholar only reports 24 citations to it before
1999. But especially after 2010, it became standard to distinguish modal
claims from claims about essence.\footnote{I think there is an
  identifiable \emph{modal era} in analytic philosophy, running from
  roughly 1970 to 2010. The equation of various claims with claims in
  the language of modal logic or conditional logic is a key part of that
  era. But this volume isn't the place to go into differences between
  contemporary philosophy and (very) early 21st century philosophy.}

This is all by way of saying that when I was discussing content
essentialism back in 2004, I wasn't distinguishing it from the claim any
cognitive state is necessarily such that if it exists, it has the
content it actually has. Now it is plausible that the essentialist
thesis entails the necessity thesis. More generally, it is plausible
that any essential properties of a thing are properties the thing has
necessarily (if it exists). The intuitive counterexamples are all
counterexamples to the converse of that claim, that if a thing has a
property necessarily it has it essentially. All that we need to argue
against content essentialism though is that essential properties are
necessary properties. From that, plus the thought that a particular
guess could have had a different content while being the very same
guess, it follows that the content is not essential to the guess. And
perhaps what goes for guesses also goes for other cognitive states. With
that in mind, let's return to the three theses about intrinsic
determination we were trying to validate.

\section{Inferences and The Genetic
Conception}\label{inferences-and-the-genetic-conception}

First, I'll argue that even if we say content essentialism is not true
in general, and if we say that not all necessary properties are
essential properties, there is still an argument that pieces of
inferential knowledge are intrinsically determined to be knowledge. The
argument makes these three assumptions:

\begin{enumerate}
\def\labelenumi{\arabic{enumi}.}
\tightlist
\item
  Any essential property is an intrinsic property.
\item
  Any knowledge state is essentially a knowledge state.
\item
  Any cognitive state, including any knowledge state, has its genesis
  essentially.
\end{enumerate}

Now consider any case where someone knows some propositions, and they
properly infer some further proposition \emph{p} from them. Call the
resulting attitude towards \emph{p} Fred. Since Fred is the result of
making a proper inference from some pieces of knowledge, Fred is itself
a knowledge state.

By 2, the prior propositions Fred is inferred from are essentially bits
of knowledge. By 3, it is essential to Fred that it was inferred from
these particular prior states. Putting these together, it seems to
follow that it is essential to Fred that it was inferred from knowledge.

Now we have to be a bit careful here. If we aren't equating essential
properties with necessary properties, we can't simply conclude that
essential properties are closed under entailment. Still, the move here
doesn't look too controversial. All I've done is say that it isn't just
the numerical identity of the origins of a state that are essential to
the state, but also the essential features of those origins. That seems
like a plausible move even if essences are not in general closed under
entailment.

Is it also part of the essence of Fred that the inference was proper?
That isn't quite as obvious to me. But we probably don't need it. All
that we need is that the essence of Fred determines that the inference
was proper. Remember, the conclusion we're aiming for is not quite that
it is intrinsic to Fred that it is a piece of knowledge; it is that
Fred's intrinsic character determines that it is knowledge. If intrinsic
character (here taken to be essence) is not closed under entailment,
that is a slightly weaker claim, and it is all that we are going for.

Given the identity of the states Fred was inferred from, including their
status as knowledge, and the fact that Fred was inferred (directly) from
them, it seems like it should follow that the inference was proper.
Again, without assuming essences are closed under entailment this won't
mean this propriety is part of Fred's essence. But here we're just
trying to prove that Fred's essence determines that Fred is a knowledge
state. That is true as long as the essences of Fred and it's
predecessors determine that the inference to Fred is proper. That will
be true as long as propriety is non-contingent, which it plausibly is.

So putting all this together, we get that it is essential to Fred that
it is knowledge, and indeed knowledge that \emph{p}. After all, it
couldn't be knowledge of something else while being the kind of
inference it was from the premises it was from. By 1, it is intrinsic to
Fred that it is knowledge that \emph{p}. From that it follows that
\emph{p} is true, and indeed that Fred meets all of the pre-requisites
for knowledge. So inferential knowledge will be intrinsically determined
on this picture, as required.

\section{Perceptions and The Genetic Conception}\label{sec-perception}

That shows that on the picture I've just sketched, inferential knowledge
will be intrinsically determined to be knowledge. But we wanted
something stronger. We wanted this to fall out of a picture where some
knowledge is extrinsically determined. How can we get that?

The key move is going to be rejecting content essentialism. in the
circumstances where knowledge is extrinsically determined, it's
plausible that the content of a cognitive state is not essential to it.

Start with a case that's relatively simple. A person sees a rope in
terrible light. They form the true belief that it is a rope, not a
snake. But the light is terrible. That very method of belief formation
could easily have led to the belief that there was a snake. So on the
genetic conception of cognition, that very cognition could have been the
cognition that there was a snake. That is to say, this is plausibly a
case where content essentialism fails. What's essential to the cognition
is that it is based on visual appearances in terrible light. Those
appearances are both temporally and modally unstable. Even when they are
right, it isn't part of their essence that they are right.

This is plausibly a case where content essentialism fails, but it is not
yet a case where knowledge is extrinsically determined. After all, in
this kind of terrible light, even true cognitions about the world around
one are lucky guesses, and hence not knowledge. What we need, and here
I'm drawing on Williamson (\citeproc{ref-WilliamsonLofoten}{2013}), are
cases that are better than lucky guesses, but only \emph{just} better
than lucky guesses.

Change the case so that the light is poor, but not terrible. Now it is
plausible that the viewer does know that there is a rope, rather than a
snake. Even in poor light we know something about our surroundings. But
is this perceptual knowledge intrinsically determined to be knowledge?
In poor light, people do make mistakes. A state with the very same
genesis, i.e., perception in poor light, could have been a misperception
of the rope as a snake. If that's right, the essence of the state will
not determine that it is a knowledge providing veridical perception of a
rope.

What's crucial here is that the KK thesis fails for perceptual knowledge
in poor but not terrible conditions for perception. This is the case
that Stoltz sets aside at the start of §4, though from what he says
about inferential knowledge, I don't think that there's anything that
I'm particularly disagreeing with. In any case where KK fails for
perception, it seems to me that the genetic conception of cognition will
imply that intrinsic determination also fails. The perceptual state is
not essentially a perception that \emph{p}, which it needs to be for the
argument for intrinsic determination to go through.

\section{Conclusion}\label{conclusion}

Thinking about intrinsic determination helps shed light on several areas
of contemporary philosophy that could use some extra clarity. To
understand whether, and when, knowledge is intrinsically determined, we
need to at least do the following. First, we have to distinguish
intrinsicness as sharing by duplicates from intrinsicness as
internality. Second, we have to think through the implications of the
genetic conception of cognition, and how that is somewhat, but not
exactly, like origin essentialism. Third, we need to distinguish
essential properties from necessary properties. And fourth, we need to
work through how perceptual knowledge works in situations that are far
from perfect perceptual situations, but still good enough for getting
knowledge.

When we do all these things, we can see that all of the positions
adopted in the Buddhist tradition can be grounded in combinations of
views that are adopted in various places today. More importantly, by
doing this, we get a better understanding of what really separates the
different contemporary views.

\subsection*{References}\label{references}
\addcontentsline{toc}{subsection}{References}

\phantomsection\label{refs}
\begin{CSLReferences}{1}{0}
\bibitem[\citeproctext]{ref-David2002}
David, Marian. 2002. {``Content Essentialism.''} \emph{Acta Analytica}
17: 103--14. doi:
\href{https://doi.org/10.1007/bf03177510}{10.1007/bf03177510}.

\bibitem[\citeproctext]{ref-Fine1994b}
Fine, Kit. 1994. {``Essence and Modality.''} \emph{Philosophical
Perspectives} 8: 1--16. doi:
\href{https://doi.org/10.2307/2214160}{10.2307/2214160}.

\bibitem[\citeproctext]{ref-Forbes2001}
Forbes, Graham. 2002. {``Origins and Identities.''} In
\emph{Individuals, Essence and Identity: Themes of Analytic
Metaphysics}, edited by Andrea Bottani, Massimiliano Carrara, and
Pierdaniele Giaretta, 319--40. Dordrecht: Springer. doi:
\href{https://doi.org/10.1007/978-94-017-1866-0_16}{10.1007/978-94-017-1866-0\_16}.

\bibitem[\citeproctext]{ref-Kripke1980}
Kripke, Saul. (1972) 1980. \emph{Naming and Necessity}. Cambridge:
Harvard University Press.

\bibitem[\citeproctext]{ref-Lewis1980den}
Lewis, David. 1980. {``Veridical Hallucination and Prosthetic Vision.''}
\emph{Australasian Journal of Philosophy} 58 (3): 239--49. doi:
\href{https://doi.org/10.1080/00048408012341251}{10.1080/00048408012341251}.

\bibitem[\citeproctext]{ref-MacBrideJL2022}
MacBride, Fraser, and Frederique Janssen-Lauret. 2022. {``Why Lewis
Would Have Rejected Grounding.''} In \emph{Perspectives on the
Philosophy of {D}avid {K}. {L}ewis}, edited by Helen Beebee and A. R. J.
Fisher, 66--91. {O}xford {U}niversity {P}ress. doi:
\href{https://doi.org/10.1093/oso/9780192845443.003.0005}{10.1093/oso/9780192845443.003.0005}.

\bibitem[\citeproctext]{ref-sep-intrinsic-extrinsic}
Marshall, Dan, and Brian Weatherson. 2023. {``{Intrinsic vs. Extrinsic
Properties}.''} In \emph{The {Stanford} Encyclopedia of Philosophy},
edited by Edward N. Zalta and Uri Nodelman, {F}all 2023.
\url{https://plato.stanford.edu/archives/fall2023/entries/intrinsic-extrinsic/};
Metaphysics Research Lab, Stanford University.

\bibitem[\citeproctext]{ref-Moore1903}
Moore, G. E. 1903. \emph{Principia Ethica}. Cambridge: Cambridge
University Press.

\bibitem[\citeproctext]{ref-Sider2001}
Sider, Theodore. 2001. {``Maximality and Intrinsic Properties.''}
\emph{Philosophy and Phenomenological Research} 63 (2): 357--64. doi:
\href{https://doi.org/10.1111/j.1933-1592.2001.tb00109.x}{10.1111/j.1933-1592.2001.tb00109.x}.

\bibitem[\citeproctext]{ref-Weatherson2001-WEAIPA}
Weatherson, Brian. 2001. {``{Intrinsic Properties and Combinatorial
Principles}.''} \emph{Philosophy and Phenomenological Research} 63 (2):
365--80. doi: \href{https://doi.org/10.2307/3071070}{10.2307/3071070}.

\bibitem[\citeproctext]{ref-Weatherson2004-WEALMT}
---------. 2004. {``Luminous Margins.''} \emph{Australasian Journal of
Philosophy} 82 (3): 373--83. doi:
\href{https://doi.org/10.1080/713659874}{10.1080/713659874}.

\bibitem[\citeproctext]{ref-Weatherson2006-WEATAM}
---------. 2006. {``{The Asymmetric Magnets Problem}.''}
\emph{Philosophical Perspectives} 20: 479--92. doi:
\href{https://doi.org/10.1111/j.1520-8583.2006.00116.x}{10.1111/j.1520-8583.2006.00116.x}.

\bibitem[\citeproctext]{ref-Williams2016}
Williams, J. Robert G. 2016. {``Representational Scepticism: The Bubble
Puzzle.''} \emph{Philosophical Perspectives} 30 (1): 419--42. doi:
\href{https://doi.org/10.1111/phpe.12083}{10.1111/phpe.12083}.

\bibitem[\citeproctext]{ref-Williamson2000}
Williamson, Timothy. 2000. \emph{{Knowledge and its Limits}}. Oxford
University Press.

\bibitem[\citeproctext]{ref-WilliamsonLofoten}
---------. 2013. {``Gettier Cases in Epistemic Logic.''} \emph{Inquiry}
56 (1): 1--14. doi:
\href{https://doi.org/10.1080/0020174X.2013.775010}{10.1080/0020174X.2013.775010}.

\bibitem[\citeproctext]{ref-sep-value-intrinsic-extrinsic}
Zimmerman, Michael J., and Ben Bradley. 2019. {``{Intrinsic vs.
Extrinsic Value}.''} In \emph{The {Stanford} Encyclopedia of
Philosophy}, edited by Edward N. Zalta, {S}pring 2019.
\url{https://plato.stanford.edu/archives/spr2019/entries/value-intrinsic-extrinsic/};
Metaphysics Research Lab, Stanford University.

\end{CSLReferences}



\noindent Draft as of 27 July 2024.


\end{document}
