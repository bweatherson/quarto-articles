% Options for packages loaded elsewhere
\PassOptionsToPackage{unicode}{hyperref}
\PassOptionsToPackage{hyphens}{url}
%
\documentclass[
  10pt,
  letterpaper,
  DIV=11,
  numbers=noendperiod,
  twoside]{scrartcl}

\usepackage{amsmath,amssymb}
\usepackage{setspace}
\usepackage{iftex}
\ifPDFTeX
  \usepackage[T1]{fontenc}
  \usepackage[utf8]{inputenc}
  \usepackage{textcomp} % provide euro and other symbols
\else % if luatex or xetex
  \usepackage{unicode-math}
  \defaultfontfeatures{Scale=MatchLowercase}
  \defaultfontfeatures[\rmfamily]{Ligatures=TeX,Scale=1}
\fi
\usepackage{lmodern}
\ifPDFTeX\else  
    % xetex/luatex font selection
    \setmainfont[ItalicFont=EB Garamond Italic,BoldFont=EB Garamond
Bold]{EB Garamond Math}
    \setsansfont[]{Europa-Bold}
  \setmathfont[]{Garamond-Math}
\fi
% Use upquote if available, for straight quotes in verbatim environments
\IfFileExists{upquote.sty}{\usepackage{upquote}}{}
\IfFileExists{microtype.sty}{% use microtype if available
  \usepackage[]{microtype}
  \UseMicrotypeSet[protrusion]{basicmath} % disable protrusion for tt fonts
}{}
\usepackage{xcolor}
\usepackage[left=1in, right=1in, top=0.8in, bottom=0.8in,
paperheight=9.5in, paperwidth=6.5in, includemp=TRUE, marginparwidth=0in,
marginparsep=0in]{geometry}
\setlength{\emergencystretch}{3em} % prevent overfull lines
\setcounter{secnumdepth}{3}
% Make \paragraph and \subparagraph free-standing
\makeatletter
\ifx\paragraph\undefined\else
  \let\oldparagraph\paragraph
  \renewcommand{\paragraph}{
    \@ifstar
      \xxxParagraphStar
      \xxxParagraphNoStar
  }
  \newcommand{\xxxParagraphStar}[1]{\oldparagraph*{#1}\mbox{}}
  \newcommand{\xxxParagraphNoStar}[1]{\oldparagraph{#1}\mbox{}}
\fi
\ifx\subparagraph\undefined\else
  \let\oldsubparagraph\subparagraph
  \renewcommand{\subparagraph}{
    \@ifstar
      \xxxSubParagraphStar
      \xxxSubParagraphNoStar
  }
  \newcommand{\xxxSubParagraphStar}[1]{\oldsubparagraph*{#1}\mbox{}}
  \newcommand{\xxxSubParagraphNoStar}[1]{\oldsubparagraph{#1}\mbox{}}
\fi
\makeatother


\providecommand{\tightlist}{%
  \setlength{\itemsep}{0pt}\setlength{\parskip}{0pt}}\usepackage{longtable,booktabs,array}
\usepackage{calc} % for calculating minipage widths
% Correct order of tables after \paragraph or \subparagraph
\usepackage{etoolbox}
\makeatletter
\patchcmd\longtable{\par}{\if@noskipsec\mbox{}\fi\par}{}{}
\makeatother
% Allow footnotes in longtable head/foot
\IfFileExists{footnotehyper.sty}{\usepackage{footnotehyper}}{\usepackage{footnote}}
\makesavenoteenv{longtable}
\usepackage{graphicx}
\makeatletter
\newsavebox\pandoc@box
\newcommand*\pandocbounded[1]{% scales image to fit in text height/width
  \sbox\pandoc@box{#1}%
  \Gscale@div\@tempa{\textheight}{\dimexpr\ht\pandoc@box+\dp\pandoc@box\relax}%
  \Gscale@div\@tempb{\linewidth}{\wd\pandoc@box}%
  \ifdim\@tempb\p@<\@tempa\p@\let\@tempa\@tempb\fi% select the smaller of both
  \ifdim\@tempa\p@<\p@\scalebox{\@tempa}{\usebox\pandoc@box}%
  \else\usebox{\pandoc@box}%
  \fi%
}
% Set default figure placement to htbp
\def\fps@figure{htbp}
\makeatother
% definitions for citeproc citations
\NewDocumentCommand\citeproctext{}{}
\NewDocumentCommand\citeproc{mm}{%
  \begingroup\def\citeproctext{#2}\cite{#1}\endgroup}
\makeatletter
 % allow citations to break across lines
 \let\@cite@ofmt\@firstofone
 % avoid brackets around text for \cite:
 \def\@biblabel#1{}
 \def\@cite#1#2{{#1\if@tempswa , #2\fi}}
\makeatother
\newlength{\cslhangindent}
\setlength{\cslhangindent}{1.5em}
\newlength{\csllabelwidth}
\setlength{\csllabelwidth}{3em}
\newenvironment{CSLReferences}[2] % #1 hanging-indent, #2 entry-spacing
 {\begin{list}{}{%
  \setlength{\itemindent}{0pt}
  \setlength{\leftmargin}{0pt}
  \setlength{\parsep}{0pt}
  % turn on hanging indent if param 1 is 1
  \ifodd #1
   \setlength{\leftmargin}{\cslhangindent}
   \setlength{\itemindent}{-1\cslhangindent}
  \fi
  % set entry spacing
  \setlength{\itemsep}{#2\baselineskip}}}
 {\end{list}}
\usepackage{calc}
\newcommand{\CSLBlock}[1]{\hfill\break\parbox[t]{\linewidth}{\strut\ignorespaces#1\strut}}
\newcommand{\CSLLeftMargin}[1]{\parbox[t]{\csllabelwidth}{\strut#1\strut}}
\newcommand{\CSLRightInline}[1]{\parbox[t]{\linewidth - \csllabelwidth}{\strut#1\strut}}
\newcommand{\CSLIndent}[1]{\hspace{\cslhangindent}#1}

\setlength\heavyrulewidth{0ex}
\setlength\lightrulewidth{0ex}
\usepackage[automark]{scrlayer-scrpage}
\clearpairofpagestyles
\cehead{
  Brian Weatherson
  }
\cohead{
  Notes on Some Ideas in Lloyd Humberstone’s Philosophical Applications of Modal Logic
  }
\ohead{\bfseries \pagemark}
\cfoot{}
\makeatletter
\newcommand*\NoIndentAfterEnv[1]{%
  \AfterEndEnvironment{#1}{\par\@afterindentfalse\@afterheading}}
\makeatother
\NoIndentAfterEnv{itemize}
\NoIndentAfterEnv{enumerate}
\NoIndentAfterEnv{description}
\NoIndentAfterEnv{quote}
\NoIndentAfterEnv{equation}
\NoIndentAfterEnv{longtable}
\NoIndentAfterEnv{abstract}
\renewenvironment{abstract}
 {\vspace{-1.25cm}
 \quotation\small\noindent\rule{\linewidth}{.5pt}\par\smallskip
 \noindent }
 {\par\noindent\rule{\linewidth}{.5pt}\endquotation}
\cehead{
       Ishani Maitra and Brian Weatherson
        }
\cohead{Notes on Humberstone}
\KOMAoption{captions}{tableheading}
\makeatletter
\@ifpackageloaded{caption}{}{\usepackage{caption}}
\AtBeginDocument{%
\ifdefined\contentsname
  \renewcommand*\contentsname{Table of contents}
\else
  \newcommand\contentsname{Table of contents}
\fi
\ifdefined\listfigurename
  \renewcommand*\listfigurename{List of Figures}
\else
  \newcommand\listfigurename{List of Figures}
\fi
\ifdefined\listtablename
  \renewcommand*\listtablename{List of Tables}
\else
  \newcommand\listtablename{List of Tables}
\fi
\ifdefined\figurename
  \renewcommand*\figurename{Figure}
\else
  \newcommand\figurename{Figure}
\fi
\ifdefined\tablename
  \renewcommand*\tablename{Table}
\else
  \newcommand\tablename{Table}
\fi
}
\@ifpackageloaded{float}{}{\usepackage{float}}
\floatstyle{ruled}
\@ifundefined{c@chapter}{\newfloat{codelisting}{h}{lop}}{\newfloat{codelisting}{h}{lop}[chapter]}
\floatname{codelisting}{Listing}
\newcommand*\listoflistings{\listof{codelisting}{List of Listings}}
\makeatother
\makeatletter
\makeatother
\makeatletter
\@ifpackageloaded{caption}{}{\usepackage{caption}}
\@ifpackageloaded{subcaption}{}{\usepackage{subcaption}}
\makeatother

\usepackage{bookmark}

\IfFileExists{xurl.sty}{\usepackage{xurl}}{} % add URL line breaks if available
\urlstyle{same} % disable monospaced font for URLs
\hypersetup{
  pdftitle={Notes on Some Ideas in Lloyd Humberstone's Philosophical Applications of Modal Logic},
  pdfauthor={Steven Kuhn; Brian Weatherson},
  hidelinks,
  pdfcreator={LaTeX via pandoc}}


\title{Notes on Some Ideas in Lloyd Humberstone's Philosophical
Applications of Modal Logic}
\author{Steven Kuhn \and Brian Weatherson}
\date{2018}

\begin{document}
\maketitle
\begin{abstract}
Lloyd Humberstone's recently published \emph{Philosophical Applications
of Modal Logic} presents a number of new ideas in modal logic as well
explication and critique of recent work of many others. We extend some
of these ideas and answer some questions that are left open in the book.
\end{abstract}


\setstretch{1.1}
Lloyd Humberstone's recently published \emph{Philosophical Applications
of Modal Logic} (Humberstone (\citeproc{ref-Humberstone2016}{2016}))
presents a number of new ideas in modal logic as well explication and
critique of recent work of many others. In this note we extend some of
these ideas and answer some questions that are left open in the book.
Numbers without other identification refer to pages in that book.

\section{Local and Global Conditions}\label{local-and-global-conditions}

One theme sounded frequently in Humberstone
(\citeproc{ref-Humberstone2016}{2016}) is the relation between a local
condition, which describes a point in a frame and a global condition,
which concerns the frame as a whole. For example, the local conditions
of being reflexive (\(Rxx\)) and being reflexive with reflexive
successors (\(Rxx \land \forall y(Rxy \rightarrow Ryy)\)) are distinct,
but their universal possession by the points in a frame describes the
same global condition of reflexivity. As a consequence, the
non-equivalent modal axioms \(\Box p\rightarrow p\) and
\((\Box p\rightarrow p) \land \Box(\Box q\rightarrow q)\) both define
the class of reflexive frames. This example leads Humberstone to ask
(189) whether there is a local property not implying that a point
possessing it is reflexive whose universal possession makes the frame
reflexive. Affirmative answers are supplied by the following formulas:
\(\forall y(y{=}x\wedge Rxx) \vee (\exists y(y\ne x) \wedge \forall y(y\ne x \rightarrow Ryy))\)
(either x is the only world and it is reflexive or else there are other
worlds, all of which are reflexive), and
\(\exists zRxz \wedge \forall z(Rzx \rightarrow Rzz)\) (x has a
successor and every world that can see x is reflexive). The second
example implies that the tense-logical formulas
\({\mathbf{F}}\top \wedge \mathbf{H}(\mathbf{G}p\rightarrow p)\) and
\(\mathbf{G}p\rightarrow p\) both define the class of reflexive frames.

\section{Fully Modalized Logics}\label{fully-modalized-logics}

Another topic that gets well-deserved attention in Humberstone
(\citeproc{ref-Humberstone2016}{2016}) is the property of logics that
Humberstone calls being ``fully modalized.'' (See 290-304.) The idea is
that in alethic modal systems the axiom \(\Box A\rightarrow A\) provides
a logical connection between the modal and nonmodal formulas, whereas in
a doxastic or deontic logic we expect that matters concerning what is
believed or what ought to be the case should be logically independent of
those concerning what is the case. The latter, but not the former, are
fully modalized. But the idea needs to formulated with some care because
we don't want the presence of, for example, \(A\rightarrow \Box \top\)
as a theorem to count against a logic's being fully modalized. As
Humberstone puts it, in a fully modalized logic, ``\ldots we don't
expect\ldots the forging of any\ldots logical connections between
\(\Box A\) and \(A\) for any given \(A\) -- other than those which
hold\ldots derivatively'' (291). The notion is captured in a rather
complicated way by E. Zolin in Zolin (\citeproc{ref-Zolin.2000}{2000})
and Humberstone shows that the characterization there is equivalent to
the following simpler one: if there is a theorem of the form \(M\vee N\)
where \(M\) is fully a fully modalized formula (i.e., containing no
sentence letters not within the scope of a modal operator) and \(N\) is
non-modal (i.e., containing no occurrences of modal operators) then
either \(M\) or \(N\) is itself a theorem. In this section we show that
Zolin's characterization is also equivalent to an even simpler one that
is closer in spirit to the motivating remarks in Humberstone
(\citeproc{ref-Humberstone2016}{2016}): every theorem is a tautological
consequence of a fully modalized theorem. (Thus the theorems can be
divided into two categoriesthe essentially nonmodal ones, i.e., the
tautologies, and the essentially modal ones, i.e., the non-tautologies
that are tautological consquences of fully modalized theorems).

We begin by restating Zolin's definition in our own terminology. If
\(p_1,{\ldots},p_n\) are sentence letters, then a state description in
\({p_1,{\ldots},p_n}\) is a conjunction
\(p_1^*\wedge {\ldots}\wedge p_n^*\), where, for \(1\le i\le n, p_i^*\)
is either \(p_i\) or \(\neg p_i\). The (truth-functional) constituents
of a formula \(A\) are the sentence-letters and \(\Box\)-formulas
occurring in \(A\) that do not properly occur within the scope of any
\(\Box\). Zolin observes that every formula \(A\) can be ``decomposed''
into a formula of the form
\({\bigvee}\{(\vec{p}\wedge B(\vec{p}))\!:\vec{p}\) is a state
description in the sentence letter constituents of \(A\)\}, where, for
each \(\vec{p}\) , \(B(\vec{p})\) is some truth functional combination
of the modal constituents of \(A\). By fixing on a particular ordering
of formulas and taking the \(B(\vec{p})\)'s to be in a disjunctive
normal form that conforms to this ordering, we can single out a unique
decomposition of this kind. Let's call it the \emph{Zolin form} of \(A\)
and let's call the formulas \(B(\vec{p})\) that occur as right conjunct
of a disjunct in the Zolin form of \(A\), the \emph{Zolin components} of
\(A\). Note that every formula is truth-funtionally equivalent to its
Zolin form. Then, according to Zolin's definition, a logic is fully
modalized if \(\vdash A\) implies \(\vdash B(\vec{p})\) for every
\(B(\vec{p})\) that is a Zolin component of \(A\).

\textbf{Theorem 1}\\
A logic L is fully modalized (according to Zolin's definition) iff every
theorem of L is a tautological consequence of a fully modalized theorem.

\textbf{Proof}\\
(Left to right). Suppose L satisfies Zolin's definition and
\(\vdash_L A\). Let
\((\vec{p}_1\wedge B(\vec{p}_1))\vee {\ldots}\vee (\vec{p}_n\wedge B(\vec{p}_n))\)
be the Zolin form of \(A\). Then, according to Zolin's definition,
\(\vdash_L B(\vec{p}_i)\) for each \(i\), \(1\le i\le n\). Let
\(A^\prime = B(\vec{p}_1)\wedge {\ldots}\wedge B(\vec{p}_n)\).
\(A^\prime\) is fully modalized and, since each of its conjuncts is
provable in L, \(A^\prime\) is as well. All that remains is to show that
\(A\) is a truth-functional consequence of \(A^\prime\). Let \(\alpha\)
be any assignment of truth values to the constituents of \(A\) such that
\(\alpha\models A^\prime\). Let \(\vec{p}_\alpha\) be the state
description in the sentence letters that are truth-functional
constiuents of \(A\) that corresponds to \(\alpha\) in the sense that
each conjunct of \(\vec{p}_\alpha\) is the literal \emph{p} or
\(\neg p\) according to whether \(\alpha(p)\) is true or false. Then
\(\alpha\) verifies \((\vec{p}_\alpha \wedge B(\vec{p}_\alpha))\), which
is a disjunct of \(A\) and so \(\alpha\models A\) as required.

(Right to left). We are given that every theorem of L is a tautological
consequence of some fully modalized theorem. Now suppose \(\vdash_L A\)
and \(B(\vec{p})\) is a Zolin component of \(A\), with a view towards
showing \(\vdash_L B(\vec{p})\). By the initial suppositions, \(A\) is a
truth-functional consequence of some fully modalized formula
\(A^\prime\). Then the Zolin form of \(A\), call it
\((\vec{p}_1\wedge B(\vec{p}_1))\vee {\ldots}\vee (p_n\wedge B(\vec{p}_n))\),
is also a truth functional consequence of \(A^\prime\), where for some
\(i\), \(1\le i\le n\), \(B(\vec{p}_i)=B(\vec{p})\). We show that
\(B(\vec{p})\) is provable in L by showing that it is also a
truth-functional consequence of the theorem \(A^\prime\). To that end,
let \(\alpha\) be any assignment of truth values to the constituents of
\(A\), such that \(\alpha\models A^\prime\). Since \(A^\prime\) is fully
modalized, its truth value under an assignment is not affected by the
truth assignments to sentence letters, so we can assume without loss of
generality that these conform to the state description . Since the
disjunction
\((\vec{p}_1\wedge B(\vec{p}_1))\vee {\ldots}\vee (\vec{p}_n\wedge B(\vec{p}_n))\)
is a truth functional consequence of \(A^\prime\), \(\alpha\) must
verify this disjunction. But \(\vec{p}_1,{\ldots},\vec{p}_n\) are state
descriptions, so \(\alpha\) can verify only one disjunct of this
formula, namely \(\wedge B(\vec{p})\). Hence
\(\alpha\models B(\vec{p})\) as required.~◻

It is possible that there are applications for which Zolin's more
detailed normal-form characterization of a fully modalized logic is more
useful than the simple characterization given here. But the proof below
shows that property that Humberstone extracts in Humberstone
(\citeproc{ref-Humberstone2016}{2016}) can be proved at least as easily
from our simple characterization.

\textbf{Theorem 2}\\
Suppose every theorem of L is a tautological consequence of a fully
modalized theorem. Then \(\vdash_LM\vee N\) where \(M\) is fully
modalized and \(N\) is modality-free implies either \(\vdash_LM\) or
\(\vdash_LN\).

\textbf{Proof}\\
Assume the hypothesis of the claim and \(\vdash_LM\vee N\) for
appropriate \(M\) and \(N\). Then there is some fully modalized
L-theorem \(A^\prime\) such that \(M\vee N\) is a truth functional
consequence of \(A^\prime\). Suppose for reductio that neither \(M\) nor
\(N\) is provable. Then neither \(M\) nor \(N\) is a truth functional
consequence of \(A^\prime\). So there is an assignment \(\alpha\) of
truth values to the constituents of \(A^\prime\) and \(M\) that makes
the former true and the latter false. Similarly, there is an assignment
\(\beta\) to the constituents of \(A^\prime\) and \(N\) that makes
\(A^\prime\) true and \(N\) false. Now extend the assignment \(\alpha\)
to the sentence letters in \(N\) by assigning them the same truth values
as \(\beta\) does, and call the result \(\alpha^\prime\). Since
\(\alpha^\prime\) agrees with \(\alpha\) on the modal constituents it
verifies \(A^\prime\) and falsifies \(M\). Since it agrees with
\(\beta\) on sentence letters, it falsifies \(N\). This contradicts the
earlier observation that \((M\vee N)\) is a truth functional consequence
of \(A^\prime\).~◻

\section{``Nothing in Between'' and the Equivalence of Modal
Logics}\label{nothing-in-between-and-the-equivalence-of-modal-logics}

The impetus for Section 4.4 of Humberstone's book (304-324) is Arthur
Prior's observation that the logical structure of moral concepts appears
to be unlike those of quantity and alethic modality:

\begin{quote}
In between ``S must be P'' and ``S may be P'' stands the simple ``S is
in fact P'', just as ``This S is P stands in between''Every S is P and
``Some S is P''. \ldots{} But so far as I can see there is nothing among
the moral or `deontic' modalities that corresponds to these intermediary
`existential' or `alethic' modalities. (Prior
(\citeproc{ref-prior1951ethical}{1951}) p145, quoted on 304 in
Humberstone (\citeproc{ref-Humberstone2016}{2016}).)
\end{quote}

Early in the section,Humberstone notes that, in fact, there are strict
logical intermediaries between ``\(A\) is obligatory'' and ``\(A\) is
permitted'' or indeed between any sentences \(A\) and B of decreasing
logical strength in any reasonably well behaved modal logic, whether
their connectives are given a deontic reading or any other. For one can
simply take as intermediary, any formula \(A\vee (B\wedge p)\) where
\emph{p} is a sentence letter that does not occur in \(A\) or \(B\). In
this section we wish to point out that a consequence of this observation
is that there is a sense in which all modal logics meeting certain
minimal requirements are the same.

We identify a ``logic'' with a many-one deducibility relation on
formulas satisfying the usual structural conditions. (So the logic with
all tautologies as axioms and no rules of inference is distinct from a
similar logic with \emph{modus ponens} as a rule of inference.) The
minimal requirements are just that logics are classically based and
substitution-closed.By classically based we mean that their languages
contain the Booleanconnectives (or at least some truth-functionally
complete subset thereof) and that these behave classically under the
deducibility relation, so that, for example \(A \wedge B \vdash C\) iff
\(A \vdash (B \rightarrow C)\).\footnote{In the terminology of
  Humberstone (\citeproc{ref-Humberstone.2011}{2011, 62}), these logics
  are \(\#\)-classical for every Boolean connective \(\#\). Humberstone
  (\citeproc{ref-Humberstone.2011}{2011}) spells out necessary and
  sufficient conditions that are omitted here. The term \emph{classical}
  which might have been preferred over \emph{classically based} is
  avoided because \emph{classical modal logic} is sometimes used for
  other purposes.} If a logic is classical, we may safely identify it
with the set of its theorems, knowing that these will determine the
deducibility relation. By substitution-closed we mean that
\(A_1^\prime,{\ldots},A_n^\prime \vdash B^\prime\) whenever
\(A_1^\prime,{\ldots},A_n^\prime\) and \(B^\prime\) are the result of
uniformly replacing sentence letters by formulas in \(A_1,{\ldots},A_n\)
and \(B\) such that \(A_1,{\ldots},A_n \vdash B\). The requirement that
the logic is classically based ensures that \(A\vee (B\wedge p)\) is a
logical intermediary between \(A\) and \(B\). The requirement that it is
substitution-closed implies that it is a strict intermediary. For if it
provably implied \(A\), then its substitution instance
\(A\vee (B\wedge B)\) would provably imply \(A\), and \(A\) would be
provably equivalent to \(B\). And if it was provably implied by \(B\),
then \(A\vee (B\wedge A)\) would be provably implied by \(B\) and again
\(A\) would be equivalent to \(B\). By saying that these logics are the
same we mean something close to what is sometimes called
\emph{translationally equivalent}.\footnote{A paradigm case is the
  relation between classical propositional logic formulated with
  \(\neg\) and ∧ and that formulated with the Sheffer stroke. The notion
  has been defined in a number of ways, which are nicely surveyed in
  chapter 5 of French (\citeproc{ref-French2010}{2010}). The definitions
  that follow are close to those in Kuhn
  (\citeproc{ref-Kuhn.1978}{1978}). Other definitions may diverge when
  certain Boolean connectives are absent or fail to behave classically,
  but our emphasis here is on logics for which they coincide.} Let us
say that logics \(L_1\) and \(L_2\) are \emph{weakly translationally
equivalent} if there is a map \(s\colon A{\mapsto}A^s\) from formulas of
\(L_1\) to formulas of \(L_2\) and a map \(t\colon C{\mapsto}C^t\) from
formulas of \(L_2\) to formulas of \(L_1\) satisfying the following
conditions (where \(\vdash_i\) is \(\vdash_{L_i}\) for \(i=1,2\)):

\begin{enumerate}
\def\labelenumi{\arabic{enumi}.}
\tightlist
\item
  \(A_1,{\ldots},A_n \vdash_1 B\) implies
  \({A_1}\!^s,{\ldots},{A_n}\!^s \vdash_2 B^s\)
\item
  \(C_1,{\ldots},C_m \vdash_2 D\) implies
  \({C_1}^t,{\ldots},{C_m}^t \vdash_1 D^t\)
\item
  \(A ~_1\!{\dashv}{\vdash}\!_1 (A^s)^t\)
\item
  \(C ~_2\!{\dashv}{\vdash}\!_2 ~ (C^t)^s\).
\end{enumerate}

\emph{s} and \emph{t} are to be thought of as translations between the
logics. If \(L_1\) and \(L_2\) are weakly translationally equivalent
then the word \emph{implies} in i and ii can be strengthened to \emph{if
and only if}, so that \emph{s} and \emph{t} are faithful embeddings. For
example, by condition ii, \({A_1}^s,{\ldots},{A_n}^s \vdash_2 B^s\)
implies \(({A_1}^s)^t,{\ldots},({A_n}^s)^t \vdash_1({B^s})^t\), and so,
by condition iii \(A_1,{\ldots},A_n \vdash_1 B\). But the strengthened
versions of i and ii still do not imply iii and iv. (See, for example,
French (\citeproc{ref-French2010}{2010}) pp 111-124.) \(L_1\) and
\(L_2\) are said to be \emph{translationally equivalent} if the
translations securing their weak equivalence meet some additional
requirement, commonly that they be compositional, i.e., that they be
maps \(f\) such that for every n-ary connective \(\#\) in the source
language there is formula schema \(\sigma\) of the target language with
\(n\) schematic variables such that
\(f(\#A_1{\ldots}A_n)=\sigma(f(A_1),{\ldots},f(A_n))\). If we are
interested in what can be said within a logic rather than the structure
of the formulas saying it, however, the restriction to compositional
translations seems unwarranted. A translation can be ``sentence by
sentence'' rather than ``symbol by symbol.'' It is plausible to take
formulas to be saying the same thing in a logic when they are provably
equivalent. In that case the structure of the things that can be said in
a logic is given by its \emph{Lindenbaum lattice}. By this, we mean the
structure \((X,\le )\) where the members of X are the equivalence
classes \([A]_L\) of formulas \(A\) under the relation
\(_L\!{\dashv}{\vdash}\!_L~\) and \([A]_L\le [B]_L\) iff
\(A\vdash \!_LB\). In that case, we may say that two logics are the same
with regards to what they can say if their Lindenbaum lattices are
isomorphic. It is not difficult to show that under conditions of
interest here, this condition coincides with weak translational
equivalence.

\textbf{Theorem 3}

\begin{enumerate}
\def\labelenumi{\roman{enumi}.}
\tightlist
\item
  If* \(L_1\) and \(L_2\) are weakly translationally equivalent then
  they have isomorphic Lindenbaum lattices.
\item
  If \(L_1\) and \(L_2\) are classically based and they have isomorphic
  Lindenbaum lattices then they are weakly translationally equivalent.
\end{enumerate}

\textbf{Proof}\\
Here and below, we drop the subscripts from the brackets and turnstile
symbols, when the logic is intended is clear. To prove i, suppose
\emph{s} and \emph{t} satisfy conditions i-iv defining weak
translational equivalence. Let \(\Phi([A])=[A^s]\). We show that
\(\Phi\) is an isomorphism. i)\(\Phi\) is well defined. Suppose
\([A]=[B]\). Then \(A{\dashv}{\vdash}B\). By condition i, this implies
\(A^s{\dashv}{\vdash}B^s\). Hence \([A^s] = [B^s]\) and
\(\Phi([A])=\Phi([B])\), as required. ii)\(\Phi\) is 1-1. Suppose
\(\Phi([A])=\Phi([B])\). Then \([A^s]=[B^s]\) and so
\(A^s{\dashv}{\vdash}B^s\). By condition ii,
\((A^s)^t{\dashv}{\vdash}(B^s)^t\). By condition iii,
\(A{\dashv}{\vdash}B,\) and so \([A]=[B]\), as required. iii)\(\Phi\) is
onto. Take any \(C\) in the language of \(L_2\). By condition iv,
\(C{\dashv}{\vdash}(C^t)^s\). Hence \([C] = [(C^t)^s]\). Therefore
\([C]=\Phi([C^t])\), and \([C]\) is in the range of \(\Phi\), as
required.

The proof of ii is facilitated by a lemma. Let us say that a translation
\(f\colon A{\mapsto}A^f\) \emph{conforms to falsum} if
\(\bot^f {\dashv}{\vdash} \bot\); \emph{to negation} if
\((\neg A)^f {\dashv}{\vdash} \neg A^f\); \emph{to conjunction} if
\((A\wedge B)^f {\dashv}{\vdash}A^f \wedge B^f\) and similarly for all
the other Boolean connectives. To prove part ii of the theorem, we use
only that \emph{s} and \emph{t} conform to conjunction, but we take the
opportunity to prove something more general.

\textbf{Lemma 1}.\\
Suppose \emph{s} and \emph{t} are translations securing the weak
equivalence of classically based modal logics \(L_1\) and \(L_2\). Then
\emph{s} and \emph{t} conform to all the Boolean connectives.

\textbf{Proof}\\
Suppose \emph{s} and \emph{t} satisfy the hypothesis of the lemma. We
show that \emph{s} and \emph{t} conform to falsum (i) and the
conditional (ii) and that it follows that they conform to all the other
Boolean connectives (iii).

(i): (We include the subscripts for clarity here.) Since \(L_1\) and
\(L_2\) are classically based, \(\bot\vdash_2 \bot^s\) and
\(\bot \vdash_1 \bot^t\). From the second of these it follows that
\(\bot^s \vdash_2 (\bot^t)^s\), and therefore that
\(\bot^s \vdash_2 \bot\). Hence \(\bot_2\!{\dashv}{\vdash}\!_2 \bot^s\)
and so \emph{s} conforms to \(\bot\). The proof that \emph{t} conforms
to \(\bot\) is similar.

(ii): Since \(L_1\) is classically based,
\((A\rightarrow B),A \vdash B\). By condition i of weak translational
equivalence \((A\rightarrow B)^s, A^s \vdash B^s\). Since \(L_2\) is
classically based, \((A\rightarrow B)^s \vdash (A^s\rightarrow B^s)\).
Similarly, since \(L_2\) is classically based,
\((A\rightarrow B)^t,A^t \vdash B^t\). By condition (ii) of weak
translational equivalence,
\((A^s\rightarrow B^s)^t,(A^s)^t \vdash (B^s)^t\). By condition iii,
\((A^s)^t {\dashv}{\vdash}A\) and \((B^s)^t {\dashv}{\vdash}B\), and so
\((A^s\rightarrow B^s)^t,A \vdash B\). Since \(L_1\) is classically
based, \((A^s\rightarrow B^s)^t \vdash A\rightarrow B\). By condition i,
\(((A^s\rightarrow B^s)^t)^s \vdash (A\rightarrow B)^s\), which implies
by condition iv that \((A^s\rightarrow B^s) \vdash (A\rightarrow B)^s\).
We have now shown that
\((A\rightarrow B)^s{\dashv}{\vdash}(A^s\rightarrow B^s)\), and so
\emph{s} conforms to →. The proof that \emph{t} conforms to → is
similar.

(iii): It can be shown that \emph{s} and \emph{t} conform to each of the
remaining Boolean connectives by expressing them in terms of falsum and
the conditional. For example
\((\neg A)^s {\dashv}{\vdash} (A\rightarrow \bot)^s\). Since \emph{s}
conforms to the conditional and falsum,
\((\neg A)^s {\dashv}{\vdash} A^s \rightarrow \bot\). Since the logics
are classically based, \((\neg A)^s {\dashv}{\vdash} \neg A^s\). The
other cases are similar.~◻

We proceed to the proof of part ii of the theorem. Suppose \(\Phi\) is
an isomorphism between the Lindenbaum lattices \((X_1,\le _1)\) and
\((X_2,\le _2)\) of \(L_1\) and \(L_2\). Let \emph{s} map each formula
\(A\) in the language of \(L_1\) to any member of \(\Phi([A])\) and let
\emph{t} map each formula \(C\) in the language of \(L_2\) to any member
of \(\Phi^{-1}([C])\). We show that \emph{s} and \emph{t} meet the four
conditions for weak translational equivalence. For i, suppose
\(A_1,{\ldots},A_n \vdash B\). Since \(L_1\) is classically based
\(A_1\wedge {\ldots}\wedge A_n \vdash B\), and so
\([A_1\wedge {\ldots}\wedge A_n] \le [B]\). Since \(\Phi\) is an
isomorphism, \(\Phi([A_1\wedge {\ldots}\wedge A_n]) \le \Phi([B])\), and
so \((A_1\wedge {\ldots}\wedge A_n)^s \vdash B^s\). Since \emph{s}
conforms to conjunction,
\(({A_1}^s\wedge {\ldots}\wedge {A_n}^s) \vdash B^s\). Since \(L_2\) is
classically based, \({A_1}^s,{\ldots},{A_n}^s \vdash B^s\). The proof
that condition ii is satisfied is similar. For conditions iii and iv
note that, since \(A^s \in \Phi([A]), [A^s]= \Phi([A])\). Similarly,
\([C^t]= \Phi^{-1}([C^t])\). Together these two identities imply
\([(A^s)^t] = \Phi^{-1}(\Phi([A])\) and
\([(C^t)^s]= \Phi(\Phi^{-1}[C])\). It follows that \([(A^s)^t] = [A]\)
and \([(C^t)^s]=[C]\) and therefore that conditions iii and iv are
satisfied.~◻

Since the modal logics under consideration are classically based, their
Lindenbaum lattices are Boolean algebras, i.e., we can define from
\(\le\) operations ∧,∨, and \(\neg\) satisfying the usual Boolean
axioms. Humberstone's observation that these logics provide strict
intermediaries implies that they are dense. Using \(X{<}Y\) to mean
\(X\le Y\) and not \(Y\le X\), we have that \([A]{<}[B]\) implies that
there is some element \([I]\) such that \([A]{<}[I]{<}[B]\). But a
Boolean algebra is dense iff it is atomless. (If the algebra is dense
and \(0{<}X\) then there is is an element \(I\), that precedes \emph{x},
in the sense that \(0{<}I{<}X\), so \emph{x} cannot be an atom.
Conversely if the algebra has no atoms, then there is an intermediary
\(I\) between \(0\) and \emph{x}, so if \(X{<}Y\), \(X\vee (I\wedge Y)\)
is an intermediary between \emph{x} and \(Y\).) A basic theorem of
Boolean algebra states that the theory of atomless Boolean algebras is
\(\aleph_0\)-categorical, i.e., that any two countable atomless Boolean
algebras are isomorphic.It follows that any two reasonably well-behaved
modal logics are weakly translationally equivalent. There is a sense in
which adding \(\Box\) or any other non-Boolean connectives to the
language of propositional logic and axioms and rules of derivation to
the usual rules for classical logic adds nothing to what can be said.
This observation contrasts starkly with what happens when translations
are required to be compositional. In Pelletier and Urquhart
(\citeproc{ref-PelletierAndUrquhart}{2003}), it is shown that if
well-behaved modal logics \(L_1\) and \(L_2\) are are translationally
equivalent, then, for any number \(n\), the number of Kripke frames with
\(n\) worlds validating \(L_1\) is the same as the number validating
\(L_2\). It follows if two logics have the finite frame property (as all
the most familiar modal logics do) and one is a sublogic of the other,
they cannot be translationally equivalent. The observation here
demonstrates the importance for the Pelletier/Urquhart result of the
requirement that the translations be compositional.

We do know that the translations between classically based modal logics
conform to the Boolean connectives. This allows us to sharpen the result
slightly in the direction of Pelletier/Urquhart.

\textbf{Theorem 4}\\
Suppose \(L_1\) and \(L_2\) are classically based modal logics. Then
\(L_1\) and \(L_2\) are weakly translationally equivalent by way of
translations \(s^*\) and \(t^*\) that preserve the Boolean connectives.

\textbf{Proof}\\
Since the logics are classically based they have isomorphic Lindenbaum
lattices. By the previous theorem they are weakly translationally
equivalent. Let \emph{s} and \emph{t} be the translations securing this
similarity. We define \(s^*\) and \(t^*\) by cases:

\begin{enumerate}
\def\labelenumi{\arabic{enumi}.}
\tightlist
\item
  \(s^*(A)=A^s\) if \(A\) is a sentence letter or
  \(A = {\#}A_1{\ldots}A_n\) for \({\#}\) an \(n\)-ary non-Boolean
  connective
\item
  \(s^{*}(\bot)=\bot\)
\item
  \(s^{*}(\neg A)= \neg A^{s^{*}}\)
\item
  \(s^{*}(A{\#}B) = A^{s^{*}}\!{\#} B^{s^{*}}\) if \({\#}\) is
  \(\wedge ,\vee ,\rightarrow\) or \(\leftrightarrow\).
\end{enumerate}

The clauses for \(t^*\) are similar.

Induction using sentence letters and formulas \({\#}A_1{\ldots}A_n\) for
\({\#}\) non-Boolean as a base and appeal to the conformity property
establishes that \(s^*\!(A){\dashv}{\vdash}A^s\) and
\(t^*(C){\dashv}{\vdash}C^t\). It follows that \(s^*\) and \(t^*\),
which preserve the Boolean connectives, also satisfy the conditions for
weak translational equivalence.~◻

Note, however, that the result of Pelletier and Urquhart
(\citeproc{ref-PelletierAndUrquhart}{2003}) ensures that \(s^*\) and
\(t^*\) are not in general compositional. So one should not presume, for
example, that the \(s^*\)-translation of \(\Box (p\wedge q)\) is any
function of the \(s^*\) translations of \emph{p} and \(q\).

\section{\texorpdfstring{S4 \(\oplus\) 5\(^\prime\) =
S4\(\oplus\)F}{S4 \textbackslash oplus 5\^{}\textbackslash prime = S4\textbackslash oplusF}}\label{s4-oplus-5prime-s4oplusf}

Consider the following two axioms:

\begin{itemize}
\tightlist
\item
  \textbf{5}′:
  \((p \wedge \neg \Box p \wedge \Box (p \vee \Box (p \rightarrow \Box p))) \rightarrow \Box \neg \Box p\)
\item
  \textbf{F}:
  \((p\wedge \Diamond\Box q) \rightarrow \Box (\Diamond p\vee q)\)
\end{itemize}

These emerge in Humberstone's survey (402-420) of the logical terrain
between \textbf{S4} and \textbf{S5} for plausible epistemic logics.
\textbf{F} figures prominently in Stalnaker
(\citeproc{ref-Stalnaker2006-STAOLO}{2006}) and \textbf{5}′ in Voorbraak
(\citeproc{ref-Voorbraak.1991}{1991}). Humberstone (410) asks whether it
is possible to derive \textbf{F} from \textbf{S4} and \textbf{5}′. The
point of this section is to argue that it is. We'll also show something
that is already clear in Humberstone's text, which is that \textbf{5}′
can be proven in \textbf{S4F}, so \textbf{S4F} = \textbf{S45}′.
Humberstone in fact shows something considerably stronger, namely that
\textbf{S4F} is complete with respect to the class of transitive,
reflexive, semi-Euclidean frames, and \textbf{5}′ is sound with respect
to the class of those frames. (The semi-Euclidean frames are those which
satisfy \(\forall xyz((xRy \wedge xRz) \rightarrow (yRz \vee zRx))\).
The term \emph{semi-Euclidean} is taken from Voorbraak
(\citeproc{ref-Voorbraak.1991}{1991}).) From these results it follows
there must be some proof of \textbf{5}′ in \textbf{S4F}. But the status
of \textbf{F} in \textbf{S45}′ was an open question.

It will be convenient to label three additional formulas that appear in
the course of our derivation of F:

\begin{itemize}
\tightlist
\item
  \textbf{5}\(^{\prime\prime}\!: (p \wedge \Diamond\neg p \wedge \Diamond\Box p) \rightarrow \Diamond(\neg p \wedge \Diamond(p\wedge \neg \Box p))\)
\item
  \textbf{A}:
  \((\Diamond{p}\vee \Box q) \wedge \Diamond\neg (\Diamond{p}\vee \Box q) \wedge \Diamond\Box (\Diamond{p}\vee \Box q)\)
\item
  \textbf{B}:
  \(\neg (\Diamond{p}\vee \Box q) \wedge \Diamond((\Diamond{p}\vee \Box q) \wedge \neg \Box (\Diamond{p}\vee \Box q))\)
\end{itemize}

We will show the following:

\textbf{Theorem 5}

\begin{quote}
\begin{enumerate}
\def\labelenumi{\roman{enumi}.}
\tightlist
\item
  \textbf{5}\(^{\prime}~_{\text{K}}\!{\dashv}{\vdash}_\text{K}~\textbf{5}^{\prime\prime}\)
\item
  \(\vdash_\text{KT4} ~ \neg \textbf{F}\rightarrow \textbf{A}\)
\item
  \(\vdash_{\text{5}^{\prime\prime}} ~\textbf{A}\rightarrow \Diamond\textbf{B}\)
\item
  \(\vdash_\text{S4} ~\neg \Diamond\textbf{B}\)
\end{enumerate}
\end{quote}

i allows us to work within \textbf{S45}\(^{\prime\prime}\) rather than
\textbf{S45}′, and ii, iii, iv constitute a \emph{reductio} proof of
\textbf{F} within that system. It should be noted that Humberstone uses
\textbf{5}′ as a label for the schema corresponding to the axiom given
here. We work within a natural deduction system that allows us to use a
rule of truth-functional consequence under assumptions, and to apply
rules of necessitation and uniform substitution to formulas that are not
under any assumptions.

\textbf{Proof} To prove i we note the following chain of K-equivalent
formulas:

\begin{enumerate}
\def\labelenumi{\arabic{enumi}.}
\tightlist
\item
  \((p \wedge \neg \Box p \wedge \Box (p \vee \Box (p \rightarrow \Box p))) \rightarrow \Box \neg \Box p~~~\)(=\textbf{5}′)
\item
  \((p \wedge \neg \Box p \wedge \neg \Box \neg \Box p) \rightarrow \neg \Box (p \vee \Box (p \rightarrow \Box p))\)
\item
  \((p \wedge \Diamond\neg p \wedge \Diamond\Box p) \rightarrow \Diamond\neg (p \vee \Box (p \rightarrow \Box p))\)
\item
  \((p \wedge \Diamond\neg p \wedge \Diamond\Box p) \rightarrow \Diamond(\neg p \wedge \neg \Box (p \rightarrow \Box p))\)
\item
  \((p \wedge \Diamond\neg p \wedge \Diamond\Box p) \rightarrow \Diamond(\neg p \wedge \Diamond(p \wedge \neg \Box p))~~~\)(=\textbf{5}\(^{\prime\prime}\))
\end{enumerate}

A derivation sketch establishing ii is given below. We make free use of
\textbf{K} and truth functional logic, but we note steps that use
\textbf{T} or \textbf{4}.

\begin{enumerate}
\def\labelenumi{\arabic{enumi}.}
\tightlist
\item
  \(\neg ((p\wedge \Diamond\Box q) \rightarrow \Box (\Diamond{p}\vee q))\)(Assumption
  \(\neg\)\textbf{F})
\item
  \emph{p} (from 1)
\item
  \(\Diamond{p}\vee \Box q\)(from 2 using \textbf{T})
\item
  \(\Diamond(\neg \Diamond{p}\wedge \neg q)\) (from 1)
\item
  \(\Diamond(\neg \Diamond{p}\wedge \neg \Box q)\) (from 4 using
  \textbf{T})
\item
  \(\Diamond\neg (\Diamond{p}\vee \Box q)\)(from 5)
\item
  \(\Diamond\Box q\)(from 1)
\item
  \(\Diamond\Box \Box q\)(from 7 using \textbf{4})
\item
  \(\Diamond\Box (\Diamond{p}\vee \Box q)\) (from 8)
\item
  \((\Diamond{p}\vee \Box q) \wedge \Diamond\neg (\Diamond{p}\vee \Box q)\wedge \Diamond\Box (\Diamond{p}\vee \Box q)\)(from
  3,6,9)
\item
  \(\neg \textbf{F} \rightarrow \textbf{A}\)(from 1-10)
\end{enumerate}

For iii note that a substitution of \(\Diamond{p}\vee \Box q\) for
\emph{p} in \textbf{5}\(^{\prime\prime}\) results in the formula
\textbf{A}\(\rightarrow \Diamond\textbf{B}\). Finally, we establish iv
by the derivation sketch below.

\begin{enumerate}
\def\labelenumi{\arabic{enumi}.}
\tightlist
\item
  \(\neg (\Diamond{p}\vee \Box q) \wedge \Diamond((\Diamond{p}\vee \Box q) \wedge \neg \Box (\Diamond{p}\vee \Box q))\)\\
  (Assumption \textbf{B})
\item
  \(\neg (\Diamond{p}\vee \Box q)\) (from 1)
\item
  \(\Box \neg p\) (from 2)
\item
  \(\Box \Box \neg p\) (from 3 using \textbf{4})
\item
  \(\Diamond((\Diamond{p}\vee \Box q) \wedge \neg \Box (\Diamond{p}\vee \Box q))\)
  (from 1)
\item
  \(\Diamond(\Box \neg p \wedge (\Diamond{p}\vee \Box q) \wedge \neg \Box (\Diamond{p}\vee \Box q))\)
  (from 4,5)
\item
  \(\Diamond(\Box \neg p \wedge \Box q \wedge \neg \Box (\Diamond{p}\vee \Box q))\)
  (from 6)
\item
  \(\Diamond(\Box \neg p \wedge \Box \Box q \wedge \neg \Box (\Diamond{p}\vee \Box q))\)
  (from 7 using \textbf{4})
\item
  \(\Diamond(\Box \neg p \wedge \Box \Box q \wedge \neg \Box \Box q)\)
  (from 8)
\item
  \(\neg\)\textbf{B} (from 1-9 by \emph{reductio})
\item
  \(\Box \neg\)\textbf{B} (from 10 by necessitation)
\item
  \(\neg \Diamond\)\textbf{B} (from 11)
\end{enumerate}

As we mentioned above, Humberstone shows that there must be a proof of
\textbf{5}′ in \textbf{S4F}. For the sake of symmetry, we sketch that
proof. As it turns out, only \textbf{KF} is required, which we could not
have known from Humberstone's completeness result.

\begin{enumerate}
\def\labelenumi{\arabic{enumi}.}
\tightlist
\item
  \((p\wedge \Diamond\Box q) \rightarrow \Box (\Diamond{p}\vee q)\)(\textbf{F})
\item
  \((\neg (p\rightarrow \Box p) \wedge \Diamond\Box p) \rightarrow \Box (\Diamond\neg (p\rightarrow \Box p)\vee p)\)(from
  1 by substitution)
\item
  \(\neg ((p \wedge \neg \Box p \wedge \Box (p \vee \Box (p \rightarrow \Box p))) \rightarrow \Box \neg \Box p)\)(Assumption
  \(\neg \textbf{5}^\prime)\)
\item
  \(p \wedge \neg \Box p \wedge \Box (p \vee \Box (p \rightarrow \Box p)) \wedge \Diamond\Box p\)(from
  3)
\item
  \(\neg (p\rightarrow \Box p) \wedge \Diamond\Box p\)(from 4)
\item
  \(\Box (\Diamond\neg (p\rightarrow \Box p)\vee p)\)(from 2,5)
\item
  \(\Diamond\neg p\)(from 4)
\item
  \(\Diamond(\neg p \wedge \Diamond\neg (p\rightarrow \Box p)\vee p)\)(from
  6,7)
\item
  \(\Diamond(\neg p \wedge \Diamond\neg (p\rightarrow \Box p))\)(from 8)
\item
  \(\Box (p \vee \Box (p \rightarrow \Box p))\)(from 4)
\item
  \(\Diamond(\neg (p\vee \Box (p\rightarrow \Box p)) \wedge (p\vee \Box (p\wedge \Box p)))\)(from
  9, 10)
\item
  \(\textbf{5}^\prime\)(from 3-11 by \emph{reductio})
\end{enumerate}

\(\blacksquare\)

\section{Ain't Necessarily So}\label{aint-necessarily-so}

Humberstone's ``logic of coming about'' (452-469) adds to the language
of classical sentential logic a modal operator \(\mathbf{D}\).
\(\mathbf{D}A\) is to be read as it comes about that \(A\) and
understood as being something like Nuel Belnap's a \emph{sees to it
that} \(A\), except that it abstracts from the idea of agency. Models
are triples \(\langle U,f,V\rangle\) where \(U\) and \(V\) are sets and
valuations of the kind familiar from modal logic and \(f\) is a unary
function from \(U\) to \(U\). The truth definition has the usual clauses
for the classical connectives and the additional clause:

\begin{itemize}
\tightlist
\item
  \(\langle U,f,V\rangle \models _x \mathbf{D}A\) iff
  \(\langle U,f,V\!\rangle \models _x A\) and not
  \(\langle U,f,V\!\rangle \models _{f(x)} A\).
\end{itemize}

Truth in the model is truth at all \(u{\in}U\), and validity is is truth
in all models. Among the valid schemas and validity-preserving rules
that he draws attention to are the following:

\begin{itemize}
\tightlist
\item
  D0: Substitution instances of tautologies
\item
  D1: \(\mathbf{D}A\rightarrow A\)
\item
  D2: \((A\wedge\mathbf{D}B) \rightarrow \mathbf{D}(A\wedge B)\)
\item
  D3:
  \(\mathbf{D}(A\wedge B) \rightarrow (\mathbf{D}A\vee \mathbf{D}B)\)
\item
  MP: \(A\rightarrow B, A ~~/~~ B\)
\item
  RD\(_{m,n}\):
  \((B_1 \wedge {\ldots}\wedge B_m) \rightarrow (A_1\vee {\ldots}\vee A_n) ~~/ \\\)\((B_1 \wedge {\ldots}\wedge B_m) \rightarrow ((\mathbf{D}A_1\wedge {\ldots}\wedge \mathbf{D}A_n) \rightarrow (\mathbf{D}B_1 \vee {\ldots}\vee \mathbf{D}B_m))\)
\end{itemize}

The last rule schema is intended to include the cases \(m=0\) and
\(n=0\) with the usual stipulation that an empty conjunction is \(\top\)
and an empty disjunction is \(\bot\). The reader is asked to show that
D2 and D3 are provable from the remaining schemas as an exercise and the
valid formulas are then shown to be axiomatized by D0, D1, MP and all
the RD rules. In this section we consider two additional schemas.

\begin{itemize}
\tightlist
\item
  D4:
  \((\mathbf{D}A\wedge \mathbf{D}B) \rightarrow \mathbf{D}(A\vee B)\)
\item
  D5:\(\neg \mathbf{D}\top\)
\end{itemize}

We show that the valid formulas of coming-about logic are axiomatized by
D0, D1, D3, D4, D5 MP and RD\(_{1,1}\). Since we can always add a rule
of substitution while replacing the schematic variables in D1, D3, D4 by
sentence letters and replacing D0 by a finite set of axioms for
sentential logic, this shows that Humberstone's infinite axiomatization
can be replaced by a simple finite one.

D4 plays a special role among the axioms and rules considered. Suppose
\(\mathbf{D}{A}\) is interpreted as \emph{it is contingently true that}
\(A\) (or, as the section head suggests, that \(A\), while true, is not
necessarily so). More precisely, replace the function \(f\) in
Humberstone's models by an accessibility of the usual kind and his truth
clause for \(\mathbf{D}\) by the following:

\begin{itemize}
\tightlist
\item
  CT: \(\langle U,R,V\!\rangle \models _x \mathbf{D}A\) iff
  \(\langle U,R,V\!\rangle \models _x A\) and, for some \(y\) such that
  \(xRy\), not \(\langle U,R,V\!\rangle \models _y A\)
\end{itemize}

It is easy to check that D0, D1, D2, D3 and D5 all remain valid and MP
and RD\(_{1,1}\) still preserve validity. But when \(R\) is not a
function then D4 can be falsified: Let
\(U=\{w,u,v\}, R=\{(w,u),(w,v)\}, V(p)=\{w,u\}\) and \(V(q)=\{w,v\}\).
So D4 is independent of the five axioms and two rules just given. In
fact, as we shall show, these axioms and rules are sufficient to
axiomatize the \emph{contingently true} operator under the
interpretation CT.

First, however, we turn to the connection between the formula schemas
and the rule schema RD\(_{m,n}\) and the proof that the logic of coming
about is finitely axiomatizable. Note that any logic containing D0 and
closed under MP is closed under truth-functional consequence (TFC). This
facilitates the proof of the following:

\textbf{Claim 1}\\
In the presence of D0, D1 and MP: D2 is provable from RD\(_{1,1}\), D3
is provable from RD\(_{2,1}\), D4 is provable from RD\(_{1,2}\) and D5
is provable from RD\(_{0,1}\)

\textbf{Proof}\\
The required derivations are sketched below.

\begin{enumerate}
\def\labelenumi{\arabic{enumi}.}
\tightlist
\item
  \((A\wedge B)\rightarrow B\) by D0
\item
  \((A\wedge B)\rightarrow (\mathbf{D}B\rightarrow \mathbf{D}(A\wedge B))\)from
  1 by RD\(_{1,1}\)
\item
  \((A\wedge B\wedge \mathbf{D}B)\rightarrow \mathbf{D}(A\wedge B))\)from
  2 by TFC
\item
  \((A\wedge \mathbf{D}B)\rightarrow \mathbf{D}(A\wedge B))\)from 3 and
  D1 by TFC
\end{enumerate}

::: \{.content-visible when-format=``docx''

:::

\begin{enumerate}
\def\labelenumi{\arabic{enumi}.}
\tightlist
\item
  \((A\wedge B)\rightarrow (A\wedge B)\) by D0
\item
  \((A\wedge B)\rightarrow (\mathbf{D}(A\wedge B)\rightarrow (\mathbf{D}A\vee \mathbf{D}B))\)from
  1 by RD\(_{2,1}\)
\item
  \(\mathbf{D}(A\wedge B)\rightarrow (\mathbf{D}A\vee \mathbf{D}B)\)from
  2 and D1 by TFC
\end{enumerate}

::: \{.content-visible when-format=``docx''

:::

\begin{enumerate}
\def\labelenumi{\arabic{enumi}.}
\tightlist
\item
  \((A\wedge B)\rightarrow (A\vee B)\) by D0
\item
  \((A\wedge B)\rightarrow ((\mathbf{D}A\wedge \mathbf{D}B)\rightarrow \mathbf{D}(A\vee B))\)
  from 1 by RD\(_{1,2}\)
\item
  \(\mathbf{D}(A\wedge B)\rightarrow \mathbf{D}(A\vee B)\)from 2 and D1
  by TFC
\end{enumerate}

::: \{.content-visible when-format=``docx''

:::

\begin{enumerate}
\def\labelenumi{\arabic{enumi}.}
\tightlist
\item
  \(\top \rightarrow \top\)by D0
\item
  \(\top \rightarrow (D\top \rightarrow \bot)\)from 1 by RD\(_{0,1}\)
\item
  \(\neg D\top\)from 2 by TFC
\end{enumerate}

Since Humberstone has already shown that D0 and D1 are valid and that
modus ponens and every instance of RD\(_{m,n}\) preserves validity, the
claim is sufficient to show that our new axiom system is sound. To prove
sufficiency, it is sufficient to show that, for all \(m,n\!\ge\!0\),
RD\(_{m,n}\) is derivable in the new system. To this end, notice first
that D3 and D4 generalize, i.e., if \(\vdash\) indicates provability in
the new axiom system then:

\begin{itemize}
\tightlist
\item
  D3*: For all
  \(n{\ge}1, \vdash \mathbf{D}(A_1\wedge {\ldots}\wedge A_n) \rightarrow (\mathbf{D}A_1\vee {\ldots}\vee \mathbf{D}A_n)\),
  and
\item
  D4*: For all
  \(n{\ge}1, \vdash (\mathbf{D}A_1\wedge {\ldots}\wedge \mathbf{D}A_n) \rightarrow \mathbf{D}(A_1\vee {\ldots}\vee A_n)\).
\end{itemize}

For these claims to be sensible without grouping conjuncts and disjuncts
our logic must allow replacement of truth-functional equivalents.
Humberstone's proof (453) of the claim that his logic satisfies the
stronger property of being ``congruential,'' i.e., closed under
replacement of provable equivalents uses only D0 and RD\(_{1,1}\) so we
can help ourselves to this result. The claims can then be proved by
induction on \(n\). In each case the basis case follows from D0. The
inductive step for D3* uses D3 and that for D4* uses D4.

This allows us to show that RD\(_{m,n}\) is derivable for all positive
\(m\) and \(n\): Suppose
\(\vdash (B_1 \wedge {\ldots}\wedge B_m) \rightarrow (A_1\vee {\ldots}\vee A_n)\).
Then by RD\(_{1,1}\),\\
\(\vdash (B_1 \wedge {\ldots}\wedge B_m) \rightarrow (\mathbf{D}(A_1\vee {\ldots}\vee A_n) \rightarrow \mathbf{D}(B_1 \wedge {\ldots}\wedge B_m))\).
By D3* and TFC,
\(\vdash (B_1 \wedge {\ldots}\wedge B_m) \rightarrow (\mathbf{D}(A_1\vee {\ldots}\vee An) \rightarrow (\mathbf{D}B_1 \vee {\ldots}\vee \mathbf{D}B_m))\).
By D4* and TFC,
\(\vdash (B_1 \wedge {\ldots}\wedge B_m) \rightarrow ((\mathbf{D}A_1\wedge {\ldots}\wedge \mathbf{D}An) \rightarrow (\mathbf{D}B_1 \vee {\ldots}\vee \mathbf{D}B_m))\).

It remains only to check the cases \(m{=}0\) and \(n{=}0\). But for all
\(m\), RD\(_{m,0}\) is a consequence of TFC: if
\((B_1\wedge {\ldots}\wedge B_m)\rightarrow \bot\) is provable then so
is any formula with \(B_1\wedge {\ldots}\wedge B_m\) as antecedent,
including the consequence of RD\(_{m,0}\).

For the case \(m=0\) we will need D4* and D5. Suppose
\(\vdash \top \rightarrow (A_1\vee {\ldots}\vee A_n)\). Then
\((A_1\vee {\ldots}\vee A_n)\) is provably equivalent to \(\top\). Since
\(\vdash \neg D\top\) and our logic is congruential it follows that
\(\vdash \neg \mathbf{D}(A_1\vee {\ldots}\vee A_n)\). By D4,
\(\vdash \neg (\mathbf{D}A_1\wedge {\ldots}\wedge \mathbf{D}A_n)\). By
TFC,
\(\vdash \top \rightarrow ( (\mathbf{D}A_1\wedge {\ldots}\wedge \mathbf{D}An) \rightarrow \bot)\),
and so RD\(_{0,n}\) is derivable.\(\blacksquare\)

Our proof that Humberstone's logic of coming about has a simple, finite
axiomatization is complete and so we turn our attention to the
axiomatization of the logic of contingent truth.

\textbf{Theorem 6}\\
The axioms D0, D1, D3, D4, D5 and rules MP and RD\(_{1,1}\) provide a
complete axiomatization of the logic of contingently true under the
interpretation CT.

\emph{Proof}\\
Soundness was observed above so it is sufficient to prove sufficiency.
This can be done by constructing a canonical model out of maximally
consistent sets in a familiar way. Let \(M^c=(W^c\!, R^c\!,V^c)\), where
\(W^c\) is the set of all maximally consistent formulas,
\(V^c(p)=\{w{\in}W^c\!: p{\in}w\}\), and \(xR^cy\) iff \(A{\in}y\)
whenever both \(A{\in}x\) and \(\mathbf{D}A{\notin}x\).

\textbf{Lemma 2} (Witness lemma)\\
\emph{If} \(\mathbf{D}A{\in}x\) then \(\exists y (xR^cy\) and
\(A{\in}y)\).

\emph{Proof}\\
Suppose \(\mathbf{D}A {\in}x\) and let \(y^- = \{B{:}~ B{\in}x\) and
\(\mathbf{D}B{\in}x\}\cup\{\neg A\}\). Then \(y^-\) is consistent. For
otherwise either \(\vdash A\) or there are formulas \(B_1,{\ldots},B_n\)
such that for \(1\le i\le n, B_i{\in}x\) and \(\mathbf{D}B_i{\in}x\) and
\(\vdash B_1\wedge {\ldots}\wedge B_n\rightarrow A\). In the first case,
by D0, \(\vdash A{\leftrightarrow}\top\). Since the logic is
congruential, \(\vdash \mathbf{D}\top\), violating D5. So we may assume
that the second case obtains. By RD\(_{1,1}\),
\(\vdash B_1\wedge {\ldots}\wedge B_n\rightarrow (\mathbf{D}A\rightarrow \mathbf{D}(B_1\wedge {\ldots}\wedge B_n))\).
Each \(B_i\) is a member of \emph{x} by construction and \(\mathbf{D}A\)
is a member of \emph{x} by supposition, so it follows that
\(\mathbf{D}(B_1\wedge {\ldots}\wedge B_n){\in}x\). By D3* this implies
\(\mathbf{D}B_1\vee {\ldots}\vee \mathbf{D}B_n \in x\). But by
construction none of the formulas \(\mathbf{D}B_i\) is a member of
\emph{x}, so we have reached a contradiction. Thus \(y^-\) is consistent
as claimed. By Lindenbaum's lemma, it can be extended to a maximal
consistent set y satisfying the conditions of the lemma.~◻

\textbf{Lemma 3} (Truth Lemma)\\
In the canonical model for our logic, \(\models _x A\) iff \(A{\in}x\).

\emph{Proof}\\
By induction on \(A\). We consider the case \(A=\mathbf{D}B\). First
suppose \(\models _x A\). By the truth definition, \(\models _x B\) and
\(\exists y (xR^cy\) and \({\nvDash_y}B)\). By induction hypothesis,
\(B{\in}x\) and \(\exists y (xR^cy\) and \(B{\notin}y)\). By the
definition of \(R^c\), \(\mathbf{D}B{\in}x\) as required.

For the converse, suppose \(A{\in}x\). By the witness lemma,
\(\exists y (xR^cy\) and \(B{\in}y)\). By induction hypothesis,
\(\exists y( xR^cy\) and \({\models_y}B)\). Furthermore, since
\(A{\in}x\), D1 implies that \(B{\in}x\), and,by induction hypothesis,
this implies that \(\models _x B\). So, by the truth definition,
\(\models _x A\), as required.~◻

To prove the theorem note that, by Lindenbaum's lemma, any consistent
set in the logic described can be extended to a maximal consistent set,
which will be one of the worlds in the canonical model. By the truth
lemma, all the members of the set will be true at that world.~◻

\subsection*{References}\label{references}
\addcontentsline{toc}{subsection}{References}

\phantomsection\label{refs}
\begin{CSLReferences}{1}{0}
\bibitem[\citeproctext]{ref-French2010}
French, Rohan. 2010. {``Translational Embeddings in Modal Logic.''} PhD
thesis, Department of Philosophy, Monash University. doi:
\href{https://doi.org/10.4225/03/587c09ed83982}{10.4225/03/587c09ed83982}.

\bibitem[\citeproctext]{ref-Humberstone.2011}
Humberstone, Lloyd. 2011. \emph{The Connectives}. Cambridge, MA: MIT
Press.

\bibitem[\citeproctext]{ref-Humberstone2016}
---------. 2016. \emph{Philsophical Applications of Modal Logic}. Milton
Keynes: College Publications.

\bibitem[\citeproctext]{ref-Kuhn.1978}
Kuhn, Steven. 1978. \emph{Many-Sorted Modal Logics}. Uppsala: Department
of Philosophy, Uppsala University.

\bibitem[\citeproctext]{ref-PelletierAndUrquhart}
Pelletier, Francis Jeffry, and Alasdair Urquhart. 2003. {``Synonymous
Logics.''} \emph{Journal of Philosophical Logic} 32 (3): 259--85. doi:
\href{https://doi.org/10.1023/a:1024248828122}{10.1023/a:1024248828122}.
Correction: \emph{ibid.} \textbf{37}:95-100, 2008.

\bibitem[\citeproctext]{ref-prior1951ethical}
Prior, Arthur N. 1951. {``The Ethical Copula.''} \emph{Australasian
Journal of Philosophy} 29 (3): 137--54. doi:
\href{https://doi.org/10.1080/00048405185200171}{10.1080/00048405185200171}.

\bibitem[\citeproctext]{ref-Stalnaker2006-STAOLO}
Stalnaker, Robert. 2006. {``{On Logics of Knowledge and Belief}.''}
\emph{Philosophical Studies} 128 (1). doi:
\href{https://doi.org/10.1007/s11098-005-4062-y}{10.1007/s11098-005-4062-y}.

\bibitem[\citeproctext]{ref-Voorbraak.1991}
Voorbraak, Frans. 1991. {``The Logic of Objective Knowledge and Rational
Belief.''} In \emph{Logics in {AI}}, edited by J. van Eijck, 499--515.
Berlin, Heidelberg: Springer, Berlin.

\bibitem[\citeproctext]{ref-Zolin.2000}
Zolin, Evgeni E. 2000. {``Embeddings of Propositional Monomodal
Logics.''} \emph{Logic Journal of IGPL} 8 (6): 861--82. doi:
\href{https://doi.org/10.1093/jigpal/8.6.861}{10.1093/jigpal/8.6.861}.

\end{CSLReferences}



\noindent Published in\emph{
Australasian Journal of Logic}, 2018, pp. 1-18.


\end{document}
