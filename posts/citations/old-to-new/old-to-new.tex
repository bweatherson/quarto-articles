% Options for packages loaded elsewhere
\PassOptionsToPackage{unicode}{hyperref}
\PassOptionsToPackage{hyphens}{url}
%
\documentclass[
  10pt,
  letterpaper,
  DIV=11,
  numbers=noendperiod,
  twoside]{scrartcl}

\usepackage{amsmath,amssymb}
\usepackage{setspace}
\usepackage{iftex}
\ifPDFTeX
  \usepackage[T1]{fontenc}
  \usepackage[utf8]{inputenc}
  \usepackage{textcomp} % provide euro and other symbols
\else % if luatex or xetex
  \usepackage{unicode-math}
  \defaultfontfeatures{Scale=MatchLowercase}
  \defaultfontfeatures[\rmfamily]{Ligatures=TeX,Scale=1}
\fi
\usepackage{lmodern}
\ifPDFTeX\else  
    % xetex/luatex font selection
  \setmainfont[ItalicFont=EB Garamond Italic,BoldFont=EB Garamond
Bold]{EB Garamond Math}
  \setsansfont[]{Europa-Bold}
  \setmathfont[]{Garamond-Math}
\fi
% Use upquote if available, for straight quotes in verbatim environments
\IfFileExists{upquote.sty}{\usepackage{upquote}}{}
\IfFileExists{microtype.sty}{% use microtype if available
  \usepackage[]{microtype}
  \UseMicrotypeSet[protrusion]{basicmath} % disable protrusion for tt fonts
}{}
\usepackage{xcolor}
\usepackage[left=1in, right=1in, top=0.8in, bottom=0.8in,
paperheight=9.5in, paperwidth=6.5in, includemp=TRUE, marginparwidth=0in,
marginparsep=0in]{geometry}
\setlength{\emergencystretch}{3em} % prevent overfull lines
\setcounter{secnumdepth}{3}
% Make \paragraph and \subparagraph free-standing
\ifx\paragraph\undefined\else
  \let\oldparagraph\paragraph
  \renewcommand{\paragraph}[1]{\oldparagraph{#1}\mbox{}}
\fi
\ifx\subparagraph\undefined\else
  \let\oldsubparagraph\subparagraph
  \renewcommand{\subparagraph}[1]{\oldsubparagraph{#1}\mbox{}}
\fi


\providecommand{\tightlist}{%
  \setlength{\itemsep}{0pt}\setlength{\parskip}{0pt}}\usepackage{longtable,booktabs,array}
\usepackage{calc} % for calculating minipage widths
% Correct order of tables after \paragraph or \subparagraph
\usepackage{etoolbox}
\makeatletter
\patchcmd\longtable{\par}{\if@noskipsec\mbox{}\fi\par}{}{}
\makeatother
% Allow footnotes in longtable head/foot
\IfFileExists{footnotehyper.sty}{\usepackage{footnotehyper}}{\usepackage{footnote}}
\makesavenoteenv{longtable}
\usepackage{graphicx}
\makeatletter
\def\maxwidth{\ifdim\Gin@nat@width>\linewidth\linewidth\else\Gin@nat@width\fi}
\def\maxheight{\ifdim\Gin@nat@height>\textheight\textheight\else\Gin@nat@height\fi}
\makeatother
% Scale images if necessary, so that they will not overflow the page
% margins by default, and it is still possible to overwrite the defaults
% using explicit options in \includegraphics[width, height, ...]{}
\setkeys{Gin}{width=\maxwidth,height=\maxheight,keepaspectratio}
% Set default figure placement to htbp
\makeatletter
\def\fps@figure{htbp}
\makeatother
% definitions for citeproc citations
\NewDocumentCommand\citeproctext{}{}
\NewDocumentCommand\citeproc{mm}{%
  \begingroup\def\citeproctext{#2}\cite{#1}\endgroup}
\makeatletter
 % allow citations to break across lines
 \let\@cite@ofmt\@firstofone
 % avoid brackets around text for \cite:
 \def\@biblabel#1{}
 \def\@cite#1#2{{#1\if@tempswa , #2\fi}}
\makeatother
\newlength{\cslhangindent}
\setlength{\cslhangindent}{1.5em}
\newlength{\csllabelwidth}
\setlength{\csllabelwidth}{3em}
\newenvironment{CSLReferences}[2] % #1 hanging-indent, #2 entry-spacing
 {\begin{list}{}{%
  \setlength{\itemindent}{0pt}
  \setlength{\leftmargin}{0pt}
  \setlength{\parsep}{0pt}
  % turn on hanging indent if param 1 is 1
  \ifodd #1
   \setlength{\leftmargin}{\cslhangindent}
   \setlength{\itemindent}{-1\cslhangindent}
  \fi
  % set entry spacing
  \setlength{\itemsep}{#2\baselineskip}}}
 {\end{list}}
\usepackage{calc}
\newcommand{\CSLBlock}[1]{\hfill\break\parbox[t]{\linewidth}{\strut\ignorespaces#1\strut}}
\newcommand{\CSLLeftMargin}[1]{\parbox[t]{\csllabelwidth}{\strut#1\strut}}
\newcommand{\CSLRightInline}[1]{\parbox[t]{\linewidth - \csllabelwidth}{\strut#1\strut}}
\newcommand{\CSLIndent}[1]{\hspace{\cslhangindent}#1}

\setlength\heavyrulewidth{0ex}
\setlength\lightrulewidth{0ex}
\usepackage[automark]{scrlayer-scrpage}
\clearpairofpagestyles
\cehead{
  Brian Weatherson
  }
\cohead{
  Citations, Then and Now
  }
\ohead{\bfseries \pagemark}
\cfoot{}
\makeatletter
\newcommand*\NoIndentAfterEnv[1]{%
  \AfterEndEnvironment{#1}{\par\@afterindentfalse\@afterheading}}
\makeatother
\NoIndentAfterEnv{itemize}
\NoIndentAfterEnv{enumerate}
\NoIndentAfterEnv{description}
\NoIndentAfterEnv{quote}
\NoIndentAfterEnv{equation}
\NoIndentAfterEnv{longtable}
\NoIndentAfterEnv{abstract}
\renewenvironment{abstract}
 {\vspace{-1.25cm}
 \quotation\small\noindent\rule{\linewidth}{.5pt}\par\smallskip
 \noindent }
 {\par\noindent\rule{\linewidth}{.5pt}\endquotation}
\KOMAoption{captions}{tableheading}
\makeatletter
\@ifpackageloaded{caption}{}{\usepackage{caption}}
\AtBeginDocument{%
\ifdefined\contentsname
  \renewcommand*\contentsname{Table of contents}
\else
  \newcommand\contentsname{Table of contents}
\fi
\ifdefined\listfigurename
  \renewcommand*\listfigurename{List of Figures}
\else
  \newcommand\listfigurename{List of Figures}
\fi
\ifdefined\listtablename
  \renewcommand*\listtablename{List of Tables}
\else
  \newcommand\listtablename{List of Tables}
\fi
\ifdefined\figurename
  \renewcommand*\figurename{Figure}
\else
  \newcommand\figurename{Figure}
\fi
\ifdefined\tablename
  \renewcommand*\tablename{Table}
\else
  \newcommand\tablename{Table}
\fi
}
\@ifpackageloaded{float}{}{\usepackage{float}}
\floatstyle{ruled}
\@ifundefined{c@chapter}{\newfloat{codelisting}{h}{lop}}{\newfloat{codelisting}{h}{lop}[chapter]}
\floatname{codelisting}{Listing}
\newcommand*\listoflistings{\listof{codelisting}{List of Listings}}
\makeatother
\makeatletter
\makeatother
\makeatletter
\@ifpackageloaded{caption}{}{\usepackage{caption}}
\@ifpackageloaded{subcaption}{}{\usepackage{subcaption}}
\makeatother
\ifLuaTeX
  \usepackage{selnolig}  % disable illegal ligatures
\fi
\usepackage{bookmark}

\IfFileExists{xurl.sty}{\usepackage{xurl}}{} % add URL line breaks if available
\urlstyle{same} % disable monospaced font for URLs
\hypersetup{
  pdftitle={Citations, Then and Now},
  pdfauthor={Brian Weatherson},
  hidelinks,
  pdfcreator={LaTeX via pandoc}}

\title{Citations, Then and Now}
\author{Brian Weatherson}
\date{2024}

\begin{document}
\maketitle
\begin{abstract}
This note looks at articles that were relatively widely cited soon after
publication, and asks how often they have been cited in recent years.
The main finding is that between 1980 and the late 1990s, many articles
that were widely cited at the time have largely disappeared from the
citation record in recent times.
\end{abstract}

\setstretch{1.1}
This post is about how citation patterns change. In particular, it is
about what kinds of articles are widely cited when they first come out,
but less cited in the future. The key result is that there were
surprisingly many of these articles in journals between 1980 and 1995.
The main cause of the drop in citations seems to be simple changes in
trends. In particular, so much journal attention was paid to relatively
a priori investigations into mental content. That's not nearly as large
a part of the philosophical landscape now, so the articles that focus on
it are less widely cited.

The dataset I'm using is the same as in
\href{http://brian.weatherson.org/quarto/posts/citations-raw-data/citations.html}{an
earlier post}, and I won't repeat it here. The big thing to know is that
I'm just looking at one hundred Anglophone, relatively analytic,
philosophy journals, and looking at citations in those journals to those
journals. For this study I'm going to start in 1965.

Note that because of a gap in the Web of Science data, I had to manually
add citations to \emph{Journal of Philosophy} articles from 1971 to
1974, but I don't have citations by those articles. That leads to some
weirdnesses, but I don't think it drastically affects these results.

Note also that Web of Science doesn't start indexing \emph{Analysis}
until 1975. This does affect the results substantially; there is no
point looking pre-1965 because so many of the widely cited articles are
in \emph{Analysis}. I suspect having \emph{Analysis} would make a pretty
big change to 1965-1970 as well, but I can't be sure.

\section{Study 1 - Counting widely cited articles}\label{sec-study-one}

The main focus here is on articles published between 1965 and 2014. I'm
stopping in 2014 because I want to be able to compare how articles were
cited soon after publication, with how they have been cited recently.
That requires having enough non-recent years that are `soon after
publication'. Since my dataset stops in mid-2022, that implied stopping
in 2014.

Divide that fifty year period up into ten periods of five years each, in
the obvious way. So for each decade we have the early and late half of
the decade, though we just have late 1960s and early 2010s.

For each half-decade, and each article published in it, find two values.

\begin{itemize}
\tightlist
\item
  The \textbf{early cites} to the article are citations by the end of
  the subsequent half-decade. So for articles published between 1990 and
  1994, that means citations in or before 1999.
\item
  The \textbf{late cites} to the article are citations in 2020, 2021,
  and the part of 2022 covered in the dataset.
\end{itemize}

The late cites might not seem like much of a dataset. But because there
are so many articles published now, and because citation practices have
changed so much, that includes many many citations. The dataset I have
goes back to the mid-1950s, but over a quarter of the citations are in
these two and a half years.

Having done that, rank the articles by the number of early cites they
have, using the number of late cites as a tiebreaker. The tiebreakers
are important, because citations just within journals are not very high;
the typical case is that there are lots of ties.

Then choose the top twenty articles, i.e., the twenty articles that were
most cited by the end of the subsequent half-decade. (With ties broken
by looking at what was most cited recently.) For the period 1990-1994,
Table~\ref{tbl-early-1990s} lists the twenty articles in question.


\begin{longtable}[]{@{}
  >{\raggedright\arraybackslash}p{(\columnwidth - 0\tabcolsep) * \real{1.0000}}@{}}

\caption{\label{tbl-early-1990s}The twenty articles from 1990-1994 most
widely cited at the time.}

\tabularnewline

\toprule\noalign{}
\begin{minipage}[b]{\linewidth}\raggedright
Article
\end{minipage} \\
\midrule\noalign{}
\endhead
\bottomrule\noalign{}
\endlastfoot
D Davidson
(\citeproc{ref-WOSA1990EQ84600001}{1990})
``The Structure and Content of Truth'' \\
K Neander
(\citeproc{ref-WOSA1991FQ15000002}{1991a})
``Functions as Selected Effects: The Conceptual Analyst's Defense'' \\
S Yablo
(\citeproc{ref-WOSA1992JA62400001}{1992})
``Mental Causation'' \\
T Crane and DH Mellor
(\citeproc{ref-WOSA1990DA14600002}{1990})
``There is No Question of Physicalism'' \\
JW Kim
(\citeproc{ref-WOSA1990FU75100001}{1990})
``Supervenience as a Philosophical Concept'' \\
P Kitcher
(\citeproc{ref-WOSA1990CH71200001}{1990})
``The Division of Cognitive Labor'' \\
K Neander
(\citeproc{ref-WOSA1991GR92500005}{1991b})
``The Teleological Notion of Function'' \\
P Kitcher
(\citeproc{ref-WOSA1992HF90300002}{1992})
``The Naturalists Return'' \\
G Rosen
(\citeproc{ref-WOSA1990DR99100001}{1990})
``Modal Fictionalism'' \\
MB Burke
(\citeproc{ref-WOSA1992HC13100003}{1992})
``Copper Statues and Pieces of Copper: A Challenge To the Standard
Account'' \\
S Schiffer
(\citeproc{ref-WOSA1992JQ78400001}{1992})
``Belief Ascription'' \\
M Mckinsey
(\citeproc{ref-WOSA1991EW83300002}{1991})
``Anti-Individualism and Privileged Access'' \\
M Tye
(\citeproc{ref-WOSA1990EG85000002}{1990})
``Vague Objects'' \\
MB Burke
(\citeproc{ref-WOSA1994PD73500006}{1994})
``Preserving the Principle of 1 Object To a Place: A Novel Account of
the Relations Among Objects, Sorts, Sortals, and Persistance
Conditions'' \\
JA Fodor
(\citeproc{ref-WOSA1991EN62900001}{1991})
``A Modal Argument for Narrow Content'' \\
L Laudan
(\citeproc{ref-WOSA1990CX46800004}{1990})
``Normative Naturalism'' \\
PA Boghossian
(\citeproc{ref-WOSA1990CV26800001}{1990})
``The Status of Content'' \\
WC Salmon
(\citeproc{ref-WOSA1994NQ04900008}{1994})
``Causality Without Counterfactuals'' \\
H Putnam
(\citeproc{ref-WOSA1994PF23100001}{1994})
``Sense, Nonsense, and the Senses: An Inquiry into the Powers of the
Human Mind'' \\
G Forbes
(\citeproc{ref-WOSA1990EB39300002}{1990})
``The Indispensability of Sinn'' \\

\end{longtable}

If we look at all the articles in the dataset, just over 1\% have been
cited sixteen or more times since 2020. (Just under 1\% have been cited
seventeen or more times.) Call this top 1\% of cited articles the
\emph{widely cited} articles in recent philosophy. Our first question is
how many of these twenty articles that were the most cited at the time,
are widely cited in this sense.

The answer is just four: Stephen Yablo's paper on mental causation,
Philip Kitcher's paper on the division of cognitive labour, and Karen
Neander's two papers on functions. Table~\ref{tbl-early-1990s-expanded}
shows how often each of these articles were cited in the `early' years,
i.e., 1990-1999, and how often they are cited in the `late years', i.e.,
from 2020 to mid 2022.


\begin{longtable}[]{@{}
  >{\raggedleft\arraybackslash}p{(\columnwidth - 4\tabcolsep) * \real{0.0609}}
  >{\raggedleft\arraybackslash}p{(\columnwidth - 4\tabcolsep) * \real{0.0558}}
  >{\raggedright\arraybackslash}p{(\columnwidth - 4\tabcolsep) * \real{0.8832}}@{}}

\caption{\label{tbl-early-1990s-expanded}Early and late citations to
twenty articles from 1990-1994.}

\tabularnewline

\toprule\noalign{}
\begin{minipage}[b]{\linewidth}\raggedleft
Early Cites
\end{minipage} & \begin{minipage}[b]{\linewidth}\raggedleft
Late Cites
\end{minipage} & \begin{minipage}[b]{\linewidth}\raggedright
Article
\end{minipage} \\
\midrule\noalign{}
\endhead
\bottomrule\noalign{}
\endlastfoot
33 & 9 & D Davidson
(\citeproc{ref-WOSA1990EQ84600001}{1990})
``The Structure and Content of Truth'' \\
32 & 33 & K Neander
(\citeproc{ref-WOSA1991FQ15000002}{1991a})
``Functions as Selected Effects: The Conceptual Analyst's Defense'' \\
27 & 66 & S Yablo
(\citeproc{ref-WOSA1992JA62400001}{1992})
``Mental Causation'' \\
26 & 7 & T Crane and DH Mellor
(\citeproc{ref-WOSA1990DA14600002}{1990})
``There is No Question of Physicalism'' \\
21 & 6 & JW Kim
(\citeproc{ref-WOSA1990FU75100001}{1990})
``Supervenience as a Philosophical Concept'' \\
20 & 54 & P Kitcher
(\citeproc{ref-WOSA1990CH71200001}{1990})
``The Division of Cognitive Labor'' \\
20 & 18 & K Neander
(\citeproc{ref-WOSA1991GR92500005}{1991b})
``The Teleological Notion of Function'' \\
20 & 13 & P Kitcher
(\citeproc{ref-WOSA1992HF90300002}{1992})
``The Naturalists Return'' \\
19 & 11 & G Rosen
(\citeproc{ref-WOSA1990DR99100001}{1990})
``Modal Fictionalism'' \\
19 & 8 & MB Burke
(\citeproc{ref-WOSA1992HC13100003}{1992})
``Copper Statues and Pieces of Copper: A Challenge To the Standard
Account'' \\
18 & 8 & S Schiffer
(\citeproc{ref-WOSA1992JQ78400001}{1992})
``Belief Ascription'' \\
17 & 10 & M Mckinsey
(\citeproc{ref-WOSA1991EW83300002}{1991})
``Anti-Individualism and Privileged Access'' \\
17 & 6 & M Tye
(\citeproc{ref-WOSA1990EG85000002}{1990})
``Vague Objects'' \\
17 & 5 & MB Burke
(\citeproc{ref-WOSA1994PD73500006}{1994})
``Preserving the Principle of 1 Object To a Place: A Novel Account of
the Relations Among Objects, Sorts, Sortals, and Persistance
Conditions'' \\
17 & 3 & JA Fodor
(\citeproc{ref-WOSA1991EN62900001}{1991})
``A Modal Argument for Narrow Content'' \\
16 & 14 & L Laudan
(\citeproc{ref-WOSA1990CX46800004}{1990})
``Normative Naturalism'' \\
16 & 9 & PA Boghossian
(\citeproc{ref-WOSA1990CV26800001}{1990})
``The Status of Content'' \\
16 & 8 & WC Salmon
(\citeproc{ref-WOSA1994NQ04900008}{1994})
``Causality Without Counterfactuals'' \\
16 & 8 & H Putnam
(\citeproc{ref-WOSA1994PF23100001}{1994})
``Sense, Nonsense, and the Senses: An Inquiry into the Powers of the
Human Mind'' \\
16 & 4 & G Forbes
(\citeproc{ref-WOSA1990EB39300002}{1990})
``The Indispensability of Sinn'' \\

\end{longtable}

This is extremely unusual, though as we'll see in a bit it is something
of an outlier result. If we do the same thing for each of the five year
periods, we typically see about 8-10 articles be widely cited in this
way, with the numbers rising as we get closer to the present.
Figure~\ref{fig-still-standing} shows the numbers for each half-decade.
(Note that the year on the x-axis is the start of the half-decade being
shown.)

\begin{figure}

\centering{

\includegraphics{old-to-new_files/figure-pdf/fig-still-standing-1.pdf}

}

\caption{\label{fig-still-standing}How many of the twenty articles most
cited at the time are still widely cited.}

\end{figure}%

As you can see, the period 1990-1994 really stands out here. But it's a
bit crude to just look at what's above or below a threshold. Let's try
being a bit more finegrained.

\section{Study 2 - Median citations of The Twenty}\label{sec-study-two}

Instead of looking at how many of the twenty articles crossed a somewhat
arbitrarily selected threshold, we could look instead at the median
number of citations they have in the period 2020-2022. I'm using median
not mean because the means end up being largely determined by how widely
cited the one or two most cited pieces are. If we use the same twenty
articles for each five year period, and calculate the median number of
citations they have in 2020-2022, we get the results seen in
Figure~\ref{fig-median-cites}.

\begin{figure}

\centering{

\includegraphics{old-to-new_files/figure-pdf/fig-median-cites-1.pdf}

}

\caption{\label{fig-median-cites}The median number of recent citations
to twenty articles most cited at the time.}

\end{figure}%

The 1990-1994 period still does poorly, but it's not as dramatic as on
Figure~\ref{fig-still-standing}.

The 1970-1974 period does surprisingly badly on this measure. That
period includes the four very widely cited articles listed in
Table~\ref{tbl-early-1970s-sample}.


\begin{longtable}[]{@{}
  >{\raggedleft\arraybackslash}p{(\columnwidth - 4\tabcolsep) * \real{0.1071}}
  >{\raggedleft\arraybackslash}p{(\columnwidth - 4\tabcolsep) * \real{0.0982}}
  >{\raggedright\arraybackslash}p{(\columnwidth - 4\tabcolsep) * \real{0.7946}}@{}}

\caption{\label{tbl-early-1970s-sample}Four very widely cited articles
from the early 1970s}

\tabularnewline

\toprule\noalign{}
\begin{minipage}[b]{\linewidth}\raggedleft
Early Cites
\end{minipage} & \begin{minipage}[b]{\linewidth}\raggedleft
Late Cites
\end{minipage} & \begin{minipage}[b]{\linewidth}\raggedright
Article
\end{minipage} \\
\midrule\noalign{}
\endhead
\bottomrule\noalign{}
\endlastfoot
22 & 88 & David Lewis
(\citeproc{ref-10.2307_2025310}{\textbf{10.2307\_2025310?}})
``Causation'' \\
23 & 72 & Harry G. Frankfurt
(\citeproc{ref-10.2307_2024717}{\textbf{10.2307\_2024717?}})
``Freedom of the Will and the Concept of a Person'' \\
12 & 49 & Paul Benacerraf
(\citeproc{ref-10.2307_2025075}{\textbf{10.2307\_2025075?}})
``Mathematical Truth'' \\
12 & 48 & JJ Thomson
(\citeproc{ref-WOSA1971Y116900003}{1971})
``A Defense of Abortion'' \\

\end{longtable}

But several other articles that were more prominent at the time,
including two by Hartry Field and two by Richard Rorty, are not nearly
as prominent in the recent literature. The large number of citations to
these four articles doesn't move the median that much.

\section{Study 3 - Rolling Periods}\label{sec-study-three}

If you look back at Table~\ref{tbl-early-1990s}, you see a lot of
articles from 1990 and 1991 in particular. This shouldn't be a surprise.
The way the twenty articles were selected was by looking at the most
cited articles by a fixed date, in this case 1999. Articles published in
1990 and 1991 had a lot more time to accumulate citations by 1999 than
articles published later in the half-decade.

One way to fix that is to get away from having round numbers in our five
year periods. For any year \emph{y} from 1965 to 2010, we can perform
the following calculations.

\begin{itemize}
\tightlist
\item
  Select the articles published between \emph{y} and *y+4.
\item
  Sort them by the number of citations they have by \emph{y}+9 (using
  citations since 2020 as a tiebreaker).
\item
  Ask how many of the top twenty on that list are widely cited recently
  (i.e., have at least sixteen citations).
\end{itemize}

If we set \emph{y} to be 1990, that gives us the results we saw in
Table~\ref{tbl-early-1990s} and Table~\ref{tbl-early-1990s-expanded}.
But we can do it for years that don't end with 0 and 5. If we do it for
all the years from 1965 to 2010, we get the results in
Figure~\ref{fig-all-still-standing}.

\begin{figure}

\centering{

\includegraphics{old-to-new_files/figure-pdf/fig-all-still-standing-1.pdf}

}

\caption{\label{fig-all-still-standing}How many of the twenty articles
most cited at the time are still widely cited.}

\end{figure}%

The odd result for 1990 itself looks like an outlier here. Why are the
years around it so different?

It's easy to explain why the value for 1989 is different. The articles
listed in Table~\ref{tbl-top-1989} were published in 1989, and widely
cited.


\begin{longtable}[]{@{}
  >{\raggedleft\arraybackslash}p{(\columnwidth - 4\tabcolsep) * \real{0.1121}}
  >{\raggedleft\arraybackslash}p{(\columnwidth - 4\tabcolsep) * \real{0.1028}}
  >{\raggedright\arraybackslash}p{(\columnwidth - 4\tabcolsep) * \real{0.7850}}@{}}

\caption{\label{tbl-top-1989}Widely cited articles from 1989}

\tabularnewline

\toprule\noalign{}
\begin{minipage}[b]{\linewidth}\raggedleft
Early Cites
\end{minipage} & \begin{minipage}[b]{\linewidth}\raggedleft
Late Cites
\end{minipage} & \begin{minipage}[b]{\linewidth}\raggedright
Article
\end{minipage} \\
\midrule\noalign{}
\endhead
\bottomrule\noalign{}
\endlastfoot
32 & 36 & RG Millikan
(\citeproc{ref-WOSA1989AA09400006}{1989b})
``In Defense of Proper Functions'' \\
23 & 39 & GA Cohen
(\citeproc{ref-WOSA1989AE70300010}{1989})
``On the Currency of Egalitarian Justice'' \\
21 & 20 & PA Boghossian and JD Velleman
(\citeproc{ref-WOSA1989T231400005}{1989})
``Color as a Secondary Quality'' \\
30 & 44 & RG Millikan
(\citeproc{ref-WOSA1989U850300001}{1989a})
``Biosemantics'' \\

\end{longtable}

The range 1989-1993 includes those four articles which the range
1990-1994 does not. But what is going on with the range 1991-1995? How
does adding 1995 to the set make such a difference? It's a bit more
complicated than that. The nine articles from 1991-1995 which are widely
cited recently are shown in Table~\ref{tbl-top-1991}.


\begin{longtable}[]{@{}
  >{\raggedleft\arraybackslash}p{(\columnwidth - 4\tabcolsep) * \real{0.0686}}
  >{\raggedleft\arraybackslash}p{(\columnwidth - 4\tabcolsep) * \real{0.0629}}
  >{\raggedright\arraybackslash}p{(\columnwidth - 4\tabcolsep) * \real{0.8686}}@{}}

\caption{\label{tbl-top-1991}Widely cited articles from 1991}

\tabularnewline

\toprule\noalign{}
\begin{minipage}[b]{\linewidth}\raggedleft
Early Cites
\end{minipage} & \begin{minipage}[b]{\linewidth}\raggedleft
Late Cites
\end{minipage} & \begin{minipage}[b]{\linewidth}\raggedright
Article
\end{minipage} \\
\midrule\noalign{}
\endhead
\bottomrule\noalign{}
\endlastfoot
18 & 84 & D Lewis
(\citeproc{ref-WOSA1994PM10400005}{1994})
``Humean Supervenience Debugged'' \\
31 & 66 & S Yablo
(\citeproc{ref-WOSA1992JA62400001}{1992})
``Mental Causation'' \\
17 & 61 & DC Dennett
(\citeproc{ref-WOSA1991EN62900002}{1991})
``Real Patterns'' \\
28 & 54 & T Burge
(\citeproc{ref-WOSA1993ML38000001}{1993})
``Content Preservation'' \\
18 & 48 & M Johnston
(\citeproc{ref-WOSA1992KC39800002}{1992})
``How To Speak of the Colors'' \\
36 & 33 & K Neander
(\citeproc{ref-WOSA1991FQ15000002}{1991a})
``Functions as Selected Effects: The Conceptual Analyst's Defense'' \\
18 & 21 & K Neander
(\citeproc{ref-WOSA1995RP14800001}{1995})
``Misrepresenting and Malfunctioning'' \\
19 & 20 & M Forster and E Sober
(\citeproc{ref-WOSA1994NQ78600001}{1994})
``How To Tell When Simpler, More Unified, or Less \emph{Ad Hoc} Theories
Will Provide More Accurate Predictions'' \\
22 & 18 & K Neander
(\citeproc{ref-WOSA1991GR92500005}{1991b})
``The Teleological Notion of Function'' \\

\end{longtable}

What's happened here is that several of the articles, most notably the
Lewis and Dennett ones, did not get a huge number of citations straight
away. They counted for the period 1991-1995 but not the period 1990-1994
because they were so often cited in 2000.

There is something nice about this way of counting how many `early'
cites a paper has. What counts as a citation `at the time' in philosophy
is somewhat vague. Philosophy moves slowly, at least relative to some
sciences, so citations within the first five years are clearly early
citations. Citations when the paper is ten years old or more are not
early citations. Between those two, there is some vagueness.

One way to handle that vagueness would be to stipulate what we mean by
`early'. Any such stipulation would be somewhat arbitrary, and any
results would have to be checked by re-running the model for other
stipulations.

The method here of looking at rolling, overlapping, intervals allows us
to accommodate the vagueness. Consider two hypothetical papers. Paper
one gets fifty citations the year it comes out, and is never cited
again. Paper two is not cited for eight years, then gets fifty citations
every year after that. Paper one will be in the top twenty papers with
the most early cites for five different periods; all the ones it is
eligible for. Paper two will only be in the top twenty for one of those
intervals, the very last one it was eligible for. So for graphs like
Figure~\ref{fig-all-still-standing}, paper one will impact (and pull
down), five of the points, while paper two will only impact (and pull
up), one of them. If paper two had started getting citations earlier, it
would have impacted more of the points. In that way, graphs like
Figure~\ref{fig-all-still-standing} take account of the vagueness in
`early'.

Later on, we will want to focus on articles, like article one, that
affect many points in the graph, and it's useful to think about why
different articles might be differentially important to this graph.

Summing up, Figure~\ref{fig-all-still-standing} that there was something
a bit unusual about the results in Section~\ref{sec-study-one} and
Section~\ref{sec-study-two}. It was only by a very particular choice of
years that the period 1990-1994 looks so unusual.

But the broader picture shows that the particular period 1990-1994 was a
bit unusual relative to its surrounds. It still leaves us with a
question about those surrounds. And we need one last study to see the
thing that most needs explaining.

\section{Study 4 - Medians in Rolling Periods}\label{sec-study-four}

As in Section~\ref{sec-study-three}, do the following calculation for
each year \emph{y}.

\begin{itemize}
\tightlist
\item
  Select the articles published between \emph{y} and *y+4.
\item
  Sort them by the number of citations they have by \emph{y}+9 (using
  citations since 2020 as a tiebreaker).
\item
  Focus on the top twenty in that list.
\end{itemize}

But instead of asking how many of these twenty are `widely cited',
instead calculate the median number of citations since 2020 for those
twenty articles. The results are shown in Figure~\ref{fig-all-median}.

\begin{figure}

\centering{

\includegraphics{old-to-new_files/figure-pdf/fig-all-median-1.pdf}

}

\caption{\label{fig-all-median}The median number of recent cites for the
twenty articles most cited at the time.}

\end{figure}%

This is a really amazing graph. Every year from 1980 to 1995 is worse
than every year from 1974 to 1979, and every year from 1996 to the
present. How could this have happened?

\section{Three Explanations}\label{sec-three-explanations}

When we think about what Figure~\ref{fig-all-median} is measuring, we
can see that there are three ways that the result might have come about.
Each of these `ways' suggests different places to look for the
underlying explanation, but first we need to test which of them matches
the data.

First, it might be that philosophers in recent years have systematically
ignored journal articles published in the 1980s and 1990s. If there are
few widely cited articles from those years, these medians will be pulled
down.

Second, it might be that the widely cited articles from the 1980s and
1990s are particularly likely to have not been cited much at the time.
It's easy to find examples of that. Rae Langton's 1993 paper ``Speech
Acts and Unspeakable Acts'' has 49 citations between 2020 and mid-2022.
But it doesn't show up in Figure~\ref{fig-all-median} because it had
hardly any citations for more than a decade after it was published. If
the work that philosophers now focus on from the 1980s and 1990s was
systematically ignored when it first came out, we'd again see the shape
of the graph in Figure~\ref{fig-all-median}.

Third, it might be that the articles which were particularly heavily
discussed at the time are discussed much less these days. So consider
two articles from 1988, Frank Jackson and Philip Pettit's
``Functionalism and Broad Content'', and John Rawls's ``The Priority of
Right and Ideas of the Good''. Both of them were so widely discussed
that they became among the twenty most cited recent articles within five
years of publication. In the terms discussed earlier, they each impact
five points on the graph. And this impact is to drag the points down,
because each of them was only cited five times between 2020 and
mid-2022. Now I say `only' even though this is still a non-trivial
number of citations; two-thirds of the papers in the database have no
citations at all in this time. It is, however, a lot fewer citations
than you might expect given their immediate prominence.

All three of these explanations get at part of the truth, but I think
the data suggests the third of them is the most important. One striking
thing that happens around 2000 is that papers like the Jackson and
Pettit, and the Rawls, stop showing up in the data. From that point on,
if a paper is widely discussed when it's published, it's still widely
discussed from 2020 to mid-2022.

The other two explanations don't explain quite so sharp a break. The
second explanation does particularly badly on that score; papers that
don't take off right away but which eventually become very widely
discussed in recent philosophy are more common in the 2000s than in the
1990s. The first explanation does a bit better at matching the data, but
only a bit; there are plenty of papers in the 1980s and 1990s that are
discussed in recent philosophy, even if they are discussed a bit less
than papers from the 1970s and 2000s.

Let's look at these explanations in turn to back up those claims.

\section{Explanation 1 - The Missing Decades}\label{sec-missing-decades}

A simple explanation of Figure~\ref{fig-all-median} would be that no one
discusses papers from the 1980s and 1990s nowadays, so they don't
discuss papers from the 1980s and 1990s that were widely discussed at
the time. This is a quarter true at most, as the following two graphs
show.

Figure~\ref{fig-all-widely-cited} shows the number of papers published
each year which are cited 16 or more times, i.e., which are `widely
cited', between 2020 and 2022.

\begin{figure}

\centering{

\includegraphics{old-to-new_files/figure-pdf/fig-all-widely-cited-1.pdf}

}

\caption{\label{fig-all-widely-cited}Number of widely cited articles
published each year}

\end{figure}%

The number of widely cited articles goes up, but it doesn't go up
dramatically enough to explain Figure~\ref{fig-all-median}. Another way
to see this is to look at Figure~\ref{fig-all-median-unqualified}. This
modifies Figure~\ref{fig-all-median} by removing the qualification that
the articles had to be widely cited at the time. For each rolling five
year period, it records the median number of citations of the twenty
most cited articles full stop. (That is, it is the average of the
citations of the tenth most and eleventh most cited articles from that
five year period.)

\begin{figure}

\centering{

\includegraphics{old-to-new_files/figure-pdf/fig-all-median-unqualified-1.pdf}

}

\caption{\label{fig-all-median-unqualified}Median citations of the
twenty most cited articles from overlapping five year periods.}

\end{figure}%

There is a jump at 1996; that is, the range 1996-2000 does much better
on this metric than any five year range earlier than that. So that's why
I think this explanation is perhaps a quarter true. But there are two
reasons to think it isn't the whole story.

First, part of the surprise in Figure~\ref{fig-all-median} was that the
period 1980-1995 was lower than either side of it. That's not something
we see in Figure~\ref{fig-all-median-unqualified}. This perhaps explains
the jump around 1996, but it doesn't explain the drop around 1980.

Second, the absolute values on Figure~\ref{fig-all-median-unqualified}
are so high that it tells against the kind of explanation being offered
here. Apart from ranges centered on the mid-1980s, in most five year
periods there are ten or more papers that are cited forty or more times
in the recent literature. That's easily enough to make the dip in
Figure~\ref{fig-all-median} go away. The key thing here is to not just
look at the shape of Figure~\ref{fig-all-median-unqualified}, but to
look at the scale. It's just not true that philosophers are
systemtically ignoring work from the 1980s and 1990s, even if they
aren't paying it quite as much attention as they are paying to work from
the early 2000s.

\section{Explanation 2 - Late Bloomers}\label{sec-late-bloomers}

Before looking at the data, I would have guessed that this was the right
explanation: the distinctive thing about the 1980s and 1990s was that
there was so much interesting work being done whose interest was only
recognised much later. That is, to put it bluntly, not what the data
shows at all. It's true that papers like Langton's 1993 paper exist. But
there are many more such papers outside the 1980s and 1990s than inside
them.

Table~\ref{tbl-very-late-bloomers} comes from taking all the articles
that never appear on the lists of twenty most cited at the time
articles, and sorting them by how many citations they have from 2020 to
mid-2022.


\begin{longtable}[]{@{}
  >{\raggedleft\arraybackslash}p{(\columnwidth - 2\tabcolsep) * \real{0.0833}}
  >{\raggedright\arraybackslash}p{(\columnwidth - 2\tabcolsep) * \real{0.9167}}@{}}

\caption{\label{tbl-very-late-bloomers}Articles with the higher number
of recent citations that are not in the top twenty for any five year
span.}

\tabularnewline

\toprule\noalign{}
\begin{minipage}[b]{\linewidth}\raggedleft
Citations
\end{minipage} & \begin{minipage}[b]{\linewidth}\raggedright
Article
\end{minipage} \\
\midrule\noalign{}
\endhead
\bottomrule\noalign{}
\endlastfoot
156 & S Haslanger
(\citeproc{ref-WOS000085841900002}{2000})
``Gender and Race: (What) Are They? (What) Do We Want Them To Be?'' \\
96 & JM Joyce
(\citeproc{ref-WOS000077956100002}{1998})
``A Nonpragmatic Vindication of Probabilism'' \\
94 & D Lewis
(\citeproc{ref-WOSA1979HJ57600007}{1979})
``Scorekeeping in a Language Game'' \\
84 & J D'arms and D Jacobson
(\citeproc{ref-WOS000087998300003}{2000})
``The Moralistic Fallacy: On the `Appropriateness' of Emotions'' \\
74 & H Douglas
(\citeproc{ref-WOS000166575500001}{2000})
``Inductive Risk and Values in Science'' \\
67 & AI Goldman
(\citeproc{ref-WOS000170434600004}{2001})
``Experts: Which Ones Should You Trust?'' \\
64 & K Derose
(\citeproc{ref-WOSA1992KB29500008}{1992})
``Contextualism and Knowledge Attributions'' \\
60 & SL Darwall
(\citeproc{ref-WOSA1977EA35800003}{1977})
``Two Kinds of Respect'' \\
57 & AM Smith
(\citeproc{ref-WOS000227058600002}{2005})
``Responsibility for Attitudes: Activity and Passivity in Mental
Life'' \\
56 & P Hieronymi
(\citeproc{ref-WOS000234618400001}{2005})
``The Wrong Kind of Reason'' \\

\end{longtable}

There are two things to note about this list.

One is that it is much more female than any list we've seen so far. Four
of the ten papers are written by women. There is an interesting research
project here about the difference in citation dynamics for papers
written by women and men, but I'll leave that for another day.

The other is that only two of the papers are from the 1990s, and none
are from the 1980s. One of those two is Joyce's 1998 paper that
(eventually) launched the accuracy-first program in epistemology.
Because it is so late in the 1990s, it could only have affected the
pattern in Figure~\ref{fig-all-median} at the very margin.

That's so say, while it's true that the papers like Langton's, and
DeRose's 1992 paper, were published in the 1990s, that's not what makes
the decade stand out. Every decade has papers like that. If anything,
papers like that were rarer in the 1990s, and especially the 1980s, than
before or after. So this isn't the explanation of
Figure~\ref{fig-all-median}.

\section{Explanation 3 - Fade Aways}\label{sec-fade-aways}

This leaves us with the last possible explanation: the reason that the
medians are so low in the 1980s and 1990s is that many articles which
are in the list of `widely cited at the time', simply aren't cited that
much in the 2020s.

The simplest way to see how well the data support this is to focus on
those articles which appear in many of the overlapping lists of highly
cited recent articles. An article published in year \emph{y} can appear
in the twenty most recent cited articles in each range from
{[}\emph{y}-4,~\emph{y}{]} to {[}\emph{y},~\emph{y}+4{]}. Let's start by
looking at the ones that appear in four or five of these lists.
Figure~\ref{fig-four-five-recent} shows how often those articles are
cites from 2020 to mid-2022.

\begin{figure}

\centering{

\includegraphics{old-to-new_files/figure-pdf/fig-four-five-recent-1.pdf}

}

\caption{\label{fig-four-five-recent}Number of recent citations for very
highly cited articles in different years.}

\end{figure}%

Note that Figure~\ref{fig-four-five-recent} uses a log scale on the
y-axis; otherwise the whole graph would have to be compressed to
properly show how many citations ``New Work for a Theory of Universals''
has. Note also that the two dots in the lower left are a bit misleading.
These two articles have zero citations since 2020, and zero points don't
show up on a log scale. So I represented them as having 0.5 citations
each. Also, I've added a small amount of `jitter' to the graph so
overlapping points are visible.

The thing that I think is most notable is looking at how many dots are
under the 10 citations line. Before 1980 there are two such articles.
Between 1980 and 1995 there are eleven. And then after 1995 there is
just one. Table~\ref{tbl-four-five-fade-away} goes lists these articles.


\begin{longtable}[]{@{}
  >{\raggedleft\arraybackslash}p{(\columnwidth - 2\tabcolsep) * \real{0.0794}}
  >{\raggedright\arraybackslash}p{(\columnwidth - 2\tabcolsep) * \real{0.9206}}@{}}

\caption{\label{tbl-four-five-fade-away}Highly cited articles with fewer
than ten recent citations.}

\tabularnewline

\toprule\noalign{}
\begin{minipage}[b]{\linewidth}\raggedleft
Late Citations
\end{minipage} & \begin{minipage}[b]{\linewidth}\raggedright
Article
\end{minipage} \\
\midrule\noalign{}
\endhead
\bottomrule\noalign{}
\endlastfoot
0 & HN Castaneda
(\citeproc{ref-WOSA1977DV15800002}{1977})
``Perception, Belief, and Structure of Physical Objects and
Consciousness'' \\
0 & J Kim
(\citeproc{ref-WOSA1978EL93700009}{1978})
``Supervenience and Nomological Incommensurables'' \\
7 & J Kim
(\citeproc{ref-WOSA1984TV24600001}{1984})
``Concepts of Supervenience'' \\
7 & J Rawls
(\citeproc{ref-WOSA1985APA8500001}{1985})
``Justice as Fairness: Political Not Metaphysical'' \\
6 & L Laudan
(\citeproc{ref-WOSA1987F902200002}{1987})
``Progress or Rationality: The Prospects for Normative Naturalism'' \\
5 & F Jackson and P Pettit
(\citeproc{ref-WOSA1988P549200004}{1988})
``Functionalism and Broad Content'' \\
5 & J Rawls
(\citeproc{ref-WOSA1988Q394000001}{1988})
``The Priority of Right and Ideas of the Good'' \\
9 & D Davidson
(\citeproc{ref-WOSA1990EQ84600001}{1990})
``The Structure and Content of Truth'' \\
8 & MB Burke
(\citeproc{ref-WOSA1992HC13100003}{1992})
``Copper Statues and Pieces of Copper: A Challenge To the Standard
Account'' \\
8 & S Schiffer
(\citeproc{ref-WOSA1992JQ78400001}{1992})
``Belief Ascription'' \\
5 & MB Burke
(\citeproc{ref-WOSA1994PD73500006}{1994})
``Preserving the Principle of 1 Object To a Place: A Novel Account of
the Relations Among Objects, Sorts, Sortals, and Persistance
Conditions'' \\
8 & WC Salmon
(\citeproc{ref-WOSA1994NQ04900008}{1994})
``Causality Without Counterfactuals'' \\
8 & H Putnam
(\citeproc{ref-WOSA1994PF23100001}{1994})
``Sense, Nonsense, and the Senses: An Inquiry into the Powers of the
Human Mind'' \\
8 & PE Griffiths and RD Gray
(\citeproc{ref-WOSA1994NP54800001}{1994})
``Developmental Systems and Evolutionary Explanation'' \\
6 & DW Zimmerman
(\citeproc{ref-WOSA1995RC31600002}{1995})
``Theories of Masses and Problems of Constitution'' \\
8 & B Fitelson
(\citeproc{ref-WOS000183806600006}{2003})
``A Probabilistic Theory of Coherence'' \\

\end{longtable}

We see the same pattern with papers that turn up in three of the top
twenty lists. Figure~\ref{fig-three-recent} shows how many of these
papers there were each year, and how often they are cited from 2020 to
mid-2022.

\begin{figure}

\centering{

\includegraphics{old-to-new_files/figure-pdf/fig-three-recent-1.pdf}

}

\caption{\label{fig-three-recent}Number of recent citations for highly
cited articles in different years.}

\end{figure}%

As with Figure~\ref{fig-four-five-recent}, the y-axis is a log scale,
papers with zero citations are actually recorded as having 0.5
citations, and some jitter has been added.

Focus again on the papers with fewer than ten citations since 2020. On
this graph there are twelve such papers between 1980 and 1995, one in
the late 1990s, and none in any other decade. This is where the 1980s
and 1990s papers really stand out. Table~\ref{tbl-three-fade-away} lists
the papers that appear on three lists of top twenty most cited recent
papers, but which are not cited nearly as much in recent years.


\begin{longtable}[]{@{}
  >{\raggedleft\arraybackslash}p{(\columnwidth - 2\tabcolsep) * \real{0.1111}}
  >{\raggedright\arraybackslash}p{(\columnwidth - 2\tabcolsep) * \real{0.8889}}@{}}

\caption{\label{tbl-three-fade-away}Highly cited articles with fewer
than ten recent citations.}

\tabularnewline

\toprule\noalign{}
\begin{minipage}[b]{\linewidth}\raggedleft
Late Citations
\end{minipage} & \begin{minipage}[b]{\linewidth}\raggedright
Article
\end{minipage} \\
\midrule\noalign{}
\endhead
\bottomrule\noalign{}
\endlastfoot
2 & N Block
(\citeproc{ref-WOSA1980JW85100004}{1980})
``Are Absent Qualia Impossible'' \\
0 & DP Lackey
(\citeproc{ref-WOSA1982NV01500001}{1982})
``Missiles and Morals: A Utilitarian Look At Nuclear Deterrence'' \\
2 & J Haugeland
(\citeproc{ref-WOSA1982NC42600008}{1982})
``Weak Supervenience'' \\
5 & J Kim
(\citeproc{ref-WOSA1982NC90700004}{1982})
``Psychophysical Supervenience'' \\
2 & E Eells and E Sober
(\citeproc{ref-WOSA1983QJ85300002}{1983})
``Probabilistic Causality and the Question of Transitivity'' \\
4 & PM Churchland
(\citeproc{ref-WOSA1985AAC6100002}{1985})
``Reduction, Qualia, and the Direct Introspection of Brain States'' \\
6 & T Nagel
(\citeproc{ref-WOSA1987J379200001}{1987})
``Moral Conflict and Political Legitimacy'' \\
1 & N Carroll
(\citeproc{ref-WOSA1988Q411400001}{1988})
``Art, Practice, and Narrative'' \\
6 & JW Kim
(\citeproc{ref-WOSA1990FU75100001}{1990})
``Supervenience as a Philosophical Concept'' \\
7 & T Crane and DH Mellor
(\citeproc{ref-WOSA1990DA14600002}{1990})
``There is No Question of Physicalism'' \\
3 & K Falvey and J Owens
(\citeproc{ref-WOSA1994NP02400004}{1994})
``Externalism, Self-Knowledge, and Skepticism'' \\
7 & P Pietroski and G Rey
(\citeproc{ref-WOSA1995QL53800004}{1995})
``When Other Things Aren't Equal: Saving Ceteris Paribus Laws From
Vacuity'' \\
5 & T Sider
(\citeproc{ref-WOS000073301000002}{1997})
``Four Dimensionalism'' \\

\end{longtable}

Unlike the other two explanations, the timeline matches up. What's
distinctive about the period 1980-1999, and especially the period
1980-1995, is that there are so many papers that get discussed a lot at
the time, but get many fewer citations in recent years.

Still, this might feel less than fully satisfying as an explanation. The
data in this section tell us what happened. They don't tell us why it
happened. For that we need to look more closely at the papers involved.

\section{When Citations End}\label{sec-when-citations-end}

Asking why a paper doesn't get many citations is like asking why a
particular tennis player didn't win Wimbledon. The simple answer in both
cases is that not getting the citations, or the trophy, is the normal
state of affairs. It's not something that calls out for explanation.

That's not quite what we're trying to do here. To continue the analogy,
what we're asking here is why a player who used to go deep into the
tournament now isn't getting out of the second round. These papers are,
for the most part, still in the top ten percent of papers by citations.
But they aren't in the top one percent any more, and that's what is to
be explained.

In some cases, the explanation is quite simple: the citations simply got
\textbf{captured} by something else. When a paper is expanded into a
book, or becomes a chapter of a prominent book, very often the citations
will simply move to the book rather than the paper. There are three
clear cases of this in Table~\ref{tbl-four-five-fade-away} and
Table~\ref{tbl-three-fade-away}: the two papers by Rawls, and Sider's
``Four Dimensionalism''. Rawls's paper ``Justice as Fairness'' and
Sider's paper ``Four Dimensionalism'' both became books with the same
title. And this rather decreased the number of times they were cited.
The other Rawls paper, ``The Priority of Right and Ideas of the Good'',
became chapter 5 of his 1993 book \emph{Political Liberalism}.

It's a little harder to judge whether citations are affected by being
reprinted in a collected volume. Three of the four Kim papers on the
list are in his 1993 collection \emph{Supervenience and Mind}. This
study doesn't look at books, so it wouldn't capture those citations. But
it could be that the book was getting some of the citations that the
papers would have received. There are two ways this could happen. One is
that people might directly cite a reprinted passage, but simply cite the
book. I looked (via Google Scholar) at the citations to
\emph{Supervenience and Mind} in these hundred journals, and I never saw
that happen for these three Kim papers. But I wouldn't trust this manual
checking, and I did see it for other Kim papers, so it might happen. The
other way is that many people feel (fairly enough) that they should
name-check Kim somewhere in their discussion of supervenience, so there
were several undirected citations to the book that perhaps would have
ended up as citations to one of these papers if the book didn't exist. I
don't think it's as dramatic an effect as for Rawls and Sider, but it is
a consideration.

What seems more likely, especially for the 1978 paper that didn't get
reprinted, is that the papers were regarded as \textbf{superseded}. If
the later papers on supervenience were simply better than the earlier
ones, having a clearer statement of the view, or making distinctions
that were muddled in the earlier paper, then it would be natural to cite
the later papers. I think that's part of what goes on with the
supervenience papers, and, I conjecture, with the papers on causation,
laws, and explanation (Eells and Sober 1983, Salmon 1994, Griffiths and
Gray 1994, Pietroski and Rey 1995).

It isn't necessarily a bad thing if papers are superseded in this way.
If a field is progressing, you'd expect there to be papers which are
widely cited when they come out because they move the debate forward,
but which are less cited in the future, when the lessons from them have
been incorporated into other work.

A related phenomena is that citations might get \textbf{consolidated}.
I'll come back to this when I look more at the papers from the 2000s;
for now a brief description will suffice. Think about the following kind
of situation, which has happened a few times in philosophy. A
philosopher has a Big Idea, which interests a lot of people, many of
whom think it is true. They spell out this Big Idea over a series of
papers. At the time they are all widely cited. Later on, one of these
papers is generally regarded as the best statement, or at least the
canonical statement, of the Big Idea, and it keeps getting cited. But
the other papers do not. In particular, if someone just wants to gesture
at the Big Idea, they cite the paper that has become canonical; they
only cite the other papers if they want to engage with some point of
detail that isn't in the canonical work. Consolidation, like papers
being superseded, is often a sign of progress, or at least movement; it
means that there is an emerging consensus about what is valuable in the
Big Idea.

One notable thing about citations is that papers that people mostly
disagree with are still very widely cited; ``Epiphenomenal Qualia'' is
perhaps the most notable instance of this. But if philosophers
collectively come to think of the presuppositions, or methodology, or
philosophical approach of the paper is flawed, then the paper will get
ignored rather than argued against. In this case, we might say the paper
is \textbf{rejected} by most philosophers. This happens a bit before the
time I'm looking at, for example with some papers by Rorty that
completely drop off the radar of these journals. But maybe it is what
happens with the Castañeda and Putnam papers here. To be clear, I'm not
making any judgment here about whether philosophers are right to dismiss
these works; as any journal editor knows, sometimes rejections are
mistaken. What I am saying is that this kind of fundamental disagreement
would explain non-citation patterns.

Finally, some papers might \textbf{fall out of fashion}. There was a lot
more discussion of mental content, and of material constitution, in the
journals in the 1980s and 1990s than there is now. That's true in
particular of relatively a priori, armchair work on those topics. That,
I think, is what has happened with the papers by Burke, Zimmerman,
Jackson and Pettit, Schiffer, and Falvey and Owens. There are
considerably more papers on these two topics if we look at the papers
that are only in one or two of the top twenty lists, or which have ten
to fifteen citations in recent years (and hence are still pulling the
median downward).

\phantomsection\label{refs}
\begin{CSLReferences}{1}{0}
\bibitem[\citeproctext]{ref-WOSA1980JW85100004}
Block, N. 1980. {``Are Absent Qualia Impossible.''} \emph{Philosophical
Review} 89 (2): 257--74.

\bibitem[\citeproctext]{ref-WOSA1990CV26800001}
Boghossian, PA. 1990. {``The Status of Content.''} \emph{Philosophical
Review} 99 (2): 157--84.

\bibitem[\citeproctext]{ref-WOSA1989T231400005}
Boghossian, PA, and JD Velleman. 1989. {``Color as a Secondary
Quality.''} \emph{Mind} 98 (389): 81--103.

\bibitem[\citeproctext]{ref-WOSA1993ML38000001}
Burge, T. 1993. {``Content Preservation.''} \emph{Philosophical Review}
102 (4): 457--88.

\bibitem[\citeproctext]{ref-WOSA1992HC13100003}
Burke, MB. 1992. {``Copper Statues and Pieces of Copper: A Challenge to
the Standard Account.''} \emph{Analysis} 52 (1): 12--17.

\bibitem[\citeproctext]{ref-WOSA1994PD73500006}
---------. 1994. {``Preserving the Principle of 1 Object to a Place: A
Novel Account of the Relations Among Objects, Sorts, Sortals, and
Persistance Conditions.''} \emph{Philosophy And Phenomenological
Research} 54 (3): 591--624.

\bibitem[\citeproctext]{ref-WOSA1988Q411400001}
Carroll, N. 1988. {``Art, Practice, and Narrative.''} \emph{Monist} 71
(2): 140--56.

\bibitem[\citeproctext]{ref-WOSA1977DV15800002}
Castaneda, HN. 1977. {``Perception, Belief, and Structure of Physical
Objects and Consciousness.''} \emph{Synthese} 35 (3): 285--351.

\bibitem[\citeproctext]{ref-WOSA1985AAC6100002}
Churchland, PM. 1985. {``Reduction, Qualia, and the Direct Introspection
of Brain States.''} \emph{Journal Of Philosophy} 82 (1): 8--28.

\bibitem[\citeproctext]{ref-WOSA1989AE70300010}
Cohen, GA. 1989. {``On the Currency of Egalitarian Justice.''}
\emph{Ethics} 99 (4): 906--44.

\bibitem[\citeproctext]{ref-WOSA1990DA14600002}
Crane, T, and DH Mellor. 1990. {``There Is No Question of
Physicalism.''} \emph{Mind} 99 (394): 185--206.

\bibitem[\citeproctext]{ref-WOS000087998300003}
D'arms, J, and D Jacobson. 2000. {``The Moralistic Fallacy: On the
'Appropriateness' of Emotions.''} \emph{Philosophy And Phenomenological
Research} 61 (1): 65--90.

\bibitem[\citeproctext]{ref-WOSA1977EA35800003}
Darwall, SL. 1977. {``Two Kinds of Respect.''} \emph{Ethics} 88 (1):
36--49.

\bibitem[\citeproctext]{ref-WOSA1990EQ84600001}
Davidson, D. 1990. {``The Structure and Content of Truth.''}
\emph{Journal Of Philosophy} 87 (6): 279--328.

\bibitem[\citeproctext]{ref-WOSA1991EN62900002}
Dennett, DC. 1991. {``Real Patterns.''} \emph{Journal Of Philosophy} 88
(1): 27--51.

\bibitem[\citeproctext]{ref-WOSA1992KB29500008}
Derose, K. 1992. {``Contextualism and Knowledge Attributions.''}
\emph{Philosophy And Phenomenological Research} 52 (4): 913--29.

\bibitem[\citeproctext]{ref-WOS000166575500001}
Douglas, H. 2000. {``Inductive Risk and Values in Science.''}
\emph{Philosophy Of Science} 67 (4): 559--79.

\bibitem[\citeproctext]{ref-WOSA1983QJ85300002}
Eells, E, and E Sober. 1983. {``Probabilistic Causality and the Question
of Transitivity.''} \emph{Philosophy Of Science} 50 (1): 35--57.

\bibitem[\citeproctext]{ref-WOSA1994NP02400004}
Falvey, K, and J Owens. 1994. {``Externalism, Self-Knowledge, and
Skepticism.''} \emph{Philosophical Review} 103 (1): 107--37.

\bibitem[\citeproctext]{ref-WOS000183806600006}
Fitelson, B. 2003. {``A Probabilistic Theory of Coherence.''}
\emph{Analysis} 63 (3): 194--99.

\bibitem[\citeproctext]{ref-WOSA1991EN62900001}
Fodor, JA. 1991. {``A Modal Argument for Narrow Content.''}
\emph{Journal Of Philosophy} 88 (1): 5--26.

\bibitem[\citeproctext]{ref-WOSA1990EB39300002}
Forbes, G. 1990. {``The Indispensability of Sinn.''} \emph{Philosophical
Review} 99 (4): 535--63.

\bibitem[\citeproctext]{ref-WOSA1994NQ78600001}
Forster, M, and E Sober. 1994. {``How to Tell When Simpler, More
Unified, or Less \_Ad Hoc\_ Theories Will Provide More Accurate
Predictions.''} \emph{British Journal For The Philosophy Of Science} 45
(1): 1--35.

\bibitem[\citeproctext]{ref-WOS000170434600004}
Goldman, AI. 2001. {``Experts: Which Ones Should You Trust?''}
\emph{Philosophy And Phenomenological Research} 63 (1): 85--110.

\bibitem[\citeproctext]{ref-WOSA1994NP54800001}
Griffiths, PE, and RD Gray. 1994. {``Developmental Systems and
Evolutionary Explanation.''} \emph{Journal Of Philosophy} 91 (6):
277--304.

\bibitem[\citeproctext]{ref-WOS000085841900002}
Haslanger, S. 2000. {``Gender and Race: (What) Are They? (What) Do We
Want Them to Be?''} \emph{Noûs} 34 (1): 31--55.

\bibitem[\citeproctext]{ref-WOSA1982NC42600008}
Haugeland, J. 1982. {``Weak Supervenience.''} \emph{American
Philosophical Quarterly} 19 (1): 93--103.

\bibitem[\citeproctext]{ref-WOS000234618400001}
Hieronymi, P. 2005. {``The Wrong Kind of Reason.''} \emph{Journal Of
Philosophy} 102 (9): 437--57.

\bibitem[\citeproctext]{ref-WOSA1988P549200004}
Jackson, F, and P Pettit. 1988. {``Functionalism and Broad Content.''}
\emph{Mind} 97 (387): 381--400.

\bibitem[\citeproctext]{ref-WOSA1992KC39800002}
Johnston, M. 1992. {``How to Speak of the Colors.''} \emph{Philosophical
Studies} 68 (3): 221--63.

\bibitem[\citeproctext]{ref-WOS000077956100002}
Joyce, JM. 1998. {``A Nonpragmatic Vindication of Probabilism.''}
\emph{Philosophy Of Science} 65 (4): 575--603.

\bibitem[\citeproctext]{ref-WOSA1978EL93700009}
Kim, J. 1978. {``Supervenience and Nomological Incommensurables.''}
\emph{American Philosophical Quarterly} 15 (2): 149--56.

\bibitem[\citeproctext]{ref-WOSA1982NC90700004}
---------. 1982. {``Psychophysical Supervenience.''} \emph{Philosophical
Studies} 41 (1): 51--70.

\bibitem[\citeproctext]{ref-WOSA1984TV24600001}
---------. 1984. {``Concepts of Supervenience.''} \emph{Philosophy And
Phenomenological Research} 45 (2): 153--76.

\bibitem[\citeproctext]{ref-WOSA1990FU75100001}
Kim, JW. 1990. {``Supervenience as a Philosophical Concept.''}
\emph{Metaphilosophy} 21 (1-2): 1--27.

\bibitem[\citeproctext]{ref-WOSA1990CH71200001}
Kitcher, P. 1990. {``The Division of Cognitive Labor.''} \emph{Journal
Of Philosophy} 87 (1): 5--22.

\bibitem[\citeproctext]{ref-WOSA1992HF90300002}
---------. 1992. {``The Naturalists Return.''} \emph{Philosophical
Review} 101 (1): 53--114.

\bibitem[\citeproctext]{ref-WOSA1982NV01500001}
Lackey, DP. 1982. {``Missiles and Morals: A Utilitarian Look at Nuclear
Deterrence.''} \emph{Philosophy \& Public Affairs} 11 (3): 189--231.

\bibitem[\citeproctext]{ref-WOSA1987F902200002}
Laudan, L. 1987. {``Progress or Rationality: The Prospects for Normative
Naturalism.''} \emph{American Philosophical Quarterly} 24 (1): 19--31.

\bibitem[\citeproctext]{ref-WOSA1990CX46800004}
---------. 1990. {``Normative Naturalism.''} \emph{Philosophy Of
Science} 57 (1): 44--59.

\bibitem[\citeproctext]{ref-WOSA1979HJ57600007}
Lewis, D. 1979. {``Scorekeeping in a Language Game.''} \emph{Journal Of
Philosophical Logic} 8 (3): 339--59.

\bibitem[\citeproctext]{ref-WOSA1994PM10400005}
---------. 1994. {``Humean Supervenience Debugged.''} \emph{Mind} 103
(412): 473--90.

\bibitem[\citeproctext]{ref-WOSA1991EW83300002}
Mckinsey, M. 1991. {``Anti-Individualism and Privileged Access.''}
\emph{Analysis} 51 (1): 9--16.

\bibitem[\citeproctext]{ref-WOSA1989U850300001}
Millikan, RG. 1989a. {``Biosemantics.''} \emph{Journal Of Philosophy} 86
(6): 281--97.

\bibitem[\citeproctext]{ref-WOSA1989AA09400006}
---------. 1989b. {``In Defense of Proper Functions.''} \emph{Philosophy
Of Science} 56 (2): 288--302.

\bibitem[\citeproctext]{ref-WOSA1987J379200001}
Nagel, T. 1987. {``Moral Conflict and Political Legitimacy.''}
\emph{Philosophy \& Public Affairs} 16 (3): 215--40.

\bibitem[\citeproctext]{ref-WOSA1991FQ15000002}
Neander, K. 1991a. {``Functions as Selected Effects: The Conceptual
Analyst's Defense.''} \emph{Philosophy Of Science} 58 (2): 168--84.

\bibitem[\citeproctext]{ref-WOSA1991GR92500005}
---------. 1991b. {``The Teleological Notion of Function.''}
\emph{Australasian Journal Of Philosophy} 69 (4): 454--68.

\bibitem[\citeproctext]{ref-WOSA1995RP14800001}
---------. 1995. {``Misrepresenting and Malfunctioning.''}
\emph{Philosophical Studies} 79 (2): 109--41.

\bibitem[\citeproctext]{ref-WOSA1995QL53800004}
Pietroski, P, and G Rey. 1995. {``When Other Things Aren't Equal: Saving
Ceteris Paribus Laws from Vacuity.''} \emph{British Journal For The
Philosophy Of Science} 46 (1): 81--110.

\bibitem[\citeproctext]{ref-WOSA1994PF23100001}
Putnam, H. 1994. {``Sense, Nonsense, and the Senses: An Inquiry into the
Powers of the Human Mind.''} \emph{Journal Of Philosophy} 91 (9):
445--65.

\bibitem[\citeproctext]{ref-WOSA1985APA8500001}
Rawls, J. 1985. {``Justice as Fairness: Political Not Metaphysical.''}
\emph{Philosophy \& Public Affairs} 14 (3): 223--51.

\bibitem[\citeproctext]{ref-WOSA1988Q394000001}
---------. 1988. {``The Priority of Right and Ideas of the Good.''}
\emph{Philosophy \& Public Affairs} 17 (4): 251--76.

\bibitem[\citeproctext]{ref-WOSA1990DR99100001}
Rosen, G. 1990. {``Modal Fictionalism.''} \emph{Mind} 99 (395): 327--54.

\bibitem[\citeproctext]{ref-WOSA1994NQ04900008}
Salmon, WC. 1994. {``Causality Without Counterfactuals.''}
\emph{Philosophy Of Science} 61 (2): 297--312.

\bibitem[\citeproctext]{ref-WOSA1992JQ78400001}
Schiffer, S. 1992. {``Belief Ascription.''} \emph{Journal Of Philosophy}
89 (10): 499--521.

\bibitem[\citeproctext]{ref-WOS000073301000002}
Sider, T. 1997. {``Four Dimensionalism.''} \emph{Philosophical Review}
106 (2): 197--231.

\bibitem[\citeproctext]{ref-WOS000227058600002}
Smith, AM. 2005. {``Responsibility for Attitudes: Activity and Passivity
in Mental Life.''} \emph{Ethics} 115 (2): 236--71.

\bibitem[\citeproctext]{ref-WOSA1971Y116900003}
Thomson, JJ. 1971. {``A Defense of Abortion.''} \emph{Philosophy \&
Public Affairs} 1 (1): 47--66.

\bibitem[\citeproctext]{ref-WOSA1990EG85000002}
Tye, M. 1990. {``Vague Objects.''} \emph{Mind} 99 (396): 535--57.

\bibitem[\citeproctext]{ref-WOSA1992JA62400001}
Yablo, S. 1992. {``Mental Causation.''} \emph{Philosophical Review} 101
(2): 245--80.

\bibitem[\citeproctext]{ref-WOSA1995RC31600002}
Zimmerman, DW. 1995. {``Theories of Masses and Problems of
Constitution.''} \emph{Philosophical Review} 104 (1): 53--110.

\end{CSLReferences}



\noindent Published online in September 2024.

\end{document}
