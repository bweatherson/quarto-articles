% Options for packages loaded elsewhere
\PassOptionsToPackage{unicode}{hyperref}
\PassOptionsToPackage{hyphens}{url}
%
\documentclass[
  10pt,
  letterpaper,
  DIV=11,
  numbers=noendperiod,
  twoside]{scrartcl}

\usepackage{amsmath,amssymb}
\usepackage{setspace}
\usepackage{iftex}
\ifPDFTeX
  \usepackage[T1]{fontenc}
  \usepackage[utf8]{inputenc}
  \usepackage{textcomp} % provide euro and other symbols
\else % if luatex or xetex
  \usepackage{unicode-math}
  \defaultfontfeatures{Scale=MatchLowercase}
  \defaultfontfeatures[\rmfamily]{Ligatures=TeX,Scale=1}
\fi
\usepackage{lmodern}
\ifPDFTeX\else  
    % xetex/luatex font selection
  \setmainfont[ItalicFont=EB Garamond Italic,BoldFont=EB Garamond
Bold]{EB Garamond Math}
  \setsansfont[]{Europa-Bold}
  \setmathfont[]{Garamond-Math}
\fi
% Use upquote if available, for straight quotes in verbatim environments
\IfFileExists{upquote.sty}{\usepackage{upquote}}{}
\IfFileExists{microtype.sty}{% use microtype if available
  \usepackage[]{microtype}
  \UseMicrotypeSet[protrusion]{basicmath} % disable protrusion for tt fonts
}{}
\usepackage{xcolor}
\usepackage[left=1in, right=1in, top=0.8in, bottom=0.8in,
paperheight=9.5in, paperwidth=6.5in, includemp=TRUE, marginparwidth=0in,
marginparsep=0in]{geometry}
\setlength{\emergencystretch}{3em} % prevent overfull lines
\setcounter{secnumdepth}{3}
% Make \paragraph and \subparagraph free-standing
\ifx\paragraph\undefined\else
  \let\oldparagraph\paragraph
  \renewcommand{\paragraph}[1]{\oldparagraph{#1}\mbox{}}
\fi
\ifx\subparagraph\undefined\else
  \let\oldsubparagraph\subparagraph
  \renewcommand{\subparagraph}[1]{\oldsubparagraph{#1}\mbox{}}
\fi


\providecommand{\tightlist}{%
  \setlength{\itemsep}{0pt}\setlength{\parskip}{0pt}}\usepackage{longtable,booktabs,array}
\usepackage{calc} % for calculating minipage widths
% Correct order of tables after \paragraph or \subparagraph
\usepackage{etoolbox}
\makeatletter
\patchcmd\longtable{\par}{\if@noskipsec\mbox{}\fi\par}{}{}
\makeatother
% Allow footnotes in longtable head/foot
\IfFileExists{footnotehyper.sty}{\usepackage{footnotehyper}}{\usepackage{footnote}}
\makesavenoteenv{longtable}
\usepackage{graphicx}
\makeatletter
\def\maxwidth{\ifdim\Gin@nat@width>\linewidth\linewidth\else\Gin@nat@width\fi}
\def\maxheight{\ifdim\Gin@nat@height>\textheight\textheight\else\Gin@nat@height\fi}
\makeatother
% Scale images if necessary, so that they will not overflow the page
% margins by default, and it is still possible to overwrite the defaults
% using explicit options in \includegraphics[width, height, ...]{}
\setkeys{Gin}{width=\maxwidth,height=\maxheight,keepaspectratio}
% Set default figure placement to htbp
\makeatletter
\def\fps@figure{htbp}
\makeatother
% definitions for citeproc citations
\NewDocumentCommand\citeproctext{}{}
\NewDocumentCommand\citeproc{mm}{%
  \begingroup\def\citeproctext{#2}\cite{#1}\endgroup}
\makeatletter
 % allow citations to break across lines
 \let\@cite@ofmt\@firstofone
 % avoid brackets around text for \cite:
 \def\@biblabel#1{}
 \def\@cite#1#2{{#1\if@tempswa , #2\fi}}
\makeatother
\newlength{\cslhangindent}
\setlength{\cslhangindent}{1.5em}
\newlength{\csllabelwidth}
\setlength{\csllabelwidth}{3em}
\newenvironment{CSLReferences}[2] % #1 hanging-indent, #2 entry-spacing
 {\begin{list}{}{%
  \setlength{\itemindent}{0pt}
  \setlength{\leftmargin}{0pt}
  \setlength{\parsep}{0pt}
  % turn on hanging indent if param 1 is 1
  \ifodd #1
   \setlength{\leftmargin}{\cslhangindent}
   \setlength{\itemindent}{-1\cslhangindent}
  \fi
  % set entry spacing
  \setlength{\itemsep}{#2\baselineskip}}}
 {\end{list}}
\usepackage{calc}
\newcommand{\CSLBlock}[1]{\hfill\break\parbox[t]{\linewidth}{\strut\ignorespaces#1\strut}}
\newcommand{\CSLLeftMargin}[1]{\parbox[t]{\csllabelwidth}{\strut#1\strut}}
\newcommand{\CSLRightInline}[1]{\parbox[t]{\linewidth - \csllabelwidth}{\strut#1\strut}}
\newcommand{\CSLIndent}[1]{\hspace{\cslhangindent}#1}

\setlength\heavyrulewidth{0ex}
\setlength\lightrulewidth{0ex}
\usepackage[automark]{scrlayer-scrpage}
\clearpairofpagestyles
\cehead{
  Brian Weatherson
  }
\cohead{
  Epistemic Permissiveness and Symmetric Games
  }
\ohead{\bfseries \pagemark}
\cfoot{}
\makeatletter
\newcommand*\NoIndentAfterEnv[1]{%
  \AfterEndEnvironment{#1}{\par\@afterindentfalse\@afterheading}}
\makeatother
\NoIndentAfterEnv{itemize}
\NoIndentAfterEnv{enumerate}
\NoIndentAfterEnv{description}
\NoIndentAfterEnv{quote}
\NoIndentAfterEnv{equation}
\NoIndentAfterEnv{longtable}
\NoIndentAfterEnv{abstract}
\renewenvironment{abstract}
 {\vspace{-1.25cm}
 \quotation\small\noindent\rule{\linewidth}{.5pt}\par\smallskip
 \noindent }
 {\par\noindent\rule{\linewidth}{.5pt}\endquotation}
\KOMAoption{captions}{tableheading}
\makeatletter
\@ifpackageloaded{caption}{}{\usepackage{caption}}
\AtBeginDocument{%
\ifdefined\contentsname
  \renewcommand*\contentsname{Table of contents}
\else
  \newcommand\contentsname{Table of contents}
\fi
\ifdefined\listfigurename
  \renewcommand*\listfigurename{List of Figures}
\else
  \newcommand\listfigurename{List of Figures}
\fi
\ifdefined\listtablename
  \renewcommand*\listtablename{List of Tables}
\else
  \newcommand\listtablename{List of Tables}
\fi
\ifdefined\figurename
  \renewcommand*\figurename{Figure}
\else
  \newcommand\figurename{Figure}
\fi
\ifdefined\tablename
  \renewcommand*\tablename{Table}
\else
  \newcommand\tablename{Table}
\fi
}
\@ifpackageloaded{float}{}{\usepackage{float}}
\floatstyle{ruled}
\@ifundefined{c@chapter}{\newfloat{codelisting}{h}{lop}}{\newfloat{codelisting}{h}{lop}[chapter]}
\floatname{codelisting}{Listing}
\newcommand*\listoflistings{\listof{codelisting}{List of Listings}}
\makeatother
\makeatletter
\makeatother
\makeatletter
\@ifpackageloaded{caption}{}{\usepackage{caption}}
\@ifpackageloaded{subcaption}{}{\usepackage{subcaption}}
\makeatother
\ifLuaTeX
  \usepackage{selnolig}  % disable illegal ligatures
\fi
\IfFileExists{bookmark.sty}{\usepackage{bookmark}}{\usepackage{hyperref}}
\IfFileExists{xurl.sty}{\usepackage{xurl}}{} % add URL line breaks if available
\urlstyle{same} % disable monospaced font for URLs
\hypersetup{
  pdftitle={Epistemic Permissiveness and Symmetric Games},
  pdfauthor={Brian Weatherson},
  hidelinks,
  pdfcreator={LaTeX via pandoc}}

\title{Epistemic Permissiveness and Symmetric Games}
\author{Brian Weatherson}
\date{2021}

\begin{document}
\maketitle
\begin{abstract}
Permissivism in epistemology is a family of theses, each of which says
that rationality is compatible with a number of distinct attitudes. This
paper argues that thinking about symmetric games gives us new reason to
believe in permissivism. In some finite games, if permissivism is false
then we have to think that a player is more likely to take one option
rather than another, even though each have the same expected return
given that player's credences. And in some infinite games, if
permissivism is false there is no rational way to play the game,
although intuitively the games could be rationally played. The latter
set of arguments rely on the recent discovery that there are symmetric
games with only asymmetric equilibria. It was long known that there are
symmetric games with no pure strategy symmetric equilibria; the
surprising new discovery is that there are symmetric games with
asymmetric equilibria, but no symmetric equilibria involving either
mixed or pure strategies.
\end{abstract}

\setstretch{1.1}
\section{Introduction}\label{introduction}

Permissivism in epistemology is a family of theses, each of which says
that rationality is compatible with a number of distinct attitudes. This
paper argues that thinking about symmetric games gives us new reason to
believe in Permissivism. I'm going to offer two arguments, one involving
finite games, and the other involving infinite games. In finite games,
the theorist who denies Permissivism says that the players have to think
that the other player is more likely to take one action rather than
another, although they know the actions have equal expected utility. In
infinite games, the theorist who denies Permissivism has to say that it
is impossible for certain games to be played with common knowledge of
rationality and shared evidence, although there does not seem to be
anything paradoxical about the games. The latter set of arguments rely
on the recent discovery that there are symmetric games with only
asymmetric equilibria. It was long known that there are symmetric games
with no pure strategy symmetric equilibria; the surprising new discovery
is that there are symmetric games with asymmetric equilibria, but no
symmetric equilibria involving either mixed or pure strategies. In both
cases, thinking about players in symmetric games pushes us towards
accepting Permissivism.

The Permissivist theses that have been the focus on recent philosophical
attention vary along two dimensions.\footnote{For a much more thorough
  introduction to the debate, and especially into the varieties of
  Permissivist theses, see Kopec and Titelbaum
  (\citeproc{ref-KopecTitelbaum2016}{2016}) and Meacham
  (\citeproc{ref-Meacham2019}{2019}). The next three paragraphs draw
  heavily from these two papers. For more recent arguments in favor of
  Permissivism, see Callahan (\citeproc{ref-Callahan2021}{2021}), Lota
  and Hlobil (\citeproc{ref-Lota2023}{2023}), Palmira
  (\citeproc{ref-Palmira2023}{2023}), and Ye
  (\citeproc{ref-Ye2023}{2023}). For criticisms, see Schultheis
  (\citeproc{ref-Schultheis2018}{2018}) and Ross
  (\citeproc{ref-Ross2021}{2021}).}

The first dimension concerns what we hold fixed when we say that
multiple attitudes are rationally permissible. It's basically common
ground that people with different evidence can rationally believe
different things, or that a person can believe different things when
their evidence changes. But are these sufficient conditions for rational
disagreement or change also necessary conditions. What's called
\emph{Interpersonal Permissiveness} says that given some evidence, there
may be more than one doxastic state that it is rational to be
in.\footnote{In setting up the debate, I'm following the standard
  practice and assuming some kind of evidentialism. If one thinks that
  other things than evidence matter to rational credence, such as
  thinking that testimony provides non-evidential reasons for belief, it
  is a little complicated but possible to restate everything here to fit
  such an epistemological view. The general picture is that `evidence'
  here really means non-pragmatic reasons for belief. But it's simpler
  to say `evidence', and that's what I'll generally say.} What's called
\emph{Intrapersonal Permissiveness} says that given a person and some
evidence, there may be more one doxastic state that it is rational to be
in. A classic form of subjectivist Bayesianism says that a person can
pick any prior they like, but they have to update it by
conditionalisation. This is a version of a view that rejects
Intrapersonal Permissiveness, since given one's prior there is only one
permissible posterior, but endorses Interpersonal Permissiveness, since
there are multiple permissible priors.

The second dimension concerns whether the distinct views can be
acknowledged as rational. Stewart Cohen (\citeproc{ref-Cohen2013}{2013})
defends the view that multiple attitudes can be rational, but one cannot
rationally acknowledge a distinct view as rational. Following Kopec and
Titelbaum (\citeproc{ref-KopecTitelbaum2016}{2016}), call a view that
says multiple views can both be rationally held and be believed to be
rational \emph{Acknowledged Permissiveness}, and the view that says
multiple views could be rationally held, but could not be acknowledged,
\emph{Unacknowledged Permissiveness}.

Putting the last two paragraphs together gives us four varieties of
Permissive theses. The negation of a Permissive thesis is a Uniqueness
thesis. The name suggests that there is precisely one rational attitude
to take in a specified situation, but we'll interpret it as the view
that there is at most one rational attitude to take so as to ensure each
Uniqueness thesis is the negation of a Permissive thesis. So if there
are four Permissive theses, there are four Uniqueness theses that negate
them.

This paper focusses on the strongest of these four Permissive theses,
Acknowledged Intrapersonal Permissiveness, and its negation,
Acknowledged Interpersonal Uniqueness. This is, I think, the most
commonly discussed form of Permissiveness in the literature, and in any
case is interesting. And I'm going to be arguing that in some cases
involving symmetric games, coherence considerations provide a strong
argument in favor of Acknowledged Intrapersonal Permissiveness.

I am assuming that it makes sense to inquire into theses like Uniqueness
and Permissivism in the artificial context set up by orthodox decision
theory and game theory. Now this might seem like a strong assumption,
since some of the motivations for Permissiveness come from the ways in
which the messy real world differs from the idealised contexts that game
theory and decision theory typically live in. This assumption should be
allowed for two reasons. First, if the `messiness' is a motivation for
Permissiveness, then the fact that we can motivate Permissiveness while
assuming away the mess is good news for Permissiveness; it can win the
argument with one hand tied behind its back. Second, the arguments
against Permissiveness are often extremely general. Consider, for
instance, the anti-arbitrariness arguments that Meacham
(\citeproc{ref-Meacham2019}{2019}) discusses (and critiques). These
arguments purport to impose very general constraints on rational
thought. If the constraints hold, they hold everywhere. So if I can show
they don't hold somewhere, this undermines the arguments for Uniqueness.
That doesn't quite show that Permissiveness is true; there's a gap
between critiquing an argument and showing its negation is true. But it
does mean that even if one doesn't want to draw too many real world
lessons from artificial cases like the ones I'm discussing, and thinks
that the real world should be addressed directly rather than via
artificial models, I'll still have shown something. Namely, I'll have
shown that principles Uniqueness theorists often use can't be right.

When I say I'm working in this artificial context, I mean that I'll make
the following four assumptions. First, I assume that everyone is
self-aware in the sense that they know what their evidence is, and what
they are doing. Second, I assume that everyone is logically omniscient,
and has doxastic states that are coherent, and closed under logical
entailments. Third, I assume that everyone maximises expected utility.
Fourth, I assume that everyone is coherent, in the sense that they do
not have beliefs that they believe are irrational, and they do not
perform actions that they believe do not maximise expected utility.
These are very strong assumptions, and obviously real world people are
not like this. But they are also assumptions that tend to make
Uniqueness more plausible. It's much more plausible that Uniqueness is
true for idealised people and false for normal people than that it is
false for idealised people, but true for normal people. So moving to
this idealised context is not tilting the playing field in my favor.

There is one technical challenge in even stating the debate in this
framework. We're interested in the question of whether it's possible for
two people, who know they have the same evidence, but who also know that
they have different doxastic states, can acknowledge that each is
rational. But given the background assumptions, it's not clear how this
is possible. Given self-awareness, the two people, A and B, will each
know their own identity. But that means they will have different
beliefs; A will believe \emph{I'm A}, and B will believe \emph{I'm B}.
And that in turn either shows very quickly that Uniqueness is false,
since these self-identifying beliefs are distinct and rational, or (more
plausibly) that they don't really have the same evidence.

The solution to this technical challenge is two-fold. First, we say,
following Lewis (\citeproc{ref-Lewis1979a}{1979}), that the relevant
evidence and belief is de se rather than de dicto. Second,
self-awareness does not include identifying information. So A knows
things like \emph{The sky is blue}, and \emph{My evidence includes that
the sky is blue}, but not \emph{A's evidence includes that the sky is
blue.} Lewis's views on de se content are hardly uncontroversial, and
this is a somewhat artificial move. But without it we end up in a
contradiction; we need that two people can have the same evidence, and
know what their evidence is, and that is impossible if each person knows
their identity, or their identity is part of their evidence. So I'll
simply flag that I'm assuming the relevant evidence is de se, and note
that there might be some complications arising from this assumption.

The next two sections set out two symmetric games where
Uniqueness\footnote{From now on, when I say `Uniqueness' I mean
  Acknowledged Intrapersonal Uniqueness. It's easier to simply stipulate
  this now than repeating the phrase every time.} leads to surprising
results. In all cases I'll assume that if Uniqueness is true, then the
players know that it is true. This is arguably something that follows
from the standard idealising assumptions, since the idealised players
have internalised a full theory of rationality. But in case it isn't,
I'll note it here, and come back to this assumption at the end of the
paper. And while this might be obvious, I'll also note explicitly that
they players have no evidence about the game or the players beyond what
I write about the structure of the games.

\section{Chicken}\label{chicken}

Some finite symmetric games don't have a symmetric pure-strategy
equilibrium. One notable example is Chicken, one version of which is in
table Table~\ref{tbl-chicken}.\footnote{When presenting games in this
  format, I'll write Row's payout first, then Column's payout. So the
  top right cell here, for example, says that if Row plays Stay and
  Column plays Swerve, then Row gets a payout of 1, and Column gets a
  payout of -1. All payouts are in utils unless otherwise stated. For
  people unfamiliar with it, the backstory of Chicken is that the
  players are in cars driving towards each other on a one-lane road.
  They can stay on the road, possibly winning points for courage and
  possibly dying, or swerve off.}

\begin{longtable}[]{@{}lcc@{}}
\caption{A simple version of chicken.}\label{tbl-chicken}\tabularnewline
\toprule\noalign{}
& Stay & Swerve \\
\midrule\noalign{}
\endfirsthead
\toprule\noalign{}
& Stay & Swerve \\
\midrule\noalign{}
\endhead
\bottomrule\noalign{}
\endlastfoot
Stay & -100, -100 & 1, -1 \\
Swerve & -1, 1 & 0, 0 \\
\end{longtable}

The symmetric pure-strategy pairs \(\langle\)Stay,~Stay\(\rangle\) and
\(\langle\) Swerve,~Swerve\(\rangle\) are not equilibria; in each case
both parties have an incentive to defect. But the game does have a
symmetric mixed strategy equilibrium. It is that both players play the
mixed strategy of Stay with probability 0.01, and Swerve with
probability 0.99.

Now assume that Row and Column have the same de se evidence, that they
are self-aware and fully rational, and that these facts and no other are
common knowledge between them, and that they are about to play Chicken
one time. (It's also common knowledge that they won't play again;
dropping this would raise the possibility of complicated strategic
reasoning.)

Let \textbf{Swerve} be the proposition that a rational player with that
evidence will Swerve. And call the players Row and Column. Given our
assumptions so far, plus Uniqueness, we can prove that Row's credence in
\textbf{Swerve} is 0.99. Here's the proof.

\begin{enumerate}
\def\labelenumi{\arabic{enumi}.}
\tightlist
\item
  Let \emph{x} be Row's credence in \textbf{Swerve}.
\item
  By self-awareness, Row knows that \emph{x} is her credence in
  \textbf{Swerve}.
\item
  Since Row knows Row is rational, Row can infer that \emph{x} is a
  rational credence in \textbf{Swerve}.
\item
  Since Row knows Uniqueness is true, Row can infer that \emph{x} is the
  only rational credence in \textbf{Swerve}.
\item
  Since Row knows Column is rational, Row can infer that \emph{x} is
  Column's credence in \textbf{Swerve}, since (at step 4) Row has
  deduced that \emph{x} is the only rational credence in
  \textbf{Swerve}.
\item
  Since all the assumptions so far are common knowledge, Row can come to
  know that Column knows that \emph{x} is Row's credence in
  \textbf{Swerve}.
\item
  If \emph{x}~= 1, then Row can come to know that it is rational for
  Column to Swerve, while knowing that Row will also Swerve. But this is
  impossible, since if Column knows Row will Swerve, it is best for
  Column to Stay. So \emph{x}~≠~1.
\item
  If \emph{x}~= 0, then Row can come to know that it is rational for
  Column to Stay, while knowing that Row will also Stay. But this is
  impossible, since if Column knows Row will Stay, it is best for Column
  to Swerve. So \emph{x}~≠~0.
\item
  So 0~\textless~\emph{x}~\textless{} 1.
\item
  Since Row knows Column's credence that Row will Swerve (as was shown
  at step 6), and Row knows Column is rational, but Row does not know
  what Column will do, it must be that Column is indifferent between
  Stay and Swerve given her (i.e., Column's) credences about what Row
  will do.\footnote{If Column was not indifferent between their options,
    the knowledge Row has by step 6 would be sufficient to deduce with
    certainty what Column will do. But at step 9 we showed that Row does
    not know what Column will do.}
\item
  Column is indifferent between Stay and Swerve only if her credence
  that Row will Swerve is 0.99. (This is a reasonably simple bit of
  algebra to prove.)
\item
  So from 10 and 11, Column's credence that Row will Swerve is 0.99.
\item
  By (known) Uniqueness, it follows that the only rational credence in
  \textbf{Swerve} is 0.99.
\item
  So since Row is rational, it follows that \emph{x}~= 0.99.
\end{enumerate}

Now there is nothing inconsistent in this reasoning. In a sense, it is
purely textbook reasoning. But the conclusion is deeply puzzling. We've
proven that Column is indifferent between her two options. And we've
proven that Row knows this. But we've also proven that Row thinks it is
99 times more likely that Column will choose one of the options over the
other. Why is that? It isn't because there is more reason to do one than
the other; given Column's attitudes, the options are equally balanced.
It is purely because Uniqueness pushes us to a symmetric equilibrium,
and this is the only symmetric equilibrium. Given Uniqueness, the only
coherent state is to have believe the other party is 99 times more
likely to resolve a tie one way rather than another.

It's important here that we're imagining a one-shot version of Chicken.
If the game is played repeatedly, then it is natural that the players
will tend to the equilibrium of the game. What is surprising is that
Uniqueness pushes us to thinking that, if the rationality of the players
is common knowledge, the mixed strategy Nash equilibrium will be played
in a one-shot game. As Matthias Risse (\citeproc{ref-Risse2000}{2000})
argues, the argument that mixed strategy Nash equilibria are rational
requirements of one-shot games is very weak. But it's a conclusion the
Uniqueness theorist is forced into.

It's even more puzzling because of another feature of Uniqueness. It's
often very hard to see what the uniquely rational attitude could be in
cases where the evidence is very sparse. The usual way to resolve this
problem is to appeal to what Keynes (\citeproc{ref-Keynes1921}{1921})
dubbed the Principle of Indifference. That principle says, roughly, that
if the evidence available for two options is equally good, treat them as
equally likely. Here, Row thinks that Column is a utility maximiser who
has two options of equal utility available to them. And Row concludes
(and must conclude if Uniqueness is correct) that one of these options
is 99 times more likely to be played. That's not inconsistent with the
letter of the Principle of Indifference. But it is inconsistent with the
spirit of it.

All that said, I suspect many defenders of Uniqueness will be happy to
accept these conclusions. The next case is I think poses a deeper
problem for them.

\section{Elections}\label{elections}

The cases in this section come from some recent work on a rather old
question,

\begin{quote}
If a symmetric game has an equilibrium, does it have a symmetric
equilibrium?
\end{quote}

Over the years, a positive answer was given to various restricted forms
of that question. Most importantly, John Nash
(\citeproc{ref-Nash1951}{1951}) showed that if each player has finitely
many moves available, then the game does have a symmetric equilibrium.

But recently it has been proven that the answer to the general question
is no. Mark Fey (\citeproc{ref-Fey2012}{2012}) describes an example of a
positive-sum two-player game that has only asymmetric
equilibria.\footnote{In Fey's game both players pick a real in {[}0,
  1{]}. If both players pick numbers in (0, 1), the one who picks the
  larger number wins. But there are a lot of complications if one or
  both pick an extreme value, including the game not always being
  zero-sum. I'm not relying on it here because it is a little too close
  to the game I'll discuss at the end of this section where everyone
  agrees there is no way to play it given common knowledge of
  rationality. Fey's paper also includes a nice chronology of some of
  the proofs of positive answers to restricted forms of the question.}
Dimitrios Xefteris (\citeproc{ref-Xefteris2015}{2015}) showed that there
is a symmetric three-player zero-sum game that has only asymmetric
equilibria. In fact, he showed that a very familiar game, a version of a
Hotelling--Downs model of elections, has this property. Here's how he
describes the game.

\begin{quote}
Consider a unit mass of voters. Each voter is characterized by her ideal
policy. We assume that the ideal policies of the voters are uniformly
distributed in {[}0, 1{]}. We moreover assume that three candidates
\emph{A}, \emph{B} and \emph{C} compete for a single office. Each
candidate \emph{J}~∈~\{\emph{A},~\emph{B},~\emph{C}\} announces a policy
\emph{s\textsubscript{J}}~∈~{[}0,~1{]} and each voter votes for the
candidate who announced the policy platform which is nearest to her
ideal policy. If a voter is indifferent between two or among all three
candidates she evenly splits her vote between/among them. A candidate
\emph{J}~∈~\{\emph{A},~\emph{B},~\emph{C}\} gets a payoff equal to one
if she receives a vote-share strictly larger than the vote-share of each
of the two other candidates. If two candidates tie in the first place
each gets a payoff equal to one half. If all three candidates receive
the same vote-shares then each gets a payoff equal to one third. In all
other cases a candidate gets a payoff equal to zero.
(\citeproc{ref-Xefteris2015}{Xefteris 2015, 124})
\end{quote}

It is clear that there is no symmetric pure-strategy equilibrium here.
If all candidates announced the same policy, everyone would get a payoff
of ⅓. But no matter what that strategy is, if \emph{B} and \emph{C}
announce the same policy, then \emph{A} has a winning move available.
(If the number \emph{B} and \emph{C} say is not 0.5, \emph{A} wins by
saying 0.5. If they do both say 0.5, then \emph{A} wins by saying 0.4.)

What's more surprising, and what Xefteris proves, is that there is no
symmetric mixed strategy equilibria either. Again, in such an
equilibrium, any player would have a payoff of ⅓. Very roughly, the
proof that no such equilibrium exists is that random deviations from the
equilibrium are as likely to lead to winning as losing, so they have a
payoff of roughly ½. So there is no incentive to stay in the
equilibrium. So no symmetric equilibrium exists.

Using this game, I'm going to offer the following argument for
Uniqueness.

\begin{enumerate}
\def\labelenumi{\arabic{enumi}.}
\tightlist
\item
  It's possible that three people can play this (symmetric) game in a
  situation where it is commonly known that (a) each of them is
  rational, and (b) they have the same evidence. In the relevant sense
  of `rational' a person is rational iff they have credences that are
  supported by their evidence, and they perform actions that maximise
  expected utility given their credences.
\item
  If Uniqueness is true, then in a symmetric game where it is common
  knowledge that each person has the same evidence and is rational,
  every player will believe that the others have the same credences
  about what the others will do.
\item
  If premise 2 is true, then if the players have the same evidence and
  are rational, the result of the game will be a symmetric equilibrium.
\item
  The game does not have a symmetric equilibrium.
\item
  So Uniqueness is false.
\end{enumerate}

I'll go over this abstract argument, then apply it to the Xefteris game.

Premise 1 is a claim that a particular game, that has an equilibrium, is
possible to play. There is a strong assumption that mathematically
coherent games are indeed possible to play, so this feels like a safe
enough assumption. I'll come back at the very end of the next section to
whether it is safe.

Premise 2 is spelling out a consequence of Uniqueness, but it helps to
go over why it is a consequence. Assume that player \emph{x} has
credence \emph{p} that player \emph{y} will play (pure) strategy
\emph{s}. In symbols,
\emph{Cr\textsubscript{x}}(\emph{s\textsubscript{y}})~=~\emph{p}, where
\emph{Cr\textsubscript{x}} is \emph{x}'s credence function, and
\emph{s\textsubscript{y}} is the proposition that \emph{y} plays
strategy \emph{s}. By common knowledge of rationality, \emph{x} thinks
it is rational with their evidence to have credence \emph{p} in
\emph{s\textsubscript{y}}. By Uniqueness, \emph{x} thinks this is the
only rational credence to have in \emph{s\textsubscript{y}}. By common
knowledge of sameness of evidence, \emph{x} thinks that \emph{y} is
rational, and has the same evidence about \emph{x} playing \emph{s} as
\emph{x} has about \emph{y} playing \emph{s}. So \emph{y} will do the
only rational thing with that evidence, namely form credence \emph{p}
that \emph{x} will play \emph{s}. (In a game with more than two players,
this also licences inferring that \emph{y} believes that \emph{z} will
play \emph{s} with probability \emph{p}, and the same for all the other
players.) Quite generally, \emph{x} believes that \emph{y} has the same
credences about \emph{x} as \emph{x} themselves has about \emph{y}.

Premise 3 says that this suffices for there to be a symmetric
equilibrium of the game. In fact, we can say what that equilibrium would
be: everyone plays the mixed strategy that corresponds to \emph{x}'s
credences about what \emph{y} will play. By 'mixed strategy
corresponding to these credences, I mean that if
\emph{Cr\textsubscript{x}}(\emph{s\textsubscript{y}})~=~\emph{p}, then
each player \emph{y} in fact plays strategy \emph{s} with probability
\emph{p}, and so on for all strategies, and probabilities.

Why is that strategy set, where everyone does what \emph{x} thinks
\emph{y} will do, an equilibrium?\footnote{Remember that an equilibrium
  here means that if everyone else knew what everyone else was doing,
  they would have no incentive to do anything other than what they are
  doing. If they are playing a mixed strategy, that means that every
  strategy they are mixing between has the same expected value \emph{x},
  and no strategy has a higher expected value than \emph{x}.} It starts
with the fact that premise 2, and the reasoning behind it, is all a
priori. So it's knowable to a perfectly rational player, like \emph{x}.
So, if Uniqueness is true, \emph{x} knows that whatever they think about
the other players will (a) be true, and (b) be common knowledge. And
since \emph{x} takes the other players to be utility maximisers, that
means that every strategy \emph{x} gives them positive probability of
playing must maximise expected utility given these (shared) credences
about what everyone else will play. If there was some strategy that did
not maximise expected utility, and \emph{x} gave them positive
probability of playing it, then \emph{x} would think it was possible
that the other player was not maximising expected utility, contradicting
the assumption of common knowledge of rationality. That's to say,
playing the mixed strategy that corresponds to \emph{x}'s beliefs about
\emph{y} must be an equilibrium, if Uniqueness is true, and it is common
knowledge that the players have the same evidence and are rational.

Premise 4 says that something which is entailed by Uniqueness, combined
with the assumptions in premise 1 and the other two premises, is not in
fact true. So by modus tollens, Uniqueness is not true.

The arguments about the first three premises should apply to any
symmetric game, with common knowledge of rationality and shared
evidence. In any such game, the credences any player has about any other
should be convertible into a (possibly mixed) equilibrium of the game.
That's because any player should be able to conclude that what they
think about one player must be a rational belief (by the common
knowledge of rationality), so must be the only rational belief given
their own evidence (by Uniqueness), so must be the belief that everyone
has (by shared evidence), so must in fact be correct (since it's the
only rational belief, and they are rational), and since it is a correct
belief, and everyone is in fact a utility maximiser while holding these
(correct) beliefs about everyone, must be an equilibrium. So every
symmetric game must have a symmetric equilibrium, if Uniqueness is true,
and the game is played under conditions of common knowledge of
rationality and shared evidence.

That's not true in this game. It does not in fact have a symmetric
equilibrium. If we drop Uniqueness, it is easy enough to describe
rational behavior for players in this game. Here is one possible model
for the game.

\begin{itemize}
\tightlist
\item
  \emph{A} plays 0.6 (and wins), \emph{B} and \emph{C} each play 0.4
  (and lose).
\item
  Each player has a correct belief about what the other players will
  play.
\item
  But both \emph{B} and \emph{C} know they cannot win given the other
  player's moves, so they pick a move completely arbitrarily.
\item
  Further, each player has a correct belief about why each player makes
  the move they make.
\end{itemize}

This is the coherent equilibria that Xefteris describes, but it requires
some amount of luck, since it requires that \emph{B} and \emph{C} pick
0.4 when they could pick absolutely anything. Here's a slightly more
plausible model of the game.

\begin{itemize}
\tightlist
\item
  \emph{A} plays 0.6 (and wins), \emph{B} and \emph{C} each play 0.4
  (and lose).
\item
  The only two rational plays are 0.4 and 0.6, and each of them is
  rationally permissible.
\item
  In any world that a player believes to be actual, or a player believes
  another player believes to be actual, or a player believes another
  player believes another player believes to be actual, etc., the
  following two conditions hold.
\item
  If a player plays 0.6, they believe the other two players will play
  0.4, and hence playing 0.6 is a winning move.
\item
  If a player plays 0.4, they believe the other two players will play
  0.6, and hence playing 0.4 is a winning move.
\end{itemize}

The main difference between this model and Xefteris's is that it allows
that players have false beliefs. But why shouldn't they have false
beliefs? All they know is that the other players are rational, and
rationality (we're assuming) does not settle a unique verdict for what
players will do.\footnote{To use the game-theory jargon, Xefteris
  describes a Nash equilibrium of the game, but what I've described is a
  a rationalizable strategy triple (\citeproc{ref-Bernheim1984}{Bernheim
  1984, Pearce1984}). If Uniqueness is true, then strictly speaking any
  rationalizable strategy n-tuple for a symmetric game is a symmetric
  Nash equilibrium.} So I think this strategy set, where the players
have rational (but false) beliefs about the other players, is more
useful to think about.

\section{Objections}\label{objections}

The arguments for premises 2 and 3 in the argument above assumed
something slightly stronger than Uniqueness. They each assumed that each
player knew Uniqueness was true, and could use that in their reasoning.
(Most importantly, in the argument for premise 3, it's important that
each player can reason through the reasoning behind premise 2, and that
reasoning used Uniqueness.) What happens if we drop that assumption, and
consider the possibility that Uniqueness is true but unknowable?

This possibility is a little uncomfortable for philosophical defenders
of Uniqueness. If the players in these games do not know that Uniqueness
is true, then neither do the authors writing about Uniqueness. And now
we have to worry about whether it is permissible to assert in print that
Uniqueness is true. I wouldn't make too much of this though. It is
unlikely that a knowledge norm governs assertion in philosophical
journals.

The bigger worry here is that one key argument for Uniqueness seems to
require that Uniqueness is knowable. A number of recent authors have
argued that Uniqueness best explains our practice of deferring to
rational people.\footnote{There is a nice discussion of this argument,
  including citations of the papers I'm about to discuss, in Kopec and
  Titelbaum (\citeproc{ref-KopecTitelbaum2016}{2016, 195}).} For
instance, Greco and Hedden use this principle in their argument for
Uniqueness.

\begin{quote}
If agent \emph{S}\textsubscript{1} judges that
\emph{S}\textsubscript{2}'s belief that \emph{P} is rational and that
\emph{S}\textsubscript{1} does not have relevant evidence that
\emph{S}\textsubscript{2} lacks, then \emph{S}\textsubscript{1} defers
to \emph{S}\textsubscript{2}'s belief that \emph{P}.
(\citeproc{ref-GrecoHedden2016}{Greco and Hedden 2016, 373}).
\end{quote}

Similar kinds of arguments are made by Dogramaci
(\citeproc{ref-Dogramaci2012}{2012}) and Horowitz
(\citeproc{ref-Horowitz2014}{2014}). But the principle looks rather
dubious in the case of these games. Imagine that \emph{A} forms a belief
(we'll come back to how) that \emph{B} believes that a rational thing to
do in the Xefteris game is to play 0.6, and so believes that \emph{B}
will play 0.6. This last step requires Uniqueness, or, more
specifically, \emph{A}'s belief that \emph{B} believes in Uniqueness.
The reasoning is as follows. \emph{A} thinks that \emph{B} thinks that
0.6 is a rational move; so, by Uniqueness, \emph{A} believes that
\emph{B} believes that 0.6 is the only rational move; so, by \emph{B}'s
belief in their own rationality, \emph{B} believes that they will play
0.6; so, by \emph{B}'s self-control as a practically rational agent,
\emph{B} will in fact play 0.6. Now \emph{A} believes that they have the
same evidence as \emph{B}, and that a rational thing to do with that
evidence is play 0.6. By Uniqueness, they will believe that the only
rational thing for someone with the evidence that they have (and that
\emph{B} has) is to play 0.6. But that can't be right. If \emph{B} is
playing 0.6, as \emph{A} has independently judged they will, the
rational thing for \emph{A} to do is to play something other than 0.6.

And that's the general case for these symmetric games with only
asymmetric equilibria. Believing that someone else is at an equilibrium
point is a reason to not copy them. Uniqueness, combined with common
knowledge of shared evidence and rationality, implies that anyone who
believes that another player will adopt strategy \emph{s} has a reason
to adopt strategy \emph{s}. After all, another player is playing it, and
since that player is rational it is a rational thing to do in their
situation, so by common evidence it is a rational thing to do in one's
own situation, so by Uniqueness it is the only rational thing to do in
one's own situation. But since the symmetric situations are not
equilibria, believing that the other person will do \emph{s} cannot be a
reason to do \emph{s}. That means one of the three assumptions we made
here - common knowledge of rationality, common knowledge of shared
evidence, and Uniqueness, must be wrong. Since it is typically taken to
be at least coherent to have common knowledge of rationality and common
knowledge of shared evidence, it follows that Uniqueness is wrong.

But maybe the Uniqueness theorist could resist that last step. Maybe
they could deny that the game, with the assumption of common knowledge
of rationality and shared evidence, really is possible. \footnote{This
  move won't really help with Chicken; but maybe in that case they can
  simply insist that a rational player will rationally think the other
  player is more likely to make one of the two choices with equal
  expected payoffs.} This perhaps isn't as surprising as it might seem.

Note two things about the Xefteris game. First, it is an infinite game
in the sense that each player has infinitely many choices. It turns out
this matters to the proof that there is no symmetric equilibrium to the
game. Second, we are assuming it is common knowledge, and hence true,
that the players are perfectly rational. Third, we are assuming that
perfect rationality entails that people will not choose one option when
there is a better option available. When you put those three things
together, some things that do not look obviously inconsistent turn out
to be impossible. Here's one example of that.

\begin{quote}
\emph{A} and \emph{B} are playing a game. Each picks a real number in
the open interval (0, 1). They each receive a payoff equal to the
average of the two numbers picked.
\end{quote}

For any number that either player picks, there is a better option
available. It is always better to pick (\emph{x}+1)/2 than \emph{x}, for
example. So it is impossible that each player knows the other is
rational, and that rationality means never picking one option when a
better option is available.

So the Uniqueness theorist could say that the same thing is going on in
the Xefteris game. Some infinitely games cannot be played by rational
actors (understood as people who never choose sub-optimal options); this
is one of them. But if this is all the Uniqueness theorist says, it is
not a well motivated response. We can say why it is impossible to
rationally play games like the open interval game; the options get
better without end. But that isn't true in the Xefteris game. The only
thing that makes the game seem impossible is the Uniqueness assumption.
People who reject Uniqueness can easily describe how the Xefteris game
can be played by rational players. Simply saying that it is impossible,
without any motivation or explanation for this other than Uniqueness
itself, feels like an implausible move.

\section{Conclusion}\label{conclusion}

If Uniqueness is true, then the following thing happens in games between
people who know each other to have the same evidence, and to be
rational. When someone forms a belief about what the other person will
do, they can infer that this is a rational way to play the game given
knowledge that everyone else will do the same thing. But sometimes this
is a very unintuitive inference. In Chicken, it implies that we should
have asymmetric attitudes to someone who is facing a choice between two
options with equal expected value. In the election game Xefteris
describes, a game that feels consistent turns out to be impossible.

I think the conclusion to draw from these cases of symmetric
interactions this is that Uniqueness is false, and hence Permissivism is
true. Sometimes in such an interaction one simply has to form a belief
about the other player, knowing they may well form a different belief
about you. Indeed, sometimes only coherent way to form a belief about
the other player is to believe that they will form a different belief
about you. And that means giving up on Uniqueness.

\subsection*{References}\label{references}
\addcontentsline{toc}{subsection}{References}

\phantomsection\label{refs}
\begin{CSLReferences}{1}{0}
\bibitem[\citeproctext]{ref-Bernheim1984}
Bernheim, B. Douglas. 1984. {``Rationalizable Strategic Behavior.''}
\emph{Econometrica} 52 (4): 1007--28. doi:
\href{https://doi.org/10.2307/1911196}{10.2307/1911196}.

\bibitem[\citeproctext]{ref-Callahan2021}
Callahan, Laura Frances. 2021. {``Epistemic Existentialism.''}
\emph{Episteme} 18 (4): 539--54. doi:
\href{https://doi.org/10.1017/epi.2019.25}{10.1017/epi.2019.25}.

\bibitem[\citeproctext]{ref-Cohen2013}
Cohen, Stewart. 2013. {``A Defence of the (Almost) Equal Weight View.''}
In \emph{The Epistemology of Disagreement: New Essays}, edited by David
Christensen and Jennifer Lackey, 98--117. Oxford: Oxford University
Press.

\bibitem[\citeproctext]{ref-Dogramaci2012}
Dogramaci, Sinan. 2012. {``Reverse Engineering Epistemic Evaluations.''}
\emph{Philosophy and Phenomenological Research} 84 (3): 513--30. doi:
\href{https://doi.org/10.1111/j.1933-1592.2011.00566.x}{10.1111/j.1933-1592.2011.00566.x}.

\bibitem[\citeproctext]{ref-Fey2012}
Fey, Mark. 2012. {``Symmetric Games with Only Asymmetric Equilibria.''}
\emph{Games and Economic Behavior} 75 (1): 424--27. doi:
\href{https://doi.org/10.1016/j.geb.2011.09.008}{10.1016/j.geb.2011.09.008}.

\bibitem[\citeproctext]{ref-GrecoHedden2016}
Greco, Daniel, and Brian Hedden. 2016. {``Uniqueness and
Metaepistemology.''} \emph{The Journal of Philosophy} 113 (8): 365--95.
doi: \href{https://doi.org/10.1111/phc3.12318}{10.1111/phc3.12318}.

\bibitem[\citeproctext]{ref-Horowitz2014}
Horowitz, Sophie. 2014. {``Immoderately Rational.''} \emph{Philosohical
Studies} 167 (1): 41--56. doi:
\href{https://doi.org/10.1007/s11098-013-0231-6}{10.1007/s11098-013-0231-6}.

\bibitem[\citeproctext]{ref-Keynes1921}
Keynes, John Maynard. 1921. \emph{Treatise on Probability}. London:
Macmillan.

\bibitem[\citeproctext]{ref-KopecTitelbaum2016}
Kopec, Matthew, and Michael G. Titelbaum. 2016. {``The Uniqueness
Thesis.''} \emph{Philosophy Compass} 11 (4): 189--200. doi:
\href{https://doi.org/10.1111/phc3.12318}{10.1111/phc3.12318}.

\bibitem[\citeproctext]{ref-Lewis1979a}
Lewis, David. 1979. {``A Problem about Permission.''} In \emph{Essays in
Honour of {J}aakko Hintikka on the Occasion of His Fiftieth Birthday on
{J}anuary 12, 1979}, edited by Esa Saarinen, Risto Hilpinen, Illka
Niiniluoto, and Merrill Provence, 163--75. Dordrecht: Reidel. Reprinted
in his \emph{Papers in Ethics and Social Philosophy}, Cambridge:
Cambridge University Press, 2000, 20-33. References to reprint.

\bibitem[\citeproctext]{ref-Lota2023}
Lota, Kenji, and Ulf Hlobil. 2023. {``Resolutions Against Uniqueness.''}
\emph{Erkenntnis} 88 (3): 1013--33. doi:
\href{https://doi.org/10.1007/s10670-021-00391-z}{10.1007/s10670-021-00391-z}.

\bibitem[\citeproctext]{ref-Meacham2019}
Meacham, Christopher. 2019. {``Deference and Uniqueness.''}
\emph{Philosohical Studies} 176 (3): 709--32. doi:
\href{https://doi.org/10.1007/s11098-018-1036-4}{10.1007/s11098-018-1036-4}.

\bibitem[\citeproctext]{ref-Nash1951}
Nash, John. 1951. {``Non-Cooperative Games.''} \emph{Annals of
Mathematics} 54 (2): 286--95.

\bibitem[\citeproctext]{ref-Palmira2023}
Palmira, Michele. 2023. {``Permissivism and the Truth Connection.''}
\emph{Erkenntnis} 88 (2): 641--6556. doi:
\href{https://doi.org/10.1007/s10670-020-00373-7}{10.1007/s10670-020-00373-7}.

\bibitem[\citeproctext]{ref-Risse2000}
Risse, Mathias. 2000. {``What Is Rational about Nash Equilibria?''}
\emph{Synthese} 124 (3): 361--84. doi:
\href{https://doi.org/10.1023/a:1005259701040}{10.1023/a:1005259701040}.

\bibitem[\citeproctext]{ref-Ross2021}
Ross, Ryan. 2021. {``Alleged Counterexamples to Uniqueness.''}
\emph{Logos and Episteme} 12 (2): 203--13.

\bibitem[\citeproctext]{ref-Schultheis2018}
Schultheis, Ginger. 2018. {``Living on the Edge: Against Epistemic
Permissivism.''} \emph{Mind} 127 (507): 863--79. doi:
\href{https://doi.org/10.1093/mind/fzw065}{10.1093/mind/fzw065}.

\bibitem[\citeproctext]{ref-Xefteris2015}
Xefteris, Dimitrios. 2015. {``Symmetric Zero-Sum Games with Only
Asymmetric Equilibria.''} \emph{Games and Economic Behavior} 89 (1):
122--25. doi:
\href{https://doi.org/10.1016/j.geb.2014.12.001}{10.1016/j.geb.2014.12.001}.

\bibitem[\citeproctext]{ref-Ye2023}
Ye, Ru. 2023. {``Permissivism, the Value of Rationality, and a
Convergence-Theoretic Epistemology.''} \emph{Philosophy and
Phenomenological Research} 106 (1): 157--75. doi:
\href{https://doi.org/10.1111/phpr.12845}{10.1111/phpr.12845}.

\end{CSLReferences}



\noindent Unpublished. First posted in 2021.

\end{document}
