% Options for packages loaded elsewhere
\PassOptionsToPackage{unicode}{hyperref}
\PassOptionsToPackage{hyphens}{url}
%
\documentclass[
  11pt,
  letterpaper,
  DIV=11,
  numbers=noendperiod,
  twoside]{scrartcl}

\usepackage{amsmath,amssymb}
\usepackage{setspace}
\usepackage{iftex}
\ifPDFTeX
  \usepackage[T1]{fontenc}
  \usepackage[utf8]{inputenc}
  \usepackage{textcomp} % provide euro and other symbols
\else % if luatex or xetex
  \usepackage{unicode-math}
  \defaultfontfeatures{Scale=MatchLowercase}
  \defaultfontfeatures[\rmfamily]{Ligatures=TeX,Scale=1}
\fi
\usepackage{lmodern}
\ifPDFTeX\else  
    % xetex/luatex font selection
    \setmainfont[ItalicFont=EB Garamond Italic,BoldFont=EB Garamond
Bold]{EB Garamond Math}
    \setsansfont[]{EB Garamond}
  \setmathfont[]{Garamond-Math}
\fi
% Use upquote if available, for straight quotes in verbatim environments
\IfFileExists{upquote.sty}{\usepackage{upquote}}{}
\IfFileExists{microtype.sty}{% use microtype if available
  \usepackage[]{microtype}
  \UseMicrotypeSet[protrusion]{basicmath} % disable protrusion for tt fonts
}{}
\usepackage{xcolor}
\usepackage[left=1.1in, right=1in, top=0.8in, bottom=0.8in,
paperheight=9.5in, paperwidth=7in, includemp=TRUE, marginparwidth=0in,
marginparsep=0in]{geometry}
\setlength{\emergencystretch}{3em} % prevent overfull lines
\setcounter{secnumdepth}{3}
% Make \paragraph and \subparagraph free-standing
\makeatletter
\ifx\paragraph\undefined\else
  \let\oldparagraph\paragraph
  \renewcommand{\paragraph}{
    \@ifstar
      \xxxParagraphStar
      \xxxParagraphNoStar
  }
  \newcommand{\xxxParagraphStar}[1]{\oldparagraph*{#1}\mbox{}}
  \newcommand{\xxxParagraphNoStar}[1]{\oldparagraph{#1}\mbox{}}
\fi
\ifx\subparagraph\undefined\else
  \let\oldsubparagraph\subparagraph
  \renewcommand{\subparagraph}{
    \@ifstar
      \xxxSubParagraphStar
      \xxxSubParagraphNoStar
  }
  \newcommand{\xxxSubParagraphStar}[1]{\oldsubparagraph*{#1}\mbox{}}
  \newcommand{\xxxSubParagraphNoStar}[1]{\oldsubparagraph{#1}\mbox{}}
\fi
\makeatother


\providecommand{\tightlist}{%
  \setlength{\itemsep}{0pt}\setlength{\parskip}{0pt}}\usepackage{longtable,booktabs,array}
\usepackage{calc} % for calculating minipage widths
% Correct order of tables after \paragraph or \subparagraph
\usepackage{etoolbox}
\makeatletter
\patchcmd\longtable{\par}{\if@noskipsec\mbox{}\fi\par}{}{}
\makeatother
% Allow footnotes in longtable head/foot
\IfFileExists{footnotehyper.sty}{\usepackage{footnotehyper}}{\usepackage{footnote}}
\makesavenoteenv{longtable}
\usepackage{graphicx}
\makeatletter
\newsavebox\pandoc@box
\newcommand*\pandocbounded[1]{% scales image to fit in text height/width
  \sbox\pandoc@box{#1}%
  \Gscale@div\@tempa{\textheight}{\dimexpr\ht\pandoc@box+\dp\pandoc@box\relax}%
  \Gscale@div\@tempb{\linewidth}{\wd\pandoc@box}%
  \ifdim\@tempb\p@<\@tempa\p@\let\@tempa\@tempb\fi% select the smaller of both
  \ifdim\@tempa\p@<\p@\scalebox{\@tempa}{\usebox\pandoc@box}%
  \else\usebox{\pandoc@box}%
  \fi%
}
% Set default figure placement to htbp
\def\fps@figure{htbp}
\makeatother
% definitions for citeproc citations
\NewDocumentCommand\citeproctext{}{}
\NewDocumentCommand\citeproc{mm}{%
  \begingroup\def\citeproctext{#2}\cite{#1}\endgroup}
\makeatletter
 % allow citations to break across lines
 \let\@cite@ofmt\@firstofone
 % avoid brackets around text for \cite:
 \def\@biblabel#1{}
 \def\@cite#1#2{{#1\if@tempswa , #2\fi}}
\makeatother
\newlength{\cslhangindent}
\setlength{\cslhangindent}{1.5em}
\newlength{\csllabelwidth}
\setlength{\csllabelwidth}{3em}
\newenvironment{CSLReferences}[2] % #1 hanging-indent, #2 entry-spacing
 {\begin{list}{}{%
  \setlength{\itemindent}{0pt}
  \setlength{\leftmargin}{0pt}
  \setlength{\parsep}{0pt}
  % turn on hanging indent if param 1 is 1
  \ifodd #1
   \setlength{\leftmargin}{\cslhangindent}
   \setlength{\itemindent}{-1\cslhangindent}
  \fi
  % set entry spacing
  \setlength{\itemsep}{#2\baselineskip}}}
 {\end{list}}
\usepackage{calc}
\newcommand{\CSLBlock}[1]{\hfill\break\parbox[t]{\linewidth}{\strut\ignorespaces#1\strut}}
\newcommand{\CSLLeftMargin}[1]{\parbox[t]{\csllabelwidth}{\strut#1\strut}}
\newcommand{\CSLRightInline}[1]{\parbox[t]{\linewidth - \csllabelwidth}{\strut#1\strut}}
\newcommand{\CSLIndent}[1]{\hspace{\cslhangindent}#1}

\setlength\heavyrulewidth{0ex}
\setlength\lightrulewidth{0ex}
\usepackage[automark]{scrlayer-scrpage}
\clearpairofpagestyles
\cehead{
  Brian Weatherson
  }
\cohead{
  Age, Period, and Cohort Effects in Philosophy Journal Citations
  }
\ohead{\bfseries \pagemark}
\cfoot{}
\makeatletter
\newcommand*\NoIndentAfterEnv[1]{%
  \AfterEndEnvironment{#1}{\par\@afterindentfalse\@afterheading}}
\makeatother
\NoIndentAfterEnv{itemize}
\NoIndentAfterEnv{enumerate}
\NoIndentAfterEnv{description}
\NoIndentAfterEnv{quote}
\NoIndentAfterEnv{equation}
\NoIndentAfterEnv{longtable}
\NoIndentAfterEnv{abstract}
\renewenvironment{abstract}
 {\vspace{-1.25cm}
 \quotation\small\noindent\emph{Abstract}:}
 {\endquotation}
\newfontfamily\tfont{EB Garamond}
\addtokomafont{disposition}{\rmfamily}
\addtokomafont{title}{\normalfont\itshape}
\let\footnoterule\relax
\KOMAoption{captions}{tableheading}
\makeatletter
\@ifpackageloaded{caption}{}{\usepackage{caption}}
\AtBeginDocument{%
\ifdefined\contentsname
  \renewcommand*\contentsname{Table of contents}
\else
  \newcommand\contentsname{Table of contents}
\fi
\ifdefined\listfigurename
  \renewcommand*\listfigurename{List of Figures}
\else
  \newcommand\listfigurename{List of Figures}
\fi
\ifdefined\listtablename
  \renewcommand*\listtablename{List of Tables}
\else
  \newcommand\listtablename{List of Tables}
\fi
\ifdefined\figurename
  \renewcommand*\figurename{Figure}
\else
  \newcommand\figurename{Figure}
\fi
\ifdefined\tablename
  \renewcommand*\tablename{Table}
\else
  \newcommand\tablename{Table}
\fi
}
\@ifpackageloaded{float}{}{\usepackage{float}}
\floatstyle{ruled}
\@ifundefined{c@chapter}{\newfloat{codelisting}{h}{lop}}{\newfloat{codelisting}{h}{lop}[chapter]}
\floatname{codelisting}{Listing}
\newcommand*\listoflistings{\listof{codelisting}{List of Listings}}
\makeatother
\makeatletter
\makeatother
\makeatletter
\@ifpackageloaded{caption}{}{\usepackage{caption}}
\@ifpackageloaded{subcaption}{}{\usepackage{subcaption}}
\makeatother
\makeatletter
\@ifpackageloaded{sidenotes}{}{\usepackage{sidenotes}}
\@ifpackageloaded{marginnote}{}{\usepackage{marginnote}}
\makeatother

\usepackage{bookmark}

\IfFileExists{xurl.sty}{\usepackage{xurl}}{} % add URL line breaks if available
\urlstyle{same} % disable monospaced font for URLs
\hypersetup{
  pdftitle={Age, Period, and Cohort Effects in Philosophy Journal Citations},
  pdfauthor={Brian Weatherson},
  hidelinks,
  pdfcreator={LaTeX via pandoc}}


\title{Age, Period, and Cohort Effects in Philosophy Journal Citations}
\author{Brian Weatherson}
\date{2025}

\begin{document}
\maketitle
\begin{abstract}
There are extremely strong age and period effects in citations in
philosophy journals. The age effect is that citations are concentrated
on articles published two to five years prior. The period effect is that
recent years have seen an explosion in the number of articles published,
and the number of citations per articles, so many articles are getting
more citations per year than they ever had previously. But cohort
effects are trickier to detect. In this note I argue that they exist.
There are more citations to articles from eras of more dramatic change
in philosophy, such as around 1970 and around 2010. And there are fewer
citations to articles from periods of consolidation, especially in the
late 1970s through the 1980s.
\end{abstract}


\setstretch{1.1}
\section{Introduction}\label{sec-introduction}

This paper concerns citations of philosophy journal articles in other
philosophy journal articles, and in particular, citations that are
indexed in Web of Science. Via my home institution\footnote{I'll say
  what that institution is when the paper is de-anonymised}, I
downloaded the full citation records for one hundred prominent English
language philosophy journals from the time Web of Science started
indexing them. And I looked at how often each of those articles cited
each other. One simple way to summarise some trends in this data set is
to ask for any pair of years, how often are articles published in the
first year cited in articles published in the second year. For instance,
here are two simple facts about the set of citations.

\begin{itemize}
\tightlist
\item
  In 1993, articles published in 1980 were cited 62 times.
\item
  In 2010, articles published in 2007 were cited 609 times.
\end{itemize}

Those numbers are very different; what could explain the change? There
are three natural kinds of explanation available.

First, it could be an \textbf{age} effect. 2010 is only three years
after 2007, while 1993 is thirteen years after 1980. Other things equal,
articles are most frequently cited two to five years after publication.
Before that they aren't widely enough known to be cited; afterwards they
are old news.

Second, it could be a \textbf{period} effect. There were many changes
between 1993 and 2010. More journals came into existence. More journals
that had already existed were added to the Web of Science index, and so
got included in the dataset. Citation norms have been changing, and the
average number of citations per article, and especially the average
number of citations to journal articles per article, have been growing
rapidly. All of these factors could, in principle, explain the
difference between the two values.

Finally, it could be a \textbf{cohort} effect. Maybe there is something
about philosophy in 1980 which made articles published then less likely
to get cited than articles published in 2007. This kind of effect is
both harder to detect in the data, and harder to understand how it could
be possible.

The point of this paper is to argue that the results we see, like the
two values shown above, are best explained by looking at all three
effects. I'll briefly note the evidence for age and period effects,
because these are enormous and mostly obvious. Then the bulk of the
paper will argue that there is still a cohort effect after accounting
for age and period effects, and suggesting some explanations for why
such an effect exists.

\section{Age, Period, Cohort}\label{sec-apc}

Age, period, and cohort effects are most commonly considered when
investigating human populations. Imagine that you are investigating some
historical records, and see something surprising when you look at
teenagers in the late 1960s. What could explain the surprising result?
It could be an age effect: because they were \emph{teenagers}. It could
be a period effect: because it was the \emph{1960s}. Or it could be a
cohort effect: because they were \emph{boomers}.

There are two big technical problems with teasing these explanations
apart. One is that a single data point doesn't distinguish between the
three possible explanations. Another is that because there is a linear
relationship between the variables, since age is just period minus
cohort, simple statistical tests don't always tease the effects
apart.\footnote{See Keyes et al. (\citeproc{ref-KeyesEtAl2010}{2010})
  for a useful survey of attempts to resolve this problem.}

It is, however, important to distinguish the effects.
Table~\ref{tbl-presidential} shows the Democratic share of the two party
vote in US Presidential elections from 1972-2008.\footnote{The data is
  from Best and Krueger (\citeproc{ref-BestKrueger2012}{2012}), who did
  a bit of work to standardise the results in light of changes to the
  ways exit polls were conducted over this time.}

\begin{longtable}[]{@{}rrrrr@{}}

\caption{\label{tbl-presidential}Democratic share of the two-party vote
in Presidential elections, via Best and Krueger
(\citeproc{ref-BestKrueger2012}{2012}).}

\tabularnewline

\toprule\noalign{}
Year & 18-29 & 30-44 & 45-59 & 60+ \\
\midrule\noalign{}
\endhead
\bottomrule\noalign{}
\endlastfoot
1972 & 0.474 & 0.356 & 0.355 & 0.307 \\
1976 & 0.519 & 0.501 & 0.476 & 0.472 \\
1980 & 0.505 & 0.399 & 0.412 & 0.434 \\
1984 & 0.406 & 0.421 & 0.399 & 0.394 \\
1988 & 0.472 & 0.456 & 0.427 & 0.494 \\
1992 & 0.559 & 0.521 & 0.508 & 0.567 \\
1996 & 0.606 & 0.541 & 0.538 & 0.524 \\
2000 & 0.507 & 0.493 & 0.494 & 0.517 \\
2004 & 0.543 & 0.466 & 0.485 & 0.457 \\
2008 & 0.676 & 0.533 & 0.503 & 0.477 \\

\end{longtable}

The period effects in Table~\ref{tbl-presidential} are rather
pronounced. Democrats did better in their worst group in the landslide
win of 2008 than they did with their best group in the landslide loss of
1972. There also looks to be a pronounced age effect. In many years,
including 1972 and 2008, the Democratic share is strictly decreasing as
one goes from younger to older voters. This looks like evidence for the
conventional wisdom that voters get more conservative as they get
older.\footnote{There could be an age effect without any voter getting
  more conservative. If young Republicans don't vote, or old Democrats
  die earlier than old Republicans, you'd also get an age effect.} But
looking at the middle rows of the table complicates this story. In 1988,
1992, and 2000, Democrats did better among voters over 60 than they did
among any other age group. Why didn't the familiar age effects show up?

One hypothesis is that there is a large cohort effect here. Roughly,
people whose formative political experience was the Great Depression
were (on average) much more disposed to vote for Democrats than people
in other cohorts. This kind of cohort based story could be put forward
either as alternative to the posited age effect, maybe the 2008 results
just show that each generation was a little less conservative than the
one before it, or as a supplement to it. That is, one might hold both
that older people are more conservative than younger people, and that
people who came of age in the Depression are less conservative than
other people. The results we see, where Dukakis (in 1988) and Gore (in
2000) do almost equally as well with young and old voters, might be the
interaction of these effects.\footnote{For a recent careful attempt to
  tease apart these effects, see Ghitza, Gelman, and Auerbach
  (\citeproc{ref-GhitzaEtAl2023}{2023}).}

That's the kind of explanation I'll be offering for citation trends. The
age and period effects are substantial, but they need to be supplemented
with a cohort effect to fully understand the trends.

\section{Methodology}\label{sec-methodology}

As noted above, the study here is based on the Web of Science database,
which my institution makes available with a subscription. That is, it
lets members of the institution download the full database for research
purposes. This is a rather large collection of files; after
de-compression they come to over a terabyte. I selected records that
were marked as \emph{articles} (as opposed to discussion notes, book
reviews, editorial matters, and so on), and whose category was either
Philosophy or History \& Philosophy of Science. I then selected by hand
the hundred journals with the most entries which were (a) primarily
English language, (b) not primarily history of science and (c) broadly
`analytic' rather than `continental'. These were somewhat subjective
choices, but the result was a reasonable collection of the journals
which are most important for telling the story of a certain kind of
philosophy over the last several decades. The list of journals, as well
as the dates covered by the index, is shown in
Table~\ref{tbl-list-of-journals}.

\begin{longtable}[]{@{}
  >{\raggedright\arraybackslash}p{(\linewidth - 6\tabcolsep) * \real{0.5747}}
  >{\raggedleft\arraybackslash}p{(\linewidth - 6\tabcolsep) * \real{0.1034}}
  >{\raggedleft\arraybackslash}p{(\linewidth - 6\tabcolsep) * \real{0.1264}}
  >{\raggedleft\arraybackslash}p{(\linewidth - 6\tabcolsep) * \real{0.1954}}@{}}

\caption{\label{tbl-list-of-journals}The journals included in this
study.}

\tabularnewline

\toprule\noalign{}
\begin{minipage}[b]{\linewidth}\raggedright
Journal
\end{minipage} & \begin{minipage}[b]{\linewidth}\raggedleft
Articles
\end{minipage} & \begin{minipage}[b]{\linewidth}\raggedleft
First Year
\end{minipage} & \begin{minipage}[b]{\linewidth}\raggedleft
Most Recent Year
\end{minipage} \\
\midrule\noalign{}
\endhead
\bottomrule\noalign{}
\endlastfoot
Acta Philosophica & 211 & 2009 & 2022 \\
American Philosophical Quarterly & 1755 & 1964 & 2021 \\
Analysis & 2615 & 1975 & 2022 \\
Analytic Philosophy & 169 & 2016 & 2022 \\
Archiv Fur Geschichte Der Philosophie & 676 & 1975 & 2022 \\
Australasian Journal Of Philosophy & 1683 & 1975 & 2022 \\
Biology \& Philosophy & 1117 & 1988 & 2022 \\
British Journal For The History Of Philosophy & 760 & 2007 & 2022 \\
British Journal For The Philosophy Of Science & 1499 & 1956 & 2022 \\
British Journal Of Aesthetics & 1369 & 1975 & 2022 \\
Bulletin Of Symbolic Logic & 81 & 2003 & 2022 \\
Canadian Journal Of Philosophy & 1497 & 1975 & 2022 \\
Croatian Journal Of Philosophy & 329 & 2007 & 2022 \\
Dialogue & 1464 & 1975 & 2022 \\
Economics And Philosophy & 116 & 2003 & 2022 \\
Episteme & 551 & 2005 & 2022 \\
Ergo & 213 & 2016 & 2021 \\
Erkenntnis & 1744 & 2000 & 2022 \\
Ethical Theory And Moral Practice & 854 & 2008 & 2022 \\
Ethics & 1564 & 1956 & 2022 \\
European Journal For Philosophy Of Science & 449 & 2011 & 2022 \\
European Journal Of Philosophy & 914 & 1998 & 2022 \\
Heythrop Journal & 792 & 1975 & 2014 \\
History And Philosophy Of Logic & 475 & 1992 & 2022 \\
Hume Studies & 111 & 2010 & 2022 \\
Hypatia & 607 & 2009 & 2022 \\
Inquiry & 1743 & 1966 & 2022 \\
International Journal For Philosophy Of Religion & 1098 & 1975 & 2022 \\
International Philosophical Quarterly & 1543 & 1961 & 2022 \\
Journal Of Aesthetics And Art Criticism & 1473 & 1975 & 2022 \\
Journal Of Applied Philosophy & 607 & 2006 & 2022 \\
Journal Of Chinese Philosophy & 1234 & 1973 & 2022 \\
Journal Of Consciousness Studies & 1356 & 2000 & 2022 \\
Journal Of Indian Philosophy & 1034 & 1975 & 2022 \\
Journal Of Medical Ethics & 270 & 1975 & 2022 \\
Journal Of Moral Philosophy & 347 & 2005 & 2022 \\
Journal Of Philosophical Logic & 1411 & 1972 & 2022 \\
Journal Of Philosophical Research & 508 & 2005 & 2022 \\
Journal Of Philosophy & 2706 & 1956 & 2022 \\
Journal Of Political Philosophy & 282 & 2003 & 2022 \\
Journal Of Social Philosophy & 481 & 2008 & 2022 \\
Journal Of Symbolic Logic & 192 & 1968 & 2022 \\
Journal Of The American Philosophical Association & 306 & 2015 & 2022 \\
Journal Of The History Of Ideas & 2101 & 1956 & 2022 \\
Journal Of The History Of Philosophy & 1083 & 1975 & 2022 \\
Journal Of The Philosophy Of History & 239 & 2010 & 2022 \\
Journal Of Value Inquiry & 1358 & 1980 & 2022 \\
Kant-Studien & 1068 & 1975 & 2022 \\
Kantian Review & 293 & 2010 & 2022 \\
Kennedy Institute Of Ethics Journal & 551 & 1995 & 2022 \\
Law And Philosophy & 208 & 2003 & 2022 \\
Logique Et Analyse & 339 & 2007 & 2021 \\
Metaphilosophy & 1470 & 1975 & 2022 \\
Mind & 1906 & 1956 & 2022 \\
Mind \& Language & 155 & 2003 & 2022 \\
Minds And Machines & 192 & 2003 & 2022 \\
Monist & 1911 & 1962 & 2022 \\
Notre Dame Journal Of Formal Logic & 424 & 2009 & 2022 \\
Noûs & 1443 & 1975 & 2022 \\
Pacific Philosophical Quarterly & 1190 & 1980 & 2022 \\
Philosophers' Imprint & 353 & 2010 & 2022 \\
Philosophia & 2050 & 1975 & 2022 \\
Philosophia Mathematica & 221 & 2008 & 2022 \\
Philosophical Explorations & 353 & 2008 & 2022 \\
Philosophical Forum & 802 & 1971 & 2022 \\
Philosophical Investigations & 676 & 1983 & 2022 \\
Philosophical Papers & 225 & 2009 & 2022 \\
Philosophical Perspectives & 275 & 2007 & 2022 \\
Philosophical Psychology & 291 & 2003 & 2022 \\
Philosophical Quarterly & 1341 & 1975 & 2022 \\
Philosophical Review & 988 & 1956 & 2022 \\
Philosophical Studies & 5210 & 1956 & 2022 \\
Philosophical Topics & 106 & 1981 & 1986 \\
Philosophy & 1951 & 1956 & 2022 \\
Philosophy \& Public Affairs & 698 & 1971 & 2022 \\
Philosophy And Phenomenological Research & 3149 & 1956 & 2022 \\
Philosophy And Rhetoric & 886 & 1975 & 2022 \\
Philosophy Compass & 540 & 2015 & 2022 \\
Philosophy East \& West & 1488 & 1966 & 2022 \\
Philosophy Of Science & 3020 & 1956 & 2022 \\
Philosophy Of The Social Sciences & 914 & 1975 & 2022 \\
Phronesis & 743 & 1975 & 2022 \\
Politics Philosophy \& Economics & 195 & 2008 & 2022 \\
Ratio & 1040 & 1974 & 2022 \\
Review Of Metaphysics & 1560 & 1956 & 2022 \\
Review Of Symbolic Logic & 535 & 2008 & 2022 \\
Russell & 379 & 1975 & 2022 \\
Social Philosophy \& Policy & 893 & 1983 & 2021 \\
South African Journal Of Philosophy & 726 & 1987 & 2022 \\
Southern Journal Of Philosophy & 1899 & 1976 & 2022 \\
Southwestern Journal Of Philosophy & 422 & 1970 & 1980 \\
Studia Logica & 666 & 2010 & 2022 \\
Studies In History And Philosophy Of Science & 1660 & 1974 & 2022 \\
Synthese & 6978 & 1966 & 2022 \\
Theoria & 395 & 2007 & 2022 \\
Thought & 188 & 2016 & 2021 \\
Topoi & 1113 & 1982 & 2022 \\
Transactions Of The Charles S Peirce Society & 1132 & 1975 & 2022 \\
Utilitas & 360 & 2009 & 2022 \\

\end{longtable}

The column `First Year' is \emph{not} the first year the journal
published; it is the first year that Web of Science indexed the journal.
This often makes a difference; because \emph{Analysis} isn't indexed
before 1975, we don't get ``Is Knowledge Justified True Belief?''
(\citeproc{ref-Gettier1963}{Gettier 1963}), or much of the initial
literature it generated. Still, we do have a lot of information to work
with, as long as we're careful about the limitations.

The database is supposed to tell you, for each indexed article, which
things it cites. The reliability of this is mixed, especially with
citations that are in footnotes rather than in a bibliography. And the
data needs a huge amount of cleaning. Eugenio Petrovich
(\citeproc{ref-Petrovich2024}{2024}) did a similar study to this one
focussing on five high profile journals, and his first step was a rather
extensive bit of data cleaning.\footnote{See section 4.2.4 of his book
  for more details on the challenges he faced.}

That said, for one important class of citations the data seems fairly
reliable (at least as far as I could check), and not in need of much
cleaning. When the citation is to another article that Web of Science
indexes, the database includes the internal reference number of the
cited article. By simply filtering for references that have an internal
reference of this kind, we can quickly get a fairly accurate record of
when the articles in Table~\ref{tbl-list-of-journals} cite other
articles on the table.

The upside of this approach, as opposed to the more thorough approach
that Petrovich used, is that it makes it practical to study a hundred
journals over sixty years. The downside is that it means we don't see
citations to anything other than journal articles, and articles in these
journals in particular. Obviously a full study of the citations in
philosophy journals would want to pay some attention to citations of
\emph{Philosophical Investigations}, \emph{A Theory of Justice},
\emph{On the Plurality of Worlds}, and many many other books. This is
not that `full study'. Instead it's an attempt to analyse an important
part of the citation data; a part that happens to be much easier to
access.

So for the most part the method used here is that I downloaded hundreds
of XML files from Web of Science and ran some filters on them. This took
a few hours - even modern computers struggle to analyse a terabyte's
worth of information quickly - but it wasn't that sophisticated. There
were only two other things I had to do to fix the data.

The way Web of Science handles the `supplements' to \emph{Noûs}, i.e.,
\emph{Philosophical Perspectives} and \emph{Philosophical Issues}, was a
little uneven. Some years these are recorded as being their own thing,
i.e., with a source name of \_Philosophical Perspectives or
\emph{Philosophical Issues}; and some years they are recorded as special
issues of \emph{Noûs}. When they were listed as special issues, the
citations were extremely unreliable. Some high profile articles are
recorded as having no citations until several years after publication.
The bibliographic information for the articles themselves was also
spotty. So I've manually removed all records that were listed as special
or supplementary issues of \emph{Noûs} (and similarly removed the
citations to those article that did get tracked).

The other big problem is that for several journals, 1974 is missing from
the index. In a couple of cases, 1973 is also missing. And in one very
important case, 1971 and 1972 are missing as well. That `important case'
is \emph{The Journal of Philosophy}. Between 1971 and 1974 it published
groundbreaking articles by Harry Frankfurt
(\citeproc{ref-Frankfurt1971}{1971}), George Boolos
(\citeproc{ref-Boolos1971}{1971}), Paul Benacerraf
(\citeproc{ref-Benacerraf1973}{1973}), Jaegwon Kim
(\citeproc{ref-Kim1973}{1973}), Michael Friedman
(\citeproc{ref-Friedman1974}{1974}), Isaac Levi
(\citeproc{ref-Levi1974}{1974}), and David Lewis
(\citeproc{ref-Lewis1971cen}{1971}, \citeproc{ref-Lewis1973ben}{1973}).
This seemed like a break in the data that needed fixing if I was going
to tell the story correctly. So I used JSTOR to find a full list of
articles (as opposed to notes or book reviews) in \emph{Journal of
Philosophy} in those years, and then looked through the citations in
articles in Table~\ref{tbl-list-of-journals} to see which citations were
to one of those articles. This did mean I was using a different
classification of publications into articles and non-articles, and there
are some odd choices.\footnote{Notably, the JSTOR list seemed to exclude
  the symposium centered around Kenneth Arrow's ``Some
  Ordinalist-Utilitarian Notes on Rawls's Theory of Justice''; I'm not
  sure why that was.} And it meant I had to do a fair bit of data
cleaning just to track down references to those four years.\footnote{A
  non-trivial chunk of the cleaning was sorting through the many and
  varied ways that philosophers have spelled Brian O'Shaughnessy's name
  over the years.} While I've strived to make the data as consistent as
possible with the other years, it's possible that I haven't succeeded,
and some discontinuities around the early 1970s are due to this
discontinuity in how the data was acquired.

After all that, we are left with 102558 articles, from ``Aristotle and
the Sea Battle'' (\citeproc{ref-Anscombe1956}{Anscombe 1956}) to ``Your
Mother Should Know: Pregnancy, the Ethics of Abortion and Knowledge
Through Acquaintance of Moral Value''
(\citeproc{ref-Woolard2022}{Woolard 2022}). These articles collectively
cite each other 389298 times.

\section{Period Effects}\label{sec-period}

Those 389298 citations are not distributed evenly over time. Instead,
they grow rapidly. At the start, in 1956, there are only 4 citations.
That's not too surprising; without the ability to cite preprints, there
aren't going to be many citations of articles that have come out that
year. By 2021, there are 50824. There are actually slightly fewer in
2022, because as you can see in Table~\ref{tbl-list-of-journals}, not
all journals were indexed for 2022. Figure~\ref{fig-citationsperyear}
shows the growth in indexed citations over time.

\begin{figure}

\centering{

\pandocbounded{\includegraphics[keepaspectratio]{apc_files/figure-pdf/fig-citationsperyear-1.pdf}}

}

\caption{\label{fig-citationsperyear}The number of citations in the
dataset made each year.}

\end{figure}%

What explains this dramatic growth? Part of the explanation is that more
articles are being published, and more articles are being indexed.
Figure~\ref{fig-articlesperyear} shows how many articles are in the
dataset each year.

\begin{figure}

\centering{

\pandocbounded{\includegraphics[keepaspectratio]{apc_files/figure-pdf/fig-articlesperyear-1.pdf}}

}

\caption{\label{fig-articlesperyear}The number of articles in the
dataset published each year.}

\end{figure}%

That explains some of the growth, but not all of it. The curve in
Figure~\ref{fig-articlesperyear} is not nearly as steep as the curve in
Figure~\ref{fig-citationsperyear}. The number of (indexed) citations per
article is also rising. In Figure~\ref{fig-outboundcitations} I've
plotted the average number of citations to other articles in the dataset
each year.

\begin{figure}

\centering{

\pandocbounded{\includegraphics[keepaspectratio]{apc_files/figure-pdf/fig-outboundcitations-1.pdf}}

}

\caption{\label{fig-outboundcitations}The average number of citations to
indexed articles each year.}

\end{figure}%

There are a few possible explanations for the shape of this graph.

At the left-hand edge, there are obvious boundary effects. Since we're
only counting citations to articles published since 1956, it isn't
surprising that there aren't very many of them per article in the 1950s.
Since articles rarely get unpublished, there are more articles available
to cite every year.

The next three explanations are a bit more speculative, and I've put
them in increasing order of speculativeness.

First, the most casual perusal of philosophy journals over time will
tell you that the number of citations is increasing. It is now
commonplace to have bibliographies several pages long. This was
considerably less common a few decades ago. There are more citations to
indexed philosophy journals in part because there are more citations.

Second, it feels like the relative importance of \emph{journals}, as
opposed to books or edited volumes, in journal articles is growing. This
study can't verify that, since I filtered out all citations to things
other than journals. But the same kind of casual perusal that tells you
three page bibliographies are more common than they used to be, also
suggests that a greater percentage of those bibliographies consists of
other journals.

There is a third factor I'd like to study, but again this dataset won't
help much with it. Antecedently, I'd have guessed that the rate of
citation of \emph{non}-philosophy journals was increasing. In
particular, philosophers seem to spend a lot more time discussing
results in psychology now than they used to do. If that were true, it
would encourage generate a downwards slope in
Figure~\ref{fig-outboundcitations}, which obviously isn't what we see at
the end. But maybe the rate of citations to all journals is growing even
more rapidly than the rate of citations to philosophy journals. That
will be left as a study for another day.

Although the number of citations is going up, the number of articles
available to be cited is also going up. Say an article is
\emph{available} to be cited in year \emph{y} iff it is published on or
before year \emph{y}. This is perhaps misleading in two directions. On
the one hand, articles published in December are not really available to
be cited that January. On other hand, in recent years the combined
effect of Early View articles and journal backlogs have meant that some
articles get cited as much as four years \emph{before} they are
officially published. Still, it's a reasonable shorthand, and I'll
return later to whether we might want to use a different definition.
Using that definition, the number of available articles each year is
shown in Figure~\ref{fig-availablearticles}.

\begin{figure}

\centering{

\pandocbounded{\includegraphics[keepaspectratio]{apc_files/figure-pdf/fig-availablearticles-1.pdf}}

}

\caption{\label{fig-availablearticles}The number of articles available
to be cited each year.}

\end{figure}%

That graph also has a similar hockey-stick shape. Putting all these
together we can work out how often, on average, available articles are
cited in each year. The results are in Figure~\ref{fig-availablerate}.

\begin{figure}

\centering{

\pandocbounded{\includegraphics[keepaspectratio]{apc_files/figure-pdf/fig-availablerate-1.pdf}}

}

\caption{\label{fig-availablerate}The average number of citations in a
year that each available article receives.}

\end{figure}%

Two things stand out about Figure~\ref{fig-availablerate}. One is that
it is fairly flat for a long time. Between 1978 and 2003 it bounces
around a bit without doing much. It does take off after 2003 though, and
then goes through the roof in 2021. The other thing is that these are
low numbers. For most of this study, an arbitrary article in one of
these hundred journals was cited in one of those journals once a
\emph{decade}. Actually, since citation rates are extremely long-tailed,
and mean rates are well above medians, that somewhat overstates how
often the `average article' was being cited. Frequent citation is very
much not the norm.\footnote{In the long run the average number of times
  an article is cited equals the average number of citations per
  article. So it shouldn't be too surprising that most article have just
  a handful of citations in philosophy journals.}

The various period effects are substantial; to get an reliable picture
of the trends in citation patterns, we're going to have to allow for
them.

\section{Age Effects}\label{sec-age}

The size of the period effects would suggest that we can't work out age
effects by simply taking averages over the whole dataset. Surprisingly,
if we do use the simplest possible method of working out age effects, we
get roughly the right result.

Let's start with that simplest possible method. Say the \emph{age} of a
citation is the time in years between the publication date of the citing
article, and the publication date of the cited article. Then we can
calculate the number of citations with each possible age. The result of
that is shown in Figure~\ref{fig-rawage}.

\begin{figure}

\centering{

\pandocbounded{\includegraphics[keepaspectratio]{apc_files/figure-pdf/fig-rawage-1.pdf}}

}

\caption{\label{fig-rawage}The age distribution of citations in the
dataset.}

\end{figure}%

The picture in Figure~\ref{fig-rawage} is fairly intuitive. Articles
rarely get cited before they are published.\footnote{Though in ``Naive
  Validity, Internalization, and Substructural Approaches To Paradox''
  (\citeproc{ref-Rosenblatt2017}{Rosenblatt 2017}), there are three
  citations to then forthcoming papers in Synthese which eventually
  appeared in 2021, giving them an age of -4.} Then they take a little
bit of time to get noticed, before hitting their peak citations between
2 and 5 years after publication. After that it's a rapid, and then a
slow, decline. For the classic articles, citations never really stop;
Anscombe (\citeproc{ref-Anscombe1956}{1956}) is cited by Mayr
(\citeproc{ref-Mayr2022}{2022}). But most articles reach the end of
their citation life sooner or, occasionally, later.

But the fact that Figure~\ref{fig-rawage} looks plausible shouldn't
obscure the fact that this is a lousy methodology. Given how many of the
citations are in the last few years, what this graph tells us is largely
what citation practices with respect to age have been like in recent
times. It could be that the overall picture is very different, once we
look closely.

As it turns out though, this is roughly the right picture. I'll show
that with some graphs that are a bit more careful about adjusting for
the period effect. I'll start with Figure~\ref{fig-example-age-effect},
which is going to need some explaining.

\begin{figure}

\centering{

\pandocbounded{\includegraphics[keepaspectratio]{apc_files/figure-pdf/fig-example-age-effect-1.pdf}}

}

\caption{\label{fig-example-age-effect}Extract from
Figure~\ref{fig-ageeffecttibble}.}

\end{figure}%

In Figure~\ref{fig-example-age-effect}, the heading at the top, i.e.,
1988, indicates that we're talking about citations of articles published
in 1988. The x-axis indicates the year that the citations are made in.
The y-axis measures what I'm calling the `Citation Ratio'. That's around
1 between 1998 and 2003, but it gets over 2 in 1991. What does that
mean? It's my attempt to measure how often articles published in 1988
are cited at various ages, compared to how commonly articles in general
were cited at the time.

So let's look at the high point on the graph, in 1991. That year the
indexed articles made 2287 citations to articles in the dataset. If we
ignore articles cited before they were published, there were 29363
articles in the dataset they could have cited. So each of those articles
was cited, on average, about 0.0779 times, over the course of 1991.
Among those available articles were the 1286 published in indexed
journals in 1988. Those articles were, collectively, cited 207 times in
1991. So the articles published in 1988 were, on average, cited about
0.161 times each in 1991. What I'm calling the citation ratio is the
ratio of the average number of citations for articles in the target
year, i.e., 1988, divided by the average number of citations for all
available articles. Since 0.161 divided by 0.0779 is about 2.0666, the
citation ratio for 1988 in 1991, is 2.0666. And that's why there is a
dot around (1991, 2) on the graph.

In Figure~\ref{fig-ageeffecttibble} I've done the same graph for sixty
different years, from 1956 to 2015.

\begin{figure*}

\centering{

\pandocbounded{\includegraphics[keepaspectratio]{apc_files/figure-pdf/fig-ageeffecttibble-1.pdf}}

}

\caption{\label{fig-ageeffecttibble}Each facet shows the relative
citation rate for articles published that year at different ages.}

\end{figure*}%

There are several notable things about Figure~\ref{fig-ageeffecttibble}.
The most important is that after some weird results in the early years,
probably due to the small sample sizes, the graphs for each year look
remarkably similar. The citation ratio takes a year or two to take off
from zero, gets to around 2 within three or four years of publication,
then heads back down. There is a fair bit of variation in how it
declines, and that's something we'll return to a lot in what follows,
but the basic picture of a steep rise to a peak around 2 within three to
four years of publication, then a descent, is remarkably stable.

It might not be clear from Figure~\ref{fig-ageeffecttibble} just where
the graphs peak in each facet. In Figure~\ref{fig-peakratio}, I've
graphed the high point for each year.

\begin{figure}

\centering{

\pandocbounded{\includegraphics[keepaspectratio]{apc_files/figure-pdf/fig-peakratio-1.pdf}}

}

\caption{\label{fig-peakratio}The maximum citation ratio in each facet
in Figure~\ref{fig-ageeffecttibble}.}

\end{figure}%

Another way to visualise these effects is to put all of the citation
ratios onto a single graph. That's what I've done in
Figure~\ref{fig-ageeffecteverything}. Each dot represents a citation
ratio. The x-axis is now the age of the citations, not the year of the
citing articles. The colour of the dots represents the year whose
citation ratio is being calculated, but there are so many colours that
it's impossible (at least for me) to tell precisely which colour goes
with which year.\footnote{Roughly, oranges are 1960s, greens are 1970s,
  blues are 1990s, with varieties of teal between them, and pinks and
  purples are the 2000s.} I've highlighted two particularly interesting
years, 1973 and 1985. The circles are for 1985, and the triangles are
for 1973. I've also drawn the average citation ratio for each year as a
line on the graph.

\begin{figure}

\centering{

\pandocbounded{\includegraphics[keepaspectratio]{apc_files/figure-pdf/fig-ageeffecteverything-1.pdf}}

}

\caption{\label{fig-ageeffecteverything}All age effects on a single
graph, with 1973 and 1985 highlighted.}

\end{figure}%

There are two things that I want to particularly highlight about
Figure~\ref{fig-ageeffecteverything}. One is how few real outliers there
are. Between ages 2 and 5, there are almost no dots below the value 1.
There are eleven such values in total, i.e., with an age of 2 to 5, and
a citation ratio below 1. Of these eleven, ten are from the first few
years of the study, i.e., 1956-1963, where the data is much noisier. On
the other hand, after age twenty, there are only twenty dots above 1,
out of 1,128 total dots. Of those twenty, fourteen are for
1973.\footnote{1973 was a very distinctive year, but this does make me
  worry a little that having to collect the citations for \emph{Journal
  of Philosophy} in a different way has led to some inconsistencies in
  the data.} With very few exceptions, the black line in
Figure~\ref{fig-ageeffecteverything} is something like \emph{the} way
that philosophy citations age.

\section{Cohort Effects}\label{sec-cohort}

\begin{figure}

\centering{

\pandocbounded{\includegraphics[keepaspectratio]{apc_files/figure-pdf/fig-cohortsummary-1.pdf}}

}

\caption{\label{fig-cohortsummary}Citation rate for articles published
each year after adjusting for age and period effects.}

\end{figure}%

\begin{figure}

\centering{

\pandocbounded{\includegraphics[keepaspectratio]{apc_files/figure-pdf/fig-ytm1973-1.pdf}}

}

\caption{\label{fig-ytm1973}Citation rate for articles published in 1973
at various ages, adjusted for period and compared to mean for that age.}

\end{figure}%

\begin{figure}

\centering{

\pandocbounded{\includegraphics[keepaspectratio]{apc_files/figure-pdf/fig-ytm1980-1.pdf}}

}

\caption{\label{fig-ytm1980}Citation rate for articles published in 1980
at various ages, adjusted for period and compared to mean for that age.}

\end{figure}%

\begin{figure}

\centering{

\pandocbounded{\includegraphics[keepaspectratio]{apc_files/figure-pdf/fig-ytm1985-1.pdf}}

}

\caption{\label{fig-ytm1985}Citation rate for articles published in 1985
at various ages, adjusted for period and compared to mean for that age.}

\end{figure}%

\begin{figure}

\centering{

\pandocbounded{\includegraphics[keepaspectratio]{apc_files/figure-pdf/fig-ytm2007-1.pdf}}

}

\caption{\label{fig-ytm2007}Citation rate for articles published in 2007
at various ages, adjusted for period and compared to mean for that age.}

\end{figure}%

\phantomsection\label{refs}
\begin{CSLReferences}{1}{0}
\bibitem[\citeproctext]{ref-Anscombe1956}
Anscombe, Elizabeth. 1956. {``Aristotle and the Sea Battle.''}
\emph{Mind} 65 (1): 1--15. doi:
\href{https://doi.org/doi.org/10.1093/mind/65.1.1}{doi.org/10.1093/mind/65.1.1}.

\bibitem[\citeproctext]{ref-Benacerraf1973}
Benacerraf, Paul. 1973. {``Mathematical Truth.''} \emph{Journal of
Philosophy} 70 (19): 661--79.

\bibitem[\citeproctext]{ref-BestKrueger2012}
Best, Samuel J., and Brian S. Krueger. 2012. \emph{Exit Polls: Surveying
the American Electorate, 1972-2010}. Washington, DC: CQ Press. doi:
\href{https://doi.org/10.4135/9781452234410}{10.4135/9781452234410}.

\bibitem[\citeproctext]{ref-Boolos1971}
Boolos, George. 1971. {``The Iterative Conception of Set.''}
\emph{Journal of Philosophy} 68 (8): 215--31.

\bibitem[\citeproctext]{ref-Frankfurt1971}
Frankfurt, Harry. 1971. {``Freedom of the Will and the Concept of a
Person.''} \emph{Journal of Philosophy} 68 (1): 5--20. doi:
\href{https://doi.org/10.2307/2024717}{10.2307/2024717}.

\bibitem[\citeproctext]{ref-Friedman1974}
Friedman, Michael. 1974. {``Explanation and Scientific Understanding.''}
\emph{Journal of Philosophy} 71 (1): 5--19. doi:
\href{https://doi.org/10.2307/2024924}{10.2307/2024924}.

\bibitem[\citeproctext]{ref-Gettier1963}
Gettier, Edmund L. 1963. {``Is Justified True Belief Knowledge?''}
\emph{Analysis} 23 (6): 121--23. doi:
\href{https://doi.org/10.2307/3326922}{10.2307/3326922}.

\bibitem[\citeproctext]{ref-GhitzaEtAl2023}
Ghitza, Yair, Andrew Gelman, and Jonathan Auerbach. 2023. {``The Great
Society, Reagan's Revolution, and Generations of Presidential Voting.''}
\emph{American Journal of Political Science} 67 (3): 520--37. doi:
\url{https://doi.org/10.1111/ajps.12713}.

\bibitem[\citeproctext]{ref-KeyesEtAl2010}
Keyes, Katherine M., Rebecca L. Utz, Whitney Robinson, and Guohua Li.
2010. {``What Is a Cohort Effect? Comparison of Three Statistical
Methods for Modeling Cohort Effects in Obesity Prevalence in the United
States, 1971--2006.''} \emph{Social Science \& Medicine} 70 (7):
1100--1108. doi:
\href{https://doi.org/10.1016/j.socscimed.2009.12.018}{10.1016/j.socscimed.2009.12.018}.

\bibitem[\citeproctext]{ref-Kim1973}
Kim, Jaegwon. 1973. {``Causes and Counterfactuals.''} \emph{Journal of
Philosophy} 70 (17): 570--72. doi:
\href{https://doi.org/10.2307/2025312}{10.2307/2025312}.

\bibitem[\citeproctext]{ref-Levi1974}
Levi, Isaac. 1974. {``On Indeterminate Probabilities.''} \emph{Journal
of Philosophy} 71 (13): 391--418. doi:
\href{https://doi.org/10.2307/2025161}{10.2307/2025161}.

\bibitem[\citeproctext]{ref-Lewis1971cen}
Lewis, David. 1971. {``Counterparts of Persons and Their Bodies.''}
\emph{Journal of Philosophy} 68 (7): 203--11. doi:
\href{https://doi.org/10.2307/2024902}{10.2307/2024902}.

\bibitem[\citeproctext]{ref-Lewis1973ben}
---------. 1973. {``Causation.''} \emph{Journal of Philosophy} 70 (17):
556--67. doi: \href{https://doi.org/10.2307/2025310}{10.2307/2025310}.
Reprinted in his \emph{Philosophical Papers}, Volume 2, Oxford: Oxford
University Press, 1986, 159-172. References to reprint.

\bibitem[\citeproctext]{ref-Mayr2022}
Mayr, Erasmus. 2022. {``Anscombe and Intentional Agency Incompatibilism
(for Human and Animal Agents).''} \emph{Synthese} 200 (254): 1--23. doi:
\href{https://doi.org/10.1007/s11229-022-03523-2}{10.1007/s11229-022-03523-2}.

\bibitem[\citeproctext]{ref-Petrovich2024}
Petrovich, Eugenio. 2024. \emph{A Quantitative Portrait of Analytic
Philosophy: Looking Through the Margins}. Cham: Springer.

\bibitem[\citeproctext]{ref-Rosenblatt2017}
Rosenblatt, Lucas. 2017. {``Naive Validity, Internalization, and
Substructural Approaches to Paradox.''} \emph{Ergo} 4 (4): 93--120. doi:
\href{https://doi.org/10.3998/ergo.12405314.0004.004}{10.3998/ergo.12405314.0004.004}.

\bibitem[\citeproctext]{ref-Woolard2022}
Woolard, Fiona. 2022. {``Your Mother Should Know: Pregnancy, the Ethics
of Abortion and Knowledge Through Acquaintance of Moral Value.''}
\emph{Pacific Philosophical Quarterly} 103 (3): 471--92. doi:
\href{https://doi.org/doi.org/10.1111/papq.12416}{doi.org/10.1111/papq.12416}.

\end{CSLReferences}



\noindent Unpublished. Posted online in 2024.


\end{document}
