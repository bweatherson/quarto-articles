% Options for packages loaded elsewhere
\PassOptionsToPackage{unicode}{hyperref}
\PassOptionsToPackage{hyphens}{url}
%
\documentclass[
  10pt,
  letterpaper,
  DIV=11,
  numbers=noendperiod,
  twoside]{scrartcl}

\usepackage{amsmath,amssymb}
\usepackage{setspace}
\usepackage{iftex}
\ifPDFTeX
  \usepackage[T1]{fontenc}
  \usepackage[utf8]{inputenc}
  \usepackage{textcomp} % provide euro and other symbols
\else % if luatex or xetex
  \usepackage{unicode-math}
  \defaultfontfeatures{Scale=MatchLowercase}
  \defaultfontfeatures[\rmfamily]{Ligatures=TeX,Scale=1}
\fi
\usepackage{lmodern}
\ifPDFTeX\else  
    % xetex/luatex font selection
  \setmainfont[ItalicFont=EB Garamond Italic,BoldFont=EB Garamond
Bold]{EB Garamond Math}
  \setsansfont[]{Europa-Bold}
  \setmathfont[]{Garamond-Math}
\fi
% Use upquote if available, for straight quotes in verbatim environments
\IfFileExists{upquote.sty}{\usepackage{upquote}}{}
\IfFileExists{microtype.sty}{% use microtype if available
  \usepackage[]{microtype}
  \UseMicrotypeSet[protrusion]{basicmath} % disable protrusion for tt fonts
}{}
\usepackage{xcolor}
\usepackage[left=1in, right=1in, top=0.8in, bottom=0.8in,
paperheight=9.5in, paperwidth=6.5in, includemp=TRUE, marginparwidth=0in,
marginparsep=0in]{geometry}
\setlength{\emergencystretch}{3em} % prevent overfull lines
\setcounter{secnumdepth}{3}
% Make \paragraph and \subparagraph free-standing
\ifx\paragraph\undefined\else
  \let\oldparagraph\paragraph
  \renewcommand{\paragraph}[1]{\oldparagraph{#1}\mbox{}}
\fi
\ifx\subparagraph\undefined\else
  \let\oldsubparagraph\subparagraph
  \renewcommand{\subparagraph}[1]{\oldsubparagraph{#1}\mbox{}}
\fi


\providecommand{\tightlist}{%
  \setlength{\itemsep}{0pt}\setlength{\parskip}{0pt}}\usepackage{longtable,booktabs,array}
\usepackage{calc} % for calculating minipage widths
% Correct order of tables after \paragraph or \subparagraph
\usepackage{etoolbox}
\makeatletter
\patchcmd\longtable{\par}{\if@noskipsec\mbox{}\fi\par}{}{}
\makeatother
% Allow footnotes in longtable head/foot
\IfFileExists{footnotehyper.sty}{\usepackage{footnotehyper}}{\usepackage{footnote}}
\makesavenoteenv{longtable}
\usepackage{graphicx}
\makeatletter
\def\maxwidth{\ifdim\Gin@nat@width>\linewidth\linewidth\else\Gin@nat@width\fi}
\def\maxheight{\ifdim\Gin@nat@height>\textheight\textheight\else\Gin@nat@height\fi}
\makeatother
% Scale images if necessary, so that they will not overflow the page
% margins by default, and it is still possible to overwrite the defaults
% using explicit options in \includegraphics[width, height, ...]{}
\setkeys{Gin}{width=\maxwidth,height=\maxheight,keepaspectratio}
% Set default figure placement to htbp
\makeatletter
\def\fps@figure{htbp}
\makeatother
% definitions for citeproc citations
\NewDocumentCommand\citeproctext{}{}
\NewDocumentCommand\citeproc{mm}{%
  \begingroup\def\citeproctext{#2}\cite{#1}\endgroup}
\makeatletter
 % allow citations to break across lines
 \let\@cite@ofmt\@firstofone
 % avoid brackets around text for \cite:
 \def\@biblabel#1{}
 \def\@cite#1#2{{#1\if@tempswa , #2\fi}}
\makeatother
\newlength{\cslhangindent}
\setlength{\cslhangindent}{1.5em}
\newlength{\csllabelwidth}
\setlength{\csllabelwidth}{3em}
\newenvironment{CSLReferences}[2] % #1 hanging-indent, #2 entry-spacing
 {\begin{list}{}{%
  \setlength{\itemindent}{0pt}
  \setlength{\leftmargin}{0pt}
  \setlength{\parsep}{0pt}
  % turn on hanging indent if param 1 is 1
  \ifodd #1
   \setlength{\leftmargin}{\cslhangindent}
   \setlength{\itemindent}{-1\cslhangindent}
  \fi
  % set entry spacing
  \setlength{\itemsep}{#2\baselineskip}}}
 {\end{list}}
\usepackage{calc}
\newcommand{\CSLBlock}[1]{\hfill\break\parbox[t]{\linewidth}{\strut\ignorespaces#1\strut}}
\newcommand{\CSLLeftMargin}[1]{\parbox[t]{\csllabelwidth}{\strut#1\strut}}
\newcommand{\CSLRightInline}[1]{\parbox[t]{\linewidth - \csllabelwidth}{\strut#1\strut}}
\newcommand{\CSLIndent}[1]{\hspace{\cslhangindent}#1}

\setlength\heavyrulewidth{0ex}
\setlength\lightrulewidth{0ex}
\usepackage[automark]{scrlayer-scrpage}
\clearpairofpagestyles
\cehead{
  Brian Weatherson
  }
\cohead{
  Desire as Belief and Moral Newcomb Problems
  }
\ohead{\bfseries \pagemark}
\cfoot{}
\makeatletter
\newcommand*\NoIndentAfterEnv[1]{%
  \AfterEndEnvironment{#1}{\par\@afterindentfalse\@afterheading}}
\makeatother
\NoIndentAfterEnv{itemize}
\NoIndentAfterEnv{enumerate}
\NoIndentAfterEnv{description}
\NoIndentAfterEnv{quote}
\NoIndentAfterEnv{equation}
\NoIndentAfterEnv{longtable}
\NoIndentAfterEnv{abstract}
\renewenvironment{abstract}
 {\vspace{-1.25cm}
 \quotation\small\noindent\rule{\linewidth}{.5pt}\par\smallskip
 \noindent }
 {\par\noindent\rule{\linewidth}{.5pt}\endquotation}
\KOMAoption{captions}{tableheading}
\makeatletter
\@ifpackageloaded{caption}{}{\usepackage{caption}}
\AtBeginDocument{%
\ifdefined\contentsname
  \renewcommand*\contentsname{Table of contents}
\else
  \newcommand\contentsname{Table of contents}
\fi
\ifdefined\listfigurename
  \renewcommand*\listfigurename{List of Figures}
\else
  \newcommand\listfigurename{List of Figures}
\fi
\ifdefined\listtablename
  \renewcommand*\listtablename{List of Tables}
\else
  \newcommand\listtablename{List of Tables}
\fi
\ifdefined\figurename
  \renewcommand*\figurename{Figure}
\else
  \newcommand\figurename{Figure}
\fi
\ifdefined\tablename
  \renewcommand*\tablename{Table}
\else
  \newcommand\tablename{Table}
\fi
}
\@ifpackageloaded{float}{}{\usepackage{float}}
\floatstyle{ruled}
\@ifundefined{c@chapter}{\newfloat{codelisting}{h}{lop}}{\newfloat{codelisting}{h}{lop}[chapter]}
\floatname{codelisting}{Listing}
\newcommand*\listoflistings{\listof{codelisting}{List of Listings}}
\makeatother
\makeatletter
\makeatother
\makeatletter
\@ifpackageloaded{caption}{}{\usepackage{caption}}
\@ifpackageloaded{subcaption}{}{\usepackage{subcaption}}
\makeatother
\ifLuaTeX
  \usepackage{selnolig}  % disable illegal ligatures
\fi
\usepackage{bookmark}

\IfFileExists{xurl.sty}{\usepackage{xurl}}{} % add URL line breaks if available
\urlstyle{same} % disable monospaced font for URLs
\hypersetup{
  pdftitle={Desire as Belief and Moral Newcomb Problems},
  pdfauthor={Brian Weatherson},
  hidelinks,
  pdfcreator={LaTeX via pandoc}}

\title{Desire as Belief and Moral Newcomb Problems}
\author{Brian Weatherson}
\date{2024}

\begin{document}
\maketitle
\begin{abstract}
David Lewis (\citeproc{ref-Lewis1988}{1988},
\citeproc{ref-Lewis1996}{1996}) offered a formal argument against the
view desiring something was equivalent to believing it to be good. There
have been many objections to this argument. I'm going to offer a new
objection that unifies some existing replies to Lewis. Lewis's formalism
is ambiguous, and there is no way to disambiguate it without making one
or other premise implausible. On one disambiguation, the objection that
Price (\citeproc{ref-Price1989}{1989}) makes is unanswerable; on the
other, the objection Collins (\citeproc{ref-Collins2015}{2015}) makes is
unanswerable. My first contribution is to show that Lewis can't avoid
one or other of these objections. My second is to show that once we make
the disambiguation, a new kind of philosophical puzzle arises. I call
the puzzle a Moral Newcomb Problem, and suggest that it may be trickier
to solve than familiar Newcomb Problems.
\end{abstract}

\setstretch{1.1}
\section{The Ludovician Argument}\label{the-ludovician-argument}

The agent in standard versions of decision theory is characterised by
two functions: a credence function and a value function. The credence
function is meant to represent something like beliefs.\footnote{Whether
  it represents partial beliefs about factual propositions, or full
  beliefs in propositions about probability is a tricky question we
  won't address here, though it parallels the questions we will address.
  See Holton (\citeproc{ref-Holton2013}{2014}) for more on this point.}
But what does the value function represent? One idea is that it
represents something like desire. Indeed, on a broadly Ludovician view
about the nature of propositional attitudes, to be a desire just is to
be the (closest thing to being the) realiser of the cleaned up
platitudes about desire (\citeproc{ref-Lewis1994bn}{Lewis 1994}). And
when those platitudes are cleaned up, they say that desires are what
determine the value function in decision theory. So that's one picture:
credences represent beliefs, or more generally belief-like states, and
values represent desires, or more generally desire-like states.

In ordinary talk, we don't make a sharp distinction between valuing
something and believing it is in some way good. When we say that our
decider believes that A is good, it is natural to say they value A, and
to assign a high value to V(A). One way to implement this idea is to
deny that the value function represents desires. But a more common way,
and the one I'll work with here, is to say that the desires just are
beliefs about the good. This is the Desire as Belief thesis that Lewis
Lewis (\citeproc{ref-Lewis1996}{1996}) argues against.

Lewis gives a couple of versions of the argument against Desire as
Belief, and the secondary literature gives many more. The argument in
the first paper is agreed on all sides (including by Lewis in the second
paper) to be needlessly complicated, but the argument in the second
paper is incredibly condensed. But there's nothing like agreement on how
to uncompress it. I'll start with a version of the argument that relies
on relatively weak premises, and which helps make clear where the
argument ultimately fails.\footnote{It's going to fail because the V
  function is ambiguous, and there's no viable disambiguation of it.}

Assume our Hero, called Hero, has a credence function C and a value
function V defined over a (finite) set of worlds.\footnote{The
  restriction to finitely many worlds is potentially substantive, but it
  makes setting out the problem so much easier, and as far as I can
  tell, the complications that are added in the infinite case aren't
  relevant to this problem.} I'll use \emph{w} as a variable for an
arbitrary world. Since we want there to be distinct worlds for any way
Hero thinks things might be, we need worlds to be more finely
individuated than Ludovician concreta Mahtani
(\citeproc{ref-Mahtani2024}{2024}). For current purposes I'll take
worlds to be pairs of concreta and what I'll call
\emph{assessments}.\footnote{This way of setting things up owes a lot to
  Collins (\citeproc{ref-Collins2015}{2015}).} An assessment is a
function from concrete propositions to numerical values. (I'll say what
a concrete proposition is in a second) Intuitively, it means how good
things would be were that proposition actualised. So a world is a pair
⟨\emph{c},~\emph{f}⟩, where \emph{c} is a concretum, and \emph{f} is a
function from non-empty concrete propositions to reals.\footnote{Intuitivly,
  if ⟨\emph{c},~\emph{f}⟩ is a world and A a concrete proposition, then
  \emph{f}(A)~=~\emph{x} means that at \emph{c}, it would be good to
  degree \emph{x} were A true. There are some natural constraints on
  \emph{f} we could derive from this, especially if we endorse centring
  for conditionals, but those constraints aren't relevant to our story,
  so will be set aside.} The value of that world is
\emph{f}(\{\emph{c}\}). A proposition is a set of worlds. A proposition
A is concrete iff for any concretum \emph{c}, and any two functions
\emph{f} and \emph{g}, ⟨\emph{c},~\emph{f}⟩~∈~A iff
⟨\emph{c},~\emph{g}⟩~∈~A. That is, concrete propositions are only about
how the concrete worlds are, in the sense of aboutness from Lewis
(\citeproc{ref-Lewis1982c}{1982}).

We'll now show that four somewhat natural principles are incompatible.
In what follows, P is Hero's credence function, V is Hero's value
function, w is an arbitrary world, and A and B refer to arbitrary
concrete propositions. P\textsubscript{A} and V\textsubscript{A} are the
credence and value functions after updating on A. Finally,
A\textsubscript{\emph{x}} is
\{⟨\emph{c},~\emph{f}⟩:~\emph{f}(A)~=~\emph{x}\}. That is,
A\textsubscript{\emph{x}} is the proposition that the value of A is
\emph{x}.

\begin{description}
\tightlist
\item[Ludovician Desire as Belief]
If A is concrete, then V(A) =
Σ\textsubscript{\emph{x}}~\emph{x}P(A\textsubscript{\emph{x}}).
\item[Worldly Value]
V(\{⟨\emph{c},~\emph{f}⟩\})~=~\emph{f}(\{⟨\emph{c},~\emph{f}⟩\})
\item[Value Additivity]
V(A)~=~Σ\textsubscript{\emph{w}}~V(\{w\})C(\{w\}~\textbar~A)
\item[Restricted Conditionalisation]
P\textsubscript{A}(B)~=~P(B~\textbar~A)
\end{description}

We assume all of these hold after updating.\footnote{Lewis makes
  repeated use throughout both papers that the principles hold after
  updating, and that in the cases at interest, updating is by
  conditionalisation. He endorses \textbf{Value Additivity} on page 326
  of the earlier paper, and \textbf{Worldly Value} on page 332.} The
`restricted' in \textbf{Restricted Conditionalisation} is because we're
only assuming tha conditionalisation is the way to update on concrete
propositions. As Lewis himself notes, it isn't obvious that it's the
right way to update for centered-world propositions, and for similar
reasons, we should be hesitant to assume that it's the right way to
update for propositions that take a stance on value.

\textbf{Worldly Value} says that to find the value of the proposition
true at just that world, you just look at what the value assignment that
partially constitutes the world says about the proposition that's true
just there. If that wasn't true, it wouldn't really make sense to call
it a \emph{value} assignment. And note that \textbf{Worldly Value}
entails a principle I'll call \textbf{Restricted Invariance}. This is
going to be the only point I'll use \textbf{Worldly Value}.

\begin{description}
\tightlist
\item[Restricted Invariance]
V\textsubscript{A}(\{w\})~=~V(\{w\})
\end{description}

The proof is that since the right-hand-side of \textbf{Worldly Value}
doesn't contain anything that varies at all, it just contains things
that are fixed by the nature of w itself. Invariance has been often
cited as the bug in Lewis's argument, but I think that's a mistake. The
argument for the version of \textbf{Restricted Invariance} (which is all
Lewis uses) that he gives on the final page of the 1988 paper looks
pretty good. And it's the argument I've formalised here with
\textbf{Worldly Value} as the starting point.

\begin{longtable}[]{@{}
  >{\raggedleft\arraybackslash}p{(\columnwidth - 4\tabcolsep) * \real{0.3125}}
  >{\raggedright\arraybackslash}p{(\columnwidth - 4\tabcolsep) * \real{0.3125}}
  >{\raggedright\arraybackslash}p{(\columnwidth - 4\tabcolsep) * \real{0.3750}}@{}}
\toprule\noalign{}
\endhead
\bottomrule\noalign{}
\endlastfoot
Σ\textsubscript{\emph{x}}~\emph{x}P(A\textsubscript{\emph{x}}) & = V(A)
& (Ludovician Desire as Belief) \\
& = Σ\textsubscript{\emph{w}}~V(\{w\})C(\{w\}~\textbar~A) & (Value
Additivity) \\
& =
Σ\textsubscript{\emph{w}}~V\textsubscript{A}(\{w\})C(\{w\}~\textbar~A) &
(Restricted Invariance) \\
& =
Σ\textsubscript{\emph{w}}~V\textsubscript{A}(\{w\})C\textsubscript{A}(\{w\})
& (Restricted Conditionalisation) \\
& = V\textsubscript{A}(A) & (Value Additivity), after updating on A \\
& =
Σ\textsubscript{\emph{x}}~\emph{x}P\textsubscript{A}(A\textsubscript{\emph{x}})
& (Ludovician Desire as Belief), after updating on A \\
& = Σ\textsubscript{\emph{x}}~\emph{x}P(A\textsubscript{\emph{x}}
\textbar{} A) & (Restricted Conditionalisation), after updating on A \\
\end{longtable}

If we make the extra simplifying assumption that \emph{f} is always a
binary function that maps bad outcomes to 0 and good outcomes to 1, we
can identify for each concrete proposition A, an evaluative proposition
Å which is \{⟨\emph{c},~\emph{f}⟩: \emph{f}(A)~=~1\}. And then the
result of the proof is that Å and A are probabilistically independent.
It's a bit absurd to think that there are only two possible values, so
it's better to think through the general version which says that the
expected goodness of A is independent of A.

Lewis's own version of the argument continues from here to derive odd
results from this independence claim. But we shouldn't follow him down
that route for several reasons. One is that the extra derivations
require assumptions that seem much more dubious than he's already used.
In particular, he needs an unconditional version of
conditionalisation.\footnote{As Etlin (\citeproc{ref-Etlin2008}{2008,
  84}) notes, when Lewis updates on the non-concrete proposition A~∨~Å,
  as well as assuming without much motivation that this update is by
  conditionalisation, he violates his earlier stipulation that the case
  is one where the motivations for distinguishing causal from evidential
  decision theory don't arise.} For another, as far as I can tell,
everyone of his actual opponents either rejects Ludovician Desire as
Belief as the way to state their view, or rejects independence. And most
importantly of all, the independence view seems on the face of it quite
implausible.

\section{Who'll Be My Role Model?}\label{wholl-be-my-role-model}

Add one more character to Hero's tale. Hero has a moral role model,
called Peter. Hero is perfectly confident that whatever Peter does, it
will be morally good. Even when Hero isn't sure what is right, they are
sure that what Peter does will be right. Right now Peter is faced with a
decision about whether or not to do A. All of Hero's credences are
concentrated on two worlds:
⟨\emph{c}\textsubscript{1},~\emph{f}\textsubscript{1}⟩ and
⟨\emph{c}\textsubscript{2},~\emph{f}\textsubscript{2}⟩, defined as
follows.

\begin{itemize}
\tightlist
\item
  At \emph{c}\textsubscript{1}, Peter does A.
\item
  At \emph{c}\textsubscript{2}, Peter does not do A.
\item
  According to \emph{f}\textsubscript{1}, propositions consistent with
  Peter doing A are good, and others are bad.\footnote{What follows is a
    little condensed. Where I talk about a set of concreta, I should
    really talk about the set of pairs that have one of those concreta
    as their first member. But it's clearer to write it this way, and
    hardly ambiguous.} So
  \emph{f}(\{\emph{c}\textsubscript{1}\})~=~\emph{f}(\{\emph{c}\textsubscript{1},~\emph{c}\textsubscript{2}\})~=~1,
  and \emph{f}(\{\emph{c}\textsubscript{2}\})~=~0.
\item
  According to \emph{f}\textsubscript{2}, propositions consistent with
  Peter not doing A are good, and others are bad. So
  \emph{f}(\{\emph{c}\textsubscript{2}\})~=~\emph{f}(\{\emph{c}\textsubscript{1},~\emph{c}\textsubscript{2}\})~=~1,
  and \emph{f}(\{\emph{c}\textsubscript{1}\})~=~0.
\end{itemize}

Hero gives credence 0.6 to
⟨\emph{c}\textsubscript{1},~\emph{f}\textsubscript{1}⟩ and credence 0.4
to ⟨\emph{c}\textsubscript{2},~\emph{f}\textsubscript{2}⟩. Now a
question. Does Hero more highly value
⟨\emph{c}\textsubscript{1},~\emph{f}\textsubscript{1}⟩ or
⟨\emph{c}\textsubscript{2},~\emph{f}\textsubscript{2}⟩?

\phantomsection\label{refs}
\begin{CSLReferences}{1}{0}
\bibitem[\citeproctext]{ref-Chalmers2011m}
Chalmers, David J. 2011. {``Frege's Puzzle and the Objects of
Credence.''} \emph{Mind} 120 (479): 587--635. doi:
\href{https://doi.org/10.1093/mind/fzr046}{10.1093/mind/fzr046}.

\bibitem[\citeproctext]{ref-Collins2015}
Collins, Jessica. 2015. {``Decision Theory After Lewis.''} In \emph{A
Companion to David Lewis}, edited by Barry Loewer and Jonathan Schaffer,
446--58. John Wiley; Sons.

\bibitem[\citeproctext]{ref-Etlin2008}
Etlin, David. 2008. {``Desire, Belief, and Conditional Belief.''} PhD
thesis, MIT.

\bibitem[\citeproctext]{ref-Holton2013}
Holton, Richard. 2014. {``Intention as a Model for Belief.''} In
\emph{Rational and Social Agency: Essays on the Philosophy of Michael
Bratman}, edited by Manuel Vargas and Gideon Yaffe, 12--37. Oxford:
Oxford University Press.

\bibitem[\citeproctext]{ref-Lewis1982c}
Lewis, David. 1982. {``Logic for Equivocators.''} \emph{No{û}s} 16 (3):
431--41. doi:
\href{https://doi.org/10.1017/cbo9780511625237.009}{10.1017/cbo9780511625237.009}.
Reprinted in his \emph{Papers in Philosophical Logic}, Cambridge:
Cambridge University Press, 1998, 97-110. References to reprint.

\bibitem[\citeproctext]{ref-Lewis1988}
---------. 1988. {``Desire as Belief.''} \emph{Mind} 97 (387): 323--32.
doi:
\href{https://doi.org/10.1093/mind/xcvii.387.323}{10.1093/mind/xcvii.387.323}.

\bibitem[\citeproctext]{ref-Lewis1994bn}
---------. 1994. {``Reduction of Mind.''} In \emph{A Companion to the
Philosophy of Mind}, edited by Samuel Guttenplan, 412--31. Oxford:
Blackwell. doi:
\href{https://doi.org/10.1017/CBO9780511625343.019}{10.1017/CBO9780511625343.019}.

\bibitem[\citeproctext]{ref-Lewis1996}
---------. 1996. {``Desire as Belief {II}.''} \emph{Mind} 105 (418):
303--13. doi:
\href{https://doi.org/10.1093/mind/105.418.303}{10.1093/mind/105.418.303}.

\bibitem[\citeproctext]{ref-Mahtani2024}
Mahtani, Anna. 2024. \emph{The Objects of Credence}. Oxford: Oxford
University Press.

\bibitem[\citeproctext]{ref-Price1989}
Price, Huw. 1989. {``Defending Desire-as-Belief.''} \emph{Mind} 98
(389): 119--27. doi:
\href{https://doi.org/10.1093/mind/XCVIII.389.119}{10.1093/mind/XCVIII.389.119}.

\end{CSLReferences}



\noindent Unpublished. Posted online in 2024.

\end{document}
