% Options for packages loaded elsewhere
\PassOptionsToPackage{unicode}{hyperref}
\PassOptionsToPackage{hyphens}{url}
%
\documentclass[
  10pt,
  letterpaper,
  DIV=11,
  numbers=noendperiod,
  twoside]{scrartcl}

\usepackage{amsmath,amssymb}
\usepackage{setspace}
\usepackage{iftex}
\ifPDFTeX
  \usepackage[T1]{fontenc}
  \usepackage[utf8]{inputenc}
  \usepackage{textcomp} % provide euro and other symbols
\else % if luatex or xetex
  \usepackage{unicode-math}
  \defaultfontfeatures{Scale=MatchLowercase}
  \defaultfontfeatures[\rmfamily]{Ligatures=TeX,Scale=1}
\fi
\usepackage{lmodern}
\ifPDFTeX\else  
    % xetex/luatex font selection
  \setmainfont[ItalicFont=EB Garamond Italic,BoldFont=EB Garamond
Bold]{EB Garamond Math}
  \setsansfont[]{Europa-Bold}
  \setmathfont[]{Garamond-Math}
\fi
% Use upquote if available, for straight quotes in verbatim environments
\IfFileExists{upquote.sty}{\usepackage{upquote}}{}
\IfFileExists{microtype.sty}{% use microtype if available
  \usepackage[]{microtype}
  \UseMicrotypeSet[protrusion]{basicmath} % disable protrusion for tt fonts
}{}
\usepackage{xcolor}
\usepackage[left=1in, right=1in, top=0.8in, bottom=0.8in,
paperheight=9.5in, paperwidth=6.5in, includemp=TRUE, marginparwidth=0in,
marginparsep=0in]{geometry}
\setlength{\emergencystretch}{3em} % prevent overfull lines
\setcounter{secnumdepth}{3}
% Make \paragraph and \subparagraph free-standing
\ifx\paragraph\undefined\else
  \let\oldparagraph\paragraph
  \renewcommand{\paragraph}[1]{\oldparagraph{#1}\mbox{}}
\fi
\ifx\subparagraph\undefined\else
  \let\oldsubparagraph\subparagraph
  \renewcommand{\subparagraph}[1]{\oldsubparagraph{#1}\mbox{}}
\fi


\providecommand{\tightlist}{%
  \setlength{\itemsep}{0pt}\setlength{\parskip}{0pt}}\usepackage{longtable,booktabs,array}
\usepackage{calc} % for calculating minipage widths
% Correct order of tables after \paragraph or \subparagraph
\usepackage{etoolbox}
\makeatletter
\patchcmd\longtable{\par}{\if@noskipsec\mbox{}\fi\par}{}{}
\makeatother
% Allow footnotes in longtable head/foot
\IfFileExists{footnotehyper.sty}{\usepackage{footnotehyper}}{\usepackage{footnote}}
\makesavenoteenv{longtable}
\usepackage{graphicx}
\makeatletter
\def\maxwidth{\ifdim\Gin@nat@width>\linewidth\linewidth\else\Gin@nat@width\fi}
\def\maxheight{\ifdim\Gin@nat@height>\textheight\textheight\else\Gin@nat@height\fi}
\makeatother
% Scale images if necessary, so that they will not overflow the page
% margins by default, and it is still possible to overwrite the defaults
% using explicit options in \includegraphics[width, height, ...]{}
\setkeys{Gin}{width=\maxwidth,height=\maxheight,keepaspectratio}
% Set default figure placement to htbp
\makeatletter
\def\fps@figure{htbp}
\makeatother

\setlength\heavyrulewidth{0ex}
\setlength\lightrulewidth{0ex}
\usepackage[automark]{scrlayer-scrpage}
\clearpairofpagestyles
\cehead{
  Brian Weatherson
  }
\cohead{
  Deference and Infinite Frames
  }
\ohead{\bfseries \pagemark}
\cfoot{}
\makeatletter
\newcommand*\NoIndentAfterEnv[1]{%
  \AfterEndEnvironment{#1}{\par\@afterindentfalse\@afterheading}}
\makeatother
\NoIndentAfterEnv{itemize}
\NoIndentAfterEnv{enumerate}
\NoIndentAfterEnv{description}
\NoIndentAfterEnv{quote}
\NoIndentAfterEnv{equation}
\NoIndentAfterEnv{longtable}
\NoIndentAfterEnv{abstract}
\renewenvironment{abstract}
 {\vspace{-1.25cm}
 \quotation\small\noindent\rule{\linewidth}{.5pt}\par\smallskip
 \noindent }
 {\par\noindent\rule{\linewidth}{.5pt}\endquotation}
\usepackage{amsfonts}
\KOMAoption{captions}{tableheading}
\makeatletter
\@ifpackageloaded{caption}{}{\usepackage{caption}}
\AtBeginDocument{%
\ifdefined\contentsname
  \renewcommand*\contentsname{Table of contents}
\else
  \newcommand\contentsname{Table of contents}
\fi
\ifdefined\listfigurename
  \renewcommand*\listfigurename{List of Figures}
\else
  \newcommand\listfigurename{List of Figures}
\fi
\ifdefined\listtablename
  \renewcommand*\listtablename{List of Tables}
\else
  \newcommand\listtablename{List of Tables}
\fi
\ifdefined\figurename
  \renewcommand*\figurename{Figure}
\else
  \newcommand\figurename{Figure}
\fi
\ifdefined\tablename
  \renewcommand*\tablename{Table}
\else
  \newcommand\tablename{Table}
\fi
}
\@ifpackageloaded{float}{}{\usepackage{float}}
\floatstyle{ruled}
\@ifundefined{c@chapter}{\newfloat{codelisting}{h}{lop}}{\newfloat{codelisting}{h}{lop}[chapter]}
\floatname{codelisting}{Listing}
\newcommand*\listoflistings{\listof{codelisting}{List of Listings}}
\makeatother
\makeatletter
\makeatother
\makeatletter
\@ifpackageloaded{caption}{}{\usepackage{caption}}
\@ifpackageloaded{subcaption}{}{\usepackage{subcaption}}
\makeatother
\ifLuaTeX
  \usepackage{selnolig}  % disable illegal ligatures
\fi
\IfFileExists{bookmark.sty}{\usepackage{bookmark}}{\usepackage{hyperref}}
\IfFileExists{xurl.sty}{\usepackage{xurl}}{} % add URL line breaks if available
\urlstyle{same} % disable monospaced font for URLs
\hypersetup{
  pdftitle={Deference and Infinite Frames},
  pdfauthor={Brian Weatherson},
  hidelinks,
  pdfcreator={LaTeX via pandoc}}

\title{Deference and Infinite Frames}
\author{Brian Weatherson}
\date{2024}

\begin{document}
\maketitle
\begin{abstract}
Dmitri Gallow showed that it is impossible to defer to two experts, and
plan to linearly mix their views if they disagree. Snow Zhang showed
that if there are only finitely many possibilities for the experts'
credences, it is impossible to defer to both and plan to have any strict
mixture of the two views if they disagree. This note shows that if we
drop the assumption that only finitely many values are possible for the
expert credences, this kind of mixing is possible.
\end{abstract}

\setstretch{1.1}
Dmitri Gallow (2018) proved that there is no triple of probability
functions C, A, B satisfying the following constraints.

\begin{enumerate}
\def\labelenumi{\arabic{enumi}.}
\tightlist
\item
  ∀\emph{a}: C(p \textbar{} A(p) = \emph{a}) = \emph{a};
\item
  ∀\emph{b}: C(p \textbar{} B(p) = \emph{b}) = \emph{b};
\item
  C(A = B) \textless{} 1;
\item
  For some λ~∈~(0,1),
  ∀\emph{a},\emph{b}:~C(p~\textbar~A(p)~=~\emph{a}~∧~B(p)~=~\emph{b})~=~λ\emph{a}~+~(1-λ)\emph{b}.
\end{enumerate}

That is, C can't defer to both A and B individually (with respect to p's
probability), think that A and B might disagree, and in the event they
do disagree, plan to take a fixed linear mixture of A's probability and
B's probability.

Snow Zhang recently proved a result that mostly generalises Gallow's
result, though it does weaken it in one crucial respect. (I'm describing
here a simplification of Zhang's result, which also generalises the
number of possible experts.) She shows that it is impossible for A, B
and C to satisfy the following five constraints.

\begin{enumerate}
\def\labelenumi{\arabic{enumi}.}
\tightlist
\item
  ∀\emph{a}: C(p \textbar{} A(p) = \emph{a}) = \emph{a};
\item
  ∀\emph{b}: C(p \textbar{} B(p) = \emph{b}) = \emph{b};
\item
  C(A = B) \textless{} 1;
\item
  For any
  \emph{a},\emph{b}:~C(p~\textbar~A(p)~=~\emph{a}~∧~B(p)~=~\emph{b})~is
  strictly between \emph{a} and \emph{b}.
\item
  For some finite set of values S, C(A(p)~∈~S ∧~B(p)~∈~S)~=~1.
\end{enumerate}

This note shows that the last constraint is essential; it is possible to
satisfy the first four constraints without it. I'll show this by
constructing a model where the first four constraints are satisfied. In
this model there will uncountably many values that A(p) and B(p) could
take. It's an open question whether Zhang's result holds if we weaken 5
to say that S is countable.

I'll describe the model in words first, then describe it algebraically.
Let X, Y and Z be normal distributions with mean 0 and variance 1. In
symbols, each of them is \(\mathcal{N}\)(0,1). So the sum of any two of
them has distribution \(\mathcal{N}\)(0,2), and the sum of all three has
distribution \(\mathcal{N}\)(0,3). Let p be the proposition that this
sum, X~+~Y~+~Z, is positive. Let C be a credence function that
knows\footnote{I'm speaking metaphorically here; really the person whose
  credence function is C knows these things. But the point of this setup
  is to picture where the model below is coming from, so this way of
  speaking is harmless enough.} nothing about X, Y, Z except what's
stated in this paragraph, so C(p)~=~½.

C knows some things about A and B. Both of them know everything C knows
about X, Y, Z, and each are logically and mathematically omniscience.
One of them knows the value of X, and one of them knows the value of
X~+~Y. A fair coin was flipped. If it landed heads, then A knows X and B
knows X~+~Y; if it landed tails, it was the other way around. C knows
about this arrangement, but doesn't know how the coin landed. Let H be
the proposition that it landed heads.

Since both A and B know everything C knows plus something more, C should
defer to them. If C knew which knew X~+~Y and which only knew X, they
would defer to the one who knew X~+~Y. They don't know this, but
conditional on knowing the values of A(p) and B(p), they can go close to
figuring it out.

Assume for now that the coin landed heads, so H is true. We'll work out
the joint density function for A and B. Then we can work out the same
density function conditional on ¬H, and from those two facts work out
the posterior probability of H. Call this value \emph{h}. Conditional on
A(p)~=~\emph{a}, and B(p)~=~\emph{b}, C's probability for p should be
(1-h)\emph{a}~+~\emph{hb}. That's because conditional on
A(p)~=~\emph{a}, B(p)~=~\emph{b} and H, C's probability for p should be
\emph{b}, while conditional on A(p)~=~\emph{a}, B(p)~=~\emph{b} and ¬H,
C's probability for p should be \emph{a}. The short version of what
follows is that since \emph{h} is a function of \emph{a} and \emph{b}
and is always in (0,1), it follows that C obeys constraint 4.

Given H, we can work out the value of X from A(p)~=~\emph{a}. In what
follows, Φ(\emph{x}) is the cumulative distribution for the standard
normal distribution, i.e., for \(\mathcal{N}\)(0,1), and
Φ\textsuperscript{-1} is its inverse. If X~=~\emph{x}, then p is true
iff Y~+~Z~\textgreater{} -\emph{x}. Since Y~+~Z is a normal distribution
with mean 0 and variance 2, i.e., standard deviation \(\sqrt{2}\), the
probability of this is Φ(\(\frac{x}{\sqrt{2}}\)). So
\emph{x}~=~\(\sqrt{2}\)Φ\textsuperscript{-1}(\emph{a}).

Given H, that X~=~\(\sqrt{2}\)Φ\textsuperscript{-1}(\emph{a}), and B(p),
we can work out what Y must be as well. If B(p)~=~\emph{b}, that means
that the probability that Z~\textgreater~-(X~+~Y) is \emph{b}. Since Z
just is a standard normal distribution, that means that X~+~Y is
Φ\textsuperscript{-1}(\emph{b}), and hence Y is
Φ\textsuperscript{-1}(\emph{b}) -
\(\sqrt{2}\)Φ\textsuperscript{-1}(\emph{a}).

Now we can work out the joint density function for \emph{a} and \emph{b}
conditional on H. Given H, A(p)~=~\emph{a} and B(p)~=~\emph{b} just when
X = \(\sqrt{2}\)Φ\textsuperscript{-1}(\emph{a}) and
Y~=~Φ\textsuperscript{-1}(\emph{b}) -
\(\sqrt{2}\)Φ\textsuperscript{-1}(\emph{a}). And if we write d(\emph{x})
for the density function for the standard normal
distribution\footnote{i.e., d(\emph{x}) =
  \(\frac{e^{-\frac{x^2}{2}}}{\sqrt{2\pi}}\).}, the joint distribution
has density d(



\noindent \vspace{1in} In progress

\end{document}
