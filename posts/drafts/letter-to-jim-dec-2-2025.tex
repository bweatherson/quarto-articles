% Options for packages loaded elsewhere
% Options for packages loaded elsewhere
\PassOptionsToPackage{unicode}{hyperref}
\PassOptionsToPackage{hyphens}{url}
%
\documentclass[
  11pt,
  letterpaper,
  DIV=11,
  numbers=noendperiod,
  twoside]{scrartcl}
\usepackage{xcolor}
\usepackage[left=1.1in, right=1in, top=0.8in, bottom=0.8in,
paperheight=11in, paperwidth=8.5in, includemp=TRUE, marginparwidth=0in,
marginparsep=0in]{geometry}
\usepackage{amsmath,amssymb}
\setcounter{secnumdepth}{3}
\usepackage{iftex}
\ifPDFTeX
  \usepackage[T1]{fontenc}
  \usepackage[utf8]{inputenc}
  \usepackage{textcomp} % provide euro and other symbols
\else % if luatex or xetex
  \usepackage{unicode-math} % this also loads fontspec
  \defaultfontfeatures{Scale=MatchLowercase}
  \defaultfontfeatures[\rmfamily]{Ligatures=TeX,Scale=1}
\fi
\usepackage{lmodern}
\ifPDFTeX\else
  % xetex/luatex font selection
  \setmathfont[]{Garamond-Math}
\fi
% Use upquote if available, for straight quotes in verbatim environments
\IfFileExists{upquote.sty}{\usepackage{upquote}}{}
\IfFileExists{microtype.sty}{% use microtype if available
  \usepackage[]{microtype}
  \UseMicrotypeSet[protrusion]{basicmath} % disable protrusion for tt fonts
}{}
\usepackage{setspace}
% Make \paragraph and \subparagraph free-standing
\makeatletter
\ifx\paragraph\undefined\else
  \let\oldparagraph\paragraph
  \renewcommand{\paragraph}{
    \@ifstar
      \xxxParagraphStar
      \xxxParagraphNoStar
  }
  \newcommand{\xxxParagraphStar}[1]{\oldparagraph*{#1}\mbox{}}
  \newcommand{\xxxParagraphNoStar}[1]{\oldparagraph{#1}\mbox{}}
\fi
\ifx\subparagraph\undefined\else
  \let\oldsubparagraph\subparagraph
  \renewcommand{\subparagraph}{
    \@ifstar
      \xxxSubParagraphStar
      \xxxSubParagraphNoStar
  }
  \newcommand{\xxxSubParagraphStar}[1]{\oldsubparagraph*{#1}\mbox{}}
  \newcommand{\xxxSubParagraphNoStar}[1]{\oldsubparagraph{#1}\mbox{}}
\fi
\makeatother


\usepackage{longtable,booktabs,array}
\usepackage{calc} % for calculating minipage widths
% Correct order of tables after \paragraph or \subparagraph
\usepackage{etoolbox}
\makeatletter
\patchcmd\longtable{\par}{\if@noskipsec\mbox{}\fi\par}{}{}
\makeatother
% Allow footnotes in longtable head/foot
\IfFileExists{footnotehyper.sty}{\usepackage{footnotehyper}}{\usepackage{footnote}}
\makesavenoteenv{longtable}
\usepackage{graphicx}
\makeatletter
\newsavebox\pandoc@box
\newcommand*\pandocbounded[1]{% scales image to fit in text height/width
  \sbox\pandoc@box{#1}%
  \Gscale@div\@tempa{\textheight}{\dimexpr\ht\pandoc@box+\dp\pandoc@box\relax}%
  \Gscale@div\@tempb{\linewidth}{\wd\pandoc@box}%
  \ifdim\@tempb\p@<\@tempa\p@\let\@tempa\@tempb\fi% select the smaller of both
  \ifdim\@tempa\p@<\p@\scalebox{\@tempa}{\usebox\pandoc@box}%
  \else\usebox{\pandoc@box}%
  \fi%
}
% Set default figure placement to htbp
\def\fps@figure{htbp}
\makeatother


% definitions for citeproc citations
\NewDocumentCommand\citeproctext{}{}
\NewDocumentCommand\citeproc{mm}{%
  \begingroup\def\citeproctext{#2}\cite{#1}\endgroup}
\makeatletter
 % allow citations to break across lines
 \let\@cite@ofmt\@firstofone
 % avoid brackets around text for \cite:
 \def\@biblabel#1{}
 \def\@cite#1#2{{#1\if@tempswa , #2\fi}}
\makeatother
\newlength{\cslhangindent}
\setlength{\cslhangindent}{1.5em}
\newlength{\csllabelwidth}
\setlength{\csllabelwidth}{3em}
\newenvironment{CSLReferences}[2] % #1 hanging-indent, #2 entry-spacing
 {\begin{list}{}{%
  \setlength{\itemindent}{0pt}
  \setlength{\leftmargin}{0pt}
  \setlength{\parsep}{0pt}
  % turn on hanging indent if param 1 is 1
  \ifodd #1
   \setlength{\leftmargin}{\cslhangindent}
   \setlength{\itemindent}{-1\cslhangindent}
  \fi
  % set entry spacing
  \setlength{\itemsep}{#2\baselineskip}}}
 {\end{list}}
\usepackage{calc}
\newcommand{\CSLBlock}[1]{\hfill\break\parbox[t]{\linewidth}{\strut\ignorespaces#1\strut}}
\newcommand{\CSLLeftMargin}[1]{\parbox[t]{\csllabelwidth}{\strut#1\strut}}
\newcommand{\CSLRightInline}[1]{\parbox[t]{\linewidth - \csllabelwidth}{\strut#1\strut}}
\newcommand{\CSLIndent}[1]{\hspace{\cslhangindent}#1}



\setlength{\emergencystretch}{3em} % prevent overfull lines

\providecommand{\tightlist}{%
  \setlength{\itemsep}{0pt}\setlength{\parskip}{0pt}}



 


\setlength\heavyrulewidth{0ex}
\setlength\lightrulewidth{0ex}
\usepackage[automark]{scrlayer-scrpage}
\clearpairofpagestyles
\cehead{}
\cohead{
  }
\ohead{\bfseries \pagemark}
\cfoot{}
\makeatletter
\newcommand*\NoIndentAfterEnv[1]{%
  \AfterEndEnvironment{#1}{\par\@afterindentfalse\@afterheading}}
\makeatother
\NoIndentAfterEnv{itemize}
\NoIndentAfterEnv{enumerate}
\NoIndentAfterEnv{description}
\NoIndentAfterEnv{quote}
\NoIndentAfterEnv{equation}
\NoIndentAfterEnv{longtable}
\NoIndentAfterEnv{abstract}
\renewenvironment{abstract}
 {\vspace{-1.25cm}
 \quotation\small\noindent\emph{Abstract}:}
 {\endquotation}
\newfontfamily\tfont{EB Garamond}
\addtokomafont{disposition}{\rmfamily}
\addtokomafont{title}{\normalfont\itshape}
\let\footnoterule\relax

\makeatletter
\renewcommand{\@maketitle}{%
  \newpage
  \null
  \vskip 2em%
  \begin{center}%
  \let \footnote \thanks
    {\itshape\huge\@title \par}%
    \vskip 0.5em%  % Reduced from default
    {\large
      \lineskip 0.3em%  % Reduced from default 0.5em
      \begin{tabular}[t]{c}%
        \@author
      \end{tabular}\par}%
    \vskip 0.5em%  % Reduced from default
    {\large \@date}%
  \end{center}%
  \par
  }
\makeatother
\RequirePackage{lettrine}

\renewenvironment{abstract}
 {\quotation\small\noindent\emph{Abstract}:}
 {\endquotation\vspace{-0.02cm}}

\setmainfont{EB Garamond Math}[
  BoldFont = {EB Garamond SemiBold},
  ItalicFont = {EB Garamond Italic},
  RawFeature = {+smcp},
]

\newfontfamily\scfont{EB Garamond Regular}[RawFeature=+smcp]
\renewcommand{\textsc}[1]{{\scfont #1}}

\renewcommand{\LettrineTextFont}{\scfont}
\KOMAoption{captions}{tableheading}
\makeatletter
\@ifpackageloaded{caption}{}{\usepackage{caption}}
\AtBeginDocument{%
\ifdefined\contentsname
  \renewcommand*\contentsname{Table of contents}
\else
  \newcommand\contentsname{Table of contents}
\fi
\ifdefined\listfigurename
  \renewcommand*\listfigurename{List of Figures}
\else
  \newcommand\listfigurename{List of Figures}
\fi
\ifdefined\listtablename
  \renewcommand*\listtablename{List of Tables}
\else
  \newcommand\listtablename{List of Tables}
\fi
\ifdefined\figurename
  \renewcommand*\figurename{Figure}
\else
  \newcommand\figurename{Figure}
\fi
\ifdefined\tablename
  \renewcommand*\tablename{Table}
\else
  \newcommand\tablename{Table}
\fi
}
\@ifpackageloaded{float}{}{\usepackage{float}}
\floatstyle{ruled}
\@ifundefined{c@chapter}{\newfloat{codelisting}{h}{lop}}{\newfloat{codelisting}{h}{lop}[chapter]}
\floatname{codelisting}{Listing}
\newcommand*\listoflistings{\listof{codelisting}{List of Listings}}
\makeatother
\makeatletter
\makeatother
\makeatletter
\@ifpackageloaded{caption}{}{\usepackage{caption}}
\@ifpackageloaded{subcaption}{}{\usepackage{subcaption}}
\makeatother
\usepackage{bookmark}
\IfFileExists{xurl.sty}{\usepackage{xurl}}{} % add URL line breaks if available
\urlstyle{same}
\hypersetup{
  hidelinks,
  pdfcreator={LaTeX via pandoc}}


\author{}
\date{2025}
\begin{document}


\setstretch{1.1}
\noindent Hi Jim,

I'm looking forward to the EWIP on Friday. I had a couple of thoughts
that I thought would be best to get to you before then. That's mostly
because they concern bees in my bonnet that it wouldn't be worth wasting
EWIP time on. But also there are a couple of things that might be easier
to write down than not.

\subsection*{Misers}\label{misers}
\addcontentsline{toc}{subsection}{Misers}

I'm not really convinced that rational misers are coherent. Money has
exchange value not use value. (If not, it's not really money.) And
rational people first exchange it for things they want more and then
things they want less. So to a first approximation, the declining
marginal utility of money follows just from the metaphysics of money and
the rationality of actors. Now none of this matters, because you're just
doing finite games and you could say the payouts are lottery tickets
which (assuming Buchak is wrong about risk sensitivity) have constant
utility. Still, I don't like it as an assumption.

\subsection*{Probability and
Possibility}\label{probability-and-possibility}
\addcontentsline{toc}{subsection}{Probability and Possibility}

There's a really fun example (for at least some value of `fun') where
probability 1 and certainty come apart. Assume you have two options
\emph{A} and \emph{B}, and Sybill has probability 1 of being correct.
The payouts are given in Table~\ref{tbl-red-green}.

\begin{longtable}[]{@{}ccc@{}}
\caption{Probability 1 comes apart from
certainty}\label{tbl-red-green}\tabularnewline
\toprule\noalign{}
& \emph{S\textsubscript{A}} & \emph{S\textsubscript{B}} \\
\midrule\noalign{}
\endfirsthead
\toprule\noalign{}
& \emph{S\textsubscript{A}} & \emph{S\textsubscript{B}} \\
\midrule\noalign{}
\endhead
\bottomrule\noalign{}
\endlastfoot
\emph{A} & 1 & 1 \\
\emph{B} & 1 & 0 \\
\end{longtable}

In equilibrium you'll play \emph{A} with probability 1, and hence Sybill
will predict \emph{A} with probability 1. That means \emph{B} has the
same expected utility as \emph{A}, and can be \textbf{rationally}
picked. So even if you know you'll be rational, you can't be certain you
won't play \emph{B}. But it's probability 1 that you won't.

\subsection*{Buridan}\label{buridan}
\addcontentsline{toc}{subsection}{Buridan}

Is this what Buridan says? I thought he didn't even discuss the donkey;
it was a later example put forward by critics of his view. The view, by
the way, isn't much different to the view you're describing here, so it
would make sense for him to say that. My main knowledge of this comes
from Adamson (\citeproc{ref-AdamsonMedieval}{2019}, ch.~65), so it's all
very second hand.

\subsection*{Sophisticated}\label{sophisticated}
\addcontentsline{toc}{subsection}{Sophisticated}

Using this name for your view makes it seem like it will be very close
to sophisticated choice theories of rational decision. Is this
intentional? I'm not sure that it's a good idea, since there are a few
differences.

\subsection*{Mixed Strategies}\label{mixed-strategies}
\addcontentsline{toc}{subsection}{Mixed Strategies}

My current view is that saying mixed strategies are always available is
the same kind of reasonable idealisation as assuming self-knowledge (as
you explicitly do), or assuming perfect and costless computational
skills (as I think you implicitly do). Sure a randomising device might
not be available, but a way of multiplying large numbers might not be
either, yet we assume away the second complication. I'm not sure why the
first is more implausible.

If that means Frustrator isn't really possible, well that's fine with
me!!

\subsection*{Asymmetric Death in
Damascus}\label{asymmetric-death-in-damascus}
\addcontentsline{toc}{subsection}{Asymmetric Death in Damascus}

Do you have a story for how the following can be possible together?

\begin{enumerate}
\def\labelenumi{\arabic{enumi}.}
\tightlist
\item
  Chooser picks between some options that have equal expected value.
\item
  Chooser has asymmetric credences about which option they'll take.
\end{enumerate}

This should happen, if I've understood the theory correctly, if we
modify Frustrator to add a dollar to each of the \emph{A} outcomes. The
story you tell in sections 4-6 about picking doesn't make it obvious why
asymmetric probabilities should be possible.

Is it something like that you should still pick, but you have some
slightly higher credence that you'll pick B, so the expected values of
the two options are the same? If so, why should you be more confident
you'll pick B? I'm mostly puzzled about how the theory handles this
case.

\subsection*{Downstream Reach}\label{downstream-reach}
\addcontentsline{toc}{subsection}{Downstream Reach}

I didn't get exactly how the constraint in section 8 was supposed to
apply. Here are some variant cases about what might happen at 12.01 that
I think shouldn't make a difference, but I can't quite tell why, on your
view, they don't.

At 12.01 you have a choice between punching yourself in the head or not
doing that. Punching yourself has a utility of -10, and is independent
of what Sybill does. Now you've got a choice at 12.01, so it's not
completely degenerate.

But maybe dumb choices are ruled out. OK, let's say at 12.01 you have
three choices: Raise left hand, Raise right hand, Do nothing. If you do
nothing, the boxes won't be opened, and you'll get nothing. Raising a
hand is how you request opening the boxes, but it doesn't make a
difference which hand you use. Still, it's a decision. So does this mean
news value applies.

Maybe we want to collapse equivalent choices. OK, change the case a
little more. If you raise your left hand, after the box is opened a coin
will be flipped. If it lands heads, you'll get double what's in the box;
if it lands tails, you'll get nothing. If you raise your right hand,
you'll get the box. As a rational miser, you're indifferent between the
hands, but that's fairly contingent. Is this a further decision?

Or maybe some of the stuff at the end about types is meant to help with
these problems. I wasn't sure I was completely following that, but maybe
the view is that given a type, you won't have any real decisions left?

\phantomsection\label{refs}
\begin{CSLReferences}{1}{0}
\bibitem[\citeproctext]{ref-AdamsonMedieval}
Adamson, Peter. 2019. \emph{Medieval Philosophy}. Vol. 4. A History of
Philosophy Without Any Gaps. Oxford: Oxford University Press.

\end{CSLReferences}



\vspace{1pt}


\end{document}
