% Options for packages loaded elsewhere
\PassOptionsToPackage{unicode}{hyperref}
\PassOptionsToPackage{hyphens}{url}
%
\documentclass[
  10pt,
  letterpaper,
  DIV=11,
  numbers=noendperiod,
  twoside]{scrartcl}

\usepackage{amsmath,amssymb}
\usepackage{setspace}
\usepackage{iftex}
\ifPDFTeX
  \usepackage[T1]{fontenc}
  \usepackage[utf8]{inputenc}
  \usepackage{textcomp} % provide euro and other symbols
\else % if luatex or xetex
  \usepackage{unicode-math}
  \defaultfontfeatures{Scale=MatchLowercase}
  \defaultfontfeatures[\rmfamily]{Ligatures=TeX,Scale=1}
\fi
\usepackage{lmodern}
\ifPDFTeX\else  
    % xetex/luatex font selection
    \setmainfont[ItalicFont=EB Garamond Italic,BoldFont=EB Garamond
Bold]{EB Garamond Math}
    \setsansfont[]{Europa-Bold}
  \setmathfont[]{Garamond-Math}
\fi
% Use upquote if available, for straight quotes in verbatim environments
\IfFileExists{upquote.sty}{\usepackage{upquote}}{}
\IfFileExists{microtype.sty}{% use microtype if available
  \usepackage[]{microtype}
  \UseMicrotypeSet[protrusion]{basicmath} % disable protrusion for tt fonts
}{}
\usepackage{xcolor}
\usepackage[left=1in, right=1in, top=0.8in, bottom=0.8in,
paperheight=9.5in, paperwidth=6.5in, includemp=TRUE, marginparwidth=0in,
marginparsep=0in]{geometry}
\setlength{\emergencystretch}{3em} % prevent overfull lines
\setcounter{secnumdepth}{3}
% Make \paragraph and \subparagraph free-standing
\makeatletter
\ifx\paragraph\undefined\else
  \let\oldparagraph\paragraph
  \renewcommand{\paragraph}{
    \@ifstar
      \xxxParagraphStar
      \xxxParagraphNoStar
  }
  \newcommand{\xxxParagraphStar}[1]{\oldparagraph*{#1}\mbox{}}
  \newcommand{\xxxParagraphNoStar}[1]{\oldparagraph{#1}\mbox{}}
\fi
\ifx\subparagraph\undefined\else
  \let\oldsubparagraph\subparagraph
  \renewcommand{\subparagraph}{
    \@ifstar
      \xxxSubParagraphStar
      \xxxSubParagraphNoStar
  }
  \newcommand{\xxxSubParagraphStar}[1]{\oldsubparagraph*{#1}\mbox{}}
  \newcommand{\xxxSubParagraphNoStar}[1]{\oldsubparagraph{#1}\mbox{}}
\fi
\makeatother


\providecommand{\tightlist}{%
  \setlength{\itemsep}{0pt}\setlength{\parskip}{0pt}}\usepackage{longtable,booktabs,array}
\usepackage{calc} % for calculating minipage widths
% Correct order of tables after \paragraph or \subparagraph
\usepackage{etoolbox}
\makeatletter
\patchcmd\longtable{\par}{\if@noskipsec\mbox{}\fi\par}{}{}
\makeatother
% Allow footnotes in longtable head/foot
\IfFileExists{footnotehyper.sty}{\usepackage{footnotehyper}}{\usepackage{footnote}}
\makesavenoteenv{longtable}
\usepackage{graphicx}
\makeatletter
\def\maxwidth{\ifdim\Gin@nat@width>\linewidth\linewidth\else\Gin@nat@width\fi}
\def\maxheight{\ifdim\Gin@nat@height>\textheight\textheight\else\Gin@nat@height\fi}
\makeatother
% Scale images if necessary, so that they will not overflow the page
% margins by default, and it is still possible to overwrite the defaults
% using explicit options in \includegraphics[width, height, ...]{}
\setkeys{Gin}{width=\maxwidth,height=\maxheight,keepaspectratio}
% Set default figure placement to htbp
\makeatletter
\def\fps@figure{htbp}
\makeatother
% definitions for citeproc citations
\NewDocumentCommand\citeproctext{}{}
\NewDocumentCommand\citeproc{mm}{%
  \begingroup\def\citeproctext{#2}\cite{#1}\endgroup}
\makeatletter
 % allow citations to break across lines
 \let\@cite@ofmt\@firstofone
 % avoid brackets around text for \cite:
 \def\@biblabel#1{}
 \def\@cite#1#2{{#1\if@tempswa , #2\fi}}
\makeatother
\newlength{\cslhangindent}
\setlength{\cslhangindent}{1.5em}
\newlength{\csllabelwidth}
\setlength{\csllabelwidth}{3em}
\newenvironment{CSLReferences}[2] % #1 hanging-indent, #2 entry-spacing
 {\begin{list}{}{%
  \setlength{\itemindent}{0pt}
  \setlength{\leftmargin}{0pt}
  \setlength{\parsep}{0pt}
  % turn on hanging indent if param 1 is 1
  \ifodd #1
   \setlength{\leftmargin}{\cslhangindent}
   \setlength{\itemindent}{-1\cslhangindent}
  \fi
  % set entry spacing
  \setlength{\itemsep}{#2\baselineskip}}}
 {\end{list}}
\usepackage{calc}
\newcommand{\CSLBlock}[1]{\hfill\break\parbox[t]{\linewidth}{\strut\ignorespaces#1\strut}}
\newcommand{\CSLLeftMargin}[1]{\parbox[t]{\csllabelwidth}{\strut#1\strut}}
\newcommand{\CSLRightInline}[1]{\parbox[t]{\linewidth - \csllabelwidth}{\strut#1\strut}}
\newcommand{\CSLIndent}[1]{\hspace{\cslhangindent}#1}

\setlength\heavyrulewidth{0ex}
\setlength\lightrulewidth{0ex}
\usepackage[automark]{scrlayer-scrpage}
\clearpairofpagestyles
\cehead{
  Brian Weatherson
  }
\cohead{
  For Bayesians, Rational Modesty Requires Imprecision
  }
\ohead{\bfseries \pagemark}
\cfoot{}
\makeatletter
\newcommand*\NoIndentAfterEnv[1]{%
  \AfterEndEnvironment{#1}{\par\@afterindentfalse\@afterheading}}
\makeatother
\NoIndentAfterEnv{itemize}
\NoIndentAfterEnv{enumerate}
\NoIndentAfterEnv{description}
\NoIndentAfterEnv{quote}
\NoIndentAfterEnv{equation}
\NoIndentAfterEnv{longtable}
\NoIndentAfterEnv{abstract}
\renewenvironment{abstract}
 {\vspace{-1.25cm}
 \quotation\small\noindent\rule{\linewidth}{.5pt}\par\smallskip
 \noindent }
 {\par\noindent\rule{\linewidth}{.5pt}\endquotation}
\KOMAoption{captions}{tableheading}
\makeatletter
\@ifpackageloaded{caption}{}{\usepackage{caption}}
\AtBeginDocument{%
\ifdefined\contentsname
  \renewcommand*\contentsname{Table of contents}
\else
  \newcommand\contentsname{Table of contents}
\fi
\ifdefined\listfigurename
  \renewcommand*\listfigurename{List of Figures}
\else
  \newcommand\listfigurename{List of Figures}
\fi
\ifdefined\listtablename
  \renewcommand*\listtablename{List of Tables}
\else
  \newcommand\listtablename{List of Tables}
\fi
\ifdefined\figurename
  \renewcommand*\figurename{Figure}
\else
  \newcommand\figurename{Figure}
\fi
\ifdefined\tablename
  \renewcommand*\tablename{Table}
\else
  \newcommand\tablename{Table}
\fi
}
\@ifpackageloaded{float}{}{\usepackage{float}}
\floatstyle{ruled}
\@ifundefined{c@chapter}{\newfloat{codelisting}{h}{lop}}{\newfloat{codelisting}{h}{lop}[chapter]}
\floatname{codelisting}{Listing}
\newcommand*\listoflistings{\listof{codelisting}{List of Listings}}
\makeatother
\makeatletter
\makeatother
\makeatletter
\@ifpackageloaded{caption}{}{\usepackage{caption}}
\@ifpackageloaded{subcaption}{}{\usepackage{subcaption}}
\makeatother

\ifLuaTeX
  \usepackage{selnolig}  % disable illegal ligatures
\fi
\usepackage{bookmark}

\IfFileExists{xurl.sty}{\usepackage{xurl}}{} % add URL line breaks if available
\urlstyle{same} % disable monospaced font for URLs
\hypersetup{
  pdftitle={For Bayesians, Rational Modesty Requires Imprecision},
  pdfauthor={Brian Weatherson},
  hidelinks,
  pdfcreator={LaTeX via pandoc}}


\title{For Bayesians, Rational Modesty Requires Imprecision}
\author{Brian Weatherson}
\date{2016}

\begin{document}
\maketitle
\begin{abstract}
Gordon Belot has recently developed a novel argument against
Bayesianism. He shows that there is an interesting class of problems
that, intuitively, no rational belief forming method is likely to get
right. But a Bayesian agent's credence, before the problem starts, that
she will get the problem right has to be 1. This is an implausible kind
of immodesty on the part of Bayesians. My aim is to show that while this
is a good argument against traditional, precise Bayesians, the argument
doesn't neatly extend to imprecise Bayesians. As such, Belot's argument
is a reason to prefer imprecise Bayesianism to precise Bayesianism.
\end{abstract}


\setstretch{1.1}
Gordon Belot (\citeproc{ref-Belot2013}{2013}) has recently developed a
novel argument against Bayesianism. He shows that there is an
interesting class of problems that, intuitively, no rational belief
forming method is likely to get right. But a Bayesian agent's credence,
before the problem starts, that she will get the problem right has to be
1. This is an implausible kind of \emph{immodesty} on the part of
Bayesians.\footnote{There is another sense of immodesty that is often
  discussed in the literature, going back to Lewis
  (\citeproc{ref-Lewis1971d}{1971}). This is the idea that some agents
  think their attitudes are optimal by some standards; these are the
  immodest ones. And often, it is held that not being self-endorsing in
  this way is a coherence failure Elga
  (\citeproc{ref-Elga2010-ELGHTD}{2010}). I don't think this kind of
  immodesty is rationally required, for reasons set out by Miriam
  Schoenfield (\citeproc{ref-Schoenfield2014}{2015}) and Maria
  Lasonen-Aarnio (\citeproc{ref-Lasonen-Aarnio2015}{2015}), but in any
  case that's not the kind of modesty that's at issue in Belot's
  argument.} My aim is to show that while this is a good argument
against traditional, precise Bayesians, the argument doesn't neatly
extend to imprecise Bayesians. As such, Belot's argument is a reason to
prefer imprecise Bayesianism to precise Bayesianism.

For present purposes, the precise Bayesian agent has just two defining
characteristics. First, their credences in all propositions are given by
a particular countably additive probability function. Second, those
credences are updated by conditionalisation as new information comes in.
These commitments are quite strong in some respects. They say that there
is a single probability function that supplies the agent's credences no
matter which question is being investigated, and no matter how little
evidence the agent has before the investigation is started. The everyday
statistician, even one who is sympathetic to Bayesian approaches, may
feel no need to sign up for anything this strong. But many philosophers
seem to be interested in varieties of Bayesianism that are just this
strong. For instance, there has been extensive discussion in recent
epistemology of whether various epistemological approaches, such as
dogmatism, can be modeled within the Bayesian framework, with the
background assumption being that it counts against those approaches if
they cannot.\footnote{For dogmatism, see Pryor
  (\citeproc{ref-Pryor2000}{2000}). The canonical argument that it is
  inconsistent with Bayesianism is White
  (\citeproc{ref-White2006}{2006}).} In these debates, the issue is not
whether the Bayesian approach works in the context of a well-defined
question and a substantial evidential background, but whether it does so
for all questions in all contexts. Indeed, the assumption is that it
does, and epistemological theories inconsistent with it are false. So
the precise Bayesian is a figure of some interest, at least in
epistemology.

The imprecise Bayesian doesn't have a single probability function for
their credences. Rather, they have a representor consisting of a set of
probability functions. The agent is more confident in \emph{p} than
\emph{q} just in case Pr(\emph{p})~\textgreater~Pr(\emph{q}) for every
Pr in this representor.\footnote{Note that this formulation leaves it
  open which side of the biconditional is explanatorily prior. I'm going
  to defend a view on which the left hand side, i.e., the comparative
  confidences, are more explanatorily basic than the facts about what is
  in the agent's representor. I say a little more about why I take this
  stand in footnote 7. For much more detail on varieties of imprecise
  Bayesianism, see Walley (\citeproc{ref-Walley1991}{1991}), from whom I
  take the view that the representor and its members are much less
  explanatorily important than the comparative judgments the agent
  makes.} Just like the precise Bayesian, the imprecise Bayesian updates
by conditionalisation; their new representor after an update is the
result of conditionalising every member of the old representor with the
new information. The added flexibility in imprecise Bayesianism will
allow us to develop a suitably modest response to Belot's puzzle.

\section{The Puzzle}\label{the-puzzle}

The set up Belot uses is this. An agent, \emph{A}, will receive a data
stream of 0s and 1s. The data stream will go on indefinitely. I will use
\textbf{x} for the (infinite) sequence of data she would (eventually)
get, x\textsubscript{k} for the \emph{k}th element of this sequence, and
\textbf{x}\textsubscript{\emph{k}} for the sequence consisting of the
first \emph{k} elements of the stream. These variables are, as usual,
rigid designators. I'll also use the capitalised \textbf{X} and
\textbf{X}\textsubscript{\emph{k}} as random variables for the sequence
itself, and for the first \emph{k} elements of the sequence,
respectively. So \textbf{X}= \textbf{x} is the substantive and true
claim that the sequence that will be received is actually \textbf{x}.
And \textbf{X}\textsubscript{\emph{k}} =
\textbf{x}\textsubscript{\emph{k}} is the substantive and true claim
that the first \emph{k} elements of that sequence are
\textbf{x}\textsubscript{\emph{k}}. Propositions of this form will play
a major role below, since they summarise the evidence the agent has
after \emph{k} elements have been revealed. I'll use \(+\) as a sequence
concatenation operator, so \textbf{y}~+~\textbf{z} is the sequence
consisting of all of \textbf{y}, followed by all of \textbf{z}.

Belot is interested in a quite general puzzle, but I'll focus for most
of the paper on a very specific instance of the puzzle. (We'll return to
the more general puzzle in the last section.) We're going to look at the
agent's evolving credence that \textbf{X} is \emph{periodic}. Let
\emph{p} be the proposition that \textbf{X} is periodic, since we'll be
returning to that proposition a lot. And let's start by assuming the
agent is a precise Bayesian, to see the challenge Belot develops.

Say that the agent \textbf{succeeds} just in case her credence in
\emph{p} eventually gets on the correct side of ½, and stays there. (The
correct side is obviously the high side if \emph{p} is true, and the low
side otherwise.) That is, if \emph{v} is the truth value function, it
succeeds just in case this is true.\footnote{Belot lets an agent succeed
  if \textbf{X} is periodic, and the credence in \emph{p} never drops
  below ½, but I think it's neater to say that the agent is undecided in
  this case.}

\begin{quote}
∃\emph{n} ∀\emph{m}⩾\emph{n}: \textbar{}\emph{v}(\emph{p}) -
\emph{Cr}(\emph{p} \textbar{} \textbf{X}\textsubscript{\emph{m}} =
\textbf{x}\textsubscript{\emph{m}})\textbar{} \textless{} ½
\end{quote}

The agent \textbf{fails} otherwise. Given the assumption that the agent
is a classical Bayesian, we can step back from evaluating the agent and
evaluate her prior probability function directly. So a prior Pr succeeds
relative to \textbf{x} just in case this is true.

\begin{quote}
∃\emph{n} ∀\emph{m}⩾\emph{n}: \textbar{}\emph{v}(\emph{p}) - Pr(\emph{p}
\textbar{} \textbf{X}\textsubscript{\emph{m}} =
\textbf{x}\textsubscript{\emph{m}})\textbar{} \textless{} ½
\end{quote}

This is reasonably intuitive; the agent is going to get a lot of data
about \textbf{X}, and it is interesting to ask whether that data
eventually lets her credence in \emph{p} get to the right side of ½.

Given these notions of success and failure, we can naturally define the
success set of a prior (or agent) as the set of sequences it succeeds
on, and the failure set as the set of sequences it fails on.

Abusing notation a little, say that \textbf{x}\textsubscript{\emph{i}}
⊃~\textbf{x}\textsubscript{\emph{k}} iff
\textbf{x}\textsubscript{\emph{i}} is a sequence that has
\textbf{x}\textsubscript{\emph{k}} as its first \emph{k} entries. Then
we can state the first of Belot's conditions on a good Bayesian
agent/prior. A prior is \textbf{open-minded} just in case this condition
holds:

\begin{quote}
∀\textbf{x}\textsubscript{\emph{k}} ∃\textbf{x}\textsubscript{\emph{i}}
⊃ \textbf{x}\textsubscript{\emph{k}}, \textbf{x}\textsubscript{\emph{j}}
⊃ \textbf{x}\textsubscript{\emph{k}}: Pr(\emph{p} \textbar{}
\textbf{X}\textsubscript{\emph{i}} = \textbf{x}\textsubscript{\emph{i}})
\textless{} ½ ∧ Pr(\emph{p}~\textbar{}
\textbf{X}\textsubscript{\emph{j}} = \textbf{x}\textsubscript{\emph{j}})
\textgreater{} ½
\end{quote}

That is, no matter what happens, it is possible that the probability of
\emph{p} will fall below , and possible it will rise above . To motivate
the first, consider any situation where the sequence to date has looked
periodic. (If it had not looked periodic to date, presumably the
probability of \emph{p} should already be low.) Now extend that sequence
with a large of random noise. At the end of this, it should no longer be
probable that the sequence is periodic. On the other hand, assume the
sequence has not looked periodic to date. Extend it by repeating
\textbf{x}\textsubscript{\emph{k}} more than \emph{k} times. At the end
of this, it should look probable that the sequence is periodic (at least
for large enough \emph{k}). So open-mindedness looks like a good
condition to impose.

The second condition we might impose, though not one Belot names, is
\emph{modesty}. Any function might fail. One natural way it might fail
is that it might get, to use a term Belot does use, \emph{flummoxed}. It
could change its mind infinitely often about whether the sequence is
periodic. By definition, open-mindedness entails the possibility of
being flummoxed. Given the definitions of success and failure, Pr will
fail relative to any \textbf{x} that flummoxes it. So success is not a
priori guaranteed. Now for any function we can work out the set of
sequences relative to which it fails. It turns out this will be a rather
large set. Indeed, the set of sequences on which any open-minded
function succeeds is meagre.\footnote{A meagre subset of a space is any
  set built up as a countable union of nowhere dense sets.} Say a
function is \textbf{modest} if the initial probability it gives to
\textbf{X} being in its success set is less than 1. Given how large the
failure set is, modesty also seems like a good requirement.\footnote{Belot
  goes into much more detail about why modesty is a good requirement to
  put on a rational prior, but I'm omitting those details since I have
  very little to add to what Belot says.}

The argument for modesty is not that it is an immediate consequence of
regularity. It does follow from regularity, but in the case we're
considering, regularity is quite implausible. Some sets, even some quite
large sets in some sense, will have to be given probability 0. The
surprising thing is that a residual set (i.e., the complement of a
meagre set) gets probability 0.

It might be thought that modesty here is problematic for the same reason
that epistemic modesty is often problematic: it validates
Moore-paradoxical thoughts. It's bad to say \emph{p, but there is a
probability that not p}. It's even bad, though as Briggs
(\citeproc{ref-Briggs2009}{2009}) points out, not quite bad for the same
reasons, to say \emph{Whether I believe p is true or false tomorrow,
there will be a probability I'm false}. Perhaps modesty is a requirement
that someone say something like that, and hence is an improper
requirement.

But in fact the requirement of modesty is disanalogous to the
`requirement', suggested in the previous paragraph, that agents endorse
Moore-paradoxical principles. There isn't anything wrong with saying
\emph{Whichever side of one half my credence in p is tomorrow, there is
a probability that the truth will be the other side of one half}. That's
not Moore-paradoxical. Indeed, unless one is sure that one's credence in
\emph{p} tomorrow will be 0 or 1, it is something one should endorse.

Or consider a different example. There will be a sequence of 0s and 1s,
but this time there will only be three elements, and the agent will only
be shown the first of them tomorrow. Let \emph{q} be the proposition
that there are more 1s than 0s in the three-element sequence. Say the
agent \textbf{succeeds} iff tomorrow, after seeing just one element, her
credence in \emph{q} is the same side of one-half as the truth. And say
the agent is \textbf{modest} iff, right now, her credence that she
succeeds tomorrow is less than one. There is nothing incoherent about
being modest. If her credal distribution today is completely flat,
giving 1/8 credence to each of the eight possible sequences, she will be
modest, for example.

Now this case is somewhat different to the one Belot started with in a
couple of respects. On the one hand, we're asking about modesty at a
particular point, i.e., tomorrow, rather than over a long sequence. On
the other hand, we're asking about whether the agent's credences will be
on the right side of one-half after having seen one-third of the data,
rather than, as in the original case, after seeing measure zero of the
sequence. The first difference makes it easier to be modest, the second
difference makes it harder. So the cases are not perfect analogies, but
they are similar enough in respect of modesty to make it plausible that
if modesty is coherent in this case, as we've shown it is, then it
should be coherent in Belot's case as well.

So that's the argument that open-mindedness and modesty are good
conditions for priors to satisfy. Here's the worrying result that Belot
proves. There are no open-minded modest priors. If \emph{A} is a
classical Bayesian, she will either have to be closed minded or
immodest. Neither seems rational, so it seems that being a classical
Bayesian is incompatible with being rational. That is, we can't be
precise Bayesians if we accept the following two constraints.

\begin{itemize}
\tightlist
\item
  \emph{Open-Mindedness}: For any initial sequence, there is a
  continuation after which it seems probable that \textbf{X} is
  periodic, and a continuation after which it seems probable that
  \textbf{X} is not periodic.
\item
  \emph{Modesty}: The initial probability that the agent will succeed,
  i.e., that their credence in \emph{p} will eventually get to the right
  side of ½ and stay there, is less than 1.
\end{itemize}

Since both open-mindedness and modesty are very plausible constraints,
it follows that there is no good way to be a precise Bayesian in the
face of this puzzle.

\section{Making the Puzzle Less
Precise}\label{making-the-puzzle-less-precise}

What happens, though, if the agent is an imprecise Bayesian? Is there a
parallel version of Belot's argument that shows this kind of imprecise
Bayesian is necessarily irrational? I'm going to argue that the answer
is no.

The first thing we have to do is work out how to redefine the key terms
in Belot's argument once we drop the assumption that the agent is a
classical Bayesian. There are several ways of formulating our
definitions which are equivalent given that assumption, but not
equivalent given that the agent is an imprecise Bayesian. There are
three major choice points here.

\begin{enumerate}
\def\labelenumi{\arabic{enumi}.}
\tightlist
\item
  What is success?
\item
  What is open-mindedness?
\item
  What is modesty?
\end{enumerate}

Assume our agent's credal state is represented by set \emph{S} of
probability functions. Then there are two natural ways to think about
success.

\begin{enumerate}
\def\labelenumi{\arabic{enumi}.}
\tightlist
\item
  ∀Pr ∈ S: ∃\emph{n} ∀\emph{m} ⩾ \emph{n}: \textbar{}\emph{v}(\emph{p})
  - Pr(\emph{p}~\textbar{} \textbf{X}\textsubscript{\emph{m}}=
  \textbf{x}\textsubscript{\emph{m}})\textbar{} \textless{} ½
\item
  ∃\emph{n} ∀Pr ∈ S: ∀\emph{m} ⩾~\emph{n}: \textbar{}\emph{v}(\emph{p})
  - Pr(\emph{p}~\textbar{} \textbf{X}\textsubscript{\emph{m}}=
  \textbf{x}\textsubscript{\emph{m}})\textbar{} \textless{} ½
\end{enumerate}

The second is obviously stronger than the first, since it involves
moving an existential quantifier out in front of a universal quantifier.
And there are some natural cases where an agent could succeed on the
first definition, and fail on the second. Here's one such case.

Let Pr\textsubscript{0} be the \emph{fair-coin measure}. Acccording to
the fair coin measure, if \textbf{y} is any \emph{k} length sequence of
0s and 1s we have Pr\textsubscript{0}(\textbf{x}\textsubscript{\emph{k}}
= \textbf{y}) = 2\^{}-\emph{k}\textasciitilde. Intuitively, it thinks
the 0s and 1s are generated by flips of a fair coin, and it won't change
its mind about that no matter what happens.

Say a probability function Pr is \textbf{regular periodic} iff it
satisfies these two conditions.

\begin{itemize}
\tightlist
\item
  Pr(\emph{p}) = 1.
\item
  For any periodic sequence \textbf{y}, Pr(\textbf{X} = \textbf{y})
  \textgreater{} 0.
\end{itemize}

Intuitively, these functions are certain that \emph{X} is periodic, and
assign positive probability to each possible periodic sequence. Now
consider the family of functions we get by taking equal weighted
mixtures of Pr\textsubscript{0} with each regular periodic function. Let
that family represent the agent's credence. And assume for now that
\textbf{X} is the sequence ⟨0, 0, 0, \dots⟩. Does the agent succeed?

Well, each Pr in her representor succeeds. To prove this, it will be
helpful to prove a lemma that we'll again have use for below. For this
lemma, let Pr\textsubscript{0} be the fair-coin measure (as already
noted), Pr\textsubscript{1} be any measure such that
Pr\textsubscript{1}(\emph{p}) = 1, and Pr\textsubscript{2} be the equal
mixture of Pr\textsubscript{0} and Pr\textsubscript{1}.

\textbf{Lemma 1}: Pr\textsubscript{2}(\emph{p}~\textbar{}
\textbf{X}\textsubscript{\emph{k}} = \textbf{y}\textsubscript{k})
\textgreater{} ½ iff
Pr\textsubscript{1}(\textbf{X}\textsubscript{\emph{k}} =
\textbf{y}\textsubscript{k}) \textgreater{}
Pr\textsubscript{0}(\textbf{X}\textsubscript{\emph{k}} =
\textbf{y}\textsubscript{k}).

\begin{quote}
\emph{Proof.} Let
Pr\textsubscript{\emph{i}}(\textbf{X}\textsubscript{\emph{k}} =
\textbf{y}\textsubscript{k}) = \emph{a}\textsubscript{\emph{i}} for
\emph{i} ∈ \{0, 1\}. Recall that Pr\textsubscript{0}(\emph{p}) = 0 and
Pr\textsubscript{1}(\emph{p}) = 1. Then we can quickly get that
Pr\textsubscript{2}(\emph{p}~\textbar{}
\textbf{X}\textsubscript{\emph{k}} = \textbf{y}\textsubscript{k}) =
\emph{a}\textsubscript{1}/(\emph{a}\textsubscript{0}~+~\emph{a}\textsubscript{1}),
from which the lemma immediately follows.~◻
\end{quote}

For any Pr in the agent's representor, there is some \emph{k} such that
Pr(\textbf{X}= ⟨0, 0, 0, \dots⟩) \textgreater{}
2\textsuperscript{-\emph{k}}\}. So after at most \emph{k} 0s have
appeared, Pr(\emph{p}) will be above ½, and it isn't coming back. That
means it succeeds. And since Pr was arbitrary, it follows that all Pr
succeed.

But the agent in a good sense doesn't succeed. No matter how much data
she gets, there will be Pr in her representor according to which
Pr(\emph{p}) \textless{} ½. After all, for any \emph{k}, there are
regular periodic Pr such that the probability of
\textbf{x}\textsubscript{\emph{k}} being \emph{k} 0s is below
2\textsuperscript{-\emph{k}}. So if we mix that function with
Pr\textsubscript{0}, we get a function where the most probable
continuations of this initial sequence are the random sequences provided
by the fair coin measure.

In terms of our definitions of success above, the agent satisfies the
first, but not the second. Every function in her representor eventually
has the probability of \emph{p} go above ½. But at any time, there are
functions in her representor according to which the probability of
\emph{p} is arbitrarily low.

Here I think we have to make a distinction between different ways of
understanding the formalism of imprecise probabilities. (What follows is
indebted to Bradley (\citeproc{ref-Bradley2014}{2014}), especially his
section 3.1, but I'm disagreeing somewhat with his conclusions, and
following more closely the conclusions of Joyce
(\citeproc{ref-Joyce2010}{2010}) and Schoenfield
(\citeproc{ref-Schoenfield2012}{2012}).)

One way of thinking about imprecise credences is that each probability
function in the representor is something like an advisor, and the agent
who is imprecise simply hasn't settled on which advisor to trust. Call
this the \textbf{pluralist} interpretation of the formalism. On this
interpretation, it is natural to think that what is true of every
function is true of the agent.

Another way is to think of the agent's mind as constituted by, but
distinct from, the representors. An analogy to keep in mind here is the
way that a parliament is constituted by, but distinct from, its members.
Keeping with this analogy, call this the \textbf{corporate}
interpretation of the formalism. Note that corporate bodies will
typically have their own rules for how the views of the members will be
translated into being views of the whole. Even if every member of the
parliament believes that the national cricket team will win its upcoming
game, it doesn't follow that the parliament believes that; the
parliament only believes what it resolves it has believed.

Now I only want to defend the imprecise Bayesian model on the corporate
interpretation.\footnote{I have an independent metaphysical reason for
  preferring the corporate interpretation. I think that comparative
  confidences, things like being at least as confident in \emph{p} as in
  \emph{q}, are metaphysically prior to numerical credences, or even
  sets of numerical credences. On such a metaphysics, what it is for Pr
  to be in the representor just is for every \emph{p}, \emph{q},
  \emph{r}, \emph{s}, if the agent is at least as confident in \emph{p}
  given \emph{q} as in \emph{r} given \emph{s}, then
  Pr(\emph{p}~\textbar~\emph{q})~⩾~Pr(\emph{r}~\textbar~\emph{s}). And
  it seems, though I won't defend this claim here, that the corporate
  interpetation fits more naturally with the idea that comparative
  confidences are primitive.} The pluralist interpretation, it seems to
me, faces grave difficulties. For one thing, it has a hard time
explaining what's wrong with the existential claim ``There is a precise
number \emph{x} such that \emph{x} is the probability of \emph{p}''.
Every advisor believes that, so on the pluralist model the agent does
too. (Compare the criticisms of ``fanatical supervaluationism'' in Lewis
(\citeproc{ref-Lewis1993c}{1993}).) More relevant to the discussion
here, I am following Belot in thinking we have an argument that each
precise Bayesian is unreasonably proud. On the pluralist interpretation,
the agent is undecided which of these unreasonable advisors she will
follow. But such a state is itself unreasonable; she should have decided
not to follow any of them, since they are all provably unreasonable!

A surprising fact about corporate bodies is that they can be immune to
problems that beset each of their members. It would be illegitimate for
any one parliamentarian to have law-making power; it is (or at least can
be) legitimate for them all to have such power. Indeed, it would be
unreasonable for any of them to think that they individually should have
law-making powers; that would be unreasonably proud. But it is not
unreasonable for them to collectively think that they should
collectively have law-making powers. If they are a well-constituted
parliament, this is a perfectly reasonable thought. Similarly here, the
agent, the corporate body, could avoid being unreasonably proud even
though each of the representors is over-confident in its own powers.

Now going back to success and modesty, it seems to me that the first
definition of success is appropriate on the pluralist interpretation of
the imprecise framework, and the second is appropriate on the corporate
interpretation. The first interpretation says that the agent succeeds
iff every member succeeds. And the second says that the agent succeeds
iff the body of functions, collectively, succeed. Since I'm defending
the use of the imprecise framework on the corporate interpretation, it
is the second definition of success that is appropriate, and that's what
I will use here.

This understanding isn't without costs. Bradley
(\citeproc{ref-Bradley2014}{2014}) argues, in effect, that the best
responses to dilation-based arguments against imprecise probabilities
(as in White (\citeproc{ref-White2010}{2010})), are only available on
the pluralist interpretation. I'm not going to try to solve those
problems here, but I will note that the interpretative choice I'm making
generates some extra philosophical work elsewhere. Against that, the
corporate interpretation has some benefits. It lets us agree with Peter
Walley (\citeproc{ref-Walley1991}{1991}) that there are rational agents
who are represented by sets of merely finitely additive probability
functions, though no merely finitely additive probability function on
its own could represent a rational agent. So the issues between the two
interpretations are extensive. For now, I'll simply note that I'm
interested in defending the imprecise Bayesian from Belot's argument on
the corporate interpretation. And with that I'll return to translating
Belot's puzzle into the imprecise framework, with the second,
corporate-friendly, interpretation of success on board.

There are also two natural ways to generalise Belot's notion of
open-mindedness to the imprecise case. We could require that the agent
satisfies either the first or second of these conditions.

\begin{enumerate}
\def\labelenumi{\arabic{enumi}.}
\tightlist
\item
  ∀\textbf{x}\textsubscript{\emph{k}}
  ∃\textbf{x}\textsubscript{\emph{i}} ⊃
  \textbf{x}\textsubscript{\emph{k}}, \textbf{x}\textsubscript{\emph{j}}
  ⊃\textbf{x}\textsubscript{\emph{k}}: ¬(Pr(\emph{p}~\textbar{}
  \textbf{X}\textsubscript{\emph{i}} =
  \textbf{x}\textsubscript{\emph{i}}) ⩾½) ∧ ¬(Pr(\emph{p}~\textbar{}
  \textbf{X}\textsubscript{\emph{j}} =
  \textbf{x}\textsubscript{\emph{j}}) \textless{} ½)
\item
  ∀\textbf{x}\textsubscript{\emph{k}}
  ∃\textbf{x}\textsubscript{\emph{i}}
  ⊃\textbf{x}\textsubscript{\emph{k}},
  \textbf{x}\textsubscript{\emph{j}}
  ⊃\textbf{x}\textsubscript{\emph{k}}: Pr(\emph{p}~\textbar{}
  \textbf{X}\textsubscript{\emph{i}} =
  \textbf{x}\textsubscript{\emph{i}}) \textless{} ½ ∧
  Pr(\emph{p}~\textbar{} \textbf{X}\textsubscript{\emph{j}} =
  \textbf{x}\textsubscript{\emph{j}}) ⩾ ½
\end{enumerate}

The second is just the same symbols as in Belot's, and it is what I'll
end up arguing is the right constraint to put on the imprecise Bayesian
agent. And it is a considerably more demanding constraint than the
first. But the first is perhaps the more natural understanding of
\emph{open-mindedness}. It says that no matter what the initial evidence
is, the agent is not guaranteed to settle her credence in \emph{p} on
one side of ½. That's a way of being open-minded.

But if the agent satisfies that constraint, she may be open-minded, but
she won't necessarily be responsive to the evidence. Here's how I'm
using the terms `open-minded' and `evidence-responsive'. In both
clauses, the quantification is intended to be over a salient class of
propositions. (The relevant class in the application we're most
interested in is just \{\emph{X is periodic}, *X is not periodic\}.) And
I'll say an agent is `confident' in a proposition iff her credence in it
is above ½.

\begin{description}
\tightlist
\item[Open-Minded]
Any time an agent is confidence in a proposition, there is some evidence
she could get that would make her lose confidence in it.
\item[Evidence-Responsive]
For any proposition, there is some evidence the agent could get that
would make her confident in it.
\end{description}

Once we allow imprecise credences, these two notions can come apart.
Consider the agent we described above, whose representor consists of
equal mixtures of the fair-coin measure and regular periodic functions.
They are open-minded; they can always lose confidence that \emph{X} is
periodic or not. But they aren't evidence-responsive; no matter what the
evidence, their credence that \emph{X} is periodic will never rise above
½. In fact, their credence that \emph{X} is periodic will never rise
above any positive number.

That suggests open-mindedness is too weak a constraint. If the evidence
the agent gets is a string of several hundred 0s, she shouldn't just
lose any initial confidence in ¬\emph{p}, she should become confident in
\emph{p}. And arguably (though I could imagine a dissent here), if the
initial sequence is a seemingly random sequence, the credence in
\emph{p} should drop well below ½. (The imagined dissent here is from
someone who thinks that the noisier the data, the more imprecise
credences should get. That's an interesting view, but perhaps orthogonal
to the issues we're debating here.)

And when we look back at Belot's motivations for open-mindedness, we see
that they are really motivations for being evidence-responsive. One of
the distinctive (and I would say problematic) features of precise
Bayesianism is that it doesn't really have a good way of representing a
state of indecisiveness or open-mindedness. In the terms we've been
using here, there's no difference for the precise Bayesian between being
evidence responsive and open minded. The imprecise Bayesian can
distinguish these. And in Belot's puzzle, we should require that the
imprecise Bayesian agent is evidence responsive. So we should impose the
second, stronger, condition.

The final condition to discuss is modesty. There are three natural
candidates here. We could merely require that the agent's prior
probability that \textbf{x} is in her success set is not equal to 1. Or
we could require that it be less than 1. Or, even more strongly, we
could require that it be less than some number that is less than 1. If
her credence that \textbf{x} is in her success set is imprecise over
some interval {[}\emph{k}, 1{]}, she satisfies the first condition, but
not the second or third. If it is imprecise over some interval
(\emph{k}, 1), or {[}\emph{k}, 1), she satisfies the first and second
conditions, but not the third. In the interests of setting the imprecise
Bayesian the hardest possible challenge, though, let's say that modesty
requires the third criteria. Her ex ante credence in success should not
just be less than 1, it should be less than some number less than 1.

The aim of the next section is to describe a representor that satisfies
open-mindedness and modesty with respect to the question of whether the
sequence is periodic. The representor will not represent a state that it
is rational for a person to be in; we'll come back in the last section
to the significance of this. My aim is just to show that for the
imprecise Bayesian, unlike the precise Bayesian, open-mindedness and
modesty are compatible. And the proof of this will be constructive; I'll
build a representor that is, while flawed in some other ways,
open-minded and modest.

\section{Meeting the Challenge,
Imprecisely}\label{meeting-the-challenge-imprecisely}

Recall that Pr\textsubscript{0} is the \emph{fair-coin measure},
according to which, if \textbf{y} is any \emph{k} length sequence of 0s
and 1s we have Pr\textsubscript{0}(\textbf{X}\textsubscript{\emph{k}} =
\textbf{y}) = 2\textsuperscript{-\emph{k}}.

Say a finite sequence \textbf{y}\textsubscript{k} of length \emph{k} is
\textbf{repeating} iff for some \emph{n} \textgreater{} 1,
\textbf{y}\textsubscript{k} consists of \emph{n} repetitions of a
sequence of length \emph{k}/\emph{n}. For any non-repeating sequence
\textbf{y}\textsubscript{k} (of length \emph{k}) let
\(\boldsymbol{s}_{\boldsymbol{y}_k}\) be the sequence consisting of
\textbf{y}\textsubscript{k} repeated infinitely often. Let
Pr\textsubscript{1} be the function such that,

\[\Pr{}_1(\boldsymbol{X}= \boldsymbol{s}_{\boldsymbol{y}_k}) = \frac{1}{2^{2k}-1}
\]

Intuitively, we can think of Pr\textsubscript{1} as follows. Consider a
measure over representations of periodic sequences. Any periodic
sequence can be represented just as a finite sequence, plus the
instruction \emph{repeat infinitely often}, so this is really just a
measure over finite sequences. One natural such measure assigns measure
2\textsuperscript{-2\emph{k}} to each sequence of length \emph{k}. Of
course, several of these representations will be representations of the
same sequence. For instance, ⟨0, 1⟩, ⟨0, 1, 0, 1⟩ and ⟨0, 1, 0, 1, 0, 1⟩
repeated infinitely produce the same sequence. Now the probability of a
sequence, according to Pr\textsubscript{1} is just the measure, so
defined, of the class of representations of that measure. (It's a little
easier to confirm that the measures sum to 1 than that the probabilities
do, which is why I've included this little explanation.)

Now define Pr\textsubscript{2} as the equal weight mixture of
Pr\textsubscript{0} and Pr\textsubscript{1}, i.e.,
Pr\textsubscript{2}(\emph{q}) = (Pr\textsubscript{0}(\emph{q}) +
Pr\textsubscript{1}(\emph{q}))/2. Since Pr\textsubscript{0}(\emph{p}) =
0, and Pr\textsubscript{1}(\emph{p}) = 1\$,
Pr\textsubscript{2}(\emph{p}) = ½. There will be several facts about
Pr\textsubscript{2} that are useful to have in place for future
reference. (Recall I'm using \textbf{X} as a random variable for the
sequence the agent will see, \textbf{x} as a rigid designator of that
sequence, \textbf{y} and \textbf{z} are variables for arbitrary
sequences, and the \emph{k} subscript to restrict sequences to length
\emph{k}.) The first of these was proven as Lemma~1.

\begin{description}
\tightlist
\item[Lemma~1.]
Pr\textsubscript{2}(\emph{p}~\textbar{}
\textbf{X}\textsubscript{\emph{k}} = \textbf{y}\textsubscript{k})
\textgreater{} ½ iff
Pr\textsubscript{1}(\textbf{X}\textsubscript{\emph{k}} =
\textbf{y}\textsubscript{k}) \textgreater{}
Pr\textsubscript{0}(\textbf{X}\textsubscript{\emph{k}} =
\textbf{y}\textsubscript{k}).\$
\end{description}

Define a new predicate \emph{n} of finite sequences
\textbf{y}\textsubscript{k}, to hold just in case
\textbf{y}\textsubscript{k} could be the initial segment of an infinite
sequence of period at most \emph{k}/2. So \textbf{y}\textsubscript{k}
must consist of some sequence repeated twice, and anything else in
\textbf{y}\textsubscript{k} must be consistent with that sequence
repeating again (and if necessary again, and again, \ldots). Then we
get,

\begin{description}
\tightlist
\item[Lemma 2.]
For \emph{k} ⩾ 2, Pr\textsubscript{2}(\emph{p}~\textbar{}
\textbf{X}\textsubscript{2\emph{k}} =
\textbf{y}\textsubscript{2\emph{k}}) \textgreater{} ½ iff
\emph{N}\textbf{y}\textsubscript{2\emph{k}}.
\end{description}

\begin{quote}
\emph{Proof.} By Lemma~1, this reduces to the question of the
relationship Pr\textsubscript{1}(\textbf{X}\textsubscript{2\emph{k}} =
\textbf{y}\textsubscript{2\emph{k}}) \textgreater{}
Pr\textsubscript{0}(\textbf{X}\textsubscript{2\emph{k}} =
\textbf{y}\textsubscript{2\emph{k}}). Moreover, we know that
Pr\textsubscript{0}(\textbf{X}\textsubscript{2\emph{k}} =
\textbf{y}\textsubscript{2\emph{k}}) = 2\textsuperscript{-2\emph{k}}. So
the question is whether
Pr\textsubscript{1}(\textbf{X}\textsubscript{2\emph{k}} =
\textbf{y}\textsubscript{2\emph{k}}) \textgreater{}
2\textsuperscript{-2\emph{k}}.

If \emph{N}\textbf{y}\textsubscript{2\emph{k}}, then it is consistent
with \textbf{X}\textsubscript{2\emph{k}} =
\textbf{y}\textsubscript{2\emph{k}} that \textbf{x} is a particular
periodic sequence with period at most \emph{k}. Since the probability,
according to Pr\textsubscript{1} of any such sequence is greater than
2\textsuperscript{-2\emph{k}}, the right-to-left direction follows.

If ¬\emph{N}\textbf{y}\textsubscript{2\emph{k}}, then the possibilities
that get positive probability according to Pr\textsubscript{1} are at
most among the following: \textbf{X} consists of the first \emph{k}~+~1
digits of \textbf{y}\textsubscript{2\emph{k}} repeated endlessly;
\textbf{X} consists of the first \emph{k}~+~2 digits of
\textbf{y}\textsubscript{2\emph{k}} repeated endlessly; \ldots;
\textbf{x} consists of the first 2\emph{k} digits of
\textbf{y}\textsubscript{2\emph{k}} repeated endlessly; \textbf{X} is
one of the two sequences of period 2\emph{k}~+ 1 starting with
\textbf{y}\textsubscript{2\emph{k}}, or one of the four sequences of
period 2\emph{k}~+~2 starting with \textbf{y}\textsubscript{2\emph{k}}
or \ldots. So we get the following, starting with the probabilities of
each of the possibilities listed in the previous sentence,

\[
\begin{aligned}
Pr{}_1(\boldsymbol{X}~2k~ = \boldsymbol{y}~2k~) 
&\leq \frac{1}{2^{2k+2}-1} 
&+&\frac{1}{2^{2k+4}-1} 
&+\dots 
&+&\frac{1}{2^{4k}-1} 
&+&\frac{2}{2^{4k+2}-1} 
&+\dots \\
%
&< \frac{1}{2^{2k+1}} 
&+ &\frac{1}{2^{2k+3}} 
&+ \dots
&+ &\frac{1}{2^{4k-1}}
&+ &\frac{1}{2^{4k}}
&+ \dots \\
%
&< \frac{1}{2^{2k}}
\end{aligned}
\]

And from that the left-to-right direction follows.~◻
\end{quote}

\begin{description}
\tightlist
\item[Lemma 3]
Pr\textsubscript{2} is open-minded.
\end{description}

\begin{quote}
\emph{Proof.} Since any initial sequence \textbf{y}\textsubscript{k}
that is not \emph{N} can be easily extended into one that is \emph{N}
(by, e.g., repeating \textbf{y}\textsubscript{k}), and one is that is
\emph{N} can be extended into one that is not (by, e.g., having the
repeating sequence stop at the very next step), this follows immediately
from Lemma~2.~◻
\end{quote}

Define \emph{f} to be a function from sequences of length \emph{k}~⩾~2
to sequences of length \emph{k}~+~1 such that

\[
f(\boldsymbol{y}_k) = \boldsymbol{y}_k + 
    \begin{cases} 
        \langle 0 \rangle &\text{if } N\boldsymbol{y}_k \leftrightarrow Pr{}_1(x_{k+1} = 0 | \boldsymbol{X}_k = \boldsymbol{y}_k) \leq ½ \\
        \langle 1 \rangle &\text{otherwise}
    \end{cases}
\]

In the normal way, define
\emph{f~\textsuperscript{n}}(\textbf{y}\textsubscript{k}) to be the
result of applying \emph{f} \emph{n} times to
\textbf{y}\textsubscript{k}. And define
\emph{f}\textsuperscript{∞}(\textbf{y}\textsubscript{k}) to be the
infinite sequence we get by doing this infintely often.

Intuitively, the way \emph{f} works is that if
\textbf{y}\textsubscript{k} is already somewhat sequential, then we
include the less likely digit, and if it isn't, then we include the more
likely digit. (With ties resolved in favour of including 0 rather than
1.) If we define \emph{p}(\textbf{y}\textsubscript{k}) to be the
smallest \emph{n} such that \textbf{y}\textsubscript{k} could be the
initial segment of a periodic sequence of length \emph{n}, then we'll
get that
\emph{p}(\emph{f}(\textbf{y}\textsubscript{k}))~\textgreater~\emph{p}(\textbf{y}\textsubscript{k})
↔ \emph{N}\textbf{y}\textsubscript{k} in all cases, except for the case
where Pr\textsubscript{1}(\textbf{x}\textsubscript{\emph{k}} =
0~\textbar~\textbf{X}\textsubscript{\emph{k}}~=~\textbf{y}\textsubscript{k})
= ½. That is, if \emph{N}\textbf{y}\textsubscript{k}, then extending
\textbf{y}\textsubscript{k} in this way will wipe out the possibility of
that smallest sequence being extended indefinitely, while if
¬\emph{N}\textbf{y}\textsubscript{k}, then that possibility will still
be on the table.

From this, it follows that
\emph{f}\textsuperscript{∞}(\textbf{y}\textsubscript{k}) will flummox
Pr\textsubscript{2}, no matter which \textbf{y}\textsubscript{k} we
start with.

We need one last classification of finite sequences, and then we are
done. Say that \emph{O}\textbf{y}\textsubscript{k} just in case some
initial segment of \textbf{y}\textsubscript{k} of length \emph{r} could
be the initial segment of an infinite period sequence of period less
than \emph{r}/2. This contrasts with \emph{N} in two ways. First, it
requires a sequence that repeats twice, and then starts a third
repetition. Second, it does not require that the sequence be `live';
there might be subsequent parts of \textbf{y}\textsubscript{k} that are
not compatible with the sequence repeating. So the sequence ⟨0, 0, 1, 0,
0, 1⟩ satisfies \emph{N} but not \emph{O}, while the sequence ⟨0, 1, 0,
1, 0, 0⟩ satisfies \emph{O} but not \emph{N}.

There are a countable infinity of finite sequences
\textbf{y}\textsubscript{k} such that
¬\emph{O}\textbf{y}\textsubscript{k}. Produce some ordering of them,
then define Pr\textsubscript{\emph{i}}, for \emph{i}~⩾~3, to be the
probability function such that
Pr\textsubscript{\emph{i}}(\textbf{X}~=~\emph{f}\textsuperscript{∞}(\textbf{y}\textsubscript{k}))
= 1, where \textbf{y}\textsubscript{k} is the \emph{i}-2'th sequence in
this order.

Now, consider the set \emph{R} of all probability functions of the form:

\[
Pr = \sum_{i = 2}^\infty a_i \Pr{}_i
\]

where each of the Pr\textsubscript{\emph{i}} are defined as above, each
\emph{a\textsubscript{i}} is non-negative, \emph{a}\textsubscript{2} is
½, and the sum of the \emph{a\textsubscript{i}} from 3 to ∞ is also ½.
Intuitively, each function starts by halving the probability
Pr\textsubscript{2} gives to each initial (or completed) sequence, and
distributing the remaining probability over the countable infinity of
flummoxing sequences of the form
\emph{f}\textsuperscript{∞}(\textbf{y}\textsubscript{k}), where
¬\emph{O}\textbf{y}\textsubscript{k}.

I'll now prove that \emph{R} is open minded.

\begin{description}
\tightlist
\item[Lemma 4]
If ¬\emph{O}\textbf{y}\textsubscript{k}, then ¬\emph{O}
f(\textbf{y}\textsubscript{k}).
\end{description}

\begin{quote}
\emph{Proof.} Since ¬\emph{O}\textbf{y}\textsubscript{k}, the only way
that \emph{O} \emph{f}(\textbf{y}\textsubscript{k}) could be true is if
\emph{k} = 2\emph{r} +1, and \emph{f}(\textbf{y}\textsubscript{k})
consists of some sequence of length \emph{r} repeated twice, plus the
first digit repeated a third time. But that means that
\emph{N}\textbf{y}\textsubscript{k}. And if that's the case, then the
extra digit that is added by \emph{f}(\textbf{y}\textsubscript{k}) will
not be the necessary digit to repeat this sequence. So it is impossible
that \emph{O} \emph{f}(\textbf{y}\textsubscript{k}).~◻
\end{quote}

\begin{description}
\tightlist
\item[Lemma 5]
If ¬\emph{O} \textbf{y}\textsubscript{k}, then ¬\emph{O}
\emph{f}\textsuperscript{∞}(\textbf{y}\textsubscript{k}).
\end{description}

\begin{quote}
\emph{Proof.} This follows trivially from Lemma~4.~◻
\end{quote}

\begin{description}
\tightlist
\item[Theorem 6]
\emph{R} is open-minded.
\end{description}

\begin{quote}
\emph{Proof.} Any initial sequence can be extended to a sequence
satisfying \emph{O}. For example, the initial sequence can be repeated
in full twice. An immediate consequence of Lemma~5 is that for all
\emph{i}~⩾~3, \emph{O}\textbf{y}\textsubscript{k} →
Pr\textsubscript{\emph{i}}(\textbf{X}\textsubscript{\emph{k}} =
\textbf{y}\textsubscript{k}) = 0. That means that if
\emph{O}\textbf{y}\textsubscript{k}, then for any Pr ∈ \emph{R},
Pr(\emph{p}~\textbar{} \textbf{X}\textsubscript{\emph{k}} =
\textbf{y}\textsubscript{k}) = Pr\textsubscript{2}(\emph{p}~\textbar{}
\textbf{X}\textsubscript{\emph{k}} = \textbf{y}\textsubscript{k}). And
now the theorem is an immediate consequence of Lemma~3.~◻
\end{quote}

Let \emph{F} be the set of all sequences
f\textsuperscript{∞}(\textbf{y}\textsubscript{k})\$, where ¬\emph{O}
\textbf{y}\textsubscript{k}.

\begin{description}
\tightlist
\item[Lemma 7]
If \textbf{x}~∈~\emph{F}, then \emph{R} fails.
\end{description}

\begin{quote}
\emph{Proof.} Assume \textbf{x}~∈~\emph{F}, so \textbf{x} is not
periodic. Then proving the lemma requires showing that for any \emph{i},
there is a \emph{j}~⩾~\emph{i} such that, according to \emph{R}, the
probability of \emph{p} given \textbf{X}\textsubscript{\emph{j}}
=\textbf{x}\textsubscript{\emph{j}} is not less than ½. And that
requires showing that there is a Pr~∈~\emph{R} such that
Pr(\emph{p}~\textbar~\textbf{X}\textsubscript{\emph{j}}~=~\textbf{x}\textsubscript{\emph{j}})~⩾~½.
This is easy to do. Consider any sequence
\textbf{y}\textsubscript{\emph{i}} of length \emph{i} not identical to
\textbf{x}\textsubscript{\emph{i}} such that ¬\emph{O}
\textbf{y}\textsubscript{\emph{i}}. Consider the probability function
Pr\textsubscript{k}~∈~\emph{R} such that
Pr\textsubscript{k}(\textbf{X}~=~\emph{f}\textsuperscript{∞}(\textbf{y}\textsubscript{\emph{i}}))
= ½. Once we conditionalise on \textbf{X}\textsubscript{\emph{i}} =
\textbf{x}\textsubscript{\emph{i}}, that function will behave just like
Pr\textsubscript{2}. And since \textbf{X} flummoxes Pr\textsubscript{2},
that means there is a \textbf{x}\textsubscript{\emph{j}} such that
Pr(\emph{p}~\textbar{} \textbf{X}\textsubscript{\emph{j}} =
\textbf{x}\textsubscript{\emph{j}}) \textgreater{} ½, and hence
Pr(\emph{p}~\textbar{} \textbf{X}\textsubscript{\emph{j}} =
\textbf{x}\textsubscript{\emph{j}}) ⩾½.~◻
\end{quote}

\begin{description}
\tightlist
\item[Lemma 8]
For each Pr ∈ \emph{R}, Pr(\textbf{x}~∈ \emph{F}) = ½.
\end{description}

\begin{quote}
\emph{Proof.} It helps to think of each of the Pr ∈ \emph{R} as mixtures
of Pr\textsubscript{0} and Pr\textsubscript{1}, plus a mixture of the
Pr\textsubscript{\emph{i}} for \emph{i}~⩾~3. Now
Pr\textsubscript{0}(\textbf{x}~∈~\emph{F}) = 0, since for any countable
set, Pr\textsubscript{0} says the probability that \textbf{x} is in that
set is 0. And Pr\textsubscript{1}(\textbf{x}~∈~\emph{F}) = 0, since
Pr\textsubscript{1} says that the probability of \textbf{x} being
periodic is 1, and none of the members of \emph{F} are periodic. But for
each Pr\textsubscript{\emph{i}} for \(i ⩾3\),
Pr\textsubscript{\emph{i}}(\textbf{x}~∈~\emph{F}) = 1\$. Indeed, for
each such function, there is a particular sequence in \emph{F} such that
the probability that \textbf{x} is that sequence is 1. So for each Pr ∈
R, Pr(\textbf{x}~∈~\emph{F}) = ¼ × 0 + ¼ × 0 + ½ × 1 = ½.~◻
\end{quote}

\begin{description}
\tightlist
\item[Theorem 9]
According to \emph{R}, the probability of an agent whose representor is
\emph{R} failing is at least ½.
\end{description}

\begin{quote}
\emph{Proof.} Immediate from Lemma~7 and Lemma 8.~◻
\end{quote}

So if an agent's credences are represented by a non-singleton set of
probability functions, not a single probability function, it is possible
for them to be open-minded and modest. On the other hand, if an agent is
represented by a single probability function, as the precise Bayesian
desires, then it is impossible to be open-minded and modest. Since being
open-minded and modest is desirable, this is a reason to prefer the
imprecise Bayesian picture.

\section{Objections and Replies}\label{objections-and-replies}

I'm going to reply to three objections, but since my replies overlap,
I'll group the objections together.

\begin{description}
\item[Objection 1]
The model here only gives you conditional modesty. Once the initial
sequence is \emph{O}, the representor becomes the singleton of an
open-minded probability function, and Belot showed that to be immodest.
Ideally, the agent would have a prior that is in some way resiliently
modest, whereas this prior is fragilely modest.
\item[Objection 2]
This representor is open-minded and modest towards one particular
problem, namely whether \textbf{X} is periodic. But Belot was interested
in a wider range of problems, indeed in all problems of the form: does
\textbf{x} fall into some set that is measurable, dense, and has a dense
complement. Ideally, we'd have a prior which is widely open-minded and
modest, in the sense that it had an open-minded and modest attitude
towards many problems. But this prior is narrowly modest, in the sense
that it is open-minded and modest about only one problem.
\item[Objection 3]
The representor described here is clearly not a representation of a
credal state of anyone rational. Look what it does if the data is a 1
followed by thousands of 0s, or is the first few thousand digits of the
binary expansion of π, or has a frequency of 0s of 0.2 over all large
sub-intervals. No one could adopt this prior, so it doesn't show
anything about the advantages of imprecise Bayesianism.
\end{description}

\emph{Reply.} My responses are going to be (1) that we should want more
resilient modesty, and though this is a hard technical challenge, it's
possible to see a way forward on it, (2) that we should want somewhat
wider open-minded modesty, though how much wider is a hard question, and
(3) that the third objection should simply be rejected. Let's go through
those in reverse order, since it's the response to the third that
explains part of what I'm doing in response to the other two.

What we have in section three is a consistency proof. For the imprecise
Bayesian, unlike the precise Bayesian, being open-minded is consistent
with being modest. That's good, since it shows that we can't rule out a
rational response to problems like Belot's. It's obviously true that the
prior in question isn't rational, but that's not needed for a
consistency proof.

Moreover, we don't just have a consistency proof, we have a constructive
consistency proof - the prior is described in detail. It's just not
going to be possible to do a constructive proof that open-mindedness,
modesty and full rationality are consistent. And that's because to do
that would essentially be to solve all of the problems of epistemology
ever. Demonstrating a fully rational prior, even for the range of
questions Belot considers, is too much to ask.

If there's a reasonable looking argument that imprecise Bayesians are
unlikely to be able to satisfy some set of plausible constraints, then
the defender of imprecise Bayesianism is, I think, obliged to show how
those constraints can be satisfied. But to ask for a demonstration of
how all reasonable constraints can be satisfied at once, in the absence
of a decent argument that they cannot be, would clearly be asking too
much.

So I don't care that the prior I described is irrational; it serves its
purpose in proving consistency. Now what would be nice is to show that
some slightly stronger constraints can be simultaneously satisfied. But
we have to be sure that those constraints are in fact reasonable
constraints. Here's one constraint that I think isn't reasonable: be
open-minded towards any proposition of the form
\{\textbf{X}~∈~\emph{S}\}, where \emph{S} is a dense set of sequences.
Let \emph{S}, for example, be the set consisting of all sequences of the
form \textbf{y}\textsubscript{k} + \textbf{z}, where
\textbf{y}\textsubscript{k} ranges over all finite sequeneces, and
\textbf{z} is a particular arbitrary sequence that lacks finite
definition in our current language. That set is dense, and indeed
measurable. But there's no evidence that could make it reasonable to
take \textbf{X}~∈~\emph{S} to be probable. So a prior that wasn't
open-minded towards \textbf{X}~∈~\emph{S} could still be perfectly
reasonable.

That said, the prior I demonstrated is closed-minded towards several
propositions that should be taken seriously. It will never have positive
credence that \textbf{X} is eventually periodic without being periodic,
or that \textbf{X} is generated by a chance process that gives each data
point chance \emph{c}~≠~½ of being 0. It would be good to have a prior
whose open-minded modesty was wider. But before we do that technical
work, I think there's a need to figure out which propositions we should
be open-minded about.

I am more worried by the fragility of the modesty of this prior. There's
a reasonable sense in which the prior is open-minded only in virtue of
the fact that it has parts which are immodest. At any point where the
agent has credence above ½ that \emph{p}, she has credence 1 that she
will succeed.

We could try to complicate the prior a bit more to avoid that. Here's a
sketch of how it could go, with application to one particular initial
sequence of data. Consider what happens to \emph{R} if the initial input
is ⟨0, 1, 0, 0, 1, 0, 0, 1⟩, hereafter \textbf{y}. According to
Pr\textsubscript{0}, that initial sequence has probability 1/256.
According to Pr\textsubscript{1}, it has probability 1/63 + 1/4095 +
1/65535 ≈ 1/62. So given that initial sequence, Pr\textsubscript{2} says
the probability of \emph{p} is about 0.8. And since the sequence is
\emph{O}, it could be the start of the the sequence ⟨0, 1, 0⟩ repeated
indefinitely, its probability according to Pr\textsubscript{\emph{i}} is
0, for \emph{i}~⩾~3. Now consider the set of all probability functions
of the form
\emph{a}Pr\textsubscript{\emph{R}}~+~\emph{b}Pr\textsubscript{\emph{New}},
where \emph{a} + \emph{b} = 1, \emph{b} ∈ (0, 1/256),
Pr\textsubscript{\emph{R}}~∈~\emph{R} and Pr\textsubscript{\emph{New}}
is the function which gives probability 1 to \textbf{X} being
\emph{O}(\textbf{y}). That prior is open-minded, and even after
conditionalising on \textbf{y} satisfies the intermediate of the three
modesty conditions described on page - the probability of failure is
less than one, though it isn't less than some number less than one. And
this trick could be generalised to satisfy more modesty conditions, and
even (though it would take some time to prove this) be unconditionally
modest.

But I'm not going to go through those steps here. That's mostly because
I think we already have shown enough to show that imprecise Bayesianism
has an advantage over precise Bayesianism. The imprecise Bayesian can,
and the precise Bayesian can't, have an open-minded modest attitude. It
would be good to press home that advantage and show that there are other
things the imprecise Bayesian can do that the precise Bayesian can't do,
such as having a widely open-minded and resiliently modest prior. But
even before such a demonstration takes place, the advantage has been
established.

\subsection*{References}\label{references}
\addcontentsline{toc}{subsection}{References}

\phantomsection\label{refs}
\begin{CSLReferences}{1}{0}
\bibitem[\citeproctext]{ref-Belot2013}
Belot, Gordon. 2013. {``Bayesian Orgulity.''} \emph{Philosophy of
Science} 80 (4): 483--503. doi:
\href{https://doi.org/10.1086/673249}{10.1086/673249}.

\bibitem[\citeproctext]{ref-Bradley2014}
Bradley, Seamus. 2014. {``Imprecise Probabilities.''} In \emph{The
Stanford Encyclopedia of Philosophy}, edited by Edward N. Zalta, Winter
2014.
\url{http://plato.stanford.edu/archives/win2014/entries/imprecise-probabilities/};
Metaphysics Research Lab, Stanford University.

\bibitem[\citeproctext]{ref-Briggs2009}
Briggs, Ray. 2009. {``Distorted Reflection.''} \emph{Philosophical
Review} 118 (1): 59--85. doi:
\href{https://doi.org/10.1215/00318108-2008-029}{10.1215/00318108-2008-029}.

\bibitem[\citeproctext]{ref-Elga2010-ELGHTD}
Elga, Adam. 2010. {``How to Disagree about How to Disagree.''} In
\emph{Disagreement}, edited by Ted Warfield and Richard Feldman,
175--87. Oxford: Oxford University Press.

\bibitem[\citeproctext]{ref-Joyce2010}
Joyce, James M. 2010. {``A Defence of Imprecise Credences in Inference
and Decision Making.''} \emph{Philosophical Perspectives} 24 (1):
281--323. doi:
\href{https://doi.org/10.1111/j.1520-8583.2010.00194.x}{10.1111/j.1520-8583.2010.00194.x}.

\bibitem[\citeproctext]{ref-Lasonen-Aarnio2015}
Lasonen-Aarnio, Maria. 2015. {``New Rational Reflection and Internalism
about Rationality.''} \emph{Oxford Studies in Epistemology} 5: 145--71.
doi:
\href{https://doi.org/10.1093/acprof:oso/9780198722762.003.0005}{10.1093/acprof:oso/9780198722762.003.0005}.

\bibitem[\citeproctext]{ref-Lewis1971d}
Lewis, David. 1971. {``Immodest Inductive Methods.''} \emph{Philosophy
of Science} 38 (1): 54--63. doi:
\href{https://doi.org/10.1086/288339}{10.1086/288339}.

\bibitem[\citeproctext]{ref-Lewis1993c}
---------. 1993. {``Many, but Almost One.''} In \emph{Ontology,
Causality, and Mind: Essays on the Philosophy of {D. M. Armstrong}},
edited by Keith Campbell, John Bacon, and Lloyd Reinhardt, 23--38.
Cambridge: Cambridge University Press. doi:
\href{https://doi.org/10.1017/CBO9780511625343.010}{10.1017/CBO9780511625343.010}.
Reprinted in his \emph{Papers in Metaphysics and Epistemology},
Cambridge: Cambridge University Press, 1999, 164-182. References to
reprint.

\bibitem[\citeproctext]{ref-Pryor2000}
Pryor, James. 2000. {``The Sceptic and the Dogmatist.''} \emph{No{û}s}
34 (4): 517--49. doi:
\href{https://doi.org/10.1111/0029-4624.00277}{10.1111/0029-4624.00277}.

\bibitem[\citeproctext]{ref-Schoenfield2012}
Schoenfield, Miriam. 2012. {``Chilling Out on Epistemic Rationality: A
Defense of Imprecise Credences (and Other Imprecise Doxastic
Attitudes).''} \emph{Philosophical Studies} 158 (2): 197--219. doi:
\href{https://doi.org/10.1007/s11098-012-9886-7}{10.1007/s11098-012-9886-7}.

\bibitem[\citeproctext]{ref-Schoenfield2014}
---------. 2015. {``A Dilemma for Calibrationism.''} \emph{Philosophy
and Phenomenological Research} 91 (2): 425--55. doi:
\href{https://doi.org/10.1111/phpr.12125}{10.1111/phpr.12125}.

\bibitem[\citeproctext]{ref-Walley1991}
Walley, Peter. 1991. \emph{Statisical Reasoning with Imprecise
Probabilities}. London: Chapman \& Hall.

\bibitem[\citeproctext]{ref-White2006}
White, Roger. 2006. {``Problems for Dogmatism.''} \emph{Philosophical
Studies} 131 (3): 525--57. doi:
\href{https://doi.org/10.1007/s11098-004-7487-9}{10.1007/s11098-004-7487-9}.

\bibitem[\citeproctext]{ref-White2010}
---------. 2010. {``Evidential Symmetry and Mushy Credence.''}
\emph{Oxford Studies in Epistemology} 3: 161--89.

\end{CSLReferences}



\noindent Published in\emph{
Ergo}, 2016, pp. 529-545.


\end{document}
