% Options for packages loaded elsewhere
\PassOptionsToPackage{unicode}{hyperref}
\PassOptionsToPackage{hyphens}{url}
%
\documentclass[
  10pt,
  letterpaper,
  DIV=11,
  numbers=noendperiod,
  twoside]{scrartcl}

\usepackage{amsmath,amssymb}
\usepackage{setspace}
\usepackage{iftex}
\ifPDFTeX
  \usepackage[T1]{fontenc}
  \usepackage[utf8]{inputenc}
  \usepackage{textcomp} % provide euro and other symbols
\else % if luatex or xetex
  \usepackage{unicode-math}
  \defaultfontfeatures{Scale=MatchLowercase}
  \defaultfontfeatures[\rmfamily]{Ligatures=TeX,Scale=1}
\fi
\usepackage{lmodern}
\ifPDFTeX\else  
    % xetex/luatex font selection
  \setmainfont[ItalicFont=EB Garamond Italic,BoldFont=EB Garamond
Bold]{EB Garamond Math}
  \setsansfont[]{Europa-Bold}
  \setmathfont[]{Garamond-Math}
\fi
% Use upquote if available, for straight quotes in verbatim environments
\IfFileExists{upquote.sty}{\usepackage{upquote}}{}
\IfFileExists{microtype.sty}{% use microtype if available
  \usepackage[]{microtype}
  \UseMicrotypeSet[protrusion]{basicmath} % disable protrusion for tt fonts
}{}
\usepackage{xcolor}
\usepackage[left=1in, right=1in, top=0.8in, bottom=0.8in,
paperheight=9.5in, paperwidth=6.5in, includemp=TRUE, marginparwidth=0in,
marginparsep=0in]{geometry}
\setlength{\emergencystretch}{3em} % prevent overfull lines
\setcounter{secnumdepth}{3}
% Make \paragraph and \subparagraph free-standing
\ifx\paragraph\undefined\else
  \let\oldparagraph\paragraph
  \renewcommand{\paragraph}[1]{\oldparagraph{#1}\mbox{}}
\fi
\ifx\subparagraph\undefined\else
  \let\oldsubparagraph\subparagraph
  \renewcommand{\subparagraph}[1]{\oldsubparagraph{#1}\mbox{}}
\fi


\providecommand{\tightlist}{%
  \setlength{\itemsep}{0pt}\setlength{\parskip}{0pt}}\usepackage{longtable,booktabs,array}
\usepackage{calc} % for calculating minipage widths
% Correct order of tables after \paragraph or \subparagraph
\usepackage{etoolbox}
\makeatletter
\patchcmd\longtable{\par}{\if@noskipsec\mbox{}\fi\par}{}{}
\makeatother
% Allow footnotes in longtable head/foot
\IfFileExists{footnotehyper.sty}{\usepackage{footnotehyper}}{\usepackage{footnote}}
\makesavenoteenv{longtable}
\usepackage{graphicx}
\makeatletter
\def\maxwidth{\ifdim\Gin@nat@width>\linewidth\linewidth\else\Gin@nat@width\fi}
\def\maxheight{\ifdim\Gin@nat@height>\textheight\textheight\else\Gin@nat@height\fi}
\makeatother
% Scale images if necessary, so that they will not overflow the page
% margins by default, and it is still possible to overwrite the defaults
% using explicit options in \includegraphics[width, height, ...]{}
\setkeys{Gin}{width=\maxwidth,height=\maxheight,keepaspectratio}
% Set default figure placement to htbp
\makeatletter
\def\fps@figure{htbp}
\makeatother
% definitions for citeproc citations
\NewDocumentCommand\citeproctext{}{}
\NewDocumentCommand\citeproc{mm}{%
  \begingroup\def\citeproctext{#2}\cite{#1}\endgroup}
\makeatletter
 % allow citations to break across lines
 \let\@cite@ofmt\@firstofone
 % avoid brackets around text for \cite:
 \def\@biblabel#1{}
 \def\@cite#1#2{{#1\if@tempswa , #2\fi}}
\makeatother
\newlength{\cslhangindent}
\setlength{\cslhangindent}{1.5em}
\newlength{\csllabelwidth}
\setlength{\csllabelwidth}{3em}
\newenvironment{CSLReferences}[2] % #1 hanging-indent, #2 entry-spacing
 {\begin{list}{}{%
  \setlength{\itemindent}{0pt}
  \setlength{\leftmargin}{0pt}
  \setlength{\parsep}{0pt}
  % turn on hanging indent if param 1 is 1
  \ifodd #1
   \setlength{\leftmargin}{\cslhangindent}
   \setlength{\itemindent}{-1\cslhangindent}
  \fi
  % set entry spacing
  \setlength{\itemsep}{#2\baselineskip}}}
 {\end{list}}
\usepackage{calc}
\newcommand{\CSLBlock}[1]{\hfill\break\parbox[t]{\linewidth}{\strut\ignorespaces#1\strut}}
\newcommand{\CSLLeftMargin}[1]{\parbox[t]{\csllabelwidth}{\strut#1\strut}}
\newcommand{\CSLRightInline}[1]{\parbox[t]{\linewidth - \csllabelwidth}{\strut#1\strut}}
\newcommand{\CSLIndent}[1]{\hspace{\cslhangindent}#1}

\setlength\heavyrulewidth{0ex}
\setlength\lightrulewidth{0ex}
\usepackage[automark]{scrlayer-scrpage}
\clearpairofpagestyles
\cehead{
  Brian Weatherson
  }
\cohead{
  Interest-Relative Invariantism
  }
\ohead{\bfseries \pagemark}
\cfoot{}
\makeatletter
\newcommand*\NoIndentAfterEnv[1]{%
  \AfterEndEnvironment{#1}{\par\@afterindentfalse\@afterheading}}
\makeatother
\NoIndentAfterEnv{itemize}
\NoIndentAfterEnv{enumerate}
\NoIndentAfterEnv{description}
\NoIndentAfterEnv{quote}
\NoIndentAfterEnv{equation}
\NoIndentAfterEnv{longtable}
\NoIndentAfterEnv{abstract}
\renewenvironment{abstract}
 {\vspace{-1.25cm}
 \quotation\small\noindent\rule{\linewidth}{.5pt}\par\smallskip
 \noindent }
 {\par\noindent\rule{\linewidth}{.5pt}\endquotation}
\KOMAoption{captions}{tableheading}
\makeatletter
\@ifpackageloaded{caption}{}{\usepackage{caption}}
\AtBeginDocument{%
\ifdefined\contentsname
  \renewcommand*\contentsname{Table of contents}
\else
  \newcommand\contentsname{Table of contents}
\fi
\ifdefined\listfigurename
  \renewcommand*\listfigurename{List of Figures}
\else
  \newcommand\listfigurename{List of Figures}
\fi
\ifdefined\listtablename
  \renewcommand*\listtablename{List of Tables}
\else
  \newcommand\listtablename{List of Tables}
\fi
\ifdefined\figurename
  \renewcommand*\figurename{Figure}
\else
  \newcommand\figurename{Figure}
\fi
\ifdefined\tablename
  \renewcommand*\tablename{Table}
\else
  \newcommand\tablename{Table}
\fi
}
\@ifpackageloaded{float}{}{\usepackage{float}}
\floatstyle{ruled}
\@ifundefined{c@chapter}{\newfloat{codelisting}{h}{lop}}{\newfloat{codelisting}{h}{lop}[chapter]}
\floatname{codelisting}{Listing}
\newcommand*\listoflistings{\listof{codelisting}{List of Listings}}
\makeatother
\makeatletter
\makeatother
\makeatletter
\@ifpackageloaded{caption}{}{\usepackage{caption}}
\@ifpackageloaded{subcaption}{}{\usepackage{subcaption}}
\makeatother
\ifLuaTeX
  \usepackage{selnolig}  % disable illegal ligatures
\fi
\usepackage{bookmark}

\IfFileExists{xurl.sty}{\usepackage{xurl}}{} % add URL line breaks if available
\urlstyle{same} % disable monospaced font for URLs
\hypersetup{
  pdftitle={Interest-Relative Invariantism},
  pdfauthor={Brian Weatherson},
  hidelinks,
  pdfcreator={LaTeX via pandoc}}

\title{Interest-Relative Invariantism}
\author{Brian Weatherson}
\date{2017}

\begin{document}
\maketitle
\begin{abstract}
An opinionated survey of the state of the literature on
interest-relative invariantism.
\end{abstract}

\setstretch{1.1}
\section{Introduction}\label{introduction}

One of the initial motivations for epistemological contextualism was
that the appropriateness of self-ascriptions of knowledge seemed to
depend, in some circumstances, on factors that were traditionally
thought to be epistemologically irrelevant. So whether our hero \emph{S}
was prepared to say ``I know that \emph{p}'' would depend not just on
how strong \emph{S}'s evidence for \emph{p} was, or how strongly they
believed it, but on factors such as how much it mattered whether
\emph{p} was true, or what alternatives to \emph{p} were salient in
their thought or talk.

It was immediately noted that this data point, even if accepted, is
consistent with a number of theories of the truth of knoweldge
ascriptions. It might be that things like stakes and salient
alternatives affect the assertability conditions of knowledge
ascriptions, but not their truth conditions
~(\citeproc{ref-Rysiew2016}{Rysiew 2017}). But let's assume that we've
convinced ourselves that this isn't right, and that whether \emph{S} can
truly (and not just appropriately) say ``I know that \emph{p}'' depends
on things like the stakes or salient alternatives.

It still doesn't follow that contextualism is true. It might be that in
all contexts, whether an utterance of ``S knows that \emph{p}'' is true
depends on the stakes for \emph{S}, or on the salient alternatives for
\emph{S}. That would be true, the idea is, whether \emph{S} is talking
about herself, or someone else is talking about her. The stakes, or
salient alternatives, would affect the truth conditions of \emph{S}'s
utterance not because she is the one doing the talking, but the one
being talked about. The practical and theoretical situation of the
ascribee of the knowledge ascription may be relevant, even if the
practical and theoretical situation of the ascrier need not be.

This line of thought leads to the idea that knowledge itself is
interest-relative. Whether an utterance here and now of ``S knows that
\emph{p}'' is true, i.e., whether \emph{S} knows that \emph{p}, depends
on how much it matters to \emph{S} that \emph{p} is true, or on which
alternative are salient to \emph{S}. The thesis that knowledge is
interest-relative is consistent with contextualism. It could be that
whether a knowledge ascription is true depends on the interests of both
the ascriber, and the ascribee. In this entry, however, I'm going to
largely focus on the view that knowledge is interest-relative, but
contextualism is false. On this view, the interests of the ascribee do
matter to the truth of a knowledge ascription, but the interests of the
ascribee do not.

This view is naturally called \textbf{interest-relative invariantism},
since it makes knowledge interest-relative, but it is a form of
anti-contextualism, i.e., invariantism. The view is sometimes called
\textbf{subject-sensitive invariantism}, since it makes knowledge
relevant to the stakes and salient alternatives to the subject. But this
is a bad name; of course whether a knowledge ascription is true is
sensitive to who the subject of the ascription is. I know what I had for
breakfast and you (probably) don't. What is distinctive is which
features of the subject's situation that interest-relative invariantism
says are relevant, and the name interest-relative invariantism makes it
clear that it is the subject's interests. There is one potential
downside to this name; it suggests that the practical interests of the
subject are relevant to what they know. I intend to use the predicate
`interest-relative' to pick out a class of theories, including the
theory floated by John Hawthorne (\citeproc{ref-Hawthorne2004}{2004}),
where the options that are salient to the subject matter to what the
subject knows. If forced to defend the name, I'd argue that salience is
relevant to the theoretical interests of the subject, if not necessarily
to their practical interests. But the name is still potentially
misleading; my main reason for using it is that `subject-sensitive' is
even more misleading. (I'll shorten `interest-relative invariantism' to
IRI in what follows. I'll return to the question of practical and
theoretical interests in section 4.)

There are a number of ways to motivate and precisify IRI. I'll spend
most of this entry going over the choice points, starting with the
points where I think there is a clearly preferably option, and ending
with the choices where I think it's unclear which way to go. Then I'll
discuss some general objections to IRI, and say how they might be
answered.

\section{Motivations}\label{motivations}

There are two primary motivations for IRI. One comes from intuitions
about cases, the other from a pair of principles. It turns out the two
are connected, but it helps to start seeing them separately.

Jason Stanley (\citeproc{ref-Stanley2005}{2005}) starts with some
versions of the `bank cases' due originally to Keith DeRose
(\citeproc{ref-DeRose1992}{1992}). These turn on idiosyncratic, archaic
details of the US payments system, and I find it hard to have clear
intuitions about them. A cleaner pair of examples is provided by Angel
Pinillos (\citeproc{ref-Pinillos2012}{2012}); here are slightly modified
versions of his examples.

\begin{quote}
Ankita and Bojan each have an essay due. They have, surprisingly,
written word for word identical papers, and are now checking the paper
for typos. The papers have no typos, and each student has checked their
paper twice, with the same dictionary, and not found any typos. They
are, in general, equally good at finding typos, and have true beliefs
about their proficiency at typo-spotting.

The only difference between them concerns the consequence of a typo
remaining. If the paper is a borderline A/A- paper, a typo might mean
Ankita gets an A- rather than an A. But the grade doesn't matter to her;
she's already been accepted into a good graduate program next year so
long as she gets above a C. But Bojan's instructor is a stickler for
spelling. Any typo and he gets a C on the paper. And he has a very
lucrative scholarship that he loses if he doesn't get at least a B on
this paper. (Compare the Typo-Low and Typo-High examples in Pinillos
(\citeproc{ref-Pinillos2012}{2012, 199}).)
\end{quote}

The intuition that helps IRI is that Ankita knows she has no typos in
her paper, and should turn it in, while Bojan does not know this, and
should do a third (and perhaps fourth or fifth) check. Contextualists
have a hard time explaining this; in this very context I can say
``Ankita knows her paper has no typos, but Bojan does not know his paper
has no typos''. If the intuition is right, it seems to support
interest-relativity, since the difference in practical situation between
Ankita and Bojan seems best placed to explain their epistemic
difference. Alternatively, if there is a single context within which one
can truly say ``Ankita knows her paper has no typos'', and ``Bojan does
not know his paper has no typos'', that's again something an
interest-invariant contextualism can't explain. Either way, we have an
argument from cases for a form of interest-relativity.

The argument from principles takes off from the idea that knowledge
plays an important role in good deliberation, and that knowledge does
not require maximal confidence. It is easiest to introduce with an
example, though note that we aren't going to rely on epistemic
intuitions about the example. Chika looked at the baseball scores last
night before going to bed and saw that the Red Sox won. She remembers
this when she wakes up, though she knows that she does sometimes
misremember baseball scores. She is then faced with the following
choice: take the red ticket, which she knows pays \$1 if the Red Sox won
last night, and nothing otherwise or the blue ticket, which she knows
pays \$1 iff 2+2=4, and nothing otherwise. Now consider the following
principle, named by Jessica Brown (\citeproc{ref-Brown2013}{2014}):

\begin{description}
\tightlist
\item[K-Suff]
If \emph{S} knows that \emph{p}, then \emph{S} can rationally take
\emph{p} as given in practical deliberation.
\end{description}

The following trio seems to be inconsistent:

\begin{enumerate}
\def\labelenumi{\arabic{enumi}.}
\tightlist
\item
  Chika knows the Red Sox won last night.
\item
  Chika is rationally required to take the blue ticket.
\item
  K-Suff is true.
\end{enumerate}

By 1 and 3, Chika can take for granted that the Red Sox won last night.
So the value of the red ticket, for her, is equal to its value
conditional on the Red Sox winning. And that is \$1. So it is at least
as valuable as the blue ticket. So she can't be rationally required to
take the blue ticket. Hence the three propositions are inconsistent.

This is worrying for two reasons. For one thing, it is intuitive that
Chika knows that the Red Sox won. For another thing, it seems this form
of argument generalises. For almost any proposition at all, if Chika
knows the red ticket pays out iff that proposition is true, she should
prefer the blue ticket. So she knows very little.

How could this argument be resisted? One move, which we'll return to
frequently, is to deny K-Suff. Maybe Chika's knowledge that the Red Sox
won is insufficient; she needs to be certain, or to have some higher
order knowledge. But denying K-Suff alone will not explain why Chika
should take the blue ticket. After all, if K-Suff is false, the fact
that Chika knows the payout terms of the tickets is not in itself a
reason for her to choose the blue ticket.

So perhaps we could deny that she is rationally required to choose the
blue ticket. This does seem extremely unintuitive to me. Intuitions
around here do not seem maximally reliable, but this is a strong enough
intuition to make it worthwhile to explore other options.

And IRI provides a clever way out of the dilemma. Chika does not know
the Red Sox won last night. But she did know that, before the choice was
offered. Once she has that choice, her knowledge changes, and now she
does not know. The intuition that she knows is explained by the fact
that relative to a more normal choice set, she can take the fact that
the Red Sox won as a given. And scepticism is averted because Chika does
normally know a lot; it's just in the context of strange choices that
she loses knowledge.

The plotline here, that principles connecting knowledge and action run
up against anti-sceptical principles in contrived choice situations, and
that IRI provides a way out of the tangle, is familiar. It is,
simplifying greatly, the argumentative structure put forward by
Hawthorne (\citeproc{ref-Hawthorne2004}{2004}), and by
(\citeproc{ref-FantlMcGrath2002}{Fantl and McGrath 2002},
\citeproc{ref-FantlMcGrath2009}{2009}), and by Weatherson
(\citeproc{ref-Weatherson2012}{2012}). It does rely on intuitions, but
they are intuitions about choices (such as that Chika should choose the
blue ticket), not about knowledge directly.

Some discussions of IRI, especially that in Hawthorne and Stanley
(\citeproc{ref-HawthorneStanley2008}{2008}) use a converse principle.
Again following the naming convention suggested by Jessica Brown
(\citeproc{ref-Brown2013}{2014}), we'll call this K-Nec.

\begin{description}
\tightlist
\item[K-Nec]
An agent can properly use \emph{p} as a reason for action only if she
knows that \emph{p}.
\end{description}

I'll mostly set the discussion of K-Nec aside here, since my preferred
argument for IRI, the argument from Chika's case, merely relies on
K-Suff. But it is interesting to work through how K-Nec helps plug a gap
in the argument by cases for IRI.

Buckwalter and Schaffer (\citeproc{ref-BuckwalterSchaffer2015}{2015})
argue that the intuitions behind Pinillos's examples are not as solid as
we might like. It's true that experimental subjects do say that Bojan
has to check the paper more times than Ankita does before he knows that
the paper contains no typos. But those subjects also say he has to check
more times before he believes that the paper has no typos. And,
surprisingly, they say that he has to check more time before he guesses
the paper has no typos. They suggest that there might be
interest-relativity in the modal `has' as much as in the verb `knows'.
To say someone `has' to X before they Y, typically means that it is
improper, in some way, to Y without doing X first. That won't be a
problem for the proponent of IRI as long as at least in some of the
cases Pinillos studies, the relevant senses of propriety are connected
to knowledge. And that's plausible for belief; Bojan has to know the
paper is typo free before he (properly) believes it. At least, that's a
plausible move given K-Nec.\footnote{I'm suggesting here that in some
  sense, knowledge is a norm of belief. For more on the normative role
  of knowledge, see Worsnip (\citeproc{ref-Worsnip2016}{2017}).}

There is one other problem for argument from cases for IRI. Imagine that
after two checks of the paper, we tell Bojan that Ankita's paper is a
duplicate of hers, and she has checked her paper in just the same way he
has checked his. And we tell him that Ankita does not overly care
whether her paper is typo-free, but is confident that it is. We then ask
him, does Ankita know her paper is typo free? Many philosophers think
Bojan should answer ``No'' here. And that isn't something IRI can
explain. According to IRI, he should say, ``I don't know.'' He can't say
Ankita does know, since he doesn't know their common paper has no typos.
But it's hard to see why he should deny knowledge. Keith DeRose
(\citeproc{ref-DeRose2009}{2009, 185}) thinks this case is particularly
hard for IRI to explain, while Brian Kim (\citeproc{ref-Kim2015}{2016})
offers some possible explanations. This objection doesn't tell against
the claim that knowledge is interest-relative, but it does threaten the
invariantism. An interest-relative contextualist should say that
everyone should deny Bojan knows his paper is typo free, and Bojan
should deny Ankita knows her paper is typo-free.

\section{Odds and Stakes}\label{oddsandstakes}

Interest-relative invariantism says that the interests of the subject
matter to what she knows. This is a fairly vague statement though; there
are a number of ways to make it precise. Right now I have interests in
practical questions (such as whether I should keep writing or go to
lunch) and in theoretical questions (such as whether IRI is true). Do
both kinds of interests matter? We'll return to that question in the
next section. For now we want to ask a prior question: when do practical
interests matter for whether a subject knows that \emph{p}? There are
two main answers to this question in the literature.

\begin{description}
\tightlist
\item[Stakes]
When the agent has a possible bet on \emph{p} that involves large
potential losses, it is harder to know that \emph{p}.
\item[Odds]
When the agent has a possible bet on \emph{p} that involves long odds,
it is harder to know that \emph{p}.
\end{description}

The difference between these two options becomes clear in a simple class
of cases. Assume the agent is faced with a choice with the following
structure:

\begin{itemize}
\tightlist
\item
  There is a safe option, with payout \emph{S}.
\item
  And there is a risky option, with good payout \emph{G} if \emph{p} is
  true, and bad payout \emph{B} if \emph{p} is false.
\end{itemize}

These choices need not involve anything like a `bet', in the ordinary
folk sense. But they are situations where the agent has to make a choice
between a path where the payouts are \emph{p}-dependent, and one where
they are independent of \emph{p}. And those are quite common situations.

The \textbf{Stakes} option says that the relevant number here is the
magnitude \emph{S}-\emph{B}. If that is large, then the agent is in a
high-stakes situation, and knowledge is hard. If it is low, then the
agent is in a low stakes situation, and knowledge is relatively easy.
(Perhaps the magnitude of \emph{G}-\emph{S} is relevant as well, though
the focus in the literature has been on examples where \emph{S}-\emph{B}
is high.)

The \textbf{Odds} option says that the relevant number is is the ratio:

\[
\frac{S-B}{G-S}
\]

If that number is high, the agent faces a long odds bet, and knowledge
is hard. If that number is low, the agent faces a short odds bet, and
knowledge is relatively easy.

If our motivation for IRI came from cases, then it is natural to believe
\textbf{Stakes}. Both Bojan and Chika face bets on \emph{p} at long
odds, but intuition is more worried about whether Bojan knows that
\emph{p} than whether Chika does. (At least my intuition is worried
about whether Bojan knows, and I've seen little evidence that Chika's
case is intuitively a case of non-knowledge.)

But if our motivation for IRI came from principles, then it is natural
to believe \textbf{Odds}. One way to think of the argument from
principles for IRI is that it is a way to make all four of the following
intuitive claims true:

\begin{enumerate}
\def\labelenumi{\arabic{enumi}.}
\tightlist
\item
  Agents should maximise evidential expected utility; i.e., they should
  choose the option whose expected utility is highest if the utilities
  are the agent's own, and the probabilities are the evidential
  probabilities given the agent's evidence.
\item
  If an agent knows that \emph{p}, they can ignore possibilities where
  \emph{p} is false; i.e., they can make whatever choice is the rational
  choice given \emph{p}.
\item
  Chika cannot ignore possibilities where the Red Sox lost; she should
  consider those possibilities because it is in virtue of them that the
  evidential expected utility of taking the red ticket is higher.
\item
  Agents with Chika's evidence, background and dispositions typically
  know that the Red Sox won.
\end{enumerate}

The first three principles imply that Chika does not know the Red Sox
won. The only way to square that with the anti-sceptical fourth
principle is to say that Chika is in some way atypical. And the only way
she has been said to be atypical is in the practical choices she faces.
But note it is not because she faces a high-stakes choice: precisely one
dollar is at stake. It is because she faces a long (indeed infinitely
long) odds bet.

In the general case we discussed above, agents maximise expected utility
by taking the risky choice iff:

\[
\frac{S-B}{G-S} < \frac{Pr(p)}{1-Pr(p)}
\]

where \emph{Pr}(\emph{p}) is the probability of \emph{p} given the
agent's evidence. The actual magnitudes at play don't matter to what
choice maximses expected utility, just the odds the agent faces. So if
one's motivation to keep IRI is to square expected utility maxmisation
with natural principles about knowledge and action, it seems the
relevant feature of practical situations should be the stakes agents
face.

Why could it seem stakes matter then? I think it is because in high
stakes situations, the odds an agent faces are typically long ones. It
is much easier to lose large amounts of utility than to gain large
amounts of utility. Bojan stands to lose a lot from a typo in his paper;
he doesn't stand to lose much by taking the time to check it over. So a
high stakes situation will, at least typically, be a long odds
situation. So if we say the odds the agent faces are relevant to what
they know, we can explain any intuition that the stakes at play are
relevant.

Jessica Brown (\citeproc{ref-Brown2008}{2008, 176}) also notes that
cases where the agent faces long odds but low stakes raise problems for
the stakes-based version of IRI.

\section{What Kind of Interests?}\label{whatkindofinterests}

Let's return to the question of whether theoretical interests are
relevant to knowledge, or only practical interests. There is some
precedent for the more restrictive answer. Stanley's book on IRI is
called \emph{Knowledge and Practical Interests}. And he defends a theory
on which what an agent knows depends on the practical questions they
face. But there are strong reasons to think that theoretical reasons
matter as well.

In the previous section, I suggested that agents know that \emph{p} only
if they would maximise expected utility by choosing the choice that
would be rational given \emph{p}. That is, agents know that \emph{p}
only if the answer to the question ``What choice maximises expected
utility?'' is the same unconditionally as it is conditional on \emph{p}.
My preferred version of interest-relative invariantism generalises this
approach. An agent knows that \emph{p} only if the rational answer to a
question she faces is the same unconditionally as it is conditional on
\emph{p}. What it is for an agent to face a question is dependent on the
agent's interests. If that's how one thinks of IRI, the question of this
section becomes, should we restrict questions the agent faces to just
being questions about what choice to make? Or should they include
questions that turn on her thoeretical interests, but which are
irrelevant to choices before her. There are two primary motivations for
allowing theoretical interests as well as practical interests to matter.

The first comes from the arguments for what Jeremy Fantl and Matthew
McGrath call the Unity Thesis (\citeproc{ref-FantlMcGrath2009}{Fantl and
McGrath 2009, 73--76}). They are interested in the thesis that whether
or not \emph{p} is a reason for an agent is independent of whether the
agent is engaged in practical or theoretical deliberation. But we don't
have to be so invested in the ideology of reasons to appreciate their
argument. Note that if only practical interests matter, then the agent
should come up with different answers to the question ``What to do in
situation \emph{S}'' depending on whether the agent is actually in
\emph{S}, or they are merely musing about how one would deal with that
situation. And it is unintuitive that this should matter.

Let's make that a little less abstract. Imagine Chika is not actually
faced with the choice between the red and blue tickets. In fact, she has
no practical decision to make that turns on whether the Red Sox won. But
she is idly musing over what she would do if she were offered the red
ticket and the blue ticket. If she knows the Red Sox won, then she
should be indifferent between the tickets. After all, she knows they
will both return \$1. But intuitively she should think the red ticket is
preferable, even in the abstract setting. And this seems to be the
totally general case.

The general lesson is that if whether one can take \emph{p} for granted
is relevant to the choice between A and B, it is similarly relevant to
the theoretical question of whether one would choose A or B, given a
choice. And since those questions should receive the same answer, if
\emph{p} can't be known while making the practical deliberation between
A and B, it can't be known while musing on whether A or B is more
choiceworthy.

In Weatherson (\citeproc{ref-Weatherson2012}{2012}) I suggest another
reason for including theoretical interests in what's relevant to
knowledge. There is something odd about the following reasoning: The
probability of \emph{p is precisely x}, therefore \emph{p}, in any case
where \emph{x}~\textless~1. It is a little hard to say, though, why this
is problematic, since we often take ourselves to know things on what we
would admit, if pushed, are purely probabilistic grounds. The version of
IRI that includes theoretical interests allows for this. If we are
consciously thinking about whether the probability of \emph{p} is
\emph{x}, then that's a relevant question to us. Conditional on
\emph{p}, the answer to that question is clearly no, since conditional
on \emph{p}, the probability of \emph{p} is 1. So anyone who is thinking
about the precise probability of \emph{p}, and not thinking it is 1, is
not in a position to know \emph{p}. And that's why it is wrong, when
thinking about \emph{p}'s probability, to infer \emph{p} from its high
probability.

Putting the ideas so far together, we get the following picture of how
interests matter. An agent knows that \emph{p} only if the evidential
probability of \emph{p} is close enough to certainty for all the
purposes that are relevant, given the agent's theoretical and practical
interests. Assuming the background theory of knowledge is non-sceptical,
this will entail that interests matter.

\section{Global or Partial}\label{globalorpartial}

So far I've described three ways to refine the defence of IRI.

\begin{enumerate}
\def\labelenumi{\arabic{enumi}.}
\tightlist
\item
  The motivation could come from cases or principles.
\item
  The relevant feature that makes it hard to have knowledge could be
  that the agent faces a high-stakes choice, or a long-odds choice.
\item
  Only practical interests may be relevant to knowledge, or theoretical
  interests may matter as well.
\end{enumerate}

For better or worse, the version of IRI I've defended has fairly clear
commitments on all three; in each case, I prefer the latter option. From
here on, I'm much less sure of the right way to refine IRI.

IRI, like contextualism, was introduced as a thesis about knowledge. But
it need not be restricted that way. It could be generalised to a number
of other epistemically interesting notion. At the extreme, we could
argue that every epistemologially interesting notion is
interest-relative. Doing so gives us a global version of IRI.

Jason Stanley (\citeproc{ref-Stanley2005}{2005}) comes close to
defending a global version. He notes that if one has both IRI, and a
`knowledge first' epistemology
~(\citeproc{ref-Williamson2000}{Williamson 2000}), then one is a long
way to towards globalism. Even if one doesn't accept the whole knowledge
first package, but just accepts the thesis that evidence is all and only
what one knows, then one is a long way towards globalism. After all, if
evidence is interest-relative, then probability, justification,
rationality, and evidential support are interest-relative too.

Katherine Rubin (\citeproc{ref-Rubin2015}{2015}) objects to globalist
versions of IRI. But the objections she gives turn, as she notes, on
taking stakes not odds to be relevant.

If a non-global version of IRI could be made to work, it would have some
theoretical advantages. It's nice to be able to say that Chika should
take the blue ticket because the evidential probability of the Red Sox
winning is lower than the evidential probability of two plus two being
four. But that won't be a non-circular explanation if we also say that
something is part of Chika's evidence in virtue of being known.

On the other hand, the motivations for interest-relativity of knowledge
seem to generalise to all other non-gradable states. In ordinary cases,
Chika could use the fact that the Red Sox won as a given in practical or
theoretical reasoning. That is, she could properly treat it as evidence.
But she can't treat it as evidence when deciding which ticket to take.
So at least what she can properly treat as evidence seems to be
interest-relative, and from there it isn't obvious how to deny that
evidence itself is interest-relative too.

There remains a question of whether gradable notions, like epistemic
probabilities, are also interest-relative. One of the aims of my first
paper on IRI ~(\citeproc{ref-Weatherson2005}{Weatherson 2005}) was to
argue that probabilistic notions are interest-invariant while binary
notions are interest-relative. But if propositions that are part of
one's evidence have maximal probability (in the relevant sense of
probability), and evidence is interest-relative, that combination won't
be sustainable.

In short, while the non-global version of IRI allows for some nice
reductive explanations of why interests matter, the global version is
supported by the very intuitions that motivated IRI. There is a danger
here that whatever way the IRI theorist goes, they will run into
insuperable difficulties. Ichikawa, Jarvis, and Rubin
(\citeproc{ref-IchikawaEtAl2012}{2012}) argue strongly that this danger
is real; there is no plausible way to fill out IRI. I'm not convinced
that the prospects are quite so grim, but I think this is one of the
more pressing worries for IRI.

\section{Belief, Justification and
Interest}\label{beliefjustificationandinterest}

If we decide that not everything in epistemology is interest-relative,
then we face a series of questions about which things are, and are not,
interest relative. One of these concerns belief. Should we say that what
an agent believes is sensitive to what her interests are?

Note that the question here concerns whether belief is constitutively
related to interests. It is extremely plausible that belief is causally
related to interests. As Jennifer Nagel (\citeproc{ref-Nagel2008}{2008})
has shown, many agents will react to being in a high-stakes situation by
lowering their confidence in relevant propositions. In this way, being
in a high-stakes situation may cause an agent to lose beliefs. This is
not the kind of constitutive interest-relativity that's at issue here,
though the fact this happens makes it harder to tell whether there is
such a thing as constitutive interest-relativity of belief.

I find it useful to distinguish three classes of views about beliefs and
interests.

\begin{enumerate}
\def\labelenumi{\arabic{enumi}.}
\tightlist
\item
  Beliefs are not interest-relative. If knowledge is interest-relative,
  the interest-relativity is in the conditions a belief must satisfy in
  order to count as knowledge.
\item
  Beliefs are interest-relative, and the interest-relativity of belief
  fully explains why knowledge is interest-relative.
\item
  Beliefs are interest-relative, but the interest-relativity of belief
  does not fully explain why knowledge is interest-relative.
\end{enumerate}

In Weatherson (\citeproc{ref-Weatherson2005}{2005}), I suggested an
argument for option 2. I now think that argument fails, for reasons
given by Jason Stanley (\citeproc{ref-Stanley2005}{2005}). I originally
thought option 2 provided the best explanation of cases like Chika's.
Assume Chika does the rational thing, and takes the blue ticket. She
believes it is better to take the blue ticket. But that would be
incoherent if she believed the Red Sox won. So she doesn't believe the
Red Sox won. But she did believe the Red Sox won before she was offered
the bet, and she hasn't received any new evidence that they did not. So,
assuming we can understand an interest-invariant notion of confidence,
she is no less confident that the Red Sox won, but she no longer
believes it. That's because belief is interest-relative. And if all
cases of interest-relativity are like Chika's, then they will all be
cases where the interest-relativity of belief is what is ultimately
explanatory.

The problem, as Stanley had in effect already pointed out, is that not
all cases are like Chika's. If agents are mistaken about the choice they
face, the explanation I offered for Chika's case won't go through. This
is especially clear in cases where the mistake is due to irrationality.
Let's look at an example of this. Assume Dian faces the same choice as
Chika, and this is clear, but he irrationally believes that the red
ticket pays out \$2. So he prefers the red ticket to the blue ticket,
and there is no reason to deny he believes the Red Sox won. Yet taking
the red ticket is irrational; he wouldn't do it were he rational. Yet it
would be rational if he knew the Red Sox won. So Dian doesn't know the
Red Sox won, in virtue of his interests, while believing they did.

Note this isn't an argument for option 1. Everything I said about Dian
is consistent with the Chika-based argument for thinking that belief is
interest-relative. It's just that there are cases where the
interest-relativity of knowledge can't be explained by the
interest-relativity of belief. So I now think option 3 is correct.

We can ask similiar questions about whether justified belief is
interest-relative, and whether if so this explains the
interest-relativity of knowledge. I won't go into as much detail here,
save to note that on my preferred version of IRI, Dian's belief that the
Red Sox won is both justified and rational. (Roughly, this is because I
think his belief that the Red Sox won just is his high credence that the
Red Sox won, and his high credence the Red Sox won is justified and
rational. I defend this picture at more length in {[}Weatherson2005;{]}.
And while that paper makes some mistaken suggestions about knowledge, I
still think what it says about belief and justification is broadly
correct.) That is, Dian has a justified true belief that the Red Sox
won, but does not know it. This is, to put it mildly, not the most
intuitive of verdicts. I suspect the alternative verdicts lead to worse
problems elsewhere. But rather than delving deeper into the details of
IRI to confirm whether that's true, let's turn to some objections to the
view.

\section{Debunking Objections}\label{debunkingobjections}

Many arguments against IRI are, in effect, debunking arguments. The
objector's immediate conclusion is not that IRI is false, but that it is
unsupported by the arguments given for it.

Arguments that people do not have the intuition that, for exaple, Bojan
lacks knowledge that his paper is typo-free, do not immediately show
thtat IRI is false. That's because the truth of IRI can be made
compatible with that intuition in two ways. For one thing, it is
possible that people think Bojan knows because they think Bojan betting
that his paper is typo free is, in the circumstances, a good
bet.\footnote{Compare the response to Feltz and Zarpentine
  (\citeproc{ref-FeltzZarpentine2010}{2010}) that I make in Weatherson
  (\citeproc{ref-Weatherson2011-WEADIR}{2011, sec. 5}), or the response
  to Lackey (\citeproc{ref-Lackey2010}{2010}) by Masashi Kasaki
  (\citeproc{ref-Kasaki2014}{2014, sec. 5}).} For another thing,
intuitions around here might be unreliable. Remember that one of the
original motivations for IRI was that it was the lowest cost solution to
the preface paradox and lottery paradox. We shouldn't expect intuitions
to be reliable in the presence of serious paradox. That consideration
cuts both ways; it makes debunking objections to arguments for IRI from
intuitions about cases look very promising. And I think those objections
are promising; but they don't show IRI is false.

Similarly, objections to the premises of the argument from principles
don't strictly entail that IRI is false. After all, IRI is an
existential thesis; it says sometimes interests matter. The principles
used to defend it are universal claims; they say (for example) it is
always permissible to act on knowledge. Weaker versions of these
principles might still be consistent with, or even supporting of, IRI.
But this feels a little desperate. If the premises of these arguments
fail, then IRI looks implausible.

But there are still two methodological points worth remembering.
Sometimes it seems that critics of principles like K-Suff reason that
K-Suff entails IRI, and IRI is antecedently implausible, so we should
start out suspicious of K-Suff. Now why might IRI be antecedently
implausible?

I think to some extent it is because it is thought to be so
revolutionary. The denial of interest-relativity is often taken to be a
``traditional'' view. This phrasing appears, for example, in Boyd
(\citeproc{ref-Boyd2015}{2016}), and in Ichikawa, Jarvis, and Rubin
(\citeproc{ref-IchikawaEtAl2012}{2012}), and even in the title of
Buckwalter (\citeproc{ref-Buckwalter2014}{2014}). And if this were
correct, that would be a mark against interest-relativity. The
``inherited experience and acumen of many generations of men''
~(\citeproc{ref-Austin1956}{Austin 1956, 11}) should not be lightly
forsaken. The problem is that it isn't true that IRI is revolutionary.
Indeed, in historical terms there is nothing particularly novel about
contemporary IRI. As Stephen R. Grimm (\citeproc{ref-Grimm2015}{2015})
points out, you can see a version of the view in Locke, and in Clifford.
What's really radical, as Descartes acknowledged, is to think the
perspective of the Cartesian meditator is the right one for
epistemology.

Perhaps what is unintuitive about IRI is that it makes knowledge depend
on factors that are not `truth-directed', or `truth-conducive'. There is
a stronger and weaker version of the principle that might be being
appealed to here. The stronger version is that IRI makes practical
matters into one of the factors on which knowledge depends, and this is
implausible. But IRI doesn't do this. It is consistent with IRI to say
that only truth-conducive features of beliefs are relevant to whether
they amount to knowledge, but how much of each feature one needs depends
on practical matters. The weaker principle is that IRI makes knowledge
counterfactually sensitive to features irrelevant to the truth,
justification or reliability of the belief. This is true, but it isn't
an objection to IRI. Any theory that allows defeaters to knowledge, and
defeaters to those defeaters, will make knowledge counterfactually
sensitive to non-truth-conducive features in just the same way. And it
is independently plausible that there are defeaters to knowledge, and
they can be defeated.\footnote{The argument of the last two sentences is
  expanded on greatly in Weatherson
  (\citeproc{ref-Weatherson2014-ProbScept}{2014, sec. 3}). The idea that
  knowledge allows for defeaters is criticised by Maria Lasonen-Aarnio
  (\citeproc{ref-Lasonen-Aarnio2014a}{2014b}). Eaton and Pickavance
  (\citeproc{ref-EatonPickavance2015}{2015}) make an objection to IRI
  that does not take this point into account.}

These are all reasons to think that IRI is not antecedently implausible.
There is one reason to think it is antecedently plausible. On a
functionalist theory of mind, belief is a practical notion. And it is
plausible that knowledge is a kind of success condition for belief. Now
it's possible to have non-practical success conditions for a state our
concept of which is practical. But I don't find that a natural starting
assumption. It's mucn more intuitive, to me at least, that the norms of
belief and the metaphysics of belief would be tightly integrated. And
that suggests that IRI is, if anything, a natural default.

That's not an argument for IRI, or of course for K-Suff. And there are
important direct objections to K-Suff. Jessica Brown
(\citeproc{ref-Brown2008}{2008}) and Jennifer Lackey
(\citeproc{ref-Lackey2010}{2010}) have examples of people in high stakes
situations who they say are intuitively described as knowing something,
but not being in a position to act on it. I'm sympathetic to the
two-part reply that Masashi Kasaki (\citeproc{ref-Kasaki2014}{2014})
makes to these examples. The first thing to note is that these are hard
cases, in areas where several paradoxes (e.g., lottery, preface,
sceptical) are lurking. Intuitions are less reliable than usual around
here. But another thing to notice is that it is very hard to say what
actions are justified by taking \emph{p} for granted in various
settings. Brown and Lackey both describe cases where doctors have lots
of evidence for \emph{p}, and given \emph{p} a certain action would
maximise patient-welfare, but where intuitively it would be wrong for
the doctor to act that way. As it stands, that's a problem for IRI only
if doctors should maximise epistemic expected patient-welfare, and that
principle isn't true. Kasaki argues that there isn't a way to fill out
Lackey's example to get around this problem, and I suspect the same is
true for Brown's example.

Finally, note that K-Suff is an extensional claim. Kenneth Boyd
(\citeproc{ref-Boyd2015}{2016}) and Baron Reed
(\citeproc{ref-Reed2014}{2014}) object to a principle much stronger than
K-Suff: the principle that what an agent knows should explain why some
choices are rational for them. Both of them say that if IRI is
inconsistent with the stronger principle, that is a serious problem for
IRI. (In Boyd's case this is part of an argument that IRI is
unmotivated; in Reed's case he takes it to be a direct objection to
IRI.) Now I think IRI is inconsistent with this principle. Chika doesn't
know the Red Sox won because she can't rationally choose the red ticket,
not the other way around. But I don't see why the principle is so
plausible. It seems plausible to me that something else (e.g., evidence)
explains both rational choice and knowledge, and the way it explains
both things makes IRI true.

\section{Direct Objections}\label{directobjections}

Let's close with direct arguments against IRI. There are two kinds of
arguments that I won't address here. One of these is the argument,
developed in Ichikawa, Jarvis, and Rubin
(\citeproc{ref-IchikawaEtAl2012}{2012}) that there isn't a good way to
say how far interest-relativity should extend. As I noted above, I agree
this is a deep problem, and don't think there is a good answer to it in
the existing literature. The other kind are objections that only apply
to the Stakes version of IRI, not the Odds version. One instance of this
kind is the Dutch Book argument deployed by Baron Reed
(\citeproc{ref-Reed2014}{2014}). I think several instances of that kind
of argument are successful. But the theory they succeed against is not
IRI, but a sub-optimal version of IRI. So I'll stick to objections that
apply to the Odds version.

IRI does allow knowledge to depend on some unexpected factors. But so do
most contemporary theories of knowledge. Most contemporary theories
allow for knowledge to be defeated in certain ways, such as by available
but unaccessed evidence ~(\citeproc{ref-Harman1973}{Harman 1973, 75}),
or by nearby possibilities of error ~(\citeproc{ref-Goldman1976}{Goldman
1976}), or by mistakes in the background reasoning. The last category of
cases aren't really contemporary; they trace back at least to
Dharmottara ~(\citeproc{ref-Nagel2014}{Nagel 2014, 58}). And
contemporary theories of knowledge also allow for defeaters to be
defeated. Once we work through the details of what can defeat a
defeater, it turns out many surprising things can affect knowledge.

Indeed, for just about any kind of defeater, it is possible to imagine
something that in some ways makes the agent's epistemic position worse,
while simultaneously defeating the defeater.\footnote{The argument of
  the last two sentences is expanded on greatly in Weatherson
  (\citeproc{ref-Weatherson2014-ProbScept}{2014, sec. 3}), where it is
  credited to Martin Smith. The idea that knowledge allows for defeaters
  is criticised by Maria Lasonen-Aarnio
  (\citeproc{ref-Lasonen-Aarnio2014}{2014a}).} If interests matter to
knowledge because they matter to defeaters, as is true on my version of
IRI, we should expect strange events to correlate with gaining
knowledge. For example, it isn't surprising that one can gain knowledge
that \emph{p} at exactly the moment one's evidential support for
\emph{p} falls. This consequence of IRI is taken to be obviously
unacceptable by Eaton and Pickavance
(\citeproc{ref-EatonPickavance2015}{2015}), but it's just a consequence
of how defeaters generally work.

IRI has been criticised for making knowledge depend on agents not
allowing agents to get knowledge by not caring, as in these vivid
quotes:

\begin{quote}
Not giving a damn, however enviable in other respects, should not be
knowledge-making. ~(\citeproc{ref-RussellDoris2008}{Russell and Doris
2009, 433})
\end{quote}

\begin{quote}
If you don't now whether penguins eat fish, but want to know, you might
think \ldots{} you have to gather evidence. {[}But if IRI{]} were
correct, though, you have another option: You could take a drink or
shoot heroin. ~(\citeproc{ref-CappelenLepore2006}{Cappelen and Lepore
2006, 1044--45})
\end{quote}

Let's walk through Cappelen and Lepore's case. IRI says that there are
people who both have high confidence that penguins eat fish, and they
have this confidence for reasons that are appropriately connected to the
fact that penguins eat fish. But one of them really worries about
sceptical doubts, and so won't regard the question of what penguins eat
as settled. The other brushes off excessive sceptical doubts, and
rightly so; they are, after all, excessive. IRI says that the latter
knows and the former does not. If the former were to care a little less,
in particular if they cared a little less about evil demons and the
like, they'd know. Perhaps they could get themselves to care a little
less by having a drink. That doesn't sound like a bad plan; if a
sceptical doubt is destroying knowledge, and there is no gain from
holding on to it, then just let it go. From this perspective, Cappelen
and Lepore's conclusion does not seem like a reductio. Excessive doubt
can destroy knowledge, so people with strong, non-misleading evidence
can gain knowledge by setting aside doubts. And drink can set aside
doubt. So drink can lead to knowledge.\footnote{Wright
  (\citeproc{ref-Wright2004}{2004}) notes that there often is not value
  in holding on to sceptical doubts, and the considerations of this
  paragraph are somewhat inspired by his views. That's not to endorse
  the idea that using alcohol or heroin is preferable to being gripped
  by sceptical doubts, especially heroin, but I do endorse the general
  idea that those doubts are not cost-free.}

But note that the drink doesn't generate the knowledge. It blocks, or
defeats, something that threatens to block knowledge. We should say the
same thing to Russell and Doris's objection. Not giving a damn, about
scepticism for example, is not knowledge-making, but it is
knowledge-causing. In general, things that cause by double prevention do
not make things happen, although later things are counterfactually
dependent on them ~(\citeproc{ref-Lewis2004a}{Lewis 2004}). And the same
is true of not caring.

Finally, it has been argued that IRI makes knowledge unstable in a
certain kind of way ~(\citeproc{ref-Lutz2014}{Lutz 2014};
\citeproc{ref-Anderson2015}{Anderson 2015}). Practical circumstances can
change quickly; something can become a live choice and cease being one
at a moment's notice. If knowledge is sensitive to what choices are
live, then knowledge can change this quickly too. But, say the
objectors, it is counterintuitive that knowledge changes this quickly.

Now I'm not sure this is counterintuitive. I think that part of what it
takes to know \emph{p} is to treat the question of whether \emph{p} as
closed. It sounds incoherent to say, ``I know a is the F, but the
question of who is the F is still ope''. And whether a question is
treated as open or closed does, I think, change quite rapidly. One can
treat a question as closed, get some new reason to open it (perhaps new
evidence, perhaps an interlocutor who treats it as open), and then
quickly dismiss that reason. So I'm not sure this is even a problem.

But to the extent that it is, it is only a problem for a somewhat
half-hearted version of IRI. The puzzles the objectors raise turn on
cases where the relevant practical options change quickly. But even once
a practical option has ceased to be available, it can be hard in
practice to dismiss it from one's mind. One may often still think about
what to do if it becomes available again, or about exactly how
unfortunate it is that the option went away. As long as theoretical as
well as practical interests matter to knowledge, it will be unlikely
that knowledge will be unstable in just this way. Practical interests
may change quickly; theoretical ones typically do not.

\subsection*{References}\label{references}
\addcontentsline{toc}{subsection}{References}

\phantomsection\label{refs}
\begin{CSLReferences}{1}{0}
\bibitem[\citeproctext]{ref-Anderson2015}
Anderson, Charity. 2015. {``On the Intimate Relationship of Knowledge
and Action.''} \emph{Episteme} 12 (3): 343--53. doi:
\href{https://doi.org/10.1017/epi.2015.16}{10.1017/epi.2015.16}.

\bibitem[\citeproctext]{ref-Austin1956}
Austin, J. L. 1956. {``A Plea for Excuses.''} \emph{Proceedings of the
Aristotelian Society} 57 (1): 1--30. doi:
\href{https://doi.org/10.1093/aristotelian/57.1.1}{10.1093/aristotelian/57.1.1}.

\bibitem[\citeproctext]{ref-Boyd2015}
Boyd, Kenneth. 2016. {``Pragmatic Encroachment and Epistemically
Responsible Action.''} \emph{Synthese} 193 (9): 2721--45. doi:
\href{https://doi.org/10.1007/s11229-015-0878-y}{10.1007/s11229-015-0878-y}.

\bibitem[\citeproctext]{ref-Brown2008}
Brown, Jessica. 2008. {``Subject-Sensitive Invariantism and the
Knowledge Norm for Practical Reasoning.''} \emph{No{û}s} 42 (2):
167--89. doi:
\href{https://doi.org/10.1111/j.1468-0068.2008.00677.x}{10.1111/j.1468-0068.2008.00677.x}.

\bibitem[\citeproctext]{ref-Brown2013}
---------. 2014. {``Impurism, Practical Reasoning and the Threshold
Problem.''} \emph{No{û}s} 48 (1): 179--92. doi:
\href{https://doi.org/10.1111/nous.12008}{10.1111/nous.12008}.

\bibitem[\citeproctext]{ref-Buckwalter2014}
Buckwalter, Wesley. 2014. {``Non-Traditional Factors in Judgments about
Knowledge.''} \emph{Philosophy Compass} 7 (4): 278--89. doi:
\href{https://doi.org/10.1111/j.1747-9991.2011.00466.x}{10.1111/j.1747-9991.2011.00466.x}.

\bibitem[\citeproctext]{ref-BuckwalterSchaffer2015}
Buckwalter, Wesley, and Jonathan Schaffer. 2015. {``Knowledge, Stakes
and Mistakes.''} \emph{No{û}s} 49 (2): 201--34. doi:
\href{https://doi.org/10.1111/nous.12017}{10.1111/nous.12017}.

\bibitem[\citeproctext]{ref-CappelenLepore2006}
Cappelen, Herman, and Ernest Lepore. 2006. {``Shared Content.''} In
\emph{The Oxford Handbook of Philosophy of Language}, edited by Ernest
Lepore and Barry C. Smith, 1020--55. Oxford: Oxford University Press.

\bibitem[\citeproctext]{ref-DeRose1992}
DeRose, Keith. 1992. {``Contextualism and Knowledge Attributions.''}
\emph{Philosophy and Phenomenological Research} 52 (4): 913--29. doi:
\href{https://doi.org/10.2307/2107917}{10.2307/2107917}.

\bibitem[\citeproctext]{ref-DeRose2009}
---------. 2009. \emph{The Case for Contextualism: Knowledge, Skepticism
and Context}. Oxford: Oxford.

\bibitem[\citeproctext]{ref-EatonPickavance2015}
Eaton, Daniel, and Timothy Pickavance. 2015. {``Evidence Against
Pragmatic Encroachment.''} \emph{Philosophical Studies} 172: 3135--43.
doi:
\href{https://doi.org/10.1007/s11098-015-0461-x}{10.1007/s11098-015-0461-x}.

\bibitem[\citeproctext]{ref-FantlMcGrath2002}
Fantl, Jeremy, and Matthew McGrath. 2002. {``Evidence, Pragmatics, and
Justification.''} \emph{Philosophical Review} 111 (1): 67--94. doi:
\href{https://doi.org/10.2307/3182570}{10.2307/3182570}.

\bibitem[\citeproctext]{ref-FantlMcGrath2009}
---------. 2009. \emph{Knowledge in an Uncertain World}. Oxford: Oxford
University Press.

\bibitem[\citeproctext]{ref-FeltzZarpentine2010}
Feltz, Adam, and Chris Zarpentine. 2010. {``Do You Know More When It
Matters Less?''} \emph{Philosophical Psychology} 23 (5): 683--706. doi:
\href{https://doi.org/10.1080/09515089.2010.514572}{10.1080/09515089.2010.514572}.

\bibitem[\citeproctext]{ref-Goldman1976}
Goldman, Alvin I. 1976. {``Discrimination and Perceptual Knowledge.''}
\emph{The Journal of Philosophy} 73 (20): 771--91. doi:
\href{https://doi.org/10.2307/2025679}{10.2307/2025679}.

\bibitem[\citeproctext]{ref-Grimm2015}
Grimm, Stephen R. 2015. {``Knowledge, Practical Interests and Rising
Tides.''} In \emph{Epistemic Evaluation: Purposeful Epistemology},
edited by David K. Henderson and John Greco, 117--37. Oxford: Oxford
University Press.

\bibitem[\citeproctext]{ref-Harman1973}
Harman, Gilbert. 1973. \emph{Thought}. Princeton: Princeton University
Press.

\bibitem[\citeproctext]{ref-Hawthorne2004}
Hawthorne, John. 2004. \emph{Knowledge and Lotteries}. Oxford: Oxford
University Press.

\bibitem[\citeproctext]{ref-HawthorneStanley2008}
Hawthorne, John, and Jason Stanley. 2008. {``{Knowledge and Action}.''}
\emph{Journal of Philosophy} 105 (10): 571--90. doi:
\href{https://doi.org/10.5840/jphil20081051022}{10.5840/jphil20081051022}.

\bibitem[\citeproctext]{ref-IchikawaEtAl2012}
Ichikawa, Jonathan Jenkins, Benjamin Jarvis, and Katherine Rubin. 2012.
{``Pragmatic Encroachment and Belief-Desire Psychology.''}
\emph{Analytic Philosophy} 53 (4): 327--43. doi:
\href{https://doi.org/10.1111/j.2153-960X.2012.00564.x}{10.1111/j.2153-960X.2012.00564.x}.

\bibitem[\citeproctext]{ref-Kasaki2014}
Kasaki, Masashi. 2014. {``Subject-Sensitive Invariantism and Isolated
Secondhand Knowledge.''} \emph{Acta Analytica} 29: 83--98. doi:
\href{https://doi.org/10.1007/s12136-013-0215-3}{10.1007/s12136-013-0215-3}.

\bibitem[\citeproctext]{ref-Kim2015}
Kim, Brian. 2016. {``In Defense of Subject-Sensitive Invariantism.''}
\emph{Episteme} 13 (2): 233--51. doi:
\href{https://doi.org/10.1017/epi.2015.40}{10.1017/epi.2015.40}.

\bibitem[\citeproctext]{ref-Lackey2010}
Lackey, Jennifer. 2010. {``Acting on Knowledge.''} \emph{Philosophical
Perspectives} 24: 361--82. doi:
\href{https://doi.org/10.1111/j.1520-8583.2010.00196.x}{10.1111/j.1520-8583.2010.00196.x}.

\bibitem[\citeproctext]{ref-Lasonen-Aarnio2014}
Lasonen-Aarnio, Maria. 2014a. {``Higher-Order Evidence and the Limits of
Defeat.''} \emph{Philosophy and Phenomenological Research} 88 (2):
314--45. doi:
\href{https://doi.org/10.1111/phpr.12090}{10.1111/phpr.12090}.

\bibitem[\citeproctext]{ref-Lasonen-Aarnio2014a}
---------. 2014b. {``The Dogmatism Puzzle.''} \emph{Australasian Journal
of Philosophy} 92 (3): 417--32. doi:
\href{https://doi.org/10.1080/00048402.2013.834949}{10.1080/00048402.2013.834949}.

\bibitem[\citeproctext]{ref-Lewis2004a}
Lewis, David. 2004. {``Causation as Influence.''} In \emph{Causation and
Counterfactuals}, edited by John Collins, Ned Hall, and L. A. Paul,
75--106. Cambridge: {MIT} Press.

\bibitem[\citeproctext]{ref-Lutz2014}
Lutz, Matt. 2014. {``The Pragmatics of Pragmatic Encroachment.''}
\emph{Synthese} 191 (8): 1717--40. doi:
\href{https://doi.org/10.1007/s11229-013-0361-6}{10.1007/s11229-013-0361-6}.

\bibitem[\citeproctext]{ref-Nagel2008}
Nagel, Jennifer. 2008. {``Knowledge Ascriptions and the Psychological
Consequences of Changing Stakes.''} \emph{Australasian Journal of
Philosophy} 86 (2): 279--94. doi:
\href{https://doi.org/10.1080/00048400801886397}{10.1080/00048400801886397}.

\bibitem[\citeproctext]{ref-Nagel2014}
---------. 2014. \emph{Knowledge: A Very Short Introduction}. Oxford:
Oxford University Press.

\bibitem[\citeproctext]{ref-Pinillos2012}
Pinillos, Ángel. 2012. {``Knowledge, Experiments and Practical
Interests.''} In \emph{Knowledge Ascriptions}, edited by Jessica Brown
and Mikkel Gerken, 192--219. Oxford: Oxford University Press.

\bibitem[\citeproctext]{ref-Reed2014}
Reed, Baron. 2014. {``Practical Matters Do Not Affect Whether You
Know.''} In \emph{Contemporary Debates in Epistemology}, edited by
Matthias Steup, John Turri, and Ernest Sosa, 2nd ed., 95--106.
Chicester: Wiley-Blackwell.

\bibitem[\citeproctext]{ref-Rubin2015}
Rubin, Katherine. 2015. {``Total Pragmatic Encroachment and Epistemic
Permissiveness.''} \emph{Pacific Philosophical Quarterly} 96: 12--38.
doi: \href{https://doi.org/10.1111/papq.12060}{10.1111/papq.12060}.

\bibitem[\citeproctext]{ref-RussellDoris2008}
Russell, Gillian, and John M. Doris. 2009. {``Knowledge by
Indifference.''} \emph{Australasian Journal of Philosophy} 86 (3):
429--37. doi:
\href{https://doi.org/10.1080/00048400802001996}{10.1080/00048400802001996}.

\bibitem[\citeproctext]{ref-Rysiew2016}
Rysiew, Patrick. 2017. {``Warranted Assertability Maneuvers.''} In
\emph{Routledge Handbook of Epistemic Contextualism}, edited by Jonathan
Jenkins Ichikawa, n/a--a. London: Routledge.

\bibitem[\citeproctext]{ref-Stanley2005}
Stanley, Jason. 2005. \emph{{Knowledge and Practical Interests}}. Oxford
University Press.

\bibitem[\citeproctext]{ref-Weatherson2005}
Weatherson, Brian. 2005. {``{Can We Do Without Pragmatic
Encroachment?}''} \emph{Philosophical Perspectives} 19 (1): 417--43.
doi:
\href{https://doi.org/10.1111/j.1520-8583.2005.00068.x}{10.1111/j.1520-8583.2005.00068.x}.

\bibitem[\citeproctext]{ref-Weatherson2011-WEADIR}
---------. 2011. {``Defending Interest-Relative Invariantism.''}
\emph{Logos \& Episteme} 2 (4): 591--609. doi:
\href{https://doi.org/10.5840/logos-episteme2011248}{10.5840/logos-episteme2011248}.

\bibitem[\citeproctext]{ref-Weatherson2012}
---------. 2012. {``Knowledge, Bets and Interests.''} In \emph{Knowledge
Ascriptions}, edited by Jessica Brown and Mikkel Gerken, 75--103.
Oxford: Oxford University Press.

\bibitem[\citeproctext]{ref-Weatherson2014-ProbScept}
---------. 2014. {``Probability and Scepticism.''} In \emph{Scepticism
and Perceptual Justification}, edited by Dylan Dodd and Elia Zardini,
71--86. Oxford: Oxford University Press.

\bibitem[\citeproctext]{ref-Williamson2000}
Williamson, Timothy. 2000. \emph{{Knowledge and its Limits}}. Oxford
University Press.

\bibitem[\citeproctext]{ref-Worsnip2016}
Worsnip, Alex. 2017. {``Knowledge Norms.''} In \emph{Routledge Handbook
of Epistemic Contextualism}, edited by Jonathan Jenkins Ichikawa,
n/a--a. London: Routledge.

\bibitem[\citeproctext]{ref-Wright2004}
Wright, Crispin. 2004. {``Warrant for Nothing (and Foundations for
Free)?''} \emph{Proceedings of the Aristotelian Society, Supplementary
Volume} 78 (1): 167--212. doi:
\href{https://doi.org/10.1111/j.0309-7013.2004.00121.x}{10.1111/j.0309-7013.2004.00121.x}.

\end{CSLReferences}



\noindent Published in\emph{
Routledge Handbook of Epistemic Contextualism}, 2017, pp. 240-253.

\end{document}
