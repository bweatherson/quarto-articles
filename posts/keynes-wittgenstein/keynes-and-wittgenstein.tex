% Options for packages loaded elsewhere
\PassOptionsToPackage{unicode}{hyperref}
\PassOptionsToPackage{hyphens}{url}
%
\documentclass[
  10pt,
  letterpaper,
  DIV=11,
  numbers=noendperiod,
  twoside]{scrartcl}

\usepackage{amsmath,amssymb}
\usepackage{setspace}
\usepackage{iftex}
\ifPDFTeX
  \usepackage[T1]{fontenc}
  \usepackage[utf8]{inputenc}
  \usepackage{textcomp} % provide euro and other symbols
\else % if luatex or xetex
  \usepackage{unicode-math}
  \defaultfontfeatures{Scale=MatchLowercase}
  \defaultfontfeatures[\rmfamily]{Ligatures=TeX,Scale=1}
\fi
\usepackage{lmodern}
\ifPDFTeX\else  
    % xetex/luatex font selection
    \setmainfont[ItalicFont=EB Garamond Italic,BoldFont=EB Garamond
Bold]{EB Garamond Math}
    \setsansfont[]{Europa-Bold}
  \setmathfont[]{Garamond-Math}
\fi
% Use upquote if available, for straight quotes in verbatim environments
\IfFileExists{upquote.sty}{\usepackage{upquote}}{}
\IfFileExists{microtype.sty}{% use microtype if available
  \usepackage[]{microtype}
  \UseMicrotypeSet[protrusion]{basicmath} % disable protrusion for tt fonts
}{}
\usepackage{xcolor}
\usepackage[left=1in, right=1in, top=0.8in, bottom=0.8in,
paperheight=9.5in, paperwidth=6.5in, includemp=TRUE, marginparwidth=0in,
marginparsep=0in]{geometry}
\setlength{\emergencystretch}{3em} % prevent overfull lines
\setcounter{secnumdepth}{3}
% Make \paragraph and \subparagraph free-standing
\makeatletter
\ifx\paragraph\undefined\else
  \let\oldparagraph\paragraph
  \renewcommand{\paragraph}{
    \@ifstar
      \xxxParagraphStar
      \xxxParagraphNoStar
  }
  \newcommand{\xxxParagraphStar}[1]{\oldparagraph*{#1}\mbox{}}
  \newcommand{\xxxParagraphNoStar}[1]{\oldparagraph{#1}\mbox{}}
\fi
\ifx\subparagraph\undefined\else
  \let\oldsubparagraph\subparagraph
  \renewcommand{\subparagraph}{
    \@ifstar
      \xxxSubParagraphStar
      \xxxSubParagraphNoStar
  }
  \newcommand{\xxxSubParagraphStar}[1]{\oldsubparagraph*{#1}\mbox{}}
  \newcommand{\xxxSubParagraphNoStar}[1]{\oldsubparagraph{#1}\mbox{}}
\fi
\makeatother


\providecommand{\tightlist}{%
  \setlength{\itemsep}{0pt}\setlength{\parskip}{0pt}}\usepackage{longtable,booktabs,array}
\usepackage{calc} % for calculating minipage widths
% Correct order of tables after \paragraph or \subparagraph
\usepackage{etoolbox}
\makeatletter
\patchcmd\longtable{\par}{\if@noskipsec\mbox{}\fi\par}{}{}
\makeatother
% Allow footnotes in longtable head/foot
\IfFileExists{footnotehyper.sty}{\usepackage{footnotehyper}}{\usepackage{footnote}}
\makesavenoteenv{longtable}
\usepackage{graphicx}
\makeatletter
\newsavebox\pandoc@box
\newcommand*\pandocbounded[1]{% scales image to fit in text height/width
  \sbox\pandoc@box{#1}%
  \Gscale@div\@tempa{\textheight}{\dimexpr\ht\pandoc@box+\dp\pandoc@box\relax}%
  \Gscale@div\@tempb{\linewidth}{\wd\pandoc@box}%
  \ifdim\@tempb\p@<\@tempa\p@\let\@tempa\@tempb\fi% select the smaller of both
  \ifdim\@tempa\p@<\p@\scalebox{\@tempa}{\usebox\pandoc@box}%
  \else\usebox{\pandoc@box}%
  \fi%
}
% Set default figure placement to htbp
\def\fps@figure{htbp}
\makeatother
% definitions for citeproc citations
\NewDocumentCommand\citeproctext{}{}
\NewDocumentCommand\citeproc{mm}{%
  \begingroup\def\citeproctext{#2}\cite{#1}\endgroup}
\makeatletter
 % allow citations to break across lines
 \let\@cite@ofmt\@firstofone
 % avoid brackets around text for \cite:
 \def\@biblabel#1{}
 \def\@cite#1#2{{#1\if@tempswa , #2\fi}}
\makeatother
\newlength{\cslhangindent}
\setlength{\cslhangindent}{1.5em}
\newlength{\csllabelwidth}
\setlength{\csllabelwidth}{3em}
\newenvironment{CSLReferences}[2] % #1 hanging-indent, #2 entry-spacing
 {\begin{list}{}{%
  \setlength{\itemindent}{0pt}
  \setlength{\leftmargin}{0pt}
  \setlength{\parsep}{0pt}
  % turn on hanging indent if param 1 is 1
  \ifodd #1
   \setlength{\leftmargin}{\cslhangindent}
   \setlength{\itemindent}{-1\cslhangindent}
  \fi
  % set entry spacing
  \setlength{\itemsep}{#2\baselineskip}}}
 {\end{list}}
\usepackage{calc}
\newcommand{\CSLBlock}[1]{\hfill\break\parbox[t]{\linewidth}{\strut\ignorespaces#1\strut}}
\newcommand{\CSLLeftMargin}[1]{\parbox[t]{\csllabelwidth}{\strut#1\strut}}
\newcommand{\CSLRightInline}[1]{\parbox[t]{\linewidth - \csllabelwidth}{\strut#1\strut}}
\newcommand{\CSLIndent}[1]{\hspace{\cslhangindent}#1}

\setlength\heavyrulewidth{0ex}
\setlength\lightrulewidth{0ex}
\usepackage[automark]{scrlayer-scrpage}
\clearpairofpagestyles
\cehead{
  Brian Weatherson
  }
\cohead{
  Keynes and Wittgenstein
  }
\ohead{\bfseries \pagemark}
\cfoot{}
\makeatletter
\newcommand*\NoIndentAfterEnv[1]{%
  \AfterEndEnvironment{#1}{\par\@afterindentfalse\@afterheading}}
\makeatother
\NoIndentAfterEnv{itemize}
\NoIndentAfterEnv{enumerate}
\NoIndentAfterEnv{description}
\NoIndentAfterEnv{quote}
\NoIndentAfterEnv{equation}
\NoIndentAfterEnv{longtable}
\NoIndentAfterEnv{abstract}
\renewenvironment{abstract}
 {\vspace{-1.25cm}
 \quotation\small\noindent\rule{\linewidth}{.5pt}\par\smallskip
 \noindent }
 {\par\noindent\rule{\linewidth}{.5pt}\endquotation}
\KOMAoption{captions}{tableheading}
\makeatletter
\@ifpackageloaded{caption}{}{\usepackage{caption}}
\AtBeginDocument{%
\ifdefined\contentsname
  \renewcommand*\contentsname{Table of contents}
\else
  \newcommand\contentsname{Table of contents}
\fi
\ifdefined\listfigurename
  \renewcommand*\listfigurename{List of Figures}
\else
  \newcommand\listfigurename{List of Figures}
\fi
\ifdefined\listtablename
  \renewcommand*\listtablename{List of Tables}
\else
  \newcommand\listtablename{List of Tables}
\fi
\ifdefined\figurename
  \renewcommand*\figurename{Figure}
\else
  \newcommand\figurename{Figure}
\fi
\ifdefined\tablename
  \renewcommand*\tablename{Table}
\else
  \newcommand\tablename{Table}
\fi
}
\@ifpackageloaded{float}{}{\usepackage{float}}
\floatstyle{ruled}
\@ifundefined{c@chapter}{\newfloat{codelisting}{h}{lop}}{\newfloat{codelisting}{h}{lop}[chapter]}
\floatname{codelisting}{Listing}
\newcommand*\listoflistings{\listof{codelisting}{List of Listings}}
\makeatother
\makeatletter
\makeatother
\makeatletter
\@ifpackageloaded{caption}{}{\usepackage{caption}}
\@ifpackageloaded{subcaption}{}{\usepackage{subcaption}}
\makeatother

\usepackage{bookmark}

\IfFileExists{xurl.sty}{\usepackage{xurl}}{} % add URL line breaks if available
\urlstyle{same} % disable monospaced font for URLs
\hypersetup{
  pdftitle={Keynes and Wittgenstein},
  pdfauthor={Brian Weatherson},
  hidelinks,
  pdfcreator={LaTeX via pandoc}}


\title{Keynes and Wittgenstein}
\author{Brian Weatherson}
\date{2001}

\begin{document}
\maketitle
\begin{abstract}
Three recent books have argued that Keynes's philosophy, like
Wittgenstein's, underwent a radical foundational shift. It is argued
that Keynes, like Wittgenstein, moved from an atomic Cartesian
individualism to a more conventionalist, intersubjective philosophy. It
is sometimes argued this was caused by Wittgenstein's concurrent
conversion. Further, it is argued that recognising this shift is
important for understanding Keynes's later economics. In this paper I
argue that the evidence adduced for these theses is insubstantial, and
other available evidence contradicts their claims.
\end{abstract}


\setstretch{1.1}
\section{Introduction}\label{introduction}

Three recent books (\citeproc{ref-Davis1994}{Davis 1994};
\citeproc{ref-Bateman1996}{Bateman 1996};
\citeproc{ref-Coates1996}{Coates 1996}) have argued that the philosophy
behind Keynes's later economics (in particular the \emph{General
Theory}) is closer to Wittgenstein's post Tractarian theorising than to
his early philosophy as expressed in his \emph{Treatise on
Probability}.\footnote{Davis's views are also set out in his
  (\citeproc{ref-Davis1995}{1995}), and Coates's to some extent in his
  (\citeproc{ref-Coates1997}{1997}), but I will focus on the more
  detailed position in their respective books.} If Keynes did follow
Wittgenstein in the ways suggested it would represent a substantial
change from his early neoplatonist epistemology. In this paper I argue
that the evidence for this thesis is insubstantial, and the best
explanation of the evidence is that Keynes's philosophical views
remained substantially unchanged.

There are three reasons for being interested in this question. The first
is that it is worthwhile getting the views of a thinker as important as
Keynes right. The second is that it would be mildly unfortunate for
those of us attracted to Keynes's epistemology to find out that it was
eventually rejected by its creator\footnote{In the way that, for
  example, subjective Bayesianism was arguably invented by and
  eventually rejected by Ramsey. See his Ramsey
  (\citeproc{ref-RamseyTruthProb}{1926}) and Ramsey
  (\citeproc{ref-Ramsey1929}{1929/1990}).}. Most importantly, all
parties agree that Keynes thought his philosophical theories had
substantial consequences for economic theory. It is a little unusual for
philosophical theories to have practical consequences; if one is claimed
to it is worthwhile identifying and evaluating the claim.

Section 2 examines Bateman's claim that Keynes abandoned the foundations
of his early theory of probability. Bateman's arguments turn, it seems,
on an equivocation between different meanings of `Platonism'. On some
interpretations the arguments are sound but don't show what Bateman
wants, on all others they are unsound. Section 3 looks at the
conventionalist, intersubjective theory of probability Bateman and Davis
claim Keynes adopted after abandoning his early objective theory. As
they express it the theory's coherence is dubious; I show how it might
be made more plausible. Nevertheless, there is little to show that
Keynes adopted it. The only time he talks about conventions is in the
context of speculative markets and in these contexts a conventionalist
theory will give the same results as an objectivist theory.

Section 4 looks at Coates's quite different arguments for an influence
from Wittgenstein to Keynes. Part of the problem with Coates's argument
is that the textual evidence he presents is capable of several readings;
indeed competing interpretations of the pages he uses exist. A bigger
problem is that even when he has shown a change in Keynes's views
occurred, he immediately infers the change was at the foundations of
Keynes's beliefs. Section 5 notes one rather important point of
Wittgenstein's of which Keynes seemed to take no notice, leading to an
error in the \emph{General Theory}. This should cast doubt on the claim
that Keynes's later philosophy, indeed later economics, was based on
theories of Wittgenstein.

\section{Bateman's Case for Change}\label{batemans-case-for-change}

A brief biographical sketch of Keynes is in order to frame the following
discussions, though I expect most readers are familiar with the broad
outlines\footnote{For more details see Skidelsky
  (\citeproc{ref-Skidelsky1983}{1983},
  \citeproc{ref-Skidelsky1992}{1992}) or Moggridge
  (\citeproc{ref-Moggridge1992}{1992}).}. Keynes arrived as an
undergraduate at Cambridge in 1902 and was based there for the rest of
his life. For the next six years he largely studied philosophy under the
influence of Moore and Russell. In 1907 he (unsuccessfully) submitted
his theory of probability as a fellowship dissertation; this was
successfully resubmitted the following year. His plans to make a book of
this were interrupted by work on Indian finance, the war and its
aftermath. It appeared as \emph{Treatise on Probability} (hereafter,
\emph{TP}) in (\citeproc{ref-Keynes1921}{1921}), after substantial work
on it in 1920. Modern subjectivist theories of probability, generally
known as Bayesian theories, first appeared in critical reviews of this
book (e.g. Borel (\citeproc{ref-Borel1924}{1924}), Ramsey
(\citeproc{ref-RamseyTruthProb}{1926})). After leaving philosophy for
many years, Wittgenstein returned to Cambridge in 1929, and subsequently
had many discussions with Keynes. In Keynes's \emph{General Theory}
(hereafter, \emph{GT}) of (\citeproc{ref-Keynes1936}{1936}) and in some
of the ensuing debate, Keynes referred to some distinctive elements of
the \emph{TP}, leading some interpreters to suspect that there was a
theoretical link between his early philosophy and his later economics.

There are two distinctive elements of Keynes's early theory of
probability for our purposes. The first is its objectivism. Keynes held
the probability of \emph{p} given \emph{h} is the degree of
\emph{reasonable} belief in \emph{p} on evidence \emph{h}, or, as Carnap
(\citeproc{ref-Carnap1950}{1950}) put it, the degree of confirmation of
\emph{p} by \emph{h}. These degrees are determined by logic; Keynes held
that there was a partial entailment relation between \emph{p} and
\emph{h}, of which the ordinary entailment relation (then thought to
have been given its best exposition by Russell and Whitehead) was just a
limiting case. And these relations are Platonic entities, we discover
what they are by perceiving them through our powers of intuition. The
second element is that the degrees may be non-numerical. So if the
probability of \emph{p} given \emph{h} is α, we may be able to say α
\textgreater{} 0.3, and α \textless{} 0.5, but not be able to give any
finer numerical limits. As a corollary, there are now two dimensions of
confirmatory support. Keynes claimed that as well as determining the
probability of \emph{p} given \emph{h}, we could determine the `weight'
of this probability, where weight measures how much evidence we have.
The more evidence is in \emph{h}, the greater the weight. Keynes thought
the distinction between saying that on evidence \emph{h}, \emph{p} has a
low probability, and saying that the weight of that probability is low
is important for understanding investment behaviour (\emph{GT}: Ch.~12).

Bateman and Davis both claim that Keynes gave up this theory for an
intersubjective theory in the \emph{GT}. I'll focus on Bateman's book,
largely because the structure of his argument is more
straightforward\footnote{All page references in sections 2 and 3 (unless
  otherwise stated) to Bateman 1996. Space considerations preclude a
  detailed examination of Davis's arguments, which are quite different
  to Bateman's. However his conclusions are subject to the same
  criticisms I make of Bateman's in section 3, and of Coates's in
  section 5.}. Bateman sets himself to offer another solution to `das
Maynard Keynes problem', which he describes as follows.

\begin{quote}
``{[}Future theorists{]} will read \emph{Treatise on Probability}'s
account of the objective nature of probabilities and the way that
rational people employ them, and they will wonder at how this person
could have turned around 15 years later and written a book {[}the
\emph{GT}{]} in which irrational people who base their decisions on
social conventions cause mass unemployment in the capitalist system''
(7)
\end{quote}

I doubt this is the right thing to say about the \emph{GT}, but that's
another story. For now we might simply note that there's no obvious
conflict here. For one thing, if the people in the \emph{TP} are
rational, and in the \emph{GT} are irrational, as Bateman allows, it's
not too surprising they behave differently. More generally, it's to be
expected (sadly) that normative and descriptive theories are different,
and by Bateman's lights that should explain the difference between the
outlook of the explicitly normative \emph{TP} and the at least partially
descriptive \emph{GT}. If the agents in the \emph{GT} are irrational,
that book cannot but be a purely normative account of rationality. On
the other hand, if the \emph{TP} were taken to be descriptive and not
just normative, if it claimed that people really conform to its
epistemological exhortations, there could be a conflict. I can't
imagine, however, what the evidence or motivation for that reading could
be.

If there were a conflict between the \emph{GT} and \emph{TP}, there
ought be a greater one between the `rational people' of the \emph{TP}
and the blatantly irrational leaders in \emph{Economic Consequences of
the Peace} (\emph{ECP}). These books were published about 15 months
apart, not 15 years. And the most memorable parts of \emph{ECP} are the
descriptions of the mental failings of President Wilson, who Lloyd
George could `bamboozle' into believing it was just to crush Germany
completely, but not `de-bamboozle' out of this view when it became
necessary. Or maybe we should say there's a conflict because the
characters in David Hume's histories do not meet his ethical or
epistemological norms.

If we give up Bateman's claim that the actors in the \emph{GT} are
irrational, and substitute the claim that the norms of rationality in
the two books differ, then we have a real conflict. And the most
charitable interpretation of Bateman is that this is the conflict he
intends to discuss. At the bottom of page 12 he goes close to saying
exactly this, but then proceeds to support his position with evidence
that Keynes changed his position on how rational people actually are.
Once we are claiming the change of view is with regard to norms,
evidence of opinion changes about empirical questions becomes
irrelevant. This does mean much of Bateman's case goes, though not yet
all of it.

The main problem with Bateman's argument is that it rests on an
equivocation over the use of the term `Platonism'. In \emph{TP} Keynes
held that probability relations are objective, non-natural and part of
logic. I'll use `logical' for the last property. When Bateman says
Keynes believed probability relations were Platonic entities, he is
alternately referring to each of these properties. He seems to
explicitly use `Platonic' to mean `objective' on page 30, `non-natural'
on page 131, and `logical' on page 123. But this isn't the important
equivocation.

Say a theory about some entities is a `Strong Platonist' theory if it
concords with all Keynes's early beliefs: those entities are objective,
non-natural and logical. Bateman wants to conclude that by the time of
the \emph{GT}, Keynes no longer had an objectivist theory of
probability. But showing he no longer held a Strong Platonist view won't
get that conclusion, because there are 3 interesting objectivist
positions which are not Strong Platonist. The following names are my
own, but they should be helpful.

\begin{quote}
\textbf{Carnapian}\\
Probability relations are objective, natural and logical. This is what
Carnap held in his (\citeproc{ref-Carnap1950}{1950}).

\textbf{Gödelism}\\
Probability relations are objective, non-natural and non-logical. Gödel
held this view about numbers, hence the name. I'd normally call this
position Platonism, but that name's under dispute. Indeed I suspect this
is what Keynes means by Platonism in \emph{My Early Beliefs}.
(\citeproc{ref-KeynesMEB}{Keynes 1938b})

\textbf{Reductionism}\\
Probability relations are objective, natural and non-logical. Such
positions don't have to reduce probability to something else, but they
usually will. Russell held such a position in his
(\citeproc{ref-Russell1948}{1948}).
\end{quote}

These categories could apply to other entities, like numbers or moral
properties or colours, but we will be focussing on probability relations
here. Say a theory is `Weak Platonist' if it is Strong Platonist or one
of these three types. The most interesting equivocation in Bateman is
using `Platonist' to refer to either Strong or Weak Platonist positions.
He argues that Keynes gave up his early Platonist position. These
arguments are sound if he means Strong Platonist, unsound if he means
Weak Platonist. But if he means Strong Platonist he can't draw the extra
conclusion that Keynes gave up objectivism about probability relations,
which he does in fact draw. So I'll examine his arguments under the
assumption that he means to show Keynes gave up Weak Platonism.

Whatever Bateman means by Keynes's Platonism, he isn't very sympathetic
to it. It gets described as `obviously flawed' (4) and `fatally flawed'
(17), and is given as the reason for his work being ignored by `early
positivists and members of the Vienna Circle' (61). Given that the
\emph{TP} is cited extensively, and often approvingly, by Carnap in his
1950, this last claim is clearly false. Most stunningly, he claims
writers committed to the existence of Platonic entities cannot `be
considered to be a part of the analytic tradition' (39), though he does
concede in a footnote that some `early analytical philosophers' (he
gives Frege as an example) were Platonist. Bateman's paradigm of
philosophy seems to be the logical positivism of Ayer's \emph{Language,
Truth and Logic}: ``nowhere would one less expect to find metaphysics
than in modern analytical philosophy'' (\citeproc{ref-Ayer1936}{Ayer
1936, 39}).

There is an implicit argument in this derision. Keynes must, so the
argument goes, have given up (Weak) Platonism because no sensible person
could believe it. If anything like this were sound it should apply to
Weak Platonism about other entities. But the history of `modern
analytical philosophy' shows that Weak Platonism (though not under that
name) is quite widespread in metaphysical circles. Modern philosophy
includes believers in possible worlds both concrete and ersatz, in
universals and in numbers. All these positions would fall under Weak
Platonism. Even Quine's ontologically sparse \emph{Word and Object} was
Weak Platonist about classes, though he probably wouldn't like the
label. So by analogy Weak Platonism about probability relations isn't so
absurd as to assume Keynes must have seen its flaws.

Bateman's more important argument is direct quotation from Keynes. This
argument is undermined largely because of Bateman's somewhat selective
quotation. There are two sources where Keynes appears to recant some of
his early beliefs. Which early beliefs, and how early these beliefs
were, is up for debate. The two are his 1938 memoir \emph{My Early
Beliefs} (hereafter, \emph{MEB}), and his 1931 review of Ramsey's
posthumous \emph{Foundations of Mathematics}. \emph{MEB} wasn't
published until 1949, three years after Keynes's death, but according to
its introduction it is unchanged from the version Keynes gave as a talk
in 1938. In it he largely discusses the influence of Moore, and
particularly \emph{Principia Ethica}, on his beliefs before the first
world war.

There are several connections between Moore's work and Keynes. The most
pertinent here is that Keynes's metaphysics of probability in \emph{TP}
is borrowed almost completely from Moore's metaphysics of goodness. Not
only are probability relations objective and non-natural, they are
simple and unanalysable. These are all attributes Moore assigns to
goodness. The only addition Keynes makes is that his probability
relations are logical. So Moore's position on goodness is, in our
language, Gödelian.

As he says in \emph{MEB}, Keynes became convinced of Moore's metaethics,
though he differed with Moore over the implications this had for ethics
proper. In particular he disagreed with Moore's claim that individuals
are morally bound to conform to social norms. Bateman seems to assume
that at any time Keynes's metaphysics of goodness and probability will
be roughly the same, and with the exception of questions about their
logical status, this seems a safe enough assumption.

Bateman quotes Keynes saying that his, and his friends', belief in
Moore's metaethics was `a religion, some sort of relation of
neo-platonism' (\citeproc{ref-KeynesMEB}{Keynes 1938b, 438}). This is
part of the evidence that Keynes meant what I'm calling Gödelism by
`Platonism'. Not only does he use it to describe Moore's position, but
comparing Platonism with religion would be quite apt if he intends it to
involve a commitment to objective, non-natural entities. The important
point to note is that he is using `religion' to include his metaethics,
a point Bateman also makes, though it probably also includes some broad
ethical generalisations. Bateman then describes the following paragraph
as removing `any doubt that {[}Keynes{]} had thrown over his youthful
Platonism as untenable'. (40)

\begin{quote}
Thus we were brought up -- with Plato's absorption in the good in
itself, with a scholasticism which outdid St.~Thomas, in calvinistic
withdrawal from the pleasures and successes of Vanity Fair, and
oppressed with all the sorrows of Werther. It did not pervert us from
laughing most of the time and we enjoyed supreme self-confidence,
superiority and contempt towards all the rest of the unconverted world.
But it was hardly a state of mind which a grown-up person in his senses
could sustain literally. (\citeproc{ref-Keynes1938}{Keynes 1938a, 442}).
\end{quote}

As it stands, \emph{perhaps} the last sentence signals a change in
metaphysical beliefs, as opposed to say a change in the importance of
pleasure-seeking. In any case the following paragraph (which Bateman
neglects to quote) shows such an interpretation to be mistaken.

\begin{quote}
It seems to me looking back, that this religion of ours was a very good
one to grow up under. It remains nearer the truth than any other I know,
with less extraneous matter and nothing to be ashamed of \ldots{} It was
a purer, sweeter air than Freud cum Marx. It is still my religion under
the surface. (\citeproc{ref-Keynes1938}{Keynes 1938a, 442}).
\end{quote}

So was Keynes confessing to `a state of mind which a grown-up person in
his senses couldn't sustain literally'? No; his `religion' which he held
onto was a very broad, abstract doctrine. It needed supplementation with
a even general ethical view, to wit an affirmative answer to one of
Moore's `open questions'. And then it needed some bridging principles to
convert those ethics into moral conduct in the world as we find it. His
early position included all these, and it seems it was in effect his
early `bridging principles' he mocks in the above quote. These relied,
the memoir makes clear throughout, on an excessively optimistic view of
human nature, so he thought in effect that he could prevent wrong by
simply proving to its perpetrators that they were wrong. Now giving up
one's bridging principles doesn't entail abandonment of a general
ethical view, let alone one's metaethics. Indeed, let alone one's
metaphysics of probability! And as the last quote makes clear, Keynes
was quite content with the most general, most abstract parts of his
early belief. If this were all Bateman had to go on it wouldn't even
show Keynes had abandoned Strong Platonism\footnote{The above points are
  similar in all substantial respects to those made by O'Donnell
  (\citeproc{ref-ODonnell1991}{1991}) in response to an earlier version
  of Bateman's account.}.

There is more to Bateman's case. In Keynes's review\footnote{This is
  often mistakenly referred to as an obituary in the literature, e.g.
  (\citeproc{ref-Coates1996}{Coates 1996, 139}).} of Ramsey
(\citeproc{ref-RamseyFM}{1931}), he recanted on some of his theory of
probability. This is quite important to the debate, so I'll quote the
relevant section at some length.

\begin{quote}
Ramsey argues as against the view which I had put forward, that
probability is concerned not with objective relations between
propositions but (in some sense) with degrees of belief, and he succeeds
in showing that the calculus of probabilities simply amounts to a set of
rules for ensuring that the system of degrees of belief which we hold
shall be a \emph{consistent} system. Thus the calculus of probability
belongs to formal logic. But the basis of our degrees of belief -- or
the \emph{a priori} probabilities, as they used to be called -- is part
of our human outfit, perhaps given us merely by natural selection,
analogous to our perceptions and our memories rather than to formal
logic. So far I yield to Ramsey -- I think he is right. But in
attempting to distinguish `rational' degrees of belief from belief in
general he was not yet, I think, quite successful. It is not getting to
the bottom of the principle of induction to merely say it is a useful
mental habit. (\citeproc{ref-Keynes1931}{Keynes 1931, 338--39}).
\end{quote}

Tellingly, Bateman neglects to quote the final two sentences. I think
there is an ambiguity here, turning on the scope of the `so far' in the
fourth sentence. If it covers the whole section quoted, it does amount
to a wholesale recantation of Keynes's theory, and this is Bateman's
interpretation. But if we take the first sentence, or at least the first
clause, as being outside its scope it does not. And there are two
reasons for doing this. First, it seems inconsistent with Keynes's later
reliance on the \emph{TP} in parts of the \emph{GT}, as
(\citeproc{ref-ODonnell1989}{O'Donnell 1989} Ch. 6) has stressed.
Secondly, it is inconsistent with Keynes's complaint that on Ramsey's
view induction is merely a `useful habit'. If Keynes had become a
full-scale subjectivist, he ought have realised that patterns of
reasoning could only possibly be valid (if deductive) or useful
(otherwise). Since he still thought there must be something more, he
seems to believe an objectvist theory is correct, though by now he is
probably quite unsure as to its precise form. So in effect what Keynes
does in this paragraph is summarise Ramsey's view, list the details he
agrees with (that probability relations aren't logical), notes his
agreement with them, and then lists the details he disagrees with (that
probability relations aren't objective).

There is more evidence that all this quote represents is a recantation
of the view that probability relations are logical. Earlier in that
review he notes how little formal logic is now believed to achieve
compared with its promise at the start of the century.

\begin{quote}
The first impression conveyed by the work of Russell was that the field
of formal logic was enormously extended. The gradual perfection of the
formal treatment at the hands of himself, of Wittgenstein and of Ramsey
had been, however, gradually to empty it of content and to reduce it
more and more to mere dry bones, until finally it seemed to exclude not
only all experience, but most of the principles, usually reckoned
logical, of reasonable thought. (\citeproc{ref-Keynes1931}{Keynes 1931,
338}).
\end{quote}

More speculatively, I suggest Keynes's change of mind here (for this
shows he had surely given up the view that probability relations are
logical) might be influenced by Gödel's incompleteness theorem. In the
\emph{TP} Keynes had followed Russell is saying mathematics is part of
logic (\citeproc{ref-Keynes1921}{Keynes 1921, 293n}). That view was
often held to be threatened by Gödel's proof that there are mathematical
truths which can't be proven, and that the consistency of mathematics
can't be proven. But no one suggested this meant mathematics is merely
subjective, or that mathematical Platonism was therefore untenable. If
this response to Gödel is right, it shows there are objective standards
of reasoning (i.e.~mathematical standards) that are not part of logic.
This makes it less of a leap to say there are objective principles of
reasonable thought that are not `logical' in the narrow sense we've been
using.

So would Keynes have known of Gödel's theorem when he wrote this review?
I think it's possible, though some more research is needed. Keynes's
review was published in \emph{The New Statesman and Nation} on October
3, 1931. This was a weekly political and literary magazine of which
Keynes was chairman. So we can safely conclude the piece was drafted not
long before publication. Gödel's theorem was first announced at a
conference in Vienna in September 1930 (\citeproc{ref-Wang1987}{Wang
1987}), and was published in early 1931. While Keynes would certainly
have not read Gödel's paper, its content could easily have reached him
through Cambridge in that 12 month `window'. Since the explicit aim of
Gödel's paper was to show the incompleteness of \emph{Principia
Mathematica}, it would have immediately had some effect in Cambridge,
both in philosophy and mathematics. Given this evidence, the probability
Keynes knew of Gödel's theorem when he wrote the review of Ramsey still
mightn't be greater than one-half, but it mightn't be less than that
either.

In sum, I conclude that Keynes had given up his earlier belief that all
rules of reasonable belief are logical. This is what he yields to
Ramsey. This concession would be supported by the `drying up' of formal
logic that Keynes notes, perhaps most dramatically expressed in Gödel's
theorem. But he hadn't given up the belief that there are objective
rules which are extra-logical, and given the identification of
probability with degree of reasonable belief, he had no reason to reject
Gödelism or Reductionism about probability. Hence Bateman's argument
that he rejected objectivist theories of probability fails.

\section{Conventionalism}\label{conventionalism}

Bateman and Davis each argue that Keynes adopted a conventionalist,
intersubjectivist theory of probability. In Davis this is explicity
attributed to Wittgenstein's influence, however in Bateman it is less
clear what the source of this idea is. It isn't obvious what they mean
by an intersubjective theory. In particular, it isn't clear whether they
mean this to be an empirical or a normative theory; whether Keynes is
claiming that we ought set our degrees of belief by convention or that
we in general do. Since the empirical theory would be consistent with
his objectivist norms, and they stress the change in his views, I
conclude they are claiming this is a new normative view. According to
this view being reasonable is analysed as conforming to conventions.
This is not a very standard epistemological position, but something
similar is often endorsed in ethics. Bateman marshals the evidence that
Keynes moves from an objectivist to a conventionalist position in ethics
as evidence for this epistemological shift, but this doesn't seem of
overwhelming significance\footnote{If Keynes had adopted a framework
  which implied a tight connection between epistemological and ethical
  norms, such as a form of utilitarianism that stressed maximisation of
  expected utility, this would be important, since he couldn't change
  ethics and keep his epistemology. But such frameworks aren't
  compulsory, and given the vehemence with which Keynes denounced
  utilitarianism (\citeproc{ref-KeynesMEB}{Keynes 1938b, 445}) it seems
  he didn't adopt one.}.

Here's the closest Bateman gets to a definition of what he means by an
intersubjective theory of probability.

\begin{quote}
When probabilities are formed according to group norms, they are
referred to as intersubjective probabilities \ldots{} I take it to be
the case that in a world of subjective probabilities some individuals
will form their own estimates and others will form them on the basis of
group norms (50n).
\end{quote}

This makes it look very much like an empirical theory, as it refers to
how people actually form beliefs, not how they ought. So his
intersubjectivism looks perfectly consistent with Keynes's objectivism.
I am completely baffled by the `world of subjective probabilities'. I
wonder what such a world looks like, and how it compares to our world of
tables, chairs and stock markets?

Fortunately there is a theory that does the work Bateman needs. Ayer
(\citeproc{ref-Ayer1936}{1936}) rejects orthodox subjectivism about
probability on the grounds that it doesn't allow people to have mistaken
probabilistic beliefs. But he can't admit Keynesian probability
relations into his sparse ontology. The solution he adopts is to define
probability as degree of rational belief, but with this caveat.

\begin{quote}
Here we may repeat that the rationality of a belief is defined, not by
reference to any absolute standard, but by reference to part of our own
actual practice (\citeproc{ref-Ayer1936}{Ayer 1936, 101}).
\end{quote}

The `our' is a bit ambiguous; interpreting it to refer to the community
doesn't do violence to the text, though it is just as plausible that it
refers to a particular agent. The `part of our practice' referred to is
just our general rules for belief formation. These aren't justified by
an absolute standard; they are justified by the fact they are our rules,
and presumably by their generality. Given Bateman's views about
metaphysics, it seems quite reasonable to suppose he'd follow Ayer on
this point.

The evidence Keynes adopted such a position is usually taken to be some
passages from the \emph{GT} and the 1937 \emph{QJE} paper in which he
replied to some attacks on that book. Here's the key points from the two
quotes Bateman uses to support his view.

\begin{quote}
In practice we have agreed to fall back on what is, in truth, a
\emph{convention}. The essence of this convention -- though it does not,
of course, work out quite so simply -- lies in assuming that the
existing state of affairs will continue indefinitely, except in so far
as we have specific reasons for expecting a change (\emph{GT}: 152).
\end{quote}

\begin{quote}
How do we manage in such circumstances to behave in a manner which saves
out faces as rational, economic men? We have devised for the purposes a
variety of techniques, of which much the most important are the three
following: \ldots{}

(3) Knowing that our own individual judgement is worthless, we endeavour
to fall back on the judgement of the rest of the world which is perhaps
better informed. That is, we endeavour to conform with the behaviour of
the majority or the average. The psychology of a society of individuals
each of whom is endeavouring to copy the others leads to what we may
strictly term a \emph{conventional} judgement
(\citeproc{ref-Keynes1937}{Keynes 1937, 115}).
\end{quote}

There are two problems with using this evidence the way Bateman does.
The first is the old one that they seem expressly directed to empirical
questions, though perhaps appearances are deceptive here. The more
important one is that Keynes is attempting to answer a very specific
question with these passages; in ignorance of the question we can easily
misinterpret the answer.

How much ought one pay for a share in company X? Well, if one intends to
hold the share come what may, all that matters is the expected
prospective yield of X's shares, appropriately discounted, as compared
to the potential yield of that money in other uses. But as Keynes
repeatedly stresses (\emph{GT}: 149; (\citeproc{ref-Keynes1937}{Keynes
1937, 114})) we have no basis for forming such expectations. Were this
the only reason for investing then purely commercial investment may
never happen.

There is another motivation for investment, one that avoids this
problem. We might buy a share in X today on the hope that we will sell
it next week (or next month or perhaps next year) for more than we paid.
To judge whether such a purchase will be profitable, we need a theory
about how the price next week will be determined. Presumably those
buyers and sellers will be making much the same evaluations that we are.
That is, they'll be thinking about how much other people think X is
worth.

\begin{quote}
We have reached the third degree where we devote our intelligences to
anticipating what average opinion expects the average opinion to be. And
there are some, I believe, who practice the fourth, fifth and higher
degrees (\emph{GT}: 156).
\end{quote}

There is simply no solution to this except to fall back on convention.
That is, we are forced into a conventionalist theory of value, at least
of investment goods. But this doesn't mean that we have a
conventionalist epistemology. On the contrary, it means that our
ordinary (objectivist) empiricism is unimpeded. For the question that
Keynes has us solve by reference to convention is: What is the value of
X? This is equivalent to, what will be value of X be, or again, to what
are the conventional beliefs about X's value? We need to answer a
question about the state of conventions, and as good empiricists we
answer it by observing conventions.

An analogy may help here. Here's something that Hempel believed: to gain
rational beliefs about the colour of ravens, one has to look at some
birds. Did this mean he had an ornithological epistemology? No; he had
an empiricist epistemology which when applied to a question about ravens
issued the directive: Observe ravens! Similarly Keynes's belief that to
answer questions about value, i.e.~about conventions, one has to look at
conventions, does not imply a conventionalist epistemology. It just
means he has an empiricist epistemology which when applied to a question
about conventions issues the directive: Observe conventions!

There might be another motivation for using conventions, again
consistent with Keynes's objectivist empiricism. Sometimes we may have
not made enough observations, or may not have the mental power to
convert these to a theory. So we'll piggyback on someone else's
observations or mental powers. (This seems to be what's going on in the
quote from Keynes (\citeproc{ref-Keynes1937}{1937}).) Or even better,
we'll piggyback on everyone's work, the conventions. To see how this is
consistent with an objectivist epistemology (if it isn't already
obvious) consider another analogy.

What is the best way to work out the derivative of a certain function?
Unless your memory of high-school calculus is clear, the simplest
solution will be to consult an authority. Let's assume for the sake of
argument that the easiest authorities to consult are maths texts. It
seems like the rational thing to do is to act as if the method advanced
by the maths texts is the correct method. Does this mean that you have
adopted some kind of authoritarian metaphysics of mathematics, where
what it is for something to be correct is for it to be asserted by an
authority? Not at all. It is assumed that what the textbook says is
correct, but the authoritarian has to make the extra claim that the
answer is correct \emph{because} it is in the textbook. This is false;
that answer is in the textbook because it is correct. In sum, the
authoritarian gets the direction of fit wrong.

Similarly in the `piggyback' cases the intersubjectivist gets the
direction of fit wrong. We are accepting that \emph{p} has emerged as
`average opinion', then it is reasonable to believe \emph{p}. But we
aren't saying with the intersubjectivist it is reasonable to believe
\emph{p} because \emph{p} is average opinion; rather we are assuming
\emph{p} is average opinion because it is reasonable to believe
\emph{p}.

The evidence so far suggests Keynes's statements are consistent with his
denying intersubjectivism. We might be able to go further and show they
are inconsistent with his adopting that theory. After the quote on
\emph{GT} page 152 he spends the next page or so defending the use of
conventions here. The defence is, in part, that decisions made in accord
with conventions are reversible in the near future, so they won't lead
to great loss. If he really were an intersubjectivist, the use of
conventions would either not need defending, or could be defended by
general philosophical principles. Secondly, there is this quote which in
context seems inconsistent with adopting a conventionalist view.

\begin{quote}
For it is not sensible to pay 25 for an investment which you believe the
prospective yield to justify a value of 30, if you also believe that the
market will value it at 20 three months hence (\emph{GT}: 155).
\end{quote}

The context is that he is discussing why reasonable professional
investors base their valuations on convention rather than on long-term
expectation. Hence the `you' in the quote is assumed to be reasonable.
Hence it is reasonable, Keynes thinks, to believe that an investment's
prospective yield justifies a value of 30, and that conventional wisdom
is that its prospective yield is much lower. But if all reasonable
beliefs were formed by accordance with conventional wisdom, this would
be inconsistent. Hence Keynes cannot have adopted a conventionalist
epistemology.

\section{Keynes and Vagueness}\label{keynes-and-vagueness}

What a terrible state Keynes interpretation has got into! From the same
few pages (the opening of \emph{GT} Ch.~4) Coates
(\citeproc{ref-Coates1996}{1996}) reads into Keynes a preference for
basing theory on vague predicates, Bradford and Harcourt
(\citeproc{ref-BradfordHarcourt1997}{1997}) read Keynes as denying that
predicates which are unavoidably vague can be used in theory, and
O'Donnell (\citeproc{ref-ODonnell1997}{1997}) sees Keynes as holding a
position in between these.

Coates's theory is that Keynes abandoned the narrowly analytic
foundations of his early philosophy because of the problems of vagueness
that were pointed out to him by Wittgenstein. He has Keynes in 1936
adopting a middle way between analytic and Continental philosophy, which
gives up on analysis because of unavoidable vagueness, but which doesn't
follow Derrida in saying all that's left after analysis is `poetry'. He
also wants to argue for the philosophical importance of this theory. In
this essay I'll focus on his exegetical theories, though there are
concerns to be raised about his philosophy.

As in Bateman, analytic philosophy gets very narrowly defined in
Coates\footnote{All page references in this section (unless otherwise
  stated) to Coates (\citeproc{ref-Coates1996}{1996}).}. Here it
includes the claim that truth-value gaps are not allowed (xii). This
excludes from the canon some of the most important papers in analytical
philosophy of the last few decades (e.g. Dummett
(\citeproc{ref-Dummett1959}{1959}), Fraassen
(\citeproc{ref-vanFraassen1966}{1966}), Fine
(\citeproc{ref-Fine1975a}{1975}), Kripke
(\citeproc{ref-Kripke1975}{1975})), and hence must be a mistake. To use
one of Coates's favourite terms, `analytic philosophy' is a family
resemblance concept, not to be so narrowly cast. In particular, as we'll
see, analytic philosophers don't have to follow Frege in being nihilist
about vagueness.

Even more bizarrely, Coates defines empiricism so it includes both
psychologism in logic and utilitarianism in ethics (72-3). Since Ayer
(\citeproc{ref-Ayer1936}{1936}) opposes each of these doctrines, does
that makes Ayer an anti-empiricist? If Ayer is a paradigm empiricist (as
seems plausible) Keynes's rejection of psychologism and utilitarianism
can hardly count as proof of opposition to empiricism, as Coates wants
it to do. Apart from the fact that Mill believed all three, there is no
interesting connection between empiricism, psychologism and
utilitarianism.

Coates's story is that in the \emph{GT} Keynes allowed both his units
and his definitions to be quantitatively vague so as to follow natural
language. This constitutes a new `philosophy of social science' (85)
that is based on the ordinary language philosophy of the later
Wittgenstein. There are several problems with this story. The first is
that most of Coates's evidence comes from \emph{obiter dicta} in early
drafts of the \emph{GT}; by the time the book was finished most of these
suggestions are expunged. The second is that it's quite possible to
accept vagueness within a highly analytic philosophical framework. The
third is that the way Keynes uses vagueness is only consistent within
such a framework.

The first part of the story focuses on how Keynes derided his
predecessors for using concepts that were vague as if they were precise.
Coates adduces evidence to show Keynes in this context used `vague' as a
synonym for `quantitatively inexact'. The most important concept misused
by Keynes's predecessors in this way was the general price level. Of
course this was hardly a new point in the \emph{GT}; Keynes
(\citeproc{ref-Keynes1909}{1909}) says similar things. Coates claims
that Keynes's reaction to this misuse was to `criticise formal methods'
(83), and to conclude that `economic analysis can do without the ``mock
precision'' of formal methods' (85). This is all hard to square with
Keynes's explicit comments.

\begin{quote}
The well-known, but unavoidable, element of vagueness which admittedly
attends the concept of the general price-level makes this term very
unsatisfactory for the purposes of a causal analysis, which ought to be
exact (\emph{GT}:~39).
\end{quote}

Further, Keynes then defends his choice of units of quantity (quantity
of money-value and quantities of employment) on the grounds that they
are not quantitatively vague. Coates is surely right when he says that
Keynes's analysis of vagueness here is `not very controversial';
although it is perhaps misleading to say it is controversial at all.

The second, and central, part of the story focuses on how Keynes allowed
his definitions to be vague, but defended this on the grounds of
conformity to ordinary language. This `introduces what is distinctive
about his later philosophy of the social sciences' (85). The bulk of
Coates's evidence comes from Keynes's commentary on his own definitions;
usually this includes a claim that he has captured the ordinary usage of
the term. Since he uses `common usage' to explicitly mean `usage amongst
economists' (\emph{GT}: 79) the support these \emph{dicta} give to
Coates's theory might be minimal, but we'll ignore that complication.
The real problem is that this commentary extends to cases where he has
changed his mind over the best definition. For example, Coates quotes
Keynes writing in a draft of the \emph{GT} about the definition of
income.

\begin{quote}
But finally I have come to the conclusion that the use of language,
which is most convenient on a balance of considerations and involves the
least departure from current usage, is to call the actual sale proceeds
\emph{income} and the present value of the expected sale proceeds
\emph{effective demand} (\citeproc{ref-Keynes1934}{Keynes 1934, 425}).
\end{quote}

Coates comments:

\begin{quote}
By choosing definitions on the ground that they correspond with actual
usage Keynes was formulating an ordinary language social science, one
that bears a resemblance to those argued for by philosophers of
hermeneutics (90).
\end{quote}

He then goes on to note some comments from the \emph{GT} apparently
about this definition, and how it relates to common usage. The problem
is that this isn't the definition of income Keynes settles on in the
\emph{GT}. There he defines income of an agent as ``the excess of the
value of his finished output sold during the period over the prime
cost'' (\emph{GT}:~54), and \emph{net income} (which Coates fails to
distinguish) as income less supplementary cost. Given that at every
stage Keynes justified his current definitions by their (alleged)
conformity with common usage, even when he changed definitions, it is
hard to believe that these justifications are more than rhetorical
flourishes. After all, who will deny that \emph{ceteris paribus}
technical definitions should follow ordinary usage?

If Keynes's early choice of definitions showed an adherence to a
`philosophy of hermeneutics', perhaps his abandonment of those
definitions constitutes abandonment of that philosophy. One change
doesn't necessarily mean a change in foundations, so it is worth looking
at those foundations.

As I mentioned, allowing that vagueness exists doesn't mean abandoning
the Russellian program of giving a precise analysis of language. There
are two reasons for this. First, contra Wittgenstein it is possible to
analyse vague terms. Secondly, there are semantic programs very much in
the spirit of Russell which allow vagueness. I'll deal with these in
order.

In \emph{Philosophical Investigations}, Wittgenstein
(\citeproc{ref-Wittgenstein1953}{1953}) argued that the existence of
vagueness frustrated the program of analysis (ss.~60, 71). The argument
presumably is that analyses are precise, and hence they cannot
accurately capture vague terms. (See also his comments about the
impossibility of drawing the boundaries of `game' in s. 68.) This is a
simple philosophical mistake. We can easily give an analysis of a vague
term, we just have to make the analysans vague in exactly the same way
as the analysandum.

To see this in action, consider that paradigm of modern philosophy,
Lewis's analysis of subjunctive conditionals or counterfactuals. Lewis
(\citeproc{ref-Lewis1973a}{1973}) says that the conditional `If \emph{p}
were the case, it would be that \emph{q}' is true iff \emph{q} is true
in the most similar possible world in which \emph{p}. He considers the
objection that `most similar' is completely vague and imprecise.

\begin{quote}
Imprecise it may be; but that is all to the good. Counterfactuals are
imprecise too. Two imprecise concepts may be rigidly fastened to one
another, swaying together rather than separately, and we can hope to be
precise about their connection Lewis (\citeproc{ref-Lewis1973a}{1973}).
\end{quote}

Whatever the fate of Lewis's theory, his methodology seems
uncontestable. Wittgenstein's claim that analysis must be abandoned
because of vagueness is refuted by these observations of Lewis. Hence
Coates's claim that allowing vagueness (as Keynes does) means giving up
on analytic philosophy is mistaken.

The second problem with Coates's comments on vagueness is that he hasn't
allowed for what I'll call `orthodox' responses to vagueness. The aim of
the early analytics drifted between giving a precise model for natural
language, and replacing natural language with an artificial precise
language. The latter, claims Coates, ought be abandoned because of the
pragmatic virtues of a vague language. Let's agree to that; can the
spirit of the early aim of giving a precise analysis of language be
preserved?

Two approaches which seem to meet this requirement are the
supervaluational and epistemic theories of vagueness. The
supervaluationist says language can't be represented by a precise
classical model, but it can be represented by a set of such models. The
epistemic theorist says that there is a precise model of language, but
we cannot know what it is\footnote{See Williamson
  (\citeproc{ref-Williamson1994-WILV}{1994}) for the best epistemic
  account, Fine (\citeproc{ref-Fine1975a}{1975}) and Keefe
  (\citeproc{ref-Keefe2000}{2000}) for the best supervaluationist
  accounts.}. Call a theorist who adopts one of these approaches
`orthodox'. The name is chosen because supporters and critics of
orthodoxy agree that these positions represent attempts to minimise
deviations from the classical, Russellian program.

Clearly Keynes did not explicitly adopt an orthodox theory of vagueness.
Williamson (\citeproc{ref-Williamson1994-WILV}{1994}) attempts to trace
the epistemic theory back to the Stoics, but general consensus is that
these approaches were all but unknown until recently. What I want to
argue is that Keynes's intuitions are clearly with orthodoxy. Coates, on
the other hand, wants to place Keynes in a tradition that is critical of
classical analysis, and perhaps finds its best modern expression in the
exponents of fuzzy logics. To see this is wrong, note that the following
beliefs are all in the \emph{GT}.

\begin{description}
\tightlist
\item[(1)]
All goods are (definitely) investment goods or consumption goods.
\item[(2)]
For some goods it is vague whether they are an investment or consumption
good. (\emph{GT}: 61)
\item[(3)]
The yield of an investment, \emph{q}, is vague.
\item[(4)]
The carrying cost of an investment, \emph{c}, is vague.
\item[(5)]
The net yield of an investment, \emph{q}~-~\emph{c}, can be precisely
determined. (\emph{GT}: 226)
\end{description}

Since Keynes believed (1) to (5) we can safely conclude he believed they
were consistent. More importantly, since the \emph{GT} has been analysed
more thoroughly than any other economic text written this century, and
no one has criticised the consistency of (1) to (5), it seems many
people agree with him. Hence if conformity with pre-theoretic intuitions
of consistency is a central desideratum of a theory of vagueness, we can
discard any theory that does not say they are consistent. However, of
those theories on the market, only orthodox theories meet this
requirement. It might also be noted that (1) and (2) are repeated in
just about every introductory macro textbook, again without to my
knowledge any question of their consistency.

We can quickly see that these propositions are all consistent on either
orthodox theory. The supervaluationist says there is a set of classical
models for a language; a sentence is true iff it is true on all models,
false iff it is false on all models, and truth-valueless otherwise.
Vague terms have different meanings on different models. So for a
particular good, say a car, about which it is vague whether it is an
investment or consumption good, the supervaluationist says it is an
investment good on some models and a consumption good on others. So (2)
is satisfied; however on all models it, like everything else, is either
a consumption or investment good, so (1) is satisfied. Similarly because
it is vague whether some costs should be counted as deductions from the
yield of an investment or increments to its carrying cost, the values of
\emph{q} and \emph{c} will be different on different models. Hence (3)
and (4) are true, but \emph{q} -~\emph{c} is constant across
models\footnote{A particular cost will either remove an amount from
  \emph{q} or add an equal amount to \emph{c}, depending on how it is
  categorised.}, so (5) is true.

The epistemic theorist says that vagueness is just ignorance. As we can
know that a car is an investment or consumption good without knowing
which, (1) and (2) can be satisfied. Similarly, since we can know that a
cost is incurred without knowing how to account for it in Keynes's
terms, we can know \emph{q}~-~\emph{c} precisely without knowing
\emph{q} or \emph{c} precisely, and hence (3) to (5) can be satisfied.

The heterodox theorist has a harder time. The theorist who, following
Russell (\citeproc{ref-Russell1923}{1923}), says that vagueness is
infectious, if a part is vague so is the whole, will deny that (1) and
(2) can be true together. Unless it's definitely true that a car is an
investment or definitely true it's a consumption good it can't be
definitely true that it's one or the other. This also seems to be the
position taken by Wittgenstein (\citeproc{ref-Wittgenstein1953}{1953}).

The nihilist about vagueness, who follows Frege in saying vague terms
can't be used coherently, similarly can't endorse both (1) and (2). On
that view, if \emph{p} and \emph{q} are both vague, then their
disjunction can't be true. Arguably, on this position the disjunction of
\emph{p} with anything cant be true, as it is nonsense, but we don't
need anything that strong.\footnote{Compare the logic in Bochvar
  (\citeproc{ref-Bochvar1939}{1939}), where \emph{p}~∨ \emph{q} is
  truth-valueless if \emph{p} is true and \emph{q} truth-valueless.
  Summaries of this and many other many-valued logics are in Haack
  (\citeproc{ref-Haack1974}{1974}).}

The extra truth-values approach to vagueness (of which fuzzy logic is a
variant) also can't make (1) and (2) consistent. On any such approach
(whether 3-valued, \emph{n}-valued or continuum-valued) the degree of
truth of a disjunction can't be higher than the degree of truth of each
of the disjuncts. So if neither `This is an investment' nor `This is a
consumption good' is absolutely true (true to degree 1), `This is an
investment or consumption good' can't be absolutely true. Yet this is
just what Keynes asserted to be possible, and what several generations
of readers have found perfectly consistent. I have only remarked about
the problem the consistency of (1) and (2) poses for heterodox theories.
These remarks apply, \emph{mutatis mutandis}, to (3), (4) and (5), but
as theorists rarely discuss quantitative vagueness (as opposed to
truth-value vagueness) these cases involve a bit more speculation as to
what heterodoxy says.

Hence Keynes did not belong to a heterodox tradition \emph{vis a vis}
vagueness, and heterodox theories fail to capture a crucial
pre-theoretic intuition about vague terms. So Coates's claims that
Keynes followed Wittgenstein into heterodoxy here, and that he ought
have, are both mistaken.

Even if all of the above is mistaken, there remains serious doubt that
Keynes had in mind anything like what Coates attributes to him. Coates
makes the chapters on definitions in the \emph{GT} into the foundations
of a new philosophy, and constituting an important revolution in theory.
This is crucial to Coates's story about the influence of Wittgenstein on
Keynes. But this attribution is totally at odds with Keynes's comments
on these chapters, comments that not only reveal his attitudes towards
his definitions but also seem a fair commentary on them.

\begin{quote}
I have felt that these chapters were a great drag on getting on to the
real business, and would perplex the reader quite unnecessarily with a
lot of points which really do not matter to my proper theme (Keynes to
Roy Harrod, 9 August 1935, quoted in (\citeproc{ref-KeynesCW}{Keynes
1971} XIII: 537)).

But the main point I would urge is that all this is \emph{not}
fundamental. \emph{Being clear} is fundamental, but the choice of
definitions of income and investment is not (Keynes to Dennis Robertson,
29 January 1935, quoted in (\citeproc{ref-KeynesCW}{Keynes 1971} XIII,
495, italics in original)).
\end{quote}

\section{Keynes on Rules and Private
Language}\label{keynes-on-rules-and-private-language}

Had Keynes followed Wittgenstein in the ways suggested by either Bateman
or Coates he would have been led into error. Fortunately he was not
tempted. There was, however, one point on which Keynes clearly did not
follow Wittgenstein, and sadly so for Wittgenstein was right. If Kripke
(\citeproc{ref-Kripke1982}{1982}) is correct and this is the crucial
point in the later Wittgenstein's thinking, Keynes's failure to observe
it provides strong evidence that Wittgenstein's influence on him was at
best slight.

Keynes, as we saw above, thought we dealt with uncertainty by assuming
that the future would resemble the present. Call this Keynes's maxim.
But this, points out Wittgenstein, gets us nowhere. We know that the
future will resemble the present; what we don't know is how it will do
so. Wittgenstein illustrates this with examples from mathematics and
semantics, but we can apply it more broadly.

Say that a particle in a one-dimensional Euclidean space is now at
position \emph{d}, travelling at velocity \emph{v} under acceleration
\emph{a}. Assuming things stay the same, where will the particle be in 1
unit of time? This question simply can't be answered, until we know what
in what respect things will `stay the same'. If it is in respect of
position, the answer is \emph{d}, in respect of velocity it is
\emph{d}~+~\emph{v}, in respect of acceleration
\emph{d}~+~\emph{v}~+~\emph{a}/2. Perhaps our Newtonian intuitions make
us prefer the second answer, perhaps not.

The same story applies in economics. When we assume things will stay the
same, does that mean we are assuming the unemployment rate or the rate
of change of the unemployment rate to be the same; real growth or
nominal growth to be constant? At the level of the firm, we can ask
whether Keynes's maxim would have us assume real or nominal profits to
be constant, or perhaps the growth rate of real or nominal profits, or
perhaps sales figures (real or nominal, absolute or variation), or
perhaps one of the variables which play a role like acceleration (rate
of change of sales growth)? In some computing firms we might even take
some of the logarithmic variables (growth of logarithm of sales) to be
the constant. We can't in consistency assume more than one or two of
these variables to be unchanged, yet Keynes provides us with nothing to
tell between them.

More importantly, it looks like Keynes hasn't even seen the problem. The
mechanical example above looks very similar to some of the paradoxes of
indifference (\emph{TP}: Ch. 4). For example, in von Kries's cube
factory example, we know that a factory makes cubes with side length
between 0 and 2cm. If that's all we know, what should we say is the
probability that the next cube's side length will be greater than 1cm?
According to Laplace's principle of indifference we should divide the
probabilities equally between the possibilities, which seems to give an
answer of 1/2. However we could have set out the problem by saying that
the volume of cubes produced is between 0 and 8cm\textsuperscript{3} and
we want to know the probability the volume of the next cube is greater
than 1cm\textsuperscript{3}. Now the answer (to the same problem) looks
to be 7/8. And if we set out the problem in terms of surface area we
seem to get the answer 3/4. The conclusion is that the principle of
indifference could only be saved if we have a small designated set of
predicates to which we can exclusively apply it. But now it seems
Keynes's maxim can only work if we have a small designated set of
predicates to which we can exclusively apply it, and if we do that we
can avoid the paradoxes of indifference. Keynes explicitly adopts his
maxim to avoid the paradoxes of indifference (\emph{GT}: 152). He would
hardly have done this if he knew structurally similar problems beset the
maxim as best the principle of indifference. As further evidence he just
missed this point, note that while he was not averse to wielding
philosophical tools in economic writing (like the paradoxes of
indifference), Wittgenstein's point is not mentioned; not in the
\emph{GT}, not in any of its drafts and not in any of the correspondence
after it was published.

For Kripke, this point is central to Wittgenstein's private language
argument. All that we can know about the meaning of a word is how our
community has used it in the past. We must assume they'll use it the
same way in the future. But what is to count as using it the same way?
\emph{A} \emph{priori} it looks like any usage of a word could count;
the only thing that could make usage of a word wrong is the user has a
different way of using the word `the same way' to everyone else. Hence
if there is no community to set such standards there are no bars on how
words can be used. And if there are no such bars, there is nothing that
can properly be called a language. Hence there can't be a private
language.

Given the importance of that conclusion to Wittgenstein's later
philosophy, if Kripke is even close to right in his reconstruction then
it is central to the later Wittgenstein that Keynes's maxim is
contentless. As Keynes clearly didn't think this (witness the central
role it plays in summaries of the \emph{GT} like Keynes 1937) he hasn't
adopted a central tenet of the later Wittgenstein's work. This puts a
rather heavy burden on those who would say he became a Wittgensteinian.
The arguments presented so far do nothing to lift that burden.

\subsection*{References}\label{references}
\addcontentsline{toc}{subsection}{References}

\phantomsection\label{refs}
\begin{CSLReferences}{1}{0}
\bibitem[\citeproctext]{ref-Ayer1936}
Ayer, Alfred. 1936. \emph{Language, Truth and Logic.} London: Gollantz.

\bibitem[\citeproctext]{ref-Bateman1996}
Bateman, Bradley. 1996. \emph{Keynes's Uncertain Revolution.} Ann Arbor:
University of Michigan Press.

\bibitem[\citeproctext]{ref-Bochvar1939}
Bochvar, D. A. 1939. {``On a Three Valued Calculus and Its Application
to the Analysis of Contradictories.''} \emph{Matematicheskii Sbornik} 4
(2): 287--308. doi:
\href{https://doi.org/10.2307/2269081}{10.2307/2269081}.

\bibitem[\citeproctext]{ref-Borel1924}
Borel, Emile. 1924. {``A Propos d'un Trait{é} de Probabilit{é}s.''}
\emph{Revue Philosophique} 98: 321--36.

\bibitem[\citeproctext]{ref-BradfordHarcourt1997}
Bradford, Wylie, and Geoff Harcourt. 1997. {``Definitions and Units.''}
In \emph{A {``Second Edition''} of the General Theory}, edited by G. C.
Harcourt and P. A. Riach, 1:107--31. London: Routledge.

\bibitem[\citeproctext]{ref-Carnap1950}
Carnap, Rudolf. 1950. \emph{Logical Foundations of Probability}.
Chicago: University of Chicago Press.

\bibitem[\citeproctext]{ref-Coates1996}
Coates, John. 1996. \emph{The Claims of Common Sense}. Cambridge:
Cambridge University Press.

\bibitem[\citeproctext]{ref-Coates1997}
---------. 1997. {``Keynes, Vague Concepts and Fuzzy Logic.''} In
\emph{A {``Second Edition''} of the General Theory}, edited by G. C.
Harcourt and P. A. Riach, 2:244--60. London: Routledge.

\bibitem[\citeproctext]{ref-Davis1994}
Davis, John. 1994. \emph{Keynes's Philosophical Development}. Cambridge:
Cambridge University Press.

\bibitem[\citeproctext]{ref-Davis1995}
---------. 1995. {``Keynes' Later Philosophy.''} \emph{History of
Political Economy} 27 (2): 237--60. doi:
\href{https://doi.org/10.1215/00182702-27-2-237}{10.1215/00182702-27-2-237}.

\bibitem[\citeproctext]{ref-Dummett1959}
Dummett, Michael. 1959. {``Truth.''} \emph{Proceedings of the
Aristotelian Society} 59 (1): 141--62. doi:
\href{https://doi.org/10.1093/aristotelian/59.1.141}{10.1093/aristotelian/59.1.141}.

\bibitem[\citeproctext]{ref-Fine1975a}
Fine, Kit. 1975. {``Vagueness, Truth and Logic.''} \emph{Synthese} 30
(3-4): 265--300. doi:
\href{https://doi.org/10.1007/bf00485047}{10.1007/bf00485047}.

\bibitem[\citeproctext]{ref-vanFraassen1966}
Fraassen, Bas van. 1966. {``Singular Terms, Truth--Value Gaps and Free
Logic.''} \emph{Journal of Philosophy} 66 (17): 481--95. doi:
\href{https://doi.org/10.2307/2024549}{10.2307/2024549}.

\bibitem[\citeproctext]{ref-Haack1974}
Haack, Susan. 1974. \emph{Deviant Logic}. Chicago: University of Chicago
Press.

\bibitem[\citeproctext]{ref-Keefe2000}
Keefe, Rosanna. 2000. \emph{Theories of Vagueness}. Cambridge: Cambridge
University Press.

\bibitem[\citeproctext]{ref-Keynes1909}
Keynes, John Maynard. 1909. {``The Method of Index Numbers with Special
Reference to the Measurement of General Exchange Value.''} In \emph{The
Collected Writings of John Maynard Keynes}, XI:50--156. London:
Macmillan. Edited by D. E. Moggridge.

\bibitem[\citeproctext]{ref-Keynes1921}
---------. 1921. \emph{Treatise on Probability}. London: Macmillan.

\bibitem[\citeproctext]{ref-Keynes1931}
---------. 1931. {``Review of \emph{Foundations of Mathematics} by Frank
Ramsey.''} \emph{The New Statesman and Nation} 2: 407. Reprinted in
\cite[X 336-339]{KeynesCW}.

\bibitem[\citeproctext]{ref-Keynes1934}
---------. 1934. {``Draft of the \emph{General Theory}.''} In \emph{The
Collected Writings of John Maynard Keynes}, by John Maynard Keynes,
XIII:423--49. London: Macmillan. Edited by D. E. Moggridge.

\bibitem[\citeproctext]{ref-Keynes1936}
---------. 1936. \emph{The General Theory of Employment, Interest and
Money}. London: Macmillan.

\bibitem[\citeproctext]{ref-Keynes1937}
---------. 1937. {``The General Theory of Employment.''} \emph{Quarterly
Journal of Economics} 51 (2): 209--23. doi:
\href{https://doi.org/10.2307/1882087}{10.2307/1882087}. Reprinted in
\cite[XIV 109-123]{KeynesCW}, references to reprint.

\bibitem[\citeproctext]{ref-Keynes1938}
---------. 1938a. {``Letter to Hugh Townshend Dated 7 December.''} In
\emph{The Collected Writings of John Maynard Keynes}, by John Maynard
Keynes, 14:293--94. London: Macmillan. Edited by D. E. Moggridge.

\bibitem[\citeproctext]{ref-KeynesMEB}
---------. 1938b. {``My Early Beliefs.''} In \emph{The Collected
Writings of John Maynard Keynes}, X:433--51. London: Macmillan. Edited
by D. E. Moggridge.

\bibitem[\citeproctext]{ref-KeynesCW}
---------. 1971. \emph{The Collected Writings of John Maynard Keynes}.
London: Macmillan. Edited by D. E. Moggridge.

\bibitem[\citeproctext]{ref-Kripke1975}
Kripke, Saul. 1975. {``Outline of a Theory of Truth.''} \emph{Journal of
Philosophy} 72 (19): 690--716. doi:
\href{https://doi.org/10.2307/2024634}{10.2307/2024634}.

\bibitem[\citeproctext]{ref-Kripke1982}
---------. 1982. \emph{Wittgenstein on Rules and Private Language}.
Oxford: Basil Blackwell.

\bibitem[\citeproctext]{ref-Lewis1973a}
Lewis, David. 1973. \emph{Counterfactuals}. Oxford: Blackwell
Publishers.

\bibitem[\citeproctext]{ref-Moggridge1992}
Moggridge, Donald. 1992. \emph{Maynard Keynes: An Economist's
Biography}. London: Routledge.

\bibitem[\citeproctext]{ref-ODonnell1989}
O'Donnell, Rod. 1989. \emph{Keynes: Philosophy, Economics and Politics}.
London: Macmillan.

\bibitem[\citeproctext]{ref-ODonnell1991}
---------. 1991. {``Reply.''} In \emph{Keynes as Philosopher-Economist},
edited by Rod O'Donnell, 78--102. London: Macmillan.

\bibitem[\citeproctext]{ref-ODonnell1997}
---------. 1997. {``Keynes and Formalism.''} In \emph{A {``Second
Edition''} of the General Theory}, edited by G. C. Harcourt and P. A.
Riach, 2:131--65. London: Routledge.

\bibitem[\citeproctext]{ref-Ramsey1929}
Ramsey, Frank. 1929/1990. {``Probability and Partial Belief.''} In
\emph{Philosophical Papers}, edited by D. H. Mellor, 95--96. Cambridge
University Press.

\bibitem[\citeproctext]{ref-RamseyTruthProb}
---------. 1926. {``Truth and Probability.''} In \emph{Philosophical
Papers}, edited by D. H. Mellor, 52--94. Cambridge: Cambridge University
Press.

\bibitem[\citeproctext]{ref-RamseyFM}
---------. 1931. \emph{The Foundations of Mathematics and Other Logical
Essays}. Edited by R. B. Braithwaite. London: Routledge.

\bibitem[\citeproctext]{ref-Russell1923}
Russell, Bertrand. 1923. {``Vagueness.''} \emph{Australasian Journal of
Philosophy and Psychology} 1 (2): 84--92. doi:
\href{https://doi.org/10.1080/00048402308540623}{10.1080/00048402308540623}.

\bibitem[\citeproctext]{ref-Russell1948}
---------. 1948. \emph{Human Knowledge: Its Scope and Limits}. London:
Allen; Unwin.

\bibitem[\citeproctext]{ref-Skidelsky1983}
Skidelsky, Robert. 1983. \emph{John Maynard Keynes. Vol. I: Hopes
Betrayed, 1883-1920}. London: Macmillan.

\bibitem[\citeproctext]{ref-Skidelsky1992}
---------. 1992. \emph{John Maynard Keynes. Vol. II: The Economist as
Saviour, 1920-1937}. London: Macmillan.

\bibitem[\citeproctext]{ref-Wang1987}
Wang, Hao. 1987. \emph{Reflections on Gôdel}. Cambridge, MA: MIT Press.

\bibitem[\citeproctext]{ref-Williamson1994-WILV}
Williamson, Timothy. 1994. \emph{{Vagueness}}. Routledge.

\bibitem[\citeproctext]{ref-Wittgenstein1953}
Wittgenstein, Ludwig. 1953. \emph{Philosophical Investigations}. London:
Macmillan.

\end{CSLReferences}



\noindent Unpublished. Posted online in 2001.


\end{document}
