% Options for packages loaded elsewhere
\PassOptionsToPackage{unicode}{hyperref}
\PassOptionsToPackage{hyphens}{url}
%
\documentclass[
  10pt,
  letterpaper,
  DIV=11,
  numbers=noendperiod,
  twoside]{scrartcl}

\usepackage{amsmath,amssymb}
\usepackage{setspace}
\usepackage{iftex}
\ifPDFTeX
  \usepackage[T1]{fontenc}
  \usepackage[utf8]{inputenc}
  \usepackage{textcomp} % provide euro and other symbols
\else % if luatex or xetex
  \usepackage{unicode-math}
  \defaultfontfeatures{Scale=MatchLowercase}
  \defaultfontfeatures[\rmfamily]{Ligatures=TeX,Scale=1}
\fi
\usepackage{lmodern}
\ifPDFTeX\else  
    % xetex/luatex font selection
  \setmainfont[ItalicFont=EB Garamond Italic,BoldFont=EB Garamond
Bold]{EB Garamond Math}
  \setsansfont[]{Europa-Bold}
  \setmathfont[]{Garamond-Math}
\fi
% Use upquote if available, for straight quotes in verbatim environments
\IfFileExists{upquote.sty}{\usepackage{upquote}}{}
\IfFileExists{microtype.sty}{% use microtype if available
  \usepackage[]{microtype}
  \UseMicrotypeSet[protrusion]{basicmath} % disable protrusion for tt fonts
}{}
\usepackage{xcolor}
\usepackage[left=1in, right=1in, top=0.8in, bottom=0.8in,
paperheight=9.5in, paperwidth=6.5in, includemp=TRUE, marginparwidth=0in,
marginparsep=0in]{geometry}
\setlength{\emergencystretch}{3em} % prevent overfull lines
\setcounter{secnumdepth}{3}
% Make \paragraph and \subparagraph free-standing
\ifx\paragraph\undefined\else
  \let\oldparagraph\paragraph
  \renewcommand{\paragraph}[1]{\oldparagraph{#1}\mbox{}}
\fi
\ifx\subparagraph\undefined\else
  \let\oldsubparagraph\subparagraph
  \renewcommand{\subparagraph}[1]{\oldsubparagraph{#1}\mbox{}}
\fi


\providecommand{\tightlist}{%
  \setlength{\itemsep}{0pt}\setlength{\parskip}{0pt}}\usepackage{longtable,booktabs,array}
\usepackage{calc} % for calculating minipage widths
% Correct order of tables after \paragraph or \subparagraph
\usepackage{etoolbox}
\makeatletter
\patchcmd\longtable{\par}{\if@noskipsec\mbox{}\fi\par}{}{}
\makeatother
% Allow footnotes in longtable head/foot
\IfFileExists{footnotehyper.sty}{\usepackage{footnotehyper}}{\usepackage{footnote}}
\makesavenoteenv{longtable}
\usepackage{graphicx}
\makeatletter
\def\maxwidth{\ifdim\Gin@nat@width>\linewidth\linewidth\else\Gin@nat@width\fi}
\def\maxheight{\ifdim\Gin@nat@height>\textheight\textheight\else\Gin@nat@height\fi}
\makeatother
% Scale images if necessary, so that they will not overflow the page
% margins by default, and it is still possible to overwrite the defaults
% using explicit options in \includegraphics[width, height, ...]{}
\setkeys{Gin}{width=\maxwidth,height=\maxheight,keepaspectratio}
% Set default figure placement to htbp
\makeatletter
\def\fps@figure{htbp}
\makeatother
% definitions for citeproc citations
\NewDocumentCommand\citeproctext{}{}
\NewDocumentCommand\citeproc{mm}{%
  \begingroup\def\citeproctext{#2}\cite{#1}\endgroup}
\makeatletter
 % allow citations to break across lines
 \let\@cite@ofmt\@firstofone
 % avoid brackets around text for \cite:
 \def\@biblabel#1{}
 \def\@cite#1#2{{#1\if@tempswa , #2\fi}}
\makeatother
\newlength{\cslhangindent}
\setlength{\cslhangindent}{1.5em}
\newlength{\csllabelwidth}
\setlength{\csllabelwidth}{3em}
\newenvironment{CSLReferences}[2] % #1 hanging-indent, #2 entry-spacing
 {\begin{list}{}{%
  \setlength{\itemindent}{0pt}
  \setlength{\leftmargin}{0pt}
  \setlength{\parsep}{0pt}
  % turn on hanging indent if param 1 is 1
  \ifodd #1
   \setlength{\leftmargin}{\cslhangindent}
   \setlength{\itemindent}{-1\cslhangindent}
  \fi
  % set entry spacing
  \setlength{\itemsep}{#2\baselineskip}}}
 {\end{list}}
\usepackage{calc}
\newcommand{\CSLBlock}[1]{\hfill\break\parbox[t]{\linewidth}{\strut\ignorespaces#1\strut}}
\newcommand{\CSLLeftMargin}[1]{\parbox[t]{\csllabelwidth}{\strut#1\strut}}
\newcommand{\CSLRightInline}[1]{\parbox[t]{\linewidth - \csllabelwidth}{\strut#1\strut}}
\newcommand{\CSLIndent}[1]{\hspace{\cslhangindent}#1}

\setlength\heavyrulewidth{0ex}
\setlength\lightrulewidth{0ex}
\usepackage[automark]{scrlayer-scrpage}
\clearpairofpagestyles
\cehead{
  Brian Weatherson
  }
\cohead{
  Deliberation Costs
  }
\ohead{\bfseries \pagemark}
\cfoot{}
\makeatletter
\newcommand*\NoIndentAfterEnv[1]{%
  \AfterEndEnvironment{#1}{\par\@afterindentfalse\@afterheading}}
\makeatother
\NoIndentAfterEnv{itemize}
\NoIndentAfterEnv{enumerate}
\NoIndentAfterEnv{description}
\NoIndentAfterEnv{quote}
\NoIndentAfterEnv{equation}
\NoIndentAfterEnv{longtable}
\NoIndentAfterEnv{abstract}
\renewenvironment{abstract}
 {\vspace{-1.25cm}
 \quotation\small\noindent\rule{\linewidth}{.5pt}\par\smallskip
 \noindent }
 {\par\noindent\rule{\linewidth}{.5pt}\endquotation}
\KOMAoption{captions}{tableheading}
\makeatletter
\@ifpackageloaded{caption}{}{\usepackage{caption}}
\AtBeginDocument{%
\ifdefined\contentsname
  \renewcommand*\contentsname{Table of contents}
\else
  \newcommand\contentsname{Table of contents}
\fi
\ifdefined\listfigurename
  \renewcommand*\listfigurename{List of Figures}
\else
  \newcommand\listfigurename{List of Figures}
\fi
\ifdefined\listtablename
  \renewcommand*\listtablename{List of Tables}
\else
  \newcommand\listtablename{List of Tables}
\fi
\ifdefined\figurename
  \renewcommand*\figurename{Figure}
\else
  \newcommand\figurename{Figure}
\fi
\ifdefined\tablename
  \renewcommand*\tablename{Table}
\else
  \newcommand\tablename{Table}
\fi
}
\@ifpackageloaded{float}{}{\usepackage{float}}
\floatstyle{ruled}
\@ifundefined{c@chapter}{\newfloat{codelisting}{h}{lop}}{\newfloat{codelisting}{h}{lop}[chapter]}
\floatname{codelisting}{Listing}
\newcommand*\listoflistings{\listof{codelisting}{List of Listings}}
\makeatother
\makeatletter
\makeatother
\makeatletter
\@ifpackageloaded{caption}{}{\usepackage{caption}}
\@ifpackageloaded{subcaption}{}{\usepackage{subcaption}}
\makeatother
\ifLuaTeX
  \usepackage{selnolig}  % disable illegal ligatures
\fi
\IfFileExists{bookmark.sty}{\usepackage{bookmark}}{\usepackage{hyperref}}
\IfFileExists{xurl.sty}{\usepackage{xurl}}{} % add URL line breaks if available
\urlstyle{same} % disable monospaced font for URLs
\hypersetup{
  pdftitle={Deliberation Costs},
  pdfauthor={Brian Weatherson},
  hidelinks,
  pdfcreator={LaTeX via pandoc}}

\title{Deliberation Costs\thanks{Unpublished draft. Thanks to audiences
at Michigan, Toronto, and the Ranch Metaphysics Workshop for valuable
feedback, not all of which I've yet incorporated.}}
\author{Brian Weatherson}
\date{2020}

\begin{document}
\maketitle
\begin{abstract}
Our theory of rational choice should be sensitive to deliberation costs.
It is irrational to take into account minor differences between goods,
if the cost of taking those differences into account is greater than the
expected gain from doing so. It has often been held in economics that
this line of reasoning will lead to an infinite regress. I argue that
the regress can be stopped if we take the rational chooser to be skilled
at attending to the right information. On the appropriate model of
skill, the rational agent will attend to the right information without
reasoning about whether this is the right information to attend to.
\end{abstract}

\setstretch{1.1}
Humans making decisions face two big limitations. First, we are
informationally limited. We don't know everything and sometimes we don't
know what we need to know in order to make the optimal decision. Second,
we are computationally limited. We can't process all of the information
that we have available to us before a decision needs to be made. Or at
least, we can't do this in a costless manner.

Orthodox decision theory treats these two limitations very differently.
To a first approximation, the whole point of orthodox decision theory is
to handle the question of how to make decisions without full
information. But on the other hand, orthodox decision theory simply
assumes away the computational limitations. Orthodox decision theory is
a theory of rational choice, and rationality is here understood to
involve not being subject to these pesky computational limitations.

I think this is a serious mistake. In particular, I think there are
several cases where our theory of rational choice can only give us the
correct verdict if we allow it to be sensitive to both kinds of
limitation. In this paper, I will discuss three such kinds of cases, and
describe how rational choice theory might be revised so as to handle
them.

I'm far from the first to notice this asymmetry in how orthodox decision
theory handles the two limitations. For approximately as long as
decision theory has existed, there have been people who have noted the
oddity of ignoring computational limitations. But there has always been
a powerful argument against taking computational limitations seriously.
It is long been thought that attempt to do this would lead to a nasty
kind of regress. It isn't entirely clear how the regress argument here
is supposed to run; the argument is more often alluded to than carefully
stated. But it is a major challenge and I will have something to say
about it.

The short version of what I'm going to say is that while we should take
both kinds of limitations seriously, we should treat them differently in
our final theory. We should, as orthodox decision theory says, take a
broadly evidentialist approach to informational limitations. That is,
good decision makers should have credence distributions over the
possibilities left open by their evidence, those credences should be
sensitive to the evidence they have, and the choices they make should
maximize expected value given those credences. On the other hand, we
should take a broadly reliabilist approach to computational limitations.
Good decision makers will adopt procedures for managing their own
limitations that reliably produce good outcomes. There is no requirement
that they adopt the procedures that are best supported by their
evidence. The reason there is no requirement they do that is that
figuring out what those reliable procedures might be is even more
computationally taxing then the problem of deciding what to do. And if
we're going to respect the fact that people can't always complete
difficult computational tasks, we shouldn't expect them to perform the
incredibly difficult task of figuring out how to adjust their decision
procedures in light of the evidence about their own limitations.

You might think that the reason orthodox theory treats computational
limitations this way is that it is simply trying to provide a theory of
ideal decision making. There is a separate question, to be sure, of how
non-ideal agents should make decisions. But the thought, or at least the
hope, is that clearly stating what the ideal looks like will help the
non-ideal agents in this task. I think there is a little reason to
believe that this hope will be realized. In general, knowing what the
ideal looks like provides us with very little guidance as to how to get
better. Knowing that any ideal outcome has a certain attribute does not
provide a reason, even a defeasible reason, for trying to to acquire
that attribute (\citeproc{ref-LipseyLancaster}{Lipsey and Lancaster
1956}).

We can see this by simply thinking about the one limitation that
orthodox theory does take seriously. A good decision maker without full
information will in general behave nothing like a good decision maker
with full information. For example, if you put the informationally
limited agent in a casino they will do the exact opposite of what an
informationally unlimited agent will do. The informationally unlimited
agent will play every game and do quite well at them. The
informationally limited agent, on the other hand, will play none of the
games because they all have negative expected returns. I think is the
general case. It's a bad idea to emulate the ideal agent, because us non
ideal agents often have to act so as to minimize the damage that have
other limitations can do. Everyone agrees that is true in the case of
informational limitations, and I am going to try and argue that it's
also true for computational limitations.

So here's the plan for the paper. First, in sections 1-2, I will
introduce the three kinds of cases but I think motivate taking
computational limitations seriously. Then, in sections 3-4, I will
introduce the regress argument that is alleged to show that any attempt
to do this will end badly. In sections 5-6, I will show how the broadly
reliabilist approach to handling computational limitations that I favor
can be motivated, and can avoid the regress. Sections 7 and 8 are
contingent speculations about how non-ideal agents might choose
reliably, and observations on how these debates connect to other
philosophical debates

Before I start on this, it's helpful to get clear on exactly what I am
taking my orthodox opponent to be committed to. I take them to endorse
the following three constraints on a theory of rational choice.

\begin{enumerate}
\def\labelenumi{\arabic{enumi}.}
\tightlist
\item
  Rational agents have credences, and these credences are responsive to
  evidence.
\item
  These credences also respect the probability calculus.
\item
  Rational agents take actions that maximize their expected utility
  given these credences.
\end{enumerate}

There are a lot of questions that I do not take my orthodox opponent to
have a settled view on, though of course many orthodox theorists will
have one view or another on one or other of these questions. These
questions include

\begin{itemize}
\tightlist
\item
  Whether rationality puts any constraints on what can be valued;
\item
  Whether our theory of rationality divides up failures to make rational
  choices into epistemic failures, axiological failures, and practical
  failures, and if it does make such a division, exactly how it should
  be made;
\item
  Whether rationality requires that agent be self-aware, i.e., whether
  they know what their own credences and utilities are; and
\item
  Exactly what evidence is, or what it means for credences to be
  responsive to evidence.
\end{itemize}

My hope is that I can provide an objection to orthodoxy that is
insensitive to how orthodox theorists answered these questions. That's a
rather ambitious project, since the answers one gives to these questions
will help provide responses to some of the objections I shall offer. But
I'm not going to try to anticipate every possible response the orthodox
theorist could make. Indeed, I don't think that I've got anything like a
knock down watertight argument against all possible versions of
orthodoxy. What I think I do have is a set of reasons to consider an
alternative, and an outline of what that alternative may look like.

\section{Three Puzzles}\label{three-puzzles}

\subsection{Puzzle One - Close Calls}\label{puzzle-one---close-calls}

Let's start with an example from a great thinker. It will require a
little exegesis, but that's not unusual when using classic texts.

\begin{quote}
Well Frankie Lee and Judas Priest\\
They were the best of friends\\
So when Frankie Lee needed money one day\\
Judas quickly pulled out a roll of tens\\
And placed them on the footstool\\
Just above the potted plain\\
Saying ``Take your pick, Frankie boy,\\
My loss will be your gain.''\\
\strut ~~~~(``The Ballad of Frankie Lee and Judas Priest'', 1968.\\
\strut ~~~~~Lyrics from Bob Dylan (\citeproc{ref-Dylan2016}{2016}) 225)
\end{quote}

On a common reading of this, Judas Priest isn't just asking Frankie Lee
how much money he wants to take, but which invididual notes. Let's
simplify, and say that it is common ground that Frankie should only take
\$10, so the choice Frankie Lee has is which of the individual notes he
will take. This will be enough to set up the puzzle.

Assume something else that isn't in the text, but which isn't an
implausible addition to the story. The world Frankie Lee and Judas
Priest live in is not completely free of counterfeit notes. And it would
be bad for Frankie Lee to take a counterfeit note. It won't matter just
how common these notes are, or how bad it would be. But our puzzle will
be most vivid if each of these are relatively small quantities. So there
aren't that many counterfeit notes in circulation, and the (expected)
disutility to Frankie Lee of having one of them is not great. There is
some chance that he will get in trouble, but the chance isn't high, and
the trouble isn't any worse than he's suffered before. Still, other
things exactly equal, Frankie Lee would prefer a genuine note to a
counterfeit one.

Now for some terminology to help us state the problem Frankie Lee is in.
Assume there are \emph{k} notes on the footstool. Call them
\emph{n}\textsubscript{1}, \ldots, \emph{n\textsubscript{k}}. Let
\emph{c\textsubscript{i}} be the proposition that note
\emph{n\textsubscript{i}} is counterfeit, and its negation
\emph{g\textsubscript{i}} be that it is genuine. And let
\emph{t\textsubscript{i}} be the act of taking note
\emph{n\textsubscript{i}}. Let \emph{U} be Frankie Lee's utility
function, and \emph{Cr} his credence function.

In our first version of the example, we'll make two more assumptions.
Apart from the issue of whether the note is real or counterfeit, Frankie
Lee is indifferent between the notes, so for some \emph{h}, \emph{l},
\emph{U}(\emph{t\textsubscript{i}} \textbar{}
\emph{g\textsubscript{i}})~=~\emph{h} and
\emph{U}(\emph{t\textsubscript{i}} \textbar{}
\emph{c\textsubscript{i}})~=~\emph{l} for all \emph{i}, with of course
\emph{h} \textgreater{} \emph{l}. If we add an extra assumption that
Frankie Lee thinks the probability that each of the notes is genuine is
the same, we get the intuitive result back that he is indifferent
between the banknotes.

But is that really a plausible move? Here is one way to start worrying
about it. Change the example so that the country Frankie Lee and Judas
Priest live in is very slowly modernising its currency. It is getting
rid of old fashioned, and somewhat easy to counterfeit, paper money, and
joining the civilised countries that use plastic money. Moreover,
plastic bank notes are, for all intents and purposes, impossible to
counterfeit. (At least, no one has yet figured out how to do it, and
Frankie Lee knows this.)

Some of the notes Judas Priest offers are the new plastic notes, and
some are the old paper notes. Now it seems clear that Frankie Lee should
take one of the new notes, and not merely on aesthetic grounds. Rather,
the fact that the plastic notes are less likely to be counterfeit is a
reason to prefer to take them. And this is true no matter how unlikely
it is that the paper notes are counterfeit, as long as this likelihood
is non-zero.

But now go back to the base case, where all the money is paper. A small
change in probability of being counterfeit seems to be enough to give
Frankie Lee a reason to prefer some of them to the others. Indeed, the
only way for him to be indifferent between the notes is if the
probability of any one being counterfeit is exactly the same as the
probability of any other being counterfeit. But that two of the notes
have exactly the same probability of being counterfeit is a measure zero
event. It isn't happening. So Frankie Lee shouldn't be indifferent
between the notes.

Of course, if the notes look exactly the same, then the probability that
each is counterfeit is exactly the same. But that's only because that
probability is one. In that case Frankie Lee should run away as fast as
possible. That's not the realistic case.

The realistic case is that the notes look a little different to each
other in ever so many respects. (Including, one hopes, their serial
numbers.) Some will be a little more faded, or a little more torn, or a
little more smudged or crumpled, than the others. It is overwhelmingly
likely that these fades, tears, smudges, spills etc are the result of
the normal wear and tear on the currency - wear and tear that paper
notes tend to wear on their face. But every imperfection in every note
is some evidence, very very marginal evidence but still evidence, that
the note is counterfeit. And since Frankie Lee's evidence, on any extant
theory of evidence, includes visible things like the tears, smudges etc
on the notes, they are pieces of evidence that affect the evidential
expected utility of taking each note. So if Frankie Lee wants to
maximize evidential expected utility, there is precisely one note he
should take. Though it probably won't be obvious to him which one it is,
so rationality requires Frankie Lee to spend some time thinking about
which note is best.

This is intuitively the wrong result. (Though it is what happens in the
song.) Frankie Lee should just make a choice more or less arbitrarily.
Since expected utility theory does not say this, expected utility theory
is wrong.

The Frankie Lee and Judas Priest case is weird. Who offers someone
money, then asks them to pick which note to take? And intuitions about
such weird cases cases are sometimes deprecated. Perhaps the contrivance
doesn't reveal deep problems with a philosophical theory, but merely a
quirk of our intuitions. I am not going to take a stand on any big
questions about the epistemology of intuitions here. Rather, I'm going
to note that cases with the same structure as the story of Frankie Lee
and Judas Priest are incredibly common in the real world. Thinking about
the real world examples can both show us how pressing the problems are,
and eventually show us a way out of those problems.

So let's leave Frankie Lee for now, just above the potted plain, and
think about a new character. We will call this one David, and he is
buying a few groceries on the way home from work. In particular, he has
to buy a can of chickpeas, a bottle of milk, and a carton of eggs. To
make life easy, we'll assume each of these cost the same amount: five
dollars.\footnote{If that sounds implausible to you, make the
  can/bottle/carton a different size, or change the currency to some
  other dollars than the one you're instinctively using. But I think
  this examples works tolerably well when understand as involving, for
  example, East Carribean dollars.} None of these purchases is entirely
risk free. Canned goods are pretty safe, but sometimes they go bad. Milk
is normally removed from sale when it goes sour, but not always. And
eggs can crack, either in transit or just on the shelf. In David's
world, just like ours, each of these risks is greater than the one that
came before.

David has a favorite brand of chickpeas, of milk, and of eggs. And he
knows where in the store they are located. So his shopping is pretty
easy. But it isn't completely straightforward. First he gets the
chickpeas. And that's simple; he grabs the nearest can, and unless it is
badly dented, or leaking, he puts in in his basket. Next he goes onto
the milk. The milk bottles have sell-by dates printed in big letters on
the front. And David checks that he isn't picking up one that is about
to expire. His store has been known to have adjacent bottles of milk
with sell-by dates 10 days apart, so it's worth checking. But as long as
the date is far enough in the future, he takes it and moves on. Finally,
he comes to the eggs. (Nothing so alike as eggs, he always thinks to
himself.) Here he has to do a little more work. He takes the first
carton, opens it to see there are no cracks on the top of the eggs, and,
finding none, puts that in his basket too. He knows some of his friends
do more than this; flipping the carton over to check for cracks
underneath. But the one time he tried that, the eggs ended up on the
floor. And he knows some of his friends do less; just picking up the
carton by the underside, and only checking for cracks if the underside
is sticky where the eggs have leaked. He thinks that makes sense too,
but he is a little paranoid, and likes visual confirmation of what he's
getting. All done, he heads to the checkout, pays his \$15, and goes
home.

The choice David faces when getting the chickpeas is like the choice
Frankie Lee faces. He has to choose from among a bunch of very similar
seeming options. In at least the chickpeas example, he should just pick
arbitrarily. But for very similar reasons to Frankie Lee, expected
utility theory won't say that.

The standard model of practical rationality that we use in philosophy is
that of expected utility maximization. But there are both theoretical
and experimental reasons to think that this is not the right model for
choices such as that faced by Frankie or David. maximizing expected
utility is resource intensive, especially in contexts like a modern
supermarket, and the returns on this resource expenditure are
unimpressive. What people mostly do, and what they should do, is choose
in a way that is sensitive to the costs of adopting one or other way.

There are two annoying terminological issues around here that I mostly
want to set aside, but need to briefly address in order to forestall
confusion.

I'm going to assume maximizing expected utility means taking the option
with the highest expected utility given facts that are readily
available. So if one simply doesn't process a relevant but
observationally obvious fact, that can lead to an irrational choice. I
might alternatively have said that the choice was rational (given the
facts the chooser was aware of), but the observational process was
irrational. But I suspect that terminology would just add needless
complication.

I'm going to spend more time on another point that is partially
terminological, but primarily substantive. That's whether we should
identify the choice consequentialists recommend in virtue of the fact
that it maximizes expected utility with one of the options (in the
ordinary sense of option), or something antecedent. I'm going to
stipulate (more or less) that it is consistent with consequentialism
that the choice can be something antecedent - it can be something like a
choice procedure. And I'm going to argue that this is what the rational
consequentialist should choose.

I'm going to call any search procedure that is sensitive to resource
considerations a satisficing procedure. This isn't an uncommon usage.
Charles Manski (\citeproc{ref-Manski2017}{2017}) uses the term this way,
and notes that it has rarely been defined more precisely than that. But
it isn't the only way that it is used. Mauro Papi
(\citeproc{ref-Papi2013}{2013}) uses the term to exclusively mean that
the chooser has a `reservation level', and they choose the first option
that crosses it. This kind of meaning will be something that becomes
important again in a bit. And Chris Tucker
(\citeproc{ref-Tucker2016}{2016}), following a long tradition in
philosophy of religion, uses it to mean any choice procedure that does
not optimize. Elena Reutskaja et al -Reutskaja et al.
(\citeproc{ref-Reutskaja2011}{2011}) contrast a `hybrid' model that is
sensitive to resource constraints with a `satisficing' model that has a
fixed reservation level. They end up offering reasons to think ordinary
people do (and perhaps should) adopt this hybrid model. So though they
don't call this a satisficing approach, it just is a version of what
Manski calls satisficing. Andrew Caplin et al
(\citeproc{ref-Caplin2011}{2011}), on the other hand, describe a very
similar model to Reutskaja et al's hybrid model - one where agents try
to find something above a reservation level but the reservation level is
sensitive to search costs - as a form of satisficing. So the terminology
around here is a mess. I propose to use Manski's terminology: agents
satisfice if they choose in a way that is sensitive to resource
constraints.

Ideally they would maximize, subject to constraints, but saying anything
more precise than this brings back the regress problem that we started
with. Let's set it aside just a little longer, and go back to David and
the chickpeas.

When David is facing the shelf of chickpeas, he can rationally take any
one of them - apart perhaps from ones that are seriously damaged. How
can expected utility theory capture that fact? I think if it identifies
David's choices with the cans on the shelf, and not with a procedure for
choosing cans, then it cannot.

It says that more than one choice is permissible only if the choices are
equal in expected utility. So the different cans are equal in expected
utility. But on reflection, this is an implausible claim. Some of the
cans are ever so slightly easier to reach. Some of the cans will have
ever so slight damage - a tiny dint here, a small tear in the label
there - that just might indicate a more serious flaw. Of course, these
small damages are almost always irrelevant, but as long as the
probability that they indicate damage is positive, it breaks the
equality of the expected utility of the cans. Even if there is no
visible damage, some of the labels will be ever so slightly more faded,
which indicates that the cans are older, which ever so slightly
increases the probability that the goods will go bad before David gets
to use them. Of course in reality this won't matter more than one time
in a million, but one in a million chances matter if you are asking
whether two expected utilities are strictly equal.

The common thread to the last paragraph is that these objects on the
shelves are almost duplicates, but the most careful quality control
doesn't produce consumer goods that are actual duplicates. There are
always some differences. It is unlikely that these differences make
precisely zero difference to the expected utility of each choice. And
even if they do, discovering that is hard work.

So it seems likely that, according to the expected utility model, it
isn't true that David could permissibly take any can of chickpeas that
is easily reachable and not obviously flawed. Even if that is true, it
is extremely unlikely that David could know it to be true. But one thing
we know about situations like David's is that any one of the (easily
reached, not clearly flawed) cans can be permissibly chosen, and David
can easily know that. So the expected utility model, as I've so far
described it, is false.

\subsection{Puzzle Two - Psychic Costs of
Bias}\label{puzzle-two---psychic-costs-of-bias}

In all but a vanishingly small class of cases, the different cans will
not have the same expected utility. But figuring out which can has the
highest expected utility is going to be work. It's possible in
principle, I suppose, that someone could be skilled at it, in the sense
that they could instinctively pick out the can whose shape, label
fading, etc., reveal it to have the highest expected utility. Such a
skill seems likely to be rare - though I'll come back to this point
below when considering some other skills that are probably less rare.
For most people, maximizing expected utility will not be something that
can be done through skill alone; it will take effort. And this effort
will be costly, and almost certainly not worth it. Although one of the
cans will be ever so fractionally higher in expected utility than the
others, the cost of finding out which can this is will be greater than
the difference in expected utility of the cans. So aiming to maximize
expected utility will have the perverse effect of reducing one's overall
utility, in a predictable way.

The costs of trying to maximize expected utility go beyond the costs of
engaging in search and computation. There is evidence that people who
employ maximizing strategies in consumer search end up worse off than
those who don't. Schwartz et al. (\citeproc{ref-SchwartzEtAl2002}{2002})
reported that consumers could be divided in `satisficers' and
`maximizers'. And once this division is made, it turns out that the
maximizers are less happy with individual choices, and with their life
in general. This finding has been extended to work on career choice
(\citeproc{ref-IyengarEtAl2006}{Iyengar, Wells, and Schwartz 2006}),
where the maximizers end up with higher salaries but less job
satisfaction, and to friend choice (\citeproc{ref-NewmanEtAl2018}{Newman
et al. 2018}), where again the maximizers seem to end up less satisfied.

There are two things that can go wrong when you try to maximize.
Maximising requires considering the strengths and weaknesses of each of
the choices. That means, it requires giving at least some consideration
to the negative attributes of what you end up choosing. And these can
cause you to be less happy with the actual choice when those negative
attributes are realized. And it also means giving consideration to the
positive attributes of the choices not made. And this could lead to
regret when you have to adopt a choice that lacks those positive
attributes. So there are two very natural paths by which the attempt to
maximize could backfire, any incurs costs that wouldn't have been
incurred by the person who simply makes an arbitrary choice.

There is evidence here that both these paths are realised, and that
maximisers do indeed end up psychically worse off than satisficers. Now
to be sure, there are both empirical and theoretical reasons to be
cautious about accepting these results at face value. Whether the second
path, from consideration of positive attributes of the non-chosen option
to felt regret, is psychologically significant seems to be tied up with
the `paradox of choice' (\citeproc{ref-Schwartz2004}{Schwartz 2004}),
the idea that sometimes giving people even more choices makes them less
happy with their outcome, because they are more prone to regret. But it
is unclear whether such a paradox exists. One meta-analysis
(\citeproc{ref-ScheibehenneEtAl2010}{Scheibehenne, Greifeneder, and Todd
2010}) did not show the effect existing at all, though a later
meta-analysis finds a significant mediated effect
(\citeproc{ref-ChernevEtAl2015}{Chernev, Böckenholt, and Goodman 2015}).
But it could also be that the result is a feature of an idiosyncratic
way of carving up the maximizers from the satisficers. Another way of
dividing them up produces no effect at all
(\citeproc{ref-DiabEtAl2008}{Diab, Gillespie, and Highhouse 2008}).

The theoretical reasons relate to Newcomb's problem. Even if we knew
that maximizers were less satisfied with how things are going than
satisficers, it isn't obvious that any one person would be better off
switching to satisficing. They might be like a two-boxer who would get
nothing if they took one-box. There is a little evidence in Iyengar,
Wells, and Schwartz (\citeproc{ref-IyengarEtAl2006}{2006}) that this
isn't quite what is happening, but the overall situation is unclear.

But the philosophical questions here are a bit simpler than the
psychological questions. Whether maximisers in general are subject to
these two kinds of costs is a tricky empirical question. Whether there
could be one maximiser who is subject to them, and who knows that they
are, is a much easier question. Of course someone could be like that.
Indeed, it seems beyond dispute that many real people are subject to
these costs. The only empirical question is whether these people are a
significant minority or a significant majority.

And all it takes for the philosophical question to be pressing is that
some choosers are, and know that they are, disposed to incur these
psychological costs if they consciously try to maximise expected value.
Our theory of choice should have something to say to them, and orthodox
theory is silent. Especially for choices that are intended to produce
happiness, the happiness effects of the choice procedure itself should
be taken into account. But orthodox theory ignores it.

\subsection{Puzzle Three - Mathematical
Challenges}\label{puzzle-three---mathematical-challenges}

For a final case, let's consider Kyla, a student taking a mathematics
exam. It's getting towards the end of the exam and she's facing quite a
bit a time pressure. She comes to a true false question, and she knows
that she knows how to solve questions like it. But she also knows that
there are other kinds of questions that she is better at solving under
time pressure. And while this is just a true false question, the exam is
set up so that she gets a large negative score if she gets the question
wrong. The expected return of simply guessing is strongly negative.

The rational thing for Kyla to do is to go on to other questions and
come back to this one if she has time. But orthodox theory doesn't allow
for this. The probability of any mathematical truth is one. And it's
part of orthodoxy that credences are supposed to be probability
functions. So whatever the correct answer is, offering it will have
positive expected utility given Kyla's credences, assuming those
credences are rational.

So orthodoxy gets this choice, and all other choices that turn on
mathematical ignorance, badly wrong. The case where Kyla simply has to
decide whether to answer the question now or come back to it later is in
some ways a relatively easy case. The really hard decisions are about
how much time to allocate to solving various mathematical problems, when
there are both costs to spending time, and rewards to solving as many
problems as possible. These can often be important decisions, and ones
that our theory should have something to say about. But orthodoxy does
not have anything to say. It's time to look for something else.

\section{Dialectical Interludes}\label{dialectical-interludes}

\subsection{Interlude One - The Obvious
Answer}\label{interlude-one---the-obvious-answer}

By this time you might be expecting a relatively simple answer to this
question. The problem is that the orthodox theorist was focussing on the
wrong choice. We shouldn't focus on the choice to take this can of
chickpeas or that one, or to answer true or false to this question.
Rather, we should focus on the choice to choose one procedure or
another. And the rational chooser will choose the procedure that is on
average best.

That solves our cases quite well. The best procedure for Frankie Lee or
David to adopt is to choose arbitrarily. Any other procedure will take
time, and it's not going to be time well spent. The best procedure for
the person wracked by regret at choices they didn't make is also to
choose somewhat arbitrarily, before the regrets have time to embed.
Conversely, the best procedure for Kyla is to skip any questions that
she can't do quickly, and come back to them if it turns out she has
time.

Given some very weak assumptions, \emph{Maximise expected utility} will
not be an optimal procedure in this sense. Actually it's ambiguous what
it means to say someone should adopt the procedure \emph{Maximise
expected utility}, but however you disambiguate that, it's wrong. The
procedure \emph{Calculate what maximises expected utility then choose
it} is not optimal, because the calculations may not be worth the
effort. The procedure \emph{Instinctively choose what maximises expected
utility} is a very efficient procedure if it is available, but for most
agents it isn't available. We should no more criticise agents who don't
adopt it than we criticise agents who don't get to work by apparating.

I'm going to adopt a version of the view that the rational choice is the
outcome of an optimal procedure. But I'm not going to adopt the most
obvious version of this obvious answer. In particular, I'm not going to
say that agents should adopt the procedure such that adopting that
procedure maximises expected value. Rather, I think, they should adopt
the procedure that maximises something like average value. We'll return
to this in a bit, but first I want to clear up some other dialectical
points.

\subsection{Dialectical Point Two - Possible Orthodox
Solutions}\label{dialectical-point-two---possible-orthodox-solutions}

There are ways of tinkering with orthodoxy to avoid some of the problems
that I raised in the previous section. For example, dropping the
constraint that credences are probabilities would avoid giving the wrong
answer in Kyla's case. And maybe, just maybe, there is a theory of
evidence, or of evidential support, such that the evidential expected
utility of each of Frankie Lee's choices are not distinct. I'm certainly
not going to try to go through every possible theory of evidence, or of
evidential support, to show that this isn't the case.

But I do want to note three constraints on an orthodox solution to the
problems that I have raised.

First, the solution must handle all the cases. This is not a completely
trivial point. The reason orthodoxy fails in the three cases is a little
different in each case. There is not, at least as far as I can tell, a
simple way to handle all of them simultaneously while staying roughly
within orthodoxy.

Second, the solution must not introduce any more complications of its
own. For example, you could try to solve some of the problems by saying
that the decision maker's evidence includes just those facts that are
immediately available to her. Perhaps there is some sense of `immediacy'
in which this provides the start of a solution to the first two puzzles.
(I think the third puzzle won't be solved this way, but the first two
might.) But this solution introduces problems of its own. For example, a
decision can be irrational in virtue of the fact that a moment's thought
would've revealed to the decision maker that it will lead to disaster.
If we restrict evidence to what is available at less than a moment's
thought, then we get this case wrong. If we don't put such a restriction
in place, then we're back to having problems with the first two puzzles
I mentioned above. I don't want to clean there is nothing for the
orthodox theorist to do here, but I do think it will be tricky to handle
the puzzles without licencing irrational thoughtlessness.

Third, any orthodox solution should be just as simple and as well
motivated as the obvious answer I just discussed. Saying that we should
focus on procedures and not on the choices they lead to on an occasion
resolves all of these puzzles in a simple and natural way. Even if an
orthodox solution can be found to all three puzzles, if it requires
three different changes to the orthodox view, it's hard to believe that
it will be preferable to a simple solution in terms of procedures.

\subsection{Dialectical Point Three -
Terminology}\label{dialectical-point-three---terminology}

At this point, some people might want to simply stipulate that the word
``rational'' picks out the choice that a computationally ideal actor
would take. Even if it's good in some sense for David to choose
arbitrarily, there is still an ideal can to choose, and he only deserves
the honorific rational if he chooses it.

I am not going to get into a fight over terminology here. If people want
to continue inquiring into what David would ideally do, then I'm not
going to get in their way. But I found this inquiry unmotivated for
three reasons. First, if we're going to consider what David would
ideally do, then I'm more interested in what he'd do if he were really
ideal, and knew everything. I don't see the appeal of investigating what
he would do given one, but only one, kind of idealisation. Second, I
don't think the ideal is a particularly good guide. Knowing what the
gods do doesn't help the mortals, for mortals just get burned if they
try to be like gods.

But the biggest reason concerns a purpose that I think is a central
function of the concept of rationality. We have a need to make the
people around us intelligible and predictable. And the best way we have
to do this is to understand the constraints and the motivations of
people around us, and feed those into a theory of rational choice that
outputs a decision given constraints and motivations. It doesn't always
work, especially if you are trying to make predictions. But it beats
most of the alternatives by a comfortable margin.

If that's the reason for having a theory of rational choice, then the
orthodox theory is not fit for purpose. The person who stands in the
grocery store aisle deliberating over which can to get is neither
intelligible nor predictable. The theory that says rational agents adopt
procedures that do well on average, given their constraints and
motivations, does make the ordinary behavior of supermarket shoppers
intelligible and predictable.

When I say `we' need to make folks around us intelligible and
predictable, I mean this to work at two different levels. From a very
early age, we do this kind of reasoning about particular individuals to
learn about the world (\citeproc{ref-ScottBaillargeon2013}{Scott and
Baillargeon 2013}). If a child sees a competent seeming adult use a
particular method to solve a problem, and the adult does not seem to
have any constraints that the child is free of, the child will copy what
the adult does (\citeproc{ref-LevyAlfano2019}{Levy and Alfano 2020}).
This makes perfect sense; the adult is rational (and better informed
than the child), so probably the adult's procedure is optimal for the
child. If we know that children do this, we can exploit it to trick
them. For example, we can demonstrate sub-optimal procedures, and
children will mimic them for a surprisingly long time. But this isn't
because the child is a fool; it's because they have a clever way of
learning about the world that can misfire when people set out to
confound it.

But I also mean this to work at the level of social analysis. The whole
point of game theoretic explanations of social phenomena is that we can
make a pattern of behavior intelligible by simply presenting the
constraints and motivations of the choosers, and then showing how
rational behavior on everyone's part produces the outcome. The research
program this paper is a part of is motivated by the hope, and it is a
hope more than a theory, that the same theory of rationality can serve
both the child who is selectively imitating those around them, and the
social scientist with their game theoretic models. Whether that's true
in general or not, I think both the child and the theorist are better
served by a theory of rational choice that is sensitive to computational
limitations and deliberation costs. And it's their perspectives that I'm
most interested in when theorising about rationality.

There is one other terminological dispute that I have no interest in
entering into, but I need to make explicit in order to set aside. Some
philosophers use `decision theory' to refer to the study of purely
procedural aspects of rationality. On this picture, there are three
parts to rational choice: epistemology, axiology and decision theory. A
rational agent will comply with all three. Compliance with the first is
manifest in rational credences. Compliance with the second is manifest
in rational values. And compliance with the third is manifest in choices
that are rational given the first two. I don't much care for this highly
factorised model of rational choice theory. Imagine we see someone
punching themselves in the head, and ask why they are doing this. If
they say, ``I want to bring about world peace, and I believe this is the
best way to do it'', we don't reply, ``Well, I guess two out of three
isn't bad''. We just think they are irrational. But for current purposes
I don't want to debate this. This paper is about the theory of rational
choice. If you think that encompasses more than decision theory, that it
also includes epistemology and axiology, then this isn't a paper in
decision theory strictly speaking. But even someone who thinks the
theory of rational choice can be factorised in this way still thinks
there is a theory of rational choice. And my plan here is to offer a
rival theory. Whether what I offer is a rival theory of \emph{decision}
turns on terminological questions about `decision theory' that I'm
hereby setting aside.

\section{History and Regresses}\label{history-and-regresses}

The idea that rational people are sensitive to their own computational
limitations has a long history. It is often traced back to a footnote of
Frank Knight's. Here is the text that provides the context for the note.

\begin{quote}
Let us take Marshall's example of a boy gathering and eating berries
\ldots{} We can hardly suppose that the boy goes through such mental
operations as drawing curves or making estimates of utility and
disutility scales. What he does, in so far as he deliberates between the
alternatives at all*, is to consider together with reference to
successive amounts of his ``commodity,'' the utility of each increment
against its ``cost in effort,'' and evaluate the net result as either
positive or negative (\citeproc{ref-Knight1921}{Knight 1921, 66--67})
\end{quote}

And the footnote attached to `at all' says this

\begin{quote}
Which, to be sure, is not very far. Nor is this any criticism of the
boy. Quite the contrary! It is evident that the rational thing to do is
to be irrational, where deliberation and estimation cost more than they
are worth. That this is very often true, and that men still oftener
(perhaps) behave as if it were, does not vitiate economic reasoning to
the extent that might be supposed. For these irrationalities (whether
rational or irrational!) tend to offset each other.
(\citeproc{ref-Knight1921}{Knight 1921, 67fn1})
\end{quote}

Knight doesn't really give an argument for the claim that these effects
will offset. And as John Conlisk (\citeproc{ref-Conlisk1996}{1996})
shows in his fantastic survey of the late 20th century literature on
bounded rationality, it very often isn't true. Especially in game
theoretic contexts, the thought that other players might think that
``deliberation and estimation cost more than they are worth'' can have
striking consequences. But our aim here is not to think about economic
theorising, but about the nature of rationality.

There is something paradoxical, almost incoherent, about Knight's
formulation. If it is ``rational to be irrational'', then being
``irrational'' can't really be irrational. There are two natural ways to
get out of this paradox. One, loosely following David Christensen
(\citeproc{ref-Christensen2007}{2007}), would be to say that ``Murphy's
Law'' applies here. Whatever one does will be irrational in some sense.
But still some actions are less irrational than others, and the least
irrational will be to decline to engage in deliberation that costs more
than it is worth. I suspect what Knight had in mind though was something
different (if not obviously better). He is using `rational' as more or
less a rigid designator of the the property of choosing as a Marshallian
maximizer does. And what he means here is that the disposition to not
choose in that way will be, in the long run, the disposition with
maximal returns.

This latter idea is what motivates the thought that rational agents will
take what John Conlisk calls ``deliberation costs'' into account. And
Conlisk thinks that this is what rational agents will do. But he also
raises a problem for this view, and indeed offers one of the clearest
(and most widely cited) statements of this problem.

\begin{quote}
However, we quickly collide with a perplexing obstacle. Suppose that we
first formulate a decision problem as a conventional optimization based
on the assumption of unbounded rationality and thus on the assumption of
zero deliberation cost. Suppose we then recognize that deliberation cost
is positive; so we fold this further cost into the original problem. The
difficulty is that the augmented optimization problem will itself be
costly to analyze; and this new deliberation cost will be neglected. We
can then formulate a third problem which includes the cost of solving
the second, and then a fourth problem, and so on. We quickly find
ourselves in an infinite and seemingly intractable regress. In rough
notation, let \emph{P} denote the initial problem, and let \emph{F}(.)
denote the operation of folding deliberation cost into a problem. Then
the regress of problems is \emph{P}, \emph{F}(\emph{P}),
\emph{F}\textsuperscript{2}(\emph{P}), \ldots{}
(\citeproc{ref-Conlisk1996}{Conlisk 1996, 687})
\end{quote}

Conlisk's own solution to this problem is not particularly satisfying.
He notes that once we get to \emph{F}\textsuperscript{3} and
\emph{F}\textsuperscript{4}, the problems are `overly convoluted' and
seem to be safely ignored. This isn't enough for two reasons. First,
even a problem that is convoluted to state can have serious consequences
when we think about solving it. (What would \emph{Econometrica} publish
if this weren't true?) Second, as is often noted,
\emph{F}\textsuperscript{2}(\emph{P}) might be a harder problem to solve
than \emph{P}, so simply stopping the regress there and treating the
rational agent as solving this problem seems to be an unmotivated
choice.

As Conlisk notes, this problem has a long history, and is often used to
dismiss the idea that folding deliberation costs into our model of the
optimising agent is a good idea. I use `dismiss' advisedly here. As he
also notes, there is very little \emph{discussion} of this infinite
regress problem in the literature before 1996. The same remains true
after 1996. What is done is that instead people appeal to the regress in
a sentence or two to set aside approaches that incorporate deliberation
cost in the way that Conlisk suggests.

Up to around the time of Conlisk's article, the infinite regress problem
was often appealed to by people arguing that we should, in effect,
ignore deliberation costs. After his article, the appeals to the regress
comes from a different direction. It is usually from theorists arguing
that deliberation costs are real, but the regress means it will be
impossible to consistently incorporate them into a model of an
optimizing agent. So we should instead rely on experimental techniques
to see how people actually handle deliberation costs; the theory of
optimization has reached its limit. This kind of move is found in
writers as diverse as Gigerenzer and Selten
(\citeproc{ref-GigerenzerSelton2001}{2001}), Odell
(\citeproc{ref-Odell2002}{2002}), Pingle
(\citeproc{ref-Pingle2006}{2006}), Mangan, Hughes, and Slack
(\citeproc{ref-ManganEtAl2010}{2010}), Ogaki and Tanaka
(\citeproc{ref-OgakiTanaka2017}{2017}) and Chakravarti
(\citeproc{ref-Chakravarti2017}{2017}). And proponents of taking
deliberation costs seriously within broadly optimizing approaches, like
Miles Kimball (\citeproc{ref-Kimball2015}{2015}), say that solving the
regress problem is the biggest barrier to having such an approach taken
seriously by economists. So let's turn to how we might solve it.

\section{Four Non-Solutions}\label{four-non-solutions}

My solution, as I've mentioned a couple of times, is a form of
reliabilism. The rational choice is the one that would be produced by
using the procedure that does best on average. That procedure will just
be maximising expected utility when computational costs are zero, and
will involve appeal to expected utility maximisation in many other
cases. But it won't, in general, simply be expected utility
maximisation.

To get clearer on what the reliabilist solution is, and how it is
motivated, I want to first go through three other solutions that I don't
think work.

First, we could just say that the rational choice is simply the choice
that produces the best actual result. This gets some cases intutively
wrong; it says that it is never rational to leave a casino without
gambling. And it eliminates a type of choice that we think is real: the
lucky guess. We want lucky guesses to be cases that produce good
outcomes, but are not rational. If the rational choice just is the best
choice, this is impossible. Since lucky guesses are not impossible, this
theory can't be right.

Second, we could say that the rational choice is the choice that
maximises expected value. But I've already gone over why that is wrong.
There are really two things we could mean by saying the rational choice
is the one that maximises expected value, and both of them are wrong. We
could say that the rational choice is to compute what has the highest
expected value, and then choose it. But this gives the wrong result in
all the cases that I discussed at the beginning. Or we could say that
the rational choice is to insinctively pick the one with the highest
expected value. But there is no more reason to think this is something
that choosers can do than there is to think that choosers can
instinctively pick the choice with the highest actual value. So this is
unrealistic.

Third, we could say the choice is the output of the procedure such that
adopting that procedure maximises expected value, given one's evidence
about the world and about the nature of procedures. Here, I think, the
regress has bite because the same arguments from the previous paragraph
still apply. We really need to distinguish two possible things we might
mean by saying that one should adopt the procedure such that adopting it
maximises expected value. First, we could mean that choosers should
compute which procedure will maximise expected value, and then adopt it.
But this will get the wrong result in Frankie Lee's case, and in
David's. They shouldn't be doing any computation at all. So perhaps
instead we could say that the rational chooser will instinctively choose
the procedure with the highest expected value. But there is no more
reason to think that choosers could always do that there is to think
that they can simply choose the first-order option with the highest
actual or expected value. So this idea fails, and it fails for just the
same reasons as the suggestion of the previous paragraph. That doesn't
prove a regress is looming, but it doesn't look good.

The fourth option I'll discuss is designed to avoid this problem, and it
is going to look somewhat more promising. Maybe we can't all at once
choose the best procedure. But we can do it piecemeal.

In general, here's a way to adopt a complicated procedure. When faced
with a certain class of problems, adopt the simplest procedure that
agree with the complicated procedure over the range of choices you face.
Then, as the problems expand, start either complicating the procedure,
or adopt a meta-procedure for choosing which simple procedure to adopt
on an occasion. Over time, if all goes well, you'll eventually adopt
something like the complicated procedure, and do it without having to
solve impossibly hard calculations about procedural effectiveness, or
having miraculously good instincts.

One appeal of this approach is that it blocks the regress. If one
selects a procedure piecemeal in this way, there is a good sense in
which
\emph{F}(\emph{P})~=~\emph{F}\textsuperscript{2}(\emph{P})~=~\emph{F}\textsuperscript{3}(\emph{P})~=~\ldots.
After all, there won't be a difference between adopting a procedure, and
adopting a procedure for adopting that procedure. Both of them will just
involve making the choices you have to make on a given day, and looking
for the opportunity to integrate those choices into a larger and more
systematic theory. By adopting a first-order procedure piecemeal, you
also adopt a second-order procedure piecemeal. And if
\emph{F}(\emph{P})~=~\emph{F}\textsuperscript{2}(\emph{P})~=~\emph{F}\textsuperscript{3}(\emph{P})~=~\ldots,
then the regress doesn't get going.

The problem is that this is too demanding. We want choosers to maximise.
We don't expect them to be able to maximise over every possible choice
situation, just over the one in front of them. If I'm buying chickpeas,
and I arbitrarily choose one of the cans, that's all to the good. It's a
rational choice. And, crucially, it stays being a rational choice even
if I have dispositions to choose badly in other choice situations. But
on the `piecemeal' model being considered here, those dispositions to
choose badly are partially constitutive of my choice procedure. And
rational choice is a matter of choosing in virtue of adopting the
correct choice procedure. So someone who is irrational somewhere is, it
turns out, irrational everywhere. This is a bad result. There is
something right about the idea that the rational chooser will just
choose what's in front of them, and do so in a sensible way. But we
shouldn't go on to say that rational choice requires that the global
procedure one thereby implicitly adopts is the right one; that's too
high a bar.

\section{Skilled Choice}\label{skilled-choice}

The way to see what's right about the last proposal, and to see our way
to the correct solution, is to somewhat reconceptualise rational choice.
We shoudl conceive of the rational chooser as a skilled chooser. And we
should think skills are a matter of reliably doing well across realistic
situations.

The justification for conceiving of rational choice as skilled choice is
largely pragmatic. Thinking of rationality that ways results in a
plausible theory of rationality, and other ways of thinking about
rationality resulted in implausible theories. So rather than argue for
the conception of rationality as skill, I'm going to more or less assume
it, and hope to justify this assumption by its fruitfulness. What I will
argue for is the idea that skill involves reliably succeeding across
realistic situations.\footnote{This is very similar to the modal
  understanding of skill in Beddor and Pavese
  (\citeproc{ref-BeddorPavese2019}{2020}).}

Think for a bit about skilled athletes, or skilled players of chess or
other games. Part of being skilled is succeeding. But it isn't just
about success. Some people win due to luck. The skilled player won't
always win, but they will reliably win across a range of situations.

Which situations are those? They are the situations that are normal
enough for the kind of activity being engaged in. These might be
dependent on highly contingent features of the activity. A chess player
who wins international tournaments must be very skilled. We wouldn't
retract that assessment if it turned out they only played well in quiet
environments, and frequently lost chess games in noisy pubs. High level
chess is played in quiet environments, so that's what matters.

A football player whose instincts only go right when there is no wind
around is not particularly skilled. Someone who doesn't know how to
adjust their passes when the wind changes is not skilled; it is lucky
that they get ever connect on a pass. Conversely, a football player
whose instincts are finely calibrated to the actual gravitational field
strength around here could be highly skilled. It's not part of
footballing skill that one is able to adjust to changes in gravitational
field strength. Some kinds of flexibility, such as ability to adjust to
wind conditions, matter, while others, such as ability to adjust to a
different gravitational field, do not. There are intermediate cases
where the importance of the ability to adjust is dependent on contingent
attributes of the activity. Top level Australian Rules Football is
almost always played at sea level. An Australian Rules Footballer whose
instincts are calibrated for play at sea level, and who has no ability
to adjust to changes in altitude, might still be highly skilled. But in
a sporting competition where top flight games are frequently played in
Mexico City or Quito, an inability to adjust to changing altitudes is a
substantial limitation on one's skill. It is luck, not skill, that
causes one to succeed in contests at one's favoured altitude. But it
isn't luck that the Australian Rules Footballer is playing at sea level;
that's a stable generalisation about the sport.

The same kind of story holds true for the skilled chooser. They have to
do well, and not by chance. But that can involve having instincts that
are calibrated to the environment one is actually in, and which would
misfire in other environments. A skilled supermarket shopper need not be
applying procedures that would do well in a medieval market. But they
must be applying procedures that will keep working if the shelving of
various items is changed.

That's to say, the skilled chooser will adopt a procedure that will, on
average, produce the best results in circumstances like the ones they
are in.There is an implicit notion of probability in that definition.
But it isn't the notion of credence, or even of rational credence.
Rather, it is the notion of how likely it is, or how frequent it is,
that different circumstances obtain. That's the sense in which the
theory is reliabilist.

When I say `produce the best results', I mean the best results of the
available procedures. Just like we don't require rational commuters to
apparate, we don't require rational choosers to instinctively maximise
utility, or expected utility. They (just) have to do the best they can.

Skilled action frequently involves doing things where one has no
evidence for the utility of such performances. It can even involve doing
things where one has evidence against the utility of what one is doing.
To see this, imagine a junior athlete who is thriving against
competitors their own age with an unusual technique. They are told, by
seemingly trustworthy coaches, that to thrive at higher levels, they
have to adopt a more orthodox technique. But though they have reason to
believe these coaches, they keep instinctively lapsing back into their
unusual techniques. And, amazingly, the coaches are wrong, and what
looked like a technique for winning against kids in parks ends up
working at international level competition. (This isn't entirely unlike
the story of Australian cricketer Steve Smith.) Such an athlete may be
highly skilled. And their skill consists in, among other things, their
instincts to do things that they have (misleading) evidence will not
work. Their skill, that is, involves deploying a procedure that is
actually reliable, even after they get evidence it is unreliable. I
think the same is true of skilled choice. Sometimes, the skilled chooser
will deploy a technique that they think is defective, and even one that
they think is defective on reasonable grounds. As long as it works, it
can still be the basis for skilled, and hence rational, choosing.

\section{Regress Blocking}\label{regress-blocking}

With all that in place, let's return to the regress problem, and in
particular to Conlisk's statement of it. Why should we think the
rational agent solves \emph{F}(\emph{P}), and not
\emph{F\textsuperscript{n}}(\emph{P}) for some \emph{n}~\textgreater~1?
I want to say that's just what rational choice is; it's skillfully
managing one's own computational and informational limitations. And
skill in this sense involves getting it right, and doing so reliably,
not necessarily thinking through the problem. This suggests two
questions.

\begin{enumerate}
\def\labelenumi{\arabic{enumi}.}
\tightlist
\item
  Why should we allow this kind of unreflective rule-following in our
  solution to the regress?
\item
  Why should we think that \emph{F}(\emph{P}) is the point where this
  consideration kicks in, as opposed to \emph{P}, or anything else?
\end{enumerate}

There are a few ways to answer 1. One motivation traces back to the work
by the artificial intelligence researcher Stuart Russell
(\citeproc{ref-Russell1997}{1997}). (Although really it starts with the
philosophers Russell cites as inspiration, such as Cherniak
(\citeproc{ref-Cherniak1986}{1986}) and Harman
(\citeproc{ref-Harman1973}{1973}).) He stresses that we should think
about the problem from the outside, as it were, not from inside the
agent's perspective. How would we program a machine that we knew would
have to face the world with various limitations? We will give it rules
to follow, but we won't necessarily give it the desire (or even the
capacity) to follow those rules self-consciously. That might be useful
some of the time - though really what's more useful is knowing the
limitations of the rules. And that can be done without following the
rules as such. It just requires good dispositions to complicate the
rules one is following in cases where such complication will be
justified.

Another motivation is right there in the quote from Knight that set this
literature going. Most writers quote the footnote, where Knight suggests
it might be rational to be irrational. But look back at what he's saying
in the text. The point is that it can be perfectly rational to use
considerations other than drawing curves and making utility scales. What
one has to do is follow internal rules that (non-accidentally) track
what one would do if one was a self-consciously perfect Marshallian
agent. That's what I'm saying too, though I'm saying it one level up.

Finally, there is the simple point that on pain of regress any set of
rules whatsoever must say that there are some rules that are simply
followed. This is one of the less controversial conclusions of the
debates about rule-following that were started by Wittgenstein
(\citeproc{ref-Wittgenstein1953}{1953}). That we must at some stage
simply follow rules, not follow them in virtue of following another
rule, say the rule to compute how to follow the first rule and act
accordingly, is an inevitable consequence of thinking that finite
creatures can be rule followers.

So question 1 is not really a big problem. But question 2 is more
serious. Why \emph{F}(\emph{P}), and why not something else? The short
answer will be that any reason to think that rational actors maximize
\emph{expected} utility, as opposed to \emph{actual} utility, will also
be a reason to think that they solve \emph{F}(\emph{P}) and not
\emph{P}.

Start by stepping back and thinking about why we cared about expected
utility instead of actual utility in the first place. Why not just say
that the best thing to do is to produce the best outcome, and be done
with it? Well, we don't say that because we take it as a fixed point of
our inquiry that agents are informationally limited, and that the best
thing to do is what is best given that limitation. Given some plausible
assumptions, the best thing for the informationally limited agent to do
would be to maximize expected utility. This is a second-best option, but
the best is unavailable given the limitations that we are treating as
unavoidable.

But agents are not just informationally limited, they are
computationally limited too. And we could have instead treated that as
the core limitation to be modelled. As Conlisk says, it is
``entertaining to imagine'' theorists who worked in just this way,
taking the agents in their models to have computational but not
informational limitations (\citeproc{ref-Conlisk1996}{Conlisk 1996,
691}). Let's imagine that when we meet the Martian economists, that's
how they reason. Conlisk notes a few things that the Martian economists
might do. They might disparage their colleagues who take informational
limitations seriously as introducing ad hoc stipulations into their
theory. They might argue that informational limitations are bound to
cancel out, or be eliminated by competition. They might argue that
apparent informational limitations are really just computational ones,
or at least can be modelled as computational ones. And so on.

What he doesn't add is that they might suggest that there is a regress
worry for any attempt to add informational constraints. Let \emph{Q} be
the initial problem as the Martians see it. That is, \emph{Q} is the
problem of finding the best outcome given full knowledge of the
situation, but the actual computational limitations of the agent. Then
we suggest that we should also account for the informational
limitations. Let's see if this will work, they say. Let \emph{I}() be
the function that transforms a problem into one that is sensitive to the
informational limitations of the agent. But if we're really sensitive to
informational limitations, we should note that \emph{I}(\emph{Q}) is
also a problem the agent has to solve under conditions of less than full
information.\footnote{At this point the Martians might note that while
  they are grateful that Williamson
  (\citeproc{ref-Williamson2000}{2000}) has highlighted problems with
  the KK principle, and these problems show some of the reasons for
  wanting to idealise away from informational limitations, they aren't
  in fact relying on Williamson's work. All they need is that agents do
  not exactly what they know. And that will be true as long as the
  correct epistemic logic is weaker than S5. And that will be true as
  long as someone somewhere has a false belief. And it would just be
  weird, they think, to care about informational limitations but want to
  idealise away from the existence of false beliefs.} So the
informationally challenged agent will have to solve not just
\emph{I}(\emph{Q}), but \emph{I}\textsuperscript{2}(\emph{Q}), and
\emph{I}\textsuperscript{3}(\emph{Q}) and so on.\footnote{At this point,
  some of the Martians note that the existence of Elster
  (\citeproc{ref-Elster1979}{1979}) restored their faith in humanity.}

Orthodox defenders of (human versions of) rational choice theory have to
think this is a bad argument. And I think most of them will agree with
roughly the solution I'm adopting. The right problem to solve is
\emph{I}(\emph{Q}), on a model where \emph{Q} is in fact the problem of
choosing the objectively best option. If one doesn't know precisely what
one's knowledge is, then one has to maximize expected utility somewhat
speculatively. But that doesn't mean that one shouldn't maximize
expected utility.

But the bigger thing to say is that neither we nor the Martians really
started with the right original problem. The original problem, \emph{O},
is the problem of choosing the objectively best option. The humans start
by considering the problem \emph{I}(\emph{O}), i.e., \emph{P}, and then
debate whether we should stick with that problem, or move to
\emph{F}(\emph{I}(\emph{O})). The Martians start by considering the
problem \emph{F}(\emph{O}), i.e., \emph{Q}, then debate whether we
should stick with that or move to \emph{I}(F(\emph{O})). And the answer
in both cases is that we should move.

Given the plausible commutativity principle, that introducing two
limitations to theorising has the same effect whichever order we
introduce them,
\emph{I}(\emph{F}(\emph{O}))~=~\emph{F}(\emph{I}(\emph{O})). That is,
\emph{F}(\emph{P})~=~\emph{I}(\emph{Q}). And that's the problem that we
should think the rational agent is solving.

But why solve that, rather than something more or less close to
\emph{O}? Well, think about what we say about an agent in a Jackson case
who tries to solve \emph{O} not \emph{I}(\emph{O}). (A Jackson case, in
this sense, is a case where the choice with highest expected value is
known to not have the highest objective value. So trying to get the
highest objective value will mean definitely not maximizing expected
value.) We think it will be sheer luck if they succeed. We think in the
long run they will almost certainly do worse than if they tried to solve
\emph{I}(\emph{O}). And in the rare case where they do better, we think
it isn't a credit to them, but to their luck. In cases where the
well-being of others is involved, we think aiming for the solution to
\emph{O} involves needless, and often immoral, risk-taking.

The Martians can quite rightly say the same things about why
\emph{F}(\emph{O}) is a more theoretically interesting problem than
\emph{O}. Assume we are in a situation where \emph{F}(\emph{O}) is known
to differ from \emph{O}, such as the case Kyla was in. Or, for a
different example, imagine the decision maker will get a reward if they
announce the correct answer to whether a particular sentence is a
truth-functional tautology, and they are allowed to pay a small fee to
use a computer that can decide whether any given sentence is a
tautology. The solution to \emph{O} is to announce the correct answer,
whatever it is. The solution to \emph{F}(\emph{O}) is to pay to use the
computer. And the Martians might point out that in the long run, solving
\emph{F}(\emph{O}) will yield better results. That if the agent does
solve problems like \emph{O} correctly, even in the long run, this will
just mean they were lucky not rational. That if the reward is that a
third party does not suffer, then it is immorally reckless to not solve
\emph{F}(\emph{O}), i.e., to not consult the computer. And in general,
whatever we can say that motivated ``Rational Choice Theory'', as
opposed to ``Choose the Best Choice Theory'', they can say too.

Both the human and the Martian arguments look good to me. We should add
in both computational and informational limitations into our model of
the ideal agent. And that's the solution to the regress. It is
legitimate to think that there is a rule that rational creatures follow
immediately, on pain of thinking that all theories of rationality imply
regresses. And thinking about the contingency of how Rational Choice
Theory got to be the way it is suggests that the solution to what
Conlisk calls \emph{F}(\emph{P}), or what I've called
\emph{F}(\emph{I}(\emph{O})), will be that point.

\section{The Nature of Good
Procedures}\label{the-nature-of-good-procedures}

Since this is meant to be a theory of rational choice for real people,
it would be helpful to say a few words about what these reliable
procedures that stop the regress might be. In principle they could be
anything, but in practice I think three kinds of procedures are
particularly important: instincts, planning, and modelling. I'll say a
bit about each of these in turn.

Humans are surprisingly good at instinctively allocating reasonable
amounts of cognitive resources to computational tasks. In artificial
intelligence research, one of the big challenges is trying to make
machines be as good as humans at figuring out which problems to allocate
cognitive resources to. This is sometimes known as the frame problem.
Here's a typical description of this from a recent survey article.

\begin{quote}
And, more generally, how do we account for our apparent ability to make
decisions on the basis only of what is relevant to an ongoing situation
without having explicitly to consider all that is not relevant?
(\citeproc{ref-Shanahan2016}{Shanahan 2016})
\end{quote}

Note that this assumes is that humans are actually very good at this
rather hard task - setting aside the irrelevant without first thinking
that it is irrelevant. This has to be instinctive. We don't go around
thinking about how much time to spend thinking on various subjects. That
would be self-defeating. Obviously we are far from perfect at this, but
it is striking how good we are at it.

Recent work on `vigilance' has illustrated how good we are at one aspect
of this problem (\citeproc{ref-SperberEtAl2010}{Sperber et al. 2010}).
Somehow, and I don't think it is clear how, we manage to keep track of
our environment in a comprehensive enough away that it allows us to
focus on those things that need focusing on. For example, when walking
down a busy street, we don't make a model of the expected movements of
each of the individuals around us. That would be too computationally
taxing. But we do pay enough attention to each of those individuals for
us to be able to focus on any one of them if they seem to pose a
particular challenge or threat. If one of them is weaving in a drunken
manner, or carrying a sword, we are able to focus on them very quickly.
To do this we must be paying at least background attention to every one
of them. I think this turns out to be a common phenomenon. There are
many situations where we don't have the ability to carefully consider
everything that's going on, but we do manage to pick out the things
around us that need close attention. And that requires monitoring of the
entire environment, and doing some very quick and dirty processing of
the resulting inputs. As I said, it's a bit of a mystery how we do this.
But whatever we do, it's an amazing feat of insinctively solving a
cognitive resource allocation problem.

When I say we do some of these things instinctively, I don't mean that
our ability to do them is innate. We might pick them up by learning from
those around us. This learning need not be conscious. It might happen by
imitation. It is sometimes thought that humans' disposition to over
imitate those around them is a kind of irrationality
(\citeproc{ref-LevyAlfano2019}{Levy and Alfano 2020}). But my guess is
that it is part of what grounds our skill in solving these hard
cognitive resource allocation problems.

But rather than speculate further about what future research will show
about the range and limits of human instinct, let's turn to two ways of
consciously adopting reliable procedures. In his discussion of the
regress, Miles Kimball (\citeproc{ref-Kimball2015}{2015}) suggests a few
options that might work. I want to focus on two of them: planning and
modelling.

\begin{quote}
Least transgressive are models in which an agent sits down once in a
long while to think very carefully about how carefully to think about
decisions of a frequently encountered type. For example, it is not
impossible that someone might spend one afternoon considering how much
time to spend on each of many grocery-shopping trips in comparison
shopping. In this type of modelling, the infrequent computations of how
carefully to think about repeated types of decisions could be
approximated as if there were no computational cost, even though the
context of the problem implies that those computational costs are
strictly positive. (\citeproc{ref-Kimball2015}{Kimball 2015, 174})
\end{quote}

And that's obviously relevant to David in the supermarket. He could, in
principle, spend one Saturday afternoon thinking about how carefully to
check each of the items in the supermarket before putting it in his
shopping cart. And then in future trips, he could just carry out this
plan. In general, planning as a device for incurring computational costs
at a time when those costs are less costly.

This isn't a terrible strategy, but I suspect it's rarely optimal. For
one thing, there are much better things to do with Saturday afternoons.
For another, it suggests we are back in the business of equating solving
\emph{F}(\emph{P}) with approximately solving \emph{P}. And that's a
mistake. Better to just say that David is rational if he just does the
things that he would do were he to waste a Saturday afternoon this way,
and then plan it out. And that thought leads to Kimball's more radical
suggestion for how to avoid the regress,

\begin{quote}
{[}M{]}odelling economic actors as doing constrained optimization in
relation to a simpler economic model than the model treated as true in
the analysis. This simpler economic model treated as true by the agent
can be called a ``folk theory'' (\citeproc{ref-Kimball2015}{Kimball
2015, 175})
\end{quote}

It's this last idea I plan to explore in more detail. (It has some
similarities to the discussion of small worlds in Joyce
(\citeproc{ref-Joyce1999}{1999}) 70-77.) The short version is that David
can, and should, have a little toy model of the supermarket in his head,
and should optimize relative to that model. The model will be false, and
David will know it is false. And that won't matter, as long as David
treats the model the right way.

There are a lot of things that could have gone wrong with a can of
chickpeas. They could have gone bad inside the can. They could have been
contaminated, either deliberately or through carelessness. They could
have been sitting around so long they have expired. All these things
are, at least logically, possible.

But these possibilities, while serious, have two quite distinctive
features. One is that they are very rare. In some cases they may have
never happened. (I've never heard of someone deliberately contaminating
canned chickpeas, though other grocery products like strawberries have
been contamination targets.) The other is that there are few easy ways
to tell whether they are actualised. You can scan each of the cans for
an expiry date, but it is really uncommon that this is relevant, and it
takes work since the expiry dates are normally written in such small
type. If a can is really badly dented, I guess that weakens the metal
and raises ever so slightly the prospect of unintentional contamination.
But it's common to have shelves full of cans that have no dents, or at
most very minor ones.

Given these two facts - the rarity of the problems and the difficulty in
getting evidence that significantly shifts the probability that this is
one of the (rare) problems - the rational thing to do is choose in a way
that is insensitive to whether those problems are actualised. Or,
perhaps more cautiously, one should be vigilant, in the sense of Sperber
et al. (\citeproc{ref-SperberEtAl2010}{2010}), to some of these
problems, and ignore the rest. But being vigilant about a problem means,
I take it, being willing to consider it if and only if you get evidence
that it is worth considering. In the short run, you still ignore the
potential problem.

And to ignore a potential problem is to choose in a way that is
insensitive to evidence for the problem. That makes sense for both the
banknotes and the chickpeas, because engaging in a choice procedure that
is sensitive to the probability of the problem will, in the long run,
make you worse off.

In Kimball's terms, the rational shopper will have a toy model of the
supermarket in which all cans of chickpeas that aren't obviously damaged
are safe to eat. This will be a defeasible model, but on a typical
grocery trip, it won't be defeated. In Joyce's terms, the small worlds
the shopper uses in setting up the decision problem they face will all
be ones in which the chickpeas are safe.

So the suggestion is that very often, the way to be rational is to have
right model in your head, and apply it correctly. A choice is the
rational choice in your situation iff it is the recommendation of the
right model. And the right model includes just as much information, and
just as many complications, as the situation demands. The regress is
blocked, on this picture, because you don't have to have computed, or
even be in a position to compute, that the right model is the right
model. Here I am following Knight. Rational agents don't have to have
worked through Marshall's Principles; they just have to think and act as
if they had. But crucially, they don't have to even act as if they are
applying the Principles to the world. They could apply them to a good
model of the world, and that's good enough.

\section{Three Philosophical
Postscripts}\label{three-philosophical-postscripts}

\subsection{Idealisation}\label{idealisation}

The story I'm telling here about how rational agents use models is very
similar, and indeed draws heavily on, the story that Michael Strevens
(\citeproc{ref-Strevens2008}{2008}) tells about how scientists use
idealisations. On that story, to use an idealisation is to set some
messy value to a computationally more simple value (often 0 or 1), and
to (implicitly) assert that the difference between the actual value and
the computationally simpler value is irrelevant for current purposes.

One benefit Strevens gets from this is that he is spared saying that
scientists use falsehoods in their reasoning. After all, it is often
true that the difference between the messy value and the simple value is
irrelevant for current purposes - and that's all that the scientist is
committing to.

The same is true in this picture. Frankie Lee can't know that the
banknotes are all equally likely to be genuine; because that's not
strictly true. But he can know that the right model to use in his
current situation is one that sets the probability of any note's
genuineness to 1. That's both true - assuming that our picture of what
makes a model right is one that takes deliberation costs seriously - and
well supported by his evidence.

\subsection{Epistemic Luck}\label{luck}

On the story I'm telling, whether decision makers are rational or
irrational will often be a matter of luck. This is as you should expect
if rationality is a matter of successfully applying a skill. Most skills
are not infallible. Being skilled at an activity means one usually
succeeds, or at least one succeeds at a higher rate than is normal, but
on any given day one could fail. The epistemic failures I call
irrationality, even though the person in some sense does the same thing
as they do in cases where they act rationally.

Here is one version of that. Recall the version of the Frankie Lee
example where the country has just started modernising its financial
system by introducing plastic banknotes. Frankie Lee knows that plastic
banknotes are genuine - no one has figured out how to forge them yet. So
if some of them are on offer, he should take one. But, let's imagine,
he's temporarily forgotten this fact. So he takes one of the paper
notes. This is irrational.

But it's also bad luck. It's not normally required that we scour our
memories for any relevant information before making a decision like
this. Normally, Frankie Lee could have put in this much cognitive
effort, and ended up rational. But the world did not cooperate, and he
ended up irrational.

I think any story that connects rationality to succeeding via skill will
have the consequence that sometimes whether one is rational is in part a
matter of luck. But the possibility of epistemic luck shouldn't surprise
us. Assume that what one should, rationally, do and believe is a
function of what one knows. And assume that the right epistemic logic is
weaker than S5. Then one won't always know what one knows. So one won't
always know what one should do or believe. So if one believes what one
should, or does what one should, this will be in some sense a matter of
luck. And surely the right epistemic logic is weaker than S5. Even if
you think the anti-luminosity arguments are bad, and the right epistemic
logic is stronger than S4, you shouldn't think that people know what it
is they don't know. (False beliefs, for example, are typically pieces of
non-knowledge that are not known to be not knowledge.) So we shouldn't
be surprised that there is epistemic luck.

\subsection{Knowledge and Rational Choice}\label{knowledge}

So my preferred picture of rational action in cases where there are
deliberation costs is that the chooser has a model of the decision
problem in their head, and they know it is a good model. That's a
constraint on rationality, but it's also a constraint on knowledge. If
the chooser knows that \emph{p}, they can't be using a model where it
might be that ¬\emph{p}. That, I think, is the core way that practical
factors encroach on knowledge - sometimes one is in a practical
situation where the best model allows for the possibility of ¬\emph{p},
and being in such a situation defeats any putative knowledge that
\emph{p}.

But I used to say something different about how practical factors
affected knowledge. I used to say something like the following.

\begin{itemize}
\tightlist
\item
  One knows \emph{p} only if the rational choice (or choices)
  conditional on \emph{p} are the rational choice (or choices)
  unconditionally for any choice one is considering.
\item
  The rational choice, either conditional or unconditional, is the one
  with the highest expected utility, or if there are ties, then all of
  them are rational choices.
\end{itemize}

And it turns out that combination of views is untenable. This was shown
independently twice over, once by Alex Zweber
(\citeproc{ref-Zweber2016}{2016}) and then, separately, by Charity
Anderson and John Hawthorne -Anderson and Hawthorne
(\citeproc{ref-AndersonHawthorne2019a}{2019}). They considered
situations like the original Frankie Lee example, and noted that my view
had the implausible consequence that Frankie Lee did not know, of each
note, that it was genuine. After all, as it stands Frankie Lee should be
indifferent between the notes, but conditional on one of the notes being
genuine, he should prefer that one. And that's implausible. Both papers
go on to note other implausibilities that purportedly follow, but
already we should acknowledge this is a problem. (Whether my view was
really committed to the other implausibilities is something I could
argue about, but it doesn't matter because this is already a perfectly
good counterexample.)

The solution, I now think, is to qualify the second bullet point above.
What I should have said instead is

\begin{itemize}
\tightlist
\item
  The rational choice is the one with the highest expected utility on
  the model that the chooser is rationally using, or if there are ties,
  then all of them are rational choices.
\end{itemize}

And now the problem goes away. It is rational for Frankie Lee to use the
model where all the notes are genuine - it isn't worth the cost of using
a more complicated model. And on that model, conditionalising on the
hypothesis that one of the notes is genuine doesn't change anything. So
if Frankie Lee is using that model, he knows the notes are genuine. If
he isn't using that model then he doesn't know the notes are genuine.
But this isn't because of any pragmatic theory of knowledge - it's
simply that to know \emph{p} requires one actually take \emph{p} as
given, and Frankie Lee fails that criteria.

So cases like Frankie Lee, or David and the chickpeas, are perfectly
good counterexamples to the version of epistemic pragmatic encroachment
I used to endorse. But they don't show that pragmatic theories are false
in general; they just show I got an important detail wrong. To get these
details right we need a better theory of when people (rationally) ignore
the details.

\subsection*{References}\label{references}
\addcontentsline{toc}{subsection}{References}

\phantomsection\label{refs}
\begin{CSLReferences}{1}{0}
\bibitem[\citeproctext]{ref-AndersonHawthorne2019a}
Anderson, Charity, and John Hawthorne. 2019. {``Knowledge, Practical
Adequacy, and Stakes.''} \emph{Oxford Studies in Epistemology} 6:
234--57.

\bibitem[\citeproctext]{ref-BeddorPavese2019}
Beddor, Bob, and Carlotta Pavese. 2020. {``Modal Virtue Epistemology.''}
\emph{Philosophy and Phenomenological Research} 101 (1): 61--79. doi:
\href{https://doi.org/10.1111/phpr.12562}{10.1111/phpr.12562}.

\bibitem[\citeproctext]{ref-Caplin2011}
Caplin, Andrew, Mark Dean, and Daniel Martin. 2011. {``Search and
Satisficing.''} \emph{American Economic Review} 101 (7): 2899--2922.
doi:
\href{https://doi.org/10.1257/aer.101.7.2899}{10.1257/aer.101.7.2899}.

\bibitem[\citeproctext]{ref-Chakravarti2017}
Chakravarti, Ashok. 2017. {``Imperfect Information and Opportunism.''}
\emph{Journal of Economic Issues} 51 (4): 1114--36. doi:
\href{https://doi.org/10.1080/00213624.2017.1391594}{10.1080/00213624.2017.1391594}.

\bibitem[\citeproctext]{ref-ChernevEtAl2015}
Chernev, Alexander, Ulf Böckenholt, and Joseph Goodman. 2015. {``Choice
Overload: A Conceptual Review and Meta-Analysis.''} \emph{Journal of
Consumer Psychology} 25 (2): 333--58. doi:
\href{https://doi.org/10.1016/j.jcps.2014.08.002}{10.1016/j.jcps.2014.08.002}.

\bibitem[\citeproctext]{ref-Cherniak1986}
Cherniak, Christopher. 1986. \emph{Minimal Rationality}. Cambridge, MA:
MIT Press.

\bibitem[\citeproctext]{ref-Christensen2007}
Christensen, David. 2007. {``Does Murphy's Law Apply in Epistemology?
Self-Doubt and Rational Ideals.''} \emph{Oxford Studies in Epistemology}
2: 3--31.

\bibitem[\citeproctext]{ref-Conlisk1996}
Conlisk, John. 1996. {``Why Bounded Rationality?''} \emph{Journal of
Economic Literature} 34 (2): 669--700.

\bibitem[\citeproctext]{ref-DiabEtAl2008}
Diab, Dalia L., Michael A. Gillespie, and Scott Highhouse. 2008. {``Are
Maximizers Really Unhappy? The Measurement of Maximizing Tendency.''}
\emph{Judgment and Decision Making} 3 (5): 364--70.

\bibitem[\citeproctext]{ref-Dylan2016}
Dylan, Bob. 2016. \emph{The Lyrics: 1961-2012}. New York: Simon \&
Schuster.

\bibitem[\citeproctext]{ref-Elster1979}
Elster, Jon. 1979. \emph{Ulysses and the Sirens: Studies in Rationality
and Irrationality}. Cambridge: Cambridge University Press.

\bibitem[\citeproctext]{ref-GigerenzerSelton2001}
Gigerenzer, Gerd, and Reinhard Selten. 2001. \emph{Bounded Rationality:
The Adaptive Toolbox}. Cambridge, MA: MIT Press.

\bibitem[\citeproctext]{ref-Harman1973}
Harman, Gilbert. 1973. \emph{Thought}. Princeton: Princeton University
Press.

\bibitem[\citeproctext]{ref-IyengarEtAl2006}
Iyengar, Sheena S., Rachael E. Wells, and Barry Schwartz. 2006. {``Doing
Better but Feeling Worse: Looking for the {`Best'} Job Undermines
Satisfaction.''} \emph{Psychological Science} 17 (2): 143--50. doi:
\href{https://doi.org/10.1111/j.1467-9280.2006.01677.x}{10.1111/j.1467-9280.2006.01677.x}.

\bibitem[\citeproctext]{ref-Joyce1999}
Joyce, James M. 1999. \emph{The Foundations of Causal Decision Theory}.
Cambridge: Cambridge University Press.

\bibitem[\citeproctext]{ref-Kimball2015}
Kimball, Miles. 2015. {``Cognitive Economics.''} \emph{The Japanese
Economic Review} 66 (2): 167--81. doi:
\href{https://doi.org/10.1111/jere.12070}{10.1111/jere.12070}.

\bibitem[\citeproctext]{ref-Knight1921}
Knight, Frank. 1921. \emph{Risk, Uncertainty and Profit}. Chicago:
University of Chicago Press.

\bibitem[\citeproctext]{ref-LevyAlfano2019}
Levy, Neil, and Mark Alfano. 2020. {``Knowledge from Vice: Deeply Social
Epistemology.''} \emph{Mind} 129 (515): 887--915. doi:
\href{https://doi.org/10.1093/mind/fzz017}{10.1093/mind/fzz017}.

\bibitem[\citeproctext]{ref-LipseyLancaster}
Lipsey, R. G., and Kelvin Lancaster. 1956. {``The General Theory of
Second Best.''} \emph{Review of Economic Studies} 24 (1): 11--32. doi:
\href{https://doi.org/10.2307/2296233}{10.2307/2296233}.

\bibitem[\citeproctext]{ref-ManganEtAl2010}
Mangan, Jean, Amanda Hughes, and Kim Slack. 2010. {``Student Finance,
Information and Decision Making.''} \emph{Higher Education} 60 (5):
459--72. doi:
\href{https://doi.org/10.1007/s10734-010-9309-7}{10.1007/s10734-010-9309-7}.

\bibitem[\citeproctext]{ref-Manski2017}
Manski, Charles F. 2017. {``Optimize, Satisfice, or Choose Without
Deliberation? A Simple Minimax-Regret Assessment.''} \emph{Theory and
Decision} 83 (2): 155--73. doi:
\href{https://doi.org/10.1007/s11238-017-9592-1}{10.1007/s11238-017-9592-1}.

\bibitem[\citeproctext]{ref-NewmanEtAl2018}
Newman, David B., Joanna Schug, Masaki Yuki, Junko Yamada, and John B.
Nezlek. 2018. {``The Negative Consequences of Maximizing in Friendship
Selection.''} \emph{Journal of Personality and Social Psychology} 114
(5): 804--24. doi:
\href{https://doi.org/10.1037/pspp0000141}{10.1037/pspp0000141}.

\bibitem[\citeproctext]{ref-Odell2002}
Odell, John S. 2002. {``Bounded Rationality and World Political
Economy.''} In \emph{Governing the World's Money}, edited by David M.
Andrews, C. Randall Henning, and Louis W. Pauly, 168--93. Ithaca:
Cornell University Press.

\bibitem[\citeproctext]{ref-OgakiTanaka2017}
Ogaki, Masao, and Saori C. Tanaka. 2017. \emph{Behavioral Economics:
Toward a New Economics by Integration with Traditional Economics}.
Singapore: Springer.

\bibitem[\citeproctext]{ref-Papi2013}
Papi, Mario. 2013. {``Satisficing and Maximizing Consumers in a
Monopolistic Screening Model.''} \emph{Mathematical Social Sciences} 66
(3): 385--89. doi:
\href{https://doi.org/10.1016/j.mathsocsci.2013.08.005}{10.1016/j.mathsocsci.2013.08.005}.

\bibitem[\citeproctext]{ref-Pingle2006}
Pingle, Mark. 2006. {``Deliberation Cost as a Foundation for Behavioral
Economics.''} In \emph{In Handbook of Contemporary Behavioral Economics:
Foundations and Developments}, edited by Morris Altman, 340--55. New
York: Routledge.

\bibitem[\citeproctext]{ref-Reutskaja2011}
Reutskaja, Elena, Rosemarie Nagel, Colin F. Camerer, and Antonio Rangel.
2011. {``Search Dynamics in Consumer Choice Under Time Pressure: An
Eye-Tracking Study.''} \emph{American Economic Review} 101 (2):
900--926. doi:
\href{https://doi.org/10.1257/aer.101.2.900}{10.1257/aer.101.2.900}.

\bibitem[\citeproctext]{ref-Russell1997}
Russell, Stuart J. 1997. {``Rationality and Intelligence.''}
\emph{Artificial Intelligence} 94 (1-2): 57--77. doi:
\href{https://doi.org/10.1016/S0004-3702(97)00026-X}{10.1016/S0004-3702(97)00026-X}.

\bibitem[\citeproctext]{ref-ScheibehenneEtAl2010}
Scheibehenne, Benjamin, Rainer Greifeneder, and Peter M. Todd. 2010.
{``Can There Ever Be Too Many Options? A Meta-Analytic Review of Choice
Overload.''} \emph{Journal of Consumer Research} 37 (3): 409--25. doi:
\href{https://doi.org/10.1086/651235}{10.1086/651235}.

\bibitem[\citeproctext]{ref-Schwartz2004}
Schwartz, Barry. 2004. \emph{The Paradox of Choice: Why More Is Less}.
New York: Harper Collins.

\bibitem[\citeproctext]{ref-SchwartzEtAl2002}
Schwartz, Barry, Andrew Ward, John Monterosso, Sonja Lyubomirsky,
Katherine White, and Darrin R. Lehman. 2002. {``Maximizing Versus
Satisficing: Happiness Is a Matter of Choice.''} \emph{Journal of
Personality and Social Psychology} 83 (5): 1178--97. doi:
\href{https://doi.org/10.1037/0022-3514.83.5.1178}{10.1037/0022-3514.83.5.1178}.

\bibitem[\citeproctext]{ref-ScottBaillargeon2013}
Scott, Rose M., and Renée Baillargeon. 2013. {``Do Infants Really Expect
Agents to Act Efficiently? A Critical Test of the Rationality
Principle.''} \emph{Pscyhological Science} 24 (4): 466--74. doi:
\href{https://doi.org/10.1177/0956797612457395}{10.1177/0956797612457395}.

\bibitem[\citeproctext]{ref-Shanahan2016}
Shanahan, Murray. 2016. {``The Frame Problem.''} In \emph{The Stanford
Encyclopedia of Philosophy}, edited by Edward N. Zalta, Spring 2016.
Metaphysics Research Lab, Stanford University.

\bibitem[\citeproctext]{ref-SperberEtAl2010}
Sperber, Dan, Fabrice Clément, Christophe Heintz, Olivier Mascaro, Hugo
Mercier, Gloria Origgi, and Deirdre Wilson. 2010. {``Epistemic
Vigilance.''} \emph{Mind and Language} 25 (4): 359--93. doi:
\href{https://doi.org/10.1111/j.1468-0017.2010.01394.x}{10.1111/j.1468-0017.2010.01394.x}.

\bibitem[\citeproctext]{ref-Strevens2008}
Strevens, Michael. 2008. \emph{Depth: An Account of Scientific
Explanations}. Cambridge, MA: Harvard University Press.

\bibitem[\citeproctext]{ref-Tucker2016}
Tucker, Chris. 2016. {``Satisficing and Motivated Submaximization (in
the Philosophy of Religion).''} \emph{Philosophy and Phenomenological
Research} 93 (1): 127--43. doi:
\href{https://doi.org/10.1111/phpr.12191}{10.1111/phpr.12191}.

\bibitem[\citeproctext]{ref-Williamson2000}
Williamson, Timothy. 2000. \emph{{Knowledge and its Limits}}. Oxford
University Press.

\bibitem[\citeproctext]{ref-Wittgenstein1953}
Wittgenstein, Ludwig. 1953. \emph{Philosophical Investigations}. London:
Macmillan.

\bibitem[\citeproctext]{ref-Zweber2016}
Zweber, Adam. 2016. {``Fallibilism, Closure, and Pragmatic
Encroachment.''} \emph{Philosophical Studies} 173 (10): 2745--57. doi:
\href{https://doi.org/10.1007/s11098-016-0631-5}{10.1007/s11098-016-0631-5}.

\end{CSLReferences}



\noindent Unpublished. First posted in 2020.

\end{document}
