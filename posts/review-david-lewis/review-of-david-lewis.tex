% Options for packages loaded elsewhere
\PassOptionsToPackage{unicode}{hyperref}
\PassOptionsToPackage{hyphens}{url}
\PassOptionsToPackage{dvipsnames,svgnames,x11names}{xcolor}
%
\documentclass[
  11pt,
  letterpaper,
  DIV=11,
  numbers=noendperiod,
  oneside]{scrartcl}

\usepackage{amsmath,amssymb}
\usepackage{iftex}
\ifPDFTeX
  \usepackage[T1]{fontenc}
  \usepackage[utf8]{inputenc}
  \usepackage{textcomp} % provide euro and other symbols
\else % if luatex or xetex
  \ifXeTeX
    \usepackage{mathspec} % this also loads fontspec
  \else
    \usepackage{unicode-math} % this also loads fontspec
  \fi
  \defaultfontfeatures{Scale=MatchLowercase}
  \defaultfontfeatures[\rmfamily]{Ligatures=TeX,Scale=1}
\fi
\usepackage{lmodern}
\ifPDFTeX\else  
    % xetex/luatex font selection
  \setmainfont[Scale = MatchLowercase]{Scala Pro}
  \setsansfont[]{Scala Sans Pro}
  \ifXeTeX
    \setmathfont(Digits,Latin,Greek)[]{Scala Pro}
  \else
    \setmathfont[]{Scala Pro}
  \fi
\fi
% Use upquote if available, for straight quotes in verbatim environments
\IfFileExists{upquote.sty}{\usepackage{upquote}}{}
\IfFileExists{microtype.sty}{% use microtype if available
  \usepackage[]{microtype}
  \UseMicrotypeSet[protrusion]{basicmath} % disable protrusion for tt fonts
}{}
\makeatletter
\@ifundefined{KOMAClassName}{% if non-KOMA class
  \IfFileExists{parskip.sty}{%
    \usepackage{parskip}
  }{% else
    \setlength{\parindent}{0pt}
    \setlength{\parskip}{6pt plus 2pt minus 1pt}}
}{% if KOMA class
  \KOMAoptions{parskip=half}}
\makeatother
\usepackage{xcolor}
\usepackage[left=1in,marginparwidth=2.0666666666667in,textwidth=4.1333333333333in,marginparsep=0.3in]{geometry}
\setlength{\emergencystretch}{3em} % prevent overfull lines
\setcounter{secnumdepth}{3}
% Make \paragraph and \subparagraph free-standing
\ifx\paragraph\undefined\else
  \let\oldparagraph\paragraph
  \renewcommand{\paragraph}[1]{\oldparagraph{#1}\mbox{}}
\fi
\ifx\subparagraph\undefined\else
  \let\oldsubparagraph\subparagraph
  \renewcommand{\subparagraph}[1]{\oldsubparagraph{#1}\mbox{}}
\fi


\providecommand{\tightlist}{%
  \setlength{\itemsep}{0pt}\setlength{\parskip}{0pt}}\usepackage{longtable,booktabs,array}
\usepackage{calc} % for calculating minipage widths
% Correct order of tables after \paragraph or \subparagraph
\usepackage{etoolbox}
\makeatletter
\patchcmd\longtable{\par}{\if@noskipsec\mbox{}\fi\par}{}{}
\makeatother
% Allow footnotes in longtable head/foot
\IfFileExists{footnotehyper.sty}{\usepackage{footnotehyper}}{\usepackage{footnote}}
\makesavenoteenv{longtable}
\usepackage{graphicx}
\makeatletter
\def\maxwidth{\ifdim\Gin@nat@width>\linewidth\linewidth\else\Gin@nat@width\fi}
\def\maxheight{\ifdim\Gin@nat@height>\textheight\textheight\else\Gin@nat@height\fi}
\makeatother
% Scale images if necessary, so that they will not overflow the page
% margins by default, and it is still possible to overwrite the defaults
% using explicit options in \includegraphics[width, height, ...]{}
\setkeys{Gin}{width=\maxwidth,height=\maxheight,keepaspectratio}
% Set default figure placement to htbp
\makeatletter
\def\fps@figure{htbp}
\makeatother

\setlength\heavyrulewidth{0ex}
\setlength\lightrulewidth{0ex}
\makeatletter
\def\@maketitle{%
\newpage
\null
\vskip 2em%
\begin{center}%
\let \footnote \thanks
  {\LARGE \@title \par}%
  \vskip 1.5em%
  {\large
    \lineskip .5em%
    \begin{tabular}[t]{c}%
      \@author
    \end{tabular}\par}%
  %\vskip 1em%
  %{\large \@date}%
\end{center}%
\par
\vskip 1.5em}
\makeatother
\KOMAoption{captions}{tableheading}
\makeatletter
\@ifpackageloaded{caption}{}{\usepackage{caption}}
\AtBeginDocument{%
\ifdefined\contentsname
  \renewcommand*\contentsname{Table of contents}
\else
  \newcommand\contentsname{Table of contents}
\fi
\ifdefined\listfigurename
  \renewcommand*\listfigurename{List of Figures}
\else
  \newcommand\listfigurename{List of Figures}
\fi
\ifdefined\listtablename
  \renewcommand*\listtablename{List of Tables}
\else
  \newcommand\listtablename{List of Tables}
\fi
\ifdefined\figurename
  \renewcommand*\figurename{Figure}
\else
  \newcommand\figurename{Figure}
\fi
\ifdefined\tablename
  \renewcommand*\tablename{Table}
\else
  \newcommand\tablename{Table}
\fi
}
\@ifpackageloaded{float}{}{\usepackage{float}}
\floatstyle{ruled}
\@ifundefined{c@chapter}{\newfloat{codelisting}{h}{lop}}{\newfloat{codelisting}{h}{lop}[chapter]}
\floatname{codelisting}{Listing}
\newcommand*\listoflistings{\listof{codelisting}{List of Listings}}
\makeatother
\makeatletter
\makeatother
\makeatletter
\@ifpackageloaded{caption}{}{\usepackage{caption}}
\@ifpackageloaded{subcaption}{}{\usepackage{subcaption}}
\makeatother
\makeatletter
\@ifpackageloaded{sidenotes}{}{\usepackage{sidenotes}}
\@ifpackageloaded{marginnote}{}{\usepackage{marginnote}}
\makeatother
\ifLuaTeX
  \usepackage{selnolig}  % disable illegal ligatures
\fi
\IfFileExists{bookmark.sty}{\usepackage{bookmark}}{\usepackage{hyperref}}
\IfFileExists{xurl.sty}{\usepackage{xurl}}{} % add URL line breaks if available
\urlstyle{same} % disable monospaced font for URLs
\hypersetup{
  pdftitle={Review of ``David Lewis''},
  pdfauthor={Brian Weatherson},
  colorlinks=true,
  linkcolor={black},
  filecolor={Maroon},
  citecolor={Blue},
  urlcolor={Blue},
  pdfcreator={LaTeX via pandoc}}

\title{Review of ``David Lewis''}
\author{Brian Weatherson}
\date{2007-01-01}

\begin{document}
\maketitle
David Lewis left us with a rich and highly integrated body of work. Its
richness means that any serious student of philosophy should study it
closely. And at first this looks like it should be something that any
student should be able to tackle. Lewis is one of the great stylists of
his generation, and his views on most topics are expressed with
admirable clarity. But the integration makes it difficult for the
beginner to get a foothold. Any place you look it seems you will have to
master five other topics before you really understand what Lewis says on
\emph{this} topic. So a systematic introduction to Lewis's views is
needed, and Daniel Nolan's new book provides one.

There are two types of reader who will most benefit from Nolan's book.

The book will obviously be very valuable for students, especially
undergraduates. It would make an excellent textbook for upper level
classes on metaphysics or philosophy of mind where Lewis's views were an
important part of the course.

Although it is mostly not pitched at experts, the book should also have
value for professional philosophers because of how it draws out the
connections between Lewis's views. For instance, someone working on the
rule-following paradoxes who was interested in learning more about the
notion of semantic eligibility that does so much work in Lewis's
solution to these paradoxes could learn here how this notion is related
to notions from the theory of properties and the analysis of physical
law. But this is not just a textbook, and the critical element of
Nolan's exposition should be helpful at points even to experts.

Nolan's summary starts with Lewis's commitment to realism, both
scientific and metaphysical. He then spends three chapters setting out
the building blocks we need for Lewis's metaphysics (properties, other
times and other worlds) and showing how these can get used to rich
picture of reality, one replete with causation, laws, dispositions and
chances. Chapter five is on Lewis's distinctive metaphysics of mind, and
discusses how it relates to functionalist theories and the identity
theory. Chapters six and seven are on content, mental and linguistic
respectively. Chapter eight surveys Lewis's views on ethics and value
theory, and the last chapter is on Lewis's methodology, especially his
use of Ramsey sentences as a way of defining theoretical terms.

There is much to like through all of this. Although all the parts of the
metaphysical picture are set out throughout Lewis's writings, this is
the best systematic exposition of the picture in a single place. The
treatment of causation, which cuts through a lot of complicated
discussion to get to the essence of Lewis's theory, is especially
useful. The discussion of Lewis's ethical views, focussing on his
complicated relationship to consequentialism and virtue ethics, does an
excellent job of drawing a relatively systematic theory out of scattered
remarks from several obscure sources.

Obviously there is a lot that could not be covered in this kind of book.
So there is very little on perception, nothing on philosophy of
mathematics, next to nothing on formal philosophy save a small
discussion of the semantics of counterfactuals, nothing on Lewis's
arguments that desire and belief are separate existences and so on.
These must have been hard cuts, but given the target audience I think
they were the right ones. If anything I would have been tempted to cut
even more to allow a little more space to the topics covered. I doubt
that typical readers will get much out of the discussion of
indeterminate probabilities towards the end of chapter 6 for example, as
interesting as that should be to experts.

Although the book is primarily expository there is a good amount of
critical discussion interspersed throughout. Lewis's single strangest
view, that dispositions must be grounded in \emph{intrinsic} properties
of the bearer of the disposition, comes in for extended and
well-targeted criticism. And Nolan raises some interesting cases that
suggest the cases not covered by Lewis's story in ``Mad Pain and Martian
Pain'' could be closer to home than Lewis wants. It might be worried
that this much critical engagement will undermine the effectiveness of
the book as a text, but I think it is all beneficial. For one thing, the
criticisms often help highlight the contours of the theory. But there is
a deeper reason too. A student learning Lewisian philosophy shouldn't
just be learning a bunch of Lewis's theories. They should be learning
something about how to do philosophy, which means putting forward
theories and criticisms of theories. The criticisms Nolan makes, all of
them the kind of criticism that Lewis would have taken seriously and
even have made in other circumstances, help teach the student how
progress is made within the paradigm Lewis established.

Lewis had a weakness for fantastic examples. His work is littered with
stories of Martians and gods and wizards and infallible predictors. He
thought, as I do, that these were perfectly appropriate in the contexts
he used them. But they create difficulties because students, and
professionals, don't see the relevance of these fantasies to real world
analysis. Nolan does the student, and the instructor, a real service by
replacing these examples with down-to-earth ones. The sections on
causation and causal decision theory in particular are clarified by
these changed examples.

There are a few things that may have been done differently given the
target audience. The bibliography only includes those works by Lewis
that are cited in the text; it should have been a complete bibliography
of Lewis's work. Nolan from time to time refers to things that Lewis's
critics say without referring to those critics by name, let alone citing
a reference. The book would be a more useful resource if it pointed
explicitly to where the reader might see these criticisms set out in
more depth. And on one or two occasions the book presupposes much more
knowledge than its primary target reader will have. In the discussion of
the view that all belief is \emph{de se} belief, for example, Nolan
suddenly presupposes familiarity with causal descriptivism about names
without so much as introducing descriptivism.

But the virtues of the book outweigh these possible imperfections.
Lewis's philosophical work should be taught to as many of the next
generation of philosophers as possible. (Not to mention the present
generation.) Those of us engaged in this task would find our job easier
if we had a clear and systematic presentation of Lewis's philosophy. Now
we do.



\end{document}
