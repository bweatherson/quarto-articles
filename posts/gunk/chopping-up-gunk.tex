% Options for packages loaded elsewhere
\PassOptionsToPackage{unicode}{hyperref}
\PassOptionsToPackage{hyphens}{url}
%
\documentclass[
  10pt,
  letterpaper,
  DIV=11,
  numbers=noendperiod,
  twoside]{scrartcl}

\usepackage{amsmath,amssymb}
\usepackage{setspace}
\usepackage{iftex}
\ifPDFTeX
  \usepackage[T1]{fontenc}
  \usepackage[utf8]{inputenc}
  \usepackage{textcomp} % provide euro and other symbols
\else % if luatex or xetex
  \usepackage{unicode-math}
  \defaultfontfeatures{Scale=MatchLowercase}
  \defaultfontfeatures[\rmfamily]{Ligatures=TeX,Scale=1}
\fi
\usepackage{lmodern}
\ifPDFTeX\else  
    % xetex/luatex font selection
  \setmainfont[ItalicFont=EB Garamond Italic,BoldFont=EB Garamond
Bold]{EB Garamond Math}
  \setsansfont[]{Europa-Bold}
  \setmathfont[]{Garamond-Math}
\fi
% Use upquote if available, for straight quotes in verbatim environments
\IfFileExists{upquote.sty}{\usepackage{upquote}}{}
\IfFileExists{microtype.sty}{% use microtype if available
  \usepackage[]{microtype}
  \UseMicrotypeSet[protrusion]{basicmath} % disable protrusion for tt fonts
}{}
\usepackage{xcolor}
\usepackage[left=1in, right=1in, top=0.8in, bottom=0.8in,
paperheight=9.5in, paperwidth=6.5in, includemp=TRUE, marginparwidth=0in,
marginparsep=0in]{geometry}
\setlength{\emergencystretch}{3em} % prevent overfull lines
\setcounter{secnumdepth}{3}
% Make \paragraph and \subparagraph free-standing
\ifx\paragraph\undefined\else
  \let\oldparagraph\paragraph
  \renewcommand{\paragraph}[1]{\oldparagraph{#1}\mbox{}}
\fi
\ifx\subparagraph\undefined\else
  \let\oldsubparagraph\subparagraph
  \renewcommand{\subparagraph}[1]{\oldsubparagraph{#1}\mbox{}}
\fi


\providecommand{\tightlist}{%
  \setlength{\itemsep}{0pt}\setlength{\parskip}{0pt}}\usepackage{longtable,booktabs,array}
\usepackage{calc} % for calculating minipage widths
% Correct order of tables after \paragraph or \subparagraph
\usepackage{etoolbox}
\makeatletter
\patchcmd\longtable{\par}{\if@noskipsec\mbox{}\fi\par}{}{}
\makeatother
% Allow footnotes in longtable head/foot
\IfFileExists{footnotehyper.sty}{\usepackage{footnotehyper}}{\usepackage{footnote}}
\makesavenoteenv{longtable}
\usepackage{graphicx}
\makeatletter
\def\maxwidth{\ifdim\Gin@nat@width>\linewidth\linewidth\else\Gin@nat@width\fi}
\def\maxheight{\ifdim\Gin@nat@height>\textheight\textheight\else\Gin@nat@height\fi}
\makeatother
% Scale images if necessary, so that they will not overflow the page
% margins by default, and it is still possible to overwrite the defaults
% using explicit options in \includegraphics[width, height, ...]{}
\setkeys{Gin}{width=\maxwidth,height=\maxheight,keepaspectratio}
% Set default figure placement to htbp
\makeatletter
\def\fps@figure{htbp}
\makeatother
% definitions for citeproc citations
\NewDocumentCommand\citeproctext{}{}
\NewDocumentCommand\citeproc{mm}{%
  \begingroup\def\citeproctext{#2}\cite{#1}\endgroup}
\makeatletter
 % allow citations to break across lines
 \let\@cite@ofmt\@firstofone
 % avoid brackets around text for \cite:
 \def\@biblabel#1{}
 \def\@cite#1#2{{#1\if@tempswa , #2\fi}}
\makeatother
\newlength{\cslhangindent}
\setlength{\cslhangindent}{1.5em}
\newlength{\csllabelwidth}
\setlength{\csllabelwidth}{3em}
\newenvironment{CSLReferences}[2] % #1 hanging-indent, #2 entry-spacing
 {\begin{list}{}{%
  \setlength{\itemindent}{0pt}
  \setlength{\leftmargin}{0pt}
  \setlength{\parsep}{0pt}
  % turn on hanging indent if param 1 is 1
  \ifodd #1
   \setlength{\leftmargin}{\cslhangindent}
   \setlength{\itemindent}{-1\cslhangindent}
  \fi
  % set entry spacing
  \setlength{\itemsep}{#2\baselineskip}}}
 {\end{list}}
\usepackage{calc}
\newcommand{\CSLBlock}[1]{\hfill\break\parbox[t]{\linewidth}{\strut\ignorespaces#1\strut}}
\newcommand{\CSLLeftMargin}[1]{\parbox[t]{\csllabelwidth}{\strut#1\strut}}
\newcommand{\CSLRightInline}[1]{\parbox[t]{\linewidth - \csllabelwidth}{\strut#1\strut}}
\newcommand{\CSLIndent}[1]{\hspace{\cslhangindent}#1}

\setlength\heavyrulewidth{0ex}
\setlength\lightrulewidth{0ex}
\usepackage[automark]{scrlayer-scrpage}
\clearpairofpagestyles
\cehead{
  Brian Weatherson
  }
\cohead{
  Chopping up Gunk
  }
\ohead{\bfseries \pagemark}
\cfoot{}
\makeatletter
\newcommand*\NoIndentAfterEnv[1]{%
  \AfterEndEnvironment{#1}{\par\@afterindentfalse\@afterheading}}
\makeatother
\NoIndentAfterEnv{itemize}
\NoIndentAfterEnv{enumerate}
\NoIndentAfterEnv{description}
\NoIndentAfterEnv{quote}
\NoIndentAfterEnv{equation}
\NoIndentAfterEnv{longtable}
\NoIndentAfterEnv{abstract}
\renewenvironment{abstract}
 {\vspace{-1.25cm}
 \quotation\small\noindent\rule{\linewidth}{.5pt}\par\smallskip
 \noindent }
 {\par\noindent\rule{\linewidth}{.5pt}\endquotation}
\cehead{
       John Hawthorne and Brian Weatherson
        }
\KOMAoption{captions}{tableheading}
\makeatletter
\@ifpackageloaded{caption}{}{\usepackage{caption}}
\AtBeginDocument{%
\ifdefined\contentsname
  \renewcommand*\contentsname{Table of contents}
\else
  \newcommand\contentsname{Table of contents}
\fi
\ifdefined\listfigurename
  \renewcommand*\listfigurename{List of Figures}
\else
  \newcommand\listfigurename{List of Figures}
\fi
\ifdefined\listtablename
  \renewcommand*\listtablename{List of Tables}
\else
  \newcommand\listtablename{List of Tables}
\fi
\ifdefined\figurename
  \renewcommand*\figurename{Figure}
\else
  \newcommand\figurename{Figure}
\fi
\ifdefined\tablename
  \renewcommand*\tablename{Table}
\else
  \newcommand\tablename{Table}
\fi
}
\@ifpackageloaded{float}{}{\usepackage{float}}
\floatstyle{ruled}
\@ifundefined{c@chapter}{\newfloat{codelisting}{h}{lop}}{\newfloat{codelisting}{h}{lop}[chapter]}
\floatname{codelisting}{Listing}
\newcommand*\listoflistings{\listof{codelisting}{List of Listings}}
\makeatother
\makeatletter
\makeatother
\makeatletter
\@ifpackageloaded{caption}{}{\usepackage{caption}}
\@ifpackageloaded{subcaption}{}{\usepackage{subcaption}}
\makeatother
\ifLuaTeX
  \usepackage{selnolig}  % disable illegal ligatures
\fi
\IfFileExists{bookmark.sty}{\usepackage{bookmark}}{\usepackage{hyperref}}
\IfFileExists{xurl.sty}{\usepackage{xurl}}{} % add URL line breaks if available
\urlstyle{same} % disable monospaced font for URLs
\hypersetup{
  pdftitle={Chopping up Gunk},
  pdfauthor={John Hawthorne; Brian Weatherson},
  hidelinks,
  pdfcreator={LaTeX via pandoc}}

\title{Chopping up Gunk}
\author{John Hawthorne \and Brian Weatherson}
\date{2004}

\begin{document}
\maketitle
\begin{abstract}
We raise an objection to the idea that the world is gunky. Certain
plausible sounding supertasks have implausible consequences if the world
is made of gunk.
\end{abstract}

\setstretch{1.1}
Atomism, the view that indivisible atoms are the basic building blocks
of physical reality, has a distinguished history. But it might not be
true. The history of physical science certainly gives many of us pause.
Every time some class of objects appeared to be the entities that Newton
had described as ``solid, massy, hard, impenetrable, movable Particles''
out of which ``God in the Beginning formed Matter''
(\citeproc{ref-Newton1952}{Newton 1952, 400}), further research revealed
that these objects were divisible after all. One might be tempted to see
that history as confirming Leibniz' dismissal of atomism as a ``youthful
prejudice'' .\footnote{See `Nature Itself' in
  (\citeproc{ref-Leibniz1998}{Leibniz 1998, 220}).} Perhaps material
objects and their parts are always divisible. There are no extended
atoms; nor are there point particles which compose material
beings.\footnote{Cf Leibniz: `I hold that \emph{matter} is essentially
  an \emph{aggregate}, and consequently that it always has actual
  parts,' in `Third Explanation of The New System,'
  (\citeproc{ref-Leibniz1998}{Leibniz 1998, 193}).}

When first presented with this hypothesis, our imaginations are quickly
drawn to picturing the process whereby a quantity of such matter -- call
it gunk -- is chopped up into smaller and smaller pieces. Prima facie,
there is nothing problematic here: insofar as such a process continues
without end, the initial quantity gets resolves into smaller and smaller
chunks with no limit to the diminution. But suppose this process is
packed into an hour, as imagined by Jose Bernadete
(\citeproc{ref-Bernadete1964}{1964}) in his 1964 monograph Infinity:

\begin{quote}
Take a stick of wood. In 1/2 minute we are to divide the stick into two
equal parts. In the next 1/4 minute we are to divide each of the two
pieces again into two equal parts. In the next 1/8 minute we are to
divide each of the four pieces (for there are now four equal pieces)
again into two equal parts, \&c.~\emph{ad infinitum}
(\citeproc{ref-Bernadete1964}{Bernadete 1964, 184}).
\end{quote}

If matter is divisible without end there seems to be no conceptual
obstacle to each of the divisions. Yet how are we to imagine the
situation at the end of the hour, when the super-task (call it
`super-cutting') has been performed on a quantity of gunk?\footnote{What
  is important, of course, is that the sequence of separations occur: it
  does not matter whether some kind of super-sharp knife is responsible
  for them. In what follows, descriptions of cutting sequences can be
  replaced without loss of content by descriptions of separation
  sequences, leaving it open whether repulsive forces or chance events
  or knives or \ldots{} are responsible for the separation sequence.}

If there were extended atoms that were never annihilated, it is clear
enough what would happen if some super-being undertook to perform
super-cutting: the process would grind to a halt when insurmountably
hard particles resisted the chopper.

If, meanwhile, there were point-sized particles that composed planes
that were as thin as a line, it would be natural to picture the limit of
the process as a sea of separated slivers, each devoid of finite extent
along one dimension. As Benardete, notes, one might then redo
super-cutting in order to finally resolve the original stick into a sea
of ``metaphysical motes'' devoid of finite extent in any direction:

\begin{quote}
At the end of the minute how many pieces of wood will we have laid out
before us? Clearly an infinite number. If the original stick was twenty
inches in length, one inch in width, and one inch in depth, what are the
dimensions of the metaphysical chips into which the stick has been
decomposed? Each chip will be one inch by one inch by one inch by --
what? So prodigiously thin must each chip be that its value is
certifiably less then any rational (or irrational) quantity. Let us now
take up one of the metaphysical chips and decompose it further into an
infinite number of metaphysical splinters. In 1/2 minute we shall divide
the chip into two equal parts. Each pieces will be one inch by 1/2 inch.
In the next 1/4 minute we shall divide each of the two pieces again into
two equal parts, yielding four pieces each being one inch by 1/4 inch.
In the next 1/8 minute we shall divide each of the four pieces again
into two equal parts, \&c ad infinitum. At the end of the mute we shall
have composed the metaphysical chip into metaphysical splinters. Each
splinter will be one inch in length. Let us now take up one of the
metaphysical splinters and break it down into an infinite number of
metaphysical motes (\citeproc{ref-Bernadete1964}{Bernadete 1964,
184--85})
\end{quote}

The number of cuts made on the stick, the chip and the splinter
respectively is aleph zero. The number of chips, splinters and motes
left at the end of each cutting process, meanwhile, is aleph one. (Think
of numbering each piece in a super-cutting process by an infinite
expansion of one's and zero's as follows: if it lay on the left of the
first cut, the first numeral is a zero, if to the right, the first
numeral is a one; it if lay on the left of one of the pieces that was
divided at the second round of cutting its second numeral is a zero, if
to the right a one; and so on. For each decimal expansion of one's and
zero's there is a bit and the end with that expansion.). This result is
surprising to some, but poses no deep conceptual confusion. With an
ontology of chips, splinters and motes available to us, there is a
natural description available to us of the limit to the super-cutting
processes described.

But what to say when gunk is subjected to super-cutting? If each
quantity of matter has proper parts, then a sea of metaphysical motes is
not, it would seem, an available outcome. In what follows, we unpack
this puzzle, providing along the way some a priori physics for
gunk-lovers. The problem is only well formed when we make explicit some
of the assumptions that drive it. We do so below:

\begin{quote}
\begin{description}
\tightlist
\item[(1)]
Gunk

Every quantity of matter has proper parts.
\item[(2)]
Conservation of Matter

No mereological sum of quantities of matter can be destroyed by any
sequence of cuts (though it may be scattered)\footnote{The `can' here is
  one of nomological possibility.}.
\item[(3)]
Occupation

If a quantity of matter occupies a point of space, then there is some
volume, extended in all dimensions, to which that point belongs which
that quantity of matter occupies.
\item[(4)]
Super-cutting

The laws of the world permit super-cutting.
\end{description}
\end{quote}

Note that (1), the thesis that every quantity of matter has parts does
not, by itself, entail any of the other theses. One might also think
that matter sometimes vanishes as a result of some sequence of cuts,
denying (2). One might hold that there are metaphysical splinters (and
perhaps even chips), denying (3). One might hold that any given quantity
of matter does have point sized pieces but that those pieces themselves
have parts (the parts being of the same size as the whole in this case),
denying (3). One might hold that some pieces of gunk can occupy, say, a
spherical region and also a single isolated point at some considerable
distance from the spherical region (while maintaining that no part of it
merely occupies the point), also denying (3). One might imagine that
while always having parts, the parts below a certain thickness are
inseparable, denying (4). One might think there is a minimum amount of
time that any event of separation takes, also denying (4) and so on.

If the gunk hypothesis is maintained, but one or more of (2) to (4) is
jettisoned, there is no problem left to solve. For example: If we are
allowed to suppose that gunk may vanish, then it will be perfectly
consistent to say that nothing is left at the limit of super-cutting. If
we are allowed parts that lack finite extent, then it will be consistent
to adopt Benardete's picture of the outcome. And so on. Our puzzle,
properly formulated is: What would happen if super-cutting occurred in a
world where (1) to (4) are true?

In order to answer that question, we need to supplement Bernadete's
brief discussion of the super-cutting process. It is not immediately
clear from what he says that super-cutting a piece of wood will turn an
object into chips, even assuming the wood to be composed of point
particles. That is a natural description of the limit of the process,
but it is hardly one that is forced upon us by the barebones description
of the process that Benardete provides. When we divide the stick into
two pieces, and then into four pieces, where are we to put these pieces?
Presumably we must ensure that they are separated. If not, it will not
be clear that we really have splinters left at the end. If the stick is
cut into four, but the four pieces are then stored so closely together
that they are not scattered any more, then we will not have four
scattered objects after two rounds of cutting. By extension, unless we
separate the pieces sufficiently after each round (or at least after
sufficiently many of them) then even in a world where matter is composed
of point particles, it is not clear that there will be infinitely many
chips left at the end. Note in this connection that there are limits as
to how far we can separate the objects. In a world where super-cutting
produces chips, we could not, from left to right, put a one inch gap
between each chip and any other, since there are aleph one chips and not
aleph one inches of space along any vector. Nor is it even clear what
kind of spacing will do the trick: how we are to keep aleph one chips
separated from each other? What we need is a formal model showing how
super-cutting is to be performed. Only then can we answer with any
precision what would happen were super-cutting to be performed on gunk.

Assume, for simplicity, that we have a stick that is exactly one inch
long. At the first stage, cut the stick into two 1/2 inch long pieces,
move the left-hand one 1/4 inch leftwards and the right hand one 1/4
inch rightwards. This can be accomplished in 1/2 second without moving
the objects at a speed of faster than 1 inch per second, or accelerating
or decelerating the objects at a rate higher than 4 inches per second
per second.\footnote{The idea is that in the first quarter second we
  accelerate the object at 4 inches per second per second. This will
  raise its velocity to 1 inch per second, and move the object 1/8 of an
  inch. In the second quarter second we decelerate it at 4 inches per
  second per second, so its velocity ends up at zero, and it ends up
  having moved 1/4 of an inch.} At the second stage, cut each piece into
two, and move each of the left-hand pieces 1/16 of an inch leftwards,
and each of the right-hand pieces 1/16 of an inch rightwards. So if the
original piece occupied the interval {[}0,~1) on a particular axis, the
four pieces will now occupy the intervals: {[}-5/16, -1/16), {[}1/16,
5/16), {[}11/16, 15/16), {[}17/16, 21/16). (The reason we are using
these half-open intervals is to avoid questions about whether the
objects that are separated by the cut used to overlap.) This cutting and
moving can be accomplished in 1/4 of a second, without any piece
attaining a velocity higher than 1/2 inch per second, or an acceleration
higher than 4 inches per second per second.

The third stage of the cutting is to take each of these four pieces, cut
them in two, move the left-hand part of each of the four 1/64 of an inch
to the left, and the right-hand part 1/64 of an inch to the right. So
the eight pieces now occupy the intervals: {[}-21/64, -13/64),
{[}-11/64, -3/64), {[}3/64, 11/64), {[}13/64,~21/64), {[}43/64,~51/64),
{[}53/64,~61/64), {[}67/64,~75/64), {[}77/64,~85/64). Again, this
cutting and moving can be accomplished within 1/8 of a second, without
any piece attaining a velocity higher than 1/4 inch per second, or an
acceleration higher than 4 inches per second per second.\footnote{Note
  that, interestingly, if we moved the pieces 1/2 inch after the first
  round, 1/4 inch after the second round, 1/8 inch after the third round
  and so on then at the limit, each left and right edge that was once
  attached will have moved back together again. The process we have
  chosen preserves separation in a way that the aforementioned process
  does not.}

In general, at stage \emph{n}, we take the 2\textsuperscript{\emph{n}}
pieces, divide each of them in two, move the left-hand piece
1/2\textsuperscript{2\emph{n}} inches leftward, and the right-hand piece
1/2\textsuperscript{2\emph{n}} inches rightward. This can all be done in
1/2\textsuperscript{\emph{n}} seconds without any piece attaining a
velocity higher than 1/2\textsuperscript{\emph{n}-1} inches per second,
or an acceleration higher than 4 inches per second per second. So the
whole super-cut can be performed in 1 second: the first stage in 1/2
second, the second stage in 1/4 second, the third stage in 1/8 second,
and so on. Note, moreover, that the whole super-cut can be performed in
a second without the pieces ever moving at any huge velocity. If readers
doubted the possibility of super-cutting because they believed it to be
a necessary truth that no matters travels at or beyond the speed of
light, their doubts were misplaced: no piece of matter in the
super-cutting process approaches a superluminous velocity.

Further, in this kind of procedure, a quantity of matter that is
scattered during the super-cutting process remains scattered during the
process. To see this, first consider a particular example. We noted
above that at the second stage there were pieces occupying the intervals
{[}-5/16, -1/16) and {[}1/16, 5/16). Before this, the point 0 had been
occupied; at this stage a gap of 1/8 inch around 0 had been opened. This
gap keeps being closed at each stage. After the third stage there were
pieces occupying the intervals {[}/64, -3/64), {[}3/64, 11/64), so the
gap is now only 3/32 inch. After the fourth stage, there will be pieces
at {[}-27/256, -11/256), {[}11/256, 27/256), so the gap is now only
11/128 inch. This process will make the gap ever smaller, but will not
lead to its closure. As the process continues, the size of the gap will
approach 1/12 of an inch, but never cross that boundary. To see this,
note that the size of the gap in inches after stage \emph{n}
(\emph{n}~\({\geq}\) 3) is
1/8~-~1/2\textsuperscript{5}~-~1/2\textsuperscript{7}~-~\ldots~-~1/2\textsuperscript{2\emph{n}}.
The sum of the series
1/2\textsuperscript{5}~+~1/2\textsuperscript{7}~+~\ldots~is 1/24. Hence
the gap at stage \emph{n} is greater than 1/8~-~1/24~=~1/12. So once the
pieces around 0 have been separated, they will never be rejoined.

This result applies generally to all of the separated pieces in the
super-cut. Once a gap is created, parts of pieces from either side of
the gap are moved ever closer to the centre of the gap at every
subsequent stage of the super-cut. But since we decrease the distance
moved by each piece at each stage of the cut, and in particular decrease
it by a factor greater than 2, the pieces, once disjointed, will never
be united.

How is the matter arranged at the end of the super-cut? To answer this
question we need to assume that motion is continuous. For each part of
the object we can calculate its position function, the function from the
length of time the super-cut has been in progress to the position of the
part. At least, we can calculate this for all times until the end of the
super-cut. With the continuity assumption in place we can infer that its
position at the end of the cut is the limiting value of its position
function. So we make this assumption.

We assumed above that there is a Cartesian axis running along the
object; say that a part \emph{a} covers a point \emph{x} just in case
\emph{a} occupies some region {[}\emph{y}, \emph{z}), and
\emph{y}~\({\leq}\)~\emph{x} and \emph{z}~\textgreater~\emph{x}. When we
say \emph{a} occupies {[}\emph{y}, \emph{z}), we do not mean to imply it
occupies only that region, just that it occupies at least that region.
Assume then that a part \emph{a} occupies a point \emph{x}
(0~\({\leq}\)~\emph{x}~\textless~1), and that the binary representation
of \emph{x} is
0.\emph{x}\textsubscript{1}\emph{x}\textsubscript{2}\ldots{}\emph{x\textsubscript{n}}\ldots,
where for each \emph{x\textsubscript{i}}, \emph{x\textsubscript{i}}
equals 0 or 1, and for all \emph{i}, there exists a \emph{j}
\textgreater{} \emph{i} such that \emph{x\textsubscript{j}}~equals
zero.\footnote{The final condition is important to rule out numbers
  having two possible representations. For example, we have to choose
  whether the representation of 1/2 should be 0.1000\ldots{} or
  0.0111\ldots, and we somewhat arbitrarily, choose the former.} If
\emph{x}\textsubscript{1}~=~1, then \emph{x}~\({\geq}\)~1/2, so the some
part of \emph{a}, a small part that originally covered \emph{x}, will be
moved rightward at the first stage. It is possible that \emph{a} itself
may be split by the cut, but there will be a small part around \emph{x}
that is not split, and it will move rightward. If
\emph{x}\textsubscript{1}~=~0, then \emph{x}~\textless~1/2, so some part
of \emph{a}, a small part that originally covered \emph{x}, will be
moved leftward at the first stage. Indeed, in general some part of
\emph{a}, a small part that originally covered \emph{x}, will be moved
rightward at the \emph{n}'th stage if \emph{x\textsubscript{n}}~=~1, and
some part of \emph{a}, a small part that originally covered \emph{x},
will be moved leftward at the \emph{n}'th stage if
\emph{x\textsubscript{n}}~=~0.

Using the fact that a part gets moved 2\textsuperscript{-2\emph{n}}
inches at stage \emph{n}, we can infer that after \emph{n} stages, a
small part that originally covered \emph{x} and has not been split by
the cuts will cover the following point after \emph{n} cuts.
\[x + \frac{(-1)^{x_1 + 1}}{4} + \frac{(-1)^{x_2 + 1}}{16} + \dots + \frac{(-1)^{x_n + 1}}{2^{2n}}\]

Assuming continuity of motion, we can assume that \emph{a} will end up
with a part that eventually covers the following point, which we will
call \emph{f}(\emph{x}).
\[f(x) = x + \sum_{i=1}^{\infty}\frac{(-1)^{x_i + 1}}{2^{2i}}\]

From this, it follows immediately that for all \emph{x} in {[}0, 1),
\emph{f}(\emph{x}) will end up being occupied. It turns out that these
are the \emph{only} points that are occupied at the end of the
super-cut.

Assume that a point \emph{y} is occupied at the end of the super-cut. We
will construct a number \emph{c} such that
\emph{y}~=~\emph{f}(\emph{c}). Recall that we noted above that whenever
two pieces were separated, a gap was created between them that would
never be completely filled. While parts of the stick would move closer
and closer to the centre of that gap during the super-cut, the middle
two-thirds of the gap would never be reoccupied. That interval, that
would never be reoccupied, would be \emph{liberated}. The interval
{[}1/3, 2/3) is liberated at the first stage, the intervals {[}-1/24,
1/24) and {[}23/24, 25/24) are liberated at the second stage, the
intervals {[}-37/192, -35/192), {[}35/192, 37/192), {[}155/192, 157/192)
and {[}227/192, 229/192) are liberated at the third stage, and so on. If
\emph{y} is occupied, then \emph{y} must not be in any liberated
interval. Therefore it must be either to the left or to the right of any
interval that is liberated.

Let \emph{c}\textsubscript{1}~equal~0 if \emph{y} is to the left of the
first liberated interval, {[}1/3, 2/3), and 1 otherwise. Given the value
of \emph{c}\textsubscript{1}, it is already determined which side
\emph{y} is of one of the intervals liberated at the second stage. If
\emph{y} is to the left of {[}1/3, 2/3), for example, then it is to the
left of {[}23/24, 25/24). But the value of \emph{c}\textsubscript{1}
does not determine which side \emph{y} is of the other interval. Let
\emph{c}\textsubscript{2} equal 0 if \emph{y} is to the left of that
interval, and 1 otherwise. The values of \emph{c}\textsubscript{1} and
\emph{c}\textsubscript{2} determine which side \emph{y} is of three of
the four intervals liberated at the fourth stage, but leave open which
side it is of one of these four. Let \emph{c}\textsubscript{3} equal 0
if \emph{y} is to the left of that interval, 1 otherwise. If we repeat
this procedure for all stages, we will get values of
\emph{c\textsubscript{i}} for all \emph{i}. Let \emph{c} be the number
whose binary expansion is
0.\emph{c}\textsubscript{1}\emph{c}\textsubscript{2}\ldots{}\emph{c\textsubscript{n}}\ldots.
It follows that \emph{y}~=~\emph{f}(\emph{c}). The reason once it is
determined which side \emph{y} is of each of the liberated intervals,
\emph{y} has been determined to fall in an interval that is exactly one
point wide, and \emph{f}(\emph{c}) is in that interval, so
\emph{f}(\emph{c}) must equal \emph{y}. So \emph{y} is occupied iff for
some \emph{x}, \emph{y}~=~\emph{f}(\emph{x}). Say
\emph{S}~=~\{\emph{y}:~\({\exists}\)\emph{x}~(\emph{y}~=~\emph{f}(\emph{x}))\};
the conclusion is that all and only the points in \emph{S} are occupied.

Could a piece of gunk occupy the points in \emph{S}? Not given the
assumptions we have made so far. \emph{S} has two properties that might
not seem consistent at first glance. It is \emph{dense} in the sense
that for any point \emph{y} in \emph{S}, and any distance \({\delta}\),
there is another point \emph{z} in \emph{S} such that
\emph{y}~-~\emph{z}~\textless~\({\delta}\). But it is
\emph{disconnected} in the sense that for any two points \emph{y} and
\emph{z} in \emph{S}, there is an extended region \emph{r} between
\emph{y} and \emph{z} that is wholly unoccupied. The proofs of density
and disconnectedness are given in the appendix.

Given (3), disconnectedness is inconsistent with gunk occupying
\emph{S}. If a material object occupies \emph{S}, it must occupy the
points in \emph{S}. Let \emph{y} be any one of these points. By (3),
\emph{S} must occupy some extended region containing \emph{y}, say,
{[}\emph{y}\textsubscript{1}, \emph{y}\textsubscript{2}). Two cases to
consider. First case: \emph{y}\textsubscript{1}~\textless~\emph{y}. If
{[}\emph{y}\textsubscript{1},~\emph{y}\textsubscript{2})~\({\subset}\)
\emph{S}, then \emph{y}\textsubscript{1} and \emph{y} are in \emph{S},
and so are all the points in between them. Since the object occupies
\emph{S}, it follows that these points are occupied. Hence there is no
extended region between \emph{y}\textsubscript{1} and \emph{y} that is
wholly unoccupied, which is inconsistent with disconnectedness. Second
case: \emph{y}\textsubscript{1}~=~\emph{y}. Again,
{[}\emph{y}\textsubscript{1}, \emph{y}\textsubscript{2})
\({\subset}\)~\emph{S}, and since this interval is non-empty,
\emph{y}\textsubscript{2}~\textgreater~\emph{y}\textsubscript{1}. Hence
(\emph{y}\textsubscript{1} +~\emph{y}\textsubscript{2})~/~2 is greater
than \emph{y}\textsubscript{1}, and all the points between it and
\emph{y}\textsubscript{1} are occupied. This is also inconsistent with
disconnectedness. So given (3), no material object could occupy
\emph{S}.

In summary, (1) through (4) plus continuity of motion cannot be true
together. From (1), (2), and (4), we inferred that our super-cutting
process was possible, and that it would not destroy any quantity of
matter (though of course it would scatter it). Assuming continuity of
motion, we calculated which points would be occupied after the
super-cut. By (3) we concluded that no piece of gunk could occupy those
points, or indeed any subset of them, yielding an inconsistent result.
Suppose that the continuity of motion thesis is dropped. We can then
maintain (1) to (4) with consistency. One should note, however, that a
world where (1) to (4) holds would be a strange world indeed: if
super-cutting is performed, the pieces of gunk would have to jump
location at the limit. The gunk cannot occupy S: but in order to occupy
a different set of points, various quantities of matter would have to
jump position at the limit.

If one believes in gunk one has a choice: Abandon one or more of (2) to
(4) or else deny that it is nomologically necessary that motion be
continuous. Which assumption should be dropped? We leave it to the gunk
lover to select the most tolerable package. The choice for the gunk
lover is a little unenviable. Those who are attracted to the view that
the actual world is gunky are very much wedded to (1) and (3). When
philosophers take seriously the idea that that matter has parts all the
way down\footnote{See, for example,
  (\citeproc{ref-Zimmerman1996}{Zimmerman 1996}).}, they do not imagine
conjoining that thesis with point sized parts, or else immaterial
parts\footnote{Leibniz, with his monads, is an exception of course. No
  contemporary gunk lover wants a monadology, however.}, or else
quantities of matter that are as thin as a plane, and so on. With a
commitment to (1) and (3) in place, super-cutting will be loaded with
physical significance. Accept that the laws of nature permits
super-cutting and one will be committed to either denying the
conservation of matter or the continuity of motion.

\section*{Appendix}\label{appendix}
\addcontentsline{toc}{section}{Appendix}

To prove density, note that if \emph{y} is occupied, there is a point
\emph{x} with binary representation
0.\emph{x}\textsubscript{1}\emph{x}\textsubscript{2}\ldots{} such that
\emph{y}~=~\emph{f}(\emph{x}). For any positive \({\delta}\), there is
an integer \emph{n} such that
\({\delta}\)~\textgreater~2\textsuperscript{-2\emph{n}}. Let \emph{v} be
the number represented by
0.\emph{x}\textsubscript{1}\emph{x}\textsubscript{2}\ldots{}\emph{x\textsubscript{n}x\textsubscript{n}}\textsubscript{+1}′\emph{x\textsubscript{n}}\textsubscript{+2}\emph{x\textsubscript{n}}\textsubscript{+3}\ldots,
where \emph{x\textsubscript{n}}\textsubscript{+1}′~=~1 iff
\emph{x\textsubscript{n}}\textsubscript{+1}~=~0, and
\emph{x\textsubscript{n}}\textsubscript{+1}′~=~0 otherwise. The
difference between \emph{f}(\emph{x}) and \emph{f}(\emph{v}) will be
exactly 2\textsuperscript{-2\emph{n}-1}. Since \emph{f}(\emph{v}) is
occupied, and \emph{y}~=~\emph{f}(\emph{x}), there is an occupied point
exactly 2\textsuperscript{-2\emph{n}-1} inches from \emph{y}, so there
is a point less than \({\delta}\) inches from \emph{y}, as required.

To prove disconnectedness, let \emph{y} and \emph{z} be any two distinct
occupied points. So for some distinct \emph{v},~\emph{x},
\emph{y}~=~\emph{f}(\emph{x}) and \emph{z}~=~\emph{f}(\emph{v}). Say
that the binary representation of \emph{x} is
0.\emph{x}\textsubscript{1}\emph{x}\textsubscript{2}\ldots, and the
binary representation of \emph{v} is
0.\emph{v}\textsubscript{1}\emph{v}\textsubscript{2}\ldots{} Let
\emph{j} be the lowest number such that
\emph{x\textsubscript{j}}~\({\neq}\)~\emph{v\textsubscript{j}}. (Since
\emph{x} and \emph{v} are distinct, there must be at least one value
\emph{j}.) Without loss of generality, assume that
\emph{x\textsubscript{j}}~=~0~and \emph{v\textsubscript{j}}~=~1. (There
is no loss of generality because we are just trying to show that between
any two occupied points there is a gap, so it does not matter which of
the two points is the rightward one.) Let \emph{k} be the number with
binary representation
0.\emph{x}\textsubscript{1}\emph{x}\textsubscript{2}\ldots{}\emph{x\textsubscript{j}}1,
and let \emph{l}\textsubscript{2} be \emph{f}(\emph{k}). Finally, define
\emph{l}\textsubscript{1} by the following equation:
\[l_i = \sum_{i=1}^j \frac{(-1)^{x_i+1}}{2^{2i}} + \sum_{i = j+1}^\infty \frac{1}{2^{2i}}\]

It is easy enough to see that \emph{f}(\emph{x}), that is \emph{y}, must
be less that \emph{l}\textsubscript{1}. For \emph{l}\textsubscript{1} is
the value that \emph{f}(\emph{x}) would take were every digit in the
binary expansion of \emph{x} after \emph{j} be 1. But by definition
there must be some value \emph{j}′~\textgreater~\emph{j} such that
\emph{x\textsubscript{j}}′~=~0. From this it follows that:
\[\sum_{i = j+1}^\infty \frac{1}{2^{2i}} > \sum_{i = j+1}^\infty \frac{(-1)^{x_i+1}}{2^{2i}}\]

And from that it follows that
\emph{l}\textsubscript{1}~\textgreater~\emph{f}(\emph{x}). Indeed, by
similar reasoning, it follows that for all \emph{u}~\textless~\emph{k},
\emph{f}(\emph{u}) \textless~\emph{l}\textsubscript{1}. Since \emph{f}
is monotone increasing, it also follows that for all
\emph{u}~\({\geq}\)~\emph{k},
\emph{f}(\emph{u})~\({\geq}\)~\emph{l}\textsubscript{2}. And from those
facts, it follows that there does not exist a \emph{u} such that
\emph{f}(\emph{u})~\({\in}\)~{[}\emph{l}\textsubscript{1},~\emph{l}\textsubscript{2}).
And since
\emph{y}~\textless~\emph{l}\textsubscript{1}~\textless~\emph{l}\textsubscript{2}~\({\leq}\)~\emph{z},
this implies that there is an extended unoccupied region between
\emph{y} and \emph{z}, as required.

\subsection*{References}\label{references}
\addcontentsline{toc}{subsection}{References}

\phantomsection\label{refs}
\begin{CSLReferences}{1}{0}
\bibitem[\citeproctext]{ref-Bernadete1964}
Bernadete, Jose. 1964. \emph{Infinity: An Essay in Metaphysics}. Oxford:
Clarendon Press.

\bibitem[\citeproctext]{ref-Leibniz1998}
Leibniz, Gottfried Wilhelm. 1998. \emph{Philosophical Texts}. Translated
by R. S. Woolhouse and Richard Francks. Oxford: Oxford University Press.

\bibitem[\citeproctext]{ref-Newton1952}
Newton, Isaac. 1952. \emph{Opticks}. New York: Dover Press.

\bibitem[\citeproctext]{ref-Zimmerman1996}
Zimmerman, Dean. 1996. {``Could Extended Objects Be Made Out of Simple
Parts: An Argument for Atomless Gunk.''} \emph{Philosophy and
Phenomenological Research} 56 (1): 1--29. doi:
\href{https://doi.org/10.2307/2108463}{10.2307/2108463}.

\end{CSLReferences}



\noindent Published in\emph{
The Monist}, 2004, pp. 339-350.

\end{document}
