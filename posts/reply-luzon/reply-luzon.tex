% Options for packages loaded elsewhere
% Options for packages loaded elsewhere
\PassOptionsToPackage{unicode}{hyperref}
\PassOptionsToPackage{hyphens}{url}
%
\documentclass[
  11pt,
  letterpaper,
  DIV=11,
  numbers=noendperiod,
  twoside]{scrartcl}
\usepackage{xcolor}
\usepackage[left=1.1in, right=1in, top=0.8in, bottom=0.8in,
paperheight=9.5in, paperwidth=7in, includemp=TRUE, marginparwidth=0in,
marginparsep=0in]{geometry}
\usepackage{amsmath,amssymb}
\setcounter{secnumdepth}{3}
\usepackage{iftex}
\ifPDFTeX
  \usepackage[T1]{fontenc}
  \usepackage[utf8]{inputenc}
  \usepackage{textcomp} % provide euro and other symbols
\else % if luatex or xetex
  \usepackage{unicode-math} % this also loads fontspec
  \defaultfontfeatures{Scale=MatchLowercase}
  \defaultfontfeatures[\rmfamily]{Ligatures=TeX,Scale=1}
\fi
\usepackage{lmodern}
\ifPDFTeX\else
  % xetex/luatex font selection
  \setmainfont[ItalicFont=EB Garamond Italic,BoldFont=EB Garamond
SemiBold]{EB Garamond Math}
  \setsansfont[]{EB Garamond}
  \setmathfont[]{Garamond-Math}
\fi
% Use upquote if available, for straight quotes in verbatim environments
\IfFileExists{upquote.sty}{\usepackage{upquote}}{}
\IfFileExists{microtype.sty}{% use microtype if available
  \usepackage[]{microtype}
  \UseMicrotypeSet[protrusion]{basicmath} % disable protrusion for tt fonts
}{}
\usepackage{setspace}
% Make \paragraph and \subparagraph free-standing
\makeatletter
\ifx\paragraph\undefined\else
  \let\oldparagraph\paragraph
  \renewcommand{\paragraph}{
    \@ifstar
      \xxxParagraphStar
      \xxxParagraphNoStar
  }
  \newcommand{\xxxParagraphStar}[1]{\oldparagraph*{#1}\mbox{}}
  \newcommand{\xxxParagraphNoStar}[1]{\oldparagraph{#1}\mbox{}}
\fi
\ifx\subparagraph\undefined\else
  \let\oldsubparagraph\subparagraph
  \renewcommand{\subparagraph}{
    \@ifstar
      \xxxSubParagraphStar
      \xxxSubParagraphNoStar
  }
  \newcommand{\xxxSubParagraphStar}[1]{\oldsubparagraph*{#1}\mbox{}}
  \newcommand{\xxxSubParagraphNoStar}[1]{\oldsubparagraph{#1}\mbox{}}
\fi
\makeatother


\usepackage{longtable,booktabs,array}
\usepackage{calc} % for calculating minipage widths
% Correct order of tables after \paragraph or \subparagraph
\usepackage{etoolbox}
\makeatletter
\patchcmd\longtable{\par}{\if@noskipsec\mbox{}\fi\par}{}{}
\makeatother
% Allow footnotes in longtable head/foot
\IfFileExists{footnotehyper.sty}{\usepackage{footnotehyper}}{\usepackage{footnote}}
\makesavenoteenv{longtable}
\usepackage{graphicx}
\makeatletter
\newsavebox\pandoc@box
\newcommand*\pandocbounded[1]{% scales image to fit in text height/width
  \sbox\pandoc@box{#1}%
  \Gscale@div\@tempa{\textheight}{\dimexpr\ht\pandoc@box+\dp\pandoc@box\relax}%
  \Gscale@div\@tempb{\linewidth}{\wd\pandoc@box}%
  \ifdim\@tempb\p@<\@tempa\p@\let\@tempa\@tempb\fi% select the smaller of both
  \ifdim\@tempa\p@<\p@\scalebox{\@tempa}{\usebox\pandoc@box}%
  \else\usebox{\pandoc@box}%
  \fi%
}
% Set default figure placement to htbp
\def\fps@figure{htbp}
\makeatother


% definitions for citeproc citations
\NewDocumentCommand\citeproctext{}{}
\NewDocumentCommand\citeproc{mm}{%
  \begingroup\def\citeproctext{#2}\cite{#1}\endgroup}
\makeatletter
 % allow citations to break across lines
 \let\@cite@ofmt\@firstofone
 % avoid brackets around text for \cite:
 \def\@biblabel#1{}
 \def\@cite#1#2{{#1\if@tempswa , #2\fi}}
\makeatother
\newlength{\cslhangindent}
\setlength{\cslhangindent}{1.5em}
\newlength{\csllabelwidth}
\setlength{\csllabelwidth}{3em}
\newenvironment{CSLReferences}[2] % #1 hanging-indent, #2 entry-spacing
 {\begin{list}{}{%
  \setlength{\itemindent}{0pt}
  \setlength{\leftmargin}{0pt}
  \setlength{\parsep}{0pt}
  % turn on hanging indent if param 1 is 1
  \ifodd #1
   \setlength{\leftmargin}{\cslhangindent}
   \setlength{\itemindent}{-1\cslhangindent}
  \fi
  % set entry spacing
  \setlength{\itemsep}{#2\baselineskip}}}
 {\end{list}}
\usepackage{calc}
\newcommand{\CSLBlock}[1]{\hfill\break\parbox[t]{\linewidth}{\strut\ignorespaces#1\strut}}
\newcommand{\CSLLeftMargin}[1]{\parbox[t]{\csllabelwidth}{\strut#1\strut}}
\newcommand{\CSLRightInline}[1]{\parbox[t]{\linewidth - \csllabelwidth}{\strut#1\strut}}
\newcommand{\CSLIndent}[1]{\hspace{\cslhangindent}#1}



\setlength{\emergencystretch}{3em} % prevent overfull lines

\providecommand{\tightlist}{%
  \setlength{\itemsep}{0pt}\setlength{\parskip}{0pt}}



 


\setlength\heavyrulewidth{0ex}
\setlength\lightrulewidth{0ex}
\usepackage[automark]{scrlayer-scrpage}
\clearpairofpagestyles
\cehead{
  Brian Weatherson
  }
\cohead{
  Akrasia and Traitors
  }
\ohead{\bfseries \pagemark}
\cfoot{}
\makeatletter
\newcommand*\NoIndentAfterEnv[1]{%
  \AfterEndEnvironment{#1}{\par\@afterindentfalse\@afterheading}}
\makeatother
\NoIndentAfterEnv{itemize}
\NoIndentAfterEnv{enumerate}
\NoIndentAfterEnv{description}
\NoIndentAfterEnv{quote}
\NoIndentAfterEnv{equation}
\NoIndentAfterEnv{longtable}
\NoIndentAfterEnv{abstract}
\renewenvironment{abstract}
 {\vspace{-1.25cm}
 \quotation\small\noindent\emph{Abstract}:}
 {\endquotation}
\newfontfamily\tfont{EB Garamond}
\addtokomafont{disposition}{\rmfamily}
\addtokomafont{title}{\normalfont\itshape}
\let\footnoterule\relax

\makeatletter
\renewcommand{\@maketitle}{%
  \newpage
  \null
  \vskip 2em%
  \begin{center}%
  \let \footnote \thanks
    {\itshape\huge\@title \par}%
    \vskip 0.5em%  % Reduced from default
    {\large
      \lineskip 0.3em%  % Reduced from default 0.5em
      \begin{tabular}[t]{c}%
        \@author
      \end{tabular}\par}%
    \vskip 0.5em%  % Reduced from default
    {\large \@date}%
  \end{center}%
  \par
  }
\makeatother
\RequirePackage{lettrine}

\renewenvironment{abstract}
 {\quotation\small\noindent\emph{Abstract}:}
 {\endquotation\vspace{-0.02cm}}
\KOMAoption{captions}{tableheading}
\makeatletter
\@ifpackageloaded{caption}{}{\usepackage{caption}}
\AtBeginDocument{%
\ifdefined\contentsname
  \renewcommand*\contentsname{Table of contents}
\else
  \newcommand\contentsname{Table of contents}
\fi
\ifdefined\listfigurename
  \renewcommand*\listfigurename{List of Figures}
\else
  \newcommand\listfigurename{List of Figures}
\fi
\ifdefined\listtablename
  \renewcommand*\listtablename{List of Tables}
\else
  \newcommand\listtablename{List of Tables}
\fi
\ifdefined\figurename
  \renewcommand*\figurename{Figure}
\else
  \newcommand\figurename{Figure}
\fi
\ifdefined\tablename
  \renewcommand*\tablename{Table}
\else
  \newcommand\tablename{Table}
\fi
}
\@ifpackageloaded{float}{}{\usepackage{float}}
\floatstyle{ruled}
\@ifundefined{c@chapter}{\newfloat{codelisting}{h}{lop}}{\newfloat{codelisting}{h}{lop}[chapter]}
\floatname{codelisting}{Listing}
\newcommand*\listoflistings{\listof{codelisting}{List of Listings}}
\makeatother
\makeatletter
\makeatother
\makeatletter
\@ifpackageloaded{caption}{}{\usepackage{caption}}
\@ifpackageloaded{subcaption}{}{\usepackage{subcaption}}
\makeatother
\usepackage{bookmark}
\IfFileExists{xurl.sty}{\usepackage{xurl}}{} % add URL line breaks if available
\urlstyle{same}
\hypersetup{
  pdftitle={Akrasia and Traitors},
  pdfauthor={Brian Weatherson},
  hidelinks,
  pdfcreator={LaTeX via pandoc}}


\title{Akrasia and Traitors}
\author{Brian Weatherson}
\date{2025}
\begin{document}
\maketitle
\begin{abstract}
Bar Luzon argues that akrasia is irrational because it leads to
violating a principle called \textbf{Avoid Treachery}. In response, I
argue that Avoid Treachery is insufficiently motivated, that it
presupposes a picture of rational inference that defenders of akrasia
have independent reason to reject, and that there are models in which
Avoid Treachery is false.
\end{abstract}


\setstretch{1.1}
\section{The Debate}\label{sec-intro}

\noindent A prominent debate in recent epistemology has been whether it
can ever be rational to believe propositions of the form of \textbf{SA},
or of some similar forms.

\begin{description}
\tightlist
\item[SA]
\emph{q} and it is irrational for me to believe \emph{q}.
\end{description}

The \emph{enkratic} philosopher says all beliefs of that form are
irrational. The \emph{anti-enkratic} philosopher says that they are
sometimes rational.

The debate here isn't always about \textbf{SA} (Simple Akrasia); some
philosophers focus on \textbf{LA} (Likely Akrasia).

\begin{description}
\tightlist
\item[LA]
\emph{q} and it is probably irrational for me to believe \emph{q}.
\end{description}

The difference between these will be important, especially in
Section~\ref{sec-aim}, because Timothy Williamson
(\citeproc{ref-Williamson2000}{2000},
\citeproc{ref-Williamson2011}{2011},
\citeproc{ref-Williamson2014}{2014}) has offered the most influential
arguments for the anti-enkrasia position about \textbf{LA}, but agrees
with the enkratic philosopher about \textbf{SA}.

Recently in this journal, Bar Luzon
(\citeproc{ref-Luzon2025}{Forthcoming}) has argued on the side of the
enkratic philosophers about \textbf{SA}. Rather than start with a mere
appeal to intuitions, as many in this debate do, she starts with a
principle she calls \textbf{Avoid Treachery}.

\begin{description}
\tightlist
\item[Avoid Treachery (AT)]
For every proposition \emph{p} and for every positive epistemic status
\emph{E}, if one knows that {[}\emph{p} has \emph{E} for one only if
\emph{p} is false{]}, then one ought not believe \emph{p}.
\end{description}

In this principle, E ranges over the statuses epistemic justification,
epistemic rationality, evidential support and epistemic permissibility,
and the conditional is a material conditional. The `ought' is purely
epistemic; if one thought belief in God was justified on Pascalian
grounds one wouldn't be moved by an argument from \textbf{AT}. So I'll
take `one ought not believe \emph{p}' to just be that it's
(epistemically) irrational to believe that \emph{p}. So we can formalise
\textbf{AT} as \textbf{AT Formalised}. In it, KA is that Hero knows A,
RA is that Hero rationally beliefs A, and E picks out one of the four
statuses from the start of the paragraph. Whichever one E picks out, EA
is that \emph{p} has that status for Hero.

\begin{description}
\tightlist
\item[AT Formalised]
K¬(\emph{p} ∧ E\emph{p}) → ¬R\emph{p}
\end{description}

The argument for the irrationality of \textbf{SA} follows pretty
quickly. Let \emph{p} be (\emph{q} ∧ ¬R\emph{q}). Assume E(A ∧ B)
implies EA, that RA implies EA, and that Hero knows anything that can be
proven in a few lines of logic. Then it's easy to show K¬(\emph{p} ∧
E\emph{p}), and hence ¬R\emph{p}, which just is the enkratic view.

The point of this note is to argue that the anti-enkratic philosopher
mostly has good reasons to reject \textbf{AT}. I say mostly because
there is one argument for \textbf{AT} that might work, but only if like
Williamson one treats \textbf{SA} and \textbf{LA} differently.

It would be too easy to say that the anti-enkratic view view implies
\textbf{AT} is false. Of course it does, since Luzon's argument against
from \textbf{AT} to the enkratic view is valid! What I want to argue is
that the reasons behind the anti-enkratic view give us somewhat
independent reason to reject \textbf{AT}. I'm going to offer the
following arguments against \textbf{AT} in sections \ref{sec-status} to
\ref{sec-model}.

\begin{enumerate}
\def\labelenumi{\arabic{enumi}.}
\tightlist
\item
  \textbf{AT} fails for other nearby values of E, and this undermines
  the motivation for believing it holds for these values.
\item
  The main argument for \textbf{AT} turns on an understanding of what it
  means to say evidence is a guide to truth that the anti-enkratic
  philosopher rejects.
\item
  An argument for \textbf{AT} from the idea that beliefs violating
  \textbf{AT} would be `self-undermining' at most supports the enkratic
  view about \textbf{SA}, not \textbf{LA}.
\item
  There are plausible models for evidence and belief where \textbf{AT}
  fails.
\end{enumerate}

But first it helps to rehearse the arguments for the anti-enkratic view,
to see how these objections flow from them.

\section{The Arguments}\label{sec-arguments}

Simplifying greatly, the anti-enkratic view relies on one
presupposition, followed by one of two (independent) arguments. The
presupposition is easiest to see with an example.

Hero has a faculty meeting today, but they have forgotten about it.
Fortunately, they just got a reminder email from their chair saying
there's a meeting today. Now they believe, indeed know, there's a
meeting today.

The presupposition is that the following three things are in principle
distinct.

\begin{enumerate}
\def\labelenumi{\arabic{enumi}.}
\tightlist
\item
  Hero's reason for believing that there is a meeting today, i.e., the
  email they got from the chair.
\item
  The facts that make the email from the chair a reason to believe there
  is a meeting today. Just what those are turns on the full theory of
  testimony, but presumably they include things like the chair's
  reliability, the frequency of emails being faked, and so on.
\item
  The reasons Hero has for believing that the email is a reason to
  believe there's a meeting today.
\end{enumerate}

The presupposition is that 1 and 3 are distinct. The reason that they
are distinct is that 1 and 2 are distinct, and 3 requires Hero to have
thoughts about (or at least sensitive to) 2, while 1 does not.

With that in place, the first argument for the anti-enkrasia view starts
with anti-exceptionalism about epistemology.\footnote{For
  anti-exceptionalism about logic, see Martin and Thomassen Hjortland
  (\citeproc{ref-MartinHjortland2024}{2024}). This kind of argument is
  particularly prominent in Lasonen-Aarnio
  (\citeproc{ref-Lasonen-Aarnio2020}{2020}).} Just like Hero might not
know descriptive facts like when the meeting is, she might not know
epistemological facts like just why the email is a reason to believe its
content. If Hero can reasonably have false beliefs about descriptive
facts, she can have false beliefs about what makes something a reason to
believe.\footnote{If, like Williamson, one denies that false beliefs can
  be reasonable, one will treat \textbf{SA} and \textbf{LA} differently.
  As noted earlier, I'm mostly ignoring that distinction here.} If those
beliefs are \emph{false}, she could reasonably believe that the meeting
is today, while reasonably believing that 2 fails to obtain.

The second argument relies on formal models, like the model of
Williamson's unmarked clock, in which the formal translations of
\textbf{SA} (or at least \textbf{LA}) are rationally believed. What's
distinctive about these formal models is that while agents know the
epistemic facts, they know what is rational to believe in what
situation, they don't know what situation they are in. It makes the
discussion clearer to have a concrete theory about what is rational in a
situation, so I'll work with a very crude form of evidentialism.
(Everything I say could, with some work, be repurposed for an argument
that makes different assumptions about what facts about a situation are
relevant, but this is an easy one to work with.) In particular I'll
assume:

\begin{itemize}
\tightlist
\item
  What's rational to believe supervenes on one's evidnce;
\item
  One's evidence is all and only what one knows.
\item
  It is rational for a person whose evidence is E to believe \emph{p}
  iff Pr(\emph{p} \textbar{} E) ⩾ 0.9, where Pr is the evidential
  probability function.
\end{itemize}

Again, I'm not saying this theory is true; in fact it's completely
implausible. What matters is that (a) what's rational to believe varies
from one situation to another, and (b) someone might not know precisely
what situation they are in, just like they might be ignorant of any
empirical fact.

Assuming evidentialism lets us distinguish two ways in which one might
be ignorant about one's situation.

\begin{itemize}
\tightlist
\item
  One might know \emph{p}, but not know one knows it.
\item
  One might not know \emph{p}, but not know one doesn't know it.
\end{itemize}

Williamson's models typically assume the first kind of ignorance, and
this has been rather controversial. But as I'll discuss in
Section~\ref{sec-model}, we can get the problem going with just the
second kind of ignorance.

So anti-enkratic philosophers have employed two kinds of strategies:
argue that people can believe \textbf{SA} (\textbf{LA}) because they can
rationally have false beliefs (lack true beliefs) about what is rational
in a situation, or because they can rationally have false beliefs (lack
true beliefs) about what situation they are in. These strategies seem to
exhaust the options for the anti-enkratic philosopher. If agents always
know which situation they are in, and know what's rational in every
situation, they'll know what's rational for them. So they can't
rationally believe \emph{p} and not know they rationally believe it. But
both strategies seem promising.

\section{Other Statuses}\label{sec-status}

The first reason to be sceptical of \textbf{AT} is that it doesn't hold
for some nearby statuses a proposition might have. A simple case is that
since one can rationally believe \emph{p} without having Cartesian
certainty that it's true, if we took E to be Cartesian certainty the
principle, \textbf{AT} would be clearly false.

More interestingly, consider the case where E is \emph{is provable in
Peano Arithmetic}. That's not really an \emph{epistemic} status, since
it doesn't refer to an agent. But it's interesting to note how
\textbf{AT} fails for this value of E. If \emph{p} is that Peano
Arithmetic is consistent, then Hero knows that \emph{p} is E iff
\emph{p} is false. But that's no reason to reject \emph{p}; indeed, Hero
should believe \emph{p}.

The point here is not that epistemic rationality is so analogous to
Cartesian certainty, or provability in Peano arithmetic that we can
simply argue by analogy that since \textbf{AT} fails when E is one of
the latter statuses, it fails when E is epistemic rationality. That's a
weak analogy; there are too many differences between them.\footnote{I'm
  grateful here to a reviewer for steering me away from a not very
  plausible argument.} A better argument is that noting \textbf{AT}
fails for these latter statuses puts a constraint on what any argument
for an instance of \textbf{AT} must look like. An argument that
\textbf{AT} holds when E is epistemic rationality better not generalise
to an argument that \textbf{AT} holds for these two latter statuses.
That would be a clear case of overgeneration.

That last claim is what I'll argue for in the rest of this paper. The
arguments that Luzon offers for \textbf{AT} are mostly arguments that do
overgenerate.

\section{Guide}\label{sec-guide}

The example of provability in Peano Arithmetic is relevant to the main
argument Luzon gives for \textbf{AT}. She argues that \textbf{AT} must
be true for the values of E she presents because if it fails, E can't be
a good guide to truth. Since justification, rationality, etc are guides
to truth, \textbf{AT} must be true.

The simplest response is that this claim about \textbf{AT} can't be
right in general because provability in Peano Arithmetic is a good guide
to truth when discussing the natural numbers, but \textbf{AT} fails when
E is provability in Peano Arithmetic. Provability is a good guide to
arithmetic truth in general, even if there are cases where it is not in
fact a good guide.

What would be implausible is to say that provability is the \emph{only}
guide to truth, but \textbf{AT} fails for it. Assuming that any reason
is a guide to truth, then if provability were the only guide, we'd have
no reason to believe that arithmetic is consistent. Assuming also that
it is irrational to believe something we have no reason to believe, it
would follow that we're irrational to believe that arithmetic is
consistent, and we would not in fact have a counterexample to
\textbf{AT}.

The argument from the last paragraph generalises. In general, it seems
incoherent to say of any E that it is the only guide to truth, but
\textbf{AT} fails for it. It would be nice here to have a theory about
what it takes for something to be a guide, but we don't need one for the
argument to work. As long as being a reason is sufficient to being a
guide, the argument goes through. But the key point here is that this
argument only goes through on the assumption that E is the only guide.
If there is another guide, the argument fails. Just what that other
guide is in arithmetic, whether it is Gödelian intuition, or
diagrammatic reasoning, or something else, is controversial.\footnote{The
  discussion of ``proof chauvinism'' about explanation in D'Alessandro
  (\citeproc{ref-DAlessandro2020}{2020}) is helpful here, though his
  focus is on explanation not knowledge.} But all that matters is that
there is some other guide, which there must be if we rationally believe
Peano arithmetic is consistent.

At this point you might think it matters that Luzon restricted E to
things like evidential support. Surely the evidentialist does think that
evidence is the only guide to truth. Here the presupposition I noted in
Section~\ref{sec-arguments} is important. When Hero believes that
there's a meeting today, her guide is not that she has evidence for
this: it's the email. She's guided by the fact that she received this
email, not by the fact that it's evidence. If she checks her computer
and sees snow is forecast, her belief that it will snow is guided by
something different. That's so even though there are a few descriptions
we can give which make it look like she is guided by the same thing. In
both cases, for instance, she is guided by words on her computer screen.
In the same sense, she is guided in both cases by her evidence. But in
the most important sense, the email and the weather forecast are
different guides.

This I suspect is ultimately the biggest difference between the enkratic
and the anti-enkratic philosopher. The enkratic philosopher thinks that
all beliefs are guided by the same thing: one's evidence. The
anti-enkratic philosopher thinks different beliefs are guided by
different things: the different pieces of evidence. \textbf{AT} is not,
I say, a good constraint on E being a guide, but it is a good constraint
on E being the only guide. So the anti-enkratic philosopher who
distinguishes facts which constitute evidence from the fact that that
fact is a piece of evidence, has good reason to reject AT.

\section{Undermining}\label{sec-aim}

The other big argument for \textbf{AT} is that if \textbf{AT} holds, a
belief that \emph{p} would be self-undermining, and hence irrational.
Presumably this means that the belief couldn't achieve it's aim or goal.
What it is to undermine someone is to stop them achieving their aim or
goal, so to be self-undermining is to do this to yourself.

Whether \textbf{AT} implies this depends on what one thinks the aim of
belief is. If it's truth, then \textbf{AT} doesn't have this
implication. It could be that \textbf{AT} is true with E something like
evidential support, and still \emph{p} is true. Indeed, part of what's
puzzling about enkratic arguments is that beliefs like \emph{p and I'm
irrational to believe p} could well be true. I often have irrational
beliefs, and sometimes I'm lucky and they're true!

So the argument must assume a stronger view about the aim of belief. A
natural thought is that the aim of belief is knowledge. Here I think the
argument for \textbf{AT} does go through. If the aim of belief is
knowledge, then if \emph{p} satisfies the antecedent of \textbf{AT} it
can't possibly satisfy it's aim, and it's irrational to do something
that can't satisfy it's aim. So given that aim, Luzon's argument goes
through, and the enktratic philosopher is right to say that any belief
of the form \textbf{SA} is irrational.

Note here that the overgeneralisation worries that I've been leaning on
simply don't apply. No one thinks the aim of belief is Cartesian
certainty, or provability in Peano arithmetic. Violations of \textbf{AT}
for the case where E is one of those statuses are not self-undermining
because belief does not aim for those statuses. This is another reason
to not think the examples in Section~\ref{sec-status} don't work as
arguments by analogy; one important feature of the statuses Luzon
stresses is that they are all plausible aims (or entailments of aims) of
belief.

The argument for \textbf{AT} from the aim of belief generalises in one
important direction, but not in another. The important direction is if
the aim of belief is such that beliefs that achieve their aim are (a)
true and (b) satisfy E, then any belief which satisfies the antecedent
of \textbf{AT} will be self-defeating, and hence irrational. So any such
hypothesis about the aim of belief will imply, via \textbf{AT} that any
belief of the form \textbf{SA} is irrational. The direction in which it
does not generalise is that there isn't an argument here for
\textbf{LA}. This can be seen from the fact that Williamson endorses
knowledge as the aim of belief (and says that beliefs like \textbf{SA}
are irrational), but also says that beliefs like \textbf{LA} can be
rational. Assuming his position is even coherent, which I think we have
good reason to believe is true, the anti-enkratic philosopher can
coherently say that there is no argument against their position about
\textbf{LA} from considerations about the aim of belief.

We could at this point debate about whether the core debate here
concerns \textbf{SA} or \textbf{LA}, or about what akrasia/enkrasia
really is. This does not strike me as a productive line of inquiry. It's
better, I think, to note the space of possibilities here.

\begin{enumerate}
\def\labelenumi{\arabic{enumi}.}
\tightlist
\item
  On weak theories of the aim of belief, e.g., where the aim is merely
  truth, the argument for \textbf{AT} does not go through.
\item
  On strong theories of the aim of belief, where satisfaction of the aim
  implies truth and rationality, the argument for \textbf{AT} does work,
  and the anti-akratic philosopher is correct, but only about
  \textbf{SA} not \textbf{LA}.
\item
  Either way, there isn't yet an argument here for the anti-akratic view
  about \textbf{LA}. Such an argument would require an extra premise
  that it is irrational to have a belief which is probably
  self-undermining.
\end{enumerate}

So to wrap up, I'll make a small note about the status of that missing
premise, that it is irrational to have a belief that probably doesn't
achieve its aim.

\section{Formal Models}\label{sec-model}

In several places, Williamson has used formal models to show that
\textbf{LA} is compatible with the knowledge aim of belief. I think
these arguments are perfectly sound, but they have been criticised in a
number of ways. The following four stand out.

\begin{enumerate}
\def\labelenumi{\arabic{enumi}.}
\tightlist
\item
  The epistemic accessibility relation in the model is intransitive, so
  the the KK principle fails (Das and Salow
  (\citeproc{ref-DasSalow2019}{2019})).
\item
  The epistemic accessibility relation in the model is not nested, so
  intuitive principles about the value of evidence fail (Geanakoplos
  (\citeproc{ref-Geanakoplos1989}{{[}1989{]} 2021}), Dorst et al.
  (\citeproc{ref-DorstEtAl2021}{2021})).
\item
  The models are cases where the probability of the target proposition
  is not a good guide to its truth. (Horowitz
  (\citeproc{ref-Horowitz2014}{2014}))
\item
  The models assume that updating is by conditionalisation on one's
  evidence, even when isn't sure precisely what one's evidence is.
  (Gallow (\citeproc{ref-Gallow2021}{2021}))
\end{enumerate}

I'm going to present a model where \textbf{AT} fails\footnote{For ease
  of expression, from now on I mean \textbf{AT} to just mean the version
  of it where E is epistemic rationality.} (and so does \textbf{LA})
even though the model is modified to avoid the first three objections.
That is, I'll present a model where epistemic accessibility is
transitive and nested, and in a sense I'll make precise probability is a
good guide to truth, but where \textbf{AT} is still false. I won't have
anything to say about the fourth objection; I think there are good
defences of that assumption, but it would be a massive digression to
present them here.

Onto the model. There is a random variable X whose prior probability is
a uniform distribution over {[}0, 1{]}. If X = \emph{x}, Hero will learn
X ⩽ \emph{x}. That is, from the world X = \emph{x}, all worlds X =
\emph{y} are possible, as long as \emph{y} ⩽ \emph{x}. This
accessibility relation is clearly transitive and nested.

Hero will update on what they learn, i.e.~X ⩽ \emph{x}. I'll use Pr for
the initial probability of some proposition, and Cr for Hero's credence
after learning X ⩽ \emph{x}.

I'm going to focus primarily on propositions of the form X \(\in\)
(\emph{a}, \emph{b}), where 0 \textless{} \emph{a} \textless{} \emph{b}
\textless{} 1. Call this proposition \emph{i}, for interval. There are
three interesting possibilities for \emph{Cr}(\emph{i}).

\begin{enumerate}
\def\labelenumi{\arabic{enumi}.}
\tightlist
\item
  If X ⩽ \emph{a}, then \emph{Cr}(\emph{i}) = 0, and \emph{p} is false,
  so that's all good.
\item
  If X = \emph{b}, then \emph{Cr}(\emph{i}) is at its highest value,
  (\emph{b} - \emph{a})/\emph{b}. That's not great since \emph{p} is
  false, but it's just one point.
\item
  Otherwise \emph{Cr}(\emph{i}) is in ((\emph{b} - \emph{a})/\emph{b},
  \emph{b} - \emph{a}).
\end{enumerate}

In the third case, there's a striking result we can prove about
\emph{Cr}.

\begin{quote}
For any threshold \emph{t} ∈ ((\emph{b} - \emph{a})/\emph{b}, \emph{b} -
\emph{a}), Pr(\emph{i} \textbar{} Cr(\emph{i}) ⩾ \emph{t}) = \emph{t}.
\end{quote}

That is, conditional on Hero, who is inside the model, having credence
at least \emph{t} in \emph{i}, the probability that we, who are outside
the model, should have in \emph{i} is \emph{t}. That is, I think, a good
sign that in this case Hero's credence in \emph{i}, the evidential
probability of \emph{i} inside the model, is correlated with the truth
of \emph{i}. The correlation isn't perfect, the edge case in point 2
will become important, but in general the posterior probability of
\emph{i} is correlated with its truth.

I won't go over the proof of this result here. It's a trivial but
somewhat tedious bit of algebra. What's more interesting is to see how
this affects \textbf{AT}.

Consider the case where the interval is (0.03, 0.3). So \emph{i} is that
X is in that interval. And consider in particular the case where X is
0.3. In that case Cr(\emph{i}) is 0.9, which we earlier assumed was the
threshold for rational belief. Indeed, this is the only point where
Cr(\emph{i}) reaches that threshold. But in that case \emph{i} is false.
So the only case where it is rational to believe \emph{i}, \emph{i} is
false. Since Hero knows the model, they are certain before receiving any
evidence that if they rationally believe \emph{i}, it will be false. But
still, this is case satisfies all the constraints that anti-akratic
philosophers have argued were missing from Williamson's earlier models.
Accessibility is transitive and nested, and in a good sense the
evidential probability is a good (if not perfect) guide to truth.

So if the anti-enkratic philosopher was moved in the first place by
models like Williamson, then even taking on board the criticisms of
those models, they have good reason to reject \textbf{AT}: it fails in
cases like these.

\section{Conclusion}\label{sec-conclusions}

So much of the literature on enkrasia consists of raw appeals to the
unintuitiveness of \textbf{SA} and \textbf{LA}. So I think it's great to
see actual arguments from principles like \textbf{AT} for enkratic
principles. But I don't think the anti-enkratic philosophers should be
changing their minds over this.

If one's initial motivation for anti-enkrasia was based in the
metaphysical distinction between reasons and what makes something a
reason, then there are good grounds for rejecting the idea that
rationality can only be guiding if \textbf{AT} is true. And if one's
initial motivation was based in the kinds of models that Williamson
developed, then even taking on board the recent criticisms of those
models, there are variants of his models that falsify \textbf{AT}.

\phantomsection\label{refs}
\begin{CSLReferences}{1}{0}
\bibitem[\citeproctext]{ref-DAlessandro2020}
D'Alessandro, William. 2020. {``Mathematical Explanation Beyond
Explanatory Proof.''} \emph{British Journal for the Philosophy of
Science} 71 (2): 581--603. doi:
\href{https://doi.org/10.1093/bjps/axy009}{10.1093/bjps/axy009}.

\bibitem[\citeproctext]{ref-DasSalow2019}
Das, Nilanjan, and Bernard Salow. 2019. {``Transparency and the {KK}
Principle.''} \emph{No{û}s} 52 (1): 3--23. doi:
\href{https://doi.org/10.1111/nous.12158}{10.1111/nous.12158}.

\bibitem[\citeproctext]{ref-DorstEtAl2021}
Dorst, Kevin, Benjamin A. Levinstein, Bernhard Salow, Brooke E. Husic,
and Branden Fitelson. 2021. {``Deference Done Better.''}
\emph{Philosophical Perspectives} 35 (1): 99--150. doi:
\href{https://doi.org/10.1111/phpe.12156}{10.1111/phpe.12156}.

\bibitem[\citeproctext]{ref-Gallow2021}
Gallow, J. Dmitri. 2021. {``Updating for Externalists.''} \emph{Noûs} 55
(3): 487--516. doi:
\href{https://doi.org/10.1111/nous.12307}{10.1111/nous.12307}.

\bibitem[\citeproctext]{ref-Geanakoplos1989}
Geanakoplos, John. (1989) 2021. {``Game Theory Without Partitions, and
Applications to Speculation and Consensus.''} \emph{The B.E. Journal of
Theoretical Economics} 21 (2): 361--94. doi:
\url{https://doi.org/10.1515/bejte-2019-0010}.

\bibitem[\citeproctext]{ref-Horowitz2014}
Horowitz, Sophie. 2014. {``Immoderately Rational.''} \emph{Philosohical
Studies} 167 (1): 41--56. doi:
\href{https://doi.org/10.1007/s11098-013-0231-6}{10.1007/s11098-013-0231-6}.

\bibitem[\citeproctext]{ref-Lasonen-Aarnio2020}
Lasonen-Aarnio, Maria. 2020. {``Enkrasia or Evidentialism? Learning to
Love Mismatch.''} \emph{Philosophical Studies} 177 (3): 597--632. doi:
\href{https://doi.org/10.1007/s11098-018-1196-2}{10.1007/s11098-018-1196-2}.

\bibitem[\citeproctext]{ref-Luzon2025}
Luzon, Bar. Forthcoming. {``Epistemic Akrasia and Treacherous
Propositions.''} \emph{Philosophical Quarterly}, Forthcoming.

\bibitem[\citeproctext]{ref-MartinHjortland2024}
Martin, Ben, and Ole Thomassen Hjortland. 2024. {``Anti-Exceptionalism
about Logic (Part i): From Naturalism to Anti-Exceptionalism.''}
\emph{Philosophy Compass} 19 (8): e13014. doi:
\url{https://doi.org/10.1111/phc3.13014}.

\bibitem[\citeproctext]{ref-Williamson2000}
Williamson, Timothy. 2000. \emph{{Knowledge and its Limits}}. Oxford
University Press.

\bibitem[\citeproctext]{ref-Williamson2011}
---------. 2011. {``Improbable Knowing.''} In \emph{Evidentialism and
Its Discontents}, edited by Trent Dougherty, 147--64. Oxford: Oxford
University Press.

\bibitem[\citeproctext]{ref-Williamson2014}
---------. 2014. {``Very Improbable Knowing.''} \emph{Erkenntnis} 79
(5): 971--99. doi:
\href{https://doi.org/10.1007/s10670-013-9590-9}{10.1007/s10670-013-9590-9}.

\end{CSLReferences}



\noindent Published in\emph{
Philosophical Quarterly}, 2025, pp. tbc.


\end{document}
