% Options for packages loaded elsewhere
\PassOptionsToPackage{unicode}{hyperref}
\PassOptionsToPackage{hyphens}{url}
%
\documentclass[
  10pt,
  letterpaper,
  DIV=11,
  numbers=noendperiod,
  twoside]{scrartcl}

\usepackage{amsmath,amssymb}
\usepackage{setspace}
\usepackage{iftex}
\ifPDFTeX
  \usepackage[T1]{fontenc}
  \usepackage[utf8]{inputenc}
  \usepackage{textcomp} % provide euro and other symbols
\else % if luatex or xetex
  \usepackage{unicode-math}
  \defaultfontfeatures{Scale=MatchLowercase}
  \defaultfontfeatures[\rmfamily]{Ligatures=TeX,Scale=1}
\fi
\usepackage{lmodern}
\ifPDFTeX\else  
    % xetex/luatex font selection
  \setmainfont[ItalicFont=EB Garamond Italic,BoldFont=EB Garamond
Bold]{EB Garamond Math}
  \setsansfont[]{Europa-Bold}
  \setmathfont[]{Garamond-Math}
\fi
% Use upquote if available, for straight quotes in verbatim environments
\IfFileExists{upquote.sty}{\usepackage{upquote}}{}
\IfFileExists{microtype.sty}{% use microtype if available
  \usepackage[]{microtype}
  \UseMicrotypeSet[protrusion]{basicmath} % disable protrusion for tt fonts
}{}
\usepackage{xcolor}
\usepackage[left=1in, right=1in, top=0.8in, bottom=0.8in,
paperheight=9.5in, paperwidth=6.5in, includemp=TRUE, marginparwidth=0in,
marginparsep=0in]{geometry}
\setlength{\emergencystretch}{3em} % prevent overfull lines
\setcounter{secnumdepth}{3}
% Make \paragraph and \subparagraph free-standing
\ifx\paragraph\undefined\else
  \let\oldparagraph\paragraph
  \renewcommand{\paragraph}[1]{\oldparagraph{#1}\mbox{}}
\fi
\ifx\subparagraph\undefined\else
  \let\oldsubparagraph\subparagraph
  \renewcommand{\subparagraph}[1]{\oldsubparagraph{#1}\mbox{}}
\fi


\providecommand{\tightlist}{%
  \setlength{\itemsep}{0pt}\setlength{\parskip}{0pt}}\usepackage{longtable,booktabs,array}
\usepackage{calc} % for calculating minipage widths
% Correct order of tables after \paragraph or \subparagraph
\usepackage{etoolbox}
\makeatletter
\patchcmd\longtable{\par}{\if@noskipsec\mbox{}\fi\par}{}{}
\makeatother
% Allow footnotes in longtable head/foot
\IfFileExists{footnotehyper.sty}{\usepackage{footnotehyper}}{\usepackage{footnote}}
\makesavenoteenv{longtable}
\usepackage{graphicx}
\makeatletter
\def\maxwidth{\ifdim\Gin@nat@width>\linewidth\linewidth\else\Gin@nat@width\fi}
\def\maxheight{\ifdim\Gin@nat@height>\textheight\textheight\else\Gin@nat@height\fi}
\makeatother
% Scale images if necessary, so that they will not overflow the page
% margins by default, and it is still possible to overwrite the defaults
% using explicit options in \includegraphics[width, height, ...]{}
\setkeys{Gin}{width=\maxwidth,height=\maxheight,keepaspectratio}
% Set default figure placement to htbp
\makeatletter
\def\fps@figure{htbp}
\makeatother

\setlength\heavyrulewidth{0ex}
\setlength\lightrulewidth{0ex}
\usepackage[automark]{scrlayer-scrpage}
\clearpairofpagestyles
\cehead{
  Brian Weatherson
  }
\cohead{
  In Defense of the ACA’s Medicaid Expansion
  }
\ohead{\bfseries \pagemark}
\cfoot{}
\makeatletter
\newcommand*\NoIndentAfterEnv[1]{%
  \AfterEndEnvironment{#1}{\par\@afterindentfalse\@afterheading}}
\makeatother
\NoIndentAfterEnv{itemize}
\NoIndentAfterEnv{enumerate}
\NoIndentAfterEnv{description}
\NoIndentAfterEnv{quote}
\NoIndentAfterEnv{equation}
\NoIndentAfterEnv{longtable}
\NoIndentAfterEnv{abstract}
\renewenvironment{abstract}
 {\vspace{-1.25cm}
 \quotation\small\noindent\rule{\linewidth}{.5pt}\par\smallskip
 \noindent }
 {\par\noindent\rule{\linewidth}{.5pt}\endquotation}
\cehead{
       Ishani Maitra and Brian Weatherson
        }
\KOMAoption{captions}{tableheading}
\makeatletter
\@ifpackageloaded{caption}{}{\usepackage{caption}}
\AtBeginDocument{%
\ifdefined\contentsname
  \renewcommand*\contentsname{Table of contents}
\else
  \newcommand\contentsname{Table of contents}
\fi
\ifdefined\listfigurename
  \renewcommand*\listfigurename{List of Figures}
\else
  \newcommand\listfigurename{List of Figures}
\fi
\ifdefined\listtablename
  \renewcommand*\listtablename{List of Tables}
\else
  \newcommand\listtablename{List of Tables}
\fi
\ifdefined\figurename
  \renewcommand*\figurename{Figure}
\else
  \newcommand\figurename{Figure}
\fi
\ifdefined\tablename
  \renewcommand*\tablename{Table}
\else
  \newcommand\tablename{Table}
\fi
}
\@ifpackageloaded{float}{}{\usepackage{float}}
\floatstyle{ruled}
\@ifundefined{c@chapter}{\newfloat{codelisting}{h}{lop}}{\newfloat{codelisting}{h}{lop}[chapter]}
\floatname{codelisting}{Listing}
\newcommand*\listoflistings{\listof{codelisting}{List of Listings}}
\makeatother
\makeatletter
\makeatother
\makeatletter
\@ifpackageloaded{caption}{}{\usepackage{caption}}
\@ifpackageloaded{subcaption}{}{\usepackage{subcaption}}
\makeatother
\ifLuaTeX
  \usepackage{selnolig}  % disable illegal ligatures
\fi
\usepackage{bookmark}

\IfFileExists{xurl.sty}{\usepackage{xurl}}{} % add URL line breaks if available
\urlstyle{same} % disable monospaced font for URLs
\hypersetup{
  pdftitle={In Defense of the ACA's Medicaid Expansion},
  pdfauthor={Ishani Maitra; Brian Weatherson},
  hidelinks,
  pdfcreator={LaTeX via pandoc}}

\title{In Defense of the ACA's Medicaid Expansion}
\author{Ishani Maitra \and Brian Weatherson}
\date{2013}

\begin{document}
\maketitle
\begin{abstract}
The only part of the Patient Protection and Affordable Care Act
(hereafter, `the ACA') struck down was a provision expanding Medicaid.
We will argue that this was a mistake; the provision should not have
been struck down. We'll do this by identifying a test that C.J. Roberts
used to justify his view that this provision was unconstitutional. We'll
defend that test against some objections raised by J. Ginsburg. We'll
then go on to argue that, properly applied, that test establishes the
constitutionality of the Medicaid provision.
\end{abstract}

\setstretch{1.1}
\section{Introduction}\label{introduction}

The only part of the Patient Protection and Affordable Care Act
(hereafter, `the ACA') struck down in \emph{National Federation of
Independent Business (NFIB) et al.~v. Sebelius, Secretary of Health and
Human Services, et al.} was a provision expanding Medicaid.\footnote{567
  U.S. \_\_\_ (2012), available at
  \url{http://www.supremecourt.gov/opinions/11pdf/11--393c3a2.pdf}.} We
will argue that this was a mistake; the provision should not have been
struck down. We'll do this by identifying a test that C.J. Roberts used
to justify his view that this provision was unconstitutional. We'll
defend that test against some objections raised by J. Ginsburg. We'll
then go on to argue that, properly applied, that test establishes the
constitutionality of the Medicaid provision.

To say just what the provision in question is, it will help to have
before us the distinctive structure of Medicaid. Each state runs its own
Medicaid program, with substantial financial support from the federal
government. There are several conditions that a state Medicaid program
must satisfy in order to qualify for this federal support, and which all
fifty states do currently satisfy. In particular, there have always been
minimum coverage requirements.\footnote{Jonathan Engel, \emph{Poor
  People's Medicine: Medicaid and American Charity Care since 1965}
  (Durham, NC: Duke University Press, 2006), 48--51.} Before the ACA,
these minimum coverage requirements were a bit of a hodgepodge. As J.
Ginsburg notes,

\begin{quote}
To receive federal Medicaid funds, States must provide health benefits
to specified categories of needy persons, including pregnant women,
children, parents, and adults with disabilities. Guaranteed eligibility
varies by category: for some it is tied to the federal poverty level
(incomes up to 100\% or 133\%); for others it depends on criteria such
as eligibility for designated state or federal assistance
programs.\footnote{\emph{NFIB v. Sebelius}, 567 U.S. \_\_\_, Ginsburg,
  J., slip opin. at 38 (2012).}
\end{quote}

The ACA introduced a broad new category of Medicaid eligibility. It said
that the states must extend Medicaid eligibility to pretty much anyone
whose income is below 133\% of the federal poverty level.\footnote{42
  U.S.C. 1396a(a)(10)(A)(i)(VIII).} In addition, it provided quite
generous federal support for these newly eligible claimants: over 90\%
of the costs of covering these individuals would be reimbursed to the
states by the federal government.\footnote{42 U.S.C. 1396d(y)(1). More
  specifically, the ACA required the federal government to bear 100\% of
  the costs of covering these newly eligible claimants through 2016. The
  level of federal support was then allowed to gradually decline, but to
  no lower than 90\% of these costs.} (For other categories of
claimants, the reimbursement rate is considerably lower.\footnote{In
  fiscal year 2012, federal funds offset between 50 and approximately
  74\% of states' costs of covering claimants eligible for Medicaid
  prior to the passage of the ACA. Kaiser Commission on Medicaid and the
  Uninsured, ``An Overview of Changes in the Federal Medical Assistance
  Percentages (FMAPs) for Medicaid'', July 2011, available at
  \url{http://www.kff.org/medicaid/upload/8210.pdf}, 2.}) But
importantly to \emph{NFIB v. Sebelius}, the ACA made provision of
Medicaid services to these newly eligible individuals a condition of
continuing federal support. That is, if a state did not expand its
Medicaid program to accommodate these newly eligible individuals, the
ACA gave the Secretary of Health and Human Services the authority to
withhold \emph{all} of the Medicaid funds the state would otherwise be
entitled to.\footnote{42 U.S.C. 1396c.}

It is this last provision that the Supreme Court found to be
unconstitutional in the current case. By a margin of 7--2, the Court
ruled that it was unconstitutional to make federal support for
continuation of the old Medicaid program conditional on states'
participation in this expansion. We're going to argue that that was a
mistake. We'll mostly focus on the opinion written by C.J. Roberts (and
joined, in this respect, by J. Kagan and J. Breyer), and the dissent
written by J. Ginsburg (and joined by J. Sotomayor).

Our discussion in this paper takes place in a relatively constrained
ideological space. Our aim is to argue against one part of the Court's
decision in this case, but to keep things manageable, we'll do this
while ignoring some very interesting philosophical and policy questions
in the vicinity. Thus, for instance, we won't concern ourselves with
arguments against the constitutionality of the ACA's Medicaid provision
that, if successful, would also judge the old Medicaid program to be
unconstitutional. And we'll assume that there are constitutional
constraints on what the federal government can do via its spending
power, that go beyond what is explicitly stated in the U.S.
Constitution. While we think these issues are very much worth pursuing,
we'll leave them aside as much as possible in this paper.

In what follows, it will be helpful to have some terminology for the
various Medicaid requirements. We will use the following:

\begin{itemize}
\tightlist
\item
  `Old Medicaid' refers to the set of requirements and funding levels
  that existed prior to the passage of the ACA.
\item
  `Expanded Medicaid' refers to the requirement that those not covered
  by Old Medicaid, but whose earnings fall below 133\% of the federal
  poverty line, now be covered, \emph{plus} the set of requirements and
  funding levels for these newly eligible individuals stipulated by the
  ACA.
\item
  `New Medicaid' refers to the conjunction of Old Medicaid and Expanded
  Medicaid, i.e., Medicaid as it was envisioned to work after the
  passage of the ACA.
\item
  `The ACA's Offer' refers to the ACA's offering the states the option
  of participating in (all of) New Medicaid, or not participating at
  all, but \emph{not} the option of participating in just Old Medicaid.
\end{itemize}

One question that became surprisingly central to the ruling in
\emph{NFIB v. Sebelius} was the \emph{ontological} relationship between
Old Medicaid and New Medicaid. As we'll see, J. Ginsburg held that these
were the same program; the addition of Expanded Medicaid was just one of
the many modifications that have been made to Old Medicaid over the
years, without destroying its identity.\footnote{\emph{NFIB v.
  Sebelius}, 567 U.S. \_\_\_, Ginsburg, J., slip opin. at 41--44 (2012).}
C.J. Roberts disagreed, arguing that the Medicaid expansion provision
``accomplishes a shift in kind, not merely degree''.\footnote{\emph{Id.},
  Roberts, C.J., slip opin. at 53.}

We're not going to take a strong stand on this ontological question.
That's partly because we think that questions like this are the wrong
\emph{kinds} of questions to be asking here. We don't really have views
on the criteria of identity through time for federal-state cooperative
programs. And we're not convinced that these questions have determinate
answers. But even if they did, we still wouldn't think that these
answers were relevant to the constitutionality of proposed
changes/supplements to those programs. Rather, on our view, what matters
is the \emph{functional} relationship between the old and new programs.

We'll have more on this presently. But first, we need to look at why the
Court thought the relevant sections of the ACA should be struck down.

\section{Spending Power and Coercion}\label{spending-power-and-coercion}

The Spending Clause of the U.S. Constitution gives Congress the power
``to pay the Debts and provide for the \ldots{} general Welfare of the
United States.''\footnote{U.S. Constitution, Article I, Section 8,
  Clause 1.} The justices in \emph{NFIB v. Sebelius} agreed that the
Spending Clause gives Congress a ``broad authority'' to interpret what
that ``general Welfare'' consists in, and to apportion funds
accordingly.\footnote{\emph{NFIB v. Sebelius}, 567 U.S. \_\_\_,
  Ginsburg, J., slip opin. at 50 (2012). See also Roberts, C.J., slip
  opin. at 45--46, and Joint Dissent, slip opin. at 29--32.} This
includes the power to offer the states funds as an inducement to take
certain actions - such as establishing and operating certain programs -
that accord with Congress' understanding of the ``general Welfare''. Old
Medicaid is just such a federal-state cooperative program, established
by Spending Clause legislation.

What the Spending Clause does not allow is for the federal government to
\emph{require} the states to implement a particular federal-state
cooperative program, or to accept the associated funding package. That
includes either directly ordering the states to do these things, or
`indirectly' coercing the states into doing them. As we'll see in the
next section, one of the main issues in \emph{NFIB v. Sebelius} is
whether the ACA's Offer constitutes an attempt to unconstitutionally
coerce the states into realizing a federal spending
objective.\footnote{In our discussion, we'll leave open the question of
  whether an offer by the federal government can be coercive without
  being \emph{unconstitutionally} coercive.}

The justices in \emph{NFIB v. Sebelius} emphasized that Spending Clause
legislation has the nature of a contract.\footnote{\emph{Id}., Roberts,
  C.J., slip opin. at 46, and Joint Dissent, slip opin. at 33.} The
federal government offers the states funds conditional on their
satisfying certain conditions. That is, it offers them money in exchange
for doing something. That looks like a contract. And, arguably, some
contracts are coercive.

When we talk about contracts being coercive, we don't mean that there is
coercion involved in getting one of the parties to accept the offer.
Rather, we mean that there is something about the offer itself that
makes it coercive. Call coercion of the first kind `coercion alongside
the contract'. In simple cases, where the parties have roughly equal
power, there can be coercion of this sort - i.e., coercion alongside the
contract - but it's hard to see how the offer itself can be coercive.
And in such cases, as long as there is no coercion alongside the
contract, the party to whom the offer is made can simply refuse it. By
contrast, when the parties are unequal in power, the mere making of the
offer can sometimes amount to an abuse of the extra power. We'll
illustrate this point with some examples in section 5. For now we want
to note two further points about the notion of coercion at issue here.

First, the question of whether a particular piece of Spending Clause
legislation is unconstitutionally coercive should be distinguished from
the question of whether that legislation runs contrary to the U.S.
system of federalism. If the federal government offered each state a
million dollars to introduce a filibuster rule into their state Senate
procedures (perhaps because the U.S. Senate was embarrassed to be so
idiosyncratic), that would arguably be unconstitutional. But that's not
because the offer would be unconstitutionally coercive. Rather, it would
be because the Spending Clause isn't an open invitation to let the
federal government interfere with every power a state has, including
powers over their own legislative procedures. To put it another way, the
Spending Clause doesn't allow the federal government to do an end run
around the system of federalism.

Second, even if an offer is judged to be unconstitutionally coercive, it
doesn't follow that it will be voided. In \emph{NFIB v. Sebelius}, the
Supreme Court did not strike down the federal government's attempt to
expand Medicaid. Rather, it \emph{amended} the terms of the offer so
they were no longer unconstitutionally coercive. Rather than states
having a choice between New Medicaid and nothing, the Court held that
they would have a choice between Expanded Medicaid, Old Medicaid, and
nothing. This isn't the usual remedy when a contract is found to be
coercive; usually, it's simply voided.\footnote{This is actually a
  surprising fact about the Court's decision. Some prominent court
  watchers seemed to assume, before the decision, that if the court
  found the expansion to be constitutionally problematic, it would
  simply be voided. See, for instance, the pre-decision discussions in
  Kaiser Family Foundation, ``The Health Reform Law's Medicaid
  Expansion: A Guide to the Supreme Court Arguments'', March 2012,
  available at \url{http://www.kff.org/healthreform/upload/8288.pdf},
  and Lyle Denniston, ``Argument Preview: Health Care, Part IV - The
  Medicaid Expansion'', 23 March 2012, available at
  \url{http://www.scotusblog.com/2012/03/argument-preview-health-care-part-iv-the-medicaid-expansion/}.
  Both of these discuss the possibility the expansion will be struck
  down, but neither seems to mention the possibility that the expansion
  will be made voluntary. We suspect they were assuming the court would
  either find the expansion non-coercive, and hence let it stand, or
  coercive, and hold that coercive offers are void.}

\section{The Roberts Rule}\label{the-roberts-rule}

As we discussed in the previous section, the Spending Clause permits the
federal government to offer the states financial inducement to implement
particular programs, as long as the offer itself isn't coercive.
Typically, this just means that the states must have the option not to
participate. As C.J. Roberts put the point,

\begin{quote}
In the typical case we look to the States to defend their prerogatives
by adopting ``the simple expedient of not yielding'' to federal
blandishments when they do not want to embrace the federal policies as
their own. ... The States are separate and independent sovereigns.
Sometimes they have to act like it.\footnote{\emph{Id.}, Roberts, C.J.,
  slip opin. at 49, citation omitted.}
\end{quote}

But sometimes, merely having the option not to participate is not
enough. That's the case if, for example, there's something about the
nature of the federal government's offer that means that the states
don't have a \emph{genuine} choice about participating. According to
C.J. Roberts, there were three aspects of the ACA's Offer that,
together, made it the case that the states didn't have a genuine choice
about accepting. So, the offer was unconstitutionally coercive.

First, the ACA's Offer conditioned the granting of funds for an already
existing program (Old Medicaid) on states' participation in
\emph{another} program (Expanded Medicaid). Call this phenomenon - i.e.,
conditioning the funding for an existing program on states'
participation in another program - `bundling'. Bundling is importantly
different from conditioning the funding for a program on states'
willingness to operate that \emph{very} program in some particular way.
The latter, according to C.J. Roberts, is just the federal government
providing for the ``general Welfare'' as permitted by the Spending
Clause.\footnote{\emph{Id.} at 50.}

But second, even bundling may be permitted when it serves some
legitimate purpose. The bundling involved in the ACA's Offer, however,
``serves no purpose other than to force unwilling States to sign up for
the dramatic expansion in health care coverage effected by the
Act''.\footnote{\emph{Id}.} To put the point another way, the
\emph{only} purpose of the bundling proposed by the ACA was to force the
states to participate in Expanded Medicaid.\footnote{We'll talk
  throughout this paper about purposes served by bundling, rather than
  the purposes Congress had in mind by bundling. This is in keeping with
  both C.J. Roberts' and J. Ginsburg's discussions.}

Finally, the states' financial stake in participating (or not) in the
bundled programs was enormous. Federal support for Old Medicaid makes up
more than 10\% of the typical state's budget.\footnote{In fiscal year
  2012, spending on Old Medicaid comprised nearly 24\% of total spending
  by the states. The National Association of State Budget Officers,
  ``State Expenditure Report: Examining Fiscal 2010--2012 State
  Spending'', available at
  \url{http://www.nasbo.org/sites/default/files/State\%20Expenditure\%20Report_1.pdf},
  44. Federal support offset at least 50\% of that state spending, and
  for some states, quite a lot more. See note 6.} Threatening the states
with losses of that magnitude was, according to C.J. Roberts, a kind of
``economic dragooning''.\footnote{\emph{NFIB v. Sebelius}, 567 U.S.
  \_\_\_, Roberts, C.J., slip opin. at 52 (2012).} When the financial
stake at issue is so large, the states have no genuine choice about
participating.

Why think that these features of the ACA's Offer are sufficient to
constitute it as unconstitutionally coercive? C.J. Roberts doesn't spell
out his reasoning here, but perhaps the thought is this. Bundling of
federal-state cooperative programs can serve various purposes. For
example, bundling can encourage the states to participate in a pair of
programs, where the existence of each program helps the other one
operate more efficiently. Or else, bundling can encourage the states to
implement a new program that helps the existing program achieve its aims
better. When bundling serves these (and other) legitimate public policy
purposes, it's plausible that the federal government is attempting to
provide for the general Welfare in accordance with the Spending Clause.

If, on the other hand, the \emph{only} purpose served by the bundling is
to get the states to participate in one of the bundled programs, and
further, the financial inducement offered is so large that it
effectively serves as a ``gun to the head'' of the states, then that
just amounts to the federal government \emph{requiring} the states to
participate in a particular program.\footnote{\emph{Id.} at 51.} And as
we observed in the previous section, that's not permitted under the
Spending Clause. (We'll have much more to say about this sort of
reasoning in later sections of this paper.)

In sum, then, C.J. Roberts argued that the ACA's Offer was
unconstitutionally coercive because it satisfied the following three
conditions:

\begin{enumerate}
\def\labelenumi{\arabic{enumi}.}
\tightlist
\item
  The offer bundles two \emph{independent} programs (Old Medicaid and
  Expanded Medicaid);
\item
  That bundling serves no other purpose than to force the states to
  participate in one of the bundled programs (Expanded Medicaid);
\item
  Failure to participate in the bundled programs exposes the states to
  enormous financial loss, and so, constitutes a kind of ``economic
  dragooning''.
\end{enumerate}

As we understand the Roberts Rule, conditions (1)-(3) are not only
jointly sufficient for unconstitutional coerciveness, but each of them
is individually necessary as well. To see why, we can consider the
conditions in turn.

Without (1), the states could potentially opt out of any modification to
an existing federal-state cooperative program. Imagine that the federal
government established new standards for effectiveness of cancer
treatments, and decided that Medicaid would henceforth only fund
programs that complied with those new standards. The new parts of this
Medicaid package are bundled together with the old parts to force the
states to comply, and failure to participate in the bundled programs
exposes the states to enormous loss. But this isn't unconstitutionally
coercive; the imagined policy could just be a good instance of quality
control on government spending.\footnote{Note that Old Medicaid has
  evolved in just this way, with modifications to the program being
  bundled together with unchanged parts. For example, the Balanced
  Budget Act of 1997 (P.L. 105--33) required the states to expand their
  coverage of home health visits required by Medicare beneficiaries. The
  federal government met all of the costs of this expansion, but did not
  make it voluntary. For more on this, see Melvina Ford, Richard Price,
  and Jennifer Neisner, ``Medicaid: 105th Congress'', Congressional
  Research Service, February 4 1998, available at
  \url{http://www.policyarchive.org/handle/10207/508}, 12--13.}

Without (2), federal government actions that by hypothesis have a
legitimate purpose could be ruled out. That would be a bad result.
Here's one sort of case that illustrates this point. Imagine that the
federal government in its wisdom subsidizes widget production through
conditional grants to the states. The government then discovers that
widget production has serious environmental consequences. So it decides
to fund cleanup operations, again through conditional grants to the
states. It seems reasonable to bundle these two grants together, since
the federal government has a legitimate interest in not subsidizing a
certain industry without also subsidizing the elimination of its
external costs. But that could be true even if widget production and the
cleanup operations look like independent programs, and each is a
significant component of state budgets.

Finally, (3) is needed to ensure that \emph{South Dakota v. Dole} is not
overturned.\footnote{483 U.S. 203 (1987).} That case concerned the
introduction of a new condition on federal highway funding, namely, that
the states enforce a drinking age of 21. States that failed to comply
with this condition stood to lose up to 5\% of the highway funds they
would otherwise be entitled to. Several states, including South Dakota,
argued that this was unconstitutionally coercive. And they had a point;
setting the drinking age and repairing highways look like quite
independent programs. Further, it was clear that the point of the
bundling was simply to get the states to comply with the federal
government objective of raising drinking ages to 21. But C.J. Roberts
held that this didn't matter, because the threatened financial loss for
noncompliance with the federal objective was small enough so as not to
amount to unconstitutional coercion.\footnote{\emph{NFIB v. Sebelius},
  567 U.S. \_\_\_, Roberts, C.J., slip opin. at 50--51 (2012). But in a
  footnote, C.J. Roberts seems to argue that \emph{any} amount of
  threatened financial loss makes the federal government's offer
  coercive. \emph{Id.} at 52n2. Perhaps he intended to draw a
  distinction between an offer being coercive, and it being
  \emph{unconstitutionally} coercive. But that distinction doesn't
  figure anywhere else in his opinion. So, we're left at a loss about
  how to square this footnote with the rest of C.J. Roberts' argument.}

Our reading of C.J. Roberts' argument - and in particular, of his test
for unconstitutional coerciveness - is quite similar to J.
Ginsburg's.\footnote{\emph{Id.}, Ginsburg, J., slip opin. at 39.} But
there's one point of difference. While J. Ginsburg includes conditions
very much like our (1)-(3) in her reading of the test, she adds a
further condition as well.

\begin{enumerate}
\def\labelenumi{\arabic{enumi}.}
\setcounter{enumi}{3}
\tightlist
\item
  The expansion (Expanded Medicaid) was unforeseeable by the states when
  they signed onto the already existing program (Old
  Medicaid).\footnote{\emph{Id.}}
\end{enumerate}

We agree with J. Ginsburg that C.J. Roberts commits himself to the truth
of (4).\footnote{\emph{Id.}, Roberts, C.J., slip opin. at 54. ``A State
  could hardly anticipate that Congress' reservation of the right
  to''alter'' or ``amend'' the Medicaid program included the power to
  transform it so dramatically.''} But we don't think that (4) is
relevant to the issue of whether the ACA's Offer is unconstitutionally
coercive (and it's not clear to us that C.J. Roberts does
either).\footnote{The result of adding (4) to the Roberts Rule is
  implausible as a test for unconstitutional coerciveness. Imagine that
  the federal government conditions federal highway spending on states
  enforcing a drinking age of 21 (as in \emph{South Dakota v. Dole}),
  but this time, threatens to withhold 100\% of highway funds from any
  states that fail to comply. That offer seems unconstitutionally
  coercive. And this verdict isn't changed if the federal government
  threatens to add such a condition to federal highway spending for many
  years before actually doing so. So the foreseeability (or lack
  thereof) of the condition doesn't seem to affect whether the offer is
  unconstitutionally coercive.}

As we see it, (4) speaks to a different question, namely, whether
Medicaid program as envisioned by the ACA (New Medicaid) is the same
program as the already existing one (Old Medicaid). And as we'll argue
next, that ontological question is irrelevant to whether the ACA's Offer
is unconstitutionally coercive.

\section{Ontological Questions}\label{ontological-questions}

Recall that the Roberts Rule says that an offer by the federal
government to help establish a federal-state cooperative program is
unconstitutionally coercive if the following conditions are satisfied:

\begin{enumerate}
\def\labelenumi{\arabic{enumi}.}
\tightlist
\item
  The offer bundles two \emph{independent} programs;
\item
  That bundling serves no other purpose than to force the states to
  participate in one of the bundled programs;
\item
  Failure to participate in the bundled programs exposes the states to
  enormous financial loss, and so, constitutes a kind of ``economic
  dragooning''.
\end{enumerate}

Condition (1) requires that that the programs be \emph{independent}, not
that their conjunction (in this case, New Medicaid) be \emph{distinct}
from the previously existing program (Old Medicaid). Thus, the Roberts
Rule, as we understand it, places no emphasis on whether the federal
government's offer creates a new program, or merely modifies an already
existing program. We'll have a lot more to say about independence later
(in section 6), but it should be clear that whether two programs are
independent, and whether conjoining them creates a new program, are
different questions. Independence has to do with how the programs
\emph{function}, what they \emph{do}, not with their ontological status,
what they \emph{are}. So, even if New Medicaid turned out to be the same
program as Old Medicaid, the two major parts of New Medicaid - Old
Medicaid and Expanded Medicaid - could be quite independent of each
other.

At first reading, it seems that both C.J. Roberts and J. Ginsburg take
the ontological question (about whether New Medicaid is the same program
as Old Medicaid) seriously. For instance, C.J. Roberts notes, in what
seems like a positive way, the states' claim that ``the expansion is in
reality a new program and that Congress is forcing them to accept it by
threatening the funds for the existing Medicaid program.''\footnote{\emph{Id}.
  at 52.} And he says that Ginsburg's reply, which assumes New Medicaid
and Old Medicaid are the same program, ``begs the question'' against the
states.\footnote{\emph{Id}.} But we think that while J. Ginsburg does
take a stance on the ontological question, C.J. Roberts in the end does
not. And on this point, we side with the latter, at least to the extent
that we think that the ontological question is irrelevant to the issue
of unconstitutional coerciveness.

To see why, consider a rival test to the Roberts Rule which says that
what matters for unconstitutional coerciveness is that the programs in
question (say, A and B) be numerically distinct. Would we get a rule
that is better than the Roberts Rule? Actually, that breaks down into
two questions. First, would the revised rule make for better law? And
second, would the revised rule be a better interpretation of C.J.
Roberts? We answer both questions negatively, with the first negative
answer being some part of our reason for the second negative answer.

Questions about individuation criteria, and criteria of identity over
time, for governmental programs are rather hard. We might think we could
make progress by looking at the area where philosophers have made the
most thorough investigation of identity criteria - namely, personal
identity - and carrying the lessons from there over to debates about
identity of governmental programs. But this would be useless twice over.
For one thing, the debates about personal identity are so far from being
settled that we have little to go on. For another, different views are
going to be plausible in the two cases. A thoroughgoing conventionalism
about personal identity is a rather unpopular view (though it is ably
defended by Caroline West).\footnote{Caroline West, ``Personal Identity:
  Practical or Metaphysical?'', in Kim Atkins and Catriona Mackenzie
  (eds.), \emph{Practical Identity and Narrative Agency} (New York:
  Routledge, 2008), 56--77.} But conventionalism about identity criteria
for conventionally established programs, like Medicaid, seems much more
plausible.

So in general we start knowing very little about identity criteria for
governmental programs. But what we do know should give us pause before
putting identity criteria into a substantial legal rule. It will often
be indeterminate whether A and B are the same program, or different
ones. (Governmental programs provide as clear an example of
indeterminate identity as anything in Terrence Parsons's study of
indeterminate identity.)\footnote{Terrence Parsons, \emph{Indeterminate
  Identity: Metaphysics and Semantics} (Oxford: Oxford University Press,
  2000).} If we made the identity, or otherwise, of A and B relevant to
whether a particular law was unconstitutional, we would risk concluding
that it is indeterminate whether that law is unconstitutional. That
doesn't feel like an acceptable outcome.

In the next section, we'll consider a thought experiment suggested by J.
Ginsburg about what would have happened if Congress had \emph{repealed}
Old Medicaid and then enacted New Medicaid as a replacement.\footnote{\emph{NFIB
  v. Sebelius}, 567 U.S. \_\_\_, Ginsburg, J., slip opin. at 51 (2012).}
We'll argue that the thought experiment isn't particularly revealing,
because it differs from what actually happened in a striking way. But
perhaps we should question this assumption. Why should we say that the
ACA merely enacted Expanded Medicaid, rather than saying that it
actually repealed Old Medicaid, and enacted New Medicaid in its place?
Indeed, if such a reading would make the ACA constitutional (as J.
Ginsburg suggests), wasn't the Court obliged to read the Act that
way?\footnote{\emph{Id}.}

The relevant reason is presumably that neither the ACA, nor the debate
around it, reads like it was repealing and replacing Old Medicaid. That
is, the rhetoric around the ACA reads like it was an expansion of Old
Medicaid, not a replacement of it. And that's enough to make it be the
case that the ACA merely enacted an expansion of Old Medicaid, not a
replacement of it. This talk about how the rhetoric matters to the
ontological description of what the ACA did is part of what we meant
above by saying that a conventionalist theory of identity for
governmental programs is plausible. We all treated the ACA as expanding,
not replacing, Old Medicaid, and hence it really was an expansion, not a
replacement, of Old Medicaid. That's so even though a functionally
equivalent Act could have replaced Old Medicaid.

But while this kind of rhetoric can matter for ontological questions, it
can hardly matter for the constitutional legitimacy of the ACA. C.J.
Roberts held that one central part of the ACA, namely, the individual
mandate, was a valid exercise of the taxing power, even though it was
never marketed as such during Congressional debates.\footnote{\emph{Id.},
  Roberts, C.J., slip opin. at 33--40.} And that seems right to us. What
matters for constitutionality is whether Congress has a power, not how
they \emph{talk} about their powers.\footnote{``{[}T{]}he
  constitutionality of an action taken by Congress does not depend on
  \emph{recitals} of the powers which it undertakes to exercise.''
  \emph{Woods v. Cloyd W. Miller Co.,} 333 U.S. 138, 144 (1948),
  emphasis added. Quoted in \emph{NFIB v. Sebelius}, 567 U.S. \_\_\_,
  Roberts, C.J., slip opin. at 39 (2012).} But to make the ontological
relationships between Old Medicaid, Expanded Medicaid, and New Medicaid
relevant to the constitutionality of the latter would be to grant this
talk, these ``recitals of the powers'', too much significance. It's hard
to see that that is right, or that C.J. Roberts, in the very opinion
where he upheld the individual mandate as an exercise of the taxing
power, would do that.

So we conclude that the right way to read C.J. Roberts' test is in terms
of the relationship between what A and B \emph{do}, not what A and B
\emph{are}. That's how we take his comments that what matters is that
the Medicaid provision in the ACA brought about a ``shift in kind, not
merely degree'', which would ``transform it {[}i.e., Medicaid{]} so
dramatically''.\footnote{\emph{Id.} at 53, 54.} Looking at what Congress
is trying to do is a much better guide to the constitutionality of its
actions than looking at its talk.

Having said all that, we suspect that if one did take ontological
distinctness to be important in testing for constitutionality, then the
Medicaid provision of the ACA \emph{would} be constitutional. It's
actually rather tricky to motivate a position on the ontology of
Medicaid that makes that provision problematic. To see this, note that
there are three ontological options here. (At least, there are three
determinate options; there are also views on which the truth is
indeterminate between these.)

First, it might be that New Medicaid and Old Medicaid are the same
program, and adding Expanded Medicaid is just a familiar way for
Medicaid to grow. That's roughly J. Ginsburg's position.\footnote{\emph{Id.},
  Ginsburg, J., slip opin. at 41--44.} Second, it might be that New
Medicaid and Old Medicaid are distinct programs, with New Medicaid
having two components - Old Medicaid and Expanded Medicaid - that run in
parallel alongside each other. On this option, New Medicaid and Old
Medicaid are no more one program than Social Security plus the Defense
Department are one program. This feels like the option most in tune with
the Court's ruling. Third, it might be that Old Medicaid was repealed by
the ACA, and New Medicaid is a new program put in its place. We've
discussed this possibility a couple of times already, and will have more
to say about it in the next section.

Now to get an ontological view that supports C.J. Roberts's ruling (on
the assumption that ontology is constitutionally relevant), we'd have to
say that New Medicaid is sufficiently different from Old Medicaid that
the first option is ruled out.\footnote{For if ontology were
  constitutionally relevant, and New Medicaid and Old Medicaid were
  substantially similar, then it's hard to see why the former would be
  unconstitutionally coercive but the latter not.} But we couldn't say
that they are so different that the third option becomes the only
plausible one.\footnote{As J. Ginsburg suggests, there's little doubt
  about the constitutionality of the third option. \emph{Id.} at 51.} A
middle ground has to be found, and that ground doesn't look stable to
us. Note in particular that Expanded Medicaid is a program whose
eligibility is defined largely in terms of who is \emph{not} eligible
for Old Medicaid. It isn't that Expanded Medicaid is for everyone
earning less than 133\% of the poverty line. Rather, it's for everyone
earning less than 133\% of the poverty line, who wasn't already covered
by Old Medicaid. That's a little odd. It's especially odd because many
Old Medicaid supported state programs were already more generous than
the federal minimums, and so already covered many of the people who fall
under Expanded Medicaid.\footnote{Kaiser Commission on Medicaid and the
  Uninsured, ``Federal Core Requirements and State Options in Medicaid:
  Current Policies and Key Issues'', April 2011, available at
  \url{http://www.kff.org/medicaid/upload/8174.pdf}.}

Now ultimately none of these ontological speculations matter. But
perhaps the underlying considerations do matter a little. The oddness of
thinking of New Medicaid as a distinct program from Old Medicaid will be
reflected in the oddness of operating it as one. And as we'll consider
in sections 6--7, that latter oddness \emph{is} relevant to the test
C.J. Roberts set out.

Before turning to applications of the Roberts Rule, however, we'll
consider an important objection to the Rule itself. We'll argue that the
objection is unsuccessful, but that it highlights some important aspects
of the Rule, including the vitality of its third condition.

\section{Coercion in Continuing
Relationships}\label{coercion-in-continuing-relationships}

In her dissent, J. Ginsburg offered the following thought experiment.

\begin{quote}
Consider also that Congress could have repealed Medicaid ... Thereafter,
Congress could have enacted Medicaid II, a new program combining the
pre-2010 coverage with the expanded coverage required by the ACA. By
what right does a court stop Congress from building up without first
tearing down?\footnote{\emph{NFIB v. Sebelius}, 567 U.S. \_\_\_,
  Ginsburg, J., slip opin. at 51 (2012), citation omitted.}
\end{quote}

We take the point of this thought experiment to be something like this.
There would have been nothing unconstitutionally coercive about Congress
repealing (Old) Medicaid, and enacting Medicaid II. But the end product
of that repeal-and-replace effort (Medicaid II) would have been
functionally equivalent to New Medicaid, though arrived at in a
different manner. So, if enacting Medicaid II by repealing-and-replacing
would have been constitutional, enacting New Medicaid by just expanding
Old Medicaid (and so, altering an already existing agreement) must be
constitutional as well. Therefore, any test that holds the ACA's Offer
to be unconstitutionally coercive - such as the Roberts Rule - should be
rejected.

We think this is a bad way to assess whether a proposed use of Congress'
spending power is unconstitutionally coercive. To see why, it will help
to consider two examples of agreements. The first involves contracts, as
C.J. Roberts suggests we should understand agreements between the
federal government and the states. The second is a more informal
agreement. What the two agreements have in common is that they're both
\emph{continuing} agreements. As we'll argue below, that's significant
for assessing their coerciveness.

\begin{quote}
\textbf{The Monopsonist}

Supplier makes widgets, and their largest customer by far is MegaCorp.
For many years, MegaCorp has had an annual order for one million
widgets. The price that MegaCorp pays has basically tracked inflation
since the deal was first established, and they now pay \$10 per widget.
This is a decent deal for Supplier, since it costs them \$8 to make each
widget. Although the supply contracts are explicitly only for a year at
a time, it is generally understood that the contracts will be renewed,
and the norm in the widget industry is that these contracts are renewed.

One year, MegaCorp says it is only interested in continuing the deal if
Supplier also sells it a million gimcracks for \$10 per gimcrack. This
isn't a great deal for Supplier, since it costs \$11 to make each
gimcrack. But Supplier will likely go out of business without the deal
to sell widgets to MegaCorp.
\end{quote}

We think MegaCorp's proposal is coercive. Supplier has no real choice
but to take on an extra supply contract that does not even cover their
costs. And we think that the offer is coercive even though the combined
offer MegaCorp makes, namely, \$20 million for a million widgets and a
million gimcracks, is not a bad deal for Supplier. Indeed, Supplier
stands to profit on the combined deal. Nevertheless, the newly added
part of the deal is basically a gift to MegaCorp, and MegaCorp is using
their monopsony power to extract that gift. That makes the deal
coercive.

The inequality in power between MegaCorp and Supplier matters here. Had
Supplier been flooded with potential buyers for their widgets, MegaCorp
could still make an offer of ``Widgets \emph{and} gimcracks, or
nothing'', but they wouldn't be in a position to make that offer
credible. That's because it isn't credible, in the envisaged
circumstances, that if Supplier had countered with an offer of ``Widgets
at the old price, and nothing more'', MegaCorp would have stuck to their
guns and refused the mutually beneficial deal.\footnote{We are drawing
  here on the literature in game theory on \emph{credible threats}. For
  more detailed discussion see, for example, Avinash Dixit and Susan
  Skeath, \emph{Games of Strategy}, second edition (New York: Norton,
  2004), Chapter 10.} Under those circumstances, the offer might not
have been coercive. But those are not the circumstances in our example.

Many of the same points apply to our second example of a continuing
agreement.

\begin{quote}
\textbf{The Conditional Philanthropist}

Child has recently graduated college, and has her first job. As with
many first jobs, the pay is not fantastic. But it's enough to afford a
(barely) tolerable apartment in a safe enough neighborhood. Her Parent
offers to pay the difference in rent that would allow her to live in a
nicer apartment in a safer neighborhood, and Child takes up this offer.
We assume that Parent is not under any obligation to do this; Child's
living situation without parental support is sub-optimal, but
acceptable. It's just a nice gift from Parent.

Some years later, after Child has established roots in the neighborhood
that she can live in thanks to Parent's gift, Parent informs her that he
won't keep providing financial support unless she agrees to assist with
one of Parent's political causes. As it happens, it is a cause that
Child does not agree with.
\end{quote}

We think this offer is also coercive. It would have been acceptable for
Parent to simply never provide support for Child's rental expenses. It
would also have been acceptable for Parent to make clear from the start
that the offer of financial support was conditional on reciprocal
political support. Making an offer like that would be distasteful, and
frankly strikes us as an appalling way to relate to one's own child. But
if Child could have an acceptable standard of living without this extra
money, it isn't \emph{coercive} to offer her a little more money in
exchange for political support. Once the arrangement has commenced,
though, and Child has structured her life around it, threatening to take
it away unless Child supports a political cause does seem coercive.

And this is why we think J. Ginsburg's thought experiment fails. It's
true that it would be constitutional for Congress to repeal Old
Medicaid. It's also true that had Old Medicaid never existed, and
Congress had enacted New Medicaid all at once in the ACA, there would be
no constitutional question here. J. Ginsburg suggests that this is
enough to show that the expansion is constitutional.

But the facts in J. Ginsburg's thought experiment aren't the facts at
hand. The existing federal support for Old Medicaid creates an important
kind of relationship between the states and the federal government. The
states have structured a significant part of their operations around the
(quite reasonable) assumption that this relationship would persist.
Requiring the states to do something new in exchange for the
preservation of that relationship is potentially coercive for just the
same reasons that Parent and MegaCorp's offers are coercive.

Put another way, in the context of a continuing relationship,
particularly one in which there is an imbalance in power, we have to
look at how the relationship is changing, and not just at the end
result, to see whether we have a case of coercion. Even if the federal
government's offer of New Medicaid, appearing as a \emph{deux ex
machina}, would have been constitutionally acceptable, it doesn't follow
that \emph{adding} Expanded Medicaid to Old Medicaid (in the way the ACA
does) is acceptable.

Our argument in this section highlights the importance of condition (3)
of the Roberts Rule. That condition focuses on the possibility of
``enormous financial loss''. If Old Medicaid and Expanded Medicaid were
both first enacted as part of the ACA, and they were bundled in the
sense that a state was not free to participate in one but not the other,
then this bundling would not expose the states to any \emph{losses}.
Rather, the bundling would just make it a little harder for the states
to receive funds offered by the federal government. But the fact that
Old Medicaid existed, and had been incorporated into the states'
financial planning, means that a threat to any state's continued
participation is a threat of \emph{loss} to that state, as required
under condition (3) of the Roberts Rule. And as we've been arguing, this
is signifiant for assessing coerciveness.

When we said that MegaCorp's and Parent's offers were coercive, we were
not offering an opinion about whether there are, or should be, legal
remedies available to Supplier or Child. It could well be argued that
the potential costs of involving the courts in relationships like these
outweigh the costs of allowing some coercive offers to be made.

But when we look at legislation that alters the relationship between the
states and the federal government, it is more plausible that there is a
role for the courts in preventing coercion. If one level of government
is coercing others, that is not something that should be allowed to
stand. In fact, allowing it to stand seems incompatible with the U.S.
system of federalism.

So we reject this argument of J. Ginsburg's. As we read her, this is the
only objection she makes to the test C.J. Roberts proposes. As we'll see
in the next section, she makes several further points that can be used
as objections to his \emph{application} of the test. We will endorse
some of those objections. But she doesn't appear to offer other
objections to the test itself. And we too will assume, from here on,
that the Roberts Rule is a good test for striking down proposed uses of
Congress' spending power on grounds of unconstitutional coerciveness.

This is primarily a paper on constitutional questions, so we'll keep
this digression brief. But we do want to note that our disagreement with
J. Ginsburg here has wider ramifications. It is common in several walks
of life to have agreements between two parties that are year-to-year on
paper, but are expected by both parties to continue somewhat
indefinitely. Many employment arrangements are like that. And, although
less common in the United States, arrangements to rent housing in many
parts of the world are also like that. In those cases, when considering
whether proposed changes to the relationship by the more powerful party
(usually the employer or the landlord) are coercive, we think it's
important to look at the changes themselves, and not just to whether the
new agreement would be acceptable taken on its own. The same kind of
reasoning should apply to continuing agreements between the states and
the federal government.

\section{Relation Between Programs}\label{relation-between-programs}

While we agree that an expansion of Medicaid \emph{could} be
unconstitutionally coercive, we don't think the expansion envisioned in
the ACA actually is. Further, we think that the Roberts Rule, properly
applied, gets this result. In this section, we'll argue that Old
Medicaid and Expanded Medicaid are sufficiently closely related that the
first condition of the Roberts Rule is not satisfied. In the next
section, we'll argue that there is legitimate reason to bundle the two
programs together, so the second condition is also not satisfied. Thus,
even accepting the Roberts Rule as a good test for unconstitutional
coerciveness, the ACA's Offer turns out to not be unconstitutionally
coercive.

Here's the Roberts Rule once more. It says that an offer by the federal
government to help establish a federal-state cooperative program is
unconstitutionally coercive if the following conditions are satisfied:

\begin{enumerate}
\def\labelenumi{\arabic{enumi}.}
\tightlist
\item
  The offer bundles two \emph{independent} programs;
\item
  That bundling serves no other purpose than to force the states to
  participate in one of the bundled programs;
\item
  Failure to participate in the bundled programs exposes the states to
  enormous financial loss, and so, constitutes a kind of ``economic
  dragooning''.
\end{enumerate}

A pair of federal-state cooperative programs may be independent in at
least two different senses: first, if the programs have substantially
different purposes; and second, (even) if they have the same (or closely
related) purposes, but attempt to achieve those purposes in
substantially different ways.

When it comes to Old Medicaid and Expanded Medicaid, that the programs
are not independent in either of the senses just outlined seems obvious
on its face. After all, the two programs share an overall purpose; both
have the aim of improving access to health services for the neediest
Americans. In fact, as we've already mentioned (in section 4),
eligibility criteria for one of the programs (Expanded Medicaid) is
defined partly in terms of \emph{in}eligibility for the other one. That
suggests that the programs are designed to work \emph{together} to
achieve their overall purpose. Moreover, they try to achieve this
purpose in the same way, via the same circuitous means; both feature the
federal government encouraging the states to provide health care to
their poorest residents by paying a (large) percentage of the costs,
conditional on the states meeting certain conditions for minimum care.
That looks like enough to make it the case that the programs are closely
related, \emph{contra} condition (1) of the Roberts Rule.

Obviously, C.J. Roberts disagreed with this assessment. We find in his
opinion four considerations that might be used to argue that the
programs are not suitably related.\footnote{\emph{Id.}, Roberts, C.J.,
  slip opin. at 53--54.} The first two purport to identify significant
differences in purpose between Old Medicaid and Expanded Medicaid, while
the third and fourth point to differences in how they're intended to
operate. None of these strikes us as persuasive. (Three of these
considerations were also discussed by J. Ginsburg in her dissent, and as
will be clear below, we largely sympathize with her
responses.\footnote{\emph{Id.}, Ginsburg, J., slip opin. at 50--51. J.
  Ginsburg treats these considerations as comprising C.J. Roberts'
  argument for the view that Old Medicaid and New Medicaid are
  \emph{distinct} programs. We've explained (in section 4) why we don't
  take C.J. Roberts to subscribe to this view.})

The first consideration is that Old Medicaid covered discrete
``categories of the needy: the disabled, the blind, the elderly, and
needy families with dependent children.''\footnote{\emph{Id.}, Roberts,
  C.J., slip opin. at 53.} But Expanded Medicaid had a somewhat blunter
condition of eligibility, namely, that the claimants be earning less
than 133\% of the federal poverty level (and not already be covered by
Old Medicaid). It's hard to see how this amounts to a source of
unrelatedness. It's as if C.J. Roberts thought of Old Medicaid as not
merely having a disjunctive essence, but as being essentially
disjunctive. He seems to be suggesting that any program that didn't have
a long list of eligibility criteria could not be closely related to Old
Medicaid. This strikes us as absurd, especially since the list of
eligibility criteria for Old Medicaid were hardly arbitrary or \emph{ad
hoc}. It was meant to reflect, as C.J. Roberts recognized, ways of being
especially needy. A simpler means of determining need would accomplish
the same thing.

This brings us to the second consideration, that by covering up to 133\%
of the poverty level (rather than merely covering the discrete
categories mentioned in the previous paragraph), Expanded Medicaid was
no longer reserved for the neediest among us. But a family of four has
to earn less than \$31,322 to qualify under this condition.\footnote{U.S.
  Department of Health \& Human Services, ``2013 Poverty Guidelines'',
  available at \url{http://aspe.hhs.gov/poverty/13poverty.cfm}.} To
exclude those earning so little from among the neediest seems
bizarre.\footnote{J. Ginsburg makes a similar point. \emph{NFIB v.
  Sebelius}, 567 U.S. \_\_\_, Ginsburg, J., slip opin. at 50 (2012).}

Of course, it would be possible to \emph{keep} expanding Medicaid, in
something akin to the manner envisioned by the ACA, until it no longer
covers just the neediest, but more closely resembles a program of
universal health care. And it may even be indeterminate where the line
between these lies. But C.J. Roberts offers no reason to think that
Expanded Medicaid crosses this line. Moreover, in light of well-known
criticisms of the federal poverty measure - criticisms which charge that
the measure, developed in the 1960s, is now outdated and significantly
undercounts poverty in the U.S. - there's good reason to think that
133\% of the federal poverty level is not an especially generous
threshold.\footnote{Jared Bernstein, ``More Poverty than Meets the
  Eye'', Economic Policy Institute, April 11 2007, available at
  \url{http://www.epi.org/publication/webfeatures_snapshots_20070411/}.
  Bernstein writes, ``When it comes to poverty in America, almost every
  analyst agrees that the official measure is terribly out-of-date and
  no longer provides a valid indication of economic deprivation.'' See
  also David M. Betson, Constance F. Citro, and Robert T. Michael,
  ``Recent Developments for Poverty Measurement in U.S. Official
  Statistics'', \emph{Journal of Official Statistics}, vol.~16, no. 2,
  2000, 87--111.}

The third consideration is that the federal subsidies for Expanded
Medicaid are more generous than those for Old Medicaid.\footnote{See
  notes 5 and 6.} But it's hard to see how the generosity, or lack
thereof, of the federal government speaks to the relatedness of the two
programs. Would more stinginess on the part of the federal government
have made the two programs more closely related?

And the final consideration C.J. Roberts offers is that the conditions
imposed on the states by Old Medicaid and Expanded Medicaid are
different. But as J. Ginsburg notes, things aren't quite so simple.
While it's true that the conditions imposed under Expanded Medicaid are
different from those \emph{traditionally} required under Old Medicaid,
they aren't any different from what has been required under Old Medicaid
since 2006. So Expanded Medicaid isn't different, in this sense, from
Medicaid as it was at the time ACA was passed.\footnote{\emph{NFIB v.
  Sebelius}, 567 U.S. \_\_\_, Ginsburg, J., slip opin. at 50--51 (2012).}

None of this is meant to deny that there are differences between Old
Medicaid and Expanded Medicaid. But by itself, that doesn't tell us much
about whether the ACA's Offer was unconstitutionally coercive. Condition
(1) of the Roberts Rule requires not merely that programs in question be
different in some way(s), but that they be \emph{independent}. We've
been arguing that the differences cited by C.J. Roberts don't speak to
independence of the two programs in any relevant sense.

\section{Reasons for Bundling}\label{reasons-for-bundling}

In this section, we describe a further way in which the ACA's Offer
fails to be unconstitutionally coercive under the Roberts Rule. We argue
that the federal government has at least three legitimate reasons for
bundling the programs together. Thus, it's not true that the \emph{only}
reason for the bundling is to force the states to participate in
Expanded Medicaid, \emph{contra} condition (2) of the Roberts Rule.

The first (and more minor) reason concerns the complexity that will
arise when future modifications are made to Medicaid. As we've already
had reason to note, Medicaid requirements are not set in
stone.\footnote{See note 22.} They change, often substantially, from
year to year. But the Court's decision in \emph{NFIB v. Sebelius} has
just made it considerably more complex to make any such changes. As
things stand, if Congress wants to enact changes to both Old Medicaid
and Expanded Medicaid, it will have to apply the change to each program
separately.

Of course, this point is related to our discussion in the previous
section; if the two programs were genuinely unrelated, we wouldn't
expect many changes that would apply to both programs. But given that
the two programs work in substantially similar ways, at least some such
changes - maybe even many - will be forthcoming.

The second reason concerns potential savings for the states. Though both
C.J. Roberts\footnote{\emph{Id}., Roberts, C.J., slip opin. at 52n12.}
and the authors of the Joint Dissent\footnote{\emph{Id.}, Joint Dissent,
  slip opin. at 45--46.} emphasized the extra financial burden Expanded
Medicaid would impose on the states, there's ample research to suggest
that that burden is in fact quite minor.\footnote{See, for example,
  January Angeles, ``How Health Reform's Medicaid Expansion Will Impact
  State Budgets: Federal Government Will Pick Up Nearly All Costs, Even
  as Expansion Provides Coverage to Millions of Low-Income Uninsured
  Americans'', Center on Budget and Policy Priorities, July 25, 2012,
  available at \url{http://www.cbpp.org/files/7--12--12health.pdf}, and
  John Holahan and Irene Headen, ``Medicaid Coverage and Spending in
  Health Reform: National and State-by-State Results for Adults at or
  Below 133\% FPL'', Kaiser Commission on Medicaid and the Uninsured,
  May 2012, available at
  \url{http://www.kff.org/healthreform/upload/medicaid-coverage-and-spending-in-health-reform-national-and-state-by-state-results-for-adults-at-or-below--133-fpl.pdf}.}
For example, the Center on Budget and Policy Priorities estimated that
the expansion would cost the states just 2.8\% more than they would
otherwise spend on Medicaid between 2014 and 2022.\footnote{Angeles,
  \emph{op. cit.}, 1.} Even more strikingly, that figure
\emph{overstates} the increase in state spending once we add in the
savings to the states from no longer having to provide uncompensated
care to those currently uninsured (or underinsured). The Urban Institute
estimated that the states would end up \emph{saving} money - anywhere
from \$92 to \$129 billion between 2014 and 2019 - by taking up the
ACA's Offer.\footnote{Matthew Buttguens, Stan Dorn, and Caitlyn Carroll,
  ``Consider Savings as well as Costs: State Governments Would Spend at
  Least \$90 Billion Less With the ACA than Without It from 2014 to
  2019'', The Urban Institute, July 2011, available at
  \url{http://www.urban.org/UploadedPDF/412361-consider-savings.pdf}, 1.
  This projection includes savings from moving some adults currently
  covered by various state Medicaid programs onto federal subsidies via
  the health care exchanges to be established under the ACA.} Those are
savings that can be used by the states to bolster their other health
care programs, including Old Medicaid. So, implementing Expanded
Medicaid might in fact put the states in a position to run Old Medicaid
better.

The third reason has to do with improving health outcomes generally.
There's research suggesting that high rates of uninsurance in a
community adversely affect health outcomes for everyone there,
\emph{including} those with insurance.\footnote{For an useful overview
  of some research on this point, see the Institute of Medicine,
  \emph{America's Uninsured Crisis: Consequences for Health and Health
  Care} (Washington, D.C.: National Academies Press), especially Chapter
  4.} The difficulty and expense involved in treating the uninsured when
they are finally driven to seek health care tends to reduce the quality
of care that might otherwise be available for the insured. So, by
reducing the ranks of the uninsured, Expanded Medicaid can help states
achieve better health outcomes via their other health care programs,
including Old Medicaid.

In assessing the costs and benefits of programs like Medicaid, it's
important to remember the ways in which basic health care is unlike many
other goods. In particular, even if a state chooses not to participate
in something like Expanded Medicaid, it still must share the financial
burden of providing health care to the population that would have been
covered by such a program. States still have to provide emergency rooms,
and in practice, are not compensated for a significant portion of the
care these provide to the uninsured. If some of the people who move onto
Expanded Medicaid were previously uninsured, a state could see an
overall reduction in its health care spending. Some of these savings
would come from moving state costs onto the federal component of
Medicaid. But some would come from the fact that patients would be
moving from a form of health care that's very expensive to provide,
i.e., emergency room care, to the more efficient forms that are mostly
available to the insured.\footnote{In its first full year of
  implementing health reform substantially similar to the ACA,
  Massachusetts saw an astounding 38\% drop in its spending on
  uncompensated care. Angeles, \emph{op. cit.}, 5.} These savings could
strengthen Old Medicaid, and the health system as a whole. So expanding
Medicaid could make Old Medicaid more financially self-sufficient.

Again, the fact that the programs are not unrelated is relevant. It's
because these are both health care programs - and moreover, health care
programs for those who might otherwise be uninsured - that it's
plausible to think that savings incurred in one program strengthen the
other program, rather than just strengthen the federal government's
balance sheet.

So the expansion of Medicaid fails two of Roberts' three criteria for
being unconstitutionally coercive. That's why we can agree with his
test, but disagree with the conclusion that the expansion was
unconstitutionally coercive.

\section{Conclusion}\label{conclusion}

We've argued in this paper for several conclusions:

\begin{enumerate}
\def\labelenumi{\arabic{enumi}.}
\tightlist
\item
  The ontological relationship between Old Medicaid and New Medicaid is
  irrelevant to the constitutionality of the ACA.
\item
  The facts that Old Medicaid was well-established prior to the passage
  of the ACA, and that the states relied on its continuation, \emph{are}
  relevant to the constitutionality of the ACA, and, in fact, open up
  the possibility that the ACA's Offer is unconstitutionally coercive.
\item
  But that Offer is not, after all, unconstitutionally coercive, for at
  least two reasons: first, because it bundles together two programs
  (Old Medicaid and Expanded Medicaid) that are closely related, and
  second, because there are legitimate reasons for that bundling.
\end{enumerate}

Our discussion has left open several interesting questions. That's
partly for space reasons, and partly because we're not sure what the
right answers are. We'll end our discussion by listing three of those
questions.

First, are offers by the federal government to help establish
federal-state cooperative programs unconstitutionally coercive
\emph{only when} the states are exposed to significant losses? We are
inclined to think that this is not the case, and that this poses a
problem for C.J. Roberts' attempt to distinguish the current case from
\emph{South Dakota v. Dole} via condition (3) of the Roberts Rule.

Second, is the Roberts Rule a good test for unconstitutional
coerciveness of offers by the federal government in other federal
jurisdictions? For example, does it throw light on whether the
Australian High Court ruled correctly in the Uniform Tax
Cases?\footnote{\emph{South Australia v Commonwealth} 65 CLR 373 (1942)
  and \emph{Victoria v Commonwealth} 99 CLR 575 (1957).}

Third, is the Roberts Rule a good test for coerciveness of proposed
changes to continuing relationships in business, or in residential
tenancy? If so, this case could have implications for areas far removed
from constitutional law. Although the previous two sections have
provided a number of reasons to doubt that the ACA's Medicaid provision
is unconstitutionally coercive, the issues raised here are relevant to
the broader question of when the more powerful party in a continuing
relationship can force changes to that relationship.\footnote{Thanks to
  the editors and referees of this journal for many helpful comments.}

\vspace{1cm}



\noindent Published in\emph{
Public Affairs Quarterly}, 2013, pp. 267-288.

\end{document}
