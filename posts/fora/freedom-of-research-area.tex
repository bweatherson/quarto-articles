% Options for packages loaded elsewhere
\PassOptionsToPackage{unicode}{hyperref}
\PassOptionsToPackage{hyphens}{url}
%
\documentclass[
  10pt,
  letterpaper,
  DIV=11,
  numbers=noendperiod,
  twoside]{scrartcl}

\usepackage{amsmath,amssymb}
\usepackage{setspace}
\usepackage{iftex}
\ifPDFTeX
  \usepackage[T1]{fontenc}
  \usepackage[utf8]{inputenc}
  \usepackage{textcomp} % provide euro and other symbols
\else % if luatex or xetex
  \usepackage{unicode-math}
  \defaultfontfeatures{Scale=MatchLowercase}
  \defaultfontfeatures[\rmfamily]{Ligatures=TeX,Scale=1}
\fi
\usepackage{lmodern}
\ifPDFTeX\else  
    % xetex/luatex font selection
  \setmainfont[ItalicFont=EB Garamond Italic,BoldFont=EB Garamond
Bold]{EB Garamond Math}
  \setsansfont[]{Europa-Bold}
  \setmathfont[]{Garamond-Math}
\fi
% Use upquote if available, for straight quotes in verbatim environments
\IfFileExists{upquote.sty}{\usepackage{upquote}}{}
\IfFileExists{microtype.sty}{% use microtype if available
  \usepackage[]{microtype}
  \UseMicrotypeSet[protrusion]{basicmath} % disable protrusion for tt fonts
}{}
\usepackage{xcolor}
\usepackage[left=1in, right=1in, top=0.8in, bottom=0.8in,
paperheight=9.5in, paperwidth=6.5in, includemp=TRUE, marginparwidth=0in,
marginparsep=0in]{geometry}
\setlength{\emergencystretch}{3em} % prevent overfull lines
\setcounter{secnumdepth}{3}
% Make \paragraph and \subparagraph free-standing
\ifx\paragraph\undefined\else
  \let\oldparagraph\paragraph
  \renewcommand{\paragraph}[1]{\oldparagraph{#1}\mbox{}}
\fi
\ifx\subparagraph\undefined\else
  \let\oldsubparagraph\subparagraph
  \renewcommand{\subparagraph}[1]{\oldsubparagraph{#1}\mbox{}}
\fi


\providecommand{\tightlist}{%
  \setlength{\itemsep}{0pt}\setlength{\parskip}{0pt}}\usepackage{longtable,booktabs,array}
\usepackage{calc} % for calculating minipage widths
% Correct order of tables after \paragraph or \subparagraph
\usepackage{etoolbox}
\makeatletter
\patchcmd\longtable{\par}{\if@noskipsec\mbox{}\fi\par}{}{}
\makeatother
% Allow footnotes in longtable head/foot
\IfFileExists{footnotehyper.sty}{\usepackage{footnotehyper}}{\usepackage{footnote}}
\makesavenoteenv{longtable}
\usepackage{graphicx}
\makeatletter
\def\maxwidth{\ifdim\Gin@nat@width>\linewidth\linewidth\else\Gin@nat@width\fi}
\def\maxheight{\ifdim\Gin@nat@height>\textheight\textheight\else\Gin@nat@height\fi}
\makeatother
% Scale images if necessary, so that they will not overflow the page
% margins by default, and it is still possible to overwrite the defaults
% using explicit options in \includegraphics[width, height, ...]{}
\setkeys{Gin}{width=\maxwidth,height=\maxheight,keepaspectratio}
% Set default figure placement to htbp
\makeatletter
\def\fps@figure{htbp}
\makeatother
% definitions for citeproc citations
\NewDocumentCommand\citeproctext{}{}
\NewDocumentCommand\citeproc{mm}{%
  \begingroup\def\citeproctext{#2}\cite{#1}\endgroup}
\makeatletter
 % allow citations to break across lines
 \let\@cite@ofmt\@firstofone
 % avoid brackets around text for \cite:
 \def\@biblabel#1{}
 \def\@cite#1#2{{#1\if@tempswa , #2\fi}}
\makeatother
\newlength{\cslhangindent}
\setlength{\cslhangindent}{1.5em}
\newlength{\csllabelwidth}
\setlength{\csllabelwidth}{3em}
\newenvironment{CSLReferences}[2] % #1 hanging-indent, #2 entry-spacing
 {\begin{list}{}{%
  \setlength{\itemindent}{0pt}
  \setlength{\leftmargin}{0pt}
  \setlength{\parsep}{0pt}
  % turn on hanging indent if param 1 is 1
  \ifodd #1
   \setlength{\leftmargin}{\cslhangindent}
   \setlength{\itemindent}{-1\cslhangindent}
  \fi
  % set entry spacing
  \setlength{\itemsep}{#2\baselineskip}}}
 {\end{list}}
\usepackage{calc}
\newcommand{\CSLBlock}[1]{\hfill\break\parbox[t]{\linewidth}{\strut\ignorespaces#1\strut}}
\newcommand{\CSLLeftMargin}[1]{\parbox[t]{\csllabelwidth}{\strut#1\strut}}
\newcommand{\CSLRightInline}[1]{\parbox[t]{\linewidth - \csllabelwidth}{\strut#1\strut}}
\newcommand{\CSLIndent}[1]{\hspace{\cslhangindent}#1}

\setlength\heavyrulewidth{0ex}
\setlength\lightrulewidth{0ex}
\usepackage[automark]{scrlayer-scrpage}
\clearpairofpagestyles
\cehead{
  Brian Weatherson
  }
\cohead{
  Freedom of Research Area
  }
\ohead{\bfseries \pagemark}
\cfoot{}
\makeatletter
\newcommand*\NoIndentAfterEnv[1]{%
  \AfterEndEnvironment{#1}{\par\@afterindentfalse\@afterheading}}
\makeatother
\NoIndentAfterEnv{itemize}
\NoIndentAfterEnv{enumerate}
\NoIndentAfterEnv{description}
\NoIndentAfterEnv{quote}
\NoIndentAfterEnv{equation}
\NoIndentAfterEnv{longtable}
\NoIndentAfterEnv{abstract}
\renewenvironment{abstract}
 {\vspace{-1.25cm}
 \quotation\small\noindent\rule{\linewidth}{.5pt}\par\smallskip
 \noindent }
 {\par\noindent\rule{\linewidth}{.5pt}\endquotation}
\KOMAoption{captions}{tableheading}
\makeatletter
\@ifpackageloaded{caption}{}{\usepackage{caption}}
\AtBeginDocument{%
\ifdefined\contentsname
  \renewcommand*\contentsname{Table of contents}
\else
  \newcommand\contentsname{Table of contents}
\fi
\ifdefined\listfigurename
  \renewcommand*\listfigurename{List of Figures}
\else
  \newcommand\listfigurename{List of Figures}
\fi
\ifdefined\listtablename
  \renewcommand*\listtablename{List of Tables}
\else
  \newcommand\listtablename{List of Tables}
\fi
\ifdefined\figurename
  \renewcommand*\figurename{Figure}
\else
  \newcommand\figurename{Figure}
\fi
\ifdefined\tablename
  \renewcommand*\tablename{Table}
\else
  \newcommand\tablename{Table}
\fi
}
\@ifpackageloaded{float}{}{\usepackage{float}}
\floatstyle{ruled}
\@ifundefined{c@chapter}{\newfloat{codelisting}{h}{lop}}{\newfloat{codelisting}{h}{lop}[chapter]}
\floatname{codelisting}{Listing}
\newcommand*\listoflistings{\listof{codelisting}{List of Listings}}
\makeatother
\makeatletter
\makeatother
\makeatletter
\@ifpackageloaded{caption}{}{\usepackage{caption}}
\@ifpackageloaded{subcaption}{}{\usepackage{subcaption}}
\makeatother
\ifLuaTeX
  \usepackage{selnolig}  % disable illegal ligatures
\fi
\usepackage{bookmark}

\IfFileExists{xurl.sty}{\usepackage{xurl}}{} % add URL line breaks if available
\urlstyle{same} % disable monospaced font for URLs
\hypersetup{
  pdftitle={Freedom of Research Area},
  pdfauthor={Brian Weatherson},
  hidelinks,
  pdfcreator={LaTeX via pandoc}}

\title{Freedom of Research Area}
\author{Brian Weatherson}
\date{2018}

\begin{document}
\maketitle
\begin{abstract}
Some writers have said that academic freedom should extend to giving
academics complete freedom over what they choose to research. I argue
against this: it is consistent with academic freedom for universities to
hire people to research particular subjects, and to make continued
employment conditional on at least some of the academic's research being
in the areas they were hired to work in. In practice, many academics
think that their fellow academics should be free to choose to work on
anything that's within the disciplinary boundaries of the department
they were hired into. I argue that's both too narrow and too broad.
Academic freedom implies that researchers should be allowed to have
their research focus drift over time. But the boundaries of permissible
drift do not correspond to anything like the boundaries of contemporary
academic departments.
\end{abstract}

\setstretch{1.1}
Imagine that a young designer gets hired by Uber. The company has
decided that their smartphone logo is terrible, and it needs to be
replaced. And rather than using a design agency, they think they can do
a better job in house. So they get a designer. And the new employee is,
in a sense, really good. They aren't really good at designing logos; in
fact they don't make any progress on the logo at all. But they are
really good at researching the history of transportation regulation, and
writing about this history in a crisp and timely manner. After a while,
they spend all their time on this historical research and writing, and
the new logo languishes.

I haven't worked for Uber, or any company like much like them, so I
can't say for sure what would happen. But my impression is that the new
employee would find themselves fired rather promptly. It wouldn't matter
how good their work was. In fact, I stipulated that it was good. And it
wouldn't matter how important that work was to the overall mission of
the company. I don't know Uber's inner workings, but from the outside I
suspect that anything that can help them deal with regulatory challenges
is considerably more important to their long-term profitability than the
look of their smartphone icon. None of this matters. The employee was
hired to do a job, they were conspicuously not doing it, and in many
businesses, that will mean you get fired.

Academia is, in crucial respects, not like that. Indeed, it is a key
aspect of academic freedom that researchers get a rather large degree of
freedom in choosing what they want to work on. A department may conduct
a search in a very specialised area, and hire someone on the strength of
their work in just that area, but if that person conducts research in a
somewhat different area when they arrive at the job, there is little the
department can do about it. And, some say, this is how things should be.
Anything otherwise, they say, would be an unacceptable restriction on
academic freedom. Here, for instance, is how Cary Nelson puts the point,

\begin{quote}
Academic freedom gives both students and faculty the right to study and
do research on the topics they choose and to draw what conclusions they
find consistent with their research, though it does not prevent others
from judging whether their work is valuable and their conclusions
sound.~(\citeproc{ref-Nelson2010}{Nelson 2010})
\end{quote}

The idea that academics are not constrained in their research topics has
some history. Here is Alexander Bickel, describing an ideal university.

\begin{quote}
In universities, professionals of many disciplines can follow lines of
inquiry determined by themselves, individually and collectively, and
dictated by no one else, on grounds either ideological or practical.
~(\citeproc{ref-Bickel1975}{Bickel 1975, 127})
\end{quote}

And we see something similar in some university regulations. Here, for
example, is what the University of Chicago has to say about research
topics.

\begin{quote}
The basic policies of The University of Chicago include complete freedom
of research and the unrestricted dissemination of
information.\footnote{Retrieved from
  \href{https://provost.uchicago.edu/handbook/research/research-policies}{uchicago.edu}.
  Both this quote and the Bickel quote are cited by Richard A. Shweder
  (\citeproc{ref-Schweder2015}{2015}).}
\end{quote}

There is a principle that seems to be running through these quotes, and
that I mean to focus on here. I'll call it FRA.

\begin{description}
\item[Freedom of Research Area (FRA)]
If part of an academic's job involves doing research, then the academic
themselves gets to choose which areas they shall perform that research
in. And provided solely that the quality of the work is sufficiently
high, this research in their self-chosen area shall count as adequately
discharging their duties to their academic employer, at least as they
pertain to research.
\end{description}

I think FRA is false, or at least that it should be false. That is, I
think academics should not have complete free choice of what they
research on. I'm not really sure how many people think FRA is true,
though I probably some people think something like it is true. Perhaps
more importantly, I think the appropriate qualifications that one needs
to add to FRA to make it true are neither obvious, nor reflected in
practice. And that's why I think it's worth discussing.

In particular, while FRA does not seem to me to be reflected in the
regular practice of academic life, a related principle I'll call FRAD
is. That is, I think FRAD is both widely believed, and many people act
as if it is true.

\begin{description}
\item[Freedom of Research Area within Department (FRAD)]
If part of an academic's job involves doing research, then the academic
themselves gets to choose which areas they shall perform that research
in, provided it is within the disciplinary boundaries of their home
department. And provided solely that the quality of the work is
sufficiently high, and that they work within these disciplinary
boundaries, this research in their self-chosen area shall count as
adequately discharging their duties to their academic employer, at least
as they pertain to research.
\end{description}

FRAD, I'll argue, is also false. Something close to FRAD, however, is
true. Academics should have `elbow room' in their research; they should
be able to move from one research project to nearby projects. And if
they make such a move, they should count as having fulfilled their
research responsibilities provided their research is of a high enough
quality. FRAD is similar to the elbow room thesis, but not quite the
same as it. And the differences matter in some important cases.

In focussing on FRA and FRAD, I'm setting to one side most of the
questions usually thought central to debates about academic freedom.
(Though I trust these questions will get plenty of discussion in the
rest of this volume.) The focus here is on which questions academics
ask, not on what answers they give. Questions about how free academics
should be in answering questions (E.g., Is it ok to defend Pol Pot? Is
it ok to use seances to motivate historical interpretations?) are left
for others to address. And I'm exclusively focussing on what academics
do in the conduct of their work. Questions about whether they should be
subject to professional sanction for research activities outside work,
and, assuming they are immune from such sanction, whether this immunity
ought be related to their status as academics or simply to their status
as employees are also being set aside.

So we're focussed on what questions academics ask in the course of their
work as academics. There is one more distinction to make to really focus
the discussion. Academics have, to greater or lesser extents, both the
freedom to tackle different research topics, and the responsibility to
tackle certain topics. I'm interested in the responsibility side. What
kind of research counts as suitably discharging one's professional
responsibility to research? Put more bluntly, the focus here is not on
what research questions an academic may ask, but on what questions they
must ask.

The American academy has a rather odd structure when in comes to
enforcing this responsibility. Junior academics get reviewed after
roughly six years, and if their performance is satisfactory, they are
awarded tenure. If not, they are fired. Just what counts for tenure
varies a lot between institutions, but at research institutions, whether
one has adequately discharged one's research responsibilities is a huge
part of the equation.

It's not completely true that there is no other point in the American
academic's career where there will be an inquiry into how well they are
discharging their research responsibilities. But at no other point are
the stakes nearly as high. For instance, many departments have a small
pot of money to distribute in the form of annual raises each year, and
often enough research performance is a factor in that distribution. But
the sums involved, especially in cash-strapped times, are tiny. Since it
is very rare for one's nominal salary to fall in this process, and
inflation is so low, the worst that happens if one completely fails to
discharge all research responsibilities is that one's salary falls by a
percent or so per year. That can add up over time, but compared to being
fired, it's a minor penalty.

So I'm going to focus mostly on that tenure decision here. That isn't
because I have any sympathy for the current structure, with the stakes
being so high here and so low elsewhere. But it's what we have to work
with, so it's what is relevant here and now.\footnote{Just to be clear,
  I'm not denying that the arguments given below could apply to tenured
  academics just as easily as non-tenured ones. But doing so would be a
  very radical break from current practice, and motivating such a
  radical break would need much more careful discussion than I have
  space to do here.}

So imagine the following case. A young scholar gets hired in a US
university\footnote{I will use the term `university' here for any
  post-secondary educational institution that hires professors with an
  expectation they will produce some research. Many of these
  institutions have `college' rather than `university' in their name,
  but I'm calling them all universities.} on a tenure-track line. Six
years later, they are up for tenure review. And in the interim they have
done high quality work, with the quality and quantity of the work being
sufficient for promotion to tenure. But the work is in a different area
to the work they did before being hired, and this work is not at all
what the department had in mind when they were hired. Assuming their
promotion file is adequate in other respects (especially concerning
teaching and service), should they be promoted to tenure? Or, perhaps
more precisely, what further details of the case matter to whether they
should be promoted to tenure? Should it matter, for instance, whether
the work was inside or outside the disciplinary purview of the
department?

I'm assuming here that principles of academic freedom apply at all to
pre-tenured faculty. This doesn't seem too controversial, though it is
striking how some universities talk about tenure and academic freedom.
Here, for instance, is a passage from the University of Michigan's
Tenure Guidelines:

\begin{quote}
The University safeguards academic freedom through its policy that no
person who has been awarded tenure by the Regents or who has been
employed by the University for a total of ten years at the rank of a
full-time instructor or higher may, thereafter, be dismissed, demoted or
recommended for terminal appointment without adequate cause and an
opportunity for a review\ldots{} ~(\citeproc{ref-MichTenureA}{The
University of Michigan 2016a})
\end{quote}

It isn't hard to read that as saying that it is through tenure that
academic freedom is protected, and conclude from that that faculty
without tenure don't have academic freedom. But I'm assuming that
conclusion is false; academic freedom does extend to untenured faculty.
And the question is what it covers.

The particular puzzle case I'm interested in is not unique to
philosophy, but philosophy is considerably more prone to it than other
fields. In many fields, a central part of the tenure file consists of
the book that results from the dissertation. In such cases, there is
little danger that the tenure file will look radically different from
the research profile that was submitted in the candidate's original job
application. In many other fields, research is closely connected to
getting and spending grant money. And the mechanics of grants make it
hard for someone's research to take a sharp change of direction at a
very early stage of their career. This is not to say the case I'm
interested in cannot arise in such disciplines. But it is much more
likely to arise in disciplines that are neither grant-based nor
book-based. There are few such disciplines in existence right now, but
philosophy, at least in the US, is one. Philosophy is also the
discipline I know the most about and, to be honest, care the most about,
so it doesn't bother me that I'm writing about a problem that is more
common here than elsewhere.

I'm also going to write exclusively about fictional cases. I looked into
using some real life cases to make the discussion more vivid. But they
ended up being more of a distraction than a helpful illustration. In
particular, it was hard to find a case where a candidate for tenure was
uncontroversially doing high quality work, but there were concerns about
the area it was in. Rather than re-litigating the tenure files of these
human beings, I think it most appropriate to focus here on the abstract
case.\footnote{As I'll return to below, the main way the issues I'm
  discussing impact everyday academic life is that expectations of how
  tenure reviews will be conducted affect how pre-tenured academics
  structure their research profiles. As any game theorist knows, the
  nature of non-equilibrium outcomes can be profoundly important to the
  actual world, even if they are never reached.}

To be sure, it is hard to precisely imagine a case just like the one I
am describing. Doing high level research in any field is hard. If there
is no sign of one having worked on something before being hired, the
probability that one will be able to acquire sufficient knowledge and
skills to do top quality research in that field is not high. And perhaps
it will even be hard to get unbiased reports on the quality of the work,
if the candidate did not get into the field through the usual channels.
But it's not so unrealistic as to be unimaginable.

I hope everyone would agree that doing extra work, well away from what
one was hired to do, is not a bad thing. It shouldn't count against the
candidate for tenure. Indeed, it is good to show some ability to stretch
out. When I say FRA and FRAD are false, I very much do not mean that one
is obliged to not do any other kind of research.\footnote{I wasn't hired
  to write articles like this one, but there isn't, I hope, anything
  wrong with my writing it.} The more research the better! But first
things first.

So why do I think FRA is false? And why do I think it would be
reasonable for universities to insist that their candidates for tenure
do some research on what they were hired to do.\footnote{There is an
  important caveat here. It would be completely unreasonable for a
  university to decide, just as a candidate comes up for tenure, that
  they really wanted the candidate to have been doing different research
  for the past six years. The question is what behaviour on the part of
  universities would be reasonable if it were clearly communicated well
  in advance.} Well, let's start by looking at an extreme case that
might be thought to motivate something like FRA. The following would be
unreasonable behaviour on the part of a hiring department. A candidate
is hired on the basis of an excellent dissertation on peer disagreement,
which has led to two publications in good journals, and there are two
more papers under review from the dissertation. The hiring department
expects her to keep doing just this kind of work.

But soon after she arrives at her new job, she surveys the most recent
work on peer disagreement and decides the debate is dead. There is, she
thinks, nothing more to say about this debate. It is an ex-debate, it
has ceased to be, it is no more, it has shuffled off this mortal coil
and is now pushing up the roofs of the libraries.\footnote{Full
  disclosure: I have a book manuscript under review with a long
  discussion of peer disagreement.} So rather than scream into the void,
she decides to take what she has learned in debates about disagreement
and apply them to more vibrant debates about testimony, and about
judgment aggregation. And between getting hired and coming up for
tenure, she writes a series of high quality, widely cited, papers on
these topics in respected journals.

But then the hiring department gets upset at time for tenure review. We
hired you to work on peer disagreement, they say, and what have we here?
Nothing at all on peer disagreement, but all this stuff on these
distinct, though admittedly related, fields. That's not enough, we say,
for promotion to tenure.

This is poor behaviour on the part of the hiring department, and so
unreasonable that I find anything like this happening in a real
department almost inconceivable. (Though some departments do have an
impressively dogged commitment to unreasonableness, so perhaps I should
be careful here about the link between conceivability and possibility.)
But we can say more than just why it is unreasonable. There is a good
story about what makes it unreasonable.

Academic debates die. Everything that needs to be said is said, and it's
time to start talking about something new. When that happens, it is
wrong to keep plowing these barren fields. And the people best
positioned to spot the death of a debate are experts, with dedicated
knowledge. Outsiders may suspect that if nothing new is happening, the
participants are just tired following a prolonged squawk. Or, perhaps
more likely, outsiders will confuse mere squawking for actual progress.
The people best positioned to determine whether it is worth investing
more resources in a debate are participants to it. If our imagined
candidate has decided that the debate is dead, then it will usually be
reasonable to defer to her expert judgment. I'm assuming here that when
it comes to particular areas of debate, an assistant professor will be
an expert, even compared to her senior colleagues. She will know, and
they won't know, the details of what has been happening in the very
recent literature, and how much those details matter. Sometimes that
won't be true; her colleagues will be experts. But it will be usually
true, and so it is safe to assume it is true when considering
hypothetical cases for the purposes of policy development.

One other relevant fact about academic debates is that they are not
isolated. An expert on one debate won't automatically become an expert
on all related questions, but she won't be a novice either. In the peer
disagreement example, it is very natural to think that our expert will
know a lot about testimony and about judgment aggregation. Those debates
are both highly relevant to disagreement. So it is reasonable to expect
that if our candidate slid into those debates, she would produce
excellent work. And, recall, that is exactly what happened in the
example.

Putting these two thoughts together, we get the following conclusions.
Allowing people to drift between nearby areas of research will not, on
the whole, reduce the quality of their research. And allowing the people
who are experts in a particular debate to choose when to move between
nearby fields, we can leverage their knowledge of those fields to ensure
that their work remains relevant to lively debates.

These considerations support a freedom to drift, to move from one area
of research to adjacent areas without needing approval from a central
authority. And that's already a kind of academic freedom. The motivation
here has a family resemblance to Hayek's argument that a virtue of
markets is that they provide a way for the system to leverage the
expertise that market participants typically have, at least about areas
immediately relevant to them ~(\citeproc{ref-Hayek1945}{Hayek 1945}).
And this Hayekian flavour to the argument shouldn't be surprising. The
alternative to a model where academics have some freedom to choose the
direction of their research is one where a central planner chooses
everyone's research topic for them. And arguing against the success of
such central planning models was a central concern of Hayek's throughout
his career.

So academics, even junior ones, should have elbow room (to borrow a
metaphor from Daniel Dennett). But it's a long way from this to
endorsing FRA. Indeed, the considerations that supported a freedom to
drift could not possibly support FRA. For one thing, the fact that one
was good enough to be hired in one particular sub-field does not
indicate that one will have particularly expert judgment on whether it
is a good use of university resources to have (more) research conducted
on a particular field distant from one's own. And for another thing, the
fact that one was hired in one field is little to no evidence that one
would be the right person to conduct that research, even if it were in
the university's interests. So if there is a wide ranging freedom to
research on whatever takes one's fancy, it will need radically different
justification to this.

It is hard to see what that justification could possibly be. There are
people who are given awards that are meant to support any kind of
research that they find interesting. The MacArthur Fellows Program, the
so-called `genius grant', is like this. And it seems suitable for people
who have made spectacular contributions, and will likely continue to do
so. It seems particularly suitable for people who have already shown an
ability to create great works that require leaping between seemingly
distant fields. If, for example, you can use hip-hop to turn the story
of the most elitist of the founding fathers into a popular phenomenon,
then someone should probably give you untied funding to just see what
happens next.

But a junior professorship is not a MacArthur Fellowship. Indeed, it is
dangerous to think that it is, or that it should be. It encourages the
idea that universities should be looking to hire geniuses, rather than
hiring people who have put in the hard work to get to where they are in
their field, and are likely to keep getting further results by a
continued application of just that kind of hard work. At least some of
the time, perhaps most of the time, the question of just how smart the
candidate is should be considerably less relevant to a job search than
the question of what they have achieved, and what those achievements
signal about their likely future research contributions.\footnote{Sarah-Jane
  Leslie's work (with various colleagues) has shown that there is a
  strong correlation between how strongly people think that brilliance
  is required for producing good work in a field, and the gender
  distribution of faculty in the field
  ~(\citeproc{ref-LeslieScience2015}{Leslie et al. 2015};
  \citeproc{ref-LeslieFrontiers2015}{Meyer, Cimpian, and Leslie 2015};
  \citeproc{ref-LesliePLOS2016}{Storage et al. 2016}). Thinking that
  whoever is hired can work on anything, and it will probably be good if
  the work they originally did was good, seems similar to me to taking
  raw talent to be the primary requisite for successful work.

  The primary argument of this paper has been the requiring academics to
  do at least a certain amount of work in a particular area is not a
  violation of academic freedom. It is no violation of academic freedom
  to set up something like NYU's Marron Institute (discussed further
  below). The considerations of this paragraph suggest something
  stronger, that it is positively bad to not require (most) academics to
  do work in a particular area, because to not do this encourages an
  invidious cult of genius. I'm not endorsing this stronger claim, but
  these considerations do look like the germ of an argument for it.} Yet
if everyone who was hired was been given a free rein to work on anything
whatsoever, if every hire was the equivalent to bestowing a MacArthur
Fellowship, then whether the candidate was some kind of genius would be
a central, perhaps sole, criteria.

If we were going to say any academic should work on whatever they like,
or even whatever they like in their department's research purview, we
need to do one of two things:

\begin{enumerate}
\def\labelenumi{\arabic{enumi}.}
\tightlist
\item
  Show how this freedom is consistent with the idea that departments
  can, in hiring, take area of research into account; or
\item
  Argue that the very widespread practice of taking area of research
  into account in hiring is indefensible.
\end{enumerate}

I actually have some sympathy for option 2 here, but it would be an
incredibly radical step.\footnote{As of November 16, 2016, there were
  209 jobs advertised on PhilJobs.org, and by my count only 36 did not
  put some restrictions or desiderata on the research area of the hired
  candidate.} So let's investigate the prospects for option 1. I think
they are rather dim.

The motivation for hiring by research field seems straightforward.
Departments have (allegedly) an interest in having researchers working
on diverse fields. And hiring people who have worked in diverse fields
is one way to meet that interest. But given FRA, or even FRAD, there is
a big gap in this motivation. All that we can know by looking at a job
application file is what a person has worked on. The department,
presumably, has an interest in there being diversity in what its members
will work on. And we need a bridge between past work and future work
here.

One way of bridging this gap would be to insist that the newly hired
academic work continue to do (some) research on (roughly) the areas they
were hired to work on. I think that's the right way to bridge the gap,
but it is inconsistent with FRA and FRAD.

Another way would be to take past research interests as noisy indicators
of future research interests. So if you want to hire in philosophy of
biology, you might hire someone who has worked in philosophy of biology
to increase the probability that that's what they'll work in. The
problem with this reasoning is that hiring the person who is most likely
to do the best work in the area you want to hire in will lead to some
bad choices in realistic scenarios. Imagine you want to hire in
philosophy of biology, and you have three candidates.

\begin{itemize}
\tightlist
\item
  A is the best philosopher of the bunch, but has at best a passing
  interest in philosophy of biology.
\item
  B is the best philosopher of biology, but also has a very strong
  interest (including a book manuscript in progress) on a completely
  different field.
\item
  C is nearly as good as B at philosophy of biology, and has no other
  philosophical interests.
\end{itemize}

If you want to maximise the expected value of research your department
does on philosophy of biology, and FRA or FRAD are in place, the best
thing to do hire C. After all, there is a non-trivial chance that B will
just work on their book manuscript and related papers, and indeed use it
to get tenure. If you want to maximise the expected value of research
your department does in philosophy, the best thing to do is to hire A.
They are the best philosopher. What's hard to see is the motivation for
making what intuitively is the right choice here, hiring B. The
solution, I think, is to ditch FRAD, and hire B with the explicit
requirement that they do a certain amount of work in philosophy of
biology.

Let's say that FRA is false then, and conclude with a more focussed look
at FRAD as it applies to tenure cases. This is a somewhat more practical
matter, since FRAD is more like the rule that is applied in actual
tenure cases. Indeed, here is what the handbook at the University of
Michigan (my employer) says about tenure review,

\begin{quote}
After the appropriate probationary period (see section 6.C ``Tenure
Probationary Period''), tenure may be granted to those instructional
faculty members whose professional accomplishments indicate that they
will continue to serve with distinction in their appointed roles. Tenure
is awarded to those who demonstrate excellent teaching, outstanding
research and scholarship, and substantial additional service,
\textbf{each of which must be relevant to the goals and needs of the
University, college and department}. The award of tenure is based on the
achievement of distinction in an area of learning and the prediction of
continued eminence throughout the individual's professional career
~(\citeproc{ref-MichTenureB}{The University of Michigan 2016b} emphasis
added)
\end{quote}

As far as I can tell, in practice the bolded clause is interpreted in
line with something like FRAD. The wording is ambiguous; it could just
as easily be read as supporting the elbow room standard that I prefer.
But I don't believe that's how things work in practice.

We have two questions to answer then. First, how similar are FRAD and
the elbow room standard? And second, in cases where they differ, which
provides a better model for building a university. I think they are not
particularly similar, and the elbow room standard is much better.

There is, of course, a certain similarity between the two standards.
FRAD says work on whatever you like, provided it is in the same
discipline as the work that got you hired. The elbow room standard says
work on whatever you like, provided it is sufficiently similar (along
some salient dimension) to the work that got you hired. And being in the
same department is a dimension of similarity. But it isn't, ultimately,
a particularly important one. Making it of central importance, as FRAD
does, leads to numerous avoidable errors.

For one thing, FRAD gives some academics more freedom to switch fields
than the elbow room standard could possibly justify. Philosophy is a
very broad field. Just because one is doing really excellent work in one
field is very little evidence that one will be able to do excellent work
in another field. Thinking that it is evidence is to just relapse into a
restricted version of the myth, or cult, of genius. So at least in some
cases, the elbow room standard will be more restrictive than FRAD. But
in other cases it will be less restrictive, and those are perhaps more
important in practice.

Disciplines have boundaries. Those boundaries are vague, but they are
there. Some people work on topics that are very near to a boundary, and
some work on topics that are far from a boundary. FRAD impacts these two
groups in very different ways, and the difference is unfair. Someone
whose work is near a boundary can't just drift into any nearby field,
since the nearby fields may be outside the disciplinary bounds. To take
one clear example, a researcher hired for work on the semantics of
modals can easily drift into other areas of semantics, or onto modal
fallacies in argumentation, but not into work on the syntax of modals.
For whatever reason, we've decided the boundary between philosophy and
not-philosophy around here is very close to the the syntax-semantics
boundary. And this is a violation of the elbow room principle, since
this kind of move from the semantics to the syntax of a particular class
of expressions is just the kind of move to a closely related field that
the elbow room principle is designed to protect.

It won't help here to say that there should be overlapping areas of
research concern between departments. The crucial question is the
boundary between X and not-X, not the boundary between X and Y. If
someone is near the X/not-X boundary, they could have a tenure home in
X, and then drift into not-X. They deserve protection in their research,
and FRAD wouldn't provide it. It would perhaps help if literally every
possible area of research was such that there was some department it was
not just in, but in without being near the boundary. But a world with
academic departments organised that way feels very different to the one
we are in.

In won't help to say we should just make the boundaries larger. Unless
we abolish the boundaries altogether, the problem will persist. And
abolishing all the boundaries would create more problems than it solves.
The boundaries play useful roles right now. It is good that hiring and
tenure decisions for, say, a position in metaphysics are made by people
with a broad range of philosophical backgrounds, and not (in the first
instance) by an arbitrary collection of people from across the
university. The ideal here is not no boundaries, but porous
boundaries.\footnote{Academia is hardly the only place where this is the
  ideal.} Boundaries exist so that local experts, and people with
special local interests, get extra say on questions of local concern,
but they are porous so they get in the way of freedoms. Replacing FRAD
with the elbow room principle gets the balance right.

It also doesn't help that the boundaries are vague. In general, if
something is true no matter how a vague term is made precise, it is a
good bet that it is true.\footnote{This principle traces back at least
  to Keynes (\citeproc{ref-Keynes1936}{1936} Ch. 6).} But we can say a
bit more about why vagueness doesn't matter in this particular case. The
following principle looks both true, and the best bet for why we might
think vague disciplinary boundaries make FRAD more palatable.

\begin{itemize}
\tightlist
\item
  If someone's work is clearly within discipline X, then any related
  area they could reasonably drift into under the elbow room principle
  will not be clearly outside discipline X.
\end{itemize}

If everyone who is hired is clearly working within X, then FRAD might be
no more restrictive than the elbow room principle. After all, any
permissible drift will not take one clearly outside one's home
discipline.

The problem is that only hiring people whose work is clearly within the
hiring discipline is a terrible idea. Indeed, it is a worse idea than
FRAD. It ensures that one will only hire in safe, traditional areas of
research. And that's a plan for stagnation, not for doing the best
research. Sometimes we have to hire people on the frontiers, and
sometimes their work will drift clearly outside one's discipline. That's
just a cost of having a dynamic research program, and attempts to avoid
paying this cost will make things even worse.

There is one problem that vague disciplinary boundaries does help with.
Vague boundaries are easier to shift than precise boundaries. That's
because there was never any agreement on where they were in the first
place, so no agreement has to be overturned. And if enough people work
who were hired in X start working on an area that used to be outside X,
we should just start treating that area as inside X. But we don't have
to make this conceptual shift every time a good philosopher does good
work on a nearby topic. Some good work is in other fields, and that's
ok.

There are other odd features of FRAD. If applied consistently, it would
lead to treating some like cases in very unlike ways. There are several
fields that are set up as departments in some universities, and as
programs in other universities. For example, in America right now there
is a divide among universities about whether to set up things like
Women's Studies, Cognitive Science, and PPE as independent departments,
or as programs run collaboratively by a number of different departments.
There are administrative considerations on either side of this choice,
and these considerations vary somewhat between different universities.
If they are set up as programs, then anyone hired in to them will have a
tenure home in one of the constituent departments. And given FRAD, that
will put certain limitations on their research. Those limitations will
be very different to what they would face if the unit were its own
department. But it seems very wrong to think the administrative decision
to set up a unit as a department or a program impact the freedoms of
people hired to work in those unites. Where FRAD applies, however, this
difference is dramatic. Someone hired in a philosophy department to
support a Cognitive Science program could move to work on philosophy of
religion, but not experimental developmental psychology. Someone hired
by a Cognitive Science department would have the opposite set of
freedoms. I'm not sure what is optimal here, but it is very odd that an
administrative decision should impact researchers in this way.

But the biggest problem with FRAD is that it makes the disciplinary
boundaries too important. Young researchers shouldn't have to second
guess whether a particular development of their research is inside or
outside a vague, shifting, boundary. The solution isn't to abolish these
boundaries, any more than rights of free movement across an area is a
reason to abolish all political boundaries within that area.\footnote{By
  analogy, it's a good thing that Ann Arbor has a city council, and it's
  a good thing that there are no barriers to moving in or out of Ann
  Arbor.} Rather, the solution is to downplay them. If exercising their
elbow room rights takes an academic outside the purview of their home
department, that's just a cost of having a dynamic research program.

And it is really the effect on these younger scholars, trying to
pre-judge what their reviewers at tenure time will think, that I'm most
interested in here. As I noted above, I don't know of any clear cases
where someone was turned down for tenure because their research was in
the wrong area. But I know many cases of academics who have put off more
speculative research projects until after their tenure review. And the
reason, typically, has been that they are nervous that the outputs of
the new project would be discounted, merely in virtue of their subject
matter, at the time of tenure review. This feels like an undesirable
feature of the status quo, and one that could be remedied by rethinking
why we care about what an individual academic works on.

So while I disagree with the strongest statements of academic freedom, I
think the position I'm endorsing allows greater freedom in practice than
existing practices like FRAD. There are, I would guess, many more people
who are worried that their research is drifting away from what their
colleagues will regard as \emph{really} part of the discipline than
there are people who would like the freedom to jump to an area they have
no training, expertise or background in. Defending the elbow room
principle, or freedom to drift, will take care of those concerns. And
the principle is much easier to defend in theory than FRA or FRAD. So
it, I think, is the core important principle concerning academics'
rights to direct their own research.

\section*{Conclusion}\label{conclusion}
\addcontentsline{toc}{section}{Conclusion}

I've focussed here exclusively on research, and not at all on teaching.
But in many ways what I'm saying here could be summarised as the view
that the norms concerning topic choice are fairly similar in research
and in teaching. If I'm given a course on history of political
philosophy to teach, then I better teach history of political
philosophy, and not, say, formal logic, or Australian geography, or
baseball statistics.\footnote{Compare the discussion of academic freedom
  and the requirement to teach the topic of the class in Cole, Cole, and
  Weiss (\citeproc{ref-Cole2015}{2015}). The example of teaching
  Australian geography in a class that is not on that is taken from one
  of their survey respondents.} It isn't in any way a violation of
academic freedom if I'm required to teach the subject I signed up to
teach.

But in practice, and in theory, there is a lot of freedom within the
boundaries of a course. If I want my history of political philosophy
course to include thinkers who are not commonly central to the story
Anglophone political philosophy tells about itself, I should (and
typically would) be free to include them. If I think the most relevant
secondary literature is by people in departments other than philosophy
(e.g., history, political science, women's studies, etc) then I should
be able to base my syllabus around such writers. Now I personally
haven't taught history of political philosophy since I was a
post-doctoral fellow who was too nervous to consider any such plan. But
it's exactly the kind of thing academic freedom should protect - and I
suspect in most cases it is what academic freedom would protect.

The same I think should go for research. If one is hired to research
history of political philosophy, then it is reasonable for the
university to require that one do just that. It isn't reasonable to
require one do only that; people should be allowed to explore what they
want. But it is reasonable to require some work on what one was hired to
do. Yet if doing that takes one outside the bounds of what is
(hereabouts) considered philosophy, that should be fine too. Do what
you're hired to do is a good principle; FRAD is not.

The picture of academia I'm trying to promote is one where more units
are free to operate the way that Paul Romer describes the Marron
Institute of Urban Management at NYU as operating.

\begin{quote}
{[}I{]}nstead of giving its faculty members the usual freedom to study
anything that that seems interesting, the institute lets the problems
that cities face set its research agenda. Because these choices are not
the usual ones on campus, many people complained.
~(\citeproc{ref-Romer2016}{Romer 2016})
\end{quote}

If universities want to give people complete freedom to set their own
research agenda, I'm not going to complain about that (much). What I
want to deny is that setting up things like the Marron Institute is a
violation of academic freedom. There is a lot to be gained by hiring
people for relatively specific research tasks, and it isn't a violation
of their freedom to hire them in this way. And that's especially true if
the constraints on their research agenda are set just by the questions
that their research team is focussed on, and not by the disciplinary
homes that house the thinkers they engage with.

\section*{References}\label{references}
\addcontentsline{toc}{section}{References}

\phantomsection\label{refs}
\begin{CSLReferences}{1}{0}
\bibitem[\citeproctext]{ref-Bickel1975}
Bickel, Alexander. 1975. \emph{The Morality of Consent}. New Haven, CT:
Yale University Press.

\bibitem[\citeproctext]{ref-Cole2015}
Cole, Jonathan R., Stephen Cole, and Christopher C. Weiss. 2015.
{``Academic Freedom: A Pilot Study of Faculty Views.''} In \emph{Who's
Afraid of Academic Freedom?}, edited by Akeel Biglrami and Jonathan R.
Cole, 343--94. New York: Columbia University Press.

\bibitem[\citeproctext]{ref-Hayek1945}
Hayek, F. A. 1945. {``The Use of Knowledge in Society.''} \emph{American
Economic Review} 35 (4): 519--30.

\bibitem[\citeproctext]{ref-Keynes1936}
Keynes, John Maynard. 1936. \emph{The General Theory of Employment,
Interest and Money}. London: Macmillan.

\bibitem[\citeproctext]{ref-LeslieScience2015}
Leslie, Sarah-Jane, Andrei Cimpian, Meredith Meyer, and Edward Freeland.
2015. {``Expectations of Brilliance Underlie Gender Distributions Across
Academic Disciplines.''} \emph{Science} 347 (6219): 262--65. doi:
\href{https://doi.org/10.1126/science.1261375}{10.1126/science.1261375}.

\bibitem[\citeproctext]{ref-LeslieFrontiers2015}
Meyer, Meredith, Andrei Cimpian, and Sarah-Jane Leslie. 2015. {``Women
Are Underrepresented in Fields Where Success Is Thought to Require
Brilliance.''} \emph{Frontiers in Psychology} 6: 1--12. doi:
\href{https://doi.org/10.3389/fpsyg.2015.00235}{10.3389/fpsyg.2015.00235}.

\bibitem[\citeproctext]{ref-Nelson2010}
Nelson, Cary. 2010. {``Defining Academic Freedom.''} \emph{Inside Higher
Ed}.
\url{https://www.insidehighered.com/views/2010/12/21/defining-academic-freedom}.

\bibitem[\citeproctext]{ref-Romer2016}
Romer, Paul. 2016. {``Everybody Wants Progress; Nobody Wants Change.''}
\url{https://paulromer.net/progress-change/}.

\bibitem[\citeproctext]{ref-Schweder2015}
Shweder, Richard A. 2015. {``To Follow the Argument Where It Leads: An
Antiquarian View of the Aim of Academic Freedom at the University of
Chicago.''} In \emph{Who's Afraid of Academic Freedom?}, edited by Akeel
Biglrami and Jonathan R. Cole, 190--238. New York: Columbia University
Press.

\bibitem[\citeproctext]{ref-LesliePLOS2016}
Storage, Daniel, Zachary Horne, Andrei Cimpian, and Sarah-Jane Leslie.
2016. {``The Frequency of Words Like {`Brilliant'} and {`Genius'} in
Teaching Evaluations Predicts the Representation of Women and African
Americans Across Academia.''} \emph{PLoS One} 11 (3): 1--17. doi:
\href{https://doi.org/10.1371/journal.pone.0150194}{10.1371/journal.pone.0150194}.

\bibitem[\citeproctext]{ref-MichTenureA}
The University of Michigan. 2016a. {``The University of Michigan Faculty
Handbook, 6.a General Principles.''}
\url{http://provost.umich.edu/faculty/handbook/6/6.A.html}.

\bibitem[\citeproctext]{ref-MichTenureB}
---------. 2016b. {``The University of Michigan Faculty Handbook, 6.b
Criteria for Tenure.''}
\url{http://provost.umich.edu/faculty/handbook/6/6.B.html}.

\end{CSLReferences}



\noindent Published in\emph{
Academic Freedom}, 2018, pp. 102-115.

\end{document}
