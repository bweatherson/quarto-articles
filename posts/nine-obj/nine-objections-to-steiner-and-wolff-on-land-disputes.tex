% Options for packages loaded elsewhere
\PassOptionsToPackage{unicode}{hyperref}
\PassOptionsToPackage{hyphens}{url}
%
\documentclass[
  10pt,
  letterpaper,
  DIV=11,
  numbers=noendperiod,
  twoside]{scrartcl}

\usepackage{amsmath,amssymb}
\usepackage{setspace}
\usepackage{iftex}
\ifPDFTeX
  \usepackage[T1]{fontenc}
  \usepackage[utf8]{inputenc}
  \usepackage{textcomp} % provide euro and other symbols
\else % if luatex or xetex
  \usepackage{unicode-math}
  \defaultfontfeatures{Scale=MatchLowercase}
  \defaultfontfeatures[\rmfamily]{Ligatures=TeX,Scale=1}
\fi
\usepackage{lmodern}
\ifPDFTeX\else  
    % xetex/luatex font selection
  \setmainfont[ItalicFont=EB Garamond Italic,BoldFont=EB Garamond
Bold]{EB Garamond Math}
  \setsansfont[]{Europa-Bold}
  \setmathfont[]{Garamond-Math}
\fi
% Use upquote if available, for straight quotes in verbatim environments
\IfFileExists{upquote.sty}{\usepackage{upquote}}{}
\IfFileExists{microtype.sty}{% use microtype if available
  \usepackage[]{microtype}
  \UseMicrotypeSet[protrusion]{basicmath} % disable protrusion for tt fonts
}{}
\usepackage{xcolor}
\usepackage[left=1in, right=1in, top=0.8in, bottom=0.8in,
paperheight=9.5in, paperwidth=6.5in, includemp=TRUE, marginparwidth=0in,
marginparsep=0in]{geometry}
\setlength{\emergencystretch}{3em} % prevent overfull lines
\setcounter{secnumdepth}{3}
% Make \paragraph and \subparagraph free-standing
\ifx\paragraph\undefined\else
  \let\oldparagraph\paragraph
  \renewcommand{\paragraph}[1]{\oldparagraph{#1}\mbox{}}
\fi
\ifx\subparagraph\undefined\else
  \let\oldsubparagraph\subparagraph
  \renewcommand{\subparagraph}[1]{\oldsubparagraph{#1}\mbox{}}
\fi


\providecommand{\tightlist}{%
  \setlength{\itemsep}{0pt}\setlength{\parskip}{0pt}}\usepackage{longtable,booktabs,array}
\usepackage{calc} % for calculating minipage widths
% Correct order of tables after \paragraph or \subparagraph
\usepackage{etoolbox}
\makeatletter
\patchcmd\longtable{\par}{\if@noskipsec\mbox{}\fi\par}{}{}
\makeatother
% Allow footnotes in longtable head/foot
\IfFileExists{footnotehyper.sty}{\usepackage{footnotehyper}}{\usepackage{footnote}}
\makesavenoteenv{longtable}
\usepackage{graphicx}
\makeatletter
\def\maxwidth{\ifdim\Gin@nat@width>\linewidth\linewidth\else\Gin@nat@width\fi}
\def\maxheight{\ifdim\Gin@nat@height>\textheight\textheight\else\Gin@nat@height\fi}
\makeatother
% Scale images if necessary, so that they will not overflow the page
% margins by default, and it is still possible to overwrite the defaults
% using explicit options in \includegraphics[width, height, ...]{}
\setkeys{Gin}{width=\maxwidth,height=\maxheight,keepaspectratio}
% Set default figure placement to htbp
\makeatletter
\def\fps@figure{htbp}
\makeatother
% definitions for citeproc citations
\NewDocumentCommand\citeproctext{}{}
\NewDocumentCommand\citeproc{mm}{%
  \begingroup\def\citeproctext{#2}\cite{#1}\endgroup}
\makeatletter
 % allow citations to break across lines
 \let\@cite@ofmt\@firstofone
 % avoid brackets around text for \cite:
 \def\@biblabel#1{}
 \def\@cite#1#2{{#1\if@tempswa , #2\fi}}
\makeatother
\newlength{\cslhangindent}
\setlength{\cslhangindent}{1.5em}
\newlength{\csllabelwidth}
\setlength{\csllabelwidth}{3em}
\newenvironment{CSLReferences}[2] % #1 hanging-indent, #2 entry-spacing
 {\begin{list}{}{%
  \setlength{\itemindent}{0pt}
  \setlength{\leftmargin}{0pt}
  \setlength{\parsep}{0pt}
  % turn on hanging indent if param 1 is 1
  \ifodd #1
   \setlength{\leftmargin}{\cslhangindent}
   \setlength{\itemindent}{-1\cslhangindent}
  \fi
  % set entry spacing
  \setlength{\itemsep}{#2\baselineskip}}}
 {\end{list}}
\usepackage{calc}
\newcommand{\CSLBlock}[1]{\hfill\break\parbox[t]{\linewidth}{\strut\ignorespaces#1\strut}}
\newcommand{\CSLLeftMargin}[1]{\parbox[t]{\csllabelwidth}{\strut#1\strut}}
\newcommand{\CSLRightInline}[1]{\parbox[t]{\linewidth - \csllabelwidth}{\strut#1\strut}}
\newcommand{\CSLIndent}[1]{\hspace{\cslhangindent}#1}

\setlength\heavyrulewidth{0ex}
\setlength\lightrulewidth{0ex}
\usepackage[automark]{scrlayer-scrpage}
\clearpairofpagestyles
\cehead{
  Brian Weatherson
  }
\cohead{
  Nine Objections to Steiner and Wolff on Land Disputes
  }
\ohead{\bfseries \pagemark}
\cfoot{}
\makeatletter
\newcommand*\NoIndentAfterEnv[1]{%
  \AfterEndEnvironment{#1}{\par\@afterindentfalse\@afterheading}}
\makeatother
\NoIndentAfterEnv{itemize}
\NoIndentAfterEnv{enumerate}
\NoIndentAfterEnv{description}
\NoIndentAfterEnv{quote}
\NoIndentAfterEnv{equation}
\NoIndentAfterEnv{longtable}
\NoIndentAfterEnv{abstract}
\renewenvironment{abstract}
 {\vspace{-1.25cm}
 \quotation\small\noindent\rule{\linewidth}{.5pt}\par\smallskip
 \noindent }
 {\par\noindent\rule{\linewidth}{.5pt}\endquotation}
\KOMAoption{captions}{tableheading}
\makeatletter
\@ifpackageloaded{caption}{}{\usepackage{caption}}
\AtBeginDocument{%
\ifdefined\contentsname
  \renewcommand*\contentsname{Table of contents}
\else
  \newcommand\contentsname{Table of contents}
\fi
\ifdefined\listfigurename
  \renewcommand*\listfigurename{List of Figures}
\else
  \newcommand\listfigurename{List of Figures}
\fi
\ifdefined\listtablename
  \renewcommand*\listtablename{List of Tables}
\else
  \newcommand\listtablename{List of Tables}
\fi
\ifdefined\figurename
  \renewcommand*\figurename{Figure}
\else
  \newcommand\figurename{Figure}
\fi
\ifdefined\tablename
  \renewcommand*\tablename{Table}
\else
  \newcommand\tablename{Table}
\fi
}
\@ifpackageloaded{float}{}{\usepackage{float}}
\floatstyle{ruled}
\@ifundefined{c@chapter}{\newfloat{codelisting}{h}{lop}}{\newfloat{codelisting}{h}{lop}[chapter]}
\floatname{codelisting}{Listing}
\newcommand*\listoflistings{\listof{codelisting}{List of Listings}}
\makeatother
\makeatletter
\makeatother
\makeatletter
\@ifpackageloaded{caption}{}{\usepackage{caption}}
\@ifpackageloaded{subcaption}{}{\usepackage{subcaption}}
\makeatother
\ifLuaTeX
  \usepackage{selnolig}  % disable illegal ligatures
\fi
\usepackage{bookmark}

\IfFileExists{xurl.sty}{\usepackage{xurl}}{} % add URL line breaks if available
\urlstyle{same} % disable monospaced font for URLs
\hypersetup{
  pdftitle={Nine Objections to Steiner and Wolff on Land Disputes},
  pdfauthor={Brian Weatherson},
  hidelinks,
  pdfcreator={LaTeX via pandoc}}

\title{Nine Objections to Steiner and Wolff on Land Disputes}
\author{Brian Weatherson}
\date{2003}

\begin{document}
\maketitle
\begin{abstract}
Some objections to the idea that disputed territories should be
auctioned.
\end{abstract}

\setstretch{1.1}
In the July 2003 \emph{Analysis}, Hillel Steiner and Jonathan Wolff
(\citeproc{ref-Steiner2003}{2003}) propose a framework for ``resolving
disputed land claims between competing nations or ethnic groups.'' The
idea is that we should auction off the land, with the loser of the
auction getting the money. While this might mean that the richer party
will normally end up with the land, and this is normally not thought to
be a good thing, if the auction is conducted as they specify ``it will
turn out that the other party ends up with something which, in the
circumstances, it prefers to the land: lots of money.''

Actually, it isn't so clear that this is what will result. Let's say we
have a particular parcel of land that groups A and B want. They each
want it quite strongly, but B has deeper pockets than A, so while A
would be prepared to pay 8 for the land, B would be prepared to pay 12.
For the auction process to function, there must be a minimum bid
increment, I'll say it is ½. Assume that B has just bid 4, A must now
choose whether to bid 4½ or accept B's bid. And assume for now that A is
not bidding tactically, it only makes a bid if it would prefer to win
the auction with that bid than accept B's bid. This assumption will be
relaxed below.

So for now, A must decide whether it prefers to be given 4, or to get
the land for 4½. Since it values the land at 8, and since it will give
up 8½ to buy the land (the 4½ it will pay, plus the 4 it would have
received from B) it may well decide to just accept the bid. But now it
has ended up with something it definitely does \emph{not} prefer to the
land, since it just accepted a bid for 4. There are two assumptions at
play here. One is that A doesn't bid tactically, which I shall return to
a bit. The other is that how much A will pay for the land is not
affected by receiving B's 4. That is, I assume that the marginal utility
of money is relatively constant for A over the ranges of money at play
in the auction. This assumption might be false if we're dealing with a
very large or valuable body of land, but it's not unreasonable in most
circumstances. (Space prevents a complete study of what happens if we
take the declining marginal utility of money completely into account.
Roughly, the effect is that some of my criticisms are \emph{slightly}
vitiated.) Now while these assumptions \emph{might} be false, Steiner
and Wolff give us no reason to be certain they are false. So for all
they've said we could have a situation just like this one, where the
poorer party ends up with something it wants much less than the land.
Hence

\begin{quote}
\textbf{Objection 1}. There is no guarantee that the losing party will
end up with something they prefer to the land.
\end{quote}

While this contradicts an alleged benefit of Steiner and Wolff's plan,
it might not be thought to be a deep problem. After all, A gets half as
much as they wanted, and if they are only one of two equal claimants to
the land, then this is a fair result. This \emph{may} be true, but note
that the assumption that each party has an equal claim to the land is
doing a lot of work here. If A's claim is stronger, then only getting
half of the value of the land is quite unfair. If the two claims are
incommensurable, there may be no fact of the matter whether it is fair
that A receives 4. If we cannot tell which of the moral claims is
stronger, which is very often the case in land disputes, it may be
impossible to tell whether A's receiving 4 is fair or not. Hence

\begin{quote}
\textbf{Objection 2}. The proposal is only appropriate where each party
has a genuinely equal moral claim to the land. This doesn't happen
often, and it is quite rare that we know it happens.
\end{quote}

While Steiner and Wolff note that they are leaving questions about
enforcement and compliance to another place, so it isn't fair to press
them too strongly on these topics, it is worth noting how this feature
of their proposal makes compliance harder to enforce. If by
participating in the auction both parties are tacitly agreeing that the
other party has an equal claim to the land, and I think the above
suggests they are doing just this, that will reduce the legitimacy of
the auction process in the eyes of members of the losing group. And that
will lead to enforcement difficulties down the line.

There is an administrative problem lurking around here. Since each party
will end up with something from this process once the auction begins, we
must have a way of determining whether the competing claims warrant an
auction, or whether one party should receive the land, or whether some
kind of negotiation is possible. And once we set up a process to do
that, it could easily encourage relatively spurious land claims. Unless
there is a serious cost to suggesting that one should be party to an
auction of some block of land, there is a large incentive to get into
these auctions wherever and whenever possible. Perhaps some method could
be designed to offset this incentive, and perhaps even the desire groups
have to be approved by the court of public opinion will offset it at
times, but it seems to be a problem with the proposal as formulated.

To be sure, if A accepts B's bid, then both parties do end up with
\emph{something} from the auction. A gets 4, and B gets some land that
it values at 12 for 4, a gain of 8. Note that B does much better out of
the auction than A. If the auction stops when the richer party makes a
bid at or above half the price the poorer party would pay, then the
richer party will always end up with a higher `utility surplus'. Hence

\begin{quote}
\textbf{Objection 3}. If there's no tactical bidding the utility surplus
is given entirely to the richer party.
\end{quote}

Let's relax the assumption that A does not bid tactically. Indeed, let's
make things as good as could be realistically expected for A. It knows
that B values the land at 12 and does not bid tactically, so B will make
bids up to 6, and accept any bid over 6. Hence the auction proceeds as
follows: A bids 4½, B bids 5, A bids 5½, B bids 6, A accepts. Now things
\emph{could} go better for A, but it would require some luck and
courage. A could bid 6½ and B could reply with a bid of 7, but since
this requires B acting against its own interests (it is better off
accepting the bid of 6½ after all), and hence also requires A making a
risky move that will only yield dividends only if B acts against its own
interests in just this way, such an outcome seems unlikely. So in
practice the best case scenario for A is that B pays 6 for the land. In
this case A ends up with 6, and B ends up paying 6 for land it values at
12, a gain of 6. Hence

\begin{quote}
\textbf{Objection 4}. Among the realistic outcomes, the best case
scenario for the poorer party is that it ends up with as large a utility
surplus as the richer party.
\end{quote}

Best cases don't often happen, so in practice we should normally expect
a result somewhere between the `no tactical bargaining' option, where B
receives a larger share of the surplus, and this `best case scenario'
where the two parties get an equal share of the surplus. Hence in almost
all cases, the richer party will get a larger surplus than the poorer
party. This seems like a flaw in the proposal, but worse is to come.
Most of the ways in which B can realistically increase its share of the
surplus involve behaviour that we should not want to encourage.

Consider again A's decision to reject the bid of 5 and bid 5½. Assume,
for simplicity, that A plans to accept a bid of 6, but drop the
assumption that A knows that B will reject a bid of 5½, if it is made.
So before A makes its decision, there are three possible outcomes it
faces:

\begin{quote}
\emph{Accept the Bid}: In this case it receives 5.
\end{quote}

\begin{quote}
\emph{Bid 5}½ and have it accepted: In this case it gets the land (value
8) for 5½, net gain 2½.
\end{quote}

\begin{quote}
\emph{Bid 5}½ and have it rejected: In this case B bids 6, and A
accepts, so it gets 6.
\end{quote}

A's expected utility is higher if it bids 5½ rather than accepts B's bid
iff its degree of belief that B will bid 6 is over \(\frac{5}{7}\). If
it is less confident than that that B will bid 6, it should accept the
bid of 5. As it happens, B \emph{is} going to reject a bid of 5½ and bid
6, so it is better off if A accepts the bid of 5. If A knows B's plans,
this will not happen. But if A is ignorant of B's intentions, it is
possible it will accept the bid of 5. Indeed, since A's confidence that
B will decline must be as high as \(\frac{5}{7}\) before it makes the
bid of 5½, it might be quite likely in this case that A will just accept
the bid.

Not surprisingly, we get the result that B is better off if its
bargaining plans are kept secret than if they are revealed to A. That in
itself may not be objectionable. But remember that the agents here are
not individuals, they are states. And the decisions about how to bid
involve policy questions that will often be the most important issue the
state in question faces for many a year. Ideally, decisions about how to
approach the auction should be decided as democratically as possible.
But democratic decision making requires openness, and it is impossible
that all the stakeholders in B, including one imagines the citizens, can
participate in the decision about how to approach the auction without B
tipping its hand. In the modern world it's impossible to involve
everyone in B without opening the debate to agents of A. And this, as
we've seen, probably has costs. Since B is better off if it does not
make decisions about how to approach the auction in the open, we have

\begin{quote}
\textbf{Objection 5}. The proposal favours secretive governments over
open democratic governments.
\end{quote}

Assume that B has been somewhat secretive, but A is still fairly
confident that B will not accept a bid of 5½. Its degree of belief that
such a bid will be rejected is ¾, let's say, so it is disposed to gamble
and make that bid. But now B starts making some noises about what it
will do with any money it gets from A. The primary beneficiary of this
windfall will be B's military. And the primary use of this military is
to engage in military conflicts with A. While some of these engagements
will be defensive, if A gets the land under dispute many will be
offensive. (I don't think these assumptions are particularly fanciful in
many of the land disputes we see in the modern world.) A must take this
into account when making its decisions. It seems reasonable to say that
every 1 that A gives B has a disutility of 1.2 for A, 1 for the cost of
the money it gives up, and 0.2 for the extra damage it may suffer when
that money is turned into weaponry turned back against A. Now the
utility calculations are quite different. If B accepts A's bid of 5½,
A's balance sheet will look like this:

\begin{longtable}[]{@{}ll@{}}
\toprule\noalign{}
\endhead
\bottomrule\noalign{}
\endlastfoot
Gain: & The land, value 8. \\
Cost: & 5½ paid to B, value 5½ \\
Cost: & B's extra military capability, value (a little over) 1. \\
Net Gain: & Roughly 1½. \\
\end{longtable}

So now the expected utility of bidding 5½ is:

\begin{quote}
Prob(Bid Accepted) Utility(Bid Accepted) + Prob(Bid Rejected)
Utility(Bid Rejected)~ ≅ ¼ ✖️ 1½ + ¾ ✖️ 6\(\ 
= 4\)\frac{7}{8}\$
\end{quote}

Hence A's expected utility for accepting B's bid of 5, i.e.~5, is higher
than its expected utility of bidding 5½, so it will accept the bid, just
as B wanted it to do. So if B indicates that it will use any payments
from A to attack A, it may well be able to get the land for less. Hence

\begin{quote}
\textbf{Objection 6}. The proposal favours belligerent governments over
peaceful governments.
\end{quote}

One qualification to this objection is that what matters here is what A
thinks B will do, not what B actually does. So the objection is not that
the proposal rewards offensive behaviour, but that it rewards
belligerence, or indications of offensive behaviour. This isn't \emph{as
bad} as rewarding military action, but it is still objectionable.

Throughout I have used a particular example to make the points clearer,
none of the arguments turns on the details of this example. What matters
is that in any case where one party is able to spend more for the land
in question simply because they are richer, the richer party will almost
inevitably have a higher utility surplus, and this party can increase
their expected utility surplus by being more secretive about their
plans, and by being adopting a more belligerent tone towards their
rivals before and during the auction. So it seems the proposal
systematically rewards behaviour we should be discouraging.

The remaining objections concern the implementation of Steiner and
Wolff's proposal. While I don't have a demonstrative proof that any of
these concerns present insurmountable difficulties, they all suggest
ways in which the proposal must be qualified if it is to be just.

The proposal seems to assume that the parties to the dispute agree over
whether the land in question can be divided. As Steiner and Wolff put
it, ``The auction can thus be viewed as a device for achieving a fair
settlement for the disposition of a good when neither division nor joint
ownership is acceptable to the parties.'' In some conflicts at least
part of what is at issue is whether the land can be divided. For
instance, if we were applying this proposal as a way of settling the war
between Britain and Ireland in 1921, would we say that all of Ireland
should be auctioned off, or just that the six counties that became
Northern Ireland should be auctioned? Assuming the British had decided
that governing southern Ireland had become too much trouble and were
only interested in retaining the north, they may not have wanted to pay
for the whole country just to protect their interests in the north. But
at least some of the Irish would have been unwilling to accept a process
that may have led to the division of the country, as would have obtained
had the south been granted Home Rule, but the north left subject to an
auction. (The historical facts are, obviously, somewhat more complicated
than I've sketched here, but even when those complications are
considered the difficulties that must be overcome before we know how to
apply the proposal to a real situation are formidable.) Hence

\begin{quote}
\textbf{Objection 7}. The proposal assumes a mechanism for determining
which land is indivisible, and in some cases developing such a process
is no easier than settling the dispute.
\end{quote}

Steiner and Wolff assume that the groups, A and B, are easily
identifiable. In practice, this may not be so easy. For example, at
least some people in Scotland would prefer that Scotland was
independent. For now most people prefer devolution to independence (and
some would prefer rule from Westminster) but we can easily imagine
circumstances in which the nationalist support would rise to a level
where it became almost a majority. If a majority in Scotland wants to
secede, and the British government is willing to do this, then
presumably they will just secede. But what are we to do if a narrow
majority in Scotland wants to secede, and the British government (or
people) do not want them to go? Presumably Steiner and Wolff's proposal
is that some sort of auction should be held to determine who should be
in charge of the land. But who exactly are meant to be the parties? On
the Westminster side, is the party Britain as a whole, or Britain except
for Scotland? On the Scottish side, is it the Scottish people? The
Scottish government, which for now is a creature that exists at the
pleasure of the British Parliament? Those people who support Scottish
independence? If the last, how shall we determine just who these people
are? Perhaps some one or other of these answers can be defended, but the
proposal is seriously incomplete, hence

\begin{quote}
\textbf{Objection 8}. There is no mechanism for determining who shall
count as a member of the groups in question.
\end{quote}

Finally, the proposal simply assumes that we can agree upon the currency
in which the auction shall be conducted, but it is not ever so clear
that this can be done. Usually, the two parties to a dispute will use
different currencies, so to avoid conflicts it would be best if the
auction were conducted in a neutral currency. But finding such a
currency may be non-trivial. There are only a handful of currencies in
the world whose supply is sufficiently abundant to conduct an auction of
this size, and most of the time those currencies will be backed by
governments who favour one side in the dispute. If they use this
favouritism to provide access to credit denominated in their currency at
a discounted rate, that threatens the fairness of the auction. Hence

\begin{quote}
\textbf{Objection 9}. The proposal assumes a given currency in which to
conduct the auction, but in practice any choice of currency may favour
one side.
\end{quote}

The last three objections are, as mentioned, somewhat administrative. It
is possible that in a particular situation they could be overcome,
though I think that it is more likely that they would pose serious
difficulties to a would-be auction-wielding pacifier. But that's not the
serious problem with the proposal. The real problem, as the first six
objections show, is that it favours rich, secretive, belligerent states
that are disposed to make spurious land claims over poor, democratic,
pacifist states that only make genuine land claims.

\section*{References}\label{references}
\addcontentsline{toc}{section}{References}

\phantomsection\label{refs}
\begin{CSLReferences}{1}{0}
\bibitem[\citeproctext]{ref-Steiner2003}
Steiner, Hillel, and Jonathan Wolff. 2003. {``A General Framework for
Resolving Disputed Land Claims.''} \emph{Analysis} 63 (3): 188--89. doi:
\href{https://doi.org/10.1093/analys/63.3.188}{10.1093/analys/63.3.188}.

\end{CSLReferences}



\noindent Published in\emph{
Analysis}, 2003, pp. 321-327.

\end{document}
