% Options for packages loaded elsewhere
\PassOptionsToPackage{unicode}{hyperref}
\PassOptionsToPackage{hyphens}{url}
\PassOptionsToPackage{dvipsnames,svgnames,x11names}{xcolor}
%
\documentclass[
  11pt,
  letterpaper,
  DIV=11,
  numbers=noendperiod]{scrartcl}

\usepackage{amsmath,amssymb}
\usepackage{iftex}
\ifPDFTeX
  \usepackage[T1]{fontenc}
  \usepackage[utf8]{inputenc}
  \usepackage{textcomp} % provide euro and other symbols
\else % if luatex or xetex
  \ifXeTeX
    \usepackage{mathspec} % this also loads fontspec
  \else
    \usepackage{unicode-math} % this also loads fontspec
  \fi
  \defaultfontfeatures{Scale=MatchLowercase}
  \defaultfontfeatures[\rmfamily]{Ligatures=TeX,Scale=1}
\fi
\usepackage{lmodern}
\ifPDFTeX\else  
    % xetex/luatex font selection
  \setmainfont[Scale = MatchLowercase]{Scala Pro}
  \setsansfont[]{Scala Sans Pro}
  \ifXeTeX
    \setmathfont(Digits,Latin,Greek)[]{Scala Pro}
  \else
    \setmathfont[]{Scala Pro}
  \fi
\fi
% Use upquote if available, for straight quotes in verbatim environments
\IfFileExists{upquote.sty}{\usepackage{upquote}}{}
\IfFileExists{microtype.sty}{% use microtype if available
  \usepackage[]{microtype}
  \UseMicrotypeSet[protrusion]{basicmath} % disable protrusion for tt fonts
}{}
\makeatletter
\@ifundefined{KOMAClassName}{% if non-KOMA class
  \IfFileExists{parskip.sty}{%
    \usepackage{parskip}
  }{% else
    \setlength{\parindent}{0pt}
    \setlength{\parskip}{6pt plus 2pt minus 1pt}}
}{% if KOMA class
  \KOMAoptions{parskip=half}}
\makeatother
\usepackage{xcolor}
\setlength{\emergencystretch}{3em} % prevent overfull lines
\setcounter{secnumdepth}{3}
% Make \paragraph and \subparagraph free-standing
\ifx\paragraph\undefined\else
  \let\oldparagraph\paragraph
  \renewcommand{\paragraph}[1]{\oldparagraph{#1}\mbox{}}
\fi
\ifx\subparagraph\undefined\else
  \let\oldsubparagraph\subparagraph
  \renewcommand{\subparagraph}[1]{\oldsubparagraph{#1}\mbox{}}
\fi


\providecommand{\tightlist}{%
  \setlength{\itemsep}{0pt}\setlength{\parskip}{0pt}}\usepackage{longtable,booktabs,array}
\usepackage{calc} % for calculating minipage widths
% Correct order of tables after \paragraph or \subparagraph
\usepackage{etoolbox}
\makeatletter
\patchcmd\longtable{\par}{\if@noskipsec\mbox{}\fi\par}{}{}
\makeatother
% Allow footnotes in longtable head/foot
\IfFileExists{footnotehyper.sty}{\usepackage{footnotehyper}}{\usepackage{footnote}}
\makesavenoteenv{longtable}
\usepackage{graphicx}
\makeatletter
\def\maxwidth{\ifdim\Gin@nat@width>\linewidth\linewidth\else\Gin@nat@width\fi}
\def\maxheight{\ifdim\Gin@nat@height>\textheight\textheight\else\Gin@nat@height\fi}
\makeatother
% Scale images if necessary, so that they will not overflow the page
% margins by default, and it is still possible to overwrite the defaults
% using explicit options in \includegraphics[width, height, ...]{}
\setkeys{Gin}{width=\maxwidth,height=\maxheight,keepaspectratio}
% Set default figure placement to htbp
\makeatletter
\def\fps@figure{htbp}
\makeatother
% definitions for citeproc citations
\NewDocumentCommand\citeproctext{}{}
\NewDocumentCommand\citeproc{mm}{%
  \begingroup\def\citeproctext{#2}\cite{#1}\endgroup}
\makeatletter
 % allow citations to break across lines
 \let\@cite@ofmt\@firstofone
 % avoid brackets around text for \cite:
 \def\@biblabel#1{}
 \def\@cite#1#2{{#1\if@tempswa , #2\fi}}
\makeatother
\newlength{\cslhangindent}
\setlength{\cslhangindent}{1.5em}
\newlength{\csllabelwidth}
\setlength{\csllabelwidth}{3em}
\newenvironment{CSLReferences}[2] % #1 hanging-indent, #2 entry-spacing
 {\begin{list}{}{%
  \setlength{\itemindent}{0pt}
  \setlength{\leftmargin}{0pt}
  \setlength{\parsep}{0pt}
  % turn on hanging indent if param 1 is 1
  \ifodd #1
   \setlength{\leftmargin}{\cslhangindent}
   \setlength{\itemindent}{-1\cslhangindent}
  \fi
  % set entry spacing
  \setlength{\itemsep}{#2\baselineskip}}}
 {\end{list}}
\usepackage{calc}
\newcommand{\CSLBlock}[1]{\hfill\break#1\hfill\break}
\newcommand{\CSLLeftMargin}[1]{\parbox[t]{\csllabelwidth}{\strut#1\strut}}
\newcommand{\CSLRightInline}[1]{\parbox[t]{\linewidth - \csllabelwidth}{\strut#1\strut}}
\newcommand{\CSLIndent}[1]{\hspace{\cslhangindent}#1}

\setlength\heavyrulewidth{0ex}
\setlength\lightrulewidth{0ex}
\makeatletter
\def\@maketitle{%
\newpage
\null
\vskip 2em%
\begin{center}%
\let \footnote \thanks
  {\LARGE \@title \par}%
  \vskip 1.5em%
  {\large
    \lineskip .5em%
    \begin{tabular}[t]{c}%
      \@author
    \end{tabular}\par}%
  %\vskip 1em%
  %{\large \@date}%
\end{center}%
\par
\vskip 1.5em}
\makeatother
\KOMAoption{captions}{tableheading}
\makeatletter
\@ifpackageloaded{caption}{}{\usepackage{caption}}
\AtBeginDocument{%
\ifdefined\contentsname
  \renewcommand*\contentsname{Table of contents}
\else
  \newcommand\contentsname{Table of contents}
\fi
\ifdefined\listfigurename
  \renewcommand*\listfigurename{List of Figures}
\else
  \newcommand\listfigurename{List of Figures}
\fi
\ifdefined\listtablename
  \renewcommand*\listtablename{List of Tables}
\else
  \newcommand\listtablename{List of Tables}
\fi
\ifdefined\figurename
  \renewcommand*\figurename{Figure}
\else
  \newcommand\figurename{Figure}
\fi
\ifdefined\tablename
  \renewcommand*\tablename{Table}
\else
  \newcommand\tablename{Table}
\fi
}
\@ifpackageloaded{float}{}{\usepackage{float}}
\floatstyle{ruled}
\@ifundefined{c@chapter}{\newfloat{codelisting}{h}{lop}}{\newfloat{codelisting}{h}{lop}[chapter]}
\floatname{codelisting}{Listing}
\newcommand*\listoflistings{\listof{codelisting}{List of Listings}}
\makeatother
\makeatletter
\makeatother
\makeatletter
\@ifpackageloaded{caption}{}{\usepackage{caption}}
\@ifpackageloaded{subcaption}{}{\usepackage{subcaption}}
\makeatother
\ifLuaTeX
  \usepackage{selnolig}  % disable illegal ligatures
\fi
\IfFileExists{bookmark.sty}{\usepackage{bookmark}}{\usepackage{hyperref}}
\IfFileExists{xurl.sty}{\usepackage{xurl}}{} % add URL line breaks if available
\urlstyle{same} % disable monospaced font for URLs
\hypersetup{
  pdftitle={Reply to Blackson},
  pdfauthor={Brian Weatherson},
  colorlinks=true,
  linkcolor={black},
  filecolor={Maroon},
  citecolor={Blue},
  urlcolor={Blue},
  pdfcreator={LaTeX via pandoc}}

\title{Reply to Blackson}
\author{Brian Weatherson}
\date{2016-01-01}

\begin{document}
\maketitle
It isn't true that all interest-invariant epistemologies are alike, but
it is certainly true that every interest-relative theory is
interest-relative in its own idiosyncratic way. In fact, there are at
least four dimensions along which a theory can be interest-relative.

I used to think ~(\citeproc{ref-Weatherson2005-WEACWD}{Weatherson 2005})
that interest-relativity in knowledge was to be explained by
interest-relativity in belief, but I came to think that's not true
~(\citeproc{ref-Weatherson2012-WEAKBI}{Weatherson 2012}). Some prominent
defenders of interest-relativity in epistemology focus on practical
interests -- it's even there in the title of the book by Jason Stanley
(\citeproc{ref-Stanley2005-STAKAP}{2005}) -- but others of us think that
theoretical interests matter too. At times Stanley writes as if
interest-relativity means that there is an extra clause in the theory of
knowledge for interests, but one need not think that. It could, for
example, be that there is an interest-sensitive domain restriction on a
quantifier in one of the clauses.

But for present purposes, the key divide among interest-relative
epistemologists is between those who think that stakes are relevant, and
those who think odds are relevant. I think, following Mark Schroeder
(\citeproc{ref-Schroeder2012}{2012}), that it is odds that matter. The
key examples here are ones where there is little cost to gambling and
getting it wrong, but even less to gain by gambling and getting it
right.

So imagine that Ankita is walking to a restaurant she hasn't been to for
a few months. She is stopped at the lights, reading baseball scores on
her phone. She is almost, but not completely, certain that she should
turn left at the next block, which indeed she should. If she had been
wrong, she would have gone two blocks out of her way. She could avoid
this risk by flipping from her baseball app to the map on her phone and
checking the address, all of which she could do before the lights
change. I say that in this circumstance, she doesn't know where the
restaurant is. She should look up where it is, that's what maximises
expected utility, but she needn't look up restaurants she knows the
location of. So she doesn't know whether the restaurant is to her left
or her right. Ankita's case is not a high-stakes one. Even on a cold
Michigan Fall evening, the downside of walking two extra blocks is not
that high. But unless she really really cares about those baseball
scores she's browsing through, deciding not to flip over to the map is a
gamble on the correctness of her plans at incredibly long odds. That's
not because the stakes are high, but because the gain from gambling is
low.

This is a genuine case of interest-relativity though. The argument I
just gave wouldn't go through if Ankita would have no disutility
whatsoever from walking two blocks out of her way. (Maybe the exercise
gain would completely outweigh the frustration.) In that case, perhaps
she does know. But if the two block walk has many times the disutility
of no longer browsing baseball scores, as it would in most realistic
cases, her interests defeat her knowledge of the restaurant location.

The same thing I think is going on in the example Thomas Blackson
(\citeproc{ref-Blackson2015}{2016}) gives. The agent has a three-way
choice, between taking drug A, taking drug B and doing nothing. The
upsides of both drugs are the same; they alleviate a minor medical
condition. The downsides are the same too; they lead to death in rare
cases. But this is much rarer still in the case of B than A. So, says
Blackson, the person should take drug B. And I agree. But, says
Blackson, this is a problem, because I'm committed to the following
argument.

\begin{enumerate}
\def\labelenumi{\arabic{enumi}.}
\tightlist
\item
  The agent knows that drug A and drug B won't kill them.
\item
  Any option an agent knows not to obtain can be left off a decision
  table.
\item
  So, from 1 and 2, the decision table the agent faces has only the
  upsides, and not the downsides.
\item
  So, from 3, the decision tables for the two drugs are the same.
\item
  So, from 4, the agent can be indifferent between the drugs.
\end{enumerate}

Since 5 is false, one of the premises must be false. Blackson says that
the false premise is 2. I say that it is 1; the patient doesn't know
drug A won't kill them. Blackson anticipates this, and says that the
move won't work because of a related case. If drug B didn't exist, the
interest-relative theorist might say that the agent has enough evidence
to know the drug won't kill them. Maybe that's true, but it isn't clear
why it is relevant. The existence of drug B changes the gamble involved
in taking drug A. It must, since taking drug A is irrational in the
original case, but rational in the version where drug B doesn't exist.
So if what the agent knows is sensitive to what gambles they face, then
it isn't surprising that the presence of extra options changes what the
agent knows.

And the interest-relative theory has a nice explanation of one feature
of Blackson's case. Imagine the agent learns that drug A is only ever
fatal with people with blood type A2B negative, and that's not their
blood type. There's no similarly known marking for when drug B is fatal.
Now the agent can say, ``Now that I know drug A won't kill me, I'm going
to take it, not drug B.'' That seems like just the right thing to say,
but on Blackson's telling of the case, they can't say it. After all,
they knew all along that drug A wouldn't kill them. Instead he has to
say that the agent shouldn't take drug A before they learn this, because
of the risk it would kill them, even though the agent knows the drug
won't kill them. That doesn't sound at all right. Much better, I think,
to say they shouldn't take drug A before learning who it endangers,
because it exposes them to a needless risk of death, but once they know
it won't kill them, it is good to take it.

And that's the general strategy for defending interest-relative
treatments of cases like Blackson's, and the others he describes. It's a
strategy that I don't think is threatened by any example to date.
There's some action that all parties agree would be irrational, or at
least not rationally mandatory. But there's some evidence we can imagine
the agent getting that would improve the rational status of the action.
(E.g., change it from being rationally impermissible to rationally
permissible, or from not rationally required to rationally required.) If
we asked the rational agent why they changed their plans after getting
that evidence, it seems to make sense for them to say that they now know
the action would have a good outcome. That is, it makes sense for them
to cite their knowledge as a reason for doing something different after
the evidence-gathering event. And only the interest-relative
epistemologist can explain why that is a sensible answer for them to
give.

\section*{References}\label{references}
\addcontentsline{toc}{section}{References}

\phantomsection\label{refs}
\begin{CSLReferences}{1}{0}
\bibitem[\citeproctext]{ref-Blackson2015}
Blackson, Thomas. 2016. {``Against Weatherson on How to Frame a Decision
Problem.''} \emph{Journal of Philosophical Research} 41: 69--72.
\url{https://doi.org/10.5840/jpr201662868}.

\bibitem[\citeproctext]{ref-Schroeder2012}
Schroeder, Mark. 2012. {``Stakes, Withholding and Pragmatic Encroachment
on Knowledge.''} \emph{Philosophical Studies} 160 (2): 265--85.
\url{https://doi.org/10.1007/s11098-011-9718-1}.

\bibitem[\citeproctext]{ref-Stanley2005-STAKAP}
Stanley, Jason. 2005. \emph{{Knowledge and Practical Interests}}. Oxford
University Press.

\bibitem[\citeproctext]{ref-Weatherson2005-WEACWD}
Weatherson, Brian. 2005. {``{Can We Do Without Pragmatic
Encroachment?}''} \emph{Philosophical Perspectives} 19 (1): 417--43.
\url{https://doi.org/10.1111/j.1520-8583.2005.00068.x}.

\bibitem[\citeproctext]{ref-Weatherson2012-WEAKBI}
---------. 2012. {``Knowledge, Bets and Interests.''} In \emph{Knowledge
Ascriptions}, edited by Jessica Brown and Mikkel Gerken, 75--103.
Oxford: Oxford University Press.

\end{CSLReferences}



\end{document}
