% Options for packages loaded elsewhere
\PassOptionsToPackage{unicode}{hyperref}
\PassOptionsToPackage{hyphens}{url}
%
\documentclass[
  10pt,
  letterpaper,
  DIV=11,
  numbers=noendperiod,
  twoside]{scrartcl}

\usepackage{amsmath,amssymb}
\usepackage{setspace}
\usepackage{iftex}
\ifPDFTeX
  \usepackage[T1]{fontenc}
  \usepackage[utf8]{inputenc}
  \usepackage{textcomp} % provide euro and other symbols
\else % if luatex or xetex
  \usepackage{unicode-math}
  \defaultfontfeatures{Scale=MatchLowercase}
  \defaultfontfeatures[\rmfamily]{Ligatures=TeX,Scale=1}
\fi
\usepackage{lmodern}
\ifPDFTeX\else  
    % xetex/luatex font selection
    \setmainfont[ItalicFont=EB Garamond Italic,BoldFont=EB Garamond
Bold]{EB Garamond Math}
    \setsansfont[]{Europa-Bold}
  \setmathfont[]{Garamond-Math}
\fi
% Use upquote if available, for straight quotes in verbatim environments
\IfFileExists{upquote.sty}{\usepackage{upquote}}{}
\IfFileExists{microtype.sty}{% use microtype if available
  \usepackage[]{microtype}
  \UseMicrotypeSet[protrusion]{basicmath} % disable protrusion for tt fonts
}{}
\usepackage{xcolor}
\usepackage[left=1in, right=1in, top=0.8in, bottom=0.8in,
paperheight=9.5in, paperwidth=6.5in, includemp=TRUE, marginparwidth=0in,
marginparsep=0in]{geometry}
\setlength{\emergencystretch}{3em} % prevent overfull lines
\setcounter{secnumdepth}{3}
% Make \paragraph and \subparagraph free-standing
\makeatletter
\ifx\paragraph\undefined\else
  \let\oldparagraph\paragraph
  \renewcommand{\paragraph}{
    \@ifstar
      \xxxParagraphStar
      \xxxParagraphNoStar
  }
  \newcommand{\xxxParagraphStar}[1]{\oldparagraph*{#1}\mbox{}}
  \newcommand{\xxxParagraphNoStar}[1]{\oldparagraph{#1}\mbox{}}
\fi
\ifx\subparagraph\undefined\else
  \let\oldsubparagraph\subparagraph
  \renewcommand{\subparagraph}{
    \@ifstar
      \xxxSubParagraphStar
      \xxxSubParagraphNoStar
  }
  \newcommand{\xxxSubParagraphStar}[1]{\oldsubparagraph*{#1}\mbox{}}
  \newcommand{\xxxSubParagraphNoStar}[1]{\oldsubparagraph{#1}\mbox{}}
\fi
\makeatother


\providecommand{\tightlist}{%
  \setlength{\itemsep}{0pt}\setlength{\parskip}{0pt}}\usepackage{longtable,booktabs,array}
\usepackage{calc} % for calculating minipage widths
% Correct order of tables after \paragraph or \subparagraph
\usepackage{etoolbox}
\makeatletter
\patchcmd\longtable{\par}{\if@noskipsec\mbox{}\fi\par}{}{}
\makeatother
% Allow footnotes in longtable head/foot
\IfFileExists{footnotehyper.sty}{\usepackage{footnotehyper}}{\usepackage{footnote}}
\makesavenoteenv{longtable}
\usepackage{graphicx}
\makeatletter
\def\maxwidth{\ifdim\Gin@nat@width>\linewidth\linewidth\else\Gin@nat@width\fi}
\def\maxheight{\ifdim\Gin@nat@height>\textheight\textheight\else\Gin@nat@height\fi}
\makeatother
% Scale images if necessary, so that they will not overflow the page
% margins by default, and it is still possible to overwrite the defaults
% using explicit options in \includegraphics[width, height, ...]{}
\setkeys{Gin}{width=\maxwidth,height=\maxheight,keepaspectratio}
% Set default figure placement to htbp
\makeatletter
\def\fps@figure{htbp}
\makeatother

\setlength\heavyrulewidth{0ex}
\setlength\lightrulewidth{0ex}
\usepackage[automark]{scrlayer-scrpage}
\clearpairofpagestyles
\cehead{
  Brian Weatherson
  }
\cohead{
  Review of “Vagueness and Contradiction”
  }
\ohead{\bfseries \pagemark}
\cfoot{}
\makeatletter
\newcommand*\NoIndentAfterEnv[1]{%
  \AfterEndEnvironment{#1}{\par\@afterindentfalse\@afterheading}}
\makeatother
\NoIndentAfterEnv{itemize}
\NoIndentAfterEnv{enumerate}
\NoIndentAfterEnv{description}
\NoIndentAfterEnv{quote}
\NoIndentAfterEnv{equation}
\NoIndentAfterEnv{longtable}
\NoIndentAfterEnv{abstract}
\renewenvironment{abstract}
 {\vspace{-1.25cm}
 \quotation\small\noindent\rule{\linewidth}{.5pt}\par\smallskip
 \noindent }
 {\par\noindent\rule{\linewidth}{.5pt}\endquotation}
\KOMAoption{captions}{tableheading}
\makeatletter
\@ifpackageloaded{caption}{}{\usepackage{caption}}
\AtBeginDocument{%
\ifdefined\contentsname
  \renewcommand*\contentsname{Table of contents}
\else
  \newcommand\contentsname{Table of contents}
\fi
\ifdefined\listfigurename
  \renewcommand*\listfigurename{List of Figures}
\else
  \newcommand\listfigurename{List of Figures}
\fi
\ifdefined\listtablename
  \renewcommand*\listtablename{List of Tables}
\else
  \newcommand\listtablename{List of Tables}
\fi
\ifdefined\figurename
  \renewcommand*\figurename{Figure}
\else
  \newcommand\figurename{Figure}
\fi
\ifdefined\tablename
  \renewcommand*\tablename{Table}
\else
  \newcommand\tablename{Table}
\fi
}
\@ifpackageloaded{float}{}{\usepackage{float}}
\floatstyle{ruled}
\@ifundefined{c@chapter}{\newfloat{codelisting}{h}{lop}}{\newfloat{codelisting}{h}{lop}[chapter]}
\floatname{codelisting}{Listing}
\newcommand*\listoflistings{\listof{codelisting}{List of Listings}}
\makeatother
\makeatletter
\makeatother
\makeatletter
\@ifpackageloaded{caption}{}{\usepackage{caption}}
\@ifpackageloaded{subcaption}{}{\usepackage{subcaption}}
\makeatother

\ifLuaTeX
  \usepackage{selnolig}  % disable illegal ligatures
\fi
\usepackage{bookmark}

\IfFileExists{xurl.sty}{\usepackage{xurl}}{} % add URL line breaks if available
\urlstyle{same} % disable monospaced font for URLs
\hypersetup{
  pdftitle={Review of ``Vagueness and Contradiction''},
  pdfauthor={Brian Weatherson},
  hidelinks,
  pdfcreator={LaTeX via pandoc}}


\title{Review of ``Vagueness and Contradiction''}
\author{Brian Weatherson}
\date{2003}

\begin{document}
\maketitle
\begin{abstract}
Review of Roy Sorensen, ``Vagueness and Contradiction''. Cambridge:
Cambridge University Press, 2000.
\end{abstract}


\setstretch{1.1}
Like all epistemicists, Roy Sorensen holds that vagueness poses no
threat to classical logic, and that appearances to the contrary are the
result of mistakenly assigning semantic force to certain barriers to
inquiry. We may not be able to know whether 932 seconds after noon is
still noonish, but there is a fact of the matter about whether it is.
Hence the sentence~\emph{932 seconds after noon is noonish}~is either
true or false, just as adherents of classical logic presuppose, and the
threat from vagueness to classical logic dissolves.

But Sorensen is not an orthodox epistemicist. He does not hold that
these barriers to inquiry rise because of our limited powers of
discrimination. That would imply that a more discerning observer, say
God, could know where the boundary is. Sorensen holds that vagueness is
an~\emph{absolute}~barrier to inquiry. No one can know whether 932
seconds after noon is noonish, even God. This is because competently
using the vague predicate `noonish' requires believing a particular
analytic falsehood involving it, and having this belief
prevents~\emph{knowing} the truth about a borderline case. Much of this
book is dedicated to defending these surprising claims. The first half
of the book argues that this is the right thing to say about vague
cases, and the second half provides more general arguments that we can
and should believe analytic falsehoods.

It's illuminating to compare Sorensen's epistemicism with that of
Timothy Williamson. A core feature of Williamson's position is neatly
summarised in this quote, which Sorensen cites: ``for the epistemicist,
definiteness is truth under all sharp interpretations of the language
indiscriminable from the right one.'' (``On the Structure of Higher
Order Vagueness''~\emph{Mind}~108 (1999): 127‑43.)~~Sorensen disagrees
with the Williamson's position in four ways. Two of these disagreements
are immediate, and two are with deeper presuppositions of Williamson's.

First, Sorensen thinks that Williamson here ignores the need for
`completeness'. Williamson holds that~\emph{p}~is definite iff, roughly,
it is true on all interpretations we cannot know to be incorrect. Call
these the admissible interpretations. Sorensen claims that is not enough
for~\emph{p}~to be knowable, and hence definitely true. It must also be
knowable that these are all the admissible interpretations.

Secondly, indiscriminability is always indiscriminability by something,
so on Williamson's account definiteness is only defined relative to a
discriminator. Sorensen wants there to be absolute borderline cases, and
absolute indefiniteness, so he cannot rest with this definition.
Sorensen thinks that unless we accept absolute borderline cases, we do
not properly respect the sense in which vagueness is an absolute barrier
to inquiry. The two other innovations in Sorensen's theory guarantee
that his theory has place for absolute indefiniteness.

Consider a normal Sorites conditional, say~\emph{If 932 seconds after
noon is noonish, so is 933 seconds}. Sorensen holds that being a
competent user of `noonish' requires that one believe every such
conditional involving `noonish'. Someone who failed to believe it would
not be competent in the language. Although Sorensen always puts this in
terms of linguistic competence, he also says that one who didn't believe
this couldn't have beliefs about the extension of our predicates. The
most natural conclusion to draw is that from Sorensen's perspective, one
who doesn't believe the Sorites conditional lacks the concept NOONISH.
Sorensen talks about predicates rather than concepts, so he doesn't put
it quite this way, but it succinctly summarises the picture he sketches.
Moreover, beliefs in such Sorites conditionals are~\emph{a priori},
despite the fact that one of them is analytically false. These are
distinctive views, and they need good arguments.

A bad argument would be, ``It is always irrational to deny a Sorites
conditional, so it is always rational to believe it.'' This ignores the
possibility that agnosticism about the conditional is always possible,
and sometimes desirable. Sorensen does not endorse this argument, though
he does note it shows that Sorites conditionals satisfy a `negative
conception' of the~\emph{a priori}: no empirical evidence can make us
believe they are~\emph{false}.

Sorensen's argument seems to be that believing every Sorites conditional
gives us many true beliefs at the cost of only one false belief. He
thinks that cost is worth the benefit. But this is at best a reason
why~\emph{we}~should believe Sorites conditionals, not why God should.
And it doesn't imply much about why God needs to believe this falsehood
if He is to have the concept NOONISH. If we think subjectivism about
language implies that God can't know more about~\emph{our}~language than
we do, that may draw God into our dilemma. But it should seem very
implausible, especially to an epistemicist, that God can't know more
about our language than we do. So even if it is good advice to believe
every Sorites conditional, it does not follow that those who spurn this
advice lack any concepts, or lack linguistic competence.

In any case, there are other costs to adopting Sorensen's advice and
believing a bunch of Sorites conditionals that we know includes a
falsehood. For example, we can no longer safely believe the logical
consequences of some things we believe. Sorensen happily accepts that
consequence. He holds~\emph{p}~and~\emph{q}~can both be~\emph{a
priori}~even though their conjunction is not~\emph{a priori}. In chapter
6 Sorensen replies to several arguments against this, including a
purported proof that~\emph{a priority}~agglomerates across conjunction.
The proof assumes that all logical truths are~\emph{a priori}. Sorensen
says this is false because there are logical truths that are too complex
for us to believe,~\emph{a priori}~or otherwise. Since the only logical
truth needed in the proof
was~\emph{p}~→~(\emph{q}~→~(\emph{p}~\&~\emph{q})), this is not
obviously a sound response.

Sorensen has a more speculative reason for thinking there is absolute
vagueness. (This is the final way in which Sorensen's position differs
from Williamson.) Consider a card that has~\emph{The sentence on the
other side of this card is false}~written on each side. If the sentences
have truth values, then one is true and the other false. Whichever is
true is a truth without a truthmaker, for any truthmaker would do just
as well at making the other true. So Sorensen concludes that here we
have a truth without a truthmaker, and the truthmaker principle is
false. If there are some truths without truthmakers, there could be
several. Sorensen holds that~\emph{a is F}~is such a truth
whenever~\emph{a}~is an~\emph{F}~which is a borderline~\emph{F}. Assume
further that only truths with truthmakers are knowable, because
knowability goes via knowing truthmakers, and we conclude that no one
could know of a borderline~\emph{F}~whether it is~\emph{F}. This is
quite an interesting line of thought, and it deserves further attention.
(Sorensen is quite upfront about how speculative it is.) Two immediate
issues spring to mind. First, it is not clear how this is still a
version of epistemicism, for vagueness is now at base a metaphysical
phenomenon. There are epistemic consequences, but vagueness is
constituted by the fact that there are truths without truthmakes, not by
the unknowability of these. Secondly, and relatedly, it is no longer
clear how higher order vagueness will be incorporated into the model.
The most obvious thought is that there will be no truthmaker for the
claim that some particular truth has a truthmaker. But whether some
object is a truthmaker for some truth is not contingent, and it is
notoriously difficult to apply truthmaker theory to necessary truths.
Since every proposition entails any necessary truth, it is plausible
that any object is a truthmaker for a necessary truth.

Sorensen argues that we should believe all Sorites conditionals,
including ones that are analytically false. He notes this requires an
argument that we can, and should, believe some analytic falsehoods. (He
calls these impossibilities `contradictions', a term some may think
should be reserved for sentences of the form~\emph{p}~\&~¬\emph{p}.) His
argument that we can is fairly quick. Assume, for reductio, the
philosophical thesis that we cannot believe analytic falsehoods. As a
philosophical thesis, this is analytically true if true at all. But
Sorensen believes its negation. So it is possible for someone to believe
an analytic falsehood. The weakest premise here is that if we cannot
believe analytic falsehoods, then it is analytic that we cannot. If it
turns out that only creatures with a language of thought can believe
analytic falsehoods, and it is a contingent feature of us that we lack a
language of thought, Sorensen's premise is false.

The argument that we should believe some analytic falsehoods uses a
version of the preface paradox. If we can believe analytic falsehoods,
we should take apparent occurrences of this (as when we make arithmetic
errors) at face value. That is, reason demands we believe that we
believe an analytic falsehood. But this implies it is provable that our
beliefs are not collectively true. So if we follow the dictates of
reason, it is provable we believe something provably false. This
argument is obviously useful for removing a particular barrier to
accepting Sorensen's account of vagueness, that it seems absurd that
reason could require we believe an analytic falsehood. But even if we
reject Sorensen's theory of vagueness, they are independently
interesting contributions to the theory of belief.

Sorensen makes two distinctive contributions to the theory of vagueness
here. First, he argues that linguistic competence demands that we
believe Sorites conditionals. Secondly, he links the existence of
vagueness to the failure of the truthmaker principle. As those familiar
with Sorensen's work will suspect, he makes these contributions in a
lively and entertaining way. Anyone working on vagueness, and especially
anyone interested in investigating the range of theories of vagueness
that preserve classical logic, should pay it close attention.

\vspace{1cm}



\noindent Published in\emph{
Australasian Journal of Philosophy}, 2003, pp. 290-292.


\end{document}
