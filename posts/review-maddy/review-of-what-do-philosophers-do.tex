% Options for packages loaded elsewhere
% Options for packages loaded elsewhere
\PassOptionsToPackage{unicode}{hyperref}
\PassOptionsToPackage{hyphens}{url}
%
\documentclass[
  11pt,
  letterpaper,
  DIV=11,
  numbers=noendperiod,
  twoside]{scrartcl}
\usepackage{xcolor}
\usepackage[left=1.1in, right=1in, top=0.8in, bottom=0.8in,
paperheight=9.5in, paperwidth=7in, includemp=TRUE, marginparwidth=0in,
marginparsep=0in]{geometry}
\usepackage{amsmath,amssymb}
\setcounter{secnumdepth}{3}
\usepackage{iftex}
\ifPDFTeX
  \usepackage[T1]{fontenc}
  \usepackage[utf8]{inputenc}
  \usepackage{textcomp} % provide euro and other symbols
\else % if luatex or xetex
  \usepackage{unicode-math} % this also loads fontspec
  \defaultfontfeatures{Scale=MatchLowercase}
  \defaultfontfeatures[\rmfamily]{Ligatures=TeX,Scale=1}
\fi
\usepackage{lmodern}
\ifPDFTeX\else
  % xetex/luatex font selection
  \setmathfont[]{Garamond-Math}
\fi
% Use upquote if available, for straight quotes in verbatim environments
\IfFileExists{upquote.sty}{\usepackage{upquote}}{}
\IfFileExists{microtype.sty}{% use microtype if available
  \usepackage[]{microtype}
  \UseMicrotypeSet[protrusion]{basicmath} % disable protrusion for tt fonts
}{}
\usepackage{setspace}
% Make \paragraph and \subparagraph free-standing
\makeatletter
\ifx\paragraph\undefined\else
  \let\oldparagraph\paragraph
  \renewcommand{\paragraph}{
    \@ifstar
      \xxxParagraphStar
      \xxxParagraphNoStar
  }
  \newcommand{\xxxParagraphStar}[1]{\oldparagraph*{#1}\mbox{}}
  \newcommand{\xxxParagraphNoStar}[1]{\oldparagraph{#1}\mbox{}}
\fi
\ifx\subparagraph\undefined\else
  \let\oldsubparagraph\subparagraph
  \renewcommand{\subparagraph}{
    \@ifstar
      \xxxSubParagraphStar
      \xxxSubParagraphNoStar
  }
  \newcommand{\xxxSubParagraphStar}[1]{\oldsubparagraph*{#1}\mbox{}}
  \newcommand{\xxxSubParagraphNoStar}[1]{\oldsubparagraph{#1}\mbox{}}
\fi
\makeatother


\usepackage{longtable,booktabs,array}
\usepackage{calc} % for calculating minipage widths
% Correct order of tables after \paragraph or \subparagraph
\usepackage{etoolbox}
\makeatletter
\patchcmd\longtable{\par}{\if@noskipsec\mbox{}\fi\par}{}{}
\makeatother
% Allow footnotes in longtable head/foot
\IfFileExists{footnotehyper.sty}{\usepackage{footnotehyper}}{\usepackage{footnote}}
\makesavenoteenv{longtable}
\usepackage{graphicx}
\makeatletter
\newsavebox\pandoc@box
\newcommand*\pandocbounded[1]{% scales image to fit in text height/width
  \sbox\pandoc@box{#1}%
  \Gscale@div\@tempa{\textheight}{\dimexpr\ht\pandoc@box+\dp\pandoc@box\relax}%
  \Gscale@div\@tempb{\linewidth}{\wd\pandoc@box}%
  \ifdim\@tempb\p@<\@tempa\p@\let\@tempa\@tempb\fi% select the smaller of both
  \ifdim\@tempa\p@<\p@\scalebox{\@tempa}{\usebox\pandoc@box}%
  \else\usebox{\pandoc@box}%
  \fi%
}
% Set default figure placement to htbp
\def\fps@figure{htbp}
\makeatother





\setlength{\emergencystretch}{3em} % prevent overfull lines

\providecommand{\tightlist}{%
  \setlength{\itemsep}{0pt}\setlength{\parskip}{0pt}}



 


\setlength\heavyrulewidth{0ex}
\setlength\lightrulewidth{0ex}
\usepackage[automark]{scrlayer-scrpage}
\clearpairofpagestyles
\cehead{
  Brian Weatherson
  }
\cohead{
  Review of “What do Philosophers Do”
  }
\ohead{\bfseries \pagemark}
\cfoot{}
\makeatletter
\newcommand*\NoIndentAfterEnv[1]{%
  \AfterEndEnvironment{#1}{\par\@afterindentfalse\@afterheading}}
\makeatother
\NoIndentAfterEnv{itemize}
\NoIndentAfterEnv{enumerate}
\NoIndentAfterEnv{description}
\NoIndentAfterEnv{quote}
\NoIndentAfterEnv{equation}
\NoIndentAfterEnv{longtable}
\NoIndentAfterEnv{abstract}
\renewenvironment{abstract}
 {\vspace{-1.25cm}
 \quotation\small\noindent\emph{Abstract}:}
 {\endquotation}
\newfontfamily\tfont{EB Garamond}
\addtokomafont{disposition}{\rmfamily}
\addtokomafont{title}{\normalfont\itshape}
\let\footnoterule\relax

\makeatletter
\renewcommand{\@maketitle}{%
  \newpage
  \null
  \vskip 2em%
  \begin{center}%
  \let \footnote \thanks
    {\itshape\huge\@title \par}%
    \vskip 0.5em%  % Reduced from default
    {\large
      \lineskip 0.3em%  % Reduced from default 0.5em
      \begin{tabular}[t]{c}%
        \@author
      \end{tabular}\par}%
    \vskip 0.5em%  % Reduced from default
    {\large \@date}%
  \end{center}%
  \par
  }
\makeatother
\RequirePackage{lettrine}

\renewenvironment{abstract}
 {\quotation\small\noindent\emph{Abstract}:}
 {\endquotation\vspace{-0.02cm}}

\setmainfont{EB Garamond Math}[
  BoldFont = {EB Garamond SemiBold},
  ItalicFont = {EB Garamond Italic},
  RawFeature = {+smcp},
]

\newfontfamily\scfont{EB Garamond Regular}[RawFeature=+smcp]
\renewcommand{\textsc}[1]{{\scfont #1}}

\renewcommand{\LettrineTextFont}{\scfont}
\KOMAoption{captions}{tableheading}
\makeatletter
\@ifpackageloaded{caption}{}{\usepackage{caption}}
\AtBeginDocument{%
\ifdefined\contentsname
  \renewcommand*\contentsname{Table of contents}
\else
  \newcommand\contentsname{Table of contents}
\fi
\ifdefined\listfigurename
  \renewcommand*\listfigurename{List of Figures}
\else
  \newcommand\listfigurename{List of Figures}
\fi
\ifdefined\listtablename
  \renewcommand*\listtablename{List of Tables}
\else
  \newcommand\listtablename{List of Tables}
\fi
\ifdefined\figurename
  \renewcommand*\figurename{Figure}
\else
  \newcommand\figurename{Figure}
\fi
\ifdefined\tablename
  \renewcommand*\tablename{Table}
\else
  \newcommand\tablename{Table}
\fi
}
\@ifpackageloaded{float}{}{\usepackage{float}}
\floatstyle{ruled}
\@ifundefined{c@chapter}{\newfloat{codelisting}{h}{lop}}{\newfloat{codelisting}{h}{lop}[chapter]}
\floatname{codelisting}{Listing}
\newcommand*\listoflistings{\listof{codelisting}{List of Listings}}
\makeatother
\makeatletter
\makeatother
\makeatletter
\@ifpackageloaded{caption}{}{\usepackage{caption}}
\@ifpackageloaded{subcaption}{}{\usepackage{subcaption}}
\makeatother
\usepackage{bookmark}
\IfFileExists{xurl.sty}{\usepackage{xurl}}{} % add URL line breaks if available
\urlstyle{same}
\hypersetup{
  pdftitle={Review of ``What do Philosophers Do''},
  pdfauthor={Brian Weatherson},
  hidelinks,
  pdfcreator={LaTeX via pandoc}}


\title{Review of ``What do Philosophers Do''}
\author{Brian Weatherson}
\date{2019}
\begin{document}
\maketitle
\begin{abstract}
Review of Penelope Maddy, ``What Do Philosophers Do?: Skepticism and the
Practice of Philosophy''. Oxford: Oxford University Press, 2017.
\end{abstract}


\setstretch{1.1}
Penelope Maddy sets out to do a number of distinct things in this book.
One is to present the responses that mid-century ordinary language
philosophers, especially Austin, Moore and Wittgenstein, make to
sceptical challenges. A second is to contribute to epistemological
debates by endorsing a version of this kind of response. And a third is
to use this discussion to say something general about, as the title of
the book has it, what philosophers do. The first of these is done very
well, and shows up ways in which some common conceptions of ordinary
language philosophy misrepresents what these philosophers were up to,
and understates their achievements. But the second and third are less
successful. There is so little engagement with philosophical work of the
last 30 years, that the picture of philosophy we're left with is, at
best, a representation of what philosophers did.

The discussions of mid-century philosophy, and philosophers, is very
well done, and well worth reading for both epistemologists, and those
interested in the history of analytic philosophy. Maddy does an
excellent job of putting the views in their proper historical context.
So we see the connections between Austin and Reed, but also the
commonalities between Austin's work and Quine's contemporaneous work.
And, for that matter, we get a good discussion of the relationship
between Austin's methodological remarks, and his first-order
epistemological work on knowledge and on perception. Maddy adds nuance
to her portrayal of Moore in previous works, now seeing him as someone
who rejected the demand to reply to the sceptic, rather than offering a
particular kind of reply. And, drawing on recent work on the
construction of \emph{On Certainty}, we get a careful study of how the
best parts of that book fit into the larger project of Wittgenstein's
later work. I learned a lot from these parts of the book, and I think it
is a valuable addition to the literature on mid-century philosophy.

Maddy doesn't want to just describe these views though, she also
endorses a theory that draws heavily on their insights. Here's the broad
picture, some of which is familiar from her earlier works. The story has
two starring characters: the Plain Man and the Plain Inquirer. The Plain
Man is the voice of untutored common sense. He thinks that scepticism is
obviously false for some reason, because he can just see that there is a
table. The Plain Inquirer, called the Second Philosopher in her earlier
work, takes the Plain Man's views as a starting point, but tests as many
of the presuppositions of these views as possible. But, crucially, she
always takes for granted other things that she believes while doing
these tests. The idea is to put everything to the test, and see how it
fits into a grand scientific picture, but to do so piecemeal. What the
Plain Inquirer rejects is the Cartesian demand to put all of one's views
to the test all at once. She also, I think, rejects the demand to defend
assumptions that she can see from the start there will be no way to
defend, such as the assumption that she's not in an extraordinary dream.

This all sounds reasonable, though at this point I'd like to know more
precisely how this differs from a Quinean holism that insists we start
our inquiries in the middle of things. Quine's nominalism and meaning
scepticism do get briefly discussed in the book, but not his
epistemology. But more pressingly, I'm not entirely sure what the
response to scepticism is. At times it seems that Maddy, like Nozick and
Dretske, rejects closure, and so thinks we can know a lot about the
world even if we can't know that sceptical scenarios don't obtain. (She
talks this way at the start of Lecture III, for example.) But at other
times, such as in the closing line of the book, she talks as if she is
neutral on Closure, and perhaps thinks that we might know that sceptical
scenarios don't obtain, we just can't defend this claim to the sceptic's
satisfaction. These are both worthwhile lines of response to scepticism,
but they are very different lines, and it would be useful to know which
it is.

If it's the first line, the one that rejects closure, I don't see how
the Plain Inquirer can be so nonchalant about this rejection. Inquiry
won't get very far if we can't use logic and mathematics to carry it
out. Closure failures threaten to undermine every method that we use,
unless we get some kind of method for sealing the failures off.

But the bigger concern I have with the book is with its representation
of how epistemologists, and philosophers more generally, think. When
Maddy discusses what epistemologists do, the representation feels 30-odd
years out of date. The only `contemporary' work discussed at length is
Barry Stroud's 1984 book \emph{The Significance of Philosophical
Scepticism}. The scare quotes here are because Stroud's book is now as
old as, say, \emph{On Certainty} was when Stroud's book came out. And I
don't think anyone in 1984 viewed \emph{On Certainty} as a piece of
contemporary epistemology, or thought that one could write a book about
the state of contemporary epistemology focussing on it.

Maddy says that ``much of the effort of epistemologists'' is directed to
analysing the concept of knowledge (60), and this discussion is
``dominated by commentary on every-more-complex problem cases'' (205).
This may have been true in the late twentieth century, but it is a
really misleading representation of contemporary epistemology. There is
no discussion here of Williamson's arguments that no analysis is
possible, nor of the objections to arguments by cases in recent
meta-philosophy. She thinks epistemologists don't take Moore seriously
enough, but there is no discussion of the resurgence of interest in
Moorean views prompted by Jim Pryor's early work. She lauds Austin's
careful study of how the verb `know' is actually used, but pays no
attention to the mountains of work on contextualist theories of
knowledge in the last 20 years. In recent years, the bulk of work in
epistemology has concerned either social epistemology, or formal
epistemology, or the relationship between epistemology and ethics, but
Maddy leaves readers with a picture of philosophy where epistemologists
care about none of those things.

All of this feels like a missed opportunity. Maddy ends the body of the
book (excepting two short appendices) with a discussion of what
philosophy does well. And the short answer is that when things go well
philosophy does theory. It lays the groundwork for future sciences but
it also, as Maddy stresses, deals with those theoretical questions that
existing sciences raise but cannot on their own answer. Maddy argues
that this kind of work, work that is continuous with sciences, work that
evaluates our practical and theoretical methods while still using other
methods of our own, is what the best philosophers have traditionally
done, and what the best contemporary philosophy does.

But she leaves the reader with the impression that most epistemologists
are not engaged in this kind of valuable project. She says that they are
mostly doing conceptual analysis, and that conceptual analysis is
(typically) not helpful to valuable philosophical projects. But, as I
mentioned above, this just seems like a misrepresentation of the last
generation's work in epistemology. A more accurate picture of the way
epistemologists do work that interacts with theoretical work in
psychology, in jurisprudence, in economics, in political science, and in
linguistics, to name but a few, would not have left the reader with the
sense of such a sharp gap between what epistemology is, and what it
should be.



\noindent Published in\emph{
Mind}, 2019, pp. 269-271.


\end{document}
