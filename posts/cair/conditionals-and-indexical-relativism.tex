% Options for packages loaded elsewhere
\PassOptionsToPackage{unicode}{hyperref}
\PassOptionsToPackage{hyphens}{url}
%
\documentclass[
  10pt,
  letterpaper,
  DIV=11,
  numbers=noendperiod,
  twoside]{scrartcl}

\usepackage{amsmath,amssymb}
\usepackage{setspace}
\usepackage{iftex}
\ifPDFTeX
  \usepackage[T1]{fontenc}
  \usepackage[utf8]{inputenc}
  \usepackage{textcomp} % provide euro and other symbols
\else % if luatex or xetex
  \usepackage{unicode-math}
  \defaultfontfeatures{Scale=MatchLowercase}
  \defaultfontfeatures[\rmfamily]{Ligatures=TeX,Scale=1}
\fi
\usepackage{lmodern}
\ifPDFTeX\else  
    % xetex/luatex font selection
  \setmainfont[ItalicFont=EB Garamond Italic,BoldFont=EB Garamond
Bold]{EB Garamond Math}
  \setsansfont[]{Europa-Bold}
  \setmathfont[]{Garamond-Math}
\fi
% Use upquote if available, for straight quotes in verbatim environments
\IfFileExists{upquote.sty}{\usepackage{upquote}}{}
\IfFileExists{microtype.sty}{% use microtype if available
  \usepackage[]{microtype}
  \UseMicrotypeSet[protrusion]{basicmath} % disable protrusion for tt fonts
}{}
\usepackage{xcolor}
\usepackage[left=1in, right=1in, top=0.8in, bottom=0.8in,
paperheight=9.5in, paperwidth=6.5in, includemp=TRUE, marginparwidth=0in,
marginparsep=0in]{geometry}
\setlength{\emergencystretch}{3em} % prevent overfull lines
\setcounter{secnumdepth}{3}
% Make \paragraph and \subparagraph free-standing
\ifx\paragraph\undefined\else
  \let\oldparagraph\paragraph
  \renewcommand{\paragraph}[1]{\oldparagraph{#1}\mbox{}}
\fi
\ifx\subparagraph\undefined\else
  \let\oldsubparagraph\subparagraph
  \renewcommand{\subparagraph}[1]{\oldsubparagraph{#1}\mbox{}}
\fi


\providecommand{\tightlist}{%
  \setlength{\itemsep}{0pt}\setlength{\parskip}{0pt}}\usepackage{longtable,booktabs,array}
\usepackage{calc} % for calculating minipage widths
% Correct order of tables after \paragraph or \subparagraph
\usepackage{etoolbox}
\makeatletter
\patchcmd\longtable{\par}{\if@noskipsec\mbox{}\fi\par}{}{}
\makeatother
% Allow footnotes in longtable head/foot
\IfFileExists{footnotehyper.sty}{\usepackage{footnotehyper}}{\usepackage{footnote}}
\makesavenoteenv{longtable}
\usepackage{graphicx}
\makeatletter
\def\maxwidth{\ifdim\Gin@nat@width>\linewidth\linewidth\else\Gin@nat@width\fi}
\def\maxheight{\ifdim\Gin@nat@height>\textheight\textheight\else\Gin@nat@height\fi}
\makeatother
% Scale images if necessary, so that they will not overflow the page
% margins by default, and it is still possible to overwrite the defaults
% using explicit options in \includegraphics[width, height, ...]{}
\setkeys{Gin}{width=\maxwidth,height=\maxheight,keepaspectratio}
% Set default figure placement to htbp
\makeatletter
\def\fps@figure{htbp}
\makeatother
% definitions for citeproc citations
\NewDocumentCommand\citeproctext{}{}
\NewDocumentCommand\citeproc{mm}{%
  \begingroup\def\citeproctext{#2}\cite{#1}\endgroup}
\makeatletter
 % allow citations to break across lines
 \let\@cite@ofmt\@firstofone
 % avoid brackets around text for \cite:
 \def\@biblabel#1{}
 \def\@cite#1#2{{#1\if@tempswa , #2\fi}}
\makeatother
\newlength{\cslhangindent}
\setlength{\cslhangindent}{1.5em}
\newlength{\csllabelwidth}
\setlength{\csllabelwidth}{3em}
\newenvironment{CSLReferences}[2] % #1 hanging-indent, #2 entry-spacing
 {\begin{list}{}{%
  \setlength{\itemindent}{0pt}
  \setlength{\leftmargin}{0pt}
  \setlength{\parsep}{0pt}
  % turn on hanging indent if param 1 is 1
  \ifodd #1
   \setlength{\leftmargin}{\cslhangindent}
   \setlength{\itemindent}{-1\cslhangindent}
  \fi
  % set entry spacing
  \setlength{\itemsep}{#2\baselineskip}}}
 {\end{list}}
\usepackage{calc}
\newcommand{\CSLBlock}[1]{\hfill\break\parbox[t]{\linewidth}{\strut\ignorespaces#1\strut}}
\newcommand{\CSLLeftMargin}[1]{\parbox[t]{\csllabelwidth}{\strut#1\strut}}
\newcommand{\CSLRightInline}[1]{\parbox[t]{\linewidth - \csllabelwidth}{\strut#1\strut}}
\newcommand{\CSLIndent}[1]{\hspace{\cslhangindent}#1}

\setlength\heavyrulewidth{0ex}
\setlength\lightrulewidth{0ex}
\usepackage[automark]{scrlayer-scrpage}
\clearpairofpagestyles
\cehead{
  Brian Weatherson
  }
\cohead{
  Conditionals and Indexical Relativism
  }
\ohead{\bfseries \pagemark}
\cfoot{}
\makeatletter
\newcommand*\NoIndentAfterEnv[1]{%
  \AfterEndEnvironment{#1}{\par\@afterindentfalse\@afterheading}}
\makeatother
\NoIndentAfterEnv{itemize}
\NoIndentAfterEnv{enumerate}
\NoIndentAfterEnv{description}
\NoIndentAfterEnv{quote}
\NoIndentAfterEnv{equation}
\NoIndentAfterEnv{longtable}
\NoIndentAfterEnv{abstract}
\renewenvironment{abstract}
 {\vspace{-1.25cm}
 \quotation\small\noindent\rule{\linewidth}{.5pt}\par\smallskip
 \noindent }
 {\par\noindent\rule{\linewidth}{.5pt}\endquotation}
\KOMAoption{captions}{tableheading}
\makeatletter
\@ifpackageloaded{caption}{}{\usepackage{caption}}
\AtBeginDocument{%
\ifdefined\contentsname
  \renewcommand*\contentsname{Table of contents}
\else
  \newcommand\contentsname{Table of contents}
\fi
\ifdefined\listfigurename
  \renewcommand*\listfigurename{List of Figures}
\else
  \newcommand\listfigurename{List of Figures}
\fi
\ifdefined\listtablename
  \renewcommand*\listtablename{List of Tables}
\else
  \newcommand\listtablename{List of Tables}
\fi
\ifdefined\figurename
  \renewcommand*\figurename{Figure}
\else
  \newcommand\figurename{Figure}
\fi
\ifdefined\tablename
  \renewcommand*\tablename{Table}
\else
  \newcommand\tablename{Table}
\fi
}
\@ifpackageloaded{float}{}{\usepackage{float}}
\floatstyle{ruled}
\@ifundefined{c@chapter}{\newfloat{codelisting}{h}{lop}}{\newfloat{codelisting}{h}{lop}[chapter]}
\floatname{codelisting}{Listing}
\newcommand*\listoflistings{\listof{codelisting}{List of Listings}}
\makeatother
\makeatletter
\makeatother
\makeatletter
\@ifpackageloaded{caption}{}{\usepackage{caption}}
\@ifpackageloaded{subcaption}{}{\usepackage{subcaption}}
\makeatother
\ifLuaTeX
  \usepackage{selnolig}  % disable illegal ligatures
\fi
\IfFileExists{bookmark.sty}{\usepackage{bookmark}}{\usepackage{hyperref}}
\IfFileExists{xurl.sty}{\usepackage{xurl}}{} % add URL line breaks if available
\urlstyle{same} % disable monospaced font for URLs
\hypersetup{
  pdftitle={Conditionals and Indexical Relativism},
  pdfauthor={Brian Weatherson},
  hidelinks,
  pdfcreator={LaTeX via pandoc}}

\title{Conditionals and Indexical Relativism}
\author{Brian Weatherson}
\date{2009}

\begin{document}
\maketitle
\begin{abstract}
I set out and defend a view on indicative conditionals that I call
``indexical relativism''. The core of the view is that which proposition
is (semantically) expressed by an utterance of a conditional is a
function of (among other things) the speaker's context and the
assessor's context. This implies a kind of relativism, namely that a
single utterance may be correctly assessed as true by one assessor and
false by another.
\end{abstract}

\setstretch{1.1}
This paper is about a class of conditionals that Anthony Gillies
(\citeproc{ref-Gillies2004-GILECA}{2004}) has dubbed `open indicatives',
that is, indicative conditionals ``whose antecedents are consistent with
our picture of the world''. I believe that what I say here can
eventually be extended to all indicative conditionals, but indicatives
that aren't open raise special problems, so I'll set them aside for
today. In Weatherson (\citeproc{ref-Weatherson2001-WEAIAS}{2001}) I
argued for an epistemic treatment of open indicatives, and implemented
this in a contextualist semantics. In this paper I want to give another
argument for the epistemic approach, but retract the contextualism.
Instead I'll put forward a relativist semantics for open indicatives.
The kind of relativism I'll defend is what I'll dub `indexical
relativism'.

I've changed my mind since Weatherson
(\citeproc{ref-Weatherson2001-WEAIAS}{2001}) largely because of
developments since I wrote that paper. There have been six primary
influences on this paper, listed here in order that they become relevant
to the paper.

\begin{enumerate}
\def\labelenumi{\arabic{enumi}.}
\tightlist
\item
  The arguments that Jason Stanley (along with co-authors) puts forward
  in his (\citeproc{ref-Stanley2007-STALIC}{2007}) for the view that all
  effects of context on semantic content are syntactically triggered,
  and in particular involve context setting the value for a tacit or
  overt variable.
\item
  John MacFarlane's defences, starting with MacFarlane
  (\citeproc{ref-MacFarlane2003-MACFCA-2}{2003}) of semantic relativism.
\item
  John MacFarlane's recent work, including MacFarlane
  (\citeproc{ref-MacFarlane2009-MACNC}{2009}) at distinguishing the view
  that propositional truth can vary between different contexts in the
  same world, and the view that the truth of an utterance can be
  assessor-sensitive.
\item
  Tamina Stephenson's (\citeproc{ref-Stephenson2007}{2007}) arguments in
  favour of a variable PRO\textsubscript{J} whose value is set by
  assessors.
\item
  Philippe Schlenker's (\citeproc{ref-Schlenker2003}{2003}) idea,
  modelled on some examples from Barbara Partee
  (\citeproc{ref-Partee1989}{1989}) that plural variables can be
  `partially bound'.
\item
  Anthony Gillies's (\citeproc{ref-Gillies2009-GILOTF}{2009})
  suggestions for how to explain the acceptability of the
  `import-export' schema in an epistemic theory of indicative
  conditionals.
\end{enumerate}

I think it is noteworthy, in light of the claims one sometimes hears
about philosophy not making progress, that most of the building blocks
of the theory defended here weren't even clearly conceptualised at the
time I wrote the earlier paper.

This paper is in seven sections. The paper starts with an argument in
favour of an epistemic treatment of open indicatives, namely that only
the epistemic theory can explain our judgments about inferences
involving open indicatives. The argument isn't completely original,
indeed much of what I'll say here can be found in Stalnaker
(\citeproc{ref-Stalnaker1975-STAIC}{1975}), but I don't think the scope
of this argument has been sufficiently appreciated. There are a large
family of epistemic theories, and in section two I'll set out some of
the choice points that an epistemic theorist faces. I'll also introduce
a fairly simple epistemic theory, not the one I favour actually, that
I'll focus on in what follows. My preferred theory has several more
bells and whistles, but I don't think those are relevant to the issues
about relativism and indexicalism that I'll focus on here, and including
them would just complicate the discussion needlessly.

In section three I look at four ways a theory could say that the truth
of an utterance type is sensitive to context.\footnote{It is actually a
  little tricky to say just what the relevant types are. I mean to use
  the term so that two people make an utterance of the same type if
  their utterances use the same words, with the same syntax, and the
  same elided material, and the same meaning. So two utterances of ``I
  am Australian'' could be of the same type, even if offered by
  different people. In some cases, e.g.~``Steel is strong enough'', it
  might be controversial whether two utterances that intuitively have
  different contents are either (a) of the same type, or (b) use terms
  with different meanings, or (c) have different elided material. I'll
  try to stay neutral on this point.} The four ways are generate by the
ways the theory answers two questions. First, is the truth of the
utterance type sensitive to facts about the context of utterance, as
contextualists say, or to facts about the context of evaluation, as
relativists say? Second, does the utterance type express different
propositions in different contexts, as indexicalists say, or does it
express a proposition that takes different truth values in different
contexts, as non-indexicalists say? In using the term `indexicalist',
I'm implicitly assuming the theory, most associated with Jason Stanley
(\citeproc{ref-Stanley2007-STALIC}{2007}) , that the way an utterance
type can express different propositions in different contexts is that it
has a variable in its semantic structure, and different contexts assign
different values to this variable.

Three of the four options generated by the two questions, indexical
contextualism, non-indexical relativism, and non-indexical
contextualism, have received some coverage in the literature. The fourth
option, indexical relativism, has not been as widely discussed. In
section four I say a little about its motivations, including its
connection to recent work by Tamina Stephenson. The variable in the
semantics for open indicatives is a plural variable; roughly it takes as
values all those propositions that are known by the salient people in
the context. In section five I note some odd properties about bound
plural pronouns that will become relevant to the story that follows. The
short version is that some plural pronouns can have their values set
partially by antecedent linguistic material, and partially by context.
So if a pronoun \emph{v} refers to the Xs, it might be that \emph{a} is
one of the Xs because \emph{v} is bound to a term that denotes \emph{a},
and \emph{b} is one of the Xs because \emph{b} is contextually salient
in the right kind of way to be one of the things that \emph{v} denotes.

In section six I put forward the arguments against contextualism, and in
particular against indexical contextualism. I start with some arguments
from `faultless disagreement', and go through four reasons why these
might not be too strong. I then discuss some arguments from facts about
when two people have said the same thing, or have said different things.
These arguments tell against simple forms of indexical contextualism,
but not against more sophisticated versions. But I conclude with a
somewhat simpler argument, an argument from what I'll call easy
agreement, that does seem to undermine indexical contextualism.

Finally in section seven I'll argue against non-indexical theories of
open indicatives. The primary argument will be that the indexicalist has
a good explanation of what's going on in McGee's `counterexamples to
modus ponens', an explanation borrowed from some recent work by Anthony
Gillies, but the non-indexicalist does not. The indexicalist's
explanation is that these arguments contain fallacies of equivocation;
on the non-indexicalist position, it is hard to see how they equivocate.
The upshot of the final two sections is that indexical relativism is
correct.

\section{Inferences Involving
Conditionals}\label{inferences-involving-conditionals}

I'm going to start by offering an argument for an epistemic treatment of
conditionals. The argument isn't particularly new, I'm basically just
offering an extension of the argument in Stalnaker
(\citeproc{ref-Stalnaker1975-STAIC}{1975}), but I don't think the force
of it has been fully appreciated. The argument starts with the
observation that any instance of any of the following inference schema
seems acceptable when the conclusion is an open indicative.

\begin{enumerate}
\def\labelenumi{\arabic{enumi}.}
\tightlist
\item
  \emph{Not A or B}; so, \emph{If A then B}
\item
  \emph{All Fs are Gs}; so, \emph{If Fa then Ga}
\item
  \emph{f}(\emph{x}) = \emph{f}(\emph{y}); so \emph{If f}(\emph{a}) =
  \emph{c} \emph{then f}(\emph{b}) = \emph{c}
\end{enumerate}

Here are some instances of each of these inferences.

\begin{enumerate}
\def\labelenumi{\arabic{enumi}.}
\setcounter{enumi}{3}
\tightlist
\item
  Either Jack won't come to the party or Jill will; so if Jack comes to
  the party, so will Jill.
\item
  All of Kim's students failed; so if Alex was one of Kim's students,
  then Alex failed.
\item
  Peter's mother is Paul's mother; so if Peter's mother is Mary, Paul's
  mother is Mary.
\end{enumerate}

Quite a lot has been written about (1)/(4) and I don't propose to add to
it. It is arguable that part of the explanation for its attractiveness
comes from pragmatic properties of disjunctions, and if that's right it
would complicate the story I want to tell here. Instead I'll focus on
the other two inferences.

Jonathan Ichikawa pointed out to me that (3) is not particularly
compelling in cases where it is common ground in the conversation that
\emph{f}(\emph{a}) is not \emph{c}. For instance, it is at least odd to
say ``Peter's mother is Jane, and she's also Paul's mother. So Peter's
mother is Paul's mother; so if Peter's mother is Mary, Paul's mother is
Mary.'' I think intuitions differ on these cases. I don't find the
inferences as bad as many informants do, and generally speaking
intuitions about knowledge-contravening indicatives are rather fuzzy. So
I'm just going to focus on the case where the conclusions are open.

Ernest Adams (\citeproc{ref-Adams1998}{1998}) has objected to the claim
that instances of (2) always seem like acceptable inferences using the
following example.

\begin{enumerate}
\def\labelenumi{\arabic{enumi}.}
\setcounter{enumi}{6}
\tightlist
\item
  Everyone who was at the party is a student. So if the Chancellor was
  at the party, the Chancellor is a student. (Adams 1998: 289)
\end{enumerate}

I actually think this sounds like a perfectly fine inference. If I say
that everyone at the party was a student, and someone takes me to
thereby to be committed to the claim that if the Chancellor was at the
party, she too is a student, then I won't complain. But perhaps my
intuitions here are odd. Here is an intuition that I feel more
comfortable with. If the conclusion of (7) is an open indicative, that
is if it isn't ruled out that the Chancellor is a student, then the
inference in (7) sounds perfectly fine to me.

Adams has to object to (2) because it provides counterexamples to a
thesis he defends at some length. This thesis is that an inference from
a single premise to a conditional \emph{If p then q} is a good inference
iff necessarily the probability of \emph{q} given \emph{p} is not lower
than the probability of the premise. Schema (3) is also a problem here
as well. In each case it isn't too hard to find instances where the
probability of the premise is arbitrarily high, but the probability of
the conclusion's consequent given its antecedent is arbitrarily low. For
instance, let the salient probabilities be as below:

\begin{quote}
Pr(\emph{f}(\emph{a}) = \emph{f}(\emph{b}) = \emph{d}) = 1-\emph{x}\\
Pr(\emph{f}(\emph{a}) = \emph{c} ∧ \emph{f}(\emph{b}) = \emph{e}) =
\emph{x}
\end{quote}

If we let \emph{x} be arbitrarily small, then the probability of the
premise \emph{f}(\emph{a}) = \emph{f}(\emph{b}) will be arbitrarily
high. But the conditional probability of \emph{f}(\emph{b}) = \emph{c}
given \emph{f}(\emph{a}) = \emph{c} will be arbitrarily low. So the
probability preservation property Adams has highlighted isn't what is
always preserved in good inferences.

Each of the inferences in (1) to (3) is, in some sense, a good
inference. It is easy to prove from that fact, and the assumption that
good inferences are valid implications, that the conditional \emph{If p
then q} is true if \emph{p} is false or \emph{q} is true. If we assume
Modus Ponens is valid (as I will throughout) then we can strengthen this
conditional to a biconditional. It is obviously easy to prove this using
the goodness of (1). Here is a proof that uses just (3), and some weak
assumptions about truth. Line 1 is the only assumption, and every line
seems to follow from the one before it.

\begin{enumerate}
\def\labelenumi{\arabic{enumi}.}
\tightlist
\item
  The truth value of \emph{p} is the truth value of \emph{p} ∧ \emph{q}
\item
  So if the truth value of \emph{p} is true, then the truth value of
  \emph{p} ∧ \emph{q} is true
\item
  So if \emph{p} is true, then \emph{p} ∧ \emph{q} is true
\item
  So if \emph{p}, then \emph{p} ∧ \emph{q}
\item
  So if \emph{p}, then \emph{q}
\end{enumerate}

I've assumed here that we can treat \emph{The truth value of p is true},
\emph{p is true}, and \emph{p} as equivalent, but this seems
uncontroversial. And on pretty much any conditional logic there is, the
move from (4) to (5) will be valid. Assuming bivalence, (1) is
equivalent to the disjunction \emph{p} is false or \emph{q} is true. So
we can infer from that disjunction to \emph{If p, then q} without using
the schema (1).

This might all suggest that open indicatives should be interpreted as
material implications. But there is some data that tells against that.
This suggestion from Richard Bradley (\citeproc{ref-Bradley2000}{2000})
seems correct.

\begin{quote}
{[}O{]}ne cannot be certain that \emph{B} is not the case if one thinks
that it is possible that if \emph{A} then \emph{B}, unless one rules out
the possibility of \emph{A} as well. You cannot, for instance, hold that
we might go to the beach, but that we certainly won't go swimming and at
the same time consider it possible that if we go to the beach we will go
swimming! To do so would reveal a misunderstanding of the indicative
conditional (or just plain inconsistency).
(\citeproc{ref-Bradley2000}{Bradley 2000, 220})
\end{quote}

More generally, someone who regards \emph{A} as an epistemic
possibility, but knows that \emph{B} is false, should regard If
\emph{A}, \emph{B} also as something they know to be false. Bradley puts
this in probabilistic terms as follows.

\begin{quote}
\textbf{Preservation Condition}\\
If Pr(\emph{A}) \textgreater{} 0 but Pr(\emph{B}) = 0, then Pr(\emph{A}
→ \emph{B}) = 0
\end{quote}

This isn't obviously the best formulation of his principle. In the
example, what matters is not that \emph{A} has non-zero probability, but
that it is something that might be true. (These are different. The
probability that the average temperature in Ithaca on January 1 next
year will be \emph{exactly} 32 degree Fahrenheit is 0, but that might be
the exact temperature.) The structure of the inference looks to be what
is given in (8), where K\emph{p} means the relevant agents knows that
\emph{p}.

\begin{enumerate}
\def\labelenumi{\arabic{enumi}.}
\setcounter{enumi}{7}
\tightlist
\item
  ¬K¬\emph{A}\\
  K¬\emph{B}\\
  So, K¬(If \emph{A, B})
\end{enumerate}

But this is not valid if the conditional is a material implication. So
now it is \emph{impossible} to accept all of the intuitively plausible
principles about inference involving conditionals are truth-preserving.
There must be some other explanation of the reasonableness of all these
inferences other than their being valid implications.

The best explanation I know of this `reasonableness' is the one endorsed
by Daniel Nolan (\citeproc{ref-Nolan2003}{2003}) as an explanation of
inferences like (1). Nolan says that given an epistemic theory of the
indicative, we can say that each of the inferences has the following
property. Any speaker who knows the premise is in a position to truly
assert the conclusion. Call an inference like this, where knowledge of
the premise implies truth of the conclusion, epistemically acceptable.
If we are confusing valid implications with epistemically acceptable
inferences, this could explain why all of (1) through (3) seem
reasonable. More impressively, this hypothesis of Nolan's explains why
(8) seems reasonable, given an epistemic theory of indicatives. If we
know that \emph{A} is true in some epistemic possibilities, but \emph{B}
is false in all of them, then all the epistemically salient alternatives
where \emph{A} is true will be ones where \emph{B} is false. So (8) will
turn out to be a good inference, by Nolan's criteria. So (8), like (1)
through (3), is epistemically acceptable. So given Nolan's epistemic
account of reasonable inference, and an epistemic theory of indicative
conditionals, we can explain the reasonableness of all five problematic
inferences. In the absence of any other good explanation of this
reasonableness, this seems to me to be a good reason to accept both
Nolan's account and an epistemic theory of indicative conditionals.

\section{The Simple Epistemic Theory of
Conditionals}\label{the-simple-epistemic-theory-of-conditionals}

For concreteness, I'll work in this paper with a very simple theory of
conditionals. I assume that in general a conditional \emph{If p},
\emph{q} has the logical form C(\emph{p}, \emph{q}, X), where C is the
conditional relation, and X is a plural variable that denotes some
propositions taken as fixed in the context. The simple epistemic theory
makes two additions to this basic assumption.

First, there is some epistemic relation R such a proposition \emph{s} is
among the X iff some salient individual \emph{i} stands in relation R to
\emph{s}. We'll use \emph{R}(\emph{i}) to represent those propositions.
It will become important later that X is genuinely a plural variable, so
\emph{R}(\emph{i}) is not a set of propositions, or a fusion of
propositions (whatever that would be). Rather, I just mean to be
plurally referring to the propositions that stand in relation R to
\emph{i}. (Note that I'm not saying anything here about how \emph{i} is
determined; my preferred theory is that it is the \emph{evaluator} of
any conditional utterance, but nothing in the simple epistemic theory
turns on this.)

A very conservative version of the theory says that R is the knowledge
relation. One can liberalise the theory in two respects. First, we can
say that R is the `position to know' relation. Second, we can say that
\emph{sRi} iff someone salient to \emph{i} knows that S. A maximally
liberal version of the theory says that \emph{sRi} iff someone salient
to \emph{i} is in a position to know that S. I'm not going to argue for
this here, but I think this maximally liberal option is the way to
proceed, so that's what I'm going to adopt for the sake of exposition.
Nothing turns on this adoption in what follows. Indeed the arguments
against contextualism and for relativism are stronger the more
constrained R is, so this is tilting the playing field away from my
preferred outcome.

Second, the simple theory I have in mind says that C is basically just
the a priori entailment relation. So C(\emph{p}, \emph{q}, X) is true
iff \emph{p} plus X a priori entail \emph{q}. If you want to say that
entailment is a relation between a set and a proposition, the claim is
that the union of \{\emph{p}\} and \{\emph{s}:~\emph{s} is among the X\}
a priori entail that \emph{q}.

There are several ways in which one might want to complicate the simple
epistemic theory. My preferred theory involves some of these
complications. Here are some complications that have been proposed in
various ways.

First, we might change C to stipulate that whenever \emph{p} and
\emph{q} are true, C(\emph{p},~\emph{q},~X) is true. This is equivalent
to endorsing strong centring in the sense of Lewis
(\citeproc{ref-Lewis1973a}{1973}). Assuming every proposition in X is
true, as I've done above, means that we've already guaranteed that
C(\emph{p},~\emph{q},~X) is false when \emph{p} is true and \emph{q} is
false.

Second, we might deny bivalence in the following way. Say that
C(\emph{p},~\emph{q},~X) is true iff \emph{p} and X a priori entail that
\emph{q}, false if it is not true and also \emph{p} and X entail
¬\emph{q}, and indeterminate in truth value otherwise. Going down this
path allows one to endorse conditional excluded middle, as supported by
Stalnaker (\citeproc{ref-Stalnaker1981}{1981}). Denying bivalence does
not compel acceptance of conditional excluded middle, but it becomes an
interesting possibility once you go down this path.

Third, we might say that R is a disjunctive relation, so some
propositions are among the \emph{R}(\emph{i}) because they stand in an
epistemic relation to \emph{i}, and others are in because they are in
some sense `fixed facts' of \emph{i}'s world. Nolan
(\citeproc{ref-Nolan2003}{2003}) uses a quite different formalism, but
if we wanted to translate his theory into this formalism, that's what
we'd do.

Fourth, we could make C a more complicated relation. In particular, we
could make it in a sense non-monotonic, e.g.~by saying that
C(\emph{p},~\emph{q},~X) holds iff the epistemic probability of \emph{q}
given \emph{p} and X is sufficiently high. If C is non-monotonic in this
sense, then we can have a conditional logic that looks like some of the
conditional logics in Lewis (\citeproc{ref-Lewis1973a}{1973}).

For what it's worth, I favour the first and (a version of) the second of
these complications, but not the third and fourth. Defending those
preferences would take us too far afield however. What I mostly want to
show in this paper is that whatever form of epistemic theory we adopt,
we should adopt what I'll call an indexical relativist version of that
theory. So I'll just presuppose the simple epistemic theory throughout,
because the general form of the argument should be easily adoptable
whichever complications one adds on. The task of the next section is to
introduce indexical relativism.

\section{Four Kinds of Sensitivity}\label{four-kinds-of-sensitivity}

Let's say that one is tempted towards a kind of moral relativism. So
when old Horace, way back when, ``Premarital sex is morally worse than
driving drunk'' he said something true in some sense, and when modern
Kayla now says ``Driving drunk is morally worse than premarital sex'',
she also says something true in a sense. How might we formalise these
intuitions? (Not, I might add, intuitions that I share.) There are a few
simple options, breaking down along two distinct axes. To save space,
I'll write P for pre-marital sex, D for driving drunk, and \textless{}
for the relation is morally worse than, in what follows.\footnote{I
  don't mean to suggest that these are the \emph{only} options. I'm
  leaving off options on which there are contextual effects on semantic
  content that are not syntactically triggered, for example. My reason
  for doing that is that there are, I think, good reasons for thinking
  that the context-sensitivity of indicative conditionals \emph{is}
  syntactically triggered, so I don't need to investigate non-syntactic
  triggers here.}

The first axis concerns the nature of propositions about the moral. One
option is to say that moral codes are part of the propositions that are
the content of Horace's and Kayla's utterances. For example, we might
say that when old Horace makes his utterance, its content is the
proposition \emph{P} \textless{} \emph{D in M\textsubscript{O}}, where
M\textsubscript{O} is Horace's old moral code. Conversely, when Kayla
makes her utterance, its content is the proposition \emph{D} \textless{}
\emph{P in M\textsubscript{N}}. where M\textsubscript{N} is Kayla's new
moral code. This option, as Sayre-McCord
(\citeproc{ref-SayreMcCord1991}{1991}) notes, treats `moral' as being
like `legal'. When we say ``Insulting the Thai monarch is illegal'', the
content of our utterance is the proposition \emph{Insulting the Thai
monarch is illegal in L}, where L is some salient legal code. That's why
typical utterances of that sentence in Bangkok are true, but typical
utterances of it in St Andrews are false. Call this option
\emph{indexicalism}, since it thinks there is an indexical element in
the semantic structure of what Horace and Kayla say. Because this will
become crucial later, what I'm taking to be essential to indexicalism is
simply the view that there is a moral code in the proposition expressed,
not that it is the moral code of the speaker.

A quite different option is to say that the content of Horace's
utterance is simply the proposition \emph{P} \textless{} \emph{D}. The
relativism comes in because it turns out that propositions are true or
false relative to, inter alia, moral codes. The proposition \emph{P}
\textless{} \emph{D} is true in M\textsubscript{O}, and false in
M\textsubscript{N}. The analogy here is to the way the very same
proposition can be true in one world false in another. This option,
which I'll call \emph{non-indexicalism}, says that moral codes function
much like worlds; they are things relative to which propositions are
true or false. The non-indexicalist takes Kaplan
(\citeproc{ref-Kaplan1989}{1989}) to be on the right track in saying
that propositions are true or false relative to world-time pairs, but
thinks that the indices relative to which propositions are true or false
are even more fine-grained than that.

The second axis concerns which context is relevant to the truth of the
utterance. One option is to say that it is the context of utterance. A
second option is to say that it is the context of evaluation. Following
MacFarlane (\citeproc{ref-MacFarlane2007-SMNIC}{2007};
\citeproc{ref-MacFarlane2009-MACNC}{2009}), I'll call the first option
\emph{contextualism}, and the second option \emph{relativism}. The point
that's worth focussing on here is that what choice we make here cuts
across the choice we make on the first axis. So there are four options
available. To set these out, we need to introduce a third character
(call him Deval) who \emph{assesses} Horace's and Kayla's utterances.
For concreteness, call Deval's moral code M\textsubscript{A}, and say
that it agrees with M\textsubscript{N} on the point at issue. Then here
are the four options we have.

\begin{itemize}
\tightlist
\item
  \emph{Indexical Contextualism}. The propositions that are the content
  of Horace's and Kayla's utteranes include moral codes, and which code
  that is is determined by features of their utterance. So the content
  of Horace's utterance is the proposition \emph{P} \textless{} \emph{D
  in M\textsubscript{O}}, and Kayla's the proposition \emph{D}
  \textless{} \emph{P in M\textsubscript{N}}. Deval should assess each
  of them as having uttered truths.
\item
  \emph{Non-indexical Relativism}. The propositions that are the content
  of their utterances do not include moral codes, and their utterances
  are only true or false relative to a moral code provided by an
  assessor. So the content of Horace's utterance is simply \emph{P}
  \textless{} \emph{D}, and Kayla's \emph{D} \textless{} \emph{P}. Since
  in M\textsubscript{A} \emph{D} \textless{} \emph{P} is true, Deval
  should assess Horace's utterance as false, and Kayla's as true.
\item
  \emph{Non-indexical contextualism}. The propositions that are the
  content of their utterances do not include moral codes, but the
  truth-value a moral utterance is the truth-value of its content in the
  context it is expressed in. So the content of Horace's utterance is
  simply \emph{P} \textless{} \emph{D}, and Kayla's \emph{D} \textless{}
  \emph{P}. Since Horace makes his utterance in a context where \emph{P}
  \textless{} \emph{D} is part of the prevailing moral code, his
  utterance is true. So Deval and Kayla should assess it as true, even
  though they think the proposition Horace expressed is false. This
  isn't contradictory; a person in another possible world can make a
  true utterance by expressing a proposition that is (actually) false,
  and for the non-indexical contextualist, moral codes are in a way like
  worlds. Kayla's utterance is true as well, since it is made in a
  different context.
\item
  \emph{Indexical relativism}. The propositions that are the content of
  Horace's and Kayla's utteranes include moral codes, and which code
  that is is determined by features of the context of assessment. So
  when Deval hears of these two utterances, he should interpret the
  content of Horace's utterance to be \emph{P} \textless{} \emph{D in
  M\textsubscript{A}}, and of Kayla's to be \emph{D} \textless{} \emph{P
  in M\textsubscript{A}}. Since the latter proposition is true, he
  should interpret Kayla's utterance as true, and Horace's as false.
\end{itemize}

It might make it easier to picture these positions in a small table, as
follows.

\begin{longtable}[]{@{}
  >{\centering\arraybackslash}p{(\columnwidth - 4\tabcolsep) * \real{0.3670}}
  >{\centering\arraybackslash}p{(\columnwidth - 4\tabcolsep) * \real{0.3211}}
  >{\centering\arraybackslash}p{(\columnwidth - 4\tabcolsep) * \real{0.3119}}@{}}
\toprule\noalign{}
\begin{minipage}[b]{\linewidth}\centering
\end{minipage} & \begin{minipage}[b]{\linewidth}\centering
\textbf{The speaker's moral code matters to utterance truth
(contextualism)}
\end{minipage} & \begin{minipage}[b]{\linewidth}\centering
\textbf{The assessor's moral code matters to utterance truth
(relativism)}
\end{minipage} \\
\midrule\noalign{}
\endhead
\bottomrule\noalign{}
\endlastfoot
\textbf{Propositions include moral codes (indexicalism)} & Indexical
Contextualism & Indexical Relativism \\
\textbf{Propositions are true or false relative to moral codes
(non-indexicalism)} & Non-indexical Contextualism & Non-indexical
Relativism \\
\end{longtable}

The modern discussion of non-indexical relativism, though not under that
name, traces to MacFarlane
(\citeproc{ref-MacFarlane2003-MACFCA-2}{2003}). The modern discussion of
non-indexical contextualism, under that name, traces to MacFarlane
(\citeproc{ref-MacFarlane2007-SMNIC}{2007};
\citeproc{ref-MacFarlane2009-MACNC}{2009}). Much of this paper, and all
of this section, is about setting out the distinctions that MacFarlane
makes in the latter papers between indexicalism and contextualism. But
once we do that, we see that there is a position, indexical relativism,
that hasn't had much attention. I plan to change that.

Before we get on to the content of indexical relativism, a small note on
nomenclature is in order. I've picked the names I have so (a) we'll have
a compositional naming scheme and (b) we get `non-indexical
contextualism' to denote what it denotes in MacFarlane's terminology.
This does mean using the term `indexical relativism' in a slightly
different way to how it has been used in the part. Einheuser
(\citeproc{ref-Einheuser2008}{2008}) and López de Sa
(\citeproc{ref-LopezDeSaMs}{2007a}) each use `indexical relativism' to
mean just what I've meant by `indexical contextualism'.
@\cite{Kolbel2009-KLBTEF} also uses the term `indexical relativism',
though López de Sa (\citeproc{ref-LopezDeSa2007a}{2007b}) argues that he
too just means contextualism, and I'm inclined to agree. So though the
name had been previously used, it had not been used to express a
distinctive view.

On the other hand, there had been some discussions of the position I'm
calling `indexical relativism'. In Egan, Hawthorne, and Weatherson
(\citeproc{ref-Egan2005-EGAEMI}{2005}) we call such a position `content
relativism', though Cappelen (\citeproc{ref-Cappelen2008}{2008}) uses
that term for a slightly different position. In MacFarlane
(\citeproc{ref-MacFarlane2005-MACMSO}{2005}) he discusses `assessment
indexicality', a property sentences have if they express different
propositions in different contexts. So there doesn't seem to be a
settled terminology for this corner of the table, and I propose to take
`indexical relativism' for it.

\section{Indexical Relativism}\label{indexical-relativism}

In ``Judge Dependence, Epistemic Modals, and Predicates of Personal
Taste'', Tamina Stephenson proposes a variant on Peter Lasersohn's
(\citeproc{ref-Lasersohn2005}{2005}) relativist account of predicates of
personal taste. Stephenson proposes that predicates of personal taste
always encode relations between an object and an assessor. So when we
say ``Warm beer is tasty'' we express some proposition of the form
\emph{Warm beer is tasty to X}. So far, this is not particularly new.
What is interesting is Stephenson's suggestion that some of the time
(but not always) there is a `silent nominal' PRO\textsubscript{J}, whose
value is the `judge'. So the utterance will be true as judged by
\emph{y} iff warm beer is tasty to \emph{y}. There are several
advantages to positing a tacit parameter. One that Stephenson stresses
is that in some cases, e.g.~when we are talking about the tastiness of
various brands of cat food, we can let the value of this parameter be
the cat rather than any human. But, by letting it by default take the
value PRO\textsubscript{J}, Stephenson shows that we can accommodate
most of the intuitions that motivate Lasersohn's relativism.

Now Stephenson is not an indexical relativist, as I've defined that
position. For according to the indexical relativist, propositions are
only true or false relative to worlds. And Stephenson has propositions
be world-time-judge triples. But I think we can adopt her idea to set
out a kind of indexical relativism. I'll first say how this could go in
the moral case, then apply it to conditionals.

The moral indexical relativist says that the context-neutral content of
an utterance like Kayla's is not a complete proposition. Rather, it is a
propositional frame that we might express as \emph{D}~\textless~\emph{P}
\emph{in M}(PRO\textsubscript{J}), where \emph{M}(\emph{x}) is
\emph{x}'s moral code, and PRO\textsubscript{J} is (as always) the
judge. Relative to any judge, the content of her utterance is that
\emph{D}~\textless~\emph{P} in that judge's moral code. So relative to
Horace, the content of her utterance is the false proposition
\emph{D}~\textless~\emph{P} \emph{in M\textsubscript{O}}, and relative
to Deval it is the true proposition \emph{D}~\textless~\emph{P} \emph{in
M\textsubscript{A}}. That's why it is fine for Horace to say ``I
disagree'', or ``That's false'', or ``She speaks falsely'', and fine for
Deval to say ``I agree'', or ``That's true'', or ``She speaks truly''.
Now I reject any kind of moral relativism, so this isn't my theory for
moral language, but it's a theory that could in principle work.

What I will defend is indexical relativism for indicative conditionals.
In general, the content of an indicative conditional \emph{If p},
\emph{q} is C(\emph{p}, \emph{q}, X), where the propositions in X are
the `background' propositions relative to which the conditional is
assessed, and C is the conditional relation\footnote{I'm assuming
  throughout that it is sufficient for the truth of C(\emph{p},
  \emph{q}, X) that it is a priori that \emph{p} plus X entail \emph{q}.
  I also think that's necessary, but I won't lean on this assumption.}.
An epistemic theory of indicatives says that the value of X is (by
default) \emph{R}(\emph{x}), where \emph{r} is some epistemic relation
(on a broad construal of `epistemic') and \emph{x} is a salient
individual. The indexical relativist position is that the content of an
utterance of a conditional is (by default) a propositional frame that we
might express as C(\emph{p}, \emph{q},
\emph{R}(\emph{PRO\textsubscript{J})). Relative to an assessor }a\emph{,
the content is C(}p\emph{, }q\emph{, }R\emph{(}a*)).

The argument for this position will come in sections 6 and 7. In section
6 I'll argue against indexical contextualism. The argument will be that
if indexical contextualism were true, it should be harder to
\emph{agree} with an utterer of a conditional than it actually is. Then
in section 7, I'll argue for indexicalism. The argument will be that we
need to posit the third argument place in the conditional relation to
explain what goes wrong in some arguments that are alleged to be both
instances of modus ponens and invalid. I'll argue (not particularly
originally) that these arguments involve a shift of a tacit parameter,
namely X. This suggests that X exists. Between those two arguments, we
can conclude that indexical relativism is true. Before that, I want to
look at some arguments against indexical relativism.

In Egan, Hawthorne, and Weatherson
(\citeproc{ref-Egan2005-EGAEMI}{2005}) we mention two arguments against
this position. One is that indexical relativism is incompatible with a
Stalnakerian account of the role of assertion. Assertions, we said, are
proposals to add something, namely their content, to the context set.
But if the content of an assertion is different in different contexts,
then it is impossible to add \emph{it} to the context set. And that, we
thought, was a problem. I now think there's a relatively simple way
around this.\footnote{The move I'm about to make bears at least a family
  resemblance to some moves López de Sa
  (\citeproc{ref-LopezDeSa2008b}{2008}) makes in defending what he calls
  `indexical relativism', though he means something different by that
  phrase.} If you want to add a proposition to the context set, then
there has to be a context. And relative to any context, a conditional
does have a content. So given any context, the content of the
conditional (relative to that context) can be added to the context set.
And that's all the Stalnakerian account requires.

Perhaps a stronger version of this objection is that even if you can
figure out, given the rules, what move a speaker is making according to
this theory, this isn't a move that sensible speakers should want to
make. So imagine that A says that if \emph{p}, then \emph{q}, and says
this because they have just discovered something no one else knows,
namely ¬(\emph{p} ∧ ¬\emph{q}). Now B hears this, not because A tells
her, but because of a fortuitous echo. B takes A to be expressing the
proposition C(\emph{p},~\emph{q},~\emph{R}(\emph{B})), and proposing
that it be added to the context set. But that's a terrible proposal, we
might object, because A has no reason to know that
C(\emph{p},~\emph{q},~\emph{R}(\emph{B})), since she knows nothing about
B's knowledge. Since there is nothing wrong with A's utterance, and the
theory interprets her as making an indefensible proposal to add
something to the context set, the theory is wrong.

This objection is potentially a powerful one, and any version of
indexical relativism must say something about such an objection. What I
say is that the objection misconstrues \emph{R}(\emph{B}). If \emph{B}
is considering an utterance by \emph{A}, even if \emph{B} does not know
that \emph{A} is the author of that utterance, then any proposition that
\emph{A} knows is among the \emph{R}(\emph{B}). I think this holds quite
generally. If \emph{A} knowably asserts that \emph{p}, and \emph{B}
considers it and says ``That might not be true'', what B says is false,
even if \emph{B} does not know whether \emph{p} is true. The reason is
that \emph{B} is taking \emph{A}'s knowledge to be, for the time being,
relevant to the context of her utterance. So in short, the knowledge on
which \emph{A} relies for her utterance is carried in to every context
in which that very utterance is assessed. That's why it is acceptable
for \emph{A} to make such a sweeping proposal, namely that for every
\emph{x} who evaluates her utterance,
C(\emph{p},~\emph{q},~\emph{R}(\emph{x})) should be added to the common
ground of \emph{x}'s context. This response relies heavily on specific
features of the interaction between conditionals and context, and I
don't think it generalises very far. It may be that this style of
objection does defeat some prima facie plausible versions of indexical
relativism, though it does not touch indexical relativism about open
indicatives.

A related objection, to a quite different proposal, is made in King and
Stanley (\citeproc{ref-KingStanley2005}{2005}). They oppose the theory
that the semantic content of an utterance is something like a character.
They say the content has to be a proposition, and the reason for this is
that ``Our understanding of a sentence in a context is due to a
compositional procedure that calculates the content of the whole
sentence from the referential contents of its parts.'' This seems like a
good reason for not taking the semantic content of a sentence in general
to be its character. And we might worry that it could be extended to an
argument against the view that the content of a conditional is a
function from contexts of assessment to propositions. But on closer
inspection it seems like no such generalisation is possible. After all,
someone interpreting a conditional \emph{can} assign a value to the
variable X that takes different values in different contexts of
assessment. They can just `replace' PRO\textsubscript{J} with themselves
when interpreting the conditional. The important point here is that the
view that the utterance does not have a context-neutral semantic content
is consistent with it having a content relative to any interpreter, and
hence to interpreters discovering its content (relative to them).

The other objection we made to indexical relativism in the earlier paper
was that it left as unexplained some phenomena about the behaviour of
epistemic modals in propositional attitude reports that the
non-indexical theory could explain. I still think this is an advantage
of the non-indexical theory, but I don't think it is decisive. (In
general, it isn't a deal breaker that one theory has to take as a brute
fact something that a rival theory explains, although it does count in
favour of the rival.) I'll say more about this objection when we discuss
propositional attitude reports in more detail in section 6. But first, I
need to introduce some facts about the behaviour of plural pronouns that
my indexical relativist theory will exploit.

\section{Partial Binding}\label{partial-binding}

Philosophical orthodoxy has it that all pronouns fall into one of two
broad categories. On the one hand, there are deictic pronouns, whose job
it is to refer (presumably directly) to a contextually salient object.
On the other hand, there are pronouns whose job it is to denote (one way
or another), an object denoted earlier in the discourse. Examples of the
latter kind include the tokens of `she' in (5.1) to (5.3).

\begin{enumerate}
\def\labelenumi{\arabic{enumi}.}
\tightlist
\item
  If Suzy enters the competition, \textbf{she} will win.
\item
  Every student will get the grade that \textbf{she} deserves.
\item
  If a dictator has a daughter, \textbf{she} is pampered by the state.
\end{enumerate}

I'm not going to go into the (very interesting) debates about how many
different kinds of pronouns are represented by the three tokens of `she'
above, nor about which of these pronouns are directly referential, which
are quantifiers, and which are neither of these. All I want to note is
that the pronouns represented here fall into a different category than
simple deictic pronouns, that refer to a contextually salient
individual.

Thomas McKay (\citeproc{ref-McKay2006}{2006} Ch. 9) has argued that the
behaviour of plural pronouns mirrors the behaviour of singular pronouns.
He shows that for every different kind of singular pronoun we can find,
or even purport to find, we can find plural pronouns behaving the same
way. The following three sentences, which are about a film school where
girls make films in large groups, have pronouns that behave just like
the three tokens of `she' above.

\begin{enumerate}
\def\labelenumi{\arabic{enumi}.}
\setcounter{enumi}{3}
\tightlist
\item
  If some girls enter the competition, \textbf{they} will win.
\item
  Some students will produce a better film than we expect \textbf{them}
  to.
\item
  If a student dislikes some girls, \textbf{their} work suffers.
\end{enumerate}

What is quite noteworthy about plural pronouns, however, is that they
need not fall into one of the two major categories I mentioned at the
start of the section.\footnote{I thought this was a way in which plural
  pronouns were unlike singular pronouns, but Zoltán Szabó suggested
  persuasively that singular pronouns could also be partially bound in
  the sense described below. The interesting cases concern pronouns that
  seem to refer to objects made from multiple parts, with circumstances
  of utterance determining some parts of the referent, and the other
  parts being the denotata of the terms to which the pronoun is
  partially bound. I'm not going to take a stand on whether such
  pronouns exist, but if Szabó's suggestion is correct, then the need to
  take X to be a plural variable is lessened.} It is possible to have a
plural pronoun whose denotation is determined partially by context, and
partially by the denotation of earlier parts of the discourse. Consider,
for example, (5.7), as uttered by Jason.

\begin{enumerate}
\def\labelenumi{\arabic{enumi}.}
\setcounter{enumi}{6}
\tightlist
\item
  If Ted comes over, we'll go and get some beers.
\end{enumerate}

It seems the `we' there denotes Ted and Jason. It denotes Jason because
it's a first-person plural pronoun, and Jason is the speaker, and the
speaker is always among the denotata of a first-person plural pronoun.
Arguably, the `we' is anaphoric on `Ted', but this does not mean it
denotes only Ted. Rather, it means that Ted is among the denotata of
`we', the others being determined by context. One might object that
really `we' in (5.7) is deictic, and Ted is among its denotata because
he has been made salient. I think that's probably a mistake, but I don't
want to press the point. Rather, I'll note some other cases where such
an explanation is unacceptable. The following example, due to Jeff King
(p.c.) shows that `we' can behave like a donkey pronoun.

\begin{enumerate}
\def\labelenumi{\arabic{enumi}.}
\setcounter{enumi}{7}
\tightlist
\item
  If any friend comes over, we'll go and get some beers.
\end{enumerate}

Intuitively that's true, as uttered by Jason, just in case for some
salient class of friends, if any member of that class comes over, Jason
and that friend will go and get beers. Now it is controversial just how
to account for donkey pronouns in general, and I'm not going to take a
side on that. But however they work, donkey pronouns seem to fall on the
second side of the divide I mentioned at the top of the section. And
first person singular pronouns are paradigmatic instances of pronouns
that get their reference from context. What's notable is that
first-person plural pronouns can display both kinds of features.

This is not a particularly new point. Example (9) was introduced by
Barbara Partee in 1989, and there is a longer discussion of the
phenomena by Phillipe Schlenker in his
(\citeproc{ref-Schlenker2003}{2003}). The latter paper is the source for
(10) and (11).

\begin{enumerate}
\def\labelenumi{\arabic{enumi}.}
\setcounter{enumi}{8}
\tightlist
\item
  John often comes over for Sunday brunch. Whenever someone else comes
  over too, we (all) end up playing trios.
  (\citeproc{ref-Partee1989}{Partee 1989})
\item
  Each of my colleagues is so difficult that at some point or other
  we've had an argument. (\citeproc{ref-Schlenker2003}{Schlenker 2003})
\item
  {[}Talking about John{]} Each of his colleagues is so difficult that
  at some point or other they've had an argument.
  (\citeproc{ref-Schlenker2003}{Schlenker 2003})
\end{enumerate}

Schlenker describes what is going on here as `partial binding', and I'll
follow his lead here. The `we' in (10) is bound to the earlier
quantifier phrase `Each of my colleagues', but, as above, this does not
mean that it merely denotes (relative to a variable assignment) one of
the speaker's colleagues. Rather, it denotes a plurality that includes
the colleague, and includes the speaker. And the speaker is supplied as
one of the denotata by context.

The reason for mentioning all this here is that my theory of how
conditionals behave involves, among other things, partial binding. The
general semantic structure of a conditional is C(\emph{p},~\emph{q}, X),
where X is a plural variable that denotes some propositions. I think X
can, indeed often is, partially bound. In simple cases a proposition is
among the X just in case it stands in relation \emph{R} to a salient
individual \emph{i}. In more complex cases, X is partially bound to an
earlier phrase, and in virtue of that some proposition \emph{s} is among
the X. But the propositions that stand in relation \emph{R} to \emph{i}
are also among the X, because X is only partially bound to the earlier
proposition. If there were no other instances of partial binding in
natural language, this would be a fairly ad hoc position to take. But
there shouldn't be any theoretical problem with assuming that tacit
variables can behave the way that overt pronouns behave.

\section{Against Indexical
Contexualism}\label{against-indexical-contexualism}

One usual way to argue for relativist theories is to appeal to instances
of faultless disagreement. It is natural to think that such arguments
could work in the case of open indicatives. Since Gibbard
(\citeproc{ref-Gibbard1981}{1981}) there has been a lot of discussion
over cases where A knows ¬(\emph{p} ∧ ¬\emph{q}), and B knows
¬(\emph{p}~∧~\emph{q}). It seems A can truly, even knowledgeably, say
\emph{If p, q}, and B can truly, even knowledgeably, say \emph{If p,}
¬\emph{q}. And, in the right context, it might seem that this is a case
where A and B disagree. One might try and argue that the only way to
explain this faultless disagreement between A and B is through some
variety of relativistic semantics. I think that will be a hard argument
to make out for four reasons.

First, some people hold that the notion of a faultless disagreement is
incoherent. I suspect that's wrong, and the concept is coherent, but
making this argument stick would require showing that faultless
disagreement is indeed coherent. I want the argument for indexical
relativism about open indicatives to not rely on the coherence of
faultless disagreement.

Second, two people can disagree without there being any proposition that
one says is true and the other is false. (This should be familiar from
debates about non-cognitivism in ethics.) If A says ``I like ice cream''
and B says ``I don't like ice cream'', then there is a natural sense in
which they are disagreeing, for instance. But arguments from
disagreement for relativism generally require that when two people
disagree, there is a proposition that one accepts and the other rejects,
and that may not be true.

Third, there is some special reason to think that this is what happens
in the conditionals case. In this case A and B are having a
\emph{conditional} disagreement. Perhaps we intuit that A and B are
disagreeing merely because of this conditional disagreement. For
comparison, if A and B had made a conditional bet, we would describe
them as having made a bet in ordinary discourse, even if the bet is not
realised because the condition is not satisfied.

Finally, as Grice (\citeproc{ref-Grice1989}{1989}) showed, there can be
cases where we naturally describe A and B as disagreeing in virtue of
two utterances, even though (a) those utterances are simple assertions,
and (b) the assertions are consistent. Grice's case is where A says that
\emph{p} or \emph{q} is true, and B says that \emph{p} or \emph{r} is
true, with stress on \emph{r}, where \emph{q} and \emph{r} are obviously
incompatible. Perhaps the natural thing to say here too is that A and B
have a conditional disagreement; conditional on ¬\emph{p}, A thinks that
\emph{q} and B thinks that \emph{r}. So this argument seems to need a
lot of work.

An apparently stronger argument comes from indirect speech reports. It
seems that in any case where a speaker, say Clarke, says ``If the doctor
didn't do it, the lawyer did'', then in any other context, we can report
that by saying ``Clarke said that if the doctor didn't do it, the lawyer
did.'' This might look to pose a problem for indexical contextualism.

Assuming that the content of \emph{If p, q} is C(\emph{p}, \emph{q},
\emph{R}(\emph{i})), where \emph{i} is some person made salient by the
context of utterance. Let \emph{p} be \emph{The doctor didn't do it},
and \emph{q} be \emph{The lawyer did it}, \emph{c} be Clarke, and
\emph{h} be the person who reports what Clarke said. Then it seems that
what Clarke said is C(\emph{p}, \emph{q}, \emph{R}(\emph{c})). But it
seems that the content of what comes after the \emph{that} in the report
is C(\emph{p},~\emph{q},~\emph{R}(\emph{h})). But since
\emph{R}(\emph{c}) might not be the same as \emph{R}(\emph{h}), this
should look like a bad report.

I think this is something of a problem for the indexical contextualist,
but it isn't beyond repair. It could be that the variable \emph{i} in
the speech report is bound to the name at the start of the report, so
the value of \emph{i} in \emph{Clarke said} C(\emph{p}, \emph{q},
\emph{R}(\emph{i})) is simply Clarke herself. This is a slightly odd
kind of binding, but it isn't impossible, so this doesn't quite rule out
a contextualist theory.

As MacFarlane (\citeproc{ref-MacFarlane2009-MACNC}{2009}) argues, the
felicity of homophonic reports does not raise a problem for either kind
of non-indexical theory. I'll argue that it also doesn't pose a problem
for indexical relativism.

The indexical relativist thinks that, on its most natural
interpretation, the content of Clarke's utterance is C(\emph{p},
\emph{q}, \emph{R}(PRO\textsubscript{J})). Similarly, it might be
thought that the natural interpretation of what comes after the
\emph{that} in the report is C(\emph{p}, \emph{q},
\emph{R}(PRO\textsubscript{J})). So it isn't surprising that the report
is acceptable.

I can imagine an objector making the following speech. ``Assume that
Clarke's utterance was sincere. Then it seems natural to say that Clarke
believes that if the doctor didn't do it, the lawyer did. But it is odd
to think that Clarke believes C(\emph{p}, \emph{q},
\emph{R}(\emph{PRO\textsubscript{J})). What she believes is
C(}p\emph{,~}q\emph{, }R\emph{(}c\emph{)). The only way to get belief
reports to work on the indexical contextualist theory is to insist that
the }i* is bound to the subject of the report. But once you say that it
is ad hoc to deny that the \emph{i} is also bound to the subject in a
speech report. And not just ad hoc, it implies that (relative to our
context) Clarke doesn't believe what she says, for she says C(\emph{p},
\emph{q}, \emph{R}(PRO\textsubscript{J})) but believes
C(\emph{p},~\emph{q}, \emph{R}(\emph{c})). On the other hand if the
\emph{i} is bound to the speaker in a speech report, so what Clarke is
said to have said is C(\emph{p}, \emph{q}, \emph{R}(\emph{c})), then (a)
you have no advantage over the indexical contextualist, and (b) you
can't explain why the report is felicitous, since you say she says
something else, namely C(\emph{p}, \emph{q},
\emph{R}(PRO\textsubscript{J})).''

I have three responses to this critic. The first is that I'm not
convinced that having a different treatment of belief reports and speech
reports, letting \emph{i} be PRO\textsubscript{J} in speech reports and
the believer in belief reports, is too terrible. The argument that it
would be ad hoc to treat speech reports and belief reports separately
seems weak. It is worse if we end up, because of the structure of the
theory, accusing Clarke of insincerity. One way of avoiding that
response is to accept the binding proposal. But it isn't the only way.
If we make two assumptions about C and \emph{R}, we can sidestep the
danger.

The first assumption is that C is monotonic in the sense that
C(\emph{p},~\emph{q},~X~+~Y) is entailed by C(\emph{p},~\emph{q}, X).
The second is that \emph{xRp} is true in case someone salient to
\emph{x} bears \emph{R} to \emph{p}, and the utterer of the judged
sentence is salient (in this sense) to the judge. (Note this is exactly
the assumption that I made earlier to defend my proposal about what
effect uttering a conditional has on the Stalnakerian context.) Now not
all indexical relativists will want to make these assumptions, but I'm
happy to do so. Now if Clarke believes \emph{If p}, \emph{q}, she
believes C(\emph{p},~\emph{q}, \emph{R}(\emph{c})). So she either
believes she knows ¬(\emph{p} ∧ ¬\emph{q}), or believes she knows some
things that (perhaps unbeknownst to her) entail it. That means that,
whoever the judge of her utterance is, she believes that
\emph{R}(PRO\textsubscript{J}) either includes or entails
¬(\emph{p}~∧~¬\emph{q}). So she believes C(\emph{p},~\emph{q},
\emph{R}(PRO\textsubscript{J})), as required.

So I don't think the indexical relativist has to concede to the critic
that speech reports involve binding in this way. But I might be wrong
about this, so my second and third responses concede this point, and
argue that it doesn't harm the indexical relativist position.

The second response is that even with this concession, the indexical
relativist has a small advantage over the indexical contextualist.
Drawing on Stephenson's work, we could argue that (a)
PRO\textsubscript{J} is often the value of a tacit variable, and (b)
whenever it is the default value of a variable, then that variable is
bound to the subject of a propositional attitude report. If that is the
case, then the indexical relativist could unify a number of different
cases that would have to be treated separately by the indexical
contextualist. Still, it is true that the non-indexicalist has an even
larger advantage here, since they can explain why this (apparent)
binding holds, but I don't think this advantage is decisive. This is the
one argument for non-indexicalism that I mentioned in section 4 might
still have some force, though not I think enough to override the
argument for indexicalism in the next section.

The third response is that the indexical relativist has a simple
explanation of why the reports are natural, even on the assumption that
the \emph{i} is bound to the speaker. First consider a similar case.
Imagine Clarke had simply said ``The lawyer did it'', i.e.~\emph{q}. It
would be natural to report her as having said that the lawyer actually
did it. Now one can imagine being surprised at this. Clarke said
something contingently true, but we reported her using a proposition
\emph{that the lawyer actually did it}, that is necessarily true. How is
this possible? Well, it is because in saying \emph{q}, she immediately
and obviously commits herself to \emph{Actually q}, and if a speaker
immediately and obviously commits themselves to a proposition in virtue
of an utterance, then it is natural to report them as having said that
proposition. Speakers are generally committed to the truth of the
utterances from their own perspective, so Clarke is committed to
C(\emph{p}, \emph{q}, \emph{R}(\emph{c})). (Arguably that is all she is
committed to, as opposed to C(\emph{p}, \emph{q},
\emph{R}(PRO\textsubscript{J})).) So we can report her as having said
C(\emph{p},~\emph{q}, \emph{R}(PRO\textsubscript{J})). And if \emph{i}
is bound to the speaker, that is what we do report her as having said by
saying ``Clarke said that if the doctor didn't do it, the lawyer
did.''\footnote{This response is similar to some of the arguments for
  speech act pluralism in Cappelen and Lepore
  (\citeproc{ref-Cappelen2005}{2005}).}

So both the argument from disagreement and the argument from speech
report against indexical contextualism have run up against some blocks.
There is another argument, however, that is effective against it. This
is an argument from easy agreement. Assume again that Clarke said ``If
the doctor didn't do it, the lawyer did.'' Assume that an arbitrary
person, call him Rebus, knows that Clarke made this utterance, and knows
that either the doctor or the lawyer did it, that is knows that
¬(\emph{p} ∧ ¬\emph{q}). On that basis alone, it will be natural for
Rebus to make any of the following utterances. ``I agree''; ``That's
right''; ``That's true''; ``What she said is true''; ``She spoke
truly''. Any of these are hard to explain on the indexical contextualist
view, according to which agreement should be harder to get than this.

On the indexical contextualist view, Clarke said C(\emph{p}, \emph{q},
\emph{R}(\emph{c})). Now on most accounts of what \emph{r} is, Rebus
need not know that ¬(\emph{p} ∧ ¬\emph{q}) is one of the propositions in
\emph{R}(\emph{c}). He need not know that Clarke know this, or that
Clarke could have known this, or really anything else. As long as he
knows that Clarke made this utterance, it seems acceptable for him to
agree with it using any of the above formulations.

There are two ways that the indexical contextualist might try to explain
this agreement. First, they could try to argue that it is acceptable for
Rebus to agree with Clarke's utterance despite not agreeing with the
propositional content of it. Second, they could try to argue that it is
the case, in any case fitting the above description, that he agrees with
the propositional content of what Clarke said.

The first approach does not seem particularly attractive. Not only does
it seem theoretically implausible, it is hard to find independent reason
to believe that this is how agreement works. Generally if a speaker
utters some term with a contextually sensitive term in it, then another
speaker will not agree with the utterance unless they agree with the
proposition we get by filling in the appropriate value for the
contextually sensitive term. Or, perhaps more precisely, they will not
accept all five of above forms of agreement.

This is how agreement works when the utterance contains an explicit
indexical like `I'. Note that if Clarke had said ``I like Hibs'', Rebus
could say ``I agree'' if he too likes Hibs. But he couldn't have said,
for instance, ``What she said is true'' unless Clarke liked Hibs. This
is why the variety of forms of agreement matters. Perhaps more
contentiously, I think this is also what happens when the original
utterance involves quantifiers with tacit domain restriction, or
comparative adjectives with tacit comparison classes, or modals, or any
other kind of context sensitive language. So the indexical contextualist
should look at a second option.

The second option is to say that the proposition Rebus knows, ¬(\emph{p}
∧ ¬\emph{q}), will be one of the \emph{R}(\emph{c}). There are two ways
to do this. First, we could say that \emph{R}(\emph{c}) includes all
propositions that are known by anyone, so as long as Rebus knows
¬(\emph{p} ∧ ¬\emph{q}), it is one of the \emph{R}(\emph{c}). But this
just about reduces back to the material implication theory of
indicatives, since any conditional will be true as long as anyone knows
the corresponding material implication. And that is implausible. Second,
we could say that \emph{R}(\emph{c}) includes any proposition known by
anyone who hears \emph{c}'s utterance. That would again ensure that
¬(\emph{p} ∧ ¬\emph{q}) is one of the \emph{R}(\emph{c}). But again, it
is fairly implausible. For one thing, it doesn't \emph{seem} that the
truth of the conditionals I'm writing in this paper depend on how wide a
readership the paper has. For another, under some assumptions this again
collapses into the material implication theory. Assume that there is an
omniscient deity. Then for any conditional \emph{If p}, \emph{q}, the
deity's knowledge is among \emph{R}(\emph{c}), and if ¬(\emph{p} ∧
¬\emph{q}) is true, then it is one of the \emph{R}(\emph{c}). But then
\emph{If p}, \emph{q} will be true, which was not what we wanted. Now we
don't know that there is an omniscient deity, but it seems reasonable to
require that our semantic theories be at least consistent with the
existence of such a deity.

So I think the indexical contextualist has no explanation of this
agreement phenomena. But the indexical relativist has no such problem.
When Clarke's utterance is being judged by Rebus, it expresses the
proposition C(\emph{p}, \emph{q}, \emph{R}(Rebus)), so ¬(\emph{p} ∧
¬\emph{q}) is clearly one of the propositions in the third clause.
That's why agreement with another's utterance of a conditional is so
easy.

Similarly there is no problem for a non-indexical relativist. It is a
little trickier to know whether there is a problem here for the
non-indexical contextualist. It is easy to see why many of the locutions
Rebus could use are acceptable. Rebus does, after all, accept the
proposition that Clarke expresses, namely C(\emph{p}, \emph{q}). The
only complication concerns ``She spoke truly.'' There is a sense in
which that's not really true. After all, the non-indexical contextualist
thinks that Clarke's \emph{utterance} was false, since they think that
an utterance is true iff the proposition it expresses is true in its
context. If we think, as probably isn't compulsory, that ``She spoke
truly'' means what the theorist means by saying the utterance is true,
then there is a problem for the non-indexical contextualist.

Setting aside those complications, what is clear is that the phenomena
of agreement raises a problem for the indexical contextualist, and not
for the relativist. We can put the problem another way. If indexical
contextualism is true, it should be possible for Rebus to say ``Clarke's
utterance''If the doctor didn't do it, the lawyer did'' was not true,
but if the doctor didn't do it, the lawyer did.'' Again, this seems like
it should be possible on theoretical grounds, and it is possible for
most contextually sensitive sentences. But this doesn't seem to be a
coherent speech on Rebus's part. This is of course just another
manifestation of the phenomena that agreement with conditionals is easy.
If Rebus accepts that if the doctor didn't do it, the lawyer did, then
he accepts that Clarke's utterance was true. The indexical contextualist
can't explain this, so indexical contextualism is false.

\section{Against Non-Indexicalism}\label{against-non-indexicalism}

The argument for an indexicalist account of indicatives is that they
allow an elegant account of what is going on in apparent counterexamples
to modus ponens, such as the cases due to Vann McGee
(\citeproc{ref-McGee1985}{1985}). What these cases turn on is that
right-embedded conditionals, like \emph{If p, then if q, r} seem
equivalent, in some sense, to conditional with conjunctive antecedents,
in this case \emph{If p and q}, \emph{r}. Given this equivalence, and
the triviality of \emph{If p and q, p and q}, we get the result that (1)
is trivial.

\begin{enumerate}
\def\labelenumi{\arabic{enumi}.}
\tightlist
\item
  If \emph{p}, then if \emph{q}, \emph{p} and \emph{q}
\end{enumerate}

And if (7.1) is genuinely trivial, then a number of awkward consequences
follow. Perhaps the worst of these consequences is that we seem to get
counterexamples to modus ponens. Let \emph{p} be some truth that isn't
knowable. Since (1) is trivial, it is true. And by hypothesis \emph{p}
is true. But on pretty much any epistemic theory of conditionals,
\emph{If q}, \emph{p and q} will not be true. So we have counterexamples
to modus ponens. (This is more like McGee's `lungfish' example than the
more widely cited example about the 1980 election, but the structure I
think is basically the same, and the solution I offer will generalise to
all these cases.)

What I'm going to say about these cases borrows heavily from some
remarks by Anthony Gillies (\citeproc{ref-Gillies2009-GILOTF}{2009}).
Gillies makes two observations that point towards a solution to the
puzzle McGee's cases raise.

First, we cannot in general \emph{assert} both of the premises, namely
(7.1) and \emph{p}, in contexts where the conclusion, namely \emph{If
q}, \emph{p and q} is not assertable. This might need to be qualified in
cases where people don't know what they can assert, but it is largely
right. As Gillies demonstrates by close attention to the cases, some
kind of context shift between the premises and conclusion is needed in
order to assert the conclusion after the premises have been asserted.

Second, there are many reasons to believe that part of what why (1)
seems trivial is that we evaluate its consequent relative to a context
in which \emph{p} is taken to be part of the evidence. Gillies
formalises this by having the antecedent play two separate roles, first
as a constituent of the conditional uttered, and second as a
context-modifier relative to which the consequent is interpreted. The
formal theory I'm building here is quite different to Gillies' because
of very different starting assumptions, but I will adopt Gillies' idea
to the framework I'm using.

Despite Gillies' first observation, there are still three reasons to
take seriously the challenge McGee's cases raise. Two of these involve
using modus ponens under the scope of a supposition, and the third
involves agents who don't know what they know. The first problem
concerns the following implication.

\begin{longtable}[]{@{}
  >{\raggedleft\arraybackslash}p{(\columnwidth - 4\tabcolsep) * \real{0.2055}}
  >{\raggedright\arraybackslash}p{(\columnwidth - 4\tabcolsep) * \real{0.4932}}
  >{\raggedright\arraybackslash}p{(\columnwidth - 4\tabcolsep) * \real{0.3014}}@{}}
\toprule\noalign{}
\endhead
\bottomrule\noalign{}
\endlastfoot
(1) & If \emph{p}, if \emph{q}, \emph{p} and \emph{q} & Premise \\
(2) & If ¬\emph{p}, if \emph{q}, ¬\emph{p} and \emph{q} & Premise \\
(3) & \emph{p} or ¬\emph{p} & Logical truth \\
(4) & \emph{p} & Assumption for argument by cases \\
(5) & If \emph{q}, \emph{p} and \emph{q} & Modus Ponens, 1, 4 \\
(6) & (If \emph{q}, \emph{p} and \emph{q}) or (If \emph{q}, ¬\emph{p}
and \emph{q}) & Or introduction, 5 \\
(7) & ¬\emph{p} & Assumption for argument by cases \\
(8) & If \emph{q}, ¬\emph{p} and \emph{q} & Modus Ponens, 2, 7 \\
(9) & (If \emph{q}, \emph{p} and \emph{q}) or (If \emph{q}, ¬\emph{p}
and \emph{q}) & Or introduction, 8 \\
(10) & (If \emph{q}, \emph{p} and \emph{q}) or (If \emph{q}, ¬\emph{p}
and \emph{q}) & Argument by cases, 3, 4-6, 7-9 \\
\end{longtable}

But on the simple epistemic theory we've been using here, (10) will not
be true in cases where the truth value of \emph{p} is unknown, even
though it seems to follow from two trivialities and a logical truth.
(I'm assuming here either that classical logic is correct, or that
\emph{p} is decidable.) Now it might be noted here that on some
theories, particularly those that follow Stalnaker
(\citeproc{ref-Stalnaker1981}{1981}) in accepting conditional excluded
middle, (10) will be true. But even on those Stalnakerian theories,
there will be cases where neither disjunct of (10) will be determinately
true. And we can rerun a version of this argument, taking as premises
that (1), (2) and (3) are determinately true, to derive as a conclusion
that one or other disjunct is determinately true.

The third reason is similar to the second. We can use modus ponens in
the scope of a reductio proof. Or, more colloquially, we can use modus
tollens. But the following argument does not look to be particularly
compelling.

\begin{longtable}[]{@{}
  >{\raggedleft\arraybackslash}p{(\columnwidth - 4\tabcolsep) * \real{0.0959}}
  >{\raggedright\arraybackslash}p{(\columnwidth - 4\tabcolsep) * \real{0.6164}}
  >{\raggedright\arraybackslash}p{(\columnwidth - 4\tabcolsep) * \real{0.2877}}@{}}
\toprule\noalign{}
\endhead
\bottomrule\noalign{}
\endlastfoot
(1) & If \emph{p}, if \emph{q}, \emph{p} and \emph{q} & Premise \\
(2) & It is not the case that if \emph{q}, \emph{p} and \emph{q} &
Premise \\
(3) & Not \emph{p} & Modus Tollens, 1, 2 \\
\end{longtable}

It may be objected that modus tollens is more controversial than modus
ponens. But since we can derive it using just modus ponens and reductio
ad absurdum, this objection looks weak. So this would be a bad result.

It might be thought best to say here that modus ponens doesn't preserve
truth, but it does preserve knowledge. If a subject is in knows each
premise, they can know the conclusion. But that doesn't seem right
either, though the cases are slightly obscure. Assume a perfectly
rational S knows that \emph{p}, but does not know that she knows that
\emph{p}, and in fact for all she knows she knows, \emph{q} and
¬\emph{p} is true. Again assuming (7.1) is trivial, she knows it, and
she knows that \emph{p}, but on an epistemic interpretation of the
conditional, she won't know \emph{If q}, \emph{p and q}, since she
doesn't know she knows that \emph{p}.

So there is a serious problem here. Once we accept that (7.1) is
trivial, a lot of unfortunate consequences follow for the epistemic
theory of conditionals. Any explanation of why (7.1) seems trivial will,
I think, have to start with Gillies' insight that when we interpret
(7.1), we evaluate its consequent relative to a context where \emph{p}
is taken as given. How might we do this? Three options spring to mind.

The first option is Gillies' theory is that it is part of the meaning of
the conditional that its consequent be interpreted relative to a context
where its antecedent is part of the background information. That has the
nice result that (7.1) is indeed trivial. It seems, however, to lead to
all the problems mentioned above. Gillies' response to these is to
develop a new theory of validity, which has the effect that while modus
ponens is itself valid, it can't be used inside the scope of
suppositions, as I frequently did above. This is a very interesting
theory, and it may well work out, but I'm going to try to develop a more
conservative approach.

The second option is to say that just uttering a conditional, \emph{If
A, B}, adds \emph{A} to the background information. This seems like a
bad option. For one thing, there is no independent reason to believe
that this is true. For another, it can't explain what is wrong with the
following kind of argument.

\begin{enumerate}
\def\labelenumi{\arabic{enumi}.}
\tightlist
\item
  Burns knows that if \emph{p}, then if \emph{q}, \emph{p} and \emph{q}.
\item
  Burns knows that it is not the case that if \emph{q}, \emph{p} and
  \emph{q}.
\item
  Burns is logically perfect, and knows the logical consequences of
  everything he knows.
\item
  So, Burns knows not \emph{p}.
\end{enumerate}

In a case where Burns doesn't know whether \emph{p} is true, and Burns
is indeed logically perfect, then intuitively (1), (2) and (3) are true,
but (4) are false. And since no conditionals were asserted, it is hard
to see how the context was shifted.

The third, and best, option is to say that the variable in the semantics
of an embedded conditional is partially bound to the antecedent.
Normally when we say \emph{If q}, \emph{p and q}, the content of that is
C(\emph{q}, \emph{p} and \emph{q}, X), and normally X is
\emph{R}(PRO\textsubscript{J}). The view under consideration says that
when that conditional is itself the consequent of a conditional, the
variable X is partially bound to the antecedent of the conditional. So
the value of X is \emph{p} plus whatever is supplied by context.

The contextualist says that that value is \emph{R}(\emph{i}), where
\emph{i} is usually the speaker. So the semantic content of (7.1) is
C(\emph{p}, C(\emph{q}, \emph{p} and \emph{q}, \emph{p} +
\emph{R}(\emph{i})), \emph{R}(\emph{i})). And that will be trivial since
the middle term is trivial. The relativist says that the contribution of
context to X is \emph{R}(PRO\textsubscript{J}). So the semantic content
of (7.1) is C(\emph{p}, C(\emph{q}, \emph{p} and \emph{q}, \emph{p} +
\emph{R}(PRO\textsubscript{J})), \emph{R}(PRO\textsubscript{J})). And
again, that is trivial.

This gives us a natural explanation of what is going on in the McGee
cases. There is simply an equivocation between premise and conclusion in
all of the cases. What follows from C(\emph{p}, C(\emph{q}, \emph{p} and
\emph{q}, \emph{p} + \emph{R}(\emph{x})), \emph{R}(\emph{x})) and
\emph{p} is C(\emph{q}, \emph{p} and \emph{q}, \emph{p} +
\emph{R}(\emph{x})). But that's not what we normally express by \emph{If
q}, \emph{p and q}. At least, it isn't what we express once we've made
it clear that \emph{p} is not part of the background information. (Here
is where Gillies' observation that the McGee cases seem to require a
context shift between stating the premises and stating the conclusion
becomes relevant.) So although modus ponens is valid, the McGee cases
are simply not instances of modus ponens, since there is an equivocation
in the value of a tacit variable.

It might be argued that this is too much of a concession to McGee. Some
people have the judgment that (1) is not always trivial, in particular
that conditionals \emph{If A}, \emph{then if B}, \emph{A} are not always
trivially true. Personally I don't get these readings, but I note that
the theory allows for their possibility. After all, binding need not be
compulsory. We can interpret the `she' in \emph{If Suzy enters the
race}, \emph{she will win} deictically, if that's what makes the best
sense in the context. Perhaps in cases where people are hearing the
false readings of \emph{If A}, \emph{then if B}, \emph{A}, all that is
going on is that the tacit indexical in the embedded conditional is
unbound. Similarly, if one's reaction to seeing the McGee arguments is
to interpret the embedded conditionals as false, I suspect what is going
on is that one is hearing the variables here as unbound. As I said, I
don't get these readings, but I can explain where these readings come
from.

The story I'm telling about the McGee cases is hardly new. Indeed, the
view that the McGee cases are not strictly speaking instances of Modus
Ponens is old enough to have been disparaged by William Lycan in his
attacks on Modus Ponens.

\begin{quote}
But this very strict sense of `instance' is neither specific nor
intended in logic textbooks \ldots{} What students and professional
philosophers have always been told is that barring equivocation or overt
indexicals, arguments of the sentential form If A, B; A; therefore, B
are valid arguments, period \ldots{} One can continue to insist that
Modus Ponens is valid for the strict sense of `instance', but at the
price of keeping us from telling easily and uncontroversially when a set
of ordinary English sentences is an `instance' of an argument form.
(\citeproc{ref-Lycan1993}{Lycan 1993, 424}, notation slightly altered)
\end{quote}

But why should we give any privilege to \emph{overt} indexicals? Tacit
variables can be just as important in determining which form an argument
takes. For example, the following argument is, on the most natural
interpretation of each sentence, invalid.

\begin{enumerate}
\def\labelenumi{\arabic{enumi}.}
\tightlist
\item
  No foreigner speaks a foreign language.
\item
  Ségolinè is a foreigner.
\item
  French is a foreign language.
\item
  Ségolinè does not speak French.
\end{enumerate}

That is invalid on its most natural reading because the tacit variable
attached to `foreign' in premises 1 and 3 takes a different value. No
one would reasonably say that we should rewrite the logic books so the
argument form \emph{No F Rs a G}; \emph{Fa}; \emph{Gb}; so ¬\emph{Rab}
is not valid on this account. Lycan is right about the downside of this
point. There is no way to tell easily and uncontroversially what the
form of an argument in natural language is. But we should never have
believed such careful matters of interpretation would be easy. (They say
life wasn't meant to be.)

Having said that, on the indexical relativist proposal offered here, it
isn't \emph{that} hard to tell what the value of X in a typical
indicative is. It is usually \emph{R}(PRO\textsubscript{J}), and there
might be a very short list of circumstances where it takes any other
value. Any indexical account faces a potential cost that it makes
interpretation more difficult than it might otherwise be, since the
hearer has to determine the value for the indexical. The fact that X is
usually \emph{R}(PRO\textsubscript{J}) minimises that cost. What is new
to my proposal is that X might be partially bound in the McGee cases.
But that only helps the interpretative task, since it reduces the task
to a familiar problem interpreters face when the speaker uses a
partially bound plural pronoun.

But the primary point of this proposal is not to offer a new solution to
the McGee cases. Rather it is to note one of the requirements of this
kind of (relatively familiar) solution. An equivocation solution
requires that there be something in the semantic content of the
conditional that takes different values in the consequent of premise 1
and in the conclusion. And non-indexical theories, by definition, can't
say that there is any such thing. For the whole point of such theories
is to deny that the content of a conditional is always different in
contexts with different information sets. So they cannot say the McGee
arguments (or the other arguments I surveyed above that use Modus Ponens
in embedded contexts) involve equivocation. But then it is hard to say
what is wrong with those arguments. So these theories seem, implausibly,
to be committed to denials of Modus Ponens. That's a sufficient reason,
I think, to be an indexicalist.

Let's take stock. In section 6 I argued that the indexical contextualist
has no explanation of why it is so easy to \emph{agree} with another's
utterance of a conditional. In this section I argued that only the
indexicalist can offer a satisfactory explanation of what is going on in
the McGee argument. The upshot of these two arguments is that we should
be indexical relativists. For only the indexical relativist can (a)
explain the agreement data and (b) explain what goes wrong in the McGee
arguments.

As a small coda, let me mention one other benefit of the partial binding
account. When I presented an earlier version of this paper at the LOGOS
workshop on Relativising Utterance Truth, the following objection was
pressed to the argument in section 1 for an epistemic treatment of
indicatives. It is true that when we know that \emph{f}(\emph{a}) =
\emph{f}(\emph{b}), then we are prepared to assert \emph{If
f}(\emph{a})~=~\emph{x}, \emph{then f}(\emph{b})~=~\emph{x}. But it is
also true that when we merely suppose that \emph{f}(\emph{a}) =
\emph{f}(\emph{b}), then we are prepared to infer inside the scope of
the supposition that \emph{If f}(\emph{a})~=~\emph{x}, \emph{then
f}(\emph{b})~=~\emph{x}. The epistemic account cannot satisfactorily
explain this. At the time I didn't know how to adequately explain these
intuitions, but now it seems the partial binding story can do the work.
It seems that inside the scope of a supposition that \emph{p}, the value
of X is \emph{p} + Y, where Y is the value X would otherwise have had.
That is, the variable in the conditional is partially bound to the
supposition that governs the discourse. That explains why all the
inferences mentioned in section one are acceptable, even when the
premise is merely a supposition.

\section*{Coda: Methodological
Ruminations}\label{coda-methodological-ruminations}
\addcontentsline{toc}{section}{Coda: Methodological Ruminations}

The version of relativism defended here is conservative in a number of
respects. Three stand out.

First, it is conservative about what propositions are. The propositions
that are the content of open indicatives (relative to contexts of
assessment) are true or false relative to worlds, not to judges, or
epistemic states, or anything of the sort.

Second, it is (somewhat) conservative about how the sentences get to
have those propositions as content. The standing meaning of the sentence
contains a variable place that gets filled by context. To be sure, it is
a plural variable that can be partially bound, but there is independent
evidence that plural variables can be partially bound. And of course,
and this is a radical step, its value can be different for different
assessors of the one utterance. But from the indexical relativist
perspective, the contextualist theory that values for variables are set
by contexts of utterance is an overly hasty generalisation from the
behaviour of a few simple indexicals. (It isn't clear even clear that
the contextualist theory can account for simple pronouns, like `you' or
`now' as they appear in sentences like the one you are now reading, so
this generalisation might have been very poorly motivated in the first
place.)

Third, it is conservative about the motivation for relativism. I haven't
relied on intuitions about faultless disagreement, which is an
inherently controversial topic. Rather, I've argued that we can motivate
relativism well enough by just looking at the grounds on which people
\emph{agree} with earlier utterances. I think there is a general
methodological point here; most of the time when theorists try to
motivate relativism using cases of disagreement, they could derive most
of their conclusions from careful studies of cases of agreement. This
method won't \emph{always} work; I don't think you can replicate the
disagreement-based arguments for moral relativism with arguments from
agreement for example. But I think that is a weakness with moral
relativism, rather than a weakness with the methodology of focussing on
agreement rather than disagreement with arguing for relativism.

Now one shouldn't fetishise epistemic conservativeness. But a relativism
that requires less of a revision of our worldview should be more
plausible to a wider range of people than a more radical relativist
view. And that's what I've provided with the indexical relativist theory
defended here.

\subsection*{References}\label{references}
\addcontentsline{toc}{subsection}{References}

\phantomsection\label{refs}
\begin{CSLReferences}{1}{0}
\bibitem[\citeproctext]{ref-Adams1998}
Adams, Ernest. 1998. \emph{A Primer on Probability Logic}. Palo Alto:
CSLI.

\bibitem[\citeproctext]{ref-Bradley2000}
Bradley, Richard. 2000. {``A Preservation Condition for Conditionals.''}
\emph{Analysis} 60 (3): 219--22. doi:
\href{https://doi.org/10.1093/analys/60.3.219}{10.1093/analys/60.3.219}.

\bibitem[\citeproctext]{ref-Cappelen2008}
Cappelen, Herman. 2008. {``Content Relativism.''} In \emph{Relativising
Utterance Truth}, edited by Manuel Garcia-Carpintero and Max Kölbel,
265--86. Oxford: Oxford University Press.

\bibitem[\citeproctext]{ref-Cappelen2005}
Cappelen, Herman, and Ernest Lepore. 2005. \emph{Insensitive Semantics:
A Defence of Semantic Minimalism and Speech Act Pluralism}. Oxford:
Blackwell.

\bibitem[\citeproctext]{ref-Egan2005-EGAEMI}
Egan, Andy, John Hawthorne, and Brian Weatherson. 2005. {``{Epistemic
Modals in Context}.''} In \emph{Contextualism in Philosophy: Knowledge,
Meaning, and Truth}, edited by Gerhard Preyer and Georg Peter, 131--70.
Oxford: Oxford University Press.

\bibitem[\citeproctext]{ref-Einheuser2008}
Einheuser, Iris. 2008. {``Three Forms of Truth-Relativism.''} In
\emph{Relativising Utterance Truth}, edited by Manuel Garcia-Carpintero
and Max Kölbel, 187--203. Oxford: Oxford University Press.

\bibitem[\citeproctext]{ref-Gibbard1981}
Gibbard, Allan. 1981. {``Two Recent Theories of Conditionals.''} In
\emph{Ifs}, edited by William Harper, Robert C. Stalnaker, and Glenn
Pearce, 211--47. Dordrecht: Reidel.

\bibitem[\citeproctext]{ref-Gillies2004-GILECA}
Gillies, Anthony S. 2004. {``Epistemic Conditionals and Conditional
Epistemics.''} \emph{No{û}s} 38 (4): 585--616. doi:
\href{https://doi.org/10.1111/j.0029-4624.2004.00485.x}{10.1111/j.0029-4624.2004.00485.x}.

\bibitem[\citeproctext]{ref-Gillies2009-GILOTF}
---------. 2009. {``{On Truth-Conditions for If (but Not Quite Only If
)}.''} \emph{Philosophical Review} 118 (3): 325--49. doi:
\href{https://doi.org/10.1215/00318108-2009-002}{10.1215/00318108-2009-002}.

\bibitem[\citeproctext]{ref-Grice1989}
Grice, H. Paul. 1989. \emph{Studies in the Way of Words}. Cambridge,
MA.: Harvard University Press.

\bibitem[\citeproctext]{ref-Kaplan1989}
Kaplan, David. 1989. {``Demonstratives.''} In \emph{Themes from Kaplan},
edited by Joseph Almog, John Perry, and Howard Wettstein, 481--563.
Oxford: Oxford University Press.

\bibitem[\citeproctext]{ref-KingStanley2005}
King, Jeff, and Jason Stanley. 2005. {``Semantics, Pragmatics and the
Role of Semantic Content.''} In \emph{Semantics Vs Pragmatics}, edited
by Zoltan Szabó, 111--64. Oxford: Oxford University Press.

\bibitem[\citeproctext]{ref-Lasersohn2005}
Lasersohn, Peter. 2005. {``Context Dependence, Disagreement and
Predicates of Personal Taste.''} \emph{Linguistics and Philosophy} 28
(6): 643--86. doi:
\href{https://doi.org/10.1007/s10988-005-0596-x}{10.1007/s10988-005-0596-x}.

\bibitem[\citeproctext]{ref-Lewis1973a}
Lewis, David. 1973. \emph{Counterfactuals}. Oxford: Blackwell
Publishers.

\bibitem[\citeproctext]{ref-LopezDeSaMs}
López de Sa, Dan. 2007a. {``(Indexical) Relativism about Values: A
Presuppositional Defense.''}

\bibitem[\citeproctext]{ref-LopezDeSa2007a}
---------. 2007b. {``The Many Relativisms and the Question of
Disagreement.''} \emph{International Journal of Philosophical Studies}
15 (2): 339--48. doi:
\href{https://doi.org/10.1080/09672550701383871}{10.1080/09672550701383871}.

\bibitem[\citeproctext]{ref-LopezDeSa2008b}
---------. 2008. {``Presuppositions of Commonality.''} In
\emph{Relativising Utterance Truth}, edited by Manuel Garcia-Carpintero
and Max Kölbel, 297--310. Oxford University Press.

\bibitem[\citeproctext]{ref-Lycan1993}
Lycan, William. 1993. {``MPP, RIP.''} \emph{Philosophical Perspectives}
7: 411--28. doi:
\href{https://doi.org/10.2307/2214132}{10.2307/2214132}.

\bibitem[\citeproctext]{ref-MacFarlane2003-MACFCA-2}
MacFarlane, John. 2003. {``{Future Contingents and Relative Truth}.''}
\emph{The Philosophical Quarterly} 53 (212): 321--36. doi:
\href{https://doi.org/10.1111/1467-9213.00315}{10.1111/1467-9213.00315}.

\bibitem[\citeproctext]{ref-MacFarlane2005-MACMSO}
---------. 2005. {``{Making Sense of Relative Truth}.''}
\emph{Proceedings of the Aristotelian Society} 105 (1): 321--39. doi:
\href{https://doi.org/10.1111/j.0066-7373.2004.00116.x}{10.1111/j.0066-7373.2004.00116.x}.

\bibitem[\citeproctext]{ref-MacFarlane2007-SMNIC}
Macfarlane, John. 2007. {``Semantic Minimalism and Nonindexical
Contextualism.''} In \emph{Context-Sensitivity and Semantic Minimalism:
New Essays on Semantics and Pragmatics}, edited by Gerhard Preyer and
Georg Peter, 240--50. Oxford University Press.

\bibitem[\citeproctext]{ref-MacFarlane2009-MACNC}
MacFarlane, John. 2009. {``{Nonindexical Contextualism}.''}
\emph{Synthese} 166 (2): 231--50. doi:
\href{https://doi.org/10.1007/s11229-007-9286-2}{10.1007/s11229-007-9286-2}.

\bibitem[\citeproctext]{ref-McGee1985}
McGee, Vann. 1985. {``A Counterexample to Modus Ponens.''} \emph{Journal
of Philosophy} 82 (9): 462--71. doi:
\href{https://doi.org/10.2307/2026276}{10.2307/2026276}.

\bibitem[\citeproctext]{ref-McKay2006}
McKay, Thomas. 2006. \emph{Plural Predication}. Oxford: Oxford
University Press.

\bibitem[\citeproctext]{ref-Nolan2003}
Nolan, Daniel. 2003. {``{Defending a Possible-Worlds Account of
Indicative Conditionals}.''} \emph{Philosophical Studies} 116 (3):
215--69. doi:
\href{https://doi.org/10.1023/B:PHIL.0000007243.60727.d4}{10.1023/B:PHIL.0000007243.60727.d4}.

\bibitem[\citeproctext]{ref-Partee1989}
Partee, Barbara. 1989. {``Binding Implicit Variables in Quantified
Contexts.''} In \emph{Papers from the Twenty-Fifth Regional Meeting of
the Chicago Linguistic Society}, edited by Caroline Wiltshire, Randolph
Graczyk, and Bradley Music. Chicago: Chicago Linguistic Society.
Reprinted in\cite{Partee2004}.

\bibitem[\citeproctext]{ref-SayreMcCord1991}
Sayre-McCord, Geoffrey. 1991. {``Being a Realist about Relativism (in
Ethics).''} \emph{Philosophical Studies} 61 (1-2): 155--76. doi:
\href{https://doi.org/10.1007/bf00385839}{10.1007/bf00385839}.

\bibitem[\citeproctext]{ref-Schlenker2003}
Schlenker, Philippe. 2003. {``Indexicality, Logophoricity, and Plural
Pronouns.''} In \emph{Afroasiatic Grammar II: Selected Papers from the
Fifth Conference on Afroasiatic Languages, Paris, 2000}, edited by
Jacqueline Lecarme, 409--28. Amsterdam: John Benjamins.

\bibitem[\citeproctext]{ref-Stalnaker1975-STAIC}
Stalnaker, Robert. 1975. {``{Indicative Conditionals}.''}
\emph{Philosophia} 5 (3): 269--86. doi:
\href{https://doi.org/10.1007/bf02379021}{10.1007/bf02379021}.

\bibitem[\citeproctext]{ref-Stalnaker1981}
---------. 1981. {``A Defence of Conditional Excluded Middle.''} In
\emph{Ifs}, edited by William Harper, Robert C. Stalnaker, and Glenn
Pearce, 87--104. Dordrecht: Reidel.

\bibitem[\citeproctext]{ref-Stanley2007-STALIC}
Stanley, Jason. 2007. \emph{{Language in Context: Selected Essays}}.
Oxford University Press.

\bibitem[\citeproctext]{ref-Stephenson2007}
Stephenson, Tamina. 2007. {``Judge Dependence, Epistemic Modals, and
Predicates of Personal Taste.''} \emph{Linguistics and Philosophy} 30
(4): 487--525. doi:
\href{https://doi.org/10.1007/s10988-008-9023-4}{10.1007/s10988-008-9023-4}.

\bibitem[\citeproctext]{ref-Weatherson2001-WEAIAS}
Weatherson, Brian. 2001. {``{Indicative and Subjunctive
Conditionals}.''} \emph{The Philosophical Quarterly} 51 (203): 200--216.
doi:
\href{https://doi.org/10.1111/j.0031-8094.2001.00224.x}{10.1111/j.0031-8094.2001.00224.x}.

\end{CSLReferences}



\noindent Published in\emph{
Synthese}, 2009, pp. 333-357.

\end{document}
