% Options for packages loaded elsewhere
% Options for packages loaded elsewhere
\PassOptionsToPackage{unicode}{hyperref}
\PassOptionsToPackage{hyphens}{url}
%
\documentclass[
  11pt,
  twoside]{ergoclass}
\usepackage{xcolor}
\usepackage[twoside=true, headsep=.25in, headheight=1in, footskip=.35in,
paperwidth=7in, paperheight=10in, top=1in, bottom=1in, inner=.8in,
outer=.8in]{geometry}
\usepackage{amsmath,amssymb}
\setcounter{secnumdepth}{3}
\usepackage{iftex}
\ifPDFTeX
  \usepackage[T1]{fontenc}
  \usepackage[utf8]{inputenc}
  \usepackage{textcomp} % provide euro and other symbols
\else % if luatex or xetex
  \usepackage{unicode-math} % this also loads fontspec
  \defaultfontfeatures{Scale=MatchLowercase}
  \defaultfontfeatures[\rmfamily]{Ligatures=TeX,Scale=1}
\fi
\usepackage[]{mathpazo}
\ifPDFTeX\else
  % xetex/luatex font selection
  \setmathfont[]{Garamond-Math}
\fi
% Use upquote if available, for straight quotes in verbatim environments
\IfFileExists{upquote.sty}{\usepackage{upquote}}{}
\IfFileExists{microtype.sty}{% use microtype if available
  \usepackage[]{microtype}
  \UseMicrotypeSet[protrusion]{basicmath} % disable protrusion for tt fonts
}{}
\usepackage{setspace}
% Make \paragraph and \subparagraph free-standing
\makeatletter
\ifx\paragraph\undefined\else
  \let\oldparagraph\paragraph
  \renewcommand{\paragraph}{
    \@ifstar
      \xxxParagraphStar
      \xxxParagraphNoStar
  }
  \newcommand{\xxxParagraphStar}[1]{\oldparagraph*{#1}\mbox{}}
  \newcommand{\xxxParagraphNoStar}[1]{\oldparagraph{#1}\mbox{}}
\fi
\ifx\subparagraph\undefined\else
  \let\oldsubparagraph\subparagraph
  \renewcommand{\subparagraph}{
    \@ifstar
      \xxxSubParagraphStar
      \xxxSubParagraphNoStar
  }
  \newcommand{\xxxSubParagraphStar}[1]{\oldsubparagraph*{#1}\mbox{}}
  \newcommand{\xxxSubParagraphNoStar}[1]{\oldsubparagraph{#1}\mbox{}}
\fi
\makeatother


\usepackage{longtable,booktabs,array}
\usepackage{calc} % for calculating minipage widths
% Correct order of tables after \paragraph or \subparagraph
\usepackage{etoolbox}
\makeatletter
\patchcmd\longtable{\par}{\if@noskipsec\mbox{}\fi\par}{}{}
\makeatother
% Allow footnotes in longtable head/foot
\IfFileExists{footnotehyper.sty}{\usepackage{footnotehyper}}{\usepackage{footnote}}
\makesavenoteenv{longtable}
\usepackage{graphicx}
\makeatletter
\newsavebox\pandoc@box
\newcommand*\pandocbounded[1]{% scales image to fit in text height/width
  \sbox\pandoc@box{#1}%
  \Gscale@div\@tempa{\textheight}{\dimexpr\ht\pandoc@box+\dp\pandoc@box\relax}%
  \Gscale@div\@tempb{\linewidth}{\wd\pandoc@box}%
  \ifdim\@tempb\p@<\@tempa\p@\let\@tempa\@tempb\fi% select the smaller of both
  \ifdim\@tempa\p@<\p@\scalebox{\@tempa}{\usebox\pandoc@box}%
  \else\usebox{\pandoc@box}%
  \fi%
}
% Set default figure placement to htbp
\def\fps@figure{htbp}
\makeatother





\setlength{\emergencystretch}{3em} % prevent overfull lines

\providecommand{\tightlist}{%
  \setlength{\itemsep}{0pt}\setlength{\parskip}{0pt}}



 
\usepackage[]{natbib}
\bibliographystyle{apalike}


% Additional packages can be loaded here if needed
% The ergoclass already loads most necessary packages

% Uncomment if you need these:
% \usepackage{booktabs}
% \usepackage{longtable}
\fancyhfoffset{0pt}
\makeatletter
\@ifpackageloaded{float}{}{\usepackage{float}}
\floatstyle{plain}
\@ifundefined{c@chapter}{\newfloat{apptbl}{h}{loapptbl}}{\newfloat{apptbl}{h}{loapptbl}[chapter]}
\floatname{apptbl}{Table A}
\newcommand*\quartoapptblref[1]{Table \hyperref[#1]{A\ref{#1}}}
\@ifpackageloaded{caption}{}{\usepackage{caption}}
\DeclareCaptionLabelFormat{quartoapptblreflabelformat}{#1#2}
\captionsetup[apptbl]{labelformat=quartoapptblreflabelformat}
\newcommand*\listofapptbls{\listof{apptbl}{List of Table As}}
\makeatother
\makeatletter
\@ifpackageloaded{caption}{}{\usepackage{caption}}
\AtBeginDocument{%
\ifdefined\contentsname
  \renewcommand*\contentsname{Table of contents}
\else
  \newcommand\contentsname{Table of contents}
\fi
\ifdefined\listfigurename
  \renewcommand*\listfigurename{List of Figures}
\else
  \newcommand\listfigurename{List of Figures}
\fi
\ifdefined\listtablename
  \renewcommand*\listtablename{List of Tables}
\else
  \newcommand\listtablename{List of Tables}
\fi
\ifdefined\figurename
  \renewcommand*\figurename{Figure}
\else
  \newcommand\figurename{Figure}
\fi
\ifdefined\tablename
  \renewcommand*\tablename{Table}
\else
  \newcommand\tablename{Table}
\fi
}
\@ifpackageloaded{float}{}{\usepackage{float}}
\floatstyle{ruled}
\@ifundefined{c@chapter}{\newfloat{codelisting}{h}{lop}}{\newfloat{codelisting}{h}{lop}[chapter]}
\floatname{codelisting}{Listing}
\newcommand*\listoflistings{\listof{codelisting}{List of Listings}}
\makeatother
\makeatletter
\makeatother
\makeatletter
\@ifpackageloaded{caption}{}{\usepackage{caption}}
\@ifpackageloaded{subcaption}{}{\usepackage{subcaption}}
\makeatother
\usepackage{bookmark}
\IfFileExists{xurl.sty}{\usepackage{xurl}}{} % add URL line breaks if available
\urlstyle{same}
\hypersetup{
  pdftitle={When are Philosophy Articles Cited?},
  pdfauthor={Anon},
  hidelinks,
  pdfcreator={LaTeX via pandoc}}


\title{When are Philosophy Articles Cited?}

\author{Anon}
\affiliation{Anon Institution}
\contact{anon@institution.edu}


\textofabstract{%
It's natural to believe that philosophy citations are typically to long
ago pieces. We're still talking about philosophers from millenia ago.
More strikingly, we're still talking about papers from half a century
ago not as historical papers, but as part of the contemporary debate.
But a systematic look at the citation data shows that these cases are
outliers. Most citations are to recently published works. Surprisingly,
this is less true in recent years than it used to be. The effect of
electronic publishing and communication has been to make citations, on
average, older. After we adjust for the typical age of philosophy
citations, and this changing trend, it turns out that the 2000s were a
particularly influential time in philosophy publishing. Articles
published in that decade are cited more than earlier or later articles,
once we adjust for the typical times articles are cited, and the
changing patterns of citation. This is arguably related to broad changes
in the interests of philosophers, towards social philosophy, and
epistemology.
}%

\articledoi{https://doi.org/TBC}

\volumeissueyear{TBC}{TBC}{2025}%

\setcounter{page}{1}%

\begin{document}
\maketitle

\setstretch{1.1}
\section{Introduction}\label{sec-introduction}

This paper examines citation patterns of philosophy journal articles.
Philosophy journals obviously cite more than just other philosophy
journals, and philosophy articles get cited beyond journals. But
examining journal-to-journal citations provides a relatively complete
dataset for systematic generalization about how articles are cited over
time. Some of these generalizations are surprising.

Before examining the data, I held two beliefs about philosophy
citations. First, philosophers cite very old papers. We still regularly
teach papers over half a century old in introductory classes:
\citet{WOSA1969Y444700002}, \citet{WOSA1971Y116900003},
\citet{WOSA1972Z066400001}, and \citet{10.2307_2025310}. These aren't
taught as history but as contributions to contemporary debate. I thought
this pattern extended to less famous papers. Second, technological
changes of the last quarter century were reversing this practice.
Innovations---email, preprint archives (arXiv, SSRN, PhilPapers), and
official preprints like EarlyView---made it easier to cite newer works.
The delay between publication and wide recognition was removed, so
citations should be getting younger.

Both beliefs were wrong.

On the first point, my generalization from famous papers was mistaken.
Normal papers differ from famous ones not just in citation frequency but
in citation patterns. The main evidence is the \emph{citation ratio},
which measures how often articles from year \emph{o} (old) are cited in
year \emph{n}, adjusted for total citations in year \emph{n}. (Full
explanation in Section~\ref{sec-age}.)
Figure~\ref{fig-master-citation-ratio} shows average citation ratios for
different citation \emph{ages}---years between \emph{o} and
\emph{n}.\footnote{The graph includes jitter for visibility. Each decade
  of publication has a different color (broken out in
  Figure~\ref{fig-decades-cite-ratio}). The graph starts in 1975 because
  earlier data is noisier, for reasons discussed below.}

\begin{figure}

\centering{

\pandocbounded{\includegraphics[keepaspectratio]{apc-nov-2025_files/figure-pdf/fig-master-citation-ratio-1.pdf}}

}

\caption{\label{fig-master-citation-ratio}Age effects from 1975 onwards,
with overall average shown.}

\end{figure}%

Each dot represents a citation ratio for a year pair; the line shows the
average for each age. The pattern is clear: articles are cited far more
when young than when old.

My initial `evidence' wasn't entirely wrong. Redoing
Figure~\ref{fig-master-citation-ratio} for articles with 15+ citations
yields Figure~\ref{fig-ageeffecteverything-high}. (This captures a small
percentage of articles but a substantial percentage of citations.)

\begin{figure}

\centering{

\pandocbounded{\includegraphics[keepaspectratio]{apc-nov-2025_files/figure-pdf/fig-ageeffecteverything-high-1.pdf}}

}

\caption{\label{fig-ageeffecteverything-high}Citation ratios for highly
cited articles}

\end{figure}%

The y-axis values in Figure~\ref{fig-ageeffecteverything-high} are
higher than in Figure~\ref{fig-master-citation-ratio}---unsurprisingly,
highly cited articles get cited more frequently. What's striking is the
shape difference. Typical articles, if cited at all, are cited soon
after publication then fade. Highly cited articles continue being cited
decades later.

These results aren't obvious---things could have differed. Some articles
were ignored initially but accumulated five to ten citations decades
later. Others were frequently cited early but are now largely ignored
(particularly in philosophy of science and philosophy of mind, for
different reasons in each case). These are outliers; they could have
been typical but aren't. Most articles influential early remain so.

For the second point, we can break
Figure~\ref{fig-master-citation-ratio} into ten-year chunks.
Figure~\ref{fig-decades-cite-ratio} groups points from
Figure~\ref{fig-master-citation-ratio} into `decades' (1975-1984,
1985-1994, etc., given 1975-2024 data). For easier comparison, I removed
the incomplete final decade and points with ages over 20.

\begin{figure}

\begin{minipage}{0.50\linewidth}

\centering{

\pandocbounded{\includegraphics[keepaspectratio]{apc-nov-2025_files/figure-pdf/fig-decades-cite-ratio-1.pdf}}

}

\subcaption{\label{fig-decades-cite-ratio-1}1975-1984}

\end{minipage}%
%
\begin{minipage}{0.50\linewidth}

\centering{

\pandocbounded{\includegraphics[keepaspectratio]{apc-nov-2025_files/figure-pdf/fig-decades-cite-ratio-2.pdf}}

}

\subcaption{\label{fig-decades-cite-ratio-2}1985-1994}

\end{minipage}%
\newline
\begin{minipage}{0.50\linewidth}

\centering{

\pandocbounded{\includegraphics[keepaspectratio]{apc-nov-2025_files/figure-pdf/fig-decades-cite-ratio-3.pdf}}

}

\subcaption{\label{fig-decades-cite-ratio-3}1995-2004}

\end{minipage}%
%
\begin{minipage}{0.50\linewidth}

\centering{

\pandocbounded{\includegraphics[keepaspectratio]{apc-nov-2025_files/figure-pdf/fig-decades-cite-ratio-4.pdf}}

}

\subcaption{\label{fig-decades-cite-ratio-4}2005-2014}

\end{minipage}%

\caption{\label{fig-decades-cite-ratio}Citation ratios for different
decades}

\end{figure}%

Three trends emerge in Figure~\ref{fig-decades-cite-ratio}, especially
after the second graph:

\begin{enumerate}
\def\labelenumi{\arabic{enumi}.}
\tightlist
\item
  Peaks arrive later. In early graphs, the line declines by age 5; in
  the last, it barely falls.
\item
  Peaks are lower. The last graph barely crosses 1.
\item
  Declines are flatter. At age 15, values rise steadily over time.
\end{enumerate}

Citations are getting older. While articles from a given year are still
cited more at ages 2-5 than 12-15, this difference has fallen markedly.
Technology's effect on citations has been opposite to my expectation.

This paper has two aims.

First, I explain the methodology behind these graphs and defend my
choices. The conclusion: these graphs show that citations were
traditionally to very recent articles but are now more frequently to
older ones.

Second, I examine which years have been most influential, after
adjusting for typical citation rates. The early 1970s stand
out---unsurprisingly. More surprising: the 2000s are the next most
influential period. Several factors contribute, but the rising
importance of epistemology is likely primary (as \citet{Petrovich2024}
also found with different data). More generally, examining citation
patterns illuminates philosophical history. Most work on analytic
philosophy's history stops at the early 1970s; this is an early attempt
to quantify subsequent developments following Kripke, Lewis, Rawls, and
others.

\section{Age of Citations}\label{sec-age-of-citations}

\subsection{Methodology}\label{sec-methodology}

The data comes from Web of Science (WoS). This section explains which
data I used and how I assembled it.

Most data comes from XML files WoS makes available to subscribing
institutions. My institution's subscription (which provided data through
2021) has lapsed, so post-2021 data comes from the WoS
website.\footnote{Also via institutional subscription; the XML is more
  expensive.}

The XML file is large---over a terabyte decompressed. To make it
manageable, I filtered to \emph{articles} (excluding discussion notes,
book reviews, editorial matters) categorized as Philosophy or History \&
Philosophy of Science. I then hand-selected the hundred journals with
the most inbound citations that were (a) primarily English language, (b)
not primarily history of science, and (c) broadly `analytic' rather than
`continental'. These choices were somewhat subjective, but yielded a
reasonable collection of journals important for understanding recent
anglophone analytic philosophy.

The journal list and basic statistics appear in
Section~\ref{sec-statistics}. For these journals, I included all
articles and notes/reviews over 15 pages. I did not restrict to pieces
labeled Philosophy or History \& Philosophy of Science---for
interdisciplinary journals like \emph{Mind and Language}, these labels
were unreliable, and I wanted a complete picture.

I supplemented the XML data in two ways. First, WoS does not index
\emph{The Journal of Philosophy} from 1971-1974. Other journals are also
missing in 1974, but this was the longest and most significant gap. The
Journal published groundbreaking articles during this period by
\citet{Frankfurt1971}, \citet{Boolos1971}, \citet{Benacerraf1973},
\citet{Kim1973}, \citet{Friedman1974}, \citet{Levi1974}, and Lewis
\citetext{\citeyear{Lewis1971cen}; \citeyear{Lewis1973ben}}. Omitting
these would undermine the analysis. I used JSTOR to compile a complete
article list (excluding notes and book reviews) for \emph{Journal of
Philosophy} in these years, then searched citations in
Table~\ref{tbl-list-of-journals} for references to them. This meant
using different article/non-article classifications, with some odd
results.\footnote{The JSTOR list excluded the symposium on Kenneth
  Arrow's ``Some Ordinalist-Utilitarian Notes on Rawls's Theory of
  Justice''; I'm unsure why.} It also required substantial data
cleaning.\footnote{Much of this involved sorting through varied
  spellings of Brian O'Shaughnessy's name.} Despite efforts at
consistency, some early-1970s discontinuities may stem from this data
acquisition difference.

Tables in Section~\ref{sec-introduction} start in 1975 partly from
consistency concerns about dual-method data compilation, but mainly
because WoS begins indexing \emph{Analysis} only in 1975. Without
\emph{Analysis}---especially papers on knowledge analysis and
inferentialism---the picture of citation patterns in those years is
incomplete. I include 1956-1974 in some analyses below, but that data is
less complete and less useful for identifying trends.

Second, my XML data extends only through mid-2022. I downloaded all
articles and citations from these 100 journals for 2021-2024 from the
WoS website, processing them with the bibliometrix package
\citep{bibliometrix}. Using 2021 data as a check, the methods yielded
similar results---differences were under 1\% for article counts and
slightly over 1\% for citations. The match isn't perfect but is close
enough that I used 2022-2024 data from WoS via bibliometrix.

\subsection{Journal to Journal}\label{sec-journal-to-journal}

This study examines one citation type: philosophy journal articles
citing other philosophy journal articles. This excludes much---edited
volumes, theses, conference programs, books, and citations in adjacent
fields.

These restrictions have three justifications.

First, journal-to-journal data is cleaner. When WoS records one indexed
article citing another, the citation record includes the cited article's
WoS ID. This eliminates cleaning errors in citation details. Authors
commonly cite incorrect page numbers and, less frequently but often
enough to require checking, incorrect titles, author names (especially
hard-to-spell names), or publication years. Cleaning this is
labor-intensive. Restricting to cases with WoS IDs doesn't avoid this
problem so much as delegate it to WoS. \citet{Petrovich2024} examined
all citations in five leading philosophy journals; covering more than
five wasn't practical given the cleaning required. I sacrifice some
comprehensiveness but cover twenty times more journals. Neither approach
is wrong---examining different things, the studies complement each
other.

Second, journals enable comprehensiveness. Determining average citation
rates for philosophy books from a given year would require a database of
all books---perhaps possible via the Library of Congress, but
challenging. For edited volume chapters, I don't know where to start.
Journals number their issues; confirming completeness is
straightforward.

Third, using whole journals makes demarcating philosophy more
manageable. I can show what I mean by philosophy journals: those in
Table~\ref{tbl-list-of-journals}. Going book-by-book or
chapter-by-chapter would be massive work for both compilation and
verification. I'm not quantifying philosophy articles in journals, but
articles in philosophy journals. The demarcation problem remains
non-trivial (should I include \emph{Cognition}?), and some boundaries
are inevitably arbitrary. But this approach involves fewer such
boundaries, and they're more transparent.

Restricting to philosophy-journal-to-philosophy-journal citations has
two major downsides.

For inbound citations, books and journal articles differ in what they
cite, creating different field impressions. History of philosophy
involves less journal publishing, and published articles cite primary
sources and recent books more than journal articles. This work offers
minimal insight into developments in history of philosophy.
Also---important later---books are cited at much older ages than
articles. \citet{Petrovich2024} notes that through the 1990s, Quine,
Wittgenstein, and Davidson were among the most cited authors. None
appear near the top examining only cited journal articles.

Davidson raises another issue about journal article citations. A
citation registers as being to a journal article only if the journal is
identified---ideally by name, though DOI works---in the citing article.
Older works often cite famous articles by mentioning reprint
collections. Someone citing ``Actions, Reasons, and Causes'' who gives
only the bibliographic detail that it's chapter one of \emph{Essays on
Actions and Events} won't necessarily create a journal-to-journal
citation in WoS. Most articles aren't reprinted, and currently people
cite originals as well as or instead of reprints. Overall this isn't a
large effect, but for finding most-cited articles, it's a major error
source.\footnote{I planned to study which articles had the largest
  citation declines as a measure of changing philosophical fashion. But
  most articles I found with large falls had been reprinted so often
  that this effect explained most of the pattern. It's not a large
  effect overall, but searching for outliers mostly finds unreliable
  data.}

For outbound citations, this study doesn't show how often journals are
cited outside philosophy. It also doesn't show citations in books, but
that's less problematic---book and journal citation patterns are
similar. However, citations inside philosophy poorly predict citations
outside. In Table~\ref{tbl-list-of-journals}, \emph{Journal of Medical
Ethics} articles are collectively rarely cited. This reflects my
excluding medical journals where that journal is cited more often. The
data tells you something: if you want confirmation that `core'
philosophy journals don't publish much bioethics, \emph{Journal of
Medical Ethics} citation numbers are evidence. But they're not evidence
about the journal's overall impact---we're not looking in the right
place.

\subsection{Age, Period, and Cohort}\label{sec-apc}

To understand citation patterns, I'll borrow terminology common in
sociology and medicine. An example best introduces it. Imagine we
observe interesting patterns among teenagers in the late 1960s and
wonder about explanations. Two pattern types immediately suggest
themselves, along with tests.

First, the behavior could reflect being teenagers---an \textbf{age
effect}. Test: see if similar patterns appear with teenagers at other
times.

Second, the behavior could reflect the 1960s, when many striking things
happened---a \textbf{period effect}. Test: see if non-teenagers in the
1960s show the same pattern.

A third explanation is important. These people were born in the early
1950s---the post-war baby boom. Colloquially, they're boomers. Perhaps
that explains the pattern---a \textbf{cohort effect}. Test: examine the
same people at other life stages.

Cohort effects are easily overlooked; sometimes they resemble age
effects. \citet{GhitzaEtAl2023} argue that many hypotheses about voting
age effects (e.g., older people being naturally conservative) are really
cohort effects. \citet{Bump2023} argues that understanding boomers'
distinctive role is crucial for understanding modern American life.

Mathematical reasons also complicate separating these effects. Many
statistical techniques for separating influences fail when linear
correlations exist between variable combinations. Here the correlation
is maximal: by definition, cohort plus age equals period. Some
workarounds exist---see \citet{KeyesEtAl2010} for options and
\citet{Rohrer2025} for recent skepticism about general solutions---but
the challenge remains.

Conceptually, separating these effects is difficult when effect strength
changes over time. As noted initially, testing which effect is strongest
naturally involves examining other times. This works when age effects
are constant. When they're not (as may be true here), it's harder.

However, keeping these three effects in mind helps summarize the data:

\begin{itemize}
\tightlist
\item
  The \textbf{age effect}: articles are cited most at ages two to five
  years.
\item
  The \textbf{period effect}: citations are far more numerous in recent
  years. Partly this reflects more published articles; partly it
  reflects citations per article growing substantially over the
  2000s-2010s and exploding in the 2020s.
\item
  The \textbf{cohort effect}: articles from the 1970s and 2000s are
  cited more than expected given age and period effects, while articles
  from other times---especially before 1965 and around 1990---are cited
  less. The reasons are more complicated; I return to them below.
\end{itemize}

The period effect is largest and, in some ways, least interesting. I'll
start by quantifying it and arguing for screening it off.

\section{Period Effects}\label{sec-period}

The database contains 509,373 citations, unevenly distributed over
time---they grow rapidly. In 1956, there are only 4 citations
(unsurprisingly, without preprint citations, few articles cite pieces
from that year). By 2024, there are 38,516.
Figure~\ref{fig-citationsperyear} shows this growth.

\begin{figure}

\centering{

\pandocbounded{\includegraphics[keepaspectratio]{apc-nov-2025_files/figure-pdf/fig-citationsperyear-1.pdf}}

}

\caption{\label{fig-citationsperyear}Citations in the dataset by year.}

\end{figure}%

As noted in Section~\ref{sec-methodology}, I used different extraction
methods from 2022 onward. The 2021-2022 drop might reflect this change,
but I don't think so. First, 2021 is likely an outlier: \emph{Synthese}
published 1,439 articles in 2021 but only 509 in 2022. Second, applying
the 2022-2024 method to 2020-2021 yielded close agreement (within 1-2\%)
for each year.

What explains this dramatic growth through 2021? Partly, more articles
are being published and indexed. Figure~\ref{fig-articlesperyear} shows
article counts by year.

\begin{figure}

\centering{

\pandocbounded{\includegraphics[keepaspectratio]{apc-nov-2025_files/figure-pdf/fig-articlesperyear-1.pdf}}

}

\caption{\label{fig-articlesperyear}Articles in the dataset by
publication year.}

\end{figure}%

This explains some growth, but not all---the curve in
Figure~\ref{fig-articlesperyear} isn't nearly as steep as in
Figure~\ref{fig-citationsperyear}. Citations per article are also
rising. Figure~\ref{fig-outboundcitations} plots average citations to
other dataset articles each year.

\begin{figure}

\centering{

\pandocbounded{\includegraphics[keepaspectratio]{apc-nov-2025_files/figure-pdf/fig-outboundcitations-1.pdf}}

}

\caption{\label{fig-outboundcitations}Average citations to indexed
articles by year.}

\end{figure}%

Several factors explain this graph's shape.

At the left edge, boundary effects are obvious. Since we count only
citations to articles published since 1956, articles in the 1950s
naturally have few citations. Since articles rarely get unpublished,
more articles become available to cite each year.

This can't explain the massive jumps at
Figure~\ref{fig-outboundcitations}'s right edge. That jump reflects
converging cultural trends: simply more citations overall (casual
journal perusal confirms this), and more journal citations versus book
or edited volume citations.

Sharp jumps like this warrant data checking. Cross-checking every entry
is impractical, but spot-checks look correct. The change appears led by
prestigious journals. For each journal, I calculated average outbound
citations (to these hundred journals) for the 2010s and 2020-2024. The
ten journals with the largest increases appear in
Table~\ref{tbl-large-growth}.

\newpage

\begin{longtable}[]{@{}
  >{\raggedright\arraybackslash}p{(\linewidth - 6\tabcolsep) * \real{0.5694}}
  >{\raggedleft\arraybackslash}p{(\linewidth - 6\tabcolsep) * \real{0.1389}}
  >{\raggedleft\arraybackslash}p{(\linewidth - 6\tabcolsep) * \real{0.1389}}
  >{\raggedleft\arraybackslash}p{(\linewidth - 6\tabcolsep) * \real{0.1528}}@{}}

\caption{\label{tbl-large-growth}Mean outbound citations for selected
journals over two decades.}

\tabularnewline

\toprule\noalign{}
\begin{minipage}[b]{\linewidth}\raggedright
Journal
\end{minipage} & \begin{minipage}[b]{\linewidth}\raggedleft
2010-2019
\end{minipage} & \begin{minipage}[b]{\linewidth}\raggedleft
2020-2024
\end{minipage} & \begin{minipage}[b]{\linewidth}\raggedleft
Difference
\end{minipage} \\
\midrule\noalign{}
\endhead
\bottomrule\noalign{}
\endlastfoot
Philosophical Review & 14.8 & 25.9 & 11.1 \\
Philosophical Perspectives & 11.3 & 19.2 & 7.9 \\
Noûs & 11.5 & 18.4 & 6.9 \\
Philosophy and Phenomenological Research & 9.6 & 15.8 & 6.2 \\
Philosophical Studies & 9.0 & 14.6 & 5.6 \\
Journal of Philosophy & 9.0 & 14.5 & 5.6 \\
Philosophy & 4.0 & 8.9 & 4.9 \\
Episteme & 8.1 & 12.9 & 4.9 \\
Philosophical Quarterly & 8.8 & 13.6 & 4.7 \\
Philosophy Compass & 11.2 & 15.9 & 4.7 \\

\end{longtable}

\emph{Philosophical Review} publishes only 10-12 articles annually, so
high variation is expected. Still, the 2010s change isn't just
small-sample variation. Of its 22 articles in 2020-2021, only one
\citep{WOS000575210400003} had fewer than 14.8 outbound citations. With
just 22 articles, anything could happen, but having all but one exceed
the historical average by chance would be surprising.

We could simply ask what proportion of citations accrue to articles in a
given year, but that would overcorrect. The 2020s have more citations to
distribute, but also more articles sharing them. We must adjust for
both. Here's how.

An article is \emph{typically cited} if published 3-10 years before the
citing year. As Figure~\ref{fig-master-citation-ratio} showed, citations
typically peak then. Using this definition,
Figure~\ref{fig-articlecounts} shows typically cited article counts at
any given time (for 2000, it shows articles published 1990-1997).

\begin{figure}

\centering{

\pandocbounded{\includegraphics[keepaspectratio]{apc-nov-2025_files/figure-pdf/fig-articlecounts-1.pdf}}

}

\caption{\label{fig-articlecounts}Typically cited articles by year.}

\end{figure}%

Figure~\ref{fig-citationcounts} shows how often these `typical' articles
are cited each year; Figure~\ref{fig-citationrate} shows mean citations
to typical articles per year.

\begin{figure}

\centering{

\pandocbounded{\includegraphics[keepaspectratio]{apc-nov-2025_files/figure-pdf/fig-citationcounts-1.pdf}}

}

\caption{\label{fig-citationcounts}Citations to typical articles by
year.}

\end{figure}%

\begin{figure}

\centering{

\pandocbounded{\includegraphics[keepaspectratio]{apc-nov-2025_files/figure-pdf/fig-citationrate-1.pdf}}

}

\caption{\label{fig-citationrate}Mean annual citations to typical
articles.}

\end{figure}%

Two things stand out in Figure~\ref{fig-citationrate}. First, the graph
is flat for a long time---from mid-1970s to early-2000s, it bounces
without much movement. Then it takes off, peaks in 2021, and returns to
the long-term trend. Second, the numbers are never high. Throughout most
of this study, even peak-age articles (3-10 years old) are cited once
per \emph{decade} in these hundred journals. Since citation rates are
extremely long-tailed with means well above medians, this overstates how
often the `average article' gets cited. Frequent citation is decidedly
not the norm.\footnote{Long-term, average citations per article equals
  average times an article is cited, so most articles having just a
  handful of philosophy journal citations is unsurprising.}

My initial measure of article influence is citation frequency divided by
typical article citation frequency. This is somewhat arbitrary---I could
have chosen ranges other than 3-10 years, but this works reasonably
well. I tried other measures; they produced either implausible data
trends or implausible judgments about paper influence. This measure had
a nice property: how influential the leading 50 articles from a period
were 10-20 years later was reasonably stable, suggesting it corrects for
period effects well.

\section{Age Effects}\label{sec-age}

Next, I determine how article age affects citation frequency. The
simplest approach---examining a typical year to see how its articles are
cited over time---would be completely wrong.
Figure~\ref{fig-1990-outbound-citations} shows citation patterns for
1990 articles.

\begin{figure}

\centering{

\pandocbounded{\includegraphics[keepaspectratio]{apc-nov-2025_files/figure-pdf/fig-1990-outbound-citations-1.pdf}}

}

\caption{\label{fig-1990-outbound-citations}Citations to 1990-published
articles.}

\end{figure}%

Using citations as influence measures,
Figure~\ref{fig-1990-outbound-citations} suggests 1990 articles were
collectively most influential in 2021. That's not true---they were most
influential 2-4 years post-publication, like most articles. The 2020s
simply published so many articles, each citing so many pieces, that even
three-decade-old articles get lifted by the rising tide.

A more intuitive influence measure uses typical articles from
Section~\ref{sec-period}. Adjust
Figure~\ref{fig-1990-outbound-citations} by dividing each value by two
things: first, the typical article citation rate from
Figure~\ref{fig-citationrate} (adjusting for period effects); second,
the number of 1990-published articles (yielding per-article influence).
The result is the \textbf{citation ratio} from
Figure~\ref{fig-master-citation-ratio}.
Figure~\ref{fig-1990-outbound-citations-norm} shows 1990's citation
ratio.

\begin{figure}

\centering{

\pandocbounded{\includegraphics[keepaspectratio]{apc-nov-2025_files/figure-pdf/fig-1990-outbound-citations-norm-1.pdf}}

}

\caption{\label{fig-1990-outbound-citations-norm}Normalized citations to
1990-published articles.}

\end{figure}%

Two reasons support Figure~\ref{fig-1990-outbound-citations-norm} as a
more plausible influence measure than
Figure~\ref{fig-1990-outbound-citations}. One appeals to intuition: I
know 1990 work came up in discussions far more in the 1990s than 2020s.
While such intuitive evidence deserves some weight, it's obviously
unreliable alone. The better reason: we get very similar graphs no
matter which initial year we pick. This was visible in
Figure~\ref{fig-decades-cite-ratio}, but it's worth seeing how stable
this is.

An explicit citation ratio definition helps. Let \emph{c}(\emph{o},
\emph{n}) be citations of year-\emph{o} articles (old year) in year
\emph{n} (new year). Let \emph{a}(\emph{o}) be articles published in
year \emph{o}. Then citation ratio \emph{r}(\emph{o}, \emph{n}) is:

\[
r(o, n) = \left(\frac{c(o, n)}{a(o)}\right) / \left(\frac{\sum\limits_{i = n-10}^{n-3}c(i, n)}{\sum\limits_{i = n-10}^{n-3}a(i)}\right)
\]

In Figure~\ref{fig-ageeffecttibble-early} and
Figure~\ref{fig-ageeffecttibble-late}, each facet represents different
\emph{o} values, the x-axis is \emph{n}, and the y-axis is
\emph{r}(\emph{o}, \emph{n}). The key observation: these graphs are
remarkably steady. I've cheated slightly---showing earlier years would
reveal different shapes. 1960s citations are so sparse that noise
overwhelms signal. Since then, patterns are reasonably steady.

\begin{figure}

\centering{

\pandocbounded{\includegraphics[keepaspectratio]{apc-nov-2025_files/figure-pdf/fig-ageeffecttibble-early-1.pdf}}

}

\caption{\label{fig-ageeffecttibble-early}Citation rates for articles
published 1968-1992.}

\end{figure}%

\begin{figure}

\centering{

\pandocbounded{\includegraphics[keepaspectratio]{apc-nov-2025_files/figure-pdf/fig-ageeffecttibble-late-1.pdf}}

}

\caption{\label{fig-ageeffecttibble-late}Citation rates for articles
published 1993-2017.}

\end{figure}%

\section{Cohort Effects}\label{sec-cohort}

Period and age effects together explain much of the citation pattern
trends. But systematic deviations remain. Figure~\ref{fig-twodeviations}
shows some examples---each graph is a facet from
Figure~\ref{fig-ageeffecttibble-early} with a line showing average age
effects.

\begin{figure}

\begin{minipage}{0.50\linewidth}

\centering{

\pandocbounded{\includegraphics[keepaspectratio]{apc-nov-2025_files/figure-pdf/fig-twodeviations-1.pdf}}

}

\subcaption{\label{fig-twodeviations-1}1979}

\end{minipage}%
%
\begin{minipage}{0.50\linewidth}

\centering{

\pandocbounded{\includegraphics[keepaspectratio]{apc-nov-2025_files/figure-pdf/fig-twodeviations-2.pdf}}

}

\subcaption{\label{fig-twodeviations-2}1985}

\end{minipage}%

\caption{\label{fig-twodeviations}Citations from a given year compared
to average citations.}

\end{figure}%

In 1979, yearly values predominantly exceed the mean line; in 1987,
they're largely below it. My main cohort effect measure is the average
ratio between yearly data (dots) and average values (line) for each such
graph. 1979 dots average about 13\% above the mean; 1987, about 9\%
below. Repeating this for every dataset year yields
Figure~\ref{fig-cohort}.

\begin{figure}

\centering{

\pandocbounded{\includegraphics[keepaspectratio]{apc-nov-2025_files/figure-pdf/fig-cohort-1.pdf}}

}

\caption{\label{fig-cohort}Cohort effects by publication year.}

\end{figure}%

Technical notes on Figure~\ref{fig-cohort}: I added a rolling average
line (four years either side, or as many as available) to highlight
features. In calculating means, I included only years with at least five
years of data for calculating mean age effects. So I haven't included
what happens when 1956 papers are cited after 2019---insufficient data
exists to determine `expected' aging curves at those points.

Five periods appear in the graph:

\begin{enumerate}
\def\labelenumi{\arabic{enumi}.}
\tightlist
\item
  Pre-mid-1960s journal articles are very rarely cited.
\item
  After that, especially in the early 1970s, many highly cited articles
  appear.
\item
  A stagnation period follows---things mostly don't return to pre-1965
  lows but stay consistently below zero.
\item
  An uptick begins mid-1990s, peaking dramatically in 2007.
\item
  A dramatic dropoff follows almost immediately after 2007's high.
\end{enumerate}

The first two trends make sense; the latter three less so. The remainder
of this paper explains what's happening and what it reveals about
philosophy's history and the philosophy profession's history.

Before 1965, philosophy's most significant work wasn't in WoS-indexed
journals---partly because books were more important, partly because WoS
indexes are incomplete for early years. We lack ``Is Knowledge Justified
True Belief?'' \citep{Gettier1963} because \emph{Analysis} indexing
starts only in 1975, and Austin's major papers---``Ifs and Cans'' and
``A Plea for Excuses'' \citep{Austin1956, Austin1956b}---because their
venues aren't indexed as journals. We do have important papers by
\citet{WOSA1956CHJ4400001}, \citet{WOSA1956CEQ2500001},
\citet{WOSA1957CGZ6000005}, \citet{WOSA1958CDL1000001},
\citet{WOSA1959CGZ6600001}, and \citet{WOSA1963CEU0700001}, but these
had less impact than contemporary books, especially \emph{Intention}
\citep{Anscombe1957}, \emph{Word and Object} \citep{Quine1960}, and
\emph{The Structure of Scientific Revolutions} \citep{Kuhn1962}.

Then, starting in the late 1960s, nearly every philosophy area was
transformed, with much action in journals. The period's two most
important works---\emph{A Theory of Justice} \citep{Rawls1971} and
\emph{Naming and Necessity} \citep{Kripke1980}---weren't journal
articles. But journal articles did revolutionize many fields, including:

\begin{itemize}
\tightlist
\item
  Free will \citep{WOSA1969Y444700002, 10.2307_2024717};
\item
  Practical ethics \citep{WOSA1971Y116900003, WOSA1972Z066400001};
\item
  Meaning and reference \citep{10.2307_2025079};
\item
  Philosophy of mathematics \citep{10.2307_2025075};
\item
  Causation \citep{10.2307_2025310, 10.2307_2025096}; and
\item
  Personal identity \citep{WOSA1971Y036400001}
\end{itemize}

Additionally, surprisingly many papers became influential later rather
than immediately, including work by \citet{WOSA1970ZE33800001},
\citet{WOSA1970ZE32700001}, \citet{WOSA1970Y384700002}, and
\citet{WOSA1973P242100001}. Figure~\ref{fig-cohort}'s early-1970s story
is directionally plausible, though magnitude is hard to confirm given
spotty data. Before the late 1960s-early 1970s, philosophy journals had
never published such high-quality work in this quantity.

To understand subsequent developments, we must examine data more
closely. Figure~\ref{fig-cohort-short} is Figure~\ref{fig-cohort}
restricted to the first seven post-publication years---for each
publication year, it measures citations of that year's articles during
their first seven years, adjusted identically for age and period
effects. I start in 1975 to exclude noisy early data.

\begin{figure}

\centering{

\pandocbounded{\includegraphics[keepaspectratio]{apc-nov-2025_files/figure-pdf/fig-cohort-short-1.pdf}}

}

\caption{\label{fig-cohort-short}Short-term citation rates (first seven
years post-publication).}

\end{figure}%

In Figure~\ref{fig-cohort}, 2010 articles are cited slightly more than
1990 articles (after adjustments). But in Figure~\ref{fig-cohort-short},
they're cited 20\% less. Generally, nearly every 1980s-1990s article
batch shows solid first-seven-year citation rates. So
Figure~\ref{fig-cohort-short} lacks Figure~\ref{fig-cohort}'s 1980s dip.
This contrasts strikingly with Figure~\ref{fig-cohort-long}, measuring
only post-seven-year citations.

\begin{figure}

\centering{

\pandocbounded{\includegraphics[keepaspectratio]{apc-nov-2025_files/figure-pdf/fig-cohort-long-1.pdf}}

}

\caption{\label{fig-cohort-long}Long-term citation rates (after the
first seven years).}

\end{figure}%

In Figure~\ref{fig-cohort-long}, every 2000-2016 year averages higher
than every 1980-1999 year. Starting around 1998, a large change occurs
in how often articles over seven years old are cited.

This explains Figure~\ref{fig-cohort}'s right-edge dropoff---the fifth
period I described. Citations are aging, but for mid-2010s articles, we
lack data on citations ten or more years post-publication. Averaging
their first-few-year citations compared to a generation ago's
first-few-year citations underestimates their influence.

This doesn't explain everything. Figure~\ref{fig-cohort-long} appears to
stop rising in the 2010s. But it explains much. 1980s-1990s articles
were heavily cited soon after publication, but their citation rates
didn't hold up like later articles. 2000s articles were cited less
initially than late-20th-century articles but far more as time passed.
This might also apply to 2010s articles, but it's too early to
determine.

Even accepting this, questions remain.

Why is this temporal citation shift occurring? Shouldn't technology
shift things oppositely? I discuss this in Section~\ref{sec-technology}.

Why are citations rising so much generally, even accounting for
increased article publication? Section~\ref{sec-technology} partially
explains this, but Section~\ref{sec-culture} discusses two cultural
factors.

Finally, why do periods around 1990 and 2005 stand out? Around 1990,
Figure~\ref{fig-cohort-short} shows an upward spike and
Figure~\ref{fig-cohort-long} a low point. Around 2005, both graphs
exceed long-term trends. The answers partly involve technological
factors (Section~\ref{sec-technology}) and cultural factors
(Section~\ref{sec-culture}) but also reflect important changes in
philosophically central topics. This hints at an important
twentieth/twenty-first-century philosophy discontinuity, which I address
in Section~\ref{sec-content}.

\section{Technology and Citations}\label{sec-technology}

A common view holds that electronic publication primarily speeds
\emph{distribution}. The data doesn't support this. If true, we'd expect
short-term citations---especially very short-term---to rise over time.

By the late twentieth century, printing and postage were mature
technologies. We weren't awaiting steam ships to deliver journal issues
to distant shores. Philosophy journal distribution used the same
technology as medicine and other time-sensitive fields. From this
perspective, the internet would accelerate distribution by weeks or
months at most---barely visible on yearly graphs. Postal improvements,
especially increased airmail use, may affect 1980s-1990s citation
numbers---more citations to foreign journals soon after publication than
in earlier decades, for example.

But Online First, Early View, and similar quasi-publication forms
haven't made much difference. They appear slightly in the
data---occasionally articles are cited before official publication---but
such cases are rare.

Technology's biggest effect was on \emph{search}, not distribution.
Before widespread computer use, searching books was far easier than
searching journal articles. Classification systems placed books near
others on the same topic. Card catalogs listed book subjects. Even book
titles helped locate topics. Finding relevant journal articles was much
harder and, it seems, rarely done.

Physical storage and access also differed notably between books and
journals. Nearly every academic has a bookshelf; far fewer have large
journal collections. Departments occasionally kept physical journals,
but accessing them often meant walking across campus to the library.
Accessing a book might mean walking four steps to a shelf. This physical
difference likely contributed to books' and articles' relative
prominence in bibliographies.

One exception existed to this access pattern (at least in any
twentieth-century department I knew, though I think this was
widespread): Departments sometimes kept latest journal issues in a
department library or common room---much more prominent and accessible.
Whether this explains why pre-1995 journal citations are so often to
very recent journals is unclear, but it probably helped.

Search technology changes appear central to the story. Before widespread
computer adoption, old journal articles were very hard to find. This
changed somewhat with electronic, easily searchable \emph{Philosophers'
Index} versions, and dramatically when journals went online. This partly
explains why older articles---especially non-classic older
articles---are now more widely cited.

\section{Culture Changes}\label{sec-culture}

Two cultural changes significantly affected citation patterns: one
concerning journal articles' role, another concerning citation norms.

Looking back at century-old journals, pieces sometimes resemble blog
posts more than current journal articles. Even substantive pieces feel
like means to an end---the journal article as much a book trial run as a
complete project report.

By the twenty-first century, this changed completely. Articles are
longer, even in venues like \emph{Analysis}. More importantly, they're
often finished products, not draft runs for future books. Prominent
philosophers' reputations rest largely or entirely on articles (e.g.,
Jonathan Schaffer). Large philosophy fields---epistemic contextualism,
metaphysical grounding---have canonical texts that are almost entirely
articles, not books. This isn't entirely new (post-Gettier knowledge
analysis literature was largely article-based), but it's a growing
trend. This partially explains aging citations: fields based in articles
generate more older-article citations, especially when those articles
haven't been superseded by books.\footnote{Articles once widely cited
  but rarely cited now are heavily populated by those forming the basis
  for widely cited books.}

The bigger trend is philosophers' increasing tendency to include brief
citations to work they're not discussing in detail but which locate the
paper in a literature. Obviously if an article cites 26 other journal
articles---as Table~\ref{tbl-large-growth} showed the average recent
\emph{Philosophical Review} paper does---it can't possibly discuss all
in depth. (This count excludes books and edited volume articles.) For
long, a striking philosophy-adjacent-disciplines difference was the lack
of citations largely serving to place a paper in a literature. Their
growth partially explains aging citations: authors citing their
subfield's history include some older articles.

Why did this change occur? Technology is part of the story
(Section~\ref{sec-technology})---it's now easier. But this can't be
everything, since the practice was more widespread in other fields
before these technologies arrived. Specialization growth is another
part. Writing about content externalism in the 1990s didn't require a
citation trail back to Kripke and Putnam---everyone knew these debates'
history; it didn't need rehashing. Since then, it's become less clear
that any field's history is universally known, creating more background
need.

But the biggest explanation part is likely interdisciplinary work growth
in the 2000s. This appears in many fields, including experimental
philosophy's rise and increasing philosophy-of-mind empirical
sophistication. I'll focus on one other aspect: philosophy of language's
move toward debates also active in linguistics. Part of what we see in
the 2000s is citation norms adoption from linguistics into philosophy of
language. Philosophy of language wasn't as important to journals in the
2000s as earlier decades.\footnote{I won't argue this here---it would
  require nearly another paper and more data sources. Short version:
  nothing in twenty-first-century philosophy of language was as crucial
  to journals as debates over names and descriptions and over wide and
  narrow content had been in earlier decades.} But it was an important
vector for citation norms spreading from linguistics to philosophy,
especially since much philosophy-of-language work overlapped with that
decade's key epistemology debate: contextualism. I'll end this paper
examining how changes in what philosophers discussed interact with
citation data.

\section{Content Changes}\label{sec-content}

Philosophy publishing's center of gravity shifts over our time period.
Through at least the early 2000s, analytic philosophy is in what
\citet[2]{Sider2020} calls the ``modal era''. One aspect of this era
that Sider highlights is that essence questions were equated with
necessity questions in ways they weren't before or after.\footnote{During
  this era, the necessity-of-origins thesis and origin essentialism
  thesis were typically taken as not just mutually supporting but
  literally identical---an identity claim not widely endorsed before
  1970 or after 2010.} This should be taken as era symptom, not
definition. What's really defining was how modality became central to
disputes across the discipline.

Consider, for example, what \citet{Jackson1998} called the `location
problem'---how to locate in the world something the philosopher thinks
exists and is not fundamental. Jackson argues that saying how to locate
the non-fundamental in the fundamental is a compulsory question for
anyone doing `serious metaphysics', with the one and only answer
involving modality. As he says,

\begin{quote}
When does a putative feature of our world have a place in the account
some serious metaphysics tells of what our world is like? I have already
mentioned one answer: if the feature is entailed by the account told in
the terms favoured by the metaphysics in question, it has a place in the
account told in the favoured terms. This is hardly controversial
considered as a sufficient condition, but, I will now argue, it is also
a necessary condition: the one and only way of having a place in an
account told in some set of preferred terms is by being entailed by that
account---a view I will refer to as the entry by entailment thesis.
\citep[ 5]{Jackson1998}
\end{quote}

Jackson went on to say other things about entailment not widely
endorsed. But at this early book stage, he was largely expressing
conventional wisdom. In a review disagreeing with many book parts,
\citet[20]{Yablo2000Jackson} says ``Not many eyebrows will be raised by
Jackson's view that metaphysics is committed to `entry by entailment'
theses.'' The quoted parts aren't controversial, especially the one
Jackson flags as ever so slightly more controversial.

The idea that entailment---i.e., necessitation---had been central to
understanding how the non-fundamental relates to the fundamental was
central to philosophy for many years by this point. We can see how
central using a slightly different statistic: grand-citations.

The number of grand-citations an article \emph{a} has is the number of
triples ⟨\emph{a}, \emph{b}, \emph{c}⟩ such that \emph{c} cites \emph{b}
and \emph{b} cites \emph{a}---the sum of citations of articles citing
\emph{a}. Grand-citations over time show David Lewis's centrality to
philosophy journals: five of the six articles with the most
grand-citations are by Lewis. Looking at particular times shows
journals' changing face. Grand-citations take time to accrue, so I'll
examine twenty-year periods---for various years, which articles
published in the preceding twenty years had the most grand-citations
through that year.

Table~\ref{tbl-grand-cite-2000} lists articles published from 1980
onward with the most grand-citations through 2000.


\begin{longtable}[]{@{}
  >{\raggedleft\arraybackslash}p{(\linewidth - 6\tabcolsep) * \real{0.0413}}
  >{\raggedright\arraybackslash}p{(\linewidth - 6\tabcolsep) * \real{0.8099}}
  >{\raggedleft\arraybackslash}p{(\linewidth - 6\tabcolsep) * \real{0.0496}}
  >{\raggedleft\arraybackslash}p{(\linewidth - 6\tabcolsep) * \real{0.0992}}@{}}

\caption{\label{tbl-grand-cite-2000}1980s-1990s articles with the most
grand-citations by 2000.}

\tabularnewline

\toprule\noalign{}
\begin{minipage}[b]{\linewidth}\raggedleft
Rank
\end{minipage} & \begin{minipage}[b]{\linewidth}\raggedright
Article
\end{minipage} & \begin{minipage}[b]{\linewidth}\raggedleft
Cites
\end{minipage} & \begin{minipage}[b]{\linewidth}\raggedleft
Grand-Cites
\end{minipage} \\
\midrule\noalign{}
\endhead
\bottomrule\noalign{}
\endlastfoot
1 & Terrence Horgan
\citeyearpar{WOSA1982NN35300003}
``Supervenience and Microphysics'' & 36 & 318 \\
2 & Tyler Burge
\citeyearpar{WOSA1986AYX3200001}
``Individualism and Psychology'' & 82 & 316 \\
3 & David Lewis
\citeyearpar{WOSA1983RR51600001}
``New Work for a Theory of Universals'' & 86 & 308 \\
4 & Paul M. Churchland
\citeyearpar{WOSA1981LD54600001}
``Eliminative Materialism and the Propositional Attitudes'' & 94 &
299 \\
5 & John Haugeland
\citeyearpar{WOSA1982NC42600008}
``Weak Supervenience'' & 40 & 258 \\
6 & Jaegwon Kim
\citeyearpar{WOSA1982NC90700004}
``Psychophysical Supervenience'' & 40 & 245 \\
7 & Ruth Garrett Millikan
\citeyearpar{WOSA1989AA09400006}
``In Defense of Proper Functions'' & 43 & 232 \\
8 & Jon Barwise and Robin Cooper
\citeyearpar{WOSA1981LH67300001}
``Generalized Quantifiers and Natural-Language'' & 83 & 221 \\
9 & John Bigelow and Robert Pargetter
\citeyearpar{WOSA1987G947600001}
``Functions'' & 30 & 220 \\
10 & Jaegwon Kim
\citeyearpar{WOSA1984TV24600001}
``Concepts of Supervenience'' & 87 & 219 \\

\end{longtable}

I've included article names to show how central supervenience was to
this literature.\footnote{The relationship story between
  twentieth-century work on functions and twenty-first-century work on
  mechanisms is interesting but for another time.} Four articles have
`supervenience' in the title! Jaegwon Kim stood at this literature's
center. My citation data somewhat \emph{underestimates} his
influence---people often cited his edited collection \emph{Supervenience
and Mind} \citep{Kim1993}, usually not tracked by WoS.

These articles' subsequent history largely explains
Figure~\ref{fig-cohort}'s relatively low 1980s values. Much of this
supervenience work has fallen out of philosophical discourse.
Table~\ref{tbl-grand-cite-2000-now} shows how many citations
Table~\ref{tbl-grand-cite-2000} articles have since 2021.


\begin{longtable}[]{@{}lr@{}}

\caption{\label{tbl-grand-cite-2000-now}Citations of
Table~\ref{tbl-grand-cite-2000} articles since 2021.}

\tabularnewline

\toprule\noalign{}
Article & Citations since 2021 \\
\midrule\noalign{}
\endhead
\bottomrule\noalign{}
\endlastfoot
\citet{WOSA1982NN35300003}
& 3 \\
\citet{WOSA1986AYX3200001}
& 15 \\
\citet{WOSA1983RR51600001}
& 271 \\
\citet{WOSA1981LD54600001}
& 50 \\
\citet{WOSA1982NC42600008}
& 2 \\
\citet{WOSA1982NC90700004}
& 6 \\
\citet{WOSA1989AA09400006}
& 58 \\
\citet{WOSA1981LH67300001}
& 43 \\
\citet{WOSA1987G947600001}
& 24 \\
\citet{WOSA1984TV24600001}
& 10 \\

\end{longtable}

Supervenience articles simply aren't cited much these days. The same
pattern recurs examining other nearly-as-widely-discussed supervenience
papers (e.g., \citet{WOSA1989T680600002} and \citet{WOSA1984ST78300010})
and papers from the debate family on wide and narrow content, mental
individualism, empty names, and hallucinations---all using supervenience
notions to define debates. Articles by \citet{WOSA1986AYX3200002},
\citet{WOSA1988P549200004}, \citet{WOSA1989T680600002},
\citet{WOSA1983RU36600003}, \citet{WOSA1991FF02900001}, and
\citet{WOSA1991EN62900001}, all very influential then, have been barely
cited since 2021. Even some Lewis papers on these topics (e.g.,
\citet{WOSA1981MS19500002} and \citet{WOSA1983PZ01000001}) have just a
handful of recent citations. A debate family central to philosophy for
years---particularly central to what \emph{Philosophical Review}
published in the 1980s---has simply dropped off the agenda. This hugely
impacts citation numbers.

Moving into the 2000s, focus shifts dramatically, as
Table~\ref{tbl-grand-cite-2010} shows.


\begin{longtable}[]{@{}
  >{\raggedleft\arraybackslash}p{(\linewidth - 6\tabcolsep) * \real{0.0368}}
  >{\raggedright\arraybackslash}p{(\linewidth - 6\tabcolsep) * \real{0.8309}}
  >{\raggedleft\arraybackslash}p{(\linewidth - 6\tabcolsep) * \real{0.0441}}
  >{\raggedleft\arraybackslash}p{(\linewidth - 6\tabcolsep) * \real{0.0882}}@{}}

\caption{\label{tbl-grand-cite-2010}Top 10 1990s-2000s articles by
grand-citations through 2010.}

\tabularnewline

\toprule\noalign{}
\begin{minipage}[b]{\linewidth}\raggedleft
Rank
\end{minipage} & \begin{minipage}[b]{\linewidth}\raggedright
Article
\end{minipage} & \begin{minipage}[b]{\linewidth}\raggedleft
Cites
\end{minipage} & \begin{minipage}[b]{\linewidth}\raggedleft
Grand-Cites
\end{minipage} \\
\midrule\noalign{}
\endhead
\bottomrule\noalign{}
\endlastfoot
1 & David Lewis
\citeyearpar{WOSA1996VY21200001}
``Elusive Knowledge'' & 182 & 665 \\
2 & Keith DeRose
\citeyearpar{WOSA1995RC31600001}
``Solving the Skeptical Problem'' & 145 & 604 \\
3 & Stephen Yablo
\citeyearpar{WOSA1992JA62400001}
``Mental Causation'' & 126 & 572 \\
4 & Keith DeRose
\citeyearpar{WOSA1991GL32100002}
``Epistemic Possibilities'' & 40 & 519 \\
5 & Tyler Burge
\citeyearpar{WOSA1993ML38000001}
``Content Preservation'' & 136 & 502 \\
6 & Karen Neander
\citeyearpar{WOSA1991FQ15000002}
``Functions as Selected Effects: The Conceptual Analyst's Defense'' & 89
& 480 \\
7 & Keith DeRose
\citeyearpar{WOSA1992KB29500008}
``Contextualism and Knowledge Attributions'' & 79 & 430 \\
8 & Mark Johnston
\citeyearpar{WOSA1992KC39800002}
``How To Speak of the Colors'' & 93 & 423 \\
9 & C. B. Martin
\citeyearpar{WOSA1994MT56900001}
``Dispositions and Conditionals'' & 82 & 401 \\
10 & Michael B. Burke
\citeyearpar{WOSA1992HC13100003}
``Copper Statues and Pieces of Copper: A Challenge To the Standard
Account'' & 45 & 395 \\

\end{longtable}

The biggest single topic was epistemology contextualism, with major
papers by \citet{WOSA1996VY21200001} and \citet{WOSA1995RC31600001} at
the center. What I want to focus on particularly, though, is another
DeRose paper on that list: ``Epistemic Possibilities.'' It has fewer
cites than any other paper there but the fourth-most grand-cites. This
is partly because it was important to both epistemology and philosophy
of language. But another reason is that the debate it triggered was one
of the first places where linguistics citation norms were applied in
philosophy. This is part of the evidence for my
Section~\ref{sec-culture} claims about changing norms' importance to
citation rates.

Moving to the 2010s, focus shifts again, as
Table~\ref{tbl-grand-cite-2020} shows. Here we see evidence for my
earlier specialization-increase claim. The papers with the most
grand-citations address subjects like mechanisms, dogmatism,
disagreement, pragmatic encroachment, and priority monism---a much
broader topic range than was central to philosophy a few decades
earlier. Crucially, topics overlap so little that someone working in
them can't expect philosophers working in other topics to know even
debate basics. (Perhaps people working on various epistemology topics on
that list can expect slightly more background knowledge from people
working on other epistemology topics, but that's about it.) So writers
need more citations just so readers will have basic topic understanding.


\begin{longtable}[]{@{}
  >{\raggedleft\arraybackslash}p{(\linewidth - 6\tabcolsep) * \real{0.0407}}
  >{\raggedright\arraybackslash}p{(\linewidth - 6\tabcolsep) * \real{0.8130}}
  >{\raggedleft\arraybackslash}p{(\linewidth - 6\tabcolsep) * \real{0.0488}}
  >{\raggedleft\arraybackslash}p{(\linewidth - 6\tabcolsep) * \real{0.0976}}@{}}

\caption{\label{tbl-grand-cite-2020}Top 10 1990s-2000s articles by
grand-citations through 2020.}

\tabularnewline

\toprule\noalign{}
\begin{minipage}[b]{\linewidth}\raggedleft
Rank
\end{minipage} & \begin{minipage}[b]{\linewidth}\raggedright
Article
\end{minipage} & \begin{minipage}[b]{\linewidth}\raggedleft
Cites
\end{minipage} & \begin{minipage}[b]{\linewidth}\raggedleft
Grand-Cites
\end{minipage} \\
\midrule\noalign{}
\endhead
\bottomrule\noalign{}
\endlastfoot
1 & Peter Machamer, Lindley Darden, and Carl F. Craver
\citeyearpar{WOS000087305900001}
``Thinking About Mechanisms'' & 402 & 2807 \\
2 & James Pryor
\citeyearpar{WOS000165361800002}
``The Skeptic and the Dogmatist'' & 288 & 1680 \\
3 & David Lewis
\citeyearpar{WOS000089124200002}
``Causation as Influence'' & 173 & 1647 \\
4 & Jeremy Fantl and Matthew McGrath
\citeyearpar{WOS000181094500003}
``Evidence, Pragmatics, and Justification'' & 149 & 1557 \\
5 & Jonathan Schaffer
\citeyearpar{WOS000272855000002}
``Monism: The Priority of the Whole'' & 215 & 1429 \\
6 & Jason Stanley and Timothy Williamson
\citeyearpar{WOS000170277300002}
``Knowing How'' & 208 & 1418 \\
7 & Keith DeRose
\citeyearpar{WOS000184740400001}
``Assertion, Knowledge, and Context'' & 170 & 1415 \\
8 & David Christensen
\citeyearpar{WOS000207419300002}
``Epistemology of Disagreement: The Good News'' & 206 & 1387 \\
9 & Nishi Shah
\citeyearpar{WOS000224335200001}
``How Truth Governs Belief'' & 135 & 1303 \\
10 & Adam Elga
\citeyearpar{WOS000249103800005}
``Reflection and Disagreement'' & 216 & 1301 \\

\end{longtable}

\section{Conclusion}\label{conclusion}

I need to write a conclusion.

\section{Appendix: Summary Statistics}\label{sec-statistics}

The paper uses the journals shown in Table~\ref{tbl-list-of-journals}.

\begin{longtable}[]{@{}
  >{\raggedright\arraybackslash}p{(\linewidth - 10\tabcolsep) * \real{0.4274}}
  >{\raggedleft\arraybackslash}p{(\linewidth - 10\tabcolsep) * \real{0.0940}}
  >{\raggedleft\arraybackslash}p{(\linewidth - 10\tabcolsep) * \real{0.0855}}
  >{\raggedleft\arraybackslash}p{(\linewidth - 10\tabcolsep) * \real{0.0769}}
  >{\raggedleft\arraybackslash}p{(\linewidth - 10\tabcolsep) * \real{0.1624}}
  >{\raggedleft\arraybackslash}p{(\linewidth - 10\tabcolsep) * \real{0.1538}}@{}}

\caption{\label{tbl-list-of-journals}Journals used in this paper}

\tabularnewline

\toprule\noalign{}
\begin{minipage}[b]{\linewidth}\raggedright
Journal
\end{minipage} & \begin{minipage}[b]{\linewidth}\raggedleft
First Year
\end{minipage} & \begin{minipage}[b]{\linewidth}\raggedleft
Last Year
\end{minipage} & \begin{minipage}[b]{\linewidth}\raggedleft
Articles
\end{minipage} & \begin{minipage}[b]{\linewidth}\raggedleft
Outbound Citations
\end{minipage} & \begin{minipage}[b]{\linewidth}\raggedleft
Inbound Citations
\end{minipage} \\
\midrule\noalign{}
\endhead
\bottomrule\noalign{}
\endlastfoot
American Philosophical Quarterly & 1964 & 2024 & 1835 & 7759 & 10298 \\
Analysis & 1975 & 2024 & 2719 & 7494 & 14835 \\
Analytic Philosophy & 2016 & 2024 & 190 & 2219 & 501 \\
Archiv für Geschichte der Philosophie & 1975 & 2024 & 672 & 1598 &
1020 \\
Australasian Journal of Philosophy & 1975 & 2024 & 1736 & 10463 &
13449 \\
Biology and Philosophy & 1988 & 2024 & 1225 & 6179 & 4827 \\
British Journal for the History of Philosophy & 2007 & 2024 & 834 & 2465
& 1113 \\
British Journal for the Philosophy of Science & 1956 & 2024 & 1620 &
9330 & 13042 \\
British Journal of Aesthetics & 1975 & 2024 & 1436 & 3556 & 3614 \\
Bulletin of Symbolic Logic & 1997 & 2024 & 443 & 1326 & 1139 \\
Canadian Journal of Philosophy & 1975 & 2023 & 1552 & 7772 & 5737 \\
Croatian Journal of Philosophy & 2007 & 2024 & 376 & 1901 & 299 \\
Dialogue & 1975 & 2024 & 1555 & 3697 & 1107 \\
Economics and Philosophy & 1986 & 2024 & 580 & 2729 & 2417 \\
Episteme & 2005 & 2024 & 586 & 5196 & 3128 \\
Ergo & 2016 & 2024 & 386 & 5090 & 867 \\
Erkenntnis & 2000 & 2024 & 1789 & 17334 & 7645 \\
Ethical Theory and Moral Practice & 2008 & 2024 & 902 & 5875 & 2157 \\
Ethics & 1956 & 2024 & 1647 & 5758 & 15706 \\
Ethics and Information Technology & 2001 & 2024 & 567 & 1953 & 1032 \\
European Journal for Philosophy of Science & 2011 & 2024 & 564 & 5984 &
1475 \\
European Journal of Philosophy & 1998 & 2024 & 994 & 5841 & 3023 \\
Heythrop Journal & 1975 & 2024 & 1559 & 874 & 355 \\
History and Philosophy of Logic & 1992 & 2024 & 522 & 1447 & 930 \\
Hypatia & 2009 & 2024 & 683 & 1556 & 1712 \\
Inquiry & 1966 & 2024 & 1646 & 7186 & 4545 \\
International Journal for Philosophy of Religion & 1975 & 2024 & 1149 &
2184 & 1135 \\
International Philosophical Quarterly & 1961 & 2023 & 1588 & 1464 &
713 \\
Journal of Aesthetics and Art Criticism & 1975 & 2024 & 1539 & 3747 &
3757 \\
Journal of Applied Philosophy & 2006 & 2024 & 666 & 3473 & 1304 \\
Journal of Chinese Philosophy & 1973 & 2024 & 1278 & 1053 & 966 \\
Journal of Consciousness Studies & 2000 & 2024 & 1526 & 4945 & 3661 \\
Journal of Indian Philosophy & 1975 & 2024 & 1107 & 1477 & 1473 \\
Journal of Medical Ethics & 1975 & 2024 & 4347 & 5904 & 5106 \\
Journal of Moral Philosophy & 2005 & 2024 & 392 & 2449 & 980 \\
Journal of Philosophical Logic & 1972 & 2024 & 1497 & 7755 & 9812 \\
Journal of Philosophical Research & 2005 & 2024 & 463 & 2097 & 575 \\
Journal of Philosophy & 1956 & 2024 & 2761 & 7299 & 37891 \\
Journal of Political Philosophy & 1998 & 2023 & 609 & 2303 & 2891 \\
Journal of Social Philosophy & 2008 & 2024 & 508 & 2202 & 883 \\
Journal of Symbolic Logic & 1966 & 2024 & 4363 & 6757 & 10588 \\
Journal of Value Inquiry & 1980 & 2024 & 1369 & 2975 & 1466 \\
Journal of the American Philosophical Association & 2015 & 2024 & 341 &
2764 & 938 \\
Journal of the History of Ideas & 1956 & 2024 & 2212 & 995 & 1705 \\
Journal of the History of Philosophy & 1975 & 2024 & 1138 & 2800 &
3065 \\
Journal of the Philosophy of History & 2010 & 2024 & 284 & 618 & 187 \\
Kant-Studien & 1975 & 2024 & 1134 & 1710 & 1710 \\
Kantian Review & 2010 & 2024 & 338 & 1560 & 655 \\
Kennedy Institute of Ethics Journal & 1995 & 2024 & 574 & 1234 & 933 \\
Law and Philosophy & 1982 & 2024 & 852 & 2566 & 1519 \\
Linguistics and Philosophy & 1979 & 2024 & 878 & 5014 & 6408 \\
Logique et Analyse & 2007 & 2021 & 340 & 1565 & 336 \\
Metaphilosophy & 1975 & 2024 & 1562 & 4541 & 2837 \\
Mind & 1956 & 2024 & 1980 & 8454 & 18449 \\
Mind \& Language & 1994 & 2024 & 893 & 5807 & 6366 \\
Minds and Machines & 1992 & 2024 & 756 & 3442 & 1965 \\
Monist & 1963 & 2024 & 1976 & 4326 & 6448 \\
Notre Dame Journal of Formal Logic & 2009 & 2024 & 486 & 1661 & 707 \\
Noûs & 1975 & 2024 & 1480 & 11792 & 20567 \\
Pacific Philosophical Quarterly & 1980 & 2024 & 1231 & 7610 & 6544 \\
Philosophers' Imprint & 2010 & 2024 & 402 & 5177 & 3301 \\
Philosophia & 1975 & 2024 & 2249 & 10028 & 2919 \\
Philosophia Mathematica & 2008 & 2024 & 243 & 1613 & 897 \\
Philosophical Explorations & 2008 & 2024 & 390 & 2857 & 1272 \\
Philosophical Forum & 1971 & 2024 & 851 & 1726 & 613 \\
Philosophical Investigations & 1983 & 2024 & 708 & 1057 & 588 \\
Philosophical Papers & 2009 & 2023 & 234 & 1379 & 445 \\
Philosophical Perspectives & 2007 & 2023 & 305 & 3380 & 3493 \\
Philosophical Psychology & 1991 & 2024 & 1312 & 7178 & 4226 \\
Philosophical Quarterly & 1975 & 2024 & 1450 & 8660 & 10728 \\
Philosophical Review & 1956 & 2024 & 1034 & 5220 & 25891 \\
Philosophical Studies & 1956 & 2024 & 5485 & 35127 & 38247 \\
Philosophy & 1956 & 2024 & 1811 & 2452 & 3611 \\
Philosophy \& Public Affairs & 1971 & 2024 & 733 & 2592 & 11779 \\
Philosophy Compass & 2015 & 2024 & 651 & 7940 & 1750 \\
Philosophy East and West & 1966 & 2024 & 1604 & 1981 & 1758 \\
Philosophy and Phenomenological Research & 1956 & 2024 & 3273 & 15151 &
21537 \\
Philosophy and Rhetoric & 1975 & 2024 & 927 & 1170 & 893 \\
Philosophy of Science & 1956 & 2024 & 3259 & 12888 & 25003 \\
Philosophy of the Social Sciences & 1975 & 2024 & 998 & 2556 & 1708 \\
Phronesis & 1975 & 2024 & 783 & 1227 & 1620 \\
Politics, Philosophy and Economics & 2008 & 2024 & 325 & 2025 & 782 \\
Ratio & 1974 & 2024 & 1090 & 3699 & 3607 \\
Res Philosophica & 2013 & 2024 & 360 & 2176 & 737 \\
Review of Metaphysics & 1956 & 2024 & 1616 & 1499 & 2435 \\
Review of Symbolic Logic & 2008 & 2024 & 578 & 3906 & 2722 \\
Russell & 1981 & 2024 & 345 & 427 & 303 \\
Social Epistemology & 2011 & 2024 & 489 & 2381 & 1034 \\
Social Philosophy and Policy & 1983 & 2024 & 973 & 2109 & 2556 \\
South African Journal of Philosophy & 1987 & 2024 & 794 & 1868 & 650 \\
Southern Journal of Philosophy & 1976 & 2024 & 1984 & 5195 & 3057 \\
Studia Logica & 2010 & 2024 & 734 & 2418 & 1008 \\
Studies in History and Philosophy of Science & 1974 & 2024 & 1832 & 8893
& 6225 \\
Synthese & 1966 & 2024 & 7770 & 64831 & 32906 \\
Theoria & 2007 & 2024 & 459 & 3902 & 727 \\
Theory and Decision & 1970 & 2024 & 1937 & 2103 & 2285 \\
Thought & 2016 & 2022 & 219 & 1537 & 455 \\
Topoi & 1982 & 2024 & 1327 & 5554 & 2365 \\
Transactions of the Charles S. Peirce Society & 1975 & 2024 & 1181 &
1885 & 1567 \\
Utilitas & 2009 & 2024 & 391 & 2481 & 1160 \\

\end{longtable}

What I've called an \emph{article} here is anything that either (a)
marked as an article or research-article by WoS, or (b) marked as a
review, discussion, or note by WoS and is at least 15 pages long. I
needed to include (b) because some very important works (e.g.,
\citet{WOSA1963CEU0700001} and \citet{WOS000272855000002}) were not
recorded as articles by WoS.

The years here are \textbf{not} the first and last years that the
journals published, but the earliest and latest years that are in the
WoS index (as of the time I pulled the data). As mentioned in the main
text, this makes a big difference for some journals, especially
\emph{Analysis}.

The way WoS handles the `supplements' to \emph{Noûs}, i.e.,
\emph{Philosophical Perspectives} and \emph{Philosophical Issues}, is a
little uneven. Some years these are recorded as being their own thing,
i.e., with a source name of \emph{Philosophical Perspectives} or
\emph{Philosophical Issues}; and some years they are recorded as special
issues of \emph{Noûs}. When they were listed as special issues, the
citations were extremely unreliable. Some high profile articles are
recorded as having no citations until several years after publication.
The bibliographic information for the articles themselves was also
spotty. So I've manually removed all records that were listed as special
or supplementary issues of \emph{Noûs} (and similarly removed the
citations to those articles that did get tracked). What you see here are
just the standalone issues of \emph{Philosophical Perspectives}.


\bibliography{/Users/weath/Documents/quarto-articles/brian-quarto.bib,/Users/weath/Documents/citations-2025/autobib.bib}


\noindent

\noindent Draft for submission


\end{document}
