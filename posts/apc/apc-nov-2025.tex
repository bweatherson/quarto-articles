% Options for packages loaded elsewhere
% Options for packages loaded elsewhere
\PassOptionsToPackage{unicode}{hyperref}
\PassOptionsToPackage{hyphens}{url}
%
\documentclass[
  11pt,
  twoside]{ergoclass}
\usepackage{xcolor}
\usepackage[twoside=true, headsep=.25in, headheight=1in, footskip=.35in,
paperwidth=7in, paperheight=10in, top=1in, bottom=1in, inner=.8in,
outer=.8in]{geometry}
\usepackage{amsmath,amssymb}
\setcounter{secnumdepth}{3}
\usepackage{iftex}
\ifPDFTeX
  \usepackage[T1]{fontenc}
  \usepackage[utf8]{inputenc}
  \usepackage{textcomp} % provide euro and other symbols
\else % if luatex or xetex
  \usepackage{unicode-math} % this also loads fontspec
  \defaultfontfeatures{Scale=MatchLowercase}
  \defaultfontfeatures[\rmfamily]{Ligatures=TeX,Scale=1}
\fi
\usepackage[]{mathpazo}
\ifPDFTeX\else
  % xetex/luatex font selection
  \setmathfont[]{Garamond-Math}
\fi
% Use upquote if available, for straight quotes in verbatim environments
\IfFileExists{upquote.sty}{\usepackage{upquote}}{}
\IfFileExists{microtype.sty}{% use microtype if available
  \usepackage[]{microtype}
  \UseMicrotypeSet[protrusion]{basicmath} % disable protrusion for tt fonts
}{}
\usepackage{setspace}
% Make \paragraph and \subparagraph free-standing
\makeatletter
\ifx\paragraph\undefined\else
  \let\oldparagraph\paragraph
  \renewcommand{\paragraph}{
    \@ifstar
      \xxxParagraphStar
      \xxxParagraphNoStar
  }
  \newcommand{\xxxParagraphStar}[1]{\oldparagraph*{#1}\mbox{}}
  \newcommand{\xxxParagraphNoStar}[1]{\oldparagraph{#1}\mbox{}}
\fi
\ifx\subparagraph\undefined\else
  \let\oldsubparagraph\subparagraph
  \renewcommand{\subparagraph}{
    \@ifstar
      \xxxSubParagraphStar
      \xxxSubParagraphNoStar
  }
  \newcommand{\xxxSubParagraphStar}[1]{\oldsubparagraph*{#1}\mbox{}}
  \newcommand{\xxxSubParagraphNoStar}[1]{\oldsubparagraph{#1}\mbox{}}
\fi
\makeatother


\usepackage{longtable,booktabs,array}
\usepackage{calc} % for calculating minipage widths
% Correct order of tables after \paragraph or \subparagraph
\usepackage{etoolbox}
\makeatletter
\patchcmd\longtable{\par}{\if@noskipsec\mbox{}\fi\par}{}{}
\makeatother
% Allow footnotes in longtable head/foot
\IfFileExists{footnotehyper.sty}{\usepackage{footnotehyper}}{\usepackage{footnote}}
\makesavenoteenv{longtable}
\usepackage{graphicx}
\makeatletter
\newsavebox\pandoc@box
\newcommand*\pandocbounded[1]{% scales image to fit in text height/width
  \sbox\pandoc@box{#1}%
  \Gscale@div\@tempa{\textheight}{\dimexpr\ht\pandoc@box+\dp\pandoc@box\relax}%
  \Gscale@div\@tempb{\linewidth}{\wd\pandoc@box}%
  \ifdim\@tempb\p@<\@tempa\p@\let\@tempa\@tempb\fi% select the smaller of both
  \ifdim\@tempa\p@<\p@\scalebox{\@tempa}{\usebox\pandoc@box}%
  \else\usebox{\pandoc@box}%
  \fi%
}
% Set default figure placement to htbp
\def\fps@figure{htbp}
\makeatother





\setlength{\emergencystretch}{3em} % prevent overfull lines

\providecommand{\tightlist}{%
  \setlength{\itemsep}{0pt}\setlength{\parskip}{0pt}}



 
\usepackage[]{natbib}
\bibliographystyle{apalike}


% Additional packages can be loaded here if needed
% The ergoclass already loads most necessary packages

% Uncomment if you need these:
% \usepackage{booktabs}
% \usepackage{longtable}
\fancyhfoffset{0pt}
\makeatletter
\@ifpackageloaded{float}{}{\usepackage{float}}
\floatstyle{plain}
\@ifundefined{c@chapter}{\newfloat{apptbl}{h}{loapptbl}}{\newfloat{apptbl}{h}{loapptbl}[chapter]}
\floatname{apptbl}{Table A}
\newcommand*\quartoapptblref[1]{Table \hyperref[#1]{A\ref{#1}}}
\@ifpackageloaded{caption}{}{\usepackage{caption}}
\DeclareCaptionLabelFormat{quartoapptblreflabelformat}{#1#2}
\captionsetup[apptbl]{labelformat=quartoapptblreflabelformat}
\newcommand*\listofapptbls{\listof{apptbl}{List of Table As}}
\makeatother
\makeatletter
\@ifpackageloaded{caption}{}{\usepackage{caption}}
\AtBeginDocument{%
\ifdefined\contentsname
  \renewcommand*\contentsname{Table of contents}
\else
  \newcommand\contentsname{Table of contents}
\fi
\ifdefined\listfigurename
  \renewcommand*\listfigurename{List of Figures}
\else
  \newcommand\listfigurename{List of Figures}
\fi
\ifdefined\listtablename
  \renewcommand*\listtablename{List of Tables}
\else
  \newcommand\listtablename{List of Tables}
\fi
\ifdefined\figurename
  \renewcommand*\figurename{Figure}
\else
  \newcommand\figurename{Figure}
\fi
\ifdefined\tablename
  \renewcommand*\tablename{Table}
\else
  \newcommand\tablename{Table}
\fi
}
\@ifpackageloaded{float}{}{\usepackage{float}}
\floatstyle{ruled}
\@ifundefined{c@chapter}{\newfloat{codelisting}{h}{lop}}{\newfloat{codelisting}{h}{lop}[chapter]}
\floatname{codelisting}{Listing}
\newcommand*\listoflistings{\listof{codelisting}{List of Listings}}
\makeatother
\makeatletter
\makeatother
\makeatletter
\@ifpackageloaded{caption}{}{\usepackage{caption}}
\@ifpackageloaded{subcaption}{}{\usepackage{subcaption}}
\makeatother
\usepackage{bookmark}
\IfFileExists{xurl.sty}{\usepackage{xurl}}{} % add URL line breaks if available
\urlstyle{same}
\hypersetup{
  pdftitle={When are Philosophy Articles Cited?},
  pdfauthor={Anon},
  hidelinks,
  pdfcreator={LaTeX via pandoc}}


\title{When are Philosophy Articles Cited?}

\author{Anon}
\affiliation{Anon Institution}
\contact{anon@institution.edu}


\textofabstract{%
This paper analyzes citation patterns in philosophy journal articles
from 1956 to 2024, using Web of Science data from 100 philosophy
journals. The analysis reveals two surprising findings. First, contrary
to the perception that philosophers cite very old papers, most journal
citations have historically been to recently published work. The typical
article is cited most frequently when it is 2-5 years old. Second,
contrary to the expectation that technological advances would make
citations younger, the opposite has occurred: citations are getting
older. Electronic databases and online access have made older articles
more discoverable than they were in the era of card catalogs and
physical library visits.

After adjusting for both age effects (when articles are typically cited)
and period effects (the dramatic increase in total citations over time),
the data reveal significant cohort effects. Two publication periods
stand out as particularly influential: the early 1970s and the 2000s.
The early-1970s peak reflects the revolutionary transformation of
philosophy by Frankfurt, Rawls, Thomson, Kripke, Lewis, and others. The
2000s pattern is more striking. Articles from this time did not
immediately upend the field, as happened in the 1970s. Rather, they
established a series of new debates that have persisted for decades. The
continued high citation rates of 2000s articles reflect the growing
centrality of epistemology and the fact that fewer debates from the
2000s have faded from discourse compared to the dramatic decline of
1980s-1990s debates about supervenience, narrow content, and related
topics in metaphysics and philosophy of mind.
}%

\articledoi{https://doi.org/TBC}

\volumeissueyear{TBC}{TBC}{2025}%

\setcounter{page}{1}%

\begin{document}
\maketitle

\setstretch{1.1}
\section{Introduction}\label{sec-introduction}

This paper examines citation patterns of philosophy journal articles.
Philosophy journals obviously cite more than just other philosophy
journals, and philosophy articles get cited beyond journals. But
examining journal-to-journal citations provides a relatively complete
dataset for systematic generalization about how articles are cited over
time. Some of these generalizations are surprising.

Before examining the data, I held two beliefs about philosophy
citations. First, philosophers cite very old papers. We still regularly
teach papers over half a century old in introductory classes, such as
\citet{WOSA1969Y444700002}, \citet{WOSA1971Y116900003},
\citet{WOSA1972Z066400001}, and \citet{10.2307_2025310}. These aren't
taught as history but as contributions to contemporary debates. I
thought this pattern extended to less famous papers. Second,
technological changes meant that more recently published papers were
getting cited. Innovations such as email, preprint archives (arXiv,
SSRN, PhilPapers), and official preprints like EarlyView, made it easier
to cite newer works. The delay between publication and wide recognition
was removed, so citations should be getting younger.

Both beliefs were wrong.

On the first point, my generalization from famous papers was mistaken.
Normal papers differ from famous ones not just in citation frequency but
in citation patterns. The main evidence is the \emph{citation ratio},
which measures how often articles from year \emph{o} (old) are cited in
year \emph{n}, adjusted for total citations in year \emph{n}. (This will
be fully explained in Section~\ref{sec-age}.)
Figure~\ref{fig-master-citation-ratio} shows average citation ratios for
different citation \emph{ages}; i.e., the difference in years between
\emph{o} and \emph{n}.\footnote{The graph includes jitter for
  visibility. Each decade of publication has a different color. The
  graph starts in 1975 because earlier data is noisier, for reasons
  discussed below.}

\begin{figure}

\centering{

\pandocbounded{\includegraphics[keepaspectratio]{apc-nov-2025_files/figure-pdf/fig-master-citation-ratio-1.pdf}}

}

\caption{\label{fig-master-citation-ratio}Age effects from 1975 onwards,
with overall average shown.}

\end{figure}%

Each dot represents a citation ratio for a year pair; the line shows the
average for each age. The pattern is clear: articles are cited far more
when young than when old.

My initial `evidence' wasn't entirely wrong. Redoing
Figure~\ref{fig-master-citation-ratio} for articles with 15+ citations
yields Figure~\ref{fig-ageeffecteverything-high}. (This captures a small
percentage of articles but a substantial percentage of citations.)

\begin{figure}

\centering{

\pandocbounded{\includegraphics[keepaspectratio]{apc-nov-2025_files/figure-pdf/fig-ageeffecteverything-high-1.pdf}}

}

\caption{\label{fig-ageeffecteverything-high}Citation ratios for highly
cited articles}

\end{figure}%

The y-axis values in Figure~\ref{fig-ageeffecteverything-high} are
higher than in Figure~\ref{fig-master-citation-ratio}. No surprise
there; highly cited articles get cited more frequently. What's striking
is the difference in shape. Typical articles, if cited at all, are cited
soon after publication then fade. Highly cited articles continue being
cited decades later.

These results aren't obvious; a different outcome is easy to imagine.
There could have been more articles that were ignored initially but
accumulated five to ten citations decades later. There are some articles
that were frequently cited early but are now largely ignored,
particularly in philosophy of science and philosophy of mind. There
could have been more such articles. But in fact they are outliers. By
far the usual pattern is that articles with few lifetime citations get
them quickly, and articles with many lifetime citations get them over a
long time.

To see how citations are changing, we can break
Figure~\ref{fig-master-citation-ratio} into ten-year chunks.
Figure~\ref{fig-decades-cite-ratio-early} and
Figure~\ref{fig-decades-cite-ratio-late} groups points from
Figure~\ref{fig-master-citation-ratio} into `decades' (1975-1984,
1985-1994, etc., given 1975-2024 data). For easier comparison, I removed
the incomplete final decade and points with ages over 20.

\begin{figure}

\begin{minipage}{\linewidth}

\centering{

\pandocbounded{\includegraphics[keepaspectratio]{apc-nov-2025_files/figure-pdf/fig-decades-cite-ratio-early-1.pdf}}

}

\subcaption{\label{fig-decades-cite-ratio-early-1}1975-1984}

\end{minipage}%
\newline
\begin{minipage}{\linewidth}

\centering{

\pandocbounded{\includegraphics[keepaspectratio]{apc-nov-2025_files/figure-pdf/fig-decades-cite-ratio-early-2.pdf}}

}

\subcaption{\label{fig-decades-cite-ratio-early-2}1985-1994}

\end{minipage}%

\caption{\label{fig-decades-cite-ratio-early}Citation ratios for
different decades}

\end{figure}%

Three trends emerge in those graphs, especially in
Figure~\ref{fig-decades-cite-ratio-late}:

\begin{figure}

\begin{minipage}{\linewidth}

\centering{

\pandocbounded{\includegraphics[keepaspectratio]{apc-nov-2025_files/figure-pdf/fig-decades-cite-ratio-late-1.pdf}}

}

\subcaption{\label{fig-decades-cite-ratio-late-1}1995-2004}

\end{minipage}%
\newline
\begin{minipage}{\linewidth}

\centering{

\pandocbounded{\includegraphics[keepaspectratio]{apc-nov-2025_files/figure-pdf/fig-decades-cite-ratio-late-2.pdf}}

}

\subcaption{\label{fig-decades-cite-ratio-late-2}2005-2015}

\end{minipage}%

\caption{\label{fig-decades-cite-ratio-late}Citation ratios for
different decades}

\end{figure}%

\begin{enumerate}
\def\labelenumi{\arabic{enumi}.}
\tightlist
\item
  Peaks arrive later. In early graphs, the line declines by age 5; in
  the last, it barely falls.
\item
  Peaks are lower. The last graph barely crosses 1.
\item
  Declines are flatter. The second graph in
  Figure~\ref{fig-decades-cite-ratio-late} has a slow and stable
  decline, unlike the steeper, curved decline in the other graphs.
\end{enumerate}

Overall, citations are getting older. While articles from a given year
are still cited more at ages 2-5 than 12-15, the difference between
these rates has fallen markedly. Technology's effect on citations has
been opposite to what I expected.

This paper has two aims. First, I explain the methodology behind these
graphs and defend my choices. Second, I examine which years have been
most influential, after adjusting for typical citation rates. Two
periods stand out: the early 1970s, and the 2000s. The first of these is
not surprising; the second might be more so. Several factors contribute
to the increase in citations to articles from the 2000s, but the rising
importance of epistemology is probably important (as Eugenio
\citet{Petrovich2024} also found with different data). More generally,
examining citation patterns illuminates philosophical history. Most work
on analytic philosophy's history stops at the early 1970s; this is an
early attempt to quantify subsequent developments after the big changes
brought about by Kripke, Lewis, Rawls, Thomson, and others.

\section{Age of Citations}\label{sec-age-of-citations}

\subsection{Methodology}\label{sec-methodology}

The data comes from Web of Science (WoS). This section explains which
data I used and how I assembled it.

Most data comes from XML files WoS makes available to subscribing
institutions. My institution's subscription (which provided data through
2021) has lapsed, so post-2021 data comes from the WoS
website.\footnote{This is also via institutional subscription; the XML
  is more expensive.}

The XML file is large; it's over a terabyte decompressed. To make it
manageable, I filtered to \emph{articles} (excluding discussion notes,
book reviews, editorial matters) categorized as Philosophy or History \&
Philosophy of Science. I then hand-selected the hundred journals with
the most inbound citations that were (a) primarily English language, (b)
not primarily history of science, and (c) broadly `analytic' rather than
`continental'. These choices were somewhat subjective, but yielded a
reasonable collection of journals important for understanding recent
anglophone analytic philosophy.

The journal list and basic statistics appear in
Table~\ref{tbl-list-of-journals} in Section~\ref{sec-statistics}. For
these journals, I included all articles and notes/reviews over 15 pages.
I used all articles (and longer notes/reviews) in the journals, not just
those labeled Philosophy or History \& Philosophy of Science. For
interdisciplinary journals like \emph{Mind and Language}, these labels
were unreliable, and I wanted a complete picture.

I supplemented the XML data in two ways. First, WoS does not index
\emph{The Journal of Philosophy} from 1971-1974. Other journals are also
missing in 1974, but this was the longest and most significant gap. The
Journal published groundbreaking articles during this period by
\citet{Frankfurt1971}, \citet{Boolos1971}, \citet{10.2307_2025075},
\citet{Kim1973}, \citet{Friedman1974}, \citet{Levi1974}, and Lewis
\citetext{\citeyear{Lewis1971cen}; \citeyear{Lewis1973ben}}. Omitting
these would undermine the analysis. I used JSTOR to compile a complete
article list (excluding notes and book reviews) for \emph{Journal of
Philosophy} in these years, then searched citations in
Table~\ref{tbl-list-of-journals} for references to them. This meant
using JSTOR's article/non-article classifications rather than WoS's,
with some odd results.\footnote{The JSTOR list excluded the symposium on
  Kenneth Arrow's ``Some Ordinalist-Utilitarian Notes on Rawls's Theory
  of Justice''; I'm unsure why.} It also required substantial data
cleaning.\footnote{Much of this involved sorting through varied
  spellings of Brian O'Shaughnessy's name.} Despite efforts at
consistency, some early-1970s discontinuities may stem from this data
acquisition difference.

Tables in Section~\ref{sec-introduction} start in 1975 partly from
consistency concerns about dual-method data compilation, but mainly
because WoS begins indexing \emph{Analysis} only in 1975. Without
\emph{Analysis}, especially without papers on the analysis of knowledge
and on inferentialism, the picture of citation patterns in those years
is incomplete. I include 1956-1974 in some analyses below, but that data
is less complete and less useful for identifying trends.

Second, my XML data extends only through mid-2022, and rather than using
half years I only used it through 2021. I downloaded all articles and
citations from these 100 journals for 2021-2024 from the WoS website,
processing them with the bibliometrix package \citep{bibliometrix}.
Using 2021 data as a check, the methods yielded similar results.
Differences were under 1\% for article counts and slightly over 1\% for
citations. The match isn't perfect but is close enough that I used
2022-2024 data from WoS via bibliometrix.

\subsection{Journal to Journal}\label{sec-journal-to-journal}

This study examines one citation type: philosophy journal articles
citing other philosophy journal articles. This excludes much: edited
volumes, theses, conference programs, books, and citations in adjacent
fields.

These restrictions have three justifications.

First, journal-to-journal data is cleaner. When WoS records one indexed
article citing another, the citation record includes the cited article's
WoS ID. This eliminates cleaning errors in citation details. Authors
commonly cite incorrect page numbers and, less frequently but often
enough to require checking, incorrect titles, author names (especially
hard-to-spell names), or publication years. Cleaning this is
labor-intensive. Restricting to cases with WoS IDs doesn't avoid this
problem so much as delegate it to WoS. \citet{Petrovich2024} examined
all citations in five leading philosophy journals; covering more than
five wasn't practical given the cleaning required. I sacrifice some
comprehensiveness but cover twenty times more journals. Neither approach
is wrong; examining different things, the studies complement each other.

Second, journals enable comprehensiveness. Determining average citation
rates for philosophy books from a given year would require a database of
all books. That's perhaps possible via the Library of Congress, but
challenging. For edited volume chapters, I don't know where to start.
Journals number their issues; confirming completeness is
straightforward.

Third, using whole journals makes demarcating philosophy more
manageable. I can show what I mean by philosophy journals: those in
Table~\ref{tbl-list-of-journals}. Going book-by-book or
chapter-by-chapter to ask ``Is this philosophy?'' would be impractical
for me to do, and impossible for anyone else to verify. So I'm not
quantifying philosophy articles in journals, but articles in philosophy
journals. The demarcation problem remains non-trivial (should I include
\emph{Cognition}?), and some boundaries are inevitably arbitrary. But
this approach involves fewer such boundaries, and they're more
transparent.

Restricting to philosophy-journal-to-philosophy-journal citations has
two major downsides.

For inbound citations, the relative roles of books and journals differs
between fields of philosophy, and just looking at articles will distort
some comparisons between those fields. In particular, history of
philosophy involves less journal publishing, and those articles that are
published in journals cite primary sources and recent books and chapters
more than journal articles. This essay offers minimal insight into
developments in history of philosophy. Also, books are cited at much
older ages than articles. \citet{Petrovich2024} notes that through the
1990s, books by Quine, Wittgenstein, and Davidson were among the most
cited works. Nothing by these authors appears so near the top when we
look at what articles are cited.

Davidson's work raises another issue about journal article citations. A
citation registers as being to a journal article only if the journal is
identified in the citing article. Older works often cite famous articles
by mentioning reprint collections. Someone citing ``Actions, Reasons,
and Causes'' who gives only the bibliographic detail that it's chapter
one of \emph{Essays on Actions and Events} won't necessarily create a
journal-to-journal citation in WoS. Most articles aren't reprinted, and
currently people cite originals as well as or instead of reprints.
Overall this isn't a large effect, but it makes a big difference when
looking for extreme cases, like which articles have been most
cited.\footnote{I planned to study which articles had the largest
  citation declines as a measure of changing philosophical fashion. But
  most articles I found with large falls had been reprinted so often
  that this effect explained most of the pattern. It's not a large
  effect overall, but searching for outliers mostly finds unreliable
  data.}

For outbound citations, this study doesn't show how often journals are
cited outside philosophy. It also doesn't show citations in books, but
that's less problematic since citation patterns of journal articles in
books and journals are reasonably similar. However, citations inside
philosophy poorly predict citations outside philosophy. In
Table~\ref{tbl-list-of-journals}, \emph{Journal of Medical Ethics}
articles are collectively rarely cited. This reflects my excluding
medical journals where that journal is cited more often. The data tells
you something: if you want confirmation that `core' philosophy journals
don't publish much bioethics, \emph{Journal of Medical Ethics} citation
numbers are evidence. But they're not evidence about the journal's
overall impact; we're not looking in the right place to measure that.

\subsection{Age, Period, and Cohort}\label{sec-apc}

To understand citation patterns, I'll borrow terminology common in
sociology and medicine. An example best introduces it. Imagine we
observe interesting patterns among teenagers in the late 1960s and
wonder about explanations. Two pattern types immediately suggest
themselves, along with tests.

First, the behavior could reflect being teenagers, i.e., an \textbf{age
effect}. To test this, see if similar patterns appear with teenagers at
other times.

Second, the behavior could reflect the 1960s, when many striking things
happened, i.e., a \textbf{period effect}. To test this. see if
non-teenagers in the 1960s show the same pattern.

A third explanation is important. These people were born in the early
1950s, during the post-war baby boom. Colloquially, they're boomers.
Perhaps that explains the pattern, i.e., the pattern is a \textbf{cohort
effect}. To test this, examine the same people at other life stages.

Cohort effects are easily overlooked; sometimes they resemble age
effects. \citet{GhitzaEtAl2023} argue that many hypotheses about voting
age effects (e.g., older people being naturally conservative) are really
cohort effects. \citet{Bump2023} argues that understanding boomers'
distinctive role is crucial for understanding modern American life.

Mathematical reasons also complicate separating these effects. Many
statistical techniques for separating influences fail when linear
correlations exist between variable combinations. Here the correlation
is maximal: by definition, cohort plus age equals period. Some
workarounds exist\footnote{See \citet{KeyesEtAl2010} for options and
  \citet{Rohrer2025} for recent skepticism about general solutions}, but
the challenge remains.

Conceptually, separating these effects is difficult when effect strength
changes over time. As noted initially, testing which effect is strongest
naturally involves examining other times. This works when age effects
are constant. When they're not (as may be true here), it's harder.

However, keeping these three effects in mind helps summarize the data:

\begin{itemize}
\tightlist
\item
  The \textbf{age effect}: articles are cited most at ages two to five
  years.
\item
  The \textbf{period effect}: citations are far more numerous in recent
  years. Partly this reflects more published articles; partly it
  reflects citations per article growing substantially over the
  2000s-2010s and exploding in the 2020s.
\item
  The \textbf{cohort effect}: articles from the 1970s and 2000s are
  cited more than expected given age and period effects, while articles
  from other times, especially before 1965 and around 1990, are cited
  less. The reasons are more complicated; I return to them below.
\end{itemize}

The period effect is largest and, in some ways, least interesting. I'll
start by quantifying it and showing how I'll adjust for it.

\section{Period Effects}\label{sec-period}

The database contains 509,373 citations, unevenly distributed over time.
In 1956, there are only 4 citations to articles in the database. Since
the database starts in 1956, they are all citations to articles
published that year, and in 1956 there weren't many of those. By 2024,
there are 38,516 citations to indexed articles.
Figure~\ref{fig-citationsperyear} shows this growth.

\begin{figure}

\centering{

\pandocbounded{\includegraphics[keepaspectratio]{apc-nov-2025_files/figure-pdf/fig-citationsperyear-1.pdf}}

}

\caption{\label{fig-citationsperyear}Citations in the dataset by year.}

\end{figure}%

As noted in Section~\ref{sec-methodology}, I used different extraction
methods from 2022 onward. The 2021-2022 drop might reflect this change,
but there are reasons to think that it is largely real rather than an
artifact of the methodology. First, 2021 is an outlier in one important
respect: \emph{Synthese} published 1,439 articles in 2021 but only 509
in 2022. Second, applying the 2022-2024 method to 2020-2021 yielded
close agreement (within 1-2\%) for each year.

What explains this dramatic growth through 2021? Partly, more articles
are being published and indexed. Figure~\ref{fig-articlesperyear} shows
article counts by year.

\begin{figure}

\centering{

\pandocbounded{\includegraphics[keepaspectratio]{apc-nov-2025_files/figure-pdf/fig-articlesperyear-1.pdf}}

}

\caption{\label{fig-articlesperyear}Articles in the dataset by
publication year.}

\end{figure}%

This explains some growth, but not all of it. The curve in
Figure~\ref{fig-articlesperyear} isn't nearly as steep as in
Figure~\ref{fig-citationsperyear}. Citations per article are also
rising. Figure~\ref{fig-outboundcitations} plots average citations to
other dataset articles each year.

\begin{figure}

\centering{

\pandocbounded{\includegraphics[keepaspectratio]{apc-nov-2025_files/figure-pdf/fig-outboundcitations-1.pdf}}

}

\caption{\label{fig-outboundcitations}Average citations to indexed
articles by year.}

\end{figure}%

Several factors explain this graph's shape.

At the left edge, boundary effects are obvious. Since we count only
citations to articles published since 1956, articles in the 1950s
naturally have few citations. Since articles rarely get unpublished,
more articles become available to cite each year.

This can't explain the massive jumps at
Figure~\ref{fig-outboundcitations}'s right edge. That jump reflects
converging cultural trends: more citations overall (casual journal
perusal confirms this), and more journal citations versus book or edited
volume citations.

Sharp jumps like this warrant data checking. Cross-checking every entry
is impractical, but spot-checks look correct. The change appears led by
prestigious journals. For each journal, I calculated average outbound
citations (to these hundred journals) for the 2010s and 2020-2024. The
ten journals with the largest increases appear in
Table~\ref{tbl-large-growth}.

\begin{longtable}[]{@{}
  >{\raggedright\arraybackslash}p{(\linewidth - 6\tabcolsep) * \real{0.6630}}
  >{\raggedleft\arraybackslash}p{(\linewidth - 6\tabcolsep) * \real{0.1087}}
  >{\raggedleft\arraybackslash}p{(\linewidth - 6\tabcolsep) * \real{0.1087}}
  >{\raggedleft\arraybackslash}p{(\linewidth - 6\tabcolsep) * \real{0.1196}}@{}}

\caption{\label{tbl-large-growth}Mean outbound citations for selected
journals over two decades.}

\tabularnewline

\toprule\noalign{}
\begin{minipage}[b]{\linewidth}\raggedright
Journal
\end{minipage} & \begin{minipage}[b]{\linewidth}\raggedleft
2010-2019
\end{minipage} & \begin{minipage}[b]{\linewidth}\raggedleft
2020-2024
\end{minipage} & \begin{minipage}[b]{\linewidth}\raggedleft
Difference
\end{minipage} \\
\midrule\noalign{}
\endhead
\bottomrule\noalign{}
\endlastfoot
Philosophical Review & 14.8 & 25.9 & 11.1 \\
Philosophical Perspectives & 11.3 & 19.2 & 7.9 \\
Noûs & 11.5 & 18.4 & 6.9 \\
Philosophy and Phenomenological\newline\hspace*{1em}Research & 9.6 &
15.8 & 6.2 \\
Philosophical Studies & 9.0 & 14.6 & 5.6 \\
Journal of Philosophy & 9.0 & 14.5 & 5.6 \\
Philosophy & 4.0 & 8.9 & 4.9 \\
Episteme & 8.1 & 12.9 & 4.9 \\
Philosophical Quarterly & 8.8 & 13.6 & 4.7 \\
Philosophy Compass & 11.2 & 15.9 & 4.7 \\

\end{longtable}

\emph{Philosophical Review} publishes only 10-12 articles annually, so
high variation is expected. Still, the 2010s change isn't just
small-sample variation. Of its 22 articles in 2020-2021, only one
\citep{WOS000575210400003} had fewer than 14.8 outbound citations. With
just 22 articles, anything could happen, but having all but one exceed
the historical average by chance would be surprising.

How do we correct for this? We could simply ask what proportion of
citations accrue to articles in a given year, but that would be an
overcorrection. The 2020s have more citations to distribute, but also
more articles sharing them. We must adjust for both. Here's how.

An article is \emph{typically cited} if published 3-10 years before the
citing year. As Figure~\ref{fig-master-citation-ratio} showed, citations
typically peak then. Using this definition,
Figure~\ref{fig-articlecounts} shows typically cited article counts at
any given time (e.g., for 2000, it shows articles published 1990-1997).

\begin{figure}

\centering{

\pandocbounded{\includegraphics[keepaspectratio]{apc-nov-2025_files/figure-pdf/fig-articlecounts-1.pdf}}

}

\caption{\label{fig-articlecounts}Typically cited articles by year.}

\end{figure}%

Figure~\ref{fig-citationcounts} shows how often these `typical' articles
are cited each year; Figure~\ref{fig-citationrate} shows mean citations
to typical articles per year.

\begin{figure}

\centering{

\pandocbounded{\includegraphics[keepaspectratio]{apc-nov-2025_files/figure-pdf/fig-citationcounts-1.pdf}}

}

\caption{\label{fig-citationcounts}Citations to typical articles by
year.}

\end{figure}%

\begin{figure}

\centering{

\pandocbounded{\includegraphics[keepaspectratio]{apc-nov-2025_files/figure-pdf/fig-citationrate-1.pdf}}

}

\caption{\label{fig-citationrate}Mean annual citations to typical
articles.}

\end{figure}%

Two things stand out in Figure~\ref{fig-citationrate}. First, the graph
is flat for a long time. From mid-1970s to early-2000s, it bounces
without much movement. Then it takes off, peaks in 2021, and returns to
the long-term trend. Second, the numbers are never high. Throughout most
of this study, even peak-age articles (3-10 years old) are cited once
per \emph{decade} in these hundred journals. Since citation rates are
extremely long-tailed with means well above medians, this overstates how
often the `average article' gets cited. Frequent citation is decidedly
not the norm.\footnote{Long-term, average citations per article equals
  average times an article is cited, so most articles having just a
  handful of philosophy journal citations is unsurprising.}

My initial measure of article influence is citation frequency divided by
typical article citation frequency. This is somewhat arbitrary, I could
have chosen ranges other than 3-10 years, but this works reasonably
well. I tried other measures; they produced either implausible data
trends or implausible judgments about paper influence. This measure had
a nice property: how influential the leading 50 articles from a period
were 10-20 years later was reasonably stable, suggesting it corrects for
period effects well.

\section{Age Effects}\label{sec-age}

Next, I determine how article age affects citation frequency. The
simplest approach, examining a typical year to see how its articles are
cited over time, would be completely wrong.
Figure~\ref{fig-1990-outbound-citations} shows citation patterns for
1990 articles.

\begin{figure}

\centering{

\pandocbounded{\includegraphics[keepaspectratio]{apc-nov-2025_files/figure-pdf/fig-1990-outbound-citations-1.pdf}}

}

\caption{\label{fig-1990-outbound-citations}Citations to 1990-published
articles.}

\end{figure}%

Using citations as influence measures,
Figure~\ref{fig-1990-outbound-citations} suggests 1990 articles were
collectively most influential in 2021. That's not true; like most
articles, they were most influential 2-4 years post-publication. The
2020s simply published so many articles, each citing so many pieces,
that even three-decade-old articles get lifted by the rising tide.

A more intuitive influence measure uses typical articles from
Section~\ref{sec-period}. Adjust
Figure~\ref{fig-1990-outbound-citations} by dividing each value by two
things: first, the typical article citation rate from
Figure~\ref{fig-citationrate} (adjusting for period effects); second,
the number of 1990-published articles (yielding per-article influence).
The result is the \textbf{citation ratio} from
Figure~\ref{fig-master-citation-ratio}.
Figure~\ref{fig-1990-outbound-citations-norm} shows 1990's citation
ratio.

\begin{figure}

\centering{

\pandocbounded{\includegraphics[keepaspectratio]{apc-nov-2025_files/figure-pdf/fig-1990-outbound-citations-norm-1.pdf}}

}

\caption{\label{fig-1990-outbound-citations-norm}Normalized citations to
1990-published articles.}

\end{figure}%

Two reasons support Figure~\ref{fig-1990-outbound-citations-norm} as a
more plausible influence measure than
Figure~\ref{fig-1990-outbound-citations}. One appeals to intuition: I
know 1990 work came up in discussions far more in the 1990s than 2020s.
While such intuitive evidence deserves some weight, it's obviously
unreliable alone. The better reason: we get very similar graphs no
matter which initial year we pick. This was visible in
Figure~\ref{fig-decades-cite-ratio-early} and
Figure~\ref{fig-decades-cite-ratio-late}, but it's worth seeing how
stable this is.

First, let's get the explicit definition of citation ratio on the table.
Let \emph{c}(\emph{o}, \emph{n}) be citations of year-\emph{o} articles
(old year) in year \emph{n} (new year). Let \emph{a}(\emph{o}) be
articles published in year \emph{o}. Then citation ratio
\emph{r}(\emph{o}, \emph{n}) is:

\[
r(o, n) = \left(\frac{c(o, n)}{a(o)}\right) / \left(\frac{\sum\limits_{i = n-10}^{n-3}c(i, n)}{\sum\limits_{i = n-10}^{n-3}a(i)}\right)
\]

In Figure~\ref{fig-ageeffecttibble-early} and
Figure~\ref{fig-ageeffecttibble-late}, each facet represents different
\emph{o} values, the x-axis is \emph{n}, and the y-axis is
\emph{r}(\emph{o}, \emph{n}). The key observation: these graphs are
remarkably steady. I've cheated slightly, since showing earlier years
would reveal different shapes. 1960s citations are so sparse that noise
overwhelms signal. Since then, patterns are reasonably steady.

\begin{figure}

\centering{

\pandocbounded{\includegraphics[keepaspectratio]{apc-nov-2025_files/figure-pdf/fig-ageeffecttibble-early-1.pdf}}

}

\caption{\label{fig-ageeffecttibble-early}Citation rates for articles
published 1968-1992.}

\end{figure}%

\begin{figure}

\centering{

\pandocbounded{\includegraphics[keepaspectratio]{apc-nov-2025_files/figure-pdf/fig-ageeffecttibble-late-1.pdf}}

}

\caption{\label{fig-ageeffecttibble-late}Citation rates for articles
published 1993-2017.}

\end{figure}%

\section{Cohort Effects}\label{sec-cohort}

Period and age effects together explain much of the trends in citation
patterns. But some systematic deviations remain.
Figure~\ref{fig-twodeviations} shows some examples. Each graph is a
facet from Figure~\ref{fig-ageeffecttibble-early} with a line showing
average age effects.

\begin{figure}

\begin{minipage}{0.50\linewidth}

\centering{

\pandocbounded{\includegraphics[keepaspectratio]{apc-nov-2025_files/figure-pdf/fig-twodeviations-1.pdf}}

}

\subcaption{\label{fig-twodeviations-1}1979}

\end{minipage}%
%
\begin{minipage}{0.50\linewidth}

\centering{

\pandocbounded{\includegraphics[keepaspectratio]{apc-nov-2025_files/figure-pdf/fig-twodeviations-2.pdf}}

}

\subcaption{\label{fig-twodeviations-2}1987}

\end{minipage}%

\caption{\label{fig-twodeviations}Citations from a given year compared
to average citations.}

\end{figure}%

In 1979, yearly values predominantly exceed the mean line; in 1987,
they're largely below it. My main cohort effect measure is the average
ratio between yearly data (dots) and average values (line) for each such
graph. 1979 dots average about 13\% above the mean; 1987, about 9\%
below. Repeating this for every dataset year yields
Figure~\ref{fig-cohort}.

\begin{figure}

\centering{

\pandocbounded{\includegraphics[keepaspectratio]{apc-nov-2025_files/figure-pdf/fig-cohort-1.pdf}}

}

\caption{\label{fig-cohort}Cohort effects by publication year.}

\end{figure}%

Technical notes on Figure~\ref{fig-cohort}: I added a rolling average
line (four years either side, or as many as available) to highlight
features. In calculating means, I included only years with at least five
years of data for calculating mean age effects. So I haven't included
what happens when 1956 papers are cited after 2019. In those cases,
insufficient data exists to determine `expected' aging curves at those
points.

Five periods appear in the graph:

\begin{enumerate}
\def\labelenumi{\arabic{enumi}.}
\tightlist
\item
  Pre-mid-1960s journal articles are very rarely cited.
\item
  After that, especially in the early 1970s, many highly cited articles
  appear.
\item
  A stagnation period follows, when things mostly don't return to
  pre-1965 lows but stay consistently below zero.
\item
  An uptick begins mid-1990s, peaking dramatically in 2007.
\item
  A dramatic dropoff follows almost immediately after 2007's high.
\end{enumerate}

The first two trends make sense; the latter three less so. The remainder
of this paper explains what's happening and what it reveals about
philosophy's history and the philosophy profession's history.

Before 1965, philosophy's most significant work wasn't in WoS-indexed
journals. Instead, it was either in books, or in journals WoS didn't
index. So we lack ``Is Knowledge Justified True Belief?''
\citep{Gettier1963} because \emph{Analysis} indexing starts only in
1975. And we lack Austin's major papers, ``Ifs and Cans'' and ``A Plea
for Excuses'' \citep{Austin1956, Austin1956b}, because their venues
aren't indexed as journals. We do have important papers by
\citet{WOSA1956CHJ4400001}, \citet{WOSA1956CEQ2500001},
\citet{WOSA1957CGZ6000005}, \citet{WOSA1958CDL1000001},
\citet{WOSA1959CGZ6600001}, and \citet{WOSA1963CEU0700001}, but these
had less impact than contemporary books, especially \emph{Intention}
\citep{Anscombe1957}, \emph{Word and Object} \citep{Quine1960}, and
\emph{The Structure of Scientific Revolutions} \citep{Kuhn1962}.

Then, starting in the late 1960s, nearly every philosophy area was
transformed, with much action in journals. The period's two most
important works, \emph{A Theory of Justice} \citep{Rawls1971} and
\emph{Naming and Necessity} \citep{Kripke1980}, weren't journal
articles. But journal articles did revolutionize many fields, including:

\begin{itemize}
\tightlist
\item
  Free will \citep{WOSA1969Y444700002, 10.2307_2024717};
\item
  Practical ethics \citep{WOSA1971Y116900003, WOSA1972Z066400001};
\item
  Meaning and reference \citep{10.2307_2025079};
\item
  Philosophy of mathematics \citep{10.2307_2025075};
\item
  Causation \citep{10.2307_2025310, 10.2307_2025096}; and
\item
  Personal identity \citep{WOSA1971Y036400001}
\end{itemize}

Unusually, several more papers from the time became influential later
rather than immediately, including work by \citet{WOSA1970ZE33800001},
\citet{WOSA1970ZE32700001}, \citet{WOSA1970Y384700002}, and
\citet{WOSA1973P242100001}. Figure~\ref{fig-cohort}'s early-1970s story
is directionally plausible, though one could quibble about the magnitude
given how spotty the data is. What's clear is that before the late
1960s-early 1970s, philosophy journals had never published such
high-quality work in this quantity.

To understand subsequent developments, we must examine data more
closely. Figure~\ref{fig-cohort-short} is Figure~\ref{fig-cohort}
restricted to the first seven post-publication years. For each
publication year \emph{o}, it measures the average citation ratio
\emph{r}(\emph{o}, \emph{n}) when \emph{n}~‑~\emph{o}~is between 1 and
7. I start in 1975 to exclude noisy early data.

\begin{figure}

\centering{

\pandocbounded{\includegraphics[keepaspectratio]{apc-nov-2025_files/figure-pdf/fig-cohort-short-1.pdf}}

}

\caption{\label{fig-cohort-short}Short-term citation rates (first seven
years post-publication).}

\end{figure}%

In Figure~\ref{fig-cohort}, 2010 articles are cited slightly more than
1990 articles (after adjustments). But in Figure~\ref{fig-cohort-short},
they're cited 20\% less. Generally, nearly every 1980s-1990s article
batch shows solid first-seven-year citation rates. So
Figure~\ref{fig-cohort-short} lacks Figure~\ref{fig-cohort}'s 1980s dip.
This contrasts strikingly with Figure~\ref{fig-cohort-long}, measuring
only post-seven-year citations.\footnote{That is, for each \emph{o} it
  measures the average value of \emph{r}(\emph{o}, \emph{n}) when
  \emph{n}~‑~\emph{o} is greater than 7.}

\begin{figure}

\centering{

\pandocbounded{\includegraphics[keepaspectratio]{apc-nov-2025_files/figure-pdf/fig-cohort-long-1.pdf}}

}

\caption{\label{fig-cohort-long}Long-term citation rates (after the
first seven years).}

\end{figure}%

In Figure~\ref{fig-cohort-long}, every 2000-2016 year averages higher
than every 1980-1999 year. Starting around 1998, a large change occurs
in how often articles over seven years old are cited.

This explains Figure~\ref{fig-cohort}'s right-edge dropoff, i.e., the
fifth period I described. Citations are aging, but for mid-2010s
articles, we lack data on citations ten or more years post-publication.
Averaging their first-few-year citations compared to a generation ago's
first-few-year citations underestimates their influence.

This tells us more about what we need to explain in third and fourth
periods in Figure~\ref{fig-cohort}. 1980s-1990s articles were heavily
cited soon after publication, but their citation rates didn't hold up
like later articles. 2000s articles were cited less initially than late
20th century articles but far more as time passed.

That tells us what we need to explain, but it doesn't exactly explain
it. Several further questions remain.

Why is this temporal citation shift occurring? Shouldn't technological
changes produce the opposite effect? I discuss this in
Section~\ref{sec-technology}.

Why are citations rising so much generally, even accounting for
increased article publication? Section~\ref{sec-technology} partially
explains this, but Section~\ref{sec-culture} discusses two cultural
factors.

Finally, why do periods around 1990 and 2005 stand out? Around 1990,
Figure~\ref{fig-cohort-short} shows an upward spike and
Figure~\ref{fig-cohort-long} a low point. Around 2005, both graphs
exceed long-term trends. The answers partly involve technological
factors (Section~\ref{sec-technology}) and cultural factors
(Section~\ref{sec-culture}) but also reflect important changes which
topics were philosophically central. This hints at an important
twentieth/twenty-first-century philosophy discontinuity, which I address
in Section~\ref{sec-content}.

\section{Technology and Citations}\label{sec-technology}

A common view holds that electronic publication primarily speeds
\emph{distribution}. The data doesn't support this. If it were true,
we'd expect short-term citations, especially very short-term citations,
to rise over time.

By the late twentieth century, printing and postage were mature
technologies. We weren't awaiting steam ships to deliver journal issues
to distant shores. Philosophy journal distribution used the same
technology as medicine and other time-sensitive fields. From this
perspective, the internet would accelerate distribution by weeks or
months at most. That's too short a difference to show up on yearly
graphs. Postal improvements, especially increased airmail use, may
affect 1980s-1990s citation numbers. We do see more citations to foreign
journals soon after publication than in earlier decades, for example.
But Online First, Early View, and similar quasi-publication forms
haven't made much difference. They appear slightly in the data,
occasionally articles are cited before official publication, but such
cases are rare.

Technology's biggest effect was on \emph{search}, not distribution.
Before widespread computer use, searching books was far easier than
searching journal articles. Classification systems placed books near
others on the same topic. Card catalogs listed book subjects. Even book
titles helped locate topics. Finding relevant journal articles was much
harder and, it seems, rarely done.

Physical storage and access also differed notably between books and
journals. Nearly every academic has a bookshelf; far fewer have large
journal collections. Departments occasionally kept physical journals,
but accessing them often meant walking across campus to the library.
Accessing a book might mean walking four steps to a shelf. This physical
difference likely contributed to books' and articles' relative
prominence in bibliographies.

In many departments, there was one way journals were more accessible.
Departments often kept latest journal issues in a department library or
common room where they were prominent and accessible. Whether this
explains why pre-1995 journal citations are so often to very recent
journals is unclear, but it probably helped.

Search technology changes appear central to the changing citation
patterns. Before widespread computer adoption, old journal articles were
very hard to find. This changed somewhat with electronic, easily
searchable databases like \emph{Philosophers' Index}, and dramatically
when journals went online. This partly explains why older articles,
especially non-classic older articles, are now more widely cited.

\section{Culture Changes}\label{sec-culture}

Two cultural changes significantly affected citation patterns: one
concerning journal articles' role, another concerning citation norms.

Looking back at century-old journals, pieces sometimes resemble blog
posts more than current journal articles. Even substantive pieces feel
like means to an end. The journal article is as much a book trial run as
a complete project report.

By the twenty-first century, this changed completely. Articles are
longer, even in venues like \emph{Analysis}. More importantly, they're
often finished products, not draft runs for future books. There are
prominent philosophers whose reputations rest largely, or in some cases
entirely, on articles. Large philosophy fields such as epistemological
contextualism and metaphysical grounding have collections of canonical
texts that are almost entirely articles, not books. This isn't entirely
new\footnote{For example, the analysis of knowledge literature that
  followed \citet{Gettier1963} was almost entirely article-based}, but
it's a growing trend. This partially explains aging citations: lots of
articles cite the canonical works in a field, and if those canonical
works are older journal articles, there are more citations to older
journal articles.

The bigger trend is philosophers' increasing tendency to include brief
citations to work they're not discussing in detail but which locate the
paper in a literature. Obviously if an article cites 26 other journal
articles, as Table~\ref{tbl-large-growth} showed the average recent
\emph{Philosophical Review} paper does, it can't possibly discuss all of
them in depth. (Recall that as well as those 26 journal articles, each
article is citing books and chapters.) For many decades, a striking
difference between philosophy and adjacent was that philosophers did not
use citations to simply place a paper in a literature. In recent years
this practice, long common in cognate fields, has become common in
philosophy as well.

Why did this change occur? Technology is part of the story
(Section~\ref{sec-technology}). Citation managers mean that it's now
easier to cite more pieces. But this can't be the full story, since the
practice was more widespread in other fields before these technologies
were universally adopted. Specialization growth is another part. Writing
about content externalism in the 1990s didn't require a citation trail
back to Kripke and Putnam because the reader could be assumed to know
the debates' history. Since then, it's become less clear that any
field's history is universally known, creating more need for citations
that provide background the reader can't be assumed to have.

But probably the biggest part of the explanation is the growth in
interdisciplinary work in the 2000s. This growth takes place across a
range of fields. Experimental philosophy goes from something that is
barely recognised to a significant field. Philosophy of mind gets
increasingly empirically sophisticated. I'll focus on one other aspect:
philosophy of language's move toward debates where there is
simultaneously a lot of work going on in linguistics.

The largest twentieth century debates in philosophy of language were, on
the whole, not mirrored in linguistics. Linguists didn't talk much about
wide versus narrow content, or the relationship between reference and
description. That changed in the 2000s, when there was more discussion
of context-sensitivity, and of modality. At the same time, we see
citation norms from linguistics become more common in philosophy of
language papers.

Now, philosophy of language wasn't as important to journals in the 2000s
as earlier decades.\footnote{I won't argue this here; it would require
  nearly another paper and more data sources. The short version is
  nothing in twenty-first-century philosophy of language was as crucial
  to journals as debates over names and descriptions and over wide and
  narrow content had been in earlier decades.} But it was an important
vector for citation norms spreading from linguistics to philosophy,
especially since much philosophy of language work overlapped with that
decade's key epistemological debate: contextualism. I'll end this paper
examining how changes in what philosophers discussed interact with
citation data.

\section{Content Changes}\label{sec-content}

Some of the changes in citation patterns can be explained by changes in
what topics were central to philosophy. Conversely, looking at changing
citation patterns can tell us something interesting about how the
discipline changed.

Through at least the early 2000s, analytic philosophy is in what
\citet[2]{Sider2020} calls the ``modal era''. One aspect of this era
that Sider highlights is that essence questions were equated with
necessity questions in ways they weren't before or after.\footnote{During
  this era, the necessity of origins thesis and origin essentialism
  thesis were typically taken as not just mutually supporting but
  literally identical. That identity claim was not widely endorsed
  before 1970 or after 2010.} This isn't definitive of being in the
modal era, but it's a striking symptom. What is, I think, definitive is
how modality was central to disputes across the discipline.

Consider, for example, what Frank \citet{Jackson1998} called the
`location problem'. That's the problem of how to locate in the world
something the philosopher thinks exists and is not fundamental. Jackson
argues that saying how to locate the non-fundamental in the fundamental
is a compulsory question for anyone doing `serious metaphysics', with
the one and only answer involving modality. As he says,

\begin{quote}
When does a putative feature of our world have a place in the account
some serious metaphysics tells of what our world is like? I have already
mentioned one answer: if the feature is entailed by the account told in
the terms favoured by the metaphysics in question, it has a place in the
account told in the favoured terms. This is hardly controversial
considered as a sufficient condition, but, I will now argue, it is also
a necessary condition: the one and only way of having a place in an
account told in some set of preferred terms is by being entailed by that
account---a view I will refer to as the entry by entailment thesis.
\citep[5]{Jackson1998}
\end{quote}

Jackson went on to defend several controversial theses about entailment
in that book. But here in the book's introduction, he was largely
expressing conventional wisdom. In a review disagreeing with much of
what Jackson argues, Stephen \citet[20]{Yablo2000Jackson} says ``Not
many eyebrows will be raised by Jackson's view that metaphysics is
committed to `entry by entailment' theses.'' The quoted parts just
aren't controversial, especially the one Jackson flags as ever so
slightly more controversial.

The idea that entailment, i.e., necessitation, had been central to
understanding how the non-fundamental relates to the fundamental was
central to philosophy for many years by this point. We can see how
central using a slightly different statistic: grand-citations.

The number of grand-citations an article \emph{a} has is the number of
triples ⟨\emph{a}, \emph{b}, \emph{c}⟩ such that \emph{c} cites \emph{b}
and \emph{b} cites \emph{a}. Grand-citations over time show David
Lewis's centrality to philosophy journals: five of the six articles with
the most grand-citations are by Lewis. Looking at particular times
reveals changes in which topics are being most widely discussed in the
journals. Grand-citations take time to accrue, so I'll examine
twenty-year periods. That is, I'll take snapshots at various times and
ask which articles published in the preceding twenty years had the most
grand-citations through that year.


\begin{longtable}[]{@{}
  >{\raggedright\arraybackslash}p{(\linewidth - 4\tabcolsep) * \real{0.8448}}
  >{\raggedleft\arraybackslash}p{(\linewidth - 4\tabcolsep) * \real{0.0517}}
  >{\raggedleft\arraybackslash}p{(\linewidth - 4\tabcolsep) * \real{0.1034}}@{}}

\caption{\label{tbl-grand-cite-2000}1980s-1990s articles with the most
grand-citations by 2000.}

\tabularnewline

\toprule\noalign{}
\begin{minipage}[b]{\linewidth}\raggedright
Article
\end{minipage} & \begin{minipage}[b]{\linewidth}\raggedleft
Cites
\end{minipage} & \begin{minipage}[b]{\linewidth}\raggedleft
Grand-Cites
\end{minipage} \\
\midrule\noalign{}
\endhead
\bottomrule\noalign{}
\endlastfoot
Terrence Horgan
\citeyearpar{WOSA1982NN35300003}
``Supervenience and Microphysics'' & 36 & 318 \\
Tyler Burge
\citeyearpar{WOSA1986AYX3200001}
``Individualism and Psychology'' & 82 & 316 \\
David Lewis
\citeyearpar{WOSA1983RR51600001}
``New Work for a Theory of Universals'' & 86 & 308 \\
Paul M. Churchland
\citeyearpar{WOSA1981LD54600001}
``Eliminative Materialism and the Propositional Attitudes'' & 94 &
299 \\
John Haugeland
\citeyearpar{WOSA1982NC42600008}
``Weak Supervenience'' & 40 & 258 \\
Jaegwon Kim
\citeyearpar{WOSA1982NC90700004}
``Psychophysical Supervenience'' & 40 & 245 \\
Ruth Garrett Millikan
\citeyearpar{WOSA1989AA09400006}
``In Defense of Proper Functions'' & 43 & 232 \\
Jon Barwise and Robin Cooper
\citeyearpar{WOSA1981LH67300001}
``Generalized Quantifiers and Natural-Language'' & 83 & 221 \\
John Bigelow and Robert Pargetter
\citeyearpar{WOSA1987G947600001}
``Functions'' & 30 & 220 \\
Jaegwon Kim
\citeyearpar{WOSA1984TV24600001}
``Concepts of Supervenience'' & 87 & 219 \\

\end{longtable}

Table~\ref{tbl-grand-cite-2000} lists articles published from 1980
onward with the most grand-citations through 2000. I've included article
names to show how central supervenience was to this
literature.\footnote{The relationship story between twentieth-century
  work on functions and twenty-first-century work on mechanisms is
  interesting but for another time.} Four articles have `supervenience'
in the title! Jaegwon Kim stood at this literature's center. This
citation data somewhat \emph{underestimates} his influence since people
often cited his edited collection \emph{Supervenience and Mind}
\citep{Kim1993}, and these citations were often not tracked by WoS.

These articles' subsequent history largely explains
Figure~\ref{fig-cohort}'s relatively low 1980s values. Much of this
supervenience work has fallen out of philosophical discourse.
Table~\ref{tbl-grand-cite-2000-now} shows how many citations
Table~\ref{tbl-grand-cite-2000} articles have since 2021.


\begin{longtable}[]{@{}lr@{}}

\caption{\label{tbl-grand-cite-2000-now}Citations of
Table~\ref{tbl-grand-cite-2000} articles since 2021.}

\tabularnewline

\toprule\noalign{}
Article & Citations since 2021 \\
\midrule\noalign{}
\endhead
\bottomrule\noalign{}
\endlastfoot
\citet{WOSA1982NN35300003}
& 3 \\
\citet{WOSA1986AYX3200001}
& 15 \\
\citet{WOSA1983RR51600001}
& 271 \\
\citet{WOSA1981LD54600001}
& 50 \\
\citet{WOSA1982NC42600008}
& 2 \\
\citet{WOSA1982NC90700004}
& 6 \\
\citet{WOSA1989AA09400006}
& 58 \\
\citet{WOSA1981LH67300001}
& 43 \\
\citet{WOSA1987G947600001}
& 24 \\
\citet{WOSA1984TV24600001}
& 10 \\

\end{longtable}

Supervenience articles simply aren't cited much these days. The same
pattern recurs when we look at other widely discussed supervenience
papers (e.g., \citet{WOSA1989T680600002} and
\citet{WOSA1984ST78300010}). And it recurs when we look at papers from
debates in mind and language which made heavy use of notions of
superveience. The latter includes debates on wide and narrow content,
mental individualism, empty names, and hallucinations. Articles by
\citet{WOSA1986AYX3200002}, \citet{WOSA1988P549200004},
\citet{WOSA1989T680600002}, \citet{WOSA1983RU36600003},
\citet{WOSA1991FF02900001}, and \citet{WOSA1991EN62900001}, all very
influential then, and often in very prominent venues, have been barely
cited since 2021. Even some Lewis papers on these topics (e.g.,
\citet{WOSA1981MS19500002} and \citet{WOSA1983PZ01000001}) have just a
handful of recent citations. A debate family central to philosophy for
years, and particularly central to what \emph{Philosophical Review}
published in the 1980s, has simply dropped off the agenda. This can't
help but impact citation numbers.

\newpage


\begin{longtable}[]{@{}
  >{\raggedright\arraybackslash}p{(\linewidth - 4\tabcolsep) * \real{0.8626}}
  >{\raggedleft\arraybackslash}p{(\linewidth - 4\tabcolsep) * \real{0.0458}}
  >{\raggedleft\arraybackslash}p{(\linewidth - 4\tabcolsep) * \real{0.0916}}@{}}

\caption{\label{tbl-grand-cite-2010}Top 10 1990s-2000s articles by
grand-citations through 2010.}

\tabularnewline

\toprule\noalign{}
\begin{minipage}[b]{\linewidth}\raggedright
Article
\end{minipage} & \begin{minipage}[b]{\linewidth}\raggedleft
Cites
\end{minipage} & \begin{minipage}[b]{\linewidth}\raggedleft
Grand-Cites
\end{minipage} \\
\midrule\noalign{}
\endhead
\bottomrule\noalign{}
\endlastfoot
David Lewis
\citeyearpar{WOSA1996VY21200001}
``Elusive Knowledge'' & 182 & 665 \\
Keith DeRose
\citeyearpar{WOSA1995RC31600001}
``Solving the Skeptical Problem'' & 145 & 604 \\
Stephen Yablo
\citeyearpar{WOSA1992JA62400001}
``Mental Causation'' & 126 & 572 \\
Keith DeRose
\citeyearpar{WOSA1991GL32100002}
``Epistemic Possibilities'' & 40 & 519 \\
Tyler Burge
\citeyearpar{WOSA1993ML38000001}
``Content Preservation'' & 136 & 502 \\
Karen Neander
\citeyearpar{WOSA1991FQ15000002}
``Functions as Selected Effects: The Conceptual Analyst's Defense'' & 89
& 480 \\
Keith DeRose
\citeyearpar{WOSA1992KB29500008}
``Contextualism and Knowledge Attributions'' & 79 & 430 \\
Mark Johnston
\citeyearpar{WOSA1992KC39800002}
``How To Speak of the Colors'' & 93 & 423 \\
C. B. Martin
\citeyearpar{WOSA1994MT56900001}
``Dispositions and Conditionals'' & 82 & 401 \\
Michael B. Burke
\citeyearpar{WOSA1992HC13100003}
``Copper Statues and Pieces of Copper: A Challenge To the Standard
Account'' & 45 & 395 \\

\end{longtable}

Moving into the 2000s, the focus shifts dramatically, as
Table~\ref{tbl-grand-cite-2010} shows. The biggest single topic was
epistemological contextualism, with major papers by
\citet{WOSA1996VY21200001} and \citet{WOSA1995RC31600001} at the center.
What I want to focus on particularly, though, is another DeRose paper on
that list: ``Epistemic Possibilities.'' It has fewer cites than any
other paper there but the fourth-most grand-cites. This is partly
because it was important to both epistemology and philosophy of
language. But another reason is that the debate it triggered was one of
the first places where linguistics citation norms were applied in
philosophy. This is part of the evidence for my
Section~\ref{sec-culture} claims about changing norms' importance to
citation rates.

Moving to the 2010s, the focus shifts again, as
Table~\ref{tbl-grand-cite-2020} shows. Here we see some evidence that
philosophy is getting more specialised, and more fragmented. The papers
with the most grand-citations address subjects like mechanisms,
dogmatism, peer disagreement, pragmatic encroachment, and priority
monism. That's a much broader range of topics than was central to
philosophy a few decades earlier. Crucially, these topics overlap so
little that someone working in them can't expect philosophers working in
other topics to know even debate basics. (Perhaps people working on
various epistemology topics on that list can expect slightly more
background knowledge from people working on other epistemology topics,
but that's about it.) So writers need to include more citations just so
readers will have access to the basic information about the debate.


\begin{longtable}[]{@{}
  >{\raggedright\arraybackslash}p{(\linewidth - 4\tabcolsep) * \real{0.8475}}
  >{\raggedleft\arraybackslash}p{(\linewidth - 4\tabcolsep) * \real{0.0508}}
  >{\raggedleft\arraybackslash}p{(\linewidth - 4\tabcolsep) * \real{0.1017}}@{}}

\caption{\label{tbl-grand-cite-2020}Top 10 2000s-2010s articles by
grand-citations through 2020.}

\tabularnewline

\toprule\noalign{}
\begin{minipage}[b]{\linewidth}\raggedright
Article
\end{minipage} & \begin{minipage}[b]{\linewidth}\raggedleft
Cites
\end{minipage} & \begin{minipage}[b]{\linewidth}\raggedleft
Grand-Cites
\end{minipage} \\
\midrule\noalign{}
\endhead
\bottomrule\noalign{}
\endlastfoot
Peter Machamer, Lindley Darden, and Carl F. Craver
\citeyearpar{WOS000087305900001}
``Thinking About Mechanisms'' & 402 & 2807 \\
James Pryor
\citeyearpar{WOS000165361800002}
``The Skeptic and the Dogmatist'' & 288 & 1680 \\
David Lewis
\citeyearpar{WOS000089124200002}
``Causation as Influence'' & 173 & 1647 \\
Jeremy Fantl and Matthew McGrath
\citeyearpar{WOS000181094500003}
``Evidence, Pragmatics, and Justification'' & 149 & 1557 \\
Jonathan Schaffer
\citeyearpar{WOS000272855000002}
``Monism: The Priority of the Whole'' & 215 & 1429 \\
Jason Stanley and Timothy Williamson
\citeyearpar{WOS000170277300002}
``Knowing How'' & 208 & 1418 \\
Keith DeRose
\citeyearpar{WOS000184740400001}
``Assertion, Knowledge, and Context'' & 170 & 1415 \\
David Christensen
\citeyearpar{WOS000207419300002}
``Epistemology of Disagreement: The Good News'' & 206 & 1387 \\
Nishi Shah
\citeyearpar{WOS000224335200001}
``How Truth Governs Belief'' & 135 & 1303 \\
Adam Elga
\citeyearpar{WOS000249103800005}
``Reflection and Disagreement'' & 216 & 1301 \\

\end{longtable}

For completeness, Table~\ref{tbl-grand-cite-2024} shows the same
analysis for the 20 years ending in 2024. Since so many of the papers in
Table~\ref{tbl-grand-cite-2020} were published in 2000-2003, there is
some turnover. The turnover in topics is particularly striking, with
more ethics appearing on the list than at earlier times.\footnote{One
  puzzling feature of the grand-citation data is that ``What is the
  Point of Equality?'' (\citet{WOS000078432400003}) has relatively few
  grand-citations, despite having the second most citations of any
  article since 1999 in these journals. I suspect the explanation here
  involves differing citation practices in different fields of
  philosophy, but possibly it is telling us something about the
  relationship between political philosophy and the rest of the
  discipline over the last couple of decades. This is an interesting
  question for further research.}

\newpage


\begin{longtable}[]{@{}
  >{\raggedright\arraybackslash}p{(\linewidth - 4\tabcolsep) * \real{0.8759}}
  >{\raggedleft\arraybackslash}p{(\linewidth - 4\tabcolsep) * \real{0.0414}}
  >{\raggedleft\arraybackslash}p{(\linewidth - 4\tabcolsep) * \real{0.0828}}@{}}

\caption{\label{tbl-grand-cite-2024}Top 10 articles from 2004-2024 by
grand-citations through 2024.}

\tabularnewline

\toprule\noalign{}
\begin{minipage}[b]{\linewidth}\raggedright
Article
\end{minipage} & \begin{minipage}[b]{\linewidth}\raggedleft
Cites
\end{minipage} & \begin{minipage}[b]{\linewidth}\raggedleft
Grand-Cites
\end{minipage} \\
\midrule\noalign{}
\endhead
\bottomrule\noalign{}
\endlastfoot
Jonathan Schaffer
\citeyearpar{WOS000272855000002}
``Monism: The Priority of the Whole'' & 348 & 3372 \\
David Christensen
\citeyearpar{WOS000207419300002}
``Epistemology of Disagreement: The Good News'' & 303 & 3012 \\
Adam Elga
\citeyearpar{WOS000249103800005}
``Reflection and Disagreement'' & 315 & 2910 \\
John Hawthorne and Jason Stanley
\citeyearpar{WOS000262624000001}
``Knowledge and Action'' & 287 & 2759 \\
Niko Kolodny
\citeyearpar{WOS000231037900002}
``Why Be Rational?'' & 275 & 2222 \\
Sharon Street
\citeyearpar{WOS000234431300006}
``A Darwinian Dilemma for Realist Theories of Value'' & 300 & 2000 \\
Andy Egan
\citeyearpar{WOS000245280800001}
``Epistemic Modals, Relativism and Assertion'' & 128 & 1962 \\
Wlodek Rabinowicz and Toni Rønnow‐Rasmussen
\citeyearpar{WOS000222134800001}
``The Strike of the Demon: On Fitting Pro-Attitudes and Value'' & 186 &
1871 \\
Jessica M. Wilson
\citeyearpar{WOS000344393500001}
``No Work for a Theory of Grounding'' & 219 & 1795 \\
Nishi Shah and J. David Velleman
\citeyearpar{WOS000240571800003}
``Doxastic Deliberation'' & 244 & 1794 \\

\end{longtable}

\section{Conclusion}\label{conclusion}

I started this project with two beliefs about citations in philosophy:
they had traditionally been to rather old articles, but now more recent
works were being cited. Both beliefs were wrong. Citations of philosophy
journals in philosophy journals had traditionally been, on the whole, to
recent articles, but in recent years they were getting older.

One reason for this is that technological changes have made a bigger
difference to search than to distribution. Electronic databases and
online journals have made older articles, particularly non-classic older
articles, far more discoverable than they were when researchers relied
on card catalogs and physical library visits.

Beyond technology, two cultural shifts have led to citations getting
older. First, journal articles have become canonical texts in their own
right rather than trial runs for books. Second, philosophers have
increasingly adopted the more expansive citation habits of other fields,
and there are a lot more citations in passing to classic works. One
reason for that might be that philosophers are engaging more with
cognate fields. Another is that the increased specialisation and
fragmentation of philosophy means writers can make fewer assumptions
about what background knowledge their readers have.

After adjusting for both age effects (i.e., articles are cited most when
2-5 years old) and period effects (i.e., there are many times more
citations in recent years), cohort effects reveal that articles from the
1970s and 2000s were disproportionately influential. The early 1970s
pattern is unsurprising, since revolutionary work by Frankfurt, Rawls,
Thomson, Kripke, Lewis, and others transformed nearly every
philosophical subfield, with much of the transformation happening in
journals. The 2000s pattern is more striking. Articles from that decade
were cited less than late-20th-century articles immediately after
publication but far more as time passed, suggesting they have made a
lasting impact.

Part of the story here is the growing centrality of epistemology to
philosophy, as revealed in the grand-citations data. But another part is
that the ending of the modal era meant that there was a huge number of
classic papers whose importance to post-modal philosophy was unclear.
The result is that papers from the 1980s and 1990s in metaphysics, mind,
and language, address questions that aren't the focus of contemporary
philosophy, and so they don't get cited.

While there is no stopping philosophical fashions changing, one of the
reasons for the continued high citation rates of articles from the 2000s
is that very few debates from the 2000s have faded away as dramatically
as the supervenience debates from earlier decades faded away. There is
less discussion now of fictionalism, vagueness, or self-locating belief
than there was twenty years ago, but the prominent articles from those
fields (e.g., \citet{WOSA1990DR99100001}, \citet{Fara2000}, and
\citet{WOS000086383700001}) are still very widely cited. While new
debates have opened up (e.g., grounding, or slurs), these have mostly
taken place alongside the large debates from the 2000s. What they don't
take place alongside are the large debates in mind, metaphysics, and
language from the 1980s and 1990s.

That difference is part of what is showing up in
Figure~\ref{fig-cohort-long}, and it raises a pair of interesting
questions. One is whether we will see some debates eventually fade away
from the journals, as for instance the debates about narrow content did.
But a second is whether we have entered a new era, where philosophical
topics are added to the field, but rarely subtracted, with a resulting
increase in specialisation, fragmentation, and, presumably,
inter-disciplinary collaboration. The citation data we have doesn't
provide much help in predicting the answers to these questions, but
hopefully it is helpful in telling us what has happened over the last
few decades, and raising new questions about what will happen in the
next few decades.

\section{Appendix: Summary Statistics}\label{sec-statistics}

Table~\ref{tbl-list-of-journals} shows the journals used in this paper.

\begin{longtable}[]{@{}
  >{\raggedright\arraybackslash}p{(\linewidth - 10\tabcolsep) * \real{0.4274}}
  >{\raggedleft\arraybackslash}p{(\linewidth - 10\tabcolsep) * \real{0.0940}}
  >{\raggedleft\arraybackslash}p{(\linewidth - 10\tabcolsep) * \real{0.0855}}
  >{\raggedleft\arraybackslash}p{(\linewidth - 10\tabcolsep) * \real{0.0769}}
  >{\raggedleft\arraybackslash}p{(\linewidth - 10\tabcolsep) * \real{0.1624}}
  >{\raggedleft\arraybackslash}p{(\linewidth - 10\tabcolsep) * \real{0.1538}}@{}}

\caption{\label{tbl-list-of-journals}Journals used in this paper}

\tabularnewline

\toprule\noalign{}
\begin{minipage}[b]{\linewidth}\raggedright
Journal
\end{minipage} & \begin{minipage}[b]{\linewidth}\raggedleft
First Year
\end{minipage} & \begin{minipage}[b]{\linewidth}\raggedleft
Last Year
\end{minipage} & \begin{minipage}[b]{\linewidth}\raggedleft
Articles
\end{minipage} & \begin{minipage}[b]{\linewidth}\raggedleft
Outbound Citations
\end{minipage} & \begin{minipage}[b]{\linewidth}\raggedleft
Inbound Citations
\end{minipage} \\
\midrule\noalign{}
\endhead
\bottomrule\noalign{}
\endlastfoot
American Philosophical Quarterly & 1964 & 2024 & 1835 & 7759 & 10298 \\
Analysis & 1975 & 2024 & 2719 & 7494 & 14835 \\
Analytic Philosophy & 2016 & 2024 & 190 & 2219 & 501 \\
Archiv für Geschichte der Philosophie & 1975 & 2024 & 672 & 1598 &
1020 \\
Australasian Journal of Philosophy & 1975 & 2024 & 1736 & 10463 &
13449 \\
Biology and Philosophy & 1988 & 2024 & 1225 & 6179 & 4827 \\
British Journal for the History of Philosophy & 2007 & 2024 & 834 & 2465
& 1113 \\
British Journal for the Philosophy of Science & 1956 & 2024 & 1620 &
9330 & 13042 \\
British Journal of Aesthetics & 1975 & 2024 & 1436 & 3556 & 3614 \\
Bulletin of Symbolic Logic & 1997 & 2024 & 443 & 1326 & 1139 \\
Canadian Journal of Philosophy & 1975 & 2023 & 1552 & 7772 & 5737 \\
Croatian Journal of Philosophy & 2007 & 2024 & 376 & 1901 & 299 \\
Dialogue & 1975 & 2024 & 1555 & 3697 & 1107 \\
Economics and Philosophy & 1986 & 2024 & 580 & 2729 & 2417 \\
Episteme & 2005 & 2024 & 586 & 5196 & 3128 \\
Ergo & 2016 & 2024 & 386 & 5090 & 867 \\
Erkenntnis & 2000 & 2024 & 1789 & 17334 & 7645 \\
Ethical Theory and Moral Practice & 2008 & 2024 & 902 & 5875 & 2157 \\
Ethics & 1956 & 2024 & 1647 & 5758 & 15706 \\
Ethics and Information Technology & 2001 & 2024 & 567 & 1953 & 1032 \\
European Journal for Philosophy of Science & 2011 & 2024 & 564 & 5984 &
1475 \\
European Journal of Philosophy & 1998 & 2024 & 994 & 5841 & 3023 \\
Heythrop Journal & 1975 & 2024 & 1559 & 874 & 355 \\
History and Philosophy of Logic & 1992 & 2024 & 522 & 1447 & 930 \\
Hypatia & 2009 & 2024 & 683 & 1556 & 1712 \\
Inquiry & 1966 & 2024 & 1646 & 7186 & 4545 \\
International Journal for Philosophy of Religion & 1975 & 2024 & 1149 &
2184 & 1135 \\
International Philosophical Quarterly & 1961 & 2023 & 1588 & 1464 &
713 \\
Journal of Aesthetics and Art Criticism & 1975 & 2024 & 1539 & 3747 &
3757 \\
Journal of Applied Philosophy & 2006 & 2024 & 666 & 3473 & 1304 \\
Journal of Chinese Philosophy & 1973 & 2024 & 1278 & 1053 & 966 \\
Journal of Consciousness Studies & 2000 & 2024 & 1526 & 4945 & 3661 \\
Journal of Indian Philosophy & 1975 & 2024 & 1107 & 1477 & 1473 \\
Journal of Medical Ethics & 1975 & 2024 & 4347 & 5904 & 5106 \\
Journal of Moral Philosophy & 2005 & 2024 & 392 & 2449 & 980 \\
Journal of Philosophical Logic & 1972 & 2024 & 1497 & 7755 & 9812 \\
Journal of Philosophical Research & 2005 & 2024 & 463 & 2097 & 575 \\
Journal of Philosophy & 1956 & 2024 & 2761 & 7299 & 37891 \\
Journal of Political Philosophy & 1998 & 2023 & 609 & 2303 & 2891 \\
Journal of Social Philosophy & 2008 & 2024 & 508 & 2202 & 883 \\
Journal of Symbolic Logic & 1966 & 2024 & 4363 & 6757 & 10588 \\
Journal of Value Inquiry & 1980 & 2024 & 1369 & 2975 & 1466 \\
Journal of the American Philosophical Association & 2015 & 2024 & 341 &
2764 & 938 \\
Journal of the History of Ideas & 1956 & 2024 & 2212 & 995 & 1705 \\
Journal of the History of Philosophy & 1975 & 2024 & 1138 & 2800 &
3065 \\
Journal of the Philosophy of History & 2010 & 2024 & 284 & 618 & 187 \\
Kant-Studien & 1975 & 2024 & 1134 & 1710 & 1710 \\
Kantian Review & 2010 & 2024 & 338 & 1560 & 655 \\
Kennedy Institute of Ethics Journal & 1995 & 2024 & 574 & 1234 & 933 \\
Law and Philosophy & 1982 & 2024 & 852 & 2566 & 1519 \\
Linguistics and Philosophy & 1979 & 2024 & 878 & 5014 & 6408 \\
Logique et Analyse & 2007 & 2021 & 340 & 1565 & 336 \\
Metaphilosophy & 1975 & 2024 & 1562 & 4541 & 2837 \\
Mind & 1956 & 2024 & 1980 & 8454 & 18449 \\
Mind \& Language & 1994 & 2024 & 893 & 5807 & 6366 \\
Minds and Machines & 1992 & 2024 & 756 & 3442 & 1965 \\
Monist & 1963 & 2024 & 1976 & 4326 & 6448 \\
Notre Dame Journal of Formal Logic & 2009 & 2024 & 486 & 1661 & 707 \\
Noûs & 1975 & 2024 & 1480 & 11792 & 20567 \\
Pacific Philosophical Quarterly & 1980 & 2024 & 1231 & 7610 & 6544 \\
Philosophers' Imprint & 2010 & 2024 & 402 & 5177 & 3301 \\
Philosophia & 1975 & 2024 & 2249 & 10028 & 2919 \\
Philosophia Mathematica & 2008 & 2024 & 243 & 1613 & 897 \\
Philosophical Explorations & 2008 & 2024 & 390 & 2857 & 1272 \\
Philosophical Forum & 1971 & 2024 & 851 & 1726 & 613 \\
Philosophical Investigations & 1983 & 2024 & 708 & 1057 & 588 \\
Philosophical Papers & 2009 & 2023 & 234 & 1379 & 445 \\
Philosophical Perspectives & 2007 & 2023 & 305 & 3380 & 3493 \\
Philosophical Psychology & 1991 & 2024 & 1312 & 7178 & 4226 \\
Philosophical Quarterly & 1975 & 2024 & 1450 & 8660 & 10728 \\
Philosophical Review & 1956 & 2024 & 1034 & 5220 & 25891 \\
Philosophical Studies & 1956 & 2024 & 5485 & 35127 & 38247 \\
Philosophy & 1956 & 2024 & 1811 & 2452 & 3611 \\
Philosophy \& Public Affairs & 1971 & 2024 & 733 & 2592 & 11779 \\
Philosophy Compass & 2015 & 2024 & 651 & 7940 & 1750 \\
Philosophy East and West & 1966 & 2024 & 1604 & 1981 & 1758 \\
Philosophy and Phenomenological Research & 1956 & 2024 & 3273 & 15151 &
21537 \\
Philosophy and Rhetoric & 1975 & 2024 & 927 & 1170 & 893 \\
Philosophy of Science & 1956 & 2024 & 3259 & 12888 & 25003 \\
Philosophy of the Social Sciences & 1975 & 2024 & 998 & 2556 & 1708 \\
Phronesis & 1975 & 2024 & 783 & 1227 & 1620 \\
Politics, Philosophy and Economics & 2008 & 2024 & 325 & 2025 & 782 \\
Ratio & 1974 & 2024 & 1090 & 3699 & 3607 \\
Res Philosophica & 2013 & 2024 & 360 & 2176 & 737 \\
Review of Metaphysics & 1956 & 2024 & 1616 & 1499 & 2435 \\
Review of Symbolic Logic & 2008 & 2024 & 578 & 3906 & 2722 \\
Russell & 1981 & 2024 & 345 & 427 & 303 \\
Social Epistemology & 2011 & 2024 & 489 & 2381 & 1034 \\
Social Philosophy and Policy & 1983 & 2024 & 973 & 2109 & 2556 \\
South African Journal of Philosophy & 1987 & 2024 & 794 & 1868 & 650 \\
Southern Journal of Philosophy & 1976 & 2024 & 1984 & 5195 & 3057 \\
Studia Logica & 2010 & 2024 & 734 & 2418 & 1008 \\
Studies in History and Philosophy of Science & 1974 & 2024 & 1832 & 8893
& 6225 \\
Synthese & 1966 & 2024 & 7770 & 64831 & 32906 \\
Theoria & 2007 & 2024 & 459 & 3902 & 727 \\
Theory and Decision & 1970 & 2024 & 1937 & 2103 & 2285 \\
Thought & 2016 & 2022 & 219 & 1537 & 455 \\
Topoi & 1982 & 2024 & 1327 & 5554 & 2365 \\
Transactions of the Charles S. Peirce Society & 1975 & 2024 & 1181 &
1885 & 1567 \\
Utilitas & 2009 & 2024 & 391 & 2481 & 1160 \\

\end{longtable}

What I've called an \emph{article} here is anything that either (a)
marked as an article or research-article by WoS, or (b) marked as a
review, discussion, or note by WoS and is at least 15 pages long. I
needed to include (b) because some very important works (e.g.,
\citet{WOSA1963CEU0700001} and \citet{WOS000272855000002}) were not
recorded as articles by WoS.

The years here are \textbf{not} the first and last years that the
journals published, but the earliest and latest years that are in the
WoS index (as of the time I pulled the data). As mentioned in the main
text, this makes a big difference for some journals, especially
\emph{Analysis}.

The way WoS handles the `supplements' to \emph{Noûs}, i.e.,
\emph{Philosophical Perspectives} and \emph{Philosophical Issues}, is a
little uneven. Some years these are recorded as being their own thing,
i.e., with a source name of \emph{Philosophical Perspectives} or
\emph{Philosophical Issues}; and some years they are recorded as special
issues of \emph{Noûs}. When they were listed as special issues, the
citations were extremely unreliable. Some high profile articles are
recorded as having no citations until several years after publication.
The bibliographic information for the articles themselves was also
spotty. So I've manually removed all records that were listed as special
or supplementary issues of \emph{Noûs} (and similarly removed the
citations to those articles that did get tracked). What you see here are
just the standalone issues of \emph{Philosophical Perspectives}.


\bibliography{/Users/weath/Documents/quarto-articles/brian-quarto.bib,/Users/weath/Documents/citations-2025/autobib.bib}


\noindent


\end{document}
