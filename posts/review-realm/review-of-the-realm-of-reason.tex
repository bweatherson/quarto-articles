% Options for packages loaded elsewhere
\PassOptionsToPackage{unicode}{hyperref}
\PassOptionsToPackage{hyphens}{url}
%
\documentclass[
  10pt,
  letterpaper,
  DIV=11,
  numbers=noendperiod,
  twoside]{scrartcl}

\usepackage{amsmath,amssymb}
\usepackage{setspace}
\usepackage{iftex}
\ifPDFTeX
  \usepackage[T1]{fontenc}
  \usepackage[utf8]{inputenc}
  \usepackage{textcomp} % provide euro and other symbols
\else % if luatex or xetex
  \usepackage{unicode-math}
  \defaultfontfeatures{Scale=MatchLowercase}
  \defaultfontfeatures[\rmfamily]{Ligatures=TeX,Scale=1}
\fi
\usepackage{lmodern}
\ifPDFTeX\else  
    % xetex/luatex font selection
  \setmainfont[ItalicFont=EB Garamond Italic,BoldFont=EB Garamond
Bold]{EB Garamond Math}
  \setsansfont[]{Europa-Bold}
  \setmathfont[]{Garamond-Math}
\fi
% Use upquote if available, for straight quotes in verbatim environments
\IfFileExists{upquote.sty}{\usepackage{upquote}}{}
\IfFileExists{microtype.sty}{% use microtype if available
  \usepackage[]{microtype}
  \UseMicrotypeSet[protrusion]{basicmath} % disable protrusion for tt fonts
}{}
\usepackage{xcolor}
\usepackage[left=1in, right=1in, top=0.8in, bottom=0.8in,
paperheight=9.5in, paperwidth=6.5in, includemp=TRUE, marginparwidth=0in,
marginparsep=0in]{geometry}
\setlength{\emergencystretch}{3em} % prevent overfull lines
\setcounter{secnumdepth}{3}
% Make \paragraph and \subparagraph free-standing
\ifx\paragraph\undefined\else
  \let\oldparagraph\paragraph
  \renewcommand{\paragraph}[1]{\oldparagraph{#1}\mbox{}}
\fi
\ifx\subparagraph\undefined\else
  \let\oldsubparagraph\subparagraph
  \renewcommand{\subparagraph}[1]{\oldsubparagraph{#1}\mbox{}}
\fi


\providecommand{\tightlist}{%
  \setlength{\itemsep}{0pt}\setlength{\parskip}{0pt}}\usepackage{longtable,booktabs,array}
\usepackage{calc} % for calculating minipage widths
% Correct order of tables after \paragraph or \subparagraph
\usepackage{etoolbox}
\makeatletter
\patchcmd\longtable{\par}{\if@noskipsec\mbox{}\fi\par}{}{}
\makeatother
% Allow footnotes in longtable head/foot
\IfFileExists{footnotehyper.sty}{\usepackage{footnotehyper}}{\usepackage{footnote}}
\makesavenoteenv{longtable}
\usepackage{graphicx}
\makeatletter
\def\maxwidth{\ifdim\Gin@nat@width>\linewidth\linewidth\else\Gin@nat@width\fi}
\def\maxheight{\ifdim\Gin@nat@height>\textheight\textheight\else\Gin@nat@height\fi}
\makeatother
% Scale images if necessary, so that they will not overflow the page
% margins by default, and it is still possible to overwrite the defaults
% using explicit options in \includegraphics[width, height, ...]{}
\setkeys{Gin}{width=\maxwidth,height=\maxheight,keepaspectratio}
% Set default figure placement to htbp
\makeatletter
\def\fps@figure{htbp}
\makeatother

\setlength\heavyrulewidth{0ex}
\setlength\lightrulewidth{0ex}
\usepackage[automark]{scrlayer-scrpage}
\clearpairofpagestyles
\cehead{
  Brian Weatherson
  }
\cohead{
  Review of “The Realm of Reason”
  }
\ohead{\bfseries \pagemark}
\cfoot{}
\makeatletter
\newcommand*\NoIndentAfterEnv[1]{%
  \AfterEndEnvironment{#1}{\par\@afterindentfalse\@afterheading}}
\makeatother
\NoIndentAfterEnv{itemize}
\NoIndentAfterEnv{enumerate}
\NoIndentAfterEnv{description}
\NoIndentAfterEnv{quote}
\NoIndentAfterEnv{equation}
\NoIndentAfterEnv{longtable}
\NoIndentAfterEnv{abstract}
\renewenvironment{abstract}
 {\vspace{-1.25cm}
 \quotation\small\noindent\rule{\linewidth}{.5pt}\par\smallskip
 \noindent }
 {\par\noindent\rule{\linewidth}{.5pt}\endquotation}
\KOMAoption{captions}{tableheading}
\makeatletter
\@ifpackageloaded{caption}{}{\usepackage{caption}}
\AtBeginDocument{%
\ifdefined\contentsname
  \renewcommand*\contentsname{Table of contents}
\else
  \newcommand\contentsname{Table of contents}
\fi
\ifdefined\listfigurename
  \renewcommand*\listfigurename{List of Figures}
\else
  \newcommand\listfigurename{List of Figures}
\fi
\ifdefined\listtablename
  \renewcommand*\listtablename{List of Tables}
\else
  \newcommand\listtablename{List of Tables}
\fi
\ifdefined\figurename
  \renewcommand*\figurename{Figure}
\else
  \newcommand\figurename{Figure}
\fi
\ifdefined\tablename
  \renewcommand*\tablename{Table}
\else
  \newcommand\tablename{Table}
\fi
}
\@ifpackageloaded{float}{}{\usepackage{float}}
\floatstyle{ruled}
\@ifundefined{c@chapter}{\newfloat{codelisting}{h}{lop}}{\newfloat{codelisting}{h}{lop}[chapter]}
\floatname{codelisting}{Listing}
\newcommand*\listoflistings{\listof{codelisting}{List of Listings}}
\makeatother
\makeatletter
\makeatother
\makeatletter
\@ifpackageloaded{caption}{}{\usepackage{caption}}
\@ifpackageloaded{subcaption}{}{\usepackage{subcaption}}
\makeatother
\ifLuaTeX
  \usepackage{selnolig}  % disable illegal ligatures
\fi
\IfFileExists{bookmark.sty}{\usepackage{bookmark}}{\usepackage{hyperref}}
\IfFileExists{xurl.sty}{\usepackage{xurl}}{} % add URL line breaks if available
\urlstyle{same} % disable monospaced font for URLs
\hypersetup{
  pdftitle={Review of ``The Realm of Reason''},
  pdfauthor={Brian Weatherson},
  hidelinks,
  pdfcreator={LaTeX via pandoc}}

\title{Review of ``The Realm of Reason''}
\author{Brian Weatherson}
\date{2004}

\begin{document}
\maketitle
\begin{abstract}
Review of Christopher Peacocke, ``The Realm of Reason''. Oxford:
Clarendon Press, 2004
\end{abstract}

\setstretch{1.1}
Some of what we know we know by experience and some by reason. It's
experience not reason that teaches that Arsenal ended last season with
90 points and Chelsea with 79, it's reason not experience that teaches
90 is greater than 79, and, arguably, it's the two together that teach
that Arsenal ended with more points than Chelsea. One useful
classification of philosophers is by the relative importance they assign
to experience and reason in grounding what we know. Empiricists (on one
reading of that term) play down reason, sometimes going so far as to
declare that anything known by a means other than experience must be a
mere matter of definition. Rationalists play reason up.

Christopher Peacocke is firmly in the rationalist camp, and \emph{The
Realm of Reason} is an attempt to lay out what he takes rationalism to
be. It gives his preferred version of rationalism and some arguments in
its favour. It's much too much to attempt in a short book and it isn't
entirely persuasive on any of the applications, but it is a grand vision
for what a global rationalism might look like, one that might prove
attractive even if the details need work. Given the length of the book a
surprising amount of time is spent on relatively abstruse details.
Peacocke provides a particularly careful account of what distinguishes
rationalists from empiricists and does a lot of work classifying and
adjudicating between rationalisms of various strengths. These are the
best parts of the book, but also the least accessible.

Peacocke's preferred version of rationalism has two distinctive
components. First, he focuses not on beliefs, as is usual, but on the
``transitions'' between representational states that occur in thought,
as when we move to a new belief on the basis of one we already have.
Mental representational states are often beliefs, but they also include
things, like perceptions, that have representational content without
necessarily being believed. Peacocke's rationalist claim is that for any
justified transition, there's an a priori explanation of why it is
justified. Second, he insists that this explanation rely crucially on
the contents of the states involved in the transition.

So we get a quite strong ``foundationalist'' epistemology. Experience
provides the foundations for empirical knowledge, but how we get from
there to what we know is entirely in the domain of reason. It is
famously difficult to justify many steps by reason alone, and the most
pressing is the very first: How do we justify the transition from
appearances to reality, such as the transition from \emph{That looks
crooked} to \emph{That is crooked}? Some philosophers have thought that
we need to link appearance and reality so closely that the link is
infallible. Peacocke doesn't take that line, so he has to justify the
transition some other way.

Descartes faced a similar problem when trying to get over his radical
doubt about the existence of the material world, and solved it by appeal
to God. We can tell a priori, he thought, that a benevolent God exists,
and a benevolent God will not let us be deceived about this matter, at
least when we are careful enough to rely on clear and distinct
perceptions. Now Descartes had to be careful here to only appeal to a
priori reasons for belief in God. He couldn't, for instance, argue from
the apparent design of the universe to the existence of a designer,
because we can't tell at this stage whether the apparent design is
merely an artifact of our defective perceptual faculties. Indeed, we
can't rely on any apparent fact about the external world until we've
determined that appearances are a good guide to reality. So we need to
argue for the existence of God without appeal to perception, and then
use God's existence to justify future reliance on perception.

In keeping with the spirit of the age, Peacocke updates Descartes's
strategy by replacing God with Darwin. Very roughly, Peacocke argues
that the best explanation of our having representative capacities at all
is that we are the products of a long process of natural selection. And
if we are the products of a long process of selection, then we probably
have accurate representations. If those two claims can be justified a
priori we have an a priori argument to the (prima facie, probable)
accuracy of our representations.

Less roughly, Peacocke argues for a ``Complexity Reduction Principle''.
We are entitled, on a priori grounds, to believe that complex phenomena
have explanations, and we are entitled to regard simpler explanations as
more probably true than more complex ones. That we have representations
at all is a complex matter. How might it be explained? One explanation
is via Divine creation. Another is that we are ``brains-in-vats'' living
in a virtual reality world dreamt up by some quirky scientist (cf
\emph{The Matrix}). But neither of these explanations really reduces the
complexity, since in each case we need to appeal to a thing (God, the
scientist) that already has representational capacities. A simpler
explanation, allegedly, is that we are the product of natural selection
and having accurate representations is selected for. This is certainly a
novel argument for Darwinism. It isn't why they teach natural selection
to biology students. And of course it has flaws. Peacocke does little to
show that there are no better explanations of our having
representations. Nor does he address the question of how complicated
hereditary mechanisms must be if they are to support natural selection.
Arguably they are much more complicated than is needed for
representation, so Darwin doesn't help reduce complexity here.

So it's not clear Peacocke's rationalism can get past step one; but
let's see what would happen next. To go beyond particular perceptions,
in acquiring knowledge, we need induction. Peacocke takes the basic form
of enumerative induction to be the (defeasible) inference from \emph{All
the (many and varied) observed Fs have been Gs} to \emph{All Fs are Gs}.
The observation of only Gs, and no non-Gs, amongst these many and varied
Fs is a complex fact, and its best (ie simplest) explanation is
sometimes that all the Fs are Gs. Peacocke argues that in these cases
this explanation is the a priori justification of the transition, and in
only these cases is the transition justified; he concludes that
induction is acceptable by rationalist lights.

The chapter on induction is only fifteen pages long, and it really needs
to be much longer. Peacocke sets out the position just outlined, and
compares it in some detail to a similar position advocated by Gilbert
Harman, and that's it. There is no discussion of what we do when most,
rather than all, the observed Fs have been Gs, even though that's surely
the more important practical case. There's no discussion of the case
that's frequently central to modern discussions on induction---the case
in which a certain (stable) ratio of the Fs are Gs. Peacocke only talks
about the special case when \emph{all} Fs are Gs, and it isn't obvious
that the discussion generalizes. There is no discussion of rationalist
alternatives, such as Keynes's justification of enumerative induction in
terms of analogical inference, or D. C. Williams's probabilistic defence
of induction. And there's no discussion of empiricist attempts to
justify induction a posteriori, or to do without it. Even if Peacocke's
suggested justification works, and it is at least a serious contender, a
persuasive treatment of induction should have dealt with at least some
of these points.

The final two chapters discuss moral beliefs. Again, Peacocke thinks
that all the inferences we make in order to get from our perceptual
beliefs to our moral beliefs can be justified a priori. His view is that
we can come to know a priori some moral principles. And we can know
contingent moral facts, such as that someone's giving £1000 to Oxfam is
morally praiseworthy, by carrying out the following inference. The
person, say Joe, helped other people in need. (We learn this by
experience.) Helping those in need is morally praiseworthy. (We learn
this moral principle by deploying our reason.) Hence what Joe did is
morally praiseworthy. But there's a problem here, and Peacocke never
fully addresses it. It's only \emph{prima facie} true that helping those
in need is morally praiseworthy. There are always exceptions to the
principle. If Joe's children starved to death because that donation was
the last money Joe had to buy them food, the donation wasn't morally
praiseworthy. Moreover, it is just about impossible to state the
exceptions without using moral language. So it is far from clear how we
are meant to come to know that this case is not one of the exceptions,
because knowing this requires both empirical knowledge and moral
sensitivity. From a ``principleist'' position like Peacocke's, knowing
this is not one of the exceptions seems just as hard as the original
problem of coming to know that the action was praiseworthy. So it seems
the rationalist still has work to do here.

One can easily get the feeling from this book that rationalism runs into
problems as soon as we try to apply it to real-world cases. But it isn't
obvious these are deep problems with rationalism, and in particular it
isn't clear that the problems can't be fixed with relatively minor
adjustments. Even if there are difficulties in application throughout
\emph{The Realm of Reason}, there is a lot of important philosophical
work going on beneath the surface. Peacocke's best work is done in
classifying the various types of rationalist position that are
available, and motivating the kind of view he wants to defend. This
material remains valuable, highly valuable to anyone wanting to draw a
plausible rationalist picture, even if his real-world applications are
not yet perfect.

\vspace{1cm}



Published in the \emph{Times Literary Supplement}, 2004 \\

\end{document}
