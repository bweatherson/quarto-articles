% Options for packages loaded elsewhere
\PassOptionsToPackage{unicode}{hyperref}
\PassOptionsToPackage{hyphens}{url}
%
\documentclass[
  11pt,
  letterpaper,
  DIV=11,
  numbers=noendperiod,
  twoside]{scrartcl}

\usepackage{amsmath,amssymb}
\usepackage{setspace}
\usepackage{iftex}
\ifPDFTeX
  \usepackage[T1]{fontenc}
  \usepackage[utf8]{inputenc}
  \usepackage{textcomp} % provide euro and other symbols
\else % if luatex or xetex
  \usepackage{unicode-math}
  \defaultfontfeatures{Scale=MatchLowercase}
  \defaultfontfeatures[\rmfamily]{Ligatures=TeX,Scale=1}
\fi
\usepackage{lmodern}
\ifPDFTeX\else  
    % xetex/luatex font selection
    \setmainfont[ItalicFont=EB Garamond Italic,BoldFont=EB Garamond
Bold]{EB Garamond Math}
    \setsansfont[]{EB Garamond}
  \setmathfont[]{Garamond-Math}
\fi
% Use upquote if available, for straight quotes in verbatim environments
\IfFileExists{upquote.sty}{\usepackage{upquote}}{}
\IfFileExists{microtype.sty}{% use microtype if available
  \usepackage[]{microtype}
  \UseMicrotypeSet[protrusion]{basicmath} % disable protrusion for tt fonts
}{}
\usepackage{xcolor}
\usepackage[left=1.1in, right=1in, top=0.8in, bottom=0.8in,
paperheight=9.5in, paperwidth=7in, includemp=TRUE, marginparwidth=0in,
marginparsep=0in]{geometry}
\setlength{\emergencystretch}{3em} % prevent overfull lines
\setcounter{secnumdepth}{3}
% Make \paragraph and \subparagraph free-standing
\makeatletter
\ifx\paragraph\undefined\else
  \let\oldparagraph\paragraph
  \renewcommand{\paragraph}{
    \@ifstar
      \xxxParagraphStar
      \xxxParagraphNoStar
  }
  \newcommand{\xxxParagraphStar}[1]{\oldparagraph*{#1}\mbox{}}
  \newcommand{\xxxParagraphNoStar}[1]{\oldparagraph{#1}\mbox{}}
\fi
\ifx\subparagraph\undefined\else
  \let\oldsubparagraph\subparagraph
  \renewcommand{\subparagraph}{
    \@ifstar
      \xxxSubParagraphStar
      \xxxSubParagraphNoStar
  }
  \newcommand{\xxxSubParagraphStar}[1]{\oldsubparagraph*{#1}\mbox{}}
  \newcommand{\xxxSubParagraphNoStar}[1]{\oldsubparagraph{#1}\mbox{}}
\fi
\makeatother


\providecommand{\tightlist}{%
  \setlength{\itemsep}{0pt}\setlength{\parskip}{0pt}}\usepackage{longtable,booktabs,array}
\usepackage{calc} % for calculating minipage widths
% Correct order of tables after \paragraph or \subparagraph
\usepackage{etoolbox}
\makeatletter
\patchcmd\longtable{\par}{\if@noskipsec\mbox{}\fi\par}{}{}
\makeatother
% Allow footnotes in longtable head/foot
\IfFileExists{footnotehyper.sty}{\usepackage{footnotehyper}}{\usepackage{footnote}}
\makesavenoteenv{longtable}
\usepackage{graphicx}
\makeatletter
\newsavebox\pandoc@box
\newcommand*\pandocbounded[1]{% scales image to fit in text height/width
  \sbox\pandoc@box{#1}%
  \Gscale@div\@tempa{\textheight}{\dimexpr\ht\pandoc@box+\dp\pandoc@box\relax}%
  \Gscale@div\@tempb{\linewidth}{\wd\pandoc@box}%
  \ifdim\@tempb\p@<\@tempa\p@\let\@tempa\@tempb\fi% select the smaller of both
  \ifdim\@tempa\p@<\p@\scalebox{\@tempa}{\usebox\pandoc@box}%
  \else\usebox{\pandoc@box}%
  \fi%
}
% Set default figure placement to htbp
\def\fps@figure{htbp}
\makeatother

\KOMAoption{captions}{tableheading}
\makeatletter
\@ifpackageloaded{caption}{}{\usepackage{caption}}
\AtBeginDocument{%
\ifdefined\contentsname
  \renewcommand*\contentsname{Table of contents}
\else
  \newcommand\contentsname{Table of contents}
\fi
\ifdefined\listfigurename
  \renewcommand*\listfigurename{List of Figures}
\else
  \newcommand\listfigurename{List of Figures}
\fi
\ifdefined\listtablename
  \renewcommand*\listtablename{List of Tables}
\else
  \newcommand\listtablename{List of Tables}
\fi
\ifdefined\figurename
  \renewcommand*\figurename{Figure}
\else
  \newcommand\figurename{Figure}
\fi
\ifdefined\tablename
  \renewcommand*\tablename{Table}
\else
  \newcommand\tablename{Table}
\fi
}
\@ifpackageloaded{float}{}{\usepackage{float}}
\floatstyle{ruled}
\@ifundefined{c@chapter}{\newfloat{codelisting}{h}{lop}}{\newfloat{codelisting}{h}{lop}[chapter]}
\floatname{codelisting}{Listing}
\newcommand*\listoflistings{\listof{codelisting}{List of Listings}}
\makeatother
\makeatletter
\makeatother
\makeatletter
\@ifpackageloaded{caption}{}{\usepackage{caption}}
\@ifpackageloaded{subcaption}{}{\usepackage{subcaption}}
\makeatother

\usepackage{bookmark}

\IfFileExists{xurl.sty}{\usepackage{xurl}}{} % add URL line breaks if available
\urlstyle{same} % disable monospaced font for URLs
\hypersetup{
  pdftitle={Comments on Alexander Wentzell's ``An Argument for Mentalism About Epistemic Justification''},
  pdfauthor={Brian Weatherson},
  hidelinks,
  pdfcreator={LaTeX via pandoc}}


\title{Comments on Alexander Wentzell's ``An Argument for Mentalism
About Epistemic Justification''}
\author{Brian Weatherson}
\date{2025-04-11}

\begin{document}
\maketitle
\begin{abstract}
Comments for a session next week at the APA Pacific
\end{abstract}


\setstretch{1.1}
Alexander Wentzell's nice paper provides a reason for preferring
mentalism, the doctrine that what's rational to believe supervenes on
mental states, to accessibilism, the doctrine that what's rational to
believe supervenes on which states are accessible. The argument is that
there are intuitively good guesses which turn on inaccessible states,
like in inattentional blindness experiments. And if information which is
received but not accessible can be the basis for rational guessing, it
can be the basis for rational believing. I'll make three brief comments,
one about the last conditional, one about whether this argument
generalises further than Alexander might want it to, and one about the
notion of a `good guess'.

First, I think there is space for the accessibilist to concede the point
about guesses, but deny that it is relevant to \emph{belief}. Belief is,
as Pamela Hieronymi and Jane Friedman have stressed, a settling
attitude. To believe \emph{p} well, one must not only have considerably
more evidence for \emph{p} than for ¬\emph{p}, one must have reason to
regard the question of \emph{p} as settled. In typical cases, and we
might want to fuss about the atypical cases but let them slide for now,
it means having reason to close off inquiry into \emph{p}, and not go
looking for more evidence one way or the other. Could inaccessible
information be a reason to close inquiry in this way? I'm not an
accessibilist, so I'm not the best judge, but it feels there is
something to work with here.

Set that aside, and let's go with the idea that Alex's argument works,
and indeed let's take on board the nice idea that thinking about the
rationality of guesses is a good way to get novel insights into the
rationality of beliefs. Alexander argues that this means mental states
which are inaccessible are relevant to rationality. What I want us to
think about for a bit is whether the same argument shows that non-mental
states which are inaccessible are also relevant to rationality.

The first thing to note is that humans have ways of taking in, and
responding to, information in ways that do not involve what we normally
think of as mental states. I touch a hot stove (maybe the one labeled
\emph{Trade War with China}), and pretty quickly pull my hand away. Why?
Well, I get the information that it's hot, and (as I understand the
science) that information leads to action without involving my brain.
The signal from the hand maybe makes it as far as my spinal column when
the return message \emph{Stop doing that you idiot} comes flying back
down. We can respond to information not just before its accessible, but
before (I'd say) it's even cognised.

This might be relevant to real world cases. Expert firefighters know to
leave a building when the floor is \emph{spongy}. What makes this an
expert skill is that in any burning building the floor is a bit spongy;
the experts apparently have the ability to tell the difference between
what's normal (or at least normal sub building on fire) and a sign the
building is about to collapse. And, at least as I understand it, experts
aren't very good at explaining to novices exactly what they are
tracking. This is possibly because part of the information they are
getting, mostly through their feet, is inaccessible.

Question, one which I think turns on both empirical and conceptual
questions. Is it possible that an expert firefighter could rationally
guess the building is about to collapse on the basis of information from
their feet that is not just inaccessible, it hasn't even been cognised?
If so, we'd have a case where Alex's kind of argument didn't just
undermine accessibilism, it undermined mentalism.

Maybe you think that's far fetched, or maybe that these signals are
mental in the relevant sense. So let's try one other example. I see an
F, and immediately infer it's a G. I infer this because I have a
hard-wired disposition to make such inferences, and this is hard-wired
because it's an adaptation; in fact around here historically, Fs have
been G. Here there is a mental state - that thing is F - but what
grounds the rationality of the guess that it's G is not just my mental
state, but the evolutionary history, and the fact that Fs around here
are G. My twin earth counterpart where Fs are not Gs would be irrational
making the same guess. Again, it feels like there are problems for
mentalism here.

The big picture is one I've long worried about. Mentalism looks like an
unstable resting point between accessibilism and externalism. Once we've
rejected accessibilism, and said inaccessible mental states matter to
rationality, we may as well say that inaccessible bodily states, or
inaccessible evolutionary history, or, as the reliabilist says,
inaccessible environmental correlations, are also relevant.

OK, last point. What's a good guess? I think there are two distinct
notions here that are worth teasing out. I think Alex's argument goes
through with either understanding, so I've left this to last. But I
think it's possible relevant to the broader literature he's placing his
paper in.

Imagine a third guesser about Arenado. This person says that Arenado
isn't young, so we should discount our guesses because of his aging. How
much discount? Well he's not that old, so let's say 5\%. That gives us a
guess of 28.5.

Question: Could that be a better guess than 30?

Answer 1: No, of course not. Good guesses have some probability of being
true. And Arenado will 100\% definitely not hit 28.5 home runs.

Answer 2: Sure. What a good guess does is minimise something like
expected distance from the truth. And possibly the guess that minimises
that is 28.5.

I think in ordinary talk we use both notions - probability maximising
and expected mistake minimising. If anything I think the second is a bit
more common, but I'm not sure. The philosophical question I have no idea
about is whether there's a fact of the matter about which notion `good
guess' normally picks out, or whether there's just an ambiguity here. I
\emph{guess} that this is just an ambiguity, and there are two notions
we should be careful not to confuse. As I said, I think Alex's argument
works on either understanding. I'm a little less sure that's true for
some of the arguments he alludes to in the paper involving guessing.

Thanks again to Alexander for a really interesting paper, and I'm
looking forward to the Q\&A.




\end{document}
