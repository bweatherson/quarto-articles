% Options for packages loaded elsewhere
\PassOptionsToPackage{unicode}{hyperref}
\PassOptionsToPackage{hyphens}{url}
%
\documentclass[
  10pt,
  letterpaper,
  DIV=11,
  numbers=noendperiod,
  twoside]{scrartcl}

\usepackage{amsmath,amssymb}
\usepackage{setspace}
\usepackage{iftex}
\ifPDFTeX
  \usepackage[T1]{fontenc}
  \usepackage[utf8]{inputenc}
  \usepackage{textcomp} % provide euro and other symbols
\else % if luatex or xetex
  \usepackage{unicode-math}
  \defaultfontfeatures{Scale=MatchLowercase}
  \defaultfontfeatures[\rmfamily]{Ligatures=TeX,Scale=1}
\fi
\usepackage{lmodern}
\ifPDFTeX\else  
    % xetex/luatex font selection
    \setmainfont[ItalicFont=EB Garamond Italic,BoldFont=EB Garamond
Bold]{EB Garamond Math}
    \setsansfont[]{Europa-Bold}
  \setmathfont[]{Garamond-Math}
\fi
% Use upquote if available, for straight quotes in verbatim environments
\IfFileExists{upquote.sty}{\usepackage{upquote}}{}
\IfFileExists{microtype.sty}{% use microtype if available
  \usepackage[]{microtype}
  \UseMicrotypeSet[protrusion]{basicmath} % disable protrusion for tt fonts
}{}
\usepackage{xcolor}
\usepackage[left=1in, right=1in, top=0.8in, bottom=0.8in,
paperheight=9.5in, paperwidth=6.5in, includemp=TRUE, marginparwidth=0in,
marginparsep=0in]{geometry}
\setlength{\emergencystretch}{3em} % prevent overfull lines
\setcounter{secnumdepth}{3}
% Make \paragraph and \subparagraph free-standing
\makeatletter
\ifx\paragraph\undefined\else
  \let\oldparagraph\paragraph
  \renewcommand{\paragraph}{
    \@ifstar
      \xxxParagraphStar
      \xxxParagraphNoStar
  }
  \newcommand{\xxxParagraphStar}[1]{\oldparagraph*{#1}\mbox{}}
  \newcommand{\xxxParagraphNoStar}[1]{\oldparagraph{#1}\mbox{}}
\fi
\ifx\subparagraph\undefined\else
  \let\oldsubparagraph\subparagraph
  \renewcommand{\subparagraph}{
    \@ifstar
      \xxxSubParagraphStar
      \xxxSubParagraphNoStar
  }
  \newcommand{\xxxSubParagraphStar}[1]{\oldsubparagraph*{#1}\mbox{}}
  \newcommand{\xxxSubParagraphNoStar}[1]{\oldsubparagraph{#1}\mbox{}}
\fi
\makeatother


\providecommand{\tightlist}{%
  \setlength{\itemsep}{0pt}\setlength{\parskip}{0pt}}\usepackage{longtable,booktabs,array}
\usepackage{calc} % for calculating minipage widths
% Correct order of tables after \paragraph or \subparagraph
\usepackage{etoolbox}
\makeatletter
\patchcmd\longtable{\par}{\if@noskipsec\mbox{}\fi\par}{}{}
\makeatother
% Allow footnotes in longtable head/foot
\IfFileExists{footnotehyper.sty}{\usepackage{footnotehyper}}{\usepackage{footnote}}
\makesavenoteenv{longtable}
\usepackage{graphicx}
\makeatletter
\newsavebox\pandoc@box
\newcommand*\pandocbounded[1]{% scales image to fit in text height/width
  \sbox\pandoc@box{#1}%
  \Gscale@div\@tempa{\textheight}{\dimexpr\ht\pandoc@box+\dp\pandoc@box\relax}%
  \Gscale@div\@tempb{\linewidth}{\wd\pandoc@box}%
  \ifdim\@tempb\p@<\@tempa\p@\let\@tempa\@tempb\fi% select the smaller of both
  \ifdim\@tempa\p@<\p@\scalebox{\@tempa}{\usebox\pandoc@box}%
  \else\usebox{\pandoc@box}%
  \fi%
}
% Set default figure placement to htbp
\def\fps@figure{htbp}
\makeatother
% definitions for citeproc citations
\NewDocumentCommand\citeproctext{}{}
\NewDocumentCommand\citeproc{mm}{%
  \begingroup\def\citeproctext{#2}\cite{#1}\endgroup}
\makeatletter
 % allow citations to break across lines
 \let\@cite@ofmt\@firstofone
 % avoid brackets around text for \cite:
 \def\@biblabel#1{}
 \def\@cite#1#2{{#1\if@tempswa , #2\fi}}
\makeatother
\newlength{\cslhangindent}
\setlength{\cslhangindent}{1.5em}
\newlength{\csllabelwidth}
\setlength{\csllabelwidth}{3em}
\newenvironment{CSLReferences}[2] % #1 hanging-indent, #2 entry-spacing
 {\begin{list}{}{%
  \setlength{\itemindent}{0pt}
  \setlength{\leftmargin}{0pt}
  \setlength{\parsep}{0pt}
  % turn on hanging indent if param 1 is 1
  \ifodd #1
   \setlength{\leftmargin}{\cslhangindent}
   \setlength{\itemindent}{-1\cslhangindent}
  \fi
  % set entry spacing
  \setlength{\itemsep}{#2\baselineskip}}}
 {\end{list}}
\usepackage{calc}
\newcommand{\CSLBlock}[1]{\hfill\break\parbox[t]{\linewidth}{\strut\ignorespaces#1\strut}}
\newcommand{\CSLLeftMargin}[1]{\parbox[t]{\csllabelwidth}{\strut#1\strut}}
\newcommand{\CSLRightInline}[1]{\parbox[t]{\linewidth - \csllabelwidth}{\strut#1\strut}}
\newcommand{\CSLIndent}[1]{\hspace{\cslhangindent}#1}

\setlength\heavyrulewidth{0ex}
\setlength\lightrulewidth{0ex}
\usepackage[automark]{scrlayer-scrpage}
\clearpairofpagestyles
\cehead{
  Brian Weatherson
  }
\cohead{
  Changes in Citation Patters
  }
\ohead{\bfseries \pagemark}
\cfoot{}
\makeatletter
\newcommand*\NoIndentAfterEnv[1]{%
  \AfterEndEnvironment{#1}{\par\@afterindentfalse\@afterheading}}
\makeatother
\NoIndentAfterEnv{itemize}
\NoIndentAfterEnv{enumerate}
\NoIndentAfterEnv{description}
\NoIndentAfterEnv{quote}
\NoIndentAfterEnv{equation}
\NoIndentAfterEnv{longtable}
\NoIndentAfterEnv{abstract}
\renewenvironment{abstract}
 {\vspace{-1.25cm}
 \quotation\small\noindent\rule{\linewidth}{.5pt}\par\smallskip
 \noindent }
 {\par\noindent\rule{\linewidth}{.5pt}\endquotation}
\KOMAoption{captions}{tableheading}
\makeatletter
\@ifpackageloaded{caption}{}{\usepackage{caption}}
\AtBeginDocument{%
\ifdefined\contentsname
  \renewcommand*\contentsname{Table of contents}
\else
  \newcommand\contentsname{Table of contents}
\fi
\ifdefined\listfigurename
  \renewcommand*\listfigurename{List of Figures}
\else
  \newcommand\listfigurename{List of Figures}
\fi
\ifdefined\listtablename
  \renewcommand*\listtablename{List of Tables}
\else
  \newcommand\listtablename{List of Tables}
\fi
\ifdefined\figurename
  \renewcommand*\figurename{Figure}
\else
  \newcommand\figurename{Figure}
\fi
\ifdefined\tablename
  \renewcommand*\tablename{Table}
\else
  \newcommand\tablename{Table}
\fi
}
\@ifpackageloaded{float}{}{\usepackage{float}}
\floatstyle{ruled}
\@ifundefined{c@chapter}{\newfloat{codelisting}{h}{lop}}{\newfloat{codelisting}{h}{lop}[chapter]}
\floatname{codelisting}{Listing}
\newcommand*\listoflistings{\listof{codelisting}{List of Listings}}
\makeatother
\makeatletter
\makeatother
\makeatletter
\@ifpackageloaded{caption}{}{\usepackage{caption}}
\@ifpackageloaded{subcaption}{}{\usepackage{subcaption}}
\makeatother

\usepackage{bookmark}

\IfFileExists{xurl.sty}{\usepackage{xurl}}{} % add URL line breaks if available
\urlstyle{same} % disable monospaced font for URLs
\hypersetup{
  pdftitle={Changes in Citation Patters},
  pdfauthor={Brian Weatherson},
  hidelinks,
  pdfcreator={LaTeX via pandoc}}


\title{Changes in Citation Patters}
\author{Brian Weatherson}
\date{2025}

\begin{document}
\maketitle
\begin{abstract}
Which journal articles are cited surprisingly often in recent times,
given how often they were cited soon after publication? Conversely,
which articles have a lot more citations soon after publication than
you'd expect from their current prominence. This note goes over some
data on these questions. NB: THIS IS INCOMPLETE BUT USEFUL TABLES HERE
FOR SOMETHING THAT MIGHT BE COMPLETED.
\end{abstract}


\setstretch{1.1}
The citation history of the typical philosophy paper is nasty, brutish,
and short. If the paper gets cited at all, it gets most of those
citations within a few years of publication, then exits, gracefully or
not, into history. But a few papers buck this trend. Despite not getting
a particularly large amount of attention when they are first published,
they become widely discussed some time later. I'll call these papers
\emph{late bloomers}. This paper identifies some of these, and looks at
what they can tell us about the history of very recent philosophy.

The opposite of a late bloomer is an early bloomer, a paper that gets
most of its citations out of the way quickly. Most papers are early
bloomers, but some are more extreme than others. This could be for
several reasons. Historically, when academics often had better access to
books than journals, a paper could cease to be cited simply because it
was reprinted in an anthology. But like with the late bloomers, looking
at the early bloomers collectively can tell us something about the
history of philosophy.

Many of the late bloomers concern social, or at least interpersonal,
issues. This reflects the rise in importance in social philosophy (which
I'll hereafter understand to include interpersonal), especially since
2018. The early bloomers cover a wider range of topics, but one frequent
pattern is that they reflect the presuppositions of the modal era in
analytic philosophy.

\emph{Modal era} is Ted Sider's term for a particular era in metaphysics
(\citeproc{ref-Sider2020}{Sider 2020} x). I'm using it slightly more
broadly, to pick out a trend across philosophy as a whole. But I'm
mostly following Sider in taking it to pick out a period, very roughly
1970 to 2010, where issues about modality became central to philosophy.

There are two particularly striking things about the modal era.

The slightly less striking is that it became more or less universally
accepted that it was at least legimate to use modal and nomic notions to
set out philosophical theories. The modal and nomic notions most
commonly used were necessity, causation, and counterfactuals. The
phenomena I want to stress here can be brought out by thinking about the
reception to the counterfactual based theories of knowledge and mental
content defended by Robert
(\citeproc{ref-Nozick1979}{\textbf{Nozick1979?}}) and Jerry Fodor
(\citeproc{ref-Fodor1987}{1987}) respectively.

There are several reasons you might object to using counterfactuals
here. You might say, with some positivists, that counterfactual claims
are unverifiable and hence meaningless. You might say, with some
post-positivists, that counterfactuals are so confused and confusing
that the resulting philosophical theories will not be at all
enlightening. You might say, and this is a more common complaint since
the modal era, that counterfactuals are so context-sensitive, and
subject to human interests, that they are too subjective to use in an
account of objective things like knowledge and content. Although
Nozick's and Fodor's theories were very widely discussed at the time,
and very widely rejected, they were not largely rejected for any of
these reasons. Even the critics, and there were many critics, typically
accepted that these were reasonable tools for Nozick and Fodor to use.
They just thought the theories were wrong. It's that concession that I
want to highlight; it tells us something about the acceptance of
modality in philosophy.

The more striking thing is how many theories were understood to simply
be modal claims. Origin essentialism, as in
(\citeproc{ref-Kripke1972}{\textbf{Kripke1972?}}), was simply identified
with a necessity of origins claim. Whether mental content was `in the
head' was simply identified with the truth or falsity of a modal
supervenience claim. And, this will be the one we talk about the most,
whether the world is entirely physical was also identified with some
kind of supervenience claim. This last one is fascinating because there
was such widespread agreement that if physicalism was anything, it was
some kind of supervenience claim. This agreement persisted in the face
of disagreement about (a) which supervenience claim it was, (b) whether
physicalism was even true, and (c) whether physicalism was even an
intelligible claim.

What we see between these two `striking things' is that philosophers
thought it was almost always permissible to give a modal answer to a
philosophical question, and often enough, it was obligatory. Very few
philosophers endorsed the ontology of modality Lewis put forward in
\emph{Plurality}, but they by and large agreed with him that modal space
was ``A Philosopher's Paradise''
(\citeproc{ref-Lewis1986}{\textbf{Lewis1986?}}).

Over the last two decades, all that has changed. We'll see this in the
early bloomers data, espeically with the citation history of papers
particularly on physicalism and on content. In metaphysics, it has
become somewhat common to use grounding notions to do the work that used
to be done by modality. But this does not mean we're in a grounding era
in philosophy, or even, as Sider puts it, a `post-modal' era. Rather

There are a few reasons why articles might turn up on a list like this.

\begin{itemize}
\tightlist
\item
  Some papers are in topics that are much more central to philosophy now
  than they were at the time of publication. This is especially true for
  papers concerned with social and interpersonal topics, such as
  Haslanger (\citeproc{ref-WOS000085841900002}{2000}), Darwall
  (\citeproc{ref-WOSA1977EA35800003}{1977}), Langton
  (\citeproc{ref-WOSA1993MJ74900002}{1993}), and Goldman
  (\citeproc{ref-WOS000170434600004}{2001}).
\item
  Some papers are important instances of approaches that were not as
  prominent at the time they were published. This starts at the top of
  the list; utilitarianism wasn't nearly as prominent in the late
  twentieth century as in the twenty-first century.
\item
  And, as Eric Schwitzgebel pointed out on an earlier post, in some
  cases what we're seeing is just the normal pattern of highly cited
  papers. This is, I think, what's happening with Street
  (\citeproc{ref-WOS000234431300006}{2006}) and Pryor
  (\citeproc{ref-WOS000165361800002}{2000}). Note that I'm comparing the
  trends in citations of papers like these to the citations of all
  papers published the same year. If it takes a while for the literature
  to settle on what the most important papers are, and after that time
  the other papers drop away, then papers like these could appear on the
  list as much by not having their citation rates decay as by becoming
  particularly more prominent.
\end{itemize}

What about the other end of the scale? To look at the other end, we
create the same binomial distribution, but this time look at articles
where it is surprising that there are this few or fewer citations after
the cutoff. These are the articles that had a strikingly high number of
early citations, given what happened later. The top of that list is
shown in \textbf{?@tbl-early-bloomers}.

\begin{itemize}
\tightlist
\item
  The article might get superseded. This happens especially if the same
  author puts out a later piece that is simply better than the earlier
  one. I think this is what's going on with Kim
  (\citeproc{ref-WOSA1978EL93700009}{1978}); eventually people came to
  think of later works, such as Kim
  (\citeproc{ref-WOSA1984TV24600001}{1984}), as simply better. I suspect
  it's also true with some of the philosophy of biology papers on this
  list; that has been a fast moving field, and even very good papers get
  replaced in the canon by more up to date ones.
\item
  The article might be incorporated into a book, and the citations move
  to the book. That's pretty clearly what's happened with Rawls
  (\citeproc{ref-WOSA1985APA8500001}{1985}) and MacFarlane
  (\citeproc{ref-WOS000262577100002}{2009}).
\item
  There is a technical issue here with reprints in collected volumes.
  Web of Science \emph{sometimes} puts together citations to a paper in
  a volume with citations in the original journal, but it doesn't
  always. This also affects the citations to the Kim papers, and I think
  that's why DeRose (\citeproc{ref-WOS000228214500002}{2005}) is here.
\item
  Sometimes the discipline gets into the habit of citing one paper in
  lieu of a family of papers that developed a point. So anyone who wants
  to point in the general direction of Joshua Knobe's mid-2000s work on
  intention cites Knobe (\citeproc{ref-WOS000183806600005}{2003}), while
  his other papers from around the same time end up on lists like this.
\item
  And sometimes the discipline just loses interest in a topic. So we
  have a lot of papers here on intuitions, and on supervenience. After
  the boom in interest in vagueness in the 2000s, citations to Quine
  (\citeproc{ref-WOSA1970ZE32000003}{1970}) and Field
  (\citeproc{ref-10.2307_2025110}{1973}) came back to earth. Even what
  seemed like completely central issues about modality and about mental
  content don't seem immune to this.
\end{itemize}

I think there is an interesting story in how many of the most
influential, at least at the time, papers of the 1980s and 1990s are now
not playing a major role in philosophical conversations. This did not,
on the whole, happen to the other decades. There are obviously plenty of
papers from the 1970s and 2000s on \textbf{?@tbl-early-bloomers}. But
they did not seem, on the whole, to be quite as central to philosophy at
their time as the papers on that list from the 1980s or
1990s.\footnote{Or if they are, as MacFarlane
  (\citeproc{ref-WOS000262577100002}{2009}) surely was, the reason they
  are here is because citations to a paper were simply replaced by
  citations to a book; they didn't just fall away.}

But that's for later work. For now I just wanted to post these lists of
which papers were distinctively late bloomers and early bloomers, and
offer some tentative suggestions for what the nature of those lists
tells us about the last half-century of philosophy.

\phantomsection\label{refs}
\begin{CSLReferences}{1}{0}
\bibitem[\citeproctext]{ref-WOSA1977EA35800003}
Darwall, Stephen L. 1977. {``Two Kinds of Respect.''} \emph{Ethics} 88
(1): 36--49. doi: \href{https://doi.org/10.1086/292054}{10.1086/292054}.

\bibitem[\citeproctext]{ref-WOS000228214500002}
DeRose, Keith. 2005. {``The Ordinary Language Basis for Contextualism,
and the New Invariantism.''} \emph{Philosophical Quarterly} 55 (219):
172--98. doi:
\href{https://doi.org/10.1111/j.0031-8094.2005.00394.x}{10.1111/j.0031-8094.2005.00394.x}.

\bibitem[\citeproctext]{ref-10.2307_2025110}
Field, Hartry. 1973. {``Theory Change and the Indeterminacy of
Reference.''} \emph{Journal Of Philosophy} 70 (14): 462--81.

\bibitem[\citeproctext]{ref-Fodor1987}
Fodor, Jerry A. 1987. \emph{Psychosemantics}. Cambridge, MA: MIT Press.

\bibitem[\citeproctext]{ref-WOS000170434600004}
Goldman, Alvin I. 2001. {``Experts: Which Ones Should You Trust?''}
\emph{Philosophy And Phenomenological Research} 63 (1): 85--110. doi:
\href{https://doi.org/10.2307/3071090}{10.2307/3071090}.

\bibitem[\citeproctext]{ref-WOS000085841900002}
Haslanger, Sally. 2000. {``Gender and Race: (What) Are They? (What) Do
We Want Them To Be?''} \emph{Noûs} 34 (1): 31--55. doi:
\href{https://doi.org/10.1111/0029-4624.00201}{10.1111/0029-4624.00201}.

\bibitem[\citeproctext]{ref-WOSA1978EL93700009}
Kim, Jaegwon. 1978. {``Supervenience and Nomological
Incommensurables.''} \emph{American Philosophical Quarterly} 15 (2):
149--56.

\bibitem[\citeproctext]{ref-WOSA1984TV24600001}
---------. 1984. {``Concepts of Supervenience.''} \emph{Philosophy And
Phenomenological Research} 45 (2): 153--76. doi:
\href{https://doi.org/10.2307/2107423}{10.2307/2107423}.

\bibitem[\citeproctext]{ref-WOS000183806600005}
Knobe, Joshua. 2003. {``Intentional Action and Side Effects in Ordinary
Language.''} \emph{Analysis} 63 (3): 190--94. doi:
\href{https://doi.org/10.1111/1467-8284.00419}{10.1111/1467-8284.00419}.

\bibitem[\citeproctext]{ref-WOSA1993MJ74900002}
Langton, Rae. 1993. {``Speech Acts and Unspeakable Acts.''}
\emph{Philosophy \& Public Affairs} 22 (4): 293--330.

\bibitem[\citeproctext]{ref-WOS000262577100002}
MacFarlane, John. 2009. {``Nonindexical Contextualism.''}
\emph{Synthese} 166 (2): 231--50. doi:
\href{https://doi.org/10.1007/s11229-007-9286-2}{10.1007/s11229-007-9286-2}.

\bibitem[\citeproctext]{ref-WOS000165361800002}
Pryor, James. 2000. {``The Skeptic and the Dogmatist.''} \emph{Noûs} 34
(4): 517--49. doi:
\href{https://doi.org/10.1111/0029-4624.00277}{10.1111/0029-4624.00277}.

\bibitem[\citeproctext]{ref-WOSA1970ZE32000003}
Quine, Willard van Orman. 1970. {``On the Reasons for Indeterminacy of
Translation.''} \emph{Journal Of Philosophy} 67 (6): 178--83. doi:
\href{https://doi.org/10.2307/2023887}{10.2307/2023887}.

\bibitem[\citeproctext]{ref-WOSA1985APA8500001}
Rawls, John. 1985. {``Justice as Fairness: Political Not
Metaphysical.''} \emph{Philosophy \& Public Affairs} 14 (3): 223--51.

\bibitem[\citeproctext]{ref-Sider2020}
Sider, Theodore. 2020. \emph{The Tools of Metaphysics and the
Metaphysics of Science}. Oxford: Oxford University Press.

\bibitem[\citeproctext]{ref-WOS000234431300006}
Street, Sharon. 2006. {``A Darwinian Dilemma for Realist Theories of
Value.''} \emph{Philosophical Studies} 127 (1): 109--66. doi:
\href{https://doi.org/10.1007/s11098-005-1726-6}{10.1007/s11098-005-1726-6}.

\end{CSLReferences}



\noindent Published in\emph{
?meta:citation.container-title}, 2025, pp. ?meta:citation.page.


\end{document}
