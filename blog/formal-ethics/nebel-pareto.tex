% Options for packages loaded elsewhere
% Options for packages loaded elsewhere
\PassOptionsToPackage{unicode}{hyperref}
\PassOptionsToPackage{hyphens}{url}
\PassOptionsToPackage{dvipsnames,svgnames,x11names}{xcolor}
%
\documentclass[
  letterpaper,
  DIV=11,
  numbers=noendperiod,
  oneside]{scrartcl}
\usepackage{xcolor}
\usepackage[left=1in,marginparwidth=2.0666666666667in,textwidth=4.1333333333333in,marginparsep=0.3in]{geometry}
\usepackage{amsmath,amssymb}
\setcounter{secnumdepth}{3}
\usepackage{iftex}
\ifPDFTeX
  \usepackage[T1]{fontenc}
  \usepackage[utf8]{inputenc}
  \usepackage{textcomp} % provide euro and other symbols
\else % if luatex or xetex
  \usepackage{unicode-math} % this also loads fontspec
  \defaultfontfeatures{Scale=MatchLowercase}
  \defaultfontfeatures[\rmfamily]{Ligatures=TeX,Scale=1}
\fi
\usepackage{lmodern}
\ifPDFTeX\else
  % xetex/luatex font selection
\fi
% Use upquote if available, for straight quotes in verbatim environments
\IfFileExists{upquote.sty}{\usepackage{upquote}}{}
\IfFileExists{microtype.sty}{% use microtype if available
  \usepackage[]{microtype}
  \UseMicrotypeSet[protrusion]{basicmath} % disable protrusion for tt fonts
}{}
\makeatletter
\@ifundefined{KOMAClassName}{% if non-KOMA class
  \IfFileExists{parskip.sty}{%
    \usepackage{parskip}
  }{% else
    \setlength{\parindent}{0pt}
    \setlength{\parskip}{6pt plus 2pt minus 1pt}}
}{% if KOMA class
  \KOMAoptions{parskip=half}}
\makeatother
% Make \paragraph and \subparagraph free-standing
\makeatletter
\ifx\paragraph\undefined\else
  \let\oldparagraph\paragraph
  \renewcommand{\paragraph}{
    \@ifstar
      \xxxParagraphStar
      \xxxParagraphNoStar
  }
  \newcommand{\xxxParagraphStar}[1]{\oldparagraph*{#1}\mbox{}}
  \newcommand{\xxxParagraphNoStar}[1]{\oldparagraph{#1}\mbox{}}
\fi
\ifx\subparagraph\undefined\else
  \let\oldsubparagraph\subparagraph
  \renewcommand{\subparagraph}{
    \@ifstar
      \xxxSubParagraphStar
      \xxxSubParagraphNoStar
  }
  \newcommand{\xxxSubParagraphStar}[1]{\oldsubparagraph*{#1}\mbox{}}
  \newcommand{\xxxSubParagraphNoStar}[1]{\oldsubparagraph{#1}\mbox{}}
\fi
\makeatother


\usepackage{longtable,booktabs,array}
\usepackage{calc} % for calculating minipage widths
% Correct order of tables after \paragraph or \subparagraph
\usepackage{etoolbox}
\makeatletter
\patchcmd\longtable{\par}{\if@noskipsec\mbox{}\fi\par}{}{}
\makeatother
% Allow footnotes in longtable head/foot
\IfFileExists{footnotehyper.sty}{\usepackage{footnotehyper}}{\usepackage{footnote}}
\makesavenoteenv{longtable}
\usepackage{graphicx}
\makeatletter
\newsavebox\pandoc@box
\newcommand*\pandocbounded[1]{% scales image to fit in text height/width
  \sbox\pandoc@box{#1}%
  \Gscale@div\@tempa{\textheight}{\dimexpr\ht\pandoc@box+\dp\pandoc@box\relax}%
  \Gscale@div\@tempb{\linewidth}{\wd\pandoc@box}%
  \ifdim\@tempb\p@<\@tempa\p@\let\@tempa\@tempb\fi% select the smaller of both
  \ifdim\@tempa\p@<\p@\scalebox{\@tempa}{\usebox\pandoc@box}%
  \else\usebox{\pandoc@box}%
  \fi%
}
% Set default figure placement to htbp
\def\fps@figure{htbp}
\makeatother


% definitions for citeproc citations
\NewDocumentCommand\citeproctext{}{}
\NewDocumentCommand\citeproc{mm}{%
  \begingroup\def\citeproctext{#2}\cite{#1}\endgroup}
\makeatletter
 % allow citations to break across lines
 \let\@cite@ofmt\@firstofone
 % avoid brackets around text for \cite:
 \def\@biblabel#1{}
 \def\@cite#1#2{{#1\if@tempswa , #2\fi}}
\makeatother
\newlength{\cslhangindent}
\setlength{\cslhangindent}{1.5em}
\newlength{\csllabelwidth}
\setlength{\csllabelwidth}{3em}
\newenvironment{CSLReferences}[2] % #1 hanging-indent, #2 entry-spacing
 {\begin{list}{}{%
  \setlength{\itemindent}{0pt}
  \setlength{\leftmargin}{0pt}
  \setlength{\parsep}{0pt}
  % turn on hanging indent if param 1 is 1
  \ifodd #1
   \setlength{\leftmargin}{\cslhangindent}
   \setlength{\itemindent}{-1\cslhangindent}
  \fi
  % set entry spacing
  \setlength{\itemsep}{#2\baselineskip}}}
 {\end{list}}
\usepackage{calc}
\newcommand{\CSLBlock}[1]{\hfill\break\parbox[t]{\linewidth}{\strut\ignorespaces#1\strut}}
\newcommand{\CSLLeftMargin}[1]{\parbox[t]{\csllabelwidth}{\strut#1\strut}}
\newcommand{\CSLRightInline}[1]{\parbox[t]{\linewidth - \csllabelwidth}{\strut#1\strut}}
\newcommand{\CSLIndent}[1]{\hspace{\cslhangindent}#1}



\setlength{\emergencystretch}{3em} % prevent overfull lines

\providecommand{\tightlist}{%
  \setlength{\itemsep}{0pt}\setlength{\parskip}{0pt}}



 


\usepackage{xcolor}
\newcommand{\textred}[1]{\textcolor{red}{#1}}
\newcommand{\textblue}[1]{\textcolor{blue}{#1}}
\KOMAoption{captions}{tableheading}
\makeatletter
\@ifpackageloaded{caption}{}{\usepackage{caption}}
\AtBeginDocument{%
\ifdefined\contentsname
  \renewcommand*\contentsname{Table of contents}
\else
  \newcommand\contentsname{Table of contents}
\fi
\ifdefined\listfigurename
  \renewcommand*\listfigurename{List of Figures}
\else
  \newcommand\listfigurename{List of Figures}
\fi
\ifdefined\listtablename
  \renewcommand*\listtablename{List of Tables}
\else
  \newcommand\listtablename{List of Tables}
\fi
\ifdefined\figurename
  \renewcommand*\figurename{Figure}
\else
  \newcommand\figurename{Figure}
\fi
\ifdefined\tablename
  \renewcommand*\tablename{Table}
\else
  \newcommand\tablename{Table}
\fi
}
\@ifpackageloaded{float}{}{\usepackage{float}}
\floatstyle{ruled}
\@ifundefined{c@chapter}{\newfloat{codelisting}{h}{lop}}{\newfloat{codelisting}{h}{lop}[chapter]}
\floatname{codelisting}{Listing}
\newcommand*\listoflistings{\listof{codelisting}{List of Listings}}
\makeatother
\makeatletter
\makeatother
\makeatletter
\@ifpackageloaded{caption}{}{\usepackage{caption}}
\@ifpackageloaded{subcaption}{}{\usepackage{subcaption}}
\makeatother
\makeatletter
\@ifpackageloaded{sidenotes}{}{\usepackage{sidenotes}}
\@ifpackageloaded{marginnote}{}{\usepackage{marginnote}}
\makeatother
\usepackage{bookmark}
\IfFileExists{xurl.sty}{\usepackage{xurl}}{} % add URL line breaks if available
\urlstyle{same}
\hypersetup{
  pdftitle={The Impossibility of an Infinite Paretian},
  pdfauthor={Brian Weatherson},
  colorlinks=true,
  linkcolor={blue},
  filecolor={Maroon},
  citecolor={Blue},
  urlcolor={Blue},
  pdfcreator={LaTeX via pandoc}}


\title{The Impossibility of an Infinite Paretian}
\author{Brian Weatherson}
\date{2025-08-28}
\begin{document}
\maketitle
\begin{abstract}
Jake Nebel has a really nice example that he thinks shows that a kind of
impartiality is impossible in infinite cases. I think it shows something
else, namely that various Pareto principles are inconsistent in infinite
settings.
\end{abstract}


Jake Nebel (2025) has a nice argument coming out in \emph{Noûs} against
the possibility of having agent-neutral values in certain infinite
worlds. That's to say, he has a proof that in infinite cases a bunch of
plausible principles turn out to be inconsistent, one of them being the
possibility of agent-neutral values. He argues that's the one of the
inconsistent set we should reject.

I'm going to use his example to argue for a different conclusion. I
think what his case shows is that some very plausible looking Pareto
principles can't be correct. These infinitary cases should be thought
of, like the example Sen (1970) gives of nosy neighbours in a liberal
society, as reasons to be sceptical of Pareto optimality.

Here is the main case that Nebel uses.\footnote{His example draws on
  Goodsell (2021).} We have a world consisting of a countable infinity
of people, \emph{a}\textsubscript{1}, \emph{a}\textsubscript{2}, \ldots,
\emph{b}\textsubscript{1}, \emph{b}\textsubscript{2}, \ldots. Each of
these people has one of two possible welfare states, which we'll denote
as 0 and 1. We'll say an outcome is any assignment of a probability
distribution of welfare payouts to each individual. (This means that
we're not going to consider what happens if the world population
changes; that raises a whole host of different concerns.)

One of four lotteries will take place. The lotteries are shown in
Table~\ref{tbl-nebel}. In each lottery, there are two random devices
that contribute to the state. The first is a fair \textcolor{red}{red}
coin which will be flipped once. The second is a fair coin that will be
flipped indefinitely until it lands heads. We'll write
\textcolor{red}{H} and \textcolor{red}{T} for the \textcolor{red}{red}
coin landing heads and tails respectively, and H\textsubscript{\emph{i}}
for the proposition that the other coin will be flipped \emph{i} times.
(For completeness, say that H\textsubscript{1} also includes the
possibility that it lands tails every time.) So we have the states shown
in the top row of Table~\ref{tbl-nebel}, with their probability below
them. In each cell, we show the people who get payout 1 in that state
from that lottery; everyone else gets payout 0.

\begin{longtable}[]{@{}
  >{\raggedleft\arraybackslash}p{(\linewidth - 16\tabcolsep) * \real{0.1351}}
  >{\centering\arraybackslash}p{(\linewidth - 16\tabcolsep) * \real{0.1081}}
  >{\centering\arraybackslash}p{(\linewidth - 16\tabcolsep) * \real{0.1081}}
  >{\centering\arraybackslash}p{(\linewidth - 16\tabcolsep) * \real{0.1081}}
  >{\centering\arraybackslash}p{(\linewidth - 16\tabcolsep) * \real{0.1081}}
  >{\centering\arraybackslash}p{(\linewidth - 16\tabcolsep) * \real{0.1081}}
  >{\centering\arraybackslash}p{(\linewidth - 16\tabcolsep) * \real{0.1081}}
  >{\centering\arraybackslash}p{(\linewidth - 16\tabcolsep) * \real{0.1081}}
  >{\centering\arraybackslash}p{(\linewidth - 16\tabcolsep) * \real{0.1081}}@{}}
\caption{Nebel's example}\label{tbl-nebel}\tabularnewline
\toprule\noalign{}
\begin{minipage}[b]{\linewidth}\raggedleft
Lottery
\end{minipage} & \begin{minipage}[b]{\linewidth}\centering
\textcolor{red}{H}H\textsubscript{1}
\end{minipage} & \begin{minipage}[b]{\linewidth}\centering
\textcolor{red}{T}H\textsubscript{1}
\end{minipage} & \begin{minipage}[b]{\linewidth}\centering
\textcolor{red}{H}H\textsubscript{2}
\end{minipage} & \begin{minipage}[b]{\linewidth}\centering
\textcolor{red}{T}H\textsubscript{2}
\end{minipage} & \begin{minipage}[b]{\linewidth}\centering
\ldots{}
\end{minipage} & \begin{minipage}[b]{\linewidth}\centering
\textcolor{red}{H}H\textsubscript{\emph{k}}
\end{minipage} & \begin{minipage}[b]{\linewidth}\centering
\textcolor{red}{T}H\textsubscript{\emph{k}}
\end{minipage} & \begin{minipage}[b]{\linewidth}\centering
\ldots{}
\end{minipage} \\
\midrule\noalign{}
\endfirsthead
\toprule\noalign{}
\begin{minipage}[b]{\linewidth}\raggedleft
Lottery
\end{minipage} & \begin{minipage}[b]{\linewidth}\centering
\textcolor{red}{H}H\textsubscript{1}
\end{minipage} & \begin{minipage}[b]{\linewidth}\centering
\textcolor{red}{T}H\textsubscript{1}
\end{minipage} & \begin{minipage}[b]{\linewidth}\centering
\textcolor{red}{H}H\textsubscript{2}
\end{minipage} & \begin{minipage}[b]{\linewidth}\centering
\textcolor{red}{T}H\textsubscript{2}
\end{minipage} & \begin{minipage}[b]{\linewidth}\centering
\ldots{}
\end{minipage} & \begin{minipage}[b]{\linewidth}\centering
\textcolor{red}{H}H\textsubscript{\emph{k}}
\end{minipage} & \begin{minipage}[b]{\linewidth}\centering
\textcolor{red}{T}H\textsubscript{\emph{k}}
\end{minipage} & \begin{minipage}[b]{\linewidth}\centering
\ldots{}
\end{minipage} \\
\midrule\noalign{}
\endhead
\bottomrule\noalign{}
\endlastfoot
\textbf{Probability} & 1/4 & 1/4 & 1/8 & 1/8 & \ldots{} &
1/2\textsuperscript{\emph{k}+1} & 1/2\textsuperscript{\emph{k}+1} &
\ldots{} \\
\emph{P} & \emph{a}\textsubscript{1}, \emph{a}\textsubscript{2} &
\emph{a}\textsubscript{1}, \emph{a}\textsubscript{2} &
\emph{a}\textsubscript{1} \ldots{} \emph{a}\textsubscript{4} &
\emph{a}\textsubscript{1} \ldots{} \emph{a}\textsubscript{4} & \ldots{}
& \emph{a}\textsubscript{1} \ldots{}
\emph{a}\textsubscript{2\textsuperscript{\emph{k}}} &
\emph{a}\textsubscript{1} \ldots{}
\emph{a}\textsubscript{2\textsuperscript{\emph{k}}} & \ldots{} \\
\emph{Q} & \emph{a}\textsubscript{1}, \emph{a}\textsubscript{2} &
\emph{b}\textsubscript{1}, \emph{b}\textsubscript{2} &
\emph{a}\textsubscript{1} \ldots{} \emph{a}\textsubscript{4} &
\emph{b}\textsubscript{1} \ldots{} \emph{b}\textsubscript{4} & \ldots{}
& \emph{a}\textsubscript{1} \ldots{}
\emph{a}\textsubscript{2\textsuperscript{\emph{k}}} &
\emph{b}\textsubscript{1} \ldots{}
\emph{b}\textsubscript{2\textsuperscript{\emph{k}}} & \ldots{} \\
\emph{P}ʹ & \emph{a}\textsubscript{1}, \emph{a}\textsubscript{2} &
\emph{a}\textsubscript{1} \ldots{} \emph{a}\textsubscript{4} &
\emph{a}\textsubscript{1}, \emph{a}\textsubscript{2} &
\emph{a}\textsubscript{1} \ldots{} \emph{a}\textsubscript{8} & \ldots{}
& \emph{a}\textsubscript{1}, \emph{a}\textsubscript{2} &
\emph{a}\textsubscript{1} \ldots{}
\emph{a}\textsubscript{2\textsuperscript{\emph{k}+1}} & \ldots{} \\
\emph{Q}ʹ & \emph{a}\textsubscript{1}, \emph{b}\textsubscript{1} &
\emph{a}\textsubscript{1}, \emph{a}\textsubscript{2},
\emph{b}\textsubscript{1}, \emph{b}\textsubscript{2} &
\emph{a}\textsubscript{2}, \emph{b}\textsubscript{2} &
\emph{a}\textsubscript{1} \ldots{} \emph{a}\textsubscript{4},
\emph{b}\textsubscript{1} \ldots{} \emph{b}\textsubscript{4} & \ldots{}
& \emph{a}\textsubscript{\emph{k}-1}, \emph{b}\textsubscript{\emph{k}-1}
& \emph{a}\textsubscript{1} \ldots{}
\emph{a}\textsubscript{2\textsuperscript{\emph{k}}},
\emph{b}\textsubscript{1} \ldots{}
\emph{b}\textsubscript{2\textsuperscript{\emph{k}}} & \ldots{} \\
\end{longtable}

Now assume that that for each person \emph{x} we have relations of
strict preference ≻\textsubscript{\emph{x}} and indifference
\textasciitilde{}\textsubscript{\emph{x}}, and from a society wide
perspective a social welfare ordering, i.e., a pair of relations of
strict preference ≻\textsubscript{∀} and indifference
\textasciitilde{}\textsubscript{∀}. We'll also assume that no matter the
subscript, that \textasciitilde{} is symmetric and ≻ is asymmetric, and
that for any options \emph{A} and \emph{B}, \emph{A}~≻~\emph{B},
\emph{A}~\textasciitilde~\emph{B} and \emph{B}~≻~\emph{A} are exclusive
and exhaustive.

We'll also assume that each individual prefers a higher probability of
getting 1 to a lower probability, and is indifferent between any two
states with the same probability of getting 1.

Nebel then derives a contradiction from the following assumptions.

\begin{description}
\tightlist
\item[Transitivity of Social Indifference]
If \emph{A}~\textasciitilde{}\textsubscript{∀}~\emph{B} and
\emph{B}~\textasciitilde{}\textsubscript{∀}~\emph{C} then
\emph{A}~\textasciitilde{}\textsubscript{∀}~\emph{C}.
\item[Pareto Indifference]
If \emph{A}~\textasciitilde{}\textsubscript{\emph{x}}~\emph{B} for all
\emph{x}, then \emph{A}~\textasciitilde{}\textsubscript{∀}~\emph{B}.
\item[Pareto Strict Preference]
If \emph{A}~≻\textsubscript{\emph{x}}~\emph{B} for all \emph{x}, then
\emph{A}~≻\textsubscript{∀}~\emph{B}.
\item[Pairwise Anonymity]
For outcome \emph{O} and individual \emph{x}, write
\emph{O\textsubscript{x}} for the outcome individual \emph{x} gets in
\emph{O}. For any outcomes \emph{A} and \emph{B} if there are two
individuals \emph{i} and \emph{j} such that
\emph{A\textsubscript{i}}~\textasciitilde{}\textsubscript{\emph{i}}
\emph{B}\textsubscript{j} and
\emph{A\textsubscript{j}}~\textasciitilde{}\textsubscript{\emph{j}}
\emph{B}\textsubscript{i}, and for all other individuals \emph{h},
\emph{A\textsubscript{h}}~\textasciitilde{}\textsubscript{\emph{h}}
\emph{B}\textsubscript{h}, then
\emph{A}~\textasciitilde{}\textsubscript{∀} \emph{B}.
\item[Lottery Indifference]
For any two lotteries \emph{A} and \emph{B} defined over the same
partition of the possibilities in states, if in each state
\emph{A}~\textasciitilde{}\textsubscript{∀} \emph{B}, then prior to the
lottery taking place, \emph{A}~\textasciitilde{}\textsubscript{∀}
\emph{B}.
\end{description}

We now reason as follows.

First, to prove \emph{Q}ʹ~≻\textsubscript{∀} \emph{Q}. Note that each
individual has a higher probability of getting 1 in \emph{Q}ʹ than
\emph{Q}. So \emph{Q}ʹ~≻\textsubscript{\emph{x}}~\emph{Q} for all
\emph{x}, so by Pareto Strict Preference, \emph{Q}ʹ~≻\textsubscript{∀}
\emph{Q}.

Second, to prove \emph{Q}~\textasciitilde{}\textsubscript{∀} \emph{P}.
In each result of the lottery, the same number of people get 1. So in
each state, a finite number of applications of Pairwise Anonymity plus
one use of Pareto Indifference we have
\emph{Q}~\textasciitilde{}\textsubscript{∀} \emph{P} conditional on each
state of the lottery. So by Lottery Indifference,
\emph{Q}~\textasciitilde{}\textsubscript{∀} \emph{P}.

Third, to prove \emph{P}~\textasciitilde{}\textsubscript{∀} \emph{P}ʹ.
Each individual has the same probability of getting 1 in each lottery,
so each individual is indifferent between the lotteries, so by Pareto
Indifference, \emph{P}~\textasciitilde{}\textsubscript{∀} \emph{P}ʹ.

Fourth, to prove \emph{P}ʹ~\textasciitilde{}\textsubscript{∀} \emph{Q}ʹ.
Here it is the same reasoning as in step 2. No matter what lottery
outcome we're in, the same number of people get payout 1 in \emph{P}ʹ as
\emph{Q}ʹ, so by many (but in each state a finite number of)
applications of Pairwise Anonymity plus one use of Pareto Indifference
we have \emph{Q}~\textasciitilde{}\textsubscript{∀} \emph{P} conditional
on each state of the lottery. So by Lottery Indifference,
\emph{Q}~\textasciitilde{}\textsubscript{∀} \emph{P}.

Fifth, by two applications of Transitivity of Indifference we get from
\emph{Q}~\textasciitilde{}\textsubscript{∀} \emph{P},
\emph{P}~\textasciitilde{}\textsubscript{∀} \emph{P}ʹ, and
\emph{P}ʹ~\textasciitilde{}\textsubscript{∀} \emph{Q}ʹ to
\emph{Q}~\textasciitilde{}\textsubscript{∀} \emph{Q}ʹ, contradicting
\emph{Q}ʹ~≻\textsubscript{∀} \emph{Q}.

Nebel argues that the culprit here is Pairwise Anonymity. I think that's
not right. I'm going to develop a similar puzzle that gets to a
contradiction without Pairwise Anonymity. To be sure, I will use other
assumptions which one could reasonably argue are to blame for the
contradiction we'll get. But I hope this provides independent evidence
for

For what it's worth, in Nebel's example my first reaction is that the
culprit is Transitivity of Indifference. We have plenty of examples,
going back to Armstrong (1939) and Debreu (1960) of violations of
Transitivity of Indifference, and Sen ({[}1970{]} 2017) argues at length
that we should expect it to fail in social settings like this.

\phantomsection\label{refs}
\begin{CSLReferences}{1}{0}
\bibitem[\citeproctext]{ref-Armstrong1939}
Armstrong, W. E. 1939. {``The Determinateness of the Utility
Function.''} \emph{The Economic Journal} 49 (195): 453--67. doi:
\href{https://doi.org/10.2307/2224802}{10.2307/2224802}.

\bibitem[\citeproctext]{ref-Debreu1960}
Debreu, Gerard. 1960. {``Review of \emph{Individual Choice Behavior: A
Theoretical Analysis}, by {R. Duncan Luce}.''} \emph{American Economic
Review} 50 (1): 186--88.

\bibitem[\citeproctext]{ref-Goodsell2021}
Goodsell, Zachary. 2021. {``A St Petersburg Paradox for Risky Welfare
Aggregation.''} \emph{Analysis} 81 (3): 420--26. doi:
\href{https://doi.org/10.1093/analys/anaa079}{10.1093/analys/anaa079}.

\bibitem[\citeproctext]{ref-Nebel2025}
Nebel, Jacob M. 2025. {``Infinite Ethics and the Limits of
Impartiality.''} No{û}s. 2025. doi:
\href{https://doi.org/10.1111/nous.70010}{10.1111/nous.70010}.

\bibitem[\citeproctext]{ref-Sen1970}
Sen, Amartya. 1970. {``The Impossibility of a Paretian Liberal.''}
\emph{Journal of Political Economy} 78 (1): 152--57. doi:
\href{https://doi.org/10.1086/259614}{10.1086/259614}.

\bibitem[\citeproctext]{ref-Sen1970sec}
---------. (1970) 2017. \emph{Collective Choice and Social Welfare:} An
expanded edition. Cambridge, MA: Harvard University Press. doi:
\href{https://doi.org/10.4159/9780674974616}{10.4159/9780674974616}.

\end{CSLReferences}




\end{document}
